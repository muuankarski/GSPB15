\section{Foreword}

\bigskip
\bigskip

\large


At the first International Conference on Nutrition, held in 1992, global leaders pledged to “act in solidarity to ensure that freedom from hunger becomes a reality.”

Although great progress has been made in reducing the prevalence of hunger, over 800 million people are still unable to meet their daily calorie needs for living healthy lives. About one in nine people go to bed daily on an empty stomach. In cases where food is available, often the quality of the food does not meet micronutrient (vitamin and mineral) needs. More than two billion people continue to suffer from nutritional deficiencies such as vitamin A, iron, zinc and iodine. While the world is grappling with issues of undernutrition, there is also the growing problem of obesity, which now affects around 500 million people. Many countries are facing a triple burden of malnutrition, where undernourishment, micronutrient deficiency and obesity exist in the same community and household. 

ICN2 presents another opportunity for the global community to make a commitment and take action to address this global menace. The two outcome documents of ICN2 - the Rome Declaration and the Framework for Action - will provide the basis for renewed commitment and focused action for addressing malnutrition within the coming decade. Experiences from the Millennium Development Goals indicate that, with a united commitment, we can achieve significant results. We must now move forward with the same determination as we address new global challenges through the Sustainable Development Goals.

Having clear indicators to measure progress is very important. Statistics are a fundamental tool in this process, necessary to identify problems and monitor progress. The better the data, the better policies can be designed to improve nutrition worldwide. Without good data, it is impossible to evaluate or determine the impact of policies, or hold stakeholders accountable for pledges they make. For statistics to effectively inform food and agriculture policies, they need to be accessible and clear to policymakers at global, regional and country levels. This publication presents selected key indicators related to food and nutrition outcomes that stakeholders can use to prioritise their actions.  

This food and nutrition pocketbook was produced jointly by the FAO Statistics and Nutrition Divisions. It is part of the FAO Statistical Yearbook suite of products and is one of the tools that can be used as building blocks for evidence-based policy making. It includes data from FAOSTAT as well as from other partners in the organization and in the international community. 

There are still gaps in the information. We hope that ICN2 will  provide the  forum for discussion on ways to improve the data to better monitor nutrition.

\bigskip

{\centering
Pietro Gennari
Chief Statistician and Director, Statistics Division
}

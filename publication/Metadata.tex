\begin{MetadataCollection}

\twocolumn

\LARGE

\textbf{Definitions}

\footnotesize

\begin{metadata}{ Agricultural area organic (ha) }{ RL.AREA.AGRORG.HA.NO }
Sum of areas under "Agricultural area certified organic" and "Agricultural area in conversion to organic". Agricultural area certified organic is the land area exclusively dedicated to organic agriculture and managed by applying organic agriculture methods. It refers to the land area fully converted to organic agriculture. It is the portion of land area (including arable lands, pastures or wild areas) managed (cultivated) or wild harvested in accordance with specific organic standards or technical regulations and that has been inspected and approved by a certification body. Agricultural area in conversion to organic is the land area which is going through the organic conversion process, usually two years period of conversion to organic land. 
\source{ FAO, Statistics Division (FAOSTAT) }
\owner{ FAO }
\end{metadata}

\begin{metadata}{ Agricultural water withdrawal (m\textsuperscript{3}/yr) }{ AQ.WAT.WWAGR.MC.SH }
Annual quantity of water withdrawn for irrigation, livestock and aquaculture purposes. It includes renewable freshwater resources as well as over-abstraction of renewable groundwater or withdrawal of fossil groundwater, use of agricultural drainage water, (treated) wastewater and desalinated water. 
\source{ Land and Water Division (AQUASTAT) }
\owner{ FAO }
\end{metadata}

\begin{metadata}{ Agriculture value added per worker (constant 2000 US\$) }{ EA.PRD.AGRI.KD }
Agriculture value added per worker is a measure of agricultural productivity. Value added in agriculture measures the output of the agricultural sector (ISIC divisions 1-5) less the value of intermediate inputs. Agriculture comprises value added from forestry, hunting, and fishing as well as cultivation of crops and livestock production. Data are in constant 2000 U.S. dollars. 
\source{ World Bank (WDI) }
\owner{ Derived from World Bank national accounts files and Food and Agriculture Organization, Production Yearbook and data files. }
\end{metadata}

\begin{metadata}{ Agriculture, value added (annual \% growth) }{ NV.AGR.TOTL.KD.ZG }
Annual growth rate for agricultural value added based on constant local currency. Aggregates are based on constant 2005 U.S. dollars. Agriculture corresponds to ISIC divisions 1-5 and includes forestry, hunting, and fishing, as well as cultivation of crops and livestock production. Value added is the net output of a sector after adding up all outputs and subtracting intermediate inputs. It is calculated without making deductions for depreciation of fabricated assets or depletion and degradation of natural resources. The origin of value added is determined by the International Standard Industrial Classification (ISIC), revision 3. 
\source{ World Bank (WDI) }
\owner{ World Bank national accounts data, and OECD National Accounts data files. }
\end{metadata}

\begin{metadata}{ Agriculture, value added (percent of GDP) }{ NV.AGR.TOTL.ZS }
Agriculture corresponds to ISIC divisions 1-5 and includes forestry, hunting, and fishing, as well as cultivation of crops and livestock production. Value added is the net output of a sector after adding up all outputs and subtracting intermediate inputs. It is calculated without making deductions for depreciation of fabricated assets or depletion and degradation of natural resources. The origin of value added is determined by the International Standard Industrial Classification (ISIC), revision 3. Note: For VAB countries, gross value added at factor cost is used as the denominator. 
\source{ World Bank (WDI) }
\owner{ World Bank national accounts data, and OECD National Accounts data files. }
\end{metadata}

\begin{metadata}{ Alcoholic beverages }{ FBS.SDES.AB.PCT3D }
Includes wine, beer, fermented beverages, alcoholic beverages and non-food alcohol. 
\source{ FAO, Statistics Division (FAOSTAT) }
\owner{ FAO }
\end{metadata}

\begin{metadata}{ Average Dietary energy (available for) consumption  }{ HS.PCS.RDEC.KCD }
Measures the amount of calories consumed by the household. It is expressed in kilocalories per person per day. The dietary energy consumption is estimated from the food quantities collected in the survey. Food quantities that are collected "as purchased" (including bones, peels, etc.) first are transformed into edible quantities by taking into consideration the respective food item refuse factor and then are expressed in grams. Once all edible quantities are transformed into grams of nutrients, the nutrient densities (grams of nutrient per gram of food product) of each food item are  used to estimate the amount of calories consumed. The dietary energy consumption should be within reasonable ranges from 800 to 4,000 kcal (whichever decile), and it tends to increase as income increases (although it is also possible that better-off households purchase more expensive and less energetic food). Data are provided by gender (female and male headed households) and by area (urban and rural).
\source{ FAO, Statistics Division }
\owner{ FAO }
\end{metadata}

\begin{metadata}{ Average Dietary Energy Supply Adequacy }{ AV3YADESA.DISS }
The Dietary Energy Supply (DES) as a percentage of the Average Dietary Energy Requirement (ADER) in each country. Each country's or region's average supply of calories for food consumption is normalized by the average dietary energy requirement estimated for its population, to provide an index of adequacy of the food supply in terms of calories. Analyzed together with the prevalence of undernourishment, it allows discerning whether undernourishment is mainly due to insufficiency of the food supply or to particularly bad distribution. 
\source{ FAO, Statistics Division (FAOSTAT) }
\owner{ FAO }
\end{metadata}

\begin{metadata}{ Average fat supply (gr/caput/day) }{ FB.FSQ.GT.GCD.AV3Y.DISS }
National average fat supply (expressed in grams per caput per day).  
\source{ FAO, Statistics Division (FAOSTAT) }
\owner{ FAO }
\end{metadata}

\begin{metadata}{ Average protein supply (gr/caput/day) }{ FB.PSQ.GT.GCD.AV3Y.DISS }
National average protein supply (expressed in grams per caput per day).  
\source{ FAO, Statistics Division (FAOSTAT) }
\owner{ FAO }
\end{metadata}

\begin{metadata}{ Average supply of protein of animal origin (gr/caput/day)}{ FB.PSQ.AO.GCD.AV3Y.DISS }
National average protein supply (expressed in grams per caput per day). It includes the following groups: Meat; Offals; Animal Fats and Products; Milk and Products; Eggs, Fish, Seafood and Products; and Acquatic Products, other. 
\source{ FAO, Statistics Division (FAOSTAT) }
\owner{ FAO }
\end{metadata}

\begin{metadata}{ Cause of death }{ SH.DTH.COMM.ZS }
Includes three causes of death (in the order they appear in the tables): by communicable diseases and maternal, prenatal and nutrition conditions; by non-communicable diseases; and by injury. All three refer to the share of all deaths for all ages by underlying causes. Communicable diseases and maternal, prenatal and nutrition conditions include infectious and parasitic diseases, respiratory infections, and nutritional deficiencies such as underweight and stunting. Non-communicable diseases include cancer, diabetes mellitus, cardiovascular diseases, digestive diseases, skin diseases, musculoskeletal diseases, and congenital anomalies. Injuries include unintentional and intentional injuries. 
\source{ World Bank (WDI) }
\owner{ Derived based on the data from WHO's World Health Statistics. }
\end{metadata}

\begin{metadata}{ Cereals, excluding beer }{ FBS.SDES.CRLS.PCT3D }
Includes wheat and products, rice (milled equivalent), barley and products, maize and products, rye and products, oats, millet and products, sorghum and products, and other cereals. 
\source{ FAO, Statistics Division (FAOSTAT) }
\owner{ FAO }
\end{metadata}

\begin{metadata}{ Consumption of iodized salt (\% of households) }{ SN.ITK.SALT.ZS }
Refers to the percentage of households that use edible salt fortified with iodine. 
\source{ World Bank (WDI) }
\owner{ United Nations Children's Fund, State of the World's Children. }
\end{metadata}

\begin{metadata}{ Depth of food deficit }{ AV3YDoFD.DISS }
Indicates how many calories would be needed to lift the undernourished from their status, everything else being constant. The average intensity of food deprivation of the undernourished, estimated as the difference between the average dietary energy requirement and the average dietary energy consumption of the undernourished population (food-deprived), is multiplied by the number of undernourished to provide an estimate of the total food deficit in the country, which is then normalized by the total population 
\source{ FAO, Statistics Division (FAOSTAT) }
\owner{ FAO }
\end{metadata}

\begin{metadata}{ Dietary Energy Supply  }{ AV3YDES.DISS }
National average energy supply (expressed in calories per caput per day).  
\source{ FAO, Statistics Division }
\owner{ FAO }
\end{metadata}

\begin{metadata}{ Eggs }{ FBS.SDES.EGG.PCT3D }
Eggs. 
\source{ FAO, Statistics Division (FAOSTAT) }
\owner{ FAO }
\end{metadata}

\begin{metadata}{ Emissions in agriculture in CO\textsubscript{2}eq (gigagrams) }{ GHG.TOT.ALL.GG.NO }
Agriculture Total contains all the emissions produced in the different agricultural emissions sub-domains, providing a picture of the contribution to the total amount of GHG emissions from agriculture. GHG Emissions from agriculture consist of non-CO\textsubscript{2} gases, namely methane (CH4) and nitrous oxide (N2O), produced by crop and livestock production and management activities. 
\source{ FAO, Statistics Division (FAOSTAT) }
\owner{ FAO }
\end{metadata}

\begin{metadata}{ Exclusive breastfeeding (\% of children under 6 months) }{ SH.STA.BFED.ZS }
Refers to the percentage of children less than six months old who are fed breast milk alone (no other liquids) in the past 24 hours. 
\source{ World Bank (WDI) }
\owner{ UNICEF, State of the World's Children, Childinfo, and Demographic and Health Surveys by ICF International. }
\end{metadata}

\begin{metadata}{ Fish, seafood and aquatic products }{ FBS.SDES.FISHAP.PCT3D }
Includes freshwater fish, demersal fish, pelagic fish, other marine fish, crustaceans, cephalopods, other molluscs, other equatic animals and aquatic plants.  
\source{ FAO, Statistics Division (FAOSTAT) }
\owner{ FAO }
\end{metadata}

\begin{metadata}{ Food consumer price index }{ Value }
Covers all goods and services, and for the food and non-alcoholic beverages group. These indices measure changes over time in the prices of food that households acquire for consumption. These indices are originally compiled and disseminated by the International Labour Organisation (ILO). 
\source{ FAO, Statistics Division }
\owner{ ILO }
\end{metadata}

\begin{metadata}{ Food price index }{ Value1 }
Measures of the monthly change in international prices of a basket of food commodities. It consists of the average of five commodity group price indices, weighted with the average export shares of each of the groups for 2002-2004 
\source{ FAO, Statistics Division }
\owner{ FAO }
\end{metadata}

\begin{metadata}{ Fruit, excluding wine }{ FBS.SDES.FEW.PCT3D }
Includes oranges, madarines, lemons, limes and products, grapefruit and products, other citrus, bananas, plantains, apples and products, pineapples and products, dates, grapes and products, and other fruit. 
\source{ FAO, Statistics Division (FAOSTAT) }
\owner{ FAO }
\end{metadata}

\begin{metadata}{ GDP per capita }{ NY.GDP.PCAP.PP.KD }
Based on purchasing power parity (PPP). PPP GDP is gross domestic product converted to international dollars using purchasing power parity rates. An international dollar has the same purchasing power over GDP as the U.S. dollar has in the United States. GDP at purchaser's prices is the sum of gross value added by all resident producers in the economy plus any product taxes and minus any subsidies not included in the value of the products. It is calculated without making deductions for depreciation of fabricated assets or for depletion and degradation of natural resources. Data are in constant 2011 international dollars. 
\source{ World Bank (WDI) }
\owner{ World Bank, International Comparison Program database. }
\end{metadata}

\begin{metadata}{ Import value index (2004-2006 = 100) }{ TI.IMVAL.FOOD.IN.NO }
Value indices represent the change in the current values of Import c.i.f. (cost, insurance and freight) all expressed in US dollars. For countries which report import values on an f.o.b. (free on board) basis, these are adjusted to approximate c.i.f. values (by a standard factor of 112 percent). 
\source{ FAO, Statistics Division (FAOSTAT) }
\owner{ FAO }
\end{metadata}

\begin{metadata}{ Industry, value added (percent of GDP) }{ NV.IND.TOTL.ZS }
Industry corresponds to ISIC divisions 10-45 and includes manufacturing (ISIC divisions 15-37). It comprises value added in mining, manufacturing (also reported as a separate subgroup), construction, electricity, water, and gas. Value added is the net output of a sector after adding up all outputs and subtracting intermediate inputs. It is calculated without making deductions for depreciation of fabricated assets or depletion and degradation of natural resources. The origin of value added is determined by the International Standard Industrial Classification (ISIC), revision 3. Note: For VAB countries, gross value added at factor cost is used as the denominator. 
\source{ World Bank (WDI) }
\owner{ World Bank national accounts data, and OECD National Accounts data files. }
\end{metadata}

\begin{metadata}{ Land use, net emissions/removal in CO\textsubscript{2}eq (gigagrams) }{ GL.LU.TOT.NERCO2EQ.NO }
Greenhouse Gas (GHG) emissions data from cropland are currently limited to emissions from cultivated organic soils. They are those associated with carbon losses from drained organic soils.  
\source{ FAO, Statistics Division (FAOSTAT) }
\owner{ FAO }
\end{metadata}

\begin{metadata}{ Life expectancy at birth }{ SP.DYN.LE00.IN }
Indicates the number of years a newborn infant would live if prevailing patterns of mortality at the time of its birth were to stay the same throughout its life. 
\source{ World Bank (WDI) }
\owner{ UNPD World Population Prospects 2010 }
\end{metadata}

\begin{metadata}{ Low-birthweight babies (\% of births) }{ SH.STA.BRTW.ZS }
Newborns weighing less than 2,500 grams, with the measurement taken within the first hours of life, before significant postnatal weight loss has occurred. 
\source{ World Bank (WDI) }
\owner{ UNICEF, State of the World's Children, Childinfo, and Demographic and Health Surveys by ICF International. }
\end{metadata}

\begin{metadata}{ Manufactures Unit Value (MUV) (index) }{ MUV }
The MUV is a composite index of prices for manufactured exports from the fifteen major developed and emerging economies to low- and middle-income economies, valued in U.S. dollars. For the MUV (15) index, unit value indexes in local currency for each country are converted to U.S. dollars using market exchange rates and are combined using weights determined by the share of each country's exports in G15 exports to low- and middle-income countries. The shares are calculated using SITC revision 3 Manufactures exports data from UN COMTRADE in 2005, the base year. The primary manufacturing prices index source is OECD's Domestic Producer Price Index (PPI) for manufacturing. Whenever PPI is not available, export price indexes or the export unit values are used as proxies. The countries and relative weights (in parentheses) are: Brazil (2.95\%), Canada (0.93\%), China (11.79\%), France (5.87\%), Germany (13.29\%), India (1.77\%), Italy (6.07\%), Japan (16.70\%), Mexico (0.93\%), South Africa (0.75\%), South Korea (10.95\%), Spain (2.30\%), Thailand (2.51\%), United Kingdom (3.50\%), and United States (19.68\%). 
\source{ World Bank }
\owner{ World Bank, Development Prospects Group; Historical US GDP deflator: US Department of Commerce. }
\end{metadata}

\begin{metadata}{ Meat and offals }{ FBS.SDES.MO.PCT3D }
Includes bovine meat, mutton and goat meat, pigmeat, poultry meat, other meat, and edible offals. 
\source{ FAO, Statistics Division (FAOSTAT) }
\owner{ FAO }
\end{metadata}

\begin{metadata}{ Milk }{ FBS.SDES.MEB.PCT3D }
Excludes butter. 
\source{ FAO, Statistics Division (FAOSTAT) }
\owner{ FAO }
\end{metadata}

\begin{metadata}{ Minimum dietary diversity in infants and young children (\%) }{ N.D.MDDIYC.PCT }
Proportion of children 6-23.9 months of age who receive foods from 4 or more food groups. 
\source{ WHO }
\owner{ WHO }
\end{metadata}

\begin{metadata}{ Minimum meal frequency in infants and young children }{ N.D.MMFIYC.PCT }
Proportion of breastfed and non-breastfed children 6-23.9 months of age who receive solid, semi-solid, or soft foods or milk feeds the minimum number of times or more. 
\source{ WHO }
\owner{ WHO }
\end{metadata}

\begin{metadata}{ Mortality rate, under-5 (per 1000 live births) }{ SH.DYN.MORT }
Probability per 1,000 that a newborn baby will die before reaching age five, if subject to age-specific mortality rates of the specified year. 
\source{ World Bank (WDI) }
\owner{ Level \& Trends in Child Mortality. Report 2011. Estimates Developed by the UN Inter-agency Group for Child Mortality Estimation (UNICEF, WHO, World Bank, UN DESA, UNPD). }
\end{metadata}

\begin{metadata}{ Number of people undernourished }{ AV3YNOU.DISS }
Estimated number of people at risk of undernourishment. It is calculated by applying the estimated prevalence of undernourishment to the total population 
\source{ FAO, Statistics Division }
\owner{ FAO }
\end{metadata}

\begin{metadata}{ Protein, fat, and carbohydrates contribution to Dietary Energy Consumption }{ HS.NCDEC.RPRT.PCT }
Proportion of dietary energy provided by each macronutrient. The proportion of calories from protein and fats are estimated as their respective consumption in grams times 4 and 9, respectively. Then the calories from total carbohydrates and alcohol are estimated as the difference between total dietary energy consumption and the calories coming from protein and fats. The concept of a balanced diet is applied in more than one of the ADePT-FSM output tables. A joint WHO/FAO group of experts established guidelines for a "balanced diet", described in terms of the proportions of total dietary energy provided by diverse sources of energy (WHO 2003). These guidelines are related to the effects of chronic non deficiency diseases. So, according to the experts, a diet is determined to be balanced when: a) The proportion of dietary energy provided by protein is in the range of 10-15 percent; b) The proportion of dietary energy provided by fats is in the range of 15-30 percent; c) The proportion of total dietary energy provided by the remaining macronutrients is in the range of 55-75 percent. Data are provided by gender (female and male headed households) and by area (urban and rural).
\source{ FAO, Statistics Division }
\owner{ FAO }
\end{metadata}

\begin{metadata}{ Oilcrops }{ FBS.SDES.OCRPS.PCT3D }
Includes soyabeans, groundnuts (shelled equivalent), sunflower seed, rape and mustardseed, cottonseed, coconuts, sesame seed, palm kernels, olives and other oilcrops. 
\source{ FAO, Statistics Division (FAOSTAT) }
\owner{ FAO }
\end{metadata}

\begin{metadata}{ Open defecation (\%) }{ WSS.SAN.OD.PCT }
Defecation in fields, forests, bushes, bodies of water or other open spaces. 
\source{ WHO and UNICEF }
\owner{ JMP, Joint Monitoring Programme }
\end{metadata}

\begin{metadata}{ Percentage of population with access to improved drinking water sources  }{ SH.H2O.SAFE.ZS }
Refers to the percentage of the population with reasonable access to an adequate amount of water from an improved source, such as a household connection, public standpipe, borehole, protected well or spring, and rainwater collection. Unimproved sources include vendors, tanker trucks, and unprotected wells and springs. Reasonable access is defined as the availability of at least 20 liters a person a day from a source within one kilometer of the dwelling. 
\source{ World Bank (WDI) }
\owner{ World Health Organization and United Nations Children's Fund, Joint Measurement Programme (JMP) (http://www.wssinfo.org/). }
\end{metadata}

\begin{metadata}{ Percentage of population with access to sanitation facilities }{ SH.STA.ACSN }
Refers to the percentage of the population with at least adequate access to excreta disposal facilities that can effectively prevent human, animal, and insect contact with excreta. Improved facilities range from simple but protected pit latrines to flush toilets with a sewerage connection. To be effective, facilities must be correctly constructed and properly maintained. 
\source{ World Bank (WDI) }
\owner{ World Health Organization and United Nations Children's Fund, Joint Measurement Programme (JMP) (http://www.wssinfo.org/). }
\end{metadata}

\begin{metadata}{ Population }{ OA.TPBS.POP.P }
De facto population in a country, area or region as of 1 July of the year indicated.
\source{ FAO, Statistics Division (FAOSTAT) }
\owner{ World Population Prospects: The 2012 Revision from the UN Population Division }
\end{metadata}

\begin{metadata}{ Prevalence of anemia, children under 5 years of age }{ SH.ANM.CHLD.ZS }
Proportion of children less than 5 years showing less than 110 g/l of hemoglobine at sea level. 
\source{ World Bank (WDI) }
\owner{ 1. WHO. Global anemia prevalence and trends 1995-2011. Geneva: World Health Organization; forthcoming. 2. Stevens GA, Finucane MM, De-Regil LM, et al. Global, regional, and national trends in hemoglobin concentration and prevalence of total and severe anemia in children and pregnant and non-pregnant women for 1995-2011: a systematic analysis of population-representative data. The Lancet Global Health 2013; 1(1): e16-e25. }
\end{metadata}

\begin{metadata}{ Prevalence of anemia among non-pregnant women (\% of women ages 15-49) }{ SH.ANM.NPRG.ZS }
Percentage of non-pregnant women whose hemoglobin level is less than 120 grams per liter at sea level. 
\source{ World Bank (WDI) }
\owner{ 1. WHO. Global anemia prevalence and trends 1995-2011. Geneva: World Health Organization; forthcoming. 2. Stevens GA, Finucane MM, De-Regil LM, et al. Global, regional, and national trends in hemoglobin concentration and prevalence of total and severe anemia in children and pregnant and non-pregnant women for 1995-2011: a systematic analysis of population-representative data. The Lancet Global Health 2013; 1(1): e16-e25. }
\end{metadata}

\begin{metadata}{ Prevalence of anemia among pregnant women (\%) }{ SH.PRG.ANEM }
Percentage of pregnant women whose hemoglobin level is less than 110 grams per liter at sea level. 
\source{ World Bank (WDI) }
\owner{ 1. WHO. Global anemia prevalence and trends 1995-2011. Geneva: World Health Organization; forthcoming. 2. Stevens GA, Finucane MM, De-Regil LM, et al. Global, regional, and national trends in hemoglobin concentration and prevalence of total and severe anemia in children and pregnant and non-pregnant women for 1995-2011: a systematic analysis of population-representative data. The Lancet Global Health 2013; 1(1): e16-e25. }
\end{metadata}

\begin{metadata}{ Prevalence of food over-acquisition }{ AV3YPCO }
The percentage of individuals in a population who tend, on a regular basis, to acquire food in excess of their needs, is obtained by estimating the probability that, by randomly sampling a member of the population, the level of food consumption is found to be excessive when assessed against that person's energy requirements. 
\source{ FAO, Statistics Division }
\owner{ FAO }
\end{metadata}

\begin{metadata}{ Prevalence of iodine deficiency based on urinary excretion, children   }{ IODINE }
The proportion of school children aged 6 to 12 years of age  showing urinary iodine equal or lower than 100 g/l 
\source{ FAO, Statistics Division }
\owner{ WHO }
\end{metadata}

\begin{metadata}{ Prevalence of overweight (\% of children under 5) }{ SH.STA.OWGH.MA.ZS }
Percentage of children under age 5 whose weight for height is more than two standard deviations above the median for the international reference population of the corresponding age as established by the WHO's new child growth standards released in 2006. 
\source{ World Bank (WDI) }
\owner{ World Health Organization, Global Database on Child Growth and Malnutrition. Country-level data are unadjusted data from national surveys, and thus may not be comparable across countries. }
\end{metadata}

\begin{metadata}{ Prevalence of overweight and obesity, adults (percent) }{ N.A.OO.PCT }
The percentage of adults (ages 20+) who have a BMI (kg/m\textsuperscript{2}) greater than 25 (overweight) or greater than 30 (obese). 
\source{ WHO }
\owner{ World Health Organization }
\end{metadata}

\begin{metadata}{ Prevalence of severe wasting, weight for height (\% of children under 5) }{ SH.SVR.WAST.MA.ZS }
Proportion of children under five whose weight for height is more than three standard deviations below the median for the international reference population ages 0-59 months. 
\source{ World Bank (WDI) }
\owner{ World Health Organization, Global Database on Child Growth and Malnutrition. Country-level data are unadjusted data from national surveys, and thus may not be comparable across countries. }
\end{metadata}

\begin{metadata}{ Prevalence of undernourishment }{ AV3YPOU.DISS }
Expresses the probability that a randomly selected individual from the population consumes an amount of calories that is insufficient to cover her/his energy requirement for an active and healthy life. 
The indicator is computed by comparing a probability distribution of habitual daily Dietary Energy Consumption with a threshold level called the Minimum Dietary Energy Requirement. Both are based on the notion of an average individual in the reference population.   
\source{ FAO, Statistics Division }
\owner{ FAO }
\end{metadata}

\begin{metadata}{ Prevalence of vitamin A deficiency based on serum retinol, total pop }{ SN.ITK.VITA.ZS }
The proportion of total population with serum retinol equal or lower than 0.70 \`{e}mol/l.  
\source{ World Bank (WDI) }
\owner{ United Nations Children's Fund, State of the World's Children. }
\end{metadata}

\begin{metadata}{ Prevalence of wasting }{ SH.STA.WAST.MA.ZS }
Proportion of children under five whose weight for height is more than two standard deviations below the median for the international reference population ages 0-59 months. 
\source{ World Bank (WDI) }
\owner{ World Health Organization, Global Database on Child Growth and Malnutrition. Country-level data are unadjusted data from national surveys, and thus may not be comparable across countries. }
\end{metadata}

\begin{metadata}{ Pulses }{ FBS.SDES.PLS.PCT3D }
Includes beans, peas and other pulses and products. 
\source{ FAO, Statistics Division (FAOSTAT) }
\owner{ FAO }
\end{metadata}

\begin{metadata}{ Rural population }{ OA.TPR.POP.P }
De facto population living in areas classified as rural (that is, it is the difference between the total population of a country and its urban population). Data refer to 1 July of the year indicated and are presented in thousands. 
\source{ FAO, Statistics Division (FAOSTAT) }
\owner{ United Nations Population Division, World Urbanization Prospects. }
\end{metadata}

\begin{metadata}{ Services, etc., value added (percent of GDP) }{ NV.SRV.TETC.ZS }
Services correspond to ISIC divisions 50-99 and they include value added in wholesale and retail trade (including hotels and restaurants), transport, and government, financial, professional, and personal services such as education, health care, and real estate services. Also included are imputed bank service charges, import duties, and any statistical discrepancies noted by national compilers as well as discrepancies arising from rescaling. Value added is the net output of a sector after adding up all outputs and subtracting intermediate inputs. It is calculated without making deductions for depreciation of fabricated assets or depletion and degradation of natural resources. The industrial origin of value added is determined by the International Standard Industrial Classification (ISIC), revision 3. Note: For VAB countries, gross value added at factor cost is used as the denominator. 
\source{ World Bank (WDI) }
\owner{ World Bank national accounts data, and OECD National Accounts data files. }
\end{metadata}

\begin{metadata}{ Share of animal protein in total protein (available for) consumption (\%) }{ HS.SPC.RAP.PCT }
Proportion of protein consumption coming from food of animal origin (animal proteins). The food commodities considered to be of animal origin are meat (red and white), fish, eggs, milk, and cheese. When households are classified by income quintiles, an increasing trend in the proportion of protein of animal origin consumed as one moves from the first to the last income quintile is expected. This is mainly because richer households can afford more expensive food products such as meat and fish. However, such a trend probably is not present in pastoral regions where poor communities/households derive a sizeable part of their consumption from livestock products (i.e., milk and cheese). Data are provided by gender (female and male headed households) and by area (urban and rural).
\source{ FAO, Statistics Division }
\owner{ FAO }
\end{metadata}

\begin{metadata}{ Share of freshwater resources withdrawn by agriculture (percent) }{ AQ.WAT.RFRWAGR.MC.SH }
Water withdrawn for irrigation in a given year, expressed in percent of the total actual renewable water resources (TRWR\_actual). This parameter is an indication of the pressure on the renewable water resources caused by irrigation. 
\source{ Land and Water Division (AQUASTAT) }
\owner{ FAO }
\end{metadata}

\begin{metadata}{ Starchy roots }{ FBS.SDES.SR.PCT3D }
Includes cassava and products, potatoes and products, sweet potatoes, and other roots. 
\source{ FAO, Statistics Division (FAOSTAT) }
\owner{ FAO }
\end{metadata}

\begin{metadata}{ Stimulants }{ FBS.SDES.STM.PCT3D }
Includes coffee and products, cocoa beans and products, and tea (including mate). 
\source{ FAO, Statistics Division (FAOSTAT) }
\owner{ FAO }
\end{metadata}

\begin{metadata}{ Stunting (\% of children under 5) }{ SH.STA.STNT.MA.ZS }
Percentage of children under age 5 whose height for age (stunting) is more than two standard deviations below the median for the international reference population ages 0-59 months. For children up to two years old height is measured by recumbent length. For older children height is measured by stature while standing. The data are based on the WHO's new child growth standards released in 2006. 
\source{ World Bank (WDI) }
\owner{ World Health Organization, Global Database on Child Growth and Malnutrition. Country-level data are unadjusted data from national surveys, and thus may not be comparable across countries. }
\end{metadata}

\begin{metadata}{ Sugar and sweeteners }{ FBS.SDES.SS.PCT3D }
Includes sugar (raw equivalent), other sweeteners, and honey. 
\source{ FAO, Statistics Division (FAOSTAT) }
\owner{ FAO }
\end{metadata}

\begin{metadata}{ Total economically active population }{ OA.TEAPT.POP.PPL.NO }
Economic activity is defined by two key criteria. First economic activities take precence over non-economic activities. And two, within economic activities, the status of being employed takes precendence over the status of being unemployed. 
\source{ FAO, Statistics Division (FAOSTAT) }
\owner{ ILO, laborsta }
\end{metadata}

\begin{metadata}{ Total water withdrawal per capita (m\textsuperscript{3}/yr/person) }{ AQ.WAT.WWTOT.MC.SH }
Total annual amount of water withdrawn per capita. 
\source{ Land and Water Division (AQUASTAT) }
\owner{ FAO }
\end{metadata}

\begin{metadata}{ Tree nuts }{ FBS.SDES.TRNT.PCT3D }
Includes nuts and products. 
\source{ FAO, Statistics Division (FAOSTAT) }
\owner{ FAO }
\end{metadata}

\begin{metadata}{ Underweight (\% of children under 5) }{ SH.STA.MALN.MA.ZS }
Percentage of children under age 5 whose weight for age is more than two standard deviations below the median for the international reference population ages 0-59 months. The data are based on the WHO's new child growth standards released in 2006. 
\source{ World Bank (WDI) }
\owner{ World Health Organization, Global Database on Child Growth and Malnutrition. Country-level data are unadjusted data from national surveys, and thus may not be comparable across countries. }
\end{metadata}

\begin{metadata}{ Underweight, adults (\%) }{ SH.STA.AMALN.ZS }
Percentage of adults who are underweight, as defined by a Body Mass Index (BMI, kg/m\textsuperscript{2}) below the international reference standard of 18.5. To calculate an individual's BMI, weight and height data are need.  The BMI is weight (kg) divided by squared height (m). 
\source{ World Health Organization (WHO) }
\owner{ World Health Organization, Global Database on Body Mass Index: http://apps.who.int/bmi/index.jsp  }
\end{metadata}

\begin{metadata}{ Urban population }{ OA.TPU.POP.P }
De facto population living in areas classified as urban (that is, it is the difference between the total population of a country and its urban population). Data refer to 1 July of the year indicated and are presented in thousands. 
\source{ FAO, Statistics Division (FAOSTAT) }
\owner{ United Nations Population Division, World Urbanization Prospects. }
\end{metadata}

\begin{metadata}{ Vegetable oils and animal fats }{ FBS.SDES.VOAF.PCT3D }
Includes soyabean oil, groundnut oil, sunflowerseed oil, rape and mustard oil, cottonseed oil, palmkernel oil, palm oil, coconut oil, sesameseed oil, olive oil, maize germ oil, other oilcrops oil, butter, ghee, cream, raw animal fats, body oil (fish) and liver oil (fish). 
\source{ FAO, Statistics Division (FAOSTAT) }
\owner{ FAO }
\end{metadata}

\begin{metadata}{ Vegetables }{ FBS.SDES.VGT.PCT3D }
Includes tomatoes and products, onions, and other vegetables. 
\source{ FAO, Statistics Division (FAOSTAT) }
\owner{ FAO }
\end{metadata}

\begin{metadata}{ Vitamin A supplementation coverage rate (\% of children ages 6-59 months) }{ SN.ITK.VITA.ZS }
Refers to the percentage of children ages 6-59 months old who received at least two doses of vitamin A in the previous year. 
\source{ World Bank (WDI) }
\owner{ United Nations Children's Fund, State of the World's Children. }
\end{metadata}

\onecolumn

\newpart{white}
\phantomsection
\addcontentsline{toc}{section}{Notes}
\onecolumn
\LARGE
\textbf{Notes}

\footnotesize
The country classification adopted in this publication is based on the United Nations M49 classification \\* (\url{http://unstats.un.org/unsd/methods/m49/m49.htm}). The country names have been abbreviated for the purpose of this publication. The official FAO names can be found at \url{http://termportal.fao.org/faonocs/appl/}.

Data are shown for countries with a population greater than 120 000 people. Bermuda, Kiribati, Seychelles and St. Vincent and the Grenadines are not included due to lack of data.

Following the creation of the Republic of South Sudan in July 2011, the M49 classification considered the Sudan as part of the Northern Africa region, and South Sudan as part of Eastern Africa. In this report, data for the Sudan are therefore included in the Northern Africa region.

The asterisk in charts and maps indicates the most recent year available in the specified time interval. In the country profiles, when the country data have not been reported for the reference year, data in italics indicate that the value for the most recent year available is shown.

In the tables, a blank means not applicable or, for an aggregate, not analytically meaningful. A 0 or 0.0 means zero or a number that is small enough to round to zero at the displayed number of decimal places.

The \textasciitilde{} in the maps refers to the range specified in the class intervals.

In addition:
\begin{itemize}
\item <5.0 proportion less than 5 percent
\item <0.1 less than 100\,000 people
\item ns not statistically significant 
\end{itemize}

Two types of aggregations are used in the book: sum and weighted mean. Two restrictions are imposed when computing the aggregation: i) the sufficiency condition – the aggregation is computed only when sufficient countries have reported data, and the current threshold is set at 50 percent of the variable and the weighting variable, if present; and ii) the comparability condition – as aggregations are usually computed over time, this condition is designed to ensure that the number of countries is comparable over several years; under the current restriction the number of countries may not vary by more than 15 over time.

This publication was carried out under the direction of Pietro Gennari (Chief Statistician and Director, ESS) . It was prepared by the Statistics Division (Amy Heyman and Markus Kainu). A special thanks to colleagues in ESS, FIPS, FOE, NRC and NRL, for their invaluable input and to Nicola Selleri (ESS) for the cover design.

\clearpage

\end{MetadataCollection}

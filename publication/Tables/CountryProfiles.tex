\renewcommand{\arraystretch}{1.1}
\setlength{\tabcolsep}{4pt}
\normalsize
\CountryData{ World }
      \rowcolors{1}{FAOblue!10}{white}
      \begin{tabular}{L{3.9cm} R{1cm} R{1cm} R{1cm}}
      \toprule
      \multicolumn{1}{c}{} & \multicolumn{1}{c}{ 1992 } & \multicolumn{1}{c}{ 2002 } & \multicolumn{1}{c}{ 2014 } \\
      \midrule
	\multicolumn{4}{l}{\textcolor{FAOblue}{\textbf{\large{The setting}}}} \\ 
	 ~ Population, total (mln) & 5\,494.9 ~ \ \ & 6\,280.9 ~ \ \ & 7\,243.8 ~ \ \ \\ 
	 ~ Population, rural (\% total population) & 3\,093 ~ \ \ & 3\,284.5 ~ \ \ & 3\,362.5 ~ \ \ \\ 
	 ~ Govt expenditure on ag (\% total outlays) &  ~ \ \ &  ~ \ \ &  ~ \ \ \\ 
	 ~ Area harvested (mln ha) & 1\,974 ~ \ \ & 2\,033 ~ \ \ & 2\,781 ~ \ \ \\ 
	 ~ Cropping intensity ratio (\%) & 0.4 ~ \ \ & 0.4 ~ \ \ &  ~ \ \ \\ 
	 ~ Water resources (m\textsuperscript{3}/person/year) &  ~ \ \ &  ~ \ \ &  ~ \ \ \\ 
	 ~ Area equipped for irrigation (1000 ha) &  ~ \ \ &  ~ \ \ &  ~ \ \ \\ 
	 ~ Area irrigated (\%) &  ~ \ \ &  ~ \ \ &  ~ \ \ \\ 
	 ~ Employment in agriculture (\%) & 36.6 ~ \ \ & 31.4 ~ \ \ & \textit{30.7} ~ \ \ \\ 
	 ~ Employment in agriculture, female (\%) & 9.7 ~ \ \ & 9.8 ~ \ \ & \textit{25.2} ~ \ \ \\ 
	 ~ Fertilizers, Nitrogen (nutrients per ha) &  ~ \ \ & 17.6 ~ \ \ & \textit{24.3} ~ \ \ \\ 
	 ~ Fertilizers, Phosphate (nutrients per ha) &  ~ \ \ & 6.9 ~ \ \ & \textit{9.4} ~ \ \ \\ 
	 ~ Fertilizers, Potash (nutrients per ha) &  ~ \ \ & 4.9 ~ \ \ & \textit{5.8} ~ \ \ \\ 
	 ~ Energy consump, power irrigation (mln kWh) & 70\,398 ~ \ \ & 146\,973 ~ \ \ & \textit{325\,448} ~ \ \ \\ 
	 ~ Agr value added per worker (constant US\$) &  ~ \ \ &  ~ \ \ &  ~ \ \ \\ 
	\multicolumn{4}{l}{\textcolor{FAOblue}{\textbf{\large{Hunger dimensions}}}} \\ 
	 ~ Dietary energy supply (kcal/pc/day) & 2\,604 ~ \ \ & 2\,723 ~ \ \ & 2\,891 ~ \ \ \\ 
	 ~ Average dietary energy supply adequacy (\%) & 113 ~ \ \ & 116 ~ \ \ & 122 ~ \ \ \\ 
	 ~ Dietary en supp, cereals/roots/tubers (\%) & 58 ~ \ \ & 55 ~ \ \ & \textit{52} ~ \ \ \\ 
	 ~ Prevalence of undernourishment (\%) & 18.5 ~ \ \ & 15 ~ \ \ & 11 ~ \ \ \\ 
	 ~ GDP per capita (US\$, PPP) & 8\,786 ~ \ \ & 10\,495 ~ \ \ & \textit{13\,915} ~ \ \ \\ 
	 ~ Domestic food price volatility (index) &  ~ \ \ & 5.4 ~ \ \ & 6.4 ~ \ \ \\ 
	 ~ Cereal import dependency ratio (\%) & -1.2 ~ \ \ & -0.1 ~ \ \ & \textit{50.7} ~ \ \ \\ 
	 ~ Underweight, children under-5 (\%) &  ~ \ \ &  ~ \ \ &  ~ \ \ \\ 
	 ~ Improved water source (\% pop) & 79.3 ~ \ \ & 84 ~ \ \ & \textit{88.7} ~ \ \ \\ 
	\multicolumn{4}{l}{\textcolor{FAOblue}{\textbf{\large{Food Supply}}}} \\ 
	 ~ Food production value, (2004-2006 mln I\$) & 1\,337\,872 ~ \ \ & 1\,664\,723 ~ \ \ & \textit{2\,246\,912} ~ \ \ \\ 
	 ~ Agriculture, value added (\% GDP) &  ~ \ \ & 3 ~ \ \ & \textit{4} ~ \ \ \\ 
	 ~ Food exports (mln US\$)  & 240\,716 ~ \ \ & 305\,746 ~ \ \ & \textit{945\,572} ~ \ \ \\ 
	 ~ Food imports (mln US\$)  & 265\,438 ~ \ \ & 321\,959 ~ \ \ & \textit{967\,028} ~ \ \ \\ 
	\multicolumn{4}{l}{\textit{\normalsize{Production indices (2004-06=100)}}} \\ 
	 ~ Net food & 79 ~ \ \ & 92 ~ \ \ & \textit{121} ~ \ \ \\ 
	 ~ Net crop & 77 ~ \ \ & 89 ~ \ \ & \textit{123} ~ \ \ \\ 
	 ~ Cereal & 86 ~ \ \ & 90 ~ \ \ & \textit{123} ~ \ \ \\ 
	 ~ Vegetable oils & 54 ~ \ \ & 81 ~ \ \ & \textit{141} ~ \ \ \\ 
	 ~ Roots and tubers & 79 ~ \ \ & 95 ~ \ \ & \textit{119} ~ \ \ \\ 
	 ~ Fruit and vegetables & 62 ~ \ \ & 90 ~ \ \ & \textit{127} ~ \ \ \\ 
	 ~ Sugar & 88 ~ \ \ & 99 ~ \ \ & \textit{132} ~ \ \ \\ 
	 ~ Livestock & 83 ~ \ \ & 95 ~ \ \ & \textit{115} ~ \ \ \\ 
	 ~ Milk & 80 ~ \ \ & 93 ~ \ \ & \textit{114} ~ \ \ \\ 
	 ~ Meat & 76 ~ \ \ & 94 ~ \ \ & \textit{118} ~ \ \ \\ 
	 ~ Fish  & 74 ~ \ \ & 94 ~ \ \ & \textit{119} ~ \ \ \\ 
	\multicolumn{4}{l}{\textit{\normalsize{Net trade (min US\$)}}} \\ 
	 ~ Cereals & -2\,991 ~ \ \ & -3\,753 ~ \ \ & \textit{-7\,007} ~ \ \ \\ 
	 ~ Fruit and vegetables & -12\,186 ~ \ \ & -5\,809 ~ \ \ & \textit{-5\,826} ~ \ \ \\ 
	 ~ Meat & -1\,573 ~ \ \ & -1\,169 ~ \ \ & \textit{5\,056} ~ \ \ \\ 
	 ~ Dairy products & -43 ~ \ \ & -302 ~ \ \ & \textit{1\,169} ~ \ \ \\ 
	 ~ Fish & -5\,220 ~ \ \ & -4\,464 ~ \ \ & \textit{3\,009} ~ \ \ \\ 
	\multicolumn{4}{l}{\textcolor{FAOblue}{\textbf{\large{Environment}}}} \\ 
	 ~ Forest area (\%) & 33 ~ \ \ & 32 ~ \ \ & \textit{32} ~ \ \ \\ 
	 ~ Renewable water res withdrawn (\% of total) &  ~ \ \ &  ~ \ \ &  ~ \ \ \\ 
	 ~ Terrestrial protect areas (\% total land area)  & 9 ~ \ \ & 12 ~ \ \ & \textit{14} ~ \ \ \\ 
	 ~ Organic area (\% total agricultural area) &  ~ \ \ & \textit{1} ~ \ \ & \textit{1} ~ \ \ \\ 
	 ~ Water withdrawal by agriculture (\% of total) &  ~ \ \ &  ~ \ \ &  ~ \ \ \\ 
	 ~ Biofuel production (thousand kt of oil eq.) & 4\,634 ~ \ \ & 25\,054 ~ \ \ & \textit{381\,064} ~ \ \ \\ 
	 ~ Wood pellet prod. (min tonnes) &  ~ \ \ &  ~ \ \ & \textit{22\,096} ~ \ \ \\ 
	 ~ GHG emissions from ag (Co2 eq, gigagrams) & 7\,252 ~ \ \ & 8\,007 ~ \ \ & \textit{8\,165} ~ \ \ \\ 
       \toprule
      \end{tabular}
      \clearpage
\CountryData{ Africa }
      \rowcolors{1}{FAOblue!10}{white}
      \begin{tabular}{L{3.9cm} R{1cm} R{1cm} R{1cm}}
      \toprule
      \multicolumn{1}{c}{} & \multicolumn{1}{c}{ 1992 } & \multicolumn{1}{c}{ 2002 } & \multicolumn{1}{c}{ 2014 } \\
      \midrule
	\multicolumn{4}{l}{\textcolor{FAOblue}{\textbf{\large{The setting}}}} \\ 
	 ~ Population, total (mln) & 664 ~ \ \ & 847.9 ~ \ \ & 1\,138.2 ~ \ \ \\ 
	 ~ Population, rural (\% total population) & 445.5 ~ \ \ & 540.5 ~ \ \ & 675.5 ~ \ \ \\ 
	 ~ Govt expenditure on ag (\% total outlays) &  ~ \ \ &  ~ \ \ &  ~ \ \ \\ 
	 ~ Area harvested (mln ha) & 132 ~ \ \ & 181 ~ \ \ & 280 ~ \ \ \\ 
	 ~ Cropping intensity ratio (\%) & 0.1 ~ \ \ & 0.2 ~ \ \ &  ~ \ \ \\ 
	 ~ Water resources (m\textsuperscript{3}/person/year) &  ~ \ \ &  ~ \ \ &  ~ \ \ \\ 
	 ~ Area equipped for irrigation (1000 ha) &  ~ \ \ &  ~ \ \ &  ~ \ \ \\ 
	 ~ Area irrigated (\%) &  ~ \ \ &  ~ \ \ &  ~ \ \ \\ 
	 ~ Employment in agriculture (\%) & 30.7 ~ \ \ & 35 ~ \ \ & \textit{27.6} ~ \ \ \\ 
	 ~ Employment in agriculture, female (\%) & 42.6 ~ \ \ & 25.9 ~ \ \ & \textit{31.1} ~ \ \ \\ 
	 ~ Fertilizers, Nitrogen (nutrients per ha) &  ~ \ \ & 2.3 ~ \ \ & \textit{2.8} ~ \ \ \\ 
	 ~ Fertilizers, Phosphate (nutrients per ha) &  ~ \ \ & 0.9 ~ \ \ & \textit{1.2} ~ \ \ \\ 
	 ~ Fertilizers, Potash (nutrients per ha) &  ~ \ \ & 0.4 ~ \ \ & \textit{0.4} ~ \ \ \\ 
	 ~ Energy consump, power irrigation (mln kWh) & 10\,722 ~ \ \ & 17\,319 ~ \ \ & \textit{20\,667} ~ \ \ \\ 
	 ~ Agr value added per worker (constant US\$) &  ~ \ \ &  ~ \ \ &  ~ \ \ \\ 
	\multicolumn{4}{l}{\textcolor{FAOblue}{\textbf{\large{Hunger dimensions}}}} \\ 
	 ~ Dietary energy supply (kcal/pc/day) & 2\,329 ~ \ \ & 2\,425 ~ \ \ & 2\,572 ~ \ \ \\ 
	 ~ Average dietary energy supply adequacy (\%) & 108 ~ \ \ & 111 ~ \ \ & 117 ~ \ \ \\ 
	 ~ Dietary en supp, cereals/roots/tubers (\%) & 66 ~ \ \ & 65 ~ \ \ & \textit{63} ~ \ \ \\ 
	 ~ Prevalence of undernourishment (\%) & 27.3 ~ \ \ & 25.1 ~ \ \ & 20 ~ \ \ \\ 
	 ~ GDP per capita (US\$, PPP) & 3\,159 ~ \ \ & 3\,500 ~ \ \ & \textit{4\,575} ~ \ \ \\ 
	 ~ Domestic food price volatility (index) &  ~ \ \ & 4.1 ~ \ \ & 9.2 ~ \ \ \\ 
	 ~ Cereal import dependency ratio (\%) & 25.6 ~ \ \ & 28.2 ~ \ \ & \textit{42} ~ \ \ \\ 
	 ~ Underweight, children under-5 (\%) &  ~ \ \ &  ~ \ \ &  ~ \ \ \\ 
	 ~ Improved water source (\% pop) & 56.5 ~ \ \ & 62.3 ~ \ \ & \textit{68.7} ~ \ \ \\ 
	\multicolumn{4}{l}{\textcolor{FAOblue}{\textbf{\large{Food Supply}}}} \\ 
	 ~ Food production value, (2004-2006 mln I\$) & 99\,997 ~ \ \ & 137\,402 ~ \ \ & \textit{202\,196} ~ \ \ \\ 
	 ~ Agriculture, value added (\% GDP) &  ~ \ \ & 17 ~ \ \ & \textit{14} ~ \ \ \\ 
	 ~ Food exports (mln US\$)  & 6\,306 ~ \ \ & 9\,439 ~ \ \ & \textit{28\,015} ~ \ \ \\ 
	 ~ Food imports (mln US\$)  & 14\,267 ~ \ \ & 18\,281 ~ \ \ & \textit{70\,139} ~ \ \ \\ 
	\multicolumn{4}{l}{\textit{\normalsize{Production indices (2004-06=100)}}} \\ 
	 ~ Net food & 64 ~ \ \ & 88 ~ \ \ & \textit{129} ~ \ \ \\ 
	 ~ Net crop & 64 ~ \ \ & 87 ~ \ \ & \textit{127} ~ \ \ \\ 
	 ~ Cereal & 65 ~ \ \ & 84 ~ \ \ & \textit{131} ~ \ \ \\ 
	 ~ Vegetable oils & 60 ~ \ \ & 84 ~ \ \ & \textit{128} ~ \ \ \\ 
	 ~ Roots and tubers & 61 ~ \ \ & 86 ~ \ \ & \textit{133} ~ \ \ \\ 
	 ~ Fruit and vegetables & 64 ~ \ \ & 88 ~ \ \ & \textit{124} ~ \ \ \\ 
	 ~ Sugar & 70 ~ \ \ & 99 ~ \ \ & \textit{114} ~ \ \ \\ 
	 ~ Livestock & 67 ~ \ \ & 90 ~ \ \ & \textit{126} ~ \ \ \\ 
	 ~ Milk & 58 ~ \ \ & 88 ~ \ \ & \textit{125} ~ \ \ \\ 
	 ~ Meat & 70 ~ \ \ & 90 ~ \ \ & \textit{127} ~ \ \ \\ 
	 ~ Fish  & 72 ~ \ \ & 93 ~ \ \ & \textit{119} ~ \ \ \\ 
	\multicolumn{4}{l}{\textit{\normalsize{Net trade (min US\$)}}} \\ 
	 ~ Cereals & -6\,127 ~ \ \ & -8\,244 ~ \ \ & \textit{-28\,779} ~ \ \ \\ 
	 ~ Fruit and vegetables & 1\,110 ~ \ \ & 1\,126 ~ \ \ & \textit{4\,320} ~ \ \ \\ 
	 ~ Meat & -480 ~ \ \ & -622 ~ \ \ & \textit{-4\,517} ~ \ \ \\ 
	 ~ Dairy products & -1\,515 ~ \ \ & -1\,373 ~ \ \ & \textit{-4\,067} ~ \ \ \\ 
	 ~ Fish & 634 ~ \ \ & 1\,949 ~ \ \ & \textit{524} ~ \ \ \\ 
	\multicolumn{4}{l}{\textcolor{FAOblue}{\textbf{\large{Environment}}}} \\ 
	 ~ Forest area (\%) & 25 ~ \ \ & 24 ~ \ \ & \textit{22} ~ \ \ \\ 
	 ~ Renewable water res withdrawn (\% of total) &  ~ \ \ &  ~ \ \ &  ~ \ \ \\ 
	 ~ Terrestrial protect areas (\% total land area)  & 9 ~ \ \ & 10 ~ \ \ & \textit{14} ~ \ \ \\ 
	 ~ Organic area (\% total agricultural area) &  ~ \ \ & \textit{0} ~ \ \ & \textit{0} ~ \ \ \\ 
	 ~ Water withdrawal by agriculture (\% of total) &  ~ \ \ &  ~ \ \ &  ~ \ \ \\ 
	 ~ Biofuel production (thousand kt of oil eq.) & 214 ~ \ \ & 447 ~ \ \ & \textit{512} ~ \ \ \\ 
	 ~ Wood pellet prod. (min tonnes) &  ~ \ \ &  ~ \ \ & \textit{95} ~ \ \ \\ 
	 ~ GHG emissions from ag (Co2 eq, gigagrams) & 1\,726 ~ \ \ & 1\,720 ~ \ \ & \textit{1\,865} ~ \ \ \\ 
       \toprule
      \end{tabular}
      \clearpage
\CountryData{ Asia }
      \rowcolors{1}{FAOblue!10}{white}
      \begin{tabular}{L{3.9cm} R{1cm} R{1cm} R{1cm}}
      \toprule
      \multicolumn{1}{c}{} & \multicolumn{1}{c}{ 1992 } & \multicolumn{1}{c}{ 2002 } & \multicolumn{1}{c}{ 2014 } \\
      \midrule
	\multicolumn{4}{l}{\textcolor{FAOblue}{\textbf{\large{The setting}}}} \\ 
	 ~ Population, total (mln) & 3\,326.4 ~ \ \ & 3\,808 ~ \ \ & 4\,342.3 ~ \ \ \\ 
	 ~ Population, rural (\% total population) & 2\,221.1 ~ \ \ & 2\,330.1 ~ \ \ & 2\,293.3 ~ \ \ \\ 
	 ~ Govt expenditure on ag (\% total outlays) &  ~ \ \ &  ~ \ \ &  ~ \ \ \\ 
	 ~ Area harvested (mln ha) & 931 ~ \ \ & 985 ~ \ \ & 1\,348 ~ \ \ \\ 
	 ~ Cropping intensity ratio (\%) & 0.6 ~ \ \ & 0.6 ~ \ \ &  ~ \ \ \\ 
	 ~ Water resources (m\textsuperscript{3}/person/year) &  ~ \ \ &  ~ \ \ &  ~ \ \ \\ 
	 ~ Area equipped for irrigation (1000 ha) &  ~ \ \ &  ~ \ \ &  ~ \ \ \\ 
	 ~ Area irrigated (\%) &  ~ \ \ &  ~ \ \ &  ~ \ \ \\ 
	 ~ Employment in agriculture (\%) & 52 ~ \ \ & 44.2 ~ \ \ & \textit{38.6} ~ \ \ \\ 
	 ~ Employment in agriculture, female (\%) & 10.2 ~ \ \ & 11.5 ~ \ \ & \textit{44.2} ~ \ \ \\ 
	 ~ Fertilizers, Nitrogen (nutrients per ha) &  ~ \ \ & 31.3 ~ \ \ & \textit{47.4} ~ \ \ \\ 
	 ~ Fertilizers, Phosphate (nutrients per ha) &  ~ \ \ & 11.2 ~ \ \ & \textit{17.9} ~ \ \ \\ 
	 ~ Fertilizers, Potash (nutrients per ha) &  ~ \ \ & 5.6 ~ \ \ & \textit{8.1} ~ \ \ \\ 
	 ~ Energy consump, power irrigation (mln kWh) & 17\,878 ~ \ \ & 49\,222 ~ \ \ & \textit{82\,411} ~ \ \ \\ 
	 ~ Agr value added per worker (constant US\$) &  ~ \ \ &  ~ \ \ &  ~ \ \ \\ 
	\multicolumn{4}{l}{\textcolor{FAOblue}{\textbf{\large{Hunger dimensions}}}} \\ 
	 ~ Dietary energy supply (kcal/pc/day) & 2\,407 ~ \ \ & 2\,565 ~ \ \ & 2\,796 ~ \ \ \\ 
	 ~ Average dietary energy supply adequacy (\%) & 107 ~ \ \ & 111 ~ \ \ & 119 ~ \ \ \\ 
	 ~ Dietary en supp, cereals/roots/tubers (\%) & 67 ~ \ \ & 61 ~ \ \ & \textit{57} ~ \ \ \\ 
	 ~ Prevalence of undernourishment (\%) & 23.5 ~ \ \ & 17.8 ~ \ \ & 12.4 ~ \ \ \\ 
	 ~ GDP per capita (US\$, PPP) & 3\,237 ~ \ \ & 4\,921 ~ \ \ & \textit{9\,392} ~ \ \ \\ 
	 ~ Domestic food price volatility (index) &  ~ \ \ & 5.4 ~ \ \ & 9.3 ~ \ \ \\ 
	 ~ Cereal import dependency ratio (\%) & 5.1 ~ \ \ & 4.1 ~ \ \ & \textit{93.3} ~ \ \ \\ 
	 ~ Underweight, children under-5 (\%) &  ~ \ \ &  ~ \ \ &  ~ \ \ \\ 
	 ~ Improved water source (\% pop) &  ~ \ \ &  ~ \ \ & \textit{90.8} ~ \ \ \\ 
	\multicolumn{4}{l}{\textcolor{FAOblue}{\textbf{\large{Food Supply}}}} \\ 
	 ~ Food production value, (2004-2006 mln I\$) & 563\,113 ~ \ \ & 788\,020 ~ \ \ & \textit{1\,134\,641} ~ \ \ \\ 
	 ~ Agriculture, value added (\% GDP) &  ~ \ \ & 6 ~ \ \ & \textit{9} ~ \ \ \\ 
	 ~ Food exports (mln US\$)  & 32\,319 ~ \ \ & 47\,351 ~ \ \ & \textit{181\,141} ~ \ \ \\ 
	 ~ Food imports (mln US\$)  & 63\,308 ~ \ \ & 85\,570 ~ \ \ & \textit{325\,015} ~ \ \ \\ 
	\multicolumn{4}{l}{\textit{\normalsize{Production indices (2004-06=100)}}} \\ 
	 ~ Net food & 73 ~ \ \ & 90 ~ \ \ & \textit{128} ~ \ \ \\ 
	 ~ Net crop & 74 ~ \ \ & 88 ~ \ \ & \textit{129} ~ \ \ \\ 
	 ~ Cereal & 85 ~ \ \ & 91 ~ \ \ & \textit{123} ~ \ \ \\ 
	 ~ Vegetable oils & 50 ~ \ \ & 80 ~ \ \ & \textit{143} ~ \ \ \\ 
	 ~ Roots and tubers & 75 ~ \ \ & 99 ~ \ \ & \textit{129} ~ \ \ \\ 
	 ~ Fruit and vegetables & 47 ~ \ \ & 88 ~ \ \ & \textit{140} ~ \ \ \\ 
	 ~ Sugar & 95 ~ \ \ & 110 ~ \ \ & \textit{134} ~ \ \ \\ 
	 ~ Livestock & 71 ~ \ \ & 90 ~ \ \ & \textit{126} ~ \ \ \\ 
	 ~ Milk & 59 ~ \ \ & 84 ~ \ \ & \textit{122} ~ \ \ \\ 
	 ~ Meat & 61 ~ \ \ & 90 ~ \ \ & \textit{128} ~ \ \ \\ 
	 ~ Fish  & 58 ~ \ \ & 90 ~ \ \ & \textit{136} ~ \ \ \\ 
	\multicolumn{4}{l}{\textit{\normalsize{Net trade (min US\$)}}} \\ 
	 ~ Cereals & -11\,466 ~ \ \ & -10\,790 ~ \ \ & \textit{-35\,498} ~ \ \ \\ 
	 ~ Fruit and vegetables & -170 ~ \ \ & -1\,241 ~ \ \ & \textit{669} ~ \ \ \\ 
	 ~ Meat & -7\,151 ~ \ \ & -10\,218 ~ \ \ & \textit{-27\,384} ~ \ \ \\ 
	 ~ Dairy products & -3\,216 ~ \ \ & -4\,340 ~ \ \ & \textit{-14\,876} ~ \ \ \\ 
	 ~ Fish & -6\,533 ~ \ \ & -4\,791 ~ \ \ & \textit{6\,320} ~ \ \ \\ 
	\multicolumn{4}{l}{\textcolor{FAOblue}{\textbf{\large{Environment}}}} \\ 
	 ~ Forest area (\%) & 19 ~ \ \ & 18 ~ \ \ & \textit{18} ~ \ \ \\ 
	 ~ Renewable water res withdrawn (\% of total) &  ~ \ \ &  ~ \ \ &  ~ \ \ \\ 
	 ~ Terrestrial protect areas (\% total land area)  & 8 ~ \ \ & 12 ~ \ \ & \textit{12} ~ \ \ \\ 
	 ~ Organic area (\% total agricultural area) &  ~ \ \ & \textit{0} ~ \ \ & \textit{0} ~ \ \ \\ 
	 ~ Water withdrawal by agriculture (\% of total) &  ~ \ \ &  ~ \ \ &  ~ \ \ \\ 
	 ~ Biofuel production (thousand kt of oil eq.) & 1\,205 ~ \ \ & 4\,065 ~ \ \ & \textit{30\,089} ~ \ \ \\ 
	 ~ Wood pellet prod. (min tonnes) &  ~ \ \ &  ~ \ \ & \textit{620} ~ \ \ \\ 
	 ~ GHG emissions from ag (Co2 eq, gigagrams) & 2\,039 ~ \ \ & 2\,733 ~ \ \ & \textit{3\,409} ~ \ \ \\ 
       \toprule
      \end{tabular}
      \clearpage
\CountryData{ Latin America and the Caribbean }
      \rowcolors{1}{FAOblue!10}{white}
      \begin{tabular}{L{3.9cm} R{1cm} R{1cm} R{1cm}}
      \toprule
      \multicolumn{1}{c}{} & \multicolumn{1}{c}{ 1992 } & \multicolumn{1}{c}{ 2002 } & \multicolumn{1}{c}{ 2014 } \\
      \midrule
	\multicolumn{4}{l}{\textcolor{FAOblue}{\textbf{\large{The setting}}}} \\ 
	 ~ Population, total (mln) & 461.7 ~ \ \ & 541.2 ~ \ \ & 623.4 ~ \ \ \\ 
	 ~ Population, rural (\% total population) & 131.7 ~ \ \ & 128.8 ~ \ \ & 125 ~ \ \ \\ 
	 ~ Govt expenditure on ag (\% total outlays) &  ~ \ \ &  ~ \ \ &  ~ \ \ \\ 
	 ~ Area harvested (mln ha) &  ~ \ \ &  ~ \ \ &  ~ \ \ \\ 
	 ~ Cropping intensity ratio (\%) &  ~ \ \ &  ~ \ \ &  ~ \ \ \\ 
	 ~ Water resources (m\textsuperscript{3}/person/year) &  ~ \ \ &  ~ \ \ &  ~ \ \ \\ 
	 ~ Area equipped for irrigation (1000 ha) &  ~ \ \ &  ~ \ \ &  ~ \ \ \\ 
	 ~ Area irrigated (\%) &  ~ \ \ &  ~ \ \ &  ~ \ \ \\ 
	 ~ Employment in agriculture (\%) & 18.8 ~ \ \ & 17.7 ~ \ \ & \textit{15.8} ~ \ \ \\ 
	 ~ Employment in agriculture, female (\%) & 14.6 ~ \ \ & 10.4 ~ \ \ & \textit{7.6} ~ \ \ \\ 
	 ~ Fertilizers, Nitrogen (nutrients per ha) &  ~ \ \ &  ~ \ \ &  ~ \ \ \\ 
	 ~ Fertilizers, Phosphate (nutrients per ha) &  ~ \ \ &  ~ \ \ &  ~ \ \ \\ 
	 ~ Fertilizers, Potash (nutrients per ha) &  ~ \ \ &  ~ \ \ &  ~ \ \ \\ 
	 ~ Energy consump, power irrigation (mln kWh) &  ~ \ \ &  ~ \ \ &  ~ \ \ \\ 
	 ~ Agr value added per worker (constant US\$) &  ~ \ \ &  ~ \ \ &  ~ \ \ \\ 
	\multicolumn{4}{l}{\textcolor{FAOblue}{\textbf{\large{Hunger dimensions}}}} \\ 
	 ~ Dietary energy supply (kcal/pc/day) & 2\,687 ~ \ \ & 2\,825 ~ \ \ & 3\,046 ~ \ \ \\ 
	 ~ Average dietary energy supply adequacy (\%) & 118 ~ \ \ & 122 ~ \ \ & 128 ~ \ \ \\ 
	 ~ Dietary en supp, cereals/roots/tubers (\%) & 43 ~ \ \ & 41 ~ \ \ & \textit{40} ~ \ \ \\ 
	 ~ Prevalence of undernourishment (\%) & 14.2 ~ \ \ & 10.8 ~ \ \ & 5.7 ~ \ \ \\ 
	 ~ GDP per capita (US\$, PPP) & 9\,961 ~ \ \ & 10\,940 ~ \ \ & \textit{13\,915} ~ \ \ \\ 
	 ~ Domestic food price volatility (index) &  ~ \ \ & 3.6 ~ \ \ & 5.9 ~ \ \ \\ 
	 ~ Cereal import dependency ratio (\%) & 11.4 ~ \ \ & 11.5 ~ \ \ & \textit{49.7} ~ \ \ \\ 
	 ~ Underweight, children under-5 (\%) &  ~ \ \ &  ~ \ \ &  ~ \ \ \\ 
	 ~ Improved water source (\% pop) & 86.1 ~ \ \ & 90.5 ~ \ \ & \textit{94} ~ \ \ \\ 
	\multicolumn{4}{l}{\textcolor{FAOblue}{\textbf{\large{Food Supply}}}} \\ 
	 ~ Food production value, (2004-2006 mln I\$) & 145\,849 ~ \ \ & 204\,852 ~ \ \ & \textit{297\,533} ~ \ \ \\ 
	 ~ Agriculture, value added (\% GDP) &  ~ \ \ & 7 ~ \ \ & \textit{5} ~ \ \ \\ 
	 ~ Food exports (mln US\$)  & 20\,737 ~ \ \ & 36\,362 ~ \ \ & \textit{142\,567} ~ \ \ \\ 
	 ~ Food imports (mln US\$)  & 13\,508 ~ \ \ & 22\,995 ~ \ \ & \textit{67\,602} ~ \ \ \\ 
	\multicolumn{4}{l}{\textit{\normalsize{Production indices (2004-06=100)}}} \\ 
	 ~ Net food & 63 ~ \ \ & 88 ~ \ \ & \textit{129} ~ \ \ \\ 
	 ~ Net crop & 67 ~ \ \ & 89 ~ \ \ & \textit{131} ~ \ \ \\ 
	 ~ Cereal & 74 ~ \ \ & 87 ~ \ \ & \textit{139} ~ \ \ \\ 
	 ~ Vegetable oils &  ~ \ \ &  ~ \ \ &  ~ \ \ \\ 
	 ~ Roots and tubers &  ~ \ \ &  ~ \ \ &  ~ \ \ \\ 
	 ~ Fruit and vegetables &  ~ \ \ &  ~ \ \ &  ~ \ \ \\ 
	 ~ Sugar &  ~ \ \ &  ~ \ \ &  ~ \ \ \\ 
	 ~ Livestock & 61 ~ \ \ & 88 ~ \ \ & \textit{123} ~ \ \ \\ 
	 ~ Milk &  ~ \ \ &  ~ \ \ &  ~ \ \ \\ 
	 ~ Meat &  ~ \ \ &  ~ \ \ &  ~ \ \ \\ 
	 ~ Fish  & 95 ~ \ \ & 97 ~ \ \ & \textit{71} ~ \ \ \\ 
	\multicolumn{4}{l}{\textit{\normalsize{Net trade (min US\$)}}} \\ 
	 ~ Cereals & -2\,520 ~ \ \ & -3\,880 ~ \ \ & \textit{1\,077} ~ \ \ \\ 
	 ~ Fruit and vegetables & 6\,457 ~ \ \ & 8\,935 ~ \ \ & \textit{23\,534} ~ \ \ \\ 
	 ~ Meat & 966 ~ \ \ & 1\,752 ~ \ \ & \textit{14\,206} ~ \ \ \\ 
	 ~ Dairy products & -1\,330 ~ \ \ & -1\,020 ~ \ \ & \textit{-2\,237} ~ \ \ \\ 
	 ~ Fish & 2\,703 ~ \ \ & 3\,966 ~ \ \ & \textit{9\,752} ~ \ \ \\ 
	\multicolumn{4}{l}{\textcolor{FAOblue}{\textbf{\large{Environment}}}} \\ 
	 ~ Forest area (\%) & 51 ~ \ \ & 49 ~ \ \ & \textit{47} ~ \ \ \\ 
	 ~ Renewable water res withdrawn (\% of total) &  ~ \ \ &  ~ \ \ &  ~ \ \ \\ 
	 ~ Terrestrial protect areas (\% total land area)  & 11 ~ \ \ & 17 ~ \ \ & \textit{21} ~ \ \ \\ 
	 ~ Organic area (\% total agricultural area) &  ~ \ \ & \textit{1} ~ \ \ & \textit{1} ~ \ \ \\ 
	 ~ Water withdrawal by agriculture (\% of total) &  ~ \ \ &  ~ \ \ &  ~ \ \ \\ 
	 ~ Biofuel production (thousand kt of oil eq.) & 1\,763 ~ \ \ & 2\,146 ~ \ \ & \textit{61\,706} ~ \ \ \\ 
	 ~ Wood pellet prod. (min tonnes) &  ~ \ \ &  ~ \ \ & \textit{123} ~ \ \ \\ 
	 ~ GHG emissions from ag (Co2 eq, gigagrams) & 2\,365 ~ \ \ & 2\,588 ~ \ \ & \textit{2\,335} ~ \ \ \\ 
       \toprule
      \end{tabular}
      \clearpage
\CountryData{ Oceania }
      \rowcolors{1}{FAOblue!10}{white}
      \begin{tabular}{L{3.9cm} R{1cm} R{1cm} R{1cm}}
      \toprule
      \multicolumn{1}{c}{} & \multicolumn{1}{c}{ 1992 } & \multicolumn{1}{c}{ 2002 } & \multicolumn{1}{c}{ 2014 } \\
      \midrule
	\multicolumn{4}{l}{\textcolor{FAOblue}{\textbf{\large{The setting}}}} \\ 
	 ~ Population, total (mln) & 27.8 ~ \ \ & 32.1 ~ \ \ & 38.8 ~ \ \ \\ 
	 ~ Population, rural (\% total population) & 8.2 ~ \ \ & 9.5 ~ \ \ & 11.3 ~ \ \ \\ 
	 ~ Govt expenditure on ag (\% total outlays) &  ~ \ \ &  ~ \ \ &  ~ \ \ \\ 
	 ~ Area harvested (mln ha) & 26 ~ \ \ & 35 ~ \ \ & 37 ~ \ \ \\ 
	 ~ Cropping intensity ratio (\%) & 0.1 ~ \ \ & 0.1 ~ \ \ &  ~ \ \ \\ 
	 ~ Water resources (m\textsuperscript{3}/person/year) &  ~ \ \ &  ~ \ \ &  ~ \ \ \\ 
	 ~ Area equipped for irrigation (1000 ha) &  ~ \ \ &  ~ \ \ &  ~ \ \ \\ 
	 ~ Area irrigated (\%) &  ~ \ \ &  ~ \ \ &  ~ \ \ \\ 
	 ~ Employment in agriculture (\%) & 6.2 ~ \ \ & 5.2 ~ \ \ & \textit{3.8} ~ \ \ \\ 
	 ~ Employment in agriculture, female (\%) & 4.5 ~ \ \ & 3.5 ~ \ \ & \textit{4.4} ~ \ \ \\ 
	 ~ Fertilizers, Nitrogen (nutrients per ha) &  ~ \ \ & 2.8 ~ \ \ & \textit{3.4} ~ \ \ \\ 
	 ~ Fertilizers, Phosphate (nutrients per ha) &  ~ \ \ & 3.3 ~ \ \ & \textit{3.2} ~ \ \ \\ 
	 ~ Fertilizers, Potash (nutrients per ha) &  ~ \ \ & 0.8 ~ \ \ & \textit{0.6} ~ \ \ \\ 
	 ~ Energy consump, power irrigation (mln kWh) & 1\,028 ~ \ \ & 1\,028 ~ \ \ & \textit{8\,667} ~ \ \ \\ 
	 ~ Agr value added per worker (constant US\$) &  ~ \ \ &  ~ \ \ &  ~ \ \ \\ 
	\multicolumn{4}{l}{\textcolor{FAOblue}{\textbf{\large{Hunger dimensions}}}} \\ 
	 ~ Dietary energy supply (kcal/pc/day) & 2\,469 ~ \ \ & 2\,435 ~ \ \ & 2\,543 ~ \ \ \\ 
	 ~ Average dietary energy supply adequacy (\%) & 114 ~ \ \ & 112 ~ \ \ & 115 ~ \ \ \\ 
	 ~ Dietary en supp, cereals/roots/tubers (\%) & 49 ~ \ \ & 50 ~ \ \ & \textit{48} ~ \ \ \\ 
	 ~ Prevalence of undernourishment (\%) & 15.4 ~ \ \ & 16.9 ~ \ \ & 14 ~ \ \ \\ 
	 ~ GDP per capita (US\$, PPP) & 2\,524 ~ \ \ & 2\,467 ~ \ \ & \textit{3\,110} ~ \ \ \\ 
	 ~ Domestic food price volatility (index) &  ~ \ \ & 7.1 ~ \ \ & 8.3 ~ \ \ \\ 
	 ~ Cereal import dependency ratio (\%) & 95.1 ~ \ \ & 95.5 ~ \ \ & \textit{95.4} ~ \ \ \\ 
	 ~ Underweight, children under-5 (\%) &  ~ \ \ &  ~ \ \ &  ~ \ \ \\ 
	 ~ Improved water source (\% pop) & 49.6 ~ \ \ & 53.5 ~ \ \ & \textit{55.5} ~ \ \ \\ 
	\multicolumn{4}{l}{\textcolor{FAOblue}{\textbf{\large{Food Supply}}}} \\ 
	 ~ Food production value, (2004-2006 mln I\$) & 25\,515 ~ \ \ & 30\,955 ~ \ \ & \textit{38\,664} ~ \ \ \\ 
	 ~ Agriculture, value added (\% GDP) &  ~ \ \ & 5 ~ \ \ & \textit{3} ~ \ \ \\ 
	 ~ Food exports (mln US\$)  & 10\,866 ~ \ \ & 17\,098 ~ \ \ & \textit{45\,536} ~ \ \ \\ 
	 ~ Food imports (mln US\$)  & 2\,247 ~ \ \ & 3\,706 ~ \ \ & \textit{12\,669} ~ \ \ \\ 
	\multicolumn{4}{l}{\textit{\normalsize{Production indices (2004-06=100)}}} \\ 
	 ~ Net food & 77 ~ \ \ & 93 ~ \ \ & \textit{116} ~ \ \ \\ 
	 ~ Net crop & 76 ~ \ \ & 84 ~ \ \ & \textit{126} ~ \ \ \\ 
	 ~ Cereal & 82 ~ \ \ & 63 ~ \ \ & \textit{117} ~ \ \ \\ 
	 ~ Vegetable oils & 55 ~ \ \ & 91 ~ \ \ & \textit{215} ~ \ \ \\ 
	 ~ Roots and tubers & 80 ~ \ \ & 98 ~ \ \ & \textit{110} ~ \ \ \\ 
	 ~ Fruit and vegetables & 70 ~ \ \ & 93 ~ \ \ & \textit{104} ~ \ \ \\ 
	 ~ Sugar & 60 ~ \ \ & 86 ~ \ \ & \textit{71} ~ \ \ \\ 
	 ~ Livestock & 82 ~ \ \ & 100 ~ \ \ & \textit{107} ~ \ \ \\ 
	 ~ Milk & 60 ~ \ \ & 100 ~ \ \ & \textit{113} ~ \ \ \\ 
	 ~ Meat & 85 ~ \ \ & 98 ~ \ \ & \textit{105} ~ \ \ \\ 
	 ~ Fish  & 63 ~ \ \ & 82 ~ \ \ & \textit{85} ~ \ \ \\ 
	\multicolumn{4}{l}{\textit{\normalsize{Net trade (min US\$)}}} \\ 
	 ~ Cereals & 1\,533 ~ \ \ & 2\,695 ~ \ \ & \textit{8\,606} ~ \ \ \\ 
	 ~ Fruit and vegetables & 701 ~ \ \ & 886 ~ \ \ & \textit{1\,257} ~ \ \ \\ 
	 ~ Meat & 4\,104 ~ \ \ & 4\,905 ~ \ \ & \textit{10\,465} ~ \ \ \\ 
	 ~ Dairy products & 1\,764 ~ \ \ & 3\,706 ~ \ \ & \textit{10\,513} ~ \ \ \\ 
	 ~ Fish & 662 ~ \ \ & 872 ~ \ \ & \textit{1\,676} ~ \ \ \\ 
	\multicolumn{4}{l}{\textcolor{FAOblue}{\textbf{\large{Environment}}}} \\ 
	 ~ Forest area (\%) & 23 ~ \ \ & 23 ~ \ \ & \textit{22} ~ \ \ \\ 
	 ~ Renewable water res withdrawn (\% of total) &  ~ \ \ &  ~ \ \ &  ~ \ \ \\ 
	 ~ Terrestrial protect areas (\% total land area)  & 8 ~ \ \ & 10 ~ \ \ & \textit{13} ~ \ \ \\ 
	 ~ Organic area (\% total agricultural area) &  ~ \ \ & \textit{3} ~ \ \ & \textit{3} ~ \ \ \\ 
	 ~ Water withdrawal by agriculture (\% of total) &  ~ \ \ &  ~ \ \ &  ~ \ \ \\ 
	 ~ Biofuel production (thousand kt of oil eq.) & 215 ~ \ \ & 325 ~ \ \ & \textit{1\,990} ~ \ \ \\ 
	 ~ Wood pellet prod. (min tonnes) &  ~ \ \ &  ~ \ \ & \textit{2} ~ \ \ \\ 
	 ~ GHG emissions from ag (Co2 eq, gigagrams) & 286 ~ \ \ & 354 ~ \ \ & \textit{340} ~ \ \ \\ 
       \toprule
      \end{tabular}
      \clearpage
\CountryData{ Afghanistan }
      \rowcolors{1}{FAOblue!10}{white}
      \begin{tabular}{L{3.9cm} R{1cm} R{1cm} R{1cm}}
      \toprule
      \multicolumn{1}{c}{} & \multicolumn{1}{c}{ 1992 } & \multicolumn{1}{c}{ 2002 } & \multicolumn{1}{c}{ 2014 } \\
      \midrule
	\multicolumn{4}{l}{\textcolor{FAOblue}{\textbf{\large{The setting}}}} \\ 
	 ~ Population, total (mln) & 13.8 ~ \ \ & 22.2 ~ \ \ & 31.3 ~ \ \ \\ 
	 ~ Population, rural (\% total population) & 11.2 ~ \ \ & 17.5 ~ \ \ & 23.6 ~ \ \ \\ 
	 ~ Govt expenditure on ag (\% total outlays) &  ~ \ \ & \textit{4.7} ~ \ \ & \textit{4.3} ~ \ \ \\ 
	 ~ Area harvested (mln ha) & 2 ~ \ \ & 4 ~ \ \ & 7 ~ \ \ \\ 
	 ~ Cropping intensity ratio (\%) & 0.1 ~ \ \ & 0.1 ~ \ \ &  ~ \ \ \\ 
	 ~ Water resources (m\textsuperscript{3}/person/year) & \textit{4} ~ \ \ & \textit{3} ~ \ \ & \textit{2} ~ \ \ \\ 
	 ~ Area equipped for irrigation (1000 ha) &  ~ \ \ &  ~ \ \ & \textit{3\,208} ~ \ \ \\ 
	 ~ Area irrigated (\%) &  ~ \ \ &  ~ \ \ & \textit{59.1} ~ \ \ \\ 
	 ~ Employment in agriculture (\%) &  ~ \ \ &  ~ \ \ &  ~ \ \ \\ 
	 ~ Employment in agriculture, female (\%) &  ~ \ \ &  ~ \ \ &  ~ \ \ \\ 
	 ~ Fertilizers, Nitrogen (nutrients per ha) &  ~ \ \ & 0.6 ~ \ \ & \textit{0.9} ~ \ \ \\ 
	 ~ Fertilizers, Phosphate (nutrients per ha) &  ~ \ \ & 0.1 ~ \ \ & \textit{0} ~ \ \ \\ 
	 ~ Fertilizers, Potash (nutrients per ha) &  ~ \ \ & 0 ~ \ \ & \textit{0} ~ \ \ \\ 
	 ~ Energy consump, power irrigation (mln kWh) & 275 ~ \ \ & 275 ~ \ \ & \textit{275} ~ \ \ \\ 
	 ~ Agr value added per worker (constant US\$) &  ~ \ \ & 0.4 ~ \ \ & \textit{0.4} ~ \ \ \\ 
	\multicolumn{4}{l}{\textcolor{FAOblue}{\textbf{\large{Hunger dimensions}}}} \\ 
	 ~ Dietary energy supply (kcal/pc/day) & 1\,944 ~ \ \ & 1\,820 ~ \ \ & 2\,089 ~ \ \ \\ 
	 ~ Average dietary energy supply adequacy (\%) & 95 ~ \ \ & 90 ~ \ \ & 100 ~ \ \ \\ 
	 ~ Dietary en supp, cereals/roots/tubers (\%) & 75 ~ \ \ & 78 ~ \ \ & \textit{76} ~ \ \ \\ 
	 ~ Prevalence of undernourishment (\%) & 35.6 ~ \ \ & 46.7 ~ \ \ & 26 ~ \ \ \\ 
	 ~ GDP per capita (US\$, PPP) &  ~ \ \ & 1\,053 ~ \ \ & \textit{1\,884} ~ \ \ \\ 
	 ~ Domestic food price volatility (index) &  ~ \ \ &  ~ \ \ &  ~ \ \ \\ 
	 ~ Cereal import dependency ratio (\%) & 6.1 ~ \ \ & 27.4 ~ \ \ & \textit{22.1} ~ \ \ \\ 
	 ~ Underweight, children under-5 (\%) &  ~ \ \ & \textit{32.9} ~ \ \ &  ~ \ \ \\ 
	 ~ Improved water source (\% pop) & 4.9 ~ \ \ & 29.1 ~ \ \ & \textit{64.2} ~ \ \ \\ 
	\multicolumn{4}{l}{\textcolor{FAOblue}{\textbf{\large{Food Supply}}}} \\ 
	 ~ Food production value, (2004-2006 mln I\$) & 1\,932 ~ \ \ & 2\,711 ~ \ \ & \textit{3\,393} ~ \ \ \\ 
	 ~ Agriculture, value added (\% GDP) &  ~ \ \ & 38 ~ \ \ & \textit{24} ~ \ \ \\ 
	 ~ Food exports (mln US\$)  & 20 ~ \ \ & 35 ~ \ \ & \textit{161} ~ \ \ \\ 
	 ~ Food imports (mln US\$)  & 85 ~ \ \ & 216 ~ \ \ & \textit{1\,081} ~ \ \ \\ 
	\multicolumn{4}{l}{\textit{\normalsize{Production indices (2004-06=100)}}} \\ 
	 ~ Net food & 68 ~ \ \ & 96 ~ \ \ & \textit{120} ~ \ \ \\ 
	 ~ Net crop & 66 ~ \ \ & 88 ~ \ \ & \textit{134} ~ \ \ \\ 
	 ~ Cereal & 54 ~ \ \ & 82 ~ \ \ & \textit{142} ~ \ \ \\ 
	 ~ Vegetable oils & 104 ~ \ \ & 87 ~ \ \ & \textit{95} ~ \ \ \\ 
	 ~ Roots and tubers & 76 ~ \ \ & 75 ~ \ \ & \textit{100} ~ \ \ \\ 
	 ~ Fruit and vegetables & 77 ~ \ \ & 97 ~ \ \ & \textit{120} ~ \ \ \\ 
	 ~ Sugar & 65 ~ \ \ & 65 ~ \ \ & \textit{180} ~ \ \ \\ 
	 ~ Livestock & 73 ~ \ \ & 104 ~ \ \ & \textit{104} ~ \ \ \\ 
	 ~ Milk & 55 ~ \ \ & 108 ~ \ \ & \textit{108} ~ \ \ \\ 
	 ~ Meat & 83 ~ \ \ & 101 ~ \ \ & \textit{101} ~ \ \ \\ 
	 ~ Fish  & 103 ~ \ \ & 93 ~ \ \ & \textit{141} ~ \ \ \\ 
	\multicolumn{4}{l}{\textit{\normalsize{Net trade (min US\$)}}} \\ 
	 ~ Cereals &  ~ \ \ &  ~ \ \ & \textit{-372} ~ \ \ \\ 
	 ~ Fruit and vegetables & 18 ~ \ \ & 24 ~ \ \ & \textit{-42} ~ \ \ \\ 
	 ~ Meat &  ~ \ \ &  ~ \ \ &  ~ \ \ \\ 
	 ~ Dairy products &  ~ \ \ &  ~ \ \ & \textit{-63} ~ \ \ \\ 
	 ~ Fish &  ~ \ \ &  ~ \ \ &  ~ \ \ \\ 
	\multicolumn{4}{l}{\textcolor{FAOblue}{\textbf{\large{Environment}}}} \\ 
	 ~ Forest area (\%) & 2 ~ \ \ & 2 ~ \ \ & \textit{2} ~ \ \ \\ 
	 ~ Renewable water res withdrawn (\% of total) &  ~ \ \ & \textit{99} ~ \ \ & 99 ~ \ \ \\ 
	 ~ Terrestrial protect areas (\% total land area)  & 0 ~ \ \ & 0 ~ \ \ & \textit{0} ~ \ \ \\ 
	 ~ Organic area (\% total agricultural area) &  ~ \ \ &  ~ \ \ & \textit{0} ~ \ \ \\ 
	 ~ Water withdrawal by agriculture (\% of total) &  ~ \ \ & \textit{99} ~ \ \ & 99 ~ \ \ \\ 
	 ~ Biofuel production (thousand kt of oil eq.) &  ~ \ \ &  ~ \ \ &  ~ \ \ \\ 
	 ~ Wood pellet prod. (min tonnes) &  ~ \ \ &  ~ \ \ &  ~ \ \ \\ 
	 ~ GHG emissions from ag (Co2 eq, gigagrams) & 8 ~ \ \ & 10 ~ \ \ & \textit{14} ~ \ \ \\ 
       \toprule
      \end{tabular}
      \clearpage
\CountryData{ Albania }
      \rowcolors{1}{FAOblue!10}{white}
      \begin{tabular}{L{3.9cm} R{1cm} R{1cm} R{1cm}}
      \toprule
      \multicolumn{1}{c}{} & \multicolumn{1}{c}{ 1992 } & \multicolumn{1}{c}{ 2002 } & \multicolumn{1}{c}{ 2014 } \\
      \midrule
	\multicolumn{4}{l}{\textcolor{FAOblue}{\textbf{\large{The setting}}}} \\ 
	 ~ Population, total (mln) & 3.4 ~ \ \ & 3.3 ~ \ \ & 3.2 ~ \ \ \\ 
	 ~ Population, rural (\% total population) & 2.2 ~ \ \ & 1.8 ~ \ \ & 1.4 ~ \ \ \\ 
	 ~ Govt expenditure on ag (\% total outlays) &  ~ \ \ &  ~ \ \ &  ~ \ \ \\ 
	 ~ Area harvested (mln ha) & 1 ~ \ \ & 1 ~ \ \ & 1 ~ \ \ \\ 
	 ~ Cropping intensity ratio (\%) & 0.5 ~ \ \ & 0.6 ~ \ \ &  ~ \ \ \\ 
	 ~ Water resources (m\textsuperscript{3}/person/year) & \textit{9} ~ \ \ & \textit{9} ~ \ \ & \textit{10} ~ \ \ \\ 
	 ~ Area equipped for irrigation (1000 ha) &  ~ \ \ &  ~ \ \ & \textit{331} ~ \ \ \\ 
	 ~ Area irrigated (\%) &  ~ \ \ &  ~ \ \ & \textit{60.8} ~ \ \ \\ 
	 ~ Employment in agriculture (\%) & \textit{68.4} ~ \ \ & 57.7 ~ \ \ & \textit{41.5} ~ \ \ \\ 
	 ~ Employment in agriculture, female (\%) &  ~ \ \ &  ~ \ \ & \textit{52.6} ~ \ \ \\ 
	 ~ Fertilizers, Nitrogen (nutrients per ha) &  ~ \ \ & 33.3 ~ \ \ & \textit{28.8} ~ \ \ \\ 
	 ~ Fertilizers, Phosphate (nutrients per ha) &  ~ \ \ & 16 ~ \ \ & \textit{18} ~ \ \ \\ 
	 ~ Fertilizers, Potash (nutrients per ha) &  ~ \ \ & 0 ~ \ \ & \textit{0} ~ \ \ \\ 
	 ~ Energy consump, power irrigation (mln kWh) & 0 ~ \ \ & 0 ~ \ \ & \textit{0} ~ \ \ \\ 
	 ~ Agr value added per worker (constant US\$) & 1.5 ~ \ \ & 2.6 ~ \ \ & \textit{3.8} ~ \ \ \\ 
	\multicolumn{4}{l}{\textcolor{FAOblue}{\textbf{\large{Hunger dimensions}}}} \\ 
	 ~ Dietary energy supply (kcal/pc/day) &  ~ \ \ &  ~ \ \ &  ~ \ \ \\ 
	 ~ Average dietary energy supply adequacy (\%) & 114 ~ \ \ & 117 ~ \ \ & 120 ~ \ \ \\ 
	 ~ Dietary en supp, cereals/roots/tubers (\%) & 59 ~ \ \ & 48 ~ \ \ & \textit{41} ~ \ \ \\ 
	 ~ Prevalence of undernourishment (\%) & <5.0 ~ \ \ & <5.0 ~ \ \ & <5.0 ~ \ \ \\ 
	 ~ GDP per capita (US\$, PPP) & 2\,877 ~ \ \ & 5\,913 ~ \ \ & \textit{9\,961} ~ \ \ \\ 
	 ~ Domestic food price volatility (index) &  ~ \ \ & 12 ~ \ \ & 10.3 ~ \ \ \\ 
	 ~ Cereal import dependency ratio (\%) & 48.3 ~ \ \ & 49 ~ \ \ & \textit{40.6} ~ \ \ \\ 
	 ~ Underweight, children under-5 (\%) &  ~ \ \ & \textit{6.6} ~ \ \ & \textit{6.3} ~ \ \ \\ 
	 ~ Improved water source (\% pop) & \textit{96.4} ~ \ \ & 96.4 ~ \ \ & \textit{95.7} ~ \ \ \\ 
	\multicolumn{4}{l}{\textcolor{FAOblue}{\textbf{\large{Food Supply}}}} \\ 
	 ~ Food production value, (2004-2006 mln I\$) & 593 ~ \ \ & 861 ~ \ \ & \textit{1\,222} ~ \ \ \\ 
	 ~ Agriculture, value added (\% GDP) & 52 ~ \ \ & 26 ~ \ \ & \textit{22} ~ \ \ \\ 
	 ~ Food exports (mln US\$)  & 4 ~ \ \ & 3 ~ \ \ & \textit{40} ~ \ \ \\ 
	 ~ Food imports (mln US\$)  & 246 ~ \ \ & 227 ~ \ \ & \textit{589} ~ \ \ \\ 
	\multicolumn{4}{l}{\textit{\normalsize{Production indices (2004-06=100)}}} \\ 
	 ~ Net food & 62 ~ \ \ & 90 ~ \ \ & \textit{128} ~ \ \ \\ 
	 ~ Net crop & 67 ~ \ \ & 86 ~ \ \ & \textit{149} ~ \ \ \\ 
	 ~ Cereal & 79 ~ \ \ & 103 ~ \ \ & \textit{131} ~ \ \ \\ 
	 ~ Vegetable oils & 51 ~ \ \ & 65 ~ \ \ & \textit{196} ~ \ \ \\ 
	 ~ Roots and tubers & 35 ~ \ \ & 97 ~ \ \ & \textit{119} ~ \ \ \\ 
	 ~ Fruit and vegetables & 61 ~ \ \ & 83 ~ \ \ & \textit{152} ~ \ \ \\ 
	 ~ Sugar & 107 ~ \ \ & 90 ~ \ \ & \textit{92} ~ \ \ \\ 
	 ~ Livestock & 61 ~ \ \ & 94 ~ \ \ & \textit{110} ~ \ \ \\ 
	 ~ Milk & 57 ~ \ \ & 93 ~ \ \ & \textit{107} ~ \ \ \\ 
	 ~ Meat & 70 ~ \ \ & 96 ~ \ \ & \textit{113} ~ \ \ \\ 
	 ~ Fish  & 46 ~ \ \ & 67 ~ \ \ & \textit{110} ~ \ \ \\ 
	\multicolumn{4}{l}{\textit{\normalsize{Net trade (min US\$)}}} \\ 
	 ~ Cereals & -111 ~ \ \ & -82 ~ \ \ & \textit{-203} ~ \ \ \\ 
	 ~ Fruit and vegetables & 1 ~ \ \ & -52 ~ \ \ & \textit{-65} ~ \ \ \\ 
	 ~ Meat & -64 ~ \ \ & -27 ~ \ \ & \textit{-69} ~ \ \ \\ 
	 ~ Dairy products &  ~ \ \ & -8 ~ \ \ & \textit{-21} ~ \ \ \\ 
	 ~ Fish &  ~ \ \ &  ~ \ \ &  ~ \ \ \\ 
	\multicolumn{4}{l}{\textcolor{FAOblue}{\textbf{\large{Environment}}}} \\ 
	 ~ Forest area (\%) & 29 ~ \ \ & 28 ~ \ \ & \textit{28} ~ \ \ \\ 
	 ~ Renewable water res withdrawn (\% of total) &  ~ \ \ &  ~ \ \ & 40 ~ \ \ \\ 
	 ~ Terrestrial protect areas (\% total land area)  & 3 ~ \ \ & 7 ~ \ \ & \textit{11} ~ \ \ \\ 
	 ~ Organic area (\% total agricultural area) &  ~ \ \ & \textit{0} ~ \ \ & \textit{0} ~ \ \ \\ 
	 ~ Water withdrawal by agriculture (\% of total) &  ~ \ \ &  ~ \ \ & 40 ~ \ \ \\ 
	 ~ Biofuel production (thousand kt of oil eq.) &  ~ \ \ &  ~ \ \ &  ~ \ \ \\ 
	 ~ Wood pellet prod. (min tonnes) &  ~ \ \ &  ~ \ \ & \textit{5} ~ \ \ \\ 
	 ~ GHG emissions from ag (Co2 eq, gigagrams) & 3 ~ \ \ & 4 ~ \ \ & \textit{2} ~ \ \ \\ 
       \toprule
      \end{tabular}
      \clearpage
\CountryData{ Algeria }
      \rowcolors{1}{FAOblue!10}{white}
      \begin{tabular}{L{3.9cm} R{1cm} R{1cm} R{1cm}}
      \toprule
      \multicolumn{1}{c}{} & \multicolumn{1}{c}{ 1992 } & \multicolumn{1}{c}{ 2002 } & \multicolumn{1}{c}{ 2014 } \\
      \midrule
	\multicolumn{4}{l}{\textcolor{FAOblue}{\textbf{\large{The setting}}}} \\ 
	 ~ Population, total (mln) & 27.5 ~ \ \ & 32.6 ~ \ \ & 39.9 ~ \ \ \\ 
	 ~ Population, rural (\% total population) & 12.8 ~ \ \ & 12 ~ \ \ & 9.8 ~ \ \ \\ 
	 ~ Govt expenditure on ag (\% total outlays) &  ~ \ \ &  ~ \ \ &  ~ \ \ \\ 
	 ~ Area harvested (mln ha) & 4 ~ \ \ & 3 ~ \ \ & 5 ~ \ \ \\ 
	 ~ Cropping intensity ratio (\%) & 0.1 ~ \ \ & 0.1 ~ \ \ &  ~ \ \ \\ 
	 ~ Water resources (m\textsuperscript{3}/person/year) & \textit{0} ~ \ \ & \textit{0} ~ \ \ & \textit{0} ~ \ \ \\ 
	 ~ Area equipped for irrigation (1000 ha) &  ~ \ \ &  ~ \ \ & \textit{570} ~ \ \ \\ 
	 ~ Area irrigated (\%) &  ~ \ \ & \textit{79.6} ~ \ \ &  ~ \ \ \\ 
	 ~ Employment in agriculture (\%) &  ~ \ \ & \textit{20.7} ~ \ \ & \textit{10.8} ~ \ \ \\ 
	 ~ Employment in agriculture, female (\%) &  ~ \ \ & \textit{22.3} ~ \ \ & \textit{3} ~ \ \ \\ 
	 ~ Fertilizers, Nitrogen (nutrients per ha) &  ~ \ \ & 0.7 ~ \ \ & \textit{1.8} ~ \ \ \\ 
	 ~ Fertilizers, Phosphate (nutrients per ha) &  ~ \ \ & 0.6 ~ \ \ & \textit{1.3} ~ \ \ \\ 
	 ~ Fertilizers, Potash (nutrients per ha) &  ~ \ \ & 0.5 ~ \ \ & \textit{0.8} ~ \ \ \\ 
	 ~ Energy consump, power irrigation (mln kWh) & 96 ~ \ \ & 96 ~ \ \ & \textit{96} ~ \ \ \\ 
	 ~ Agr value added per worker (constant US\$) & 2.3 ~ \ \ & 2.2 ~ \ \ & \textit{3.7} ~ \ \ \\ 
	\multicolumn{4}{l}{\textcolor{FAOblue}{\textbf{\large{Hunger dimensions}}}} \\ 
	 ~ Dietary energy supply (kcal/pc/day) & 2\,822 ~ \ \ & 2\,927 ~ \ \ & 3\,310 ~ \ \ \\ 
	 ~ Average dietary energy supply adequacy (\%) & 130 ~ \ \ & 126 ~ \ \ & 143 ~ \ \ \\ 
	 ~ Dietary en supp, cereals/roots/tubers (\%) & 59 ~ \ \ & 60 ~ \ \ & \textit{55} ~ \ \ \\ 
	 ~ Prevalence of undernourishment (\%) & 7.4 ~ \ \ & 8 ~ \ \ & <5.0 ~ \ \ \\ 
	 ~ GDP per capita (US\$, PPP) & 9\,693 ~ \ \ & 10\,634 ~ \ \ & \textit{12\,893} ~ \ \ \\ 
	 ~ Domestic food price volatility (index) &  ~ \ \ & 12.4 ~ \ \ & 5.5 ~ \ \ \\ 
	 ~ Cereal import dependency ratio (\%) & 64.2 ~ \ \ & 71.5 ~ \ \ & \textit{67.6} ~ \ \ \\ 
	 ~ Underweight, children under-5 (\%) & 9.2 ~ \ \ & 11.1 ~ \ \ & \textit{3.7} ~ \ \ \\ 
	 ~ Improved water source (\% pop) & 94.1 ~ \ \ & 88.1 ~ \ \ & \textit{83.9} ~ \ \ \\ 
	\multicolumn{4}{l}{\textcolor{FAOblue}{\textbf{\large{Food Supply}}}} \\ 
	 ~ Food production value, (2004-2006 mln I\$) & 3\,387 ~ \ \ & 3\,975 ~ \ \ & \textit{8\,218} ~ \ \ \\ 
	 ~ Agriculture, value added (\% GDP) & 12 ~ \ \ & 10 ~ \ \ & \textit{11} ~ \ \ \\ 
	 ~ Food exports (mln US\$)  & 65 ~ \ \ & 29 ~ \ \ & \textit{277} ~ \ \ \\ 
	 ~ Food imports (mln US\$)  & 2\,204 ~ \ \ & 2\,728 ~ \ \ & \textit{8\,478} ~ \ \ \\ 
	\multicolumn{4}{l}{\textit{\normalsize{Production indices (2004-06=100)}}} \\ 
	 ~ Net food & 65 ~ \ \ & 76 ~ \ \ & \textit{158} ~ \ \ \\ 
	 ~ Net crop & 61 ~ \ \ & 67 ~ \ \ & \textit{167} ~ \ \ \\ 
	 ~ Cereal & 82 ~ \ \ & 49 ~ \ \ & \textit{128} ~ \ \ \\ 
	 ~ Vegetable oils & 79 ~ \ \ & 57 ~ \ \ & \textit{165} ~ \ \ \\ 
	 ~ Roots and tubers & 53 ~ \ \ & 63 ~ \ \ & \textit{240} ~ \ \ \\ 
	 ~ Fruit and vegetables & 55 ~ \ \ & 74 ~ \ \ & \textit{171} ~ \ \ \\ 
	 ~ Sugar &  ~ \ \ &  ~ \ \ &  ~ \ \ \\ 
	 ~ Livestock & 73 ~ \ \ & 91 ~ \ \ & \textit{145} ~ \ \ \\ 
	 ~ Milk & 56 ~ \ \ & 81 ~ \ \ & \textit{170} ~ \ \ \\ 
	 ~ Meat & 81 ~ \ \ & 96 ~ \ \ & \textit{127} ~ \ \ \\ 
	 ~ Fish  & 74 ~ \ \ & 105 ~ \ \ & \textit{79} ~ \ \ \\ 
	\multicolumn{4}{l}{\textit{\normalsize{Net trade (min US\$)}}} \\ 
	 ~ Cereals & -770 ~ \ \ & -1\,297 ~ \ \ & \textit{-3\,494} ~ \ \ \\ 
	 ~ Fruit and vegetables & -51 ~ \ \ & -245 ~ \ \ & \textit{-897} ~ \ \ \\ 
	 ~ Meat & -43 ~ \ \ & -36 ~ \ \ & \textit{-260} ~ \ \ \\ 
	 ~ Dairy products & -636 ~ \ \ & -487 ~ \ \ & \textit{-1\,263} ~ \ \ \\ 
	 ~ Fish &  ~ \ \ &  ~ \ \ &  ~ \ \ \\ 
	\multicolumn{4}{l}{\textcolor{FAOblue}{\textbf{\large{Environment}}}} \\ 
	 ~ Forest area (\%) & 1 ~ \ \ & 1 ~ \ \ & \textit{1} ~ \ \ \\ 
	 ~ Renewable water res withdrawn (\% of total) &  ~ \ \ & \textit{61} ~ \ \ & 61 ~ \ \ \\ 
	 ~ Terrestrial protect areas (\% total land area)  & 6 ~ \ \ & 6 ~ \ \ & \textit{7} ~ \ \ \\ 
	 ~ Organic area (\% total agricultural area) &  ~ \ \ & \textit{0} ~ \ \ & \textit{0} ~ \ \ \\ 
	 ~ Water withdrawal by agriculture (\% of total) &  ~ \ \ & \textit{61} ~ \ \ & 61 ~ \ \ \\ 
	 ~ Biofuel production (thousand kt of oil eq.) &  ~ \ \ &  ~ \ \ &  ~ \ \ \\ 
	 ~ Wood pellet prod. (min tonnes) &  ~ \ \ &  ~ \ \ &  ~ \ \ \\ 
	 ~ GHG emissions from ag (Co2 eq, gigagrams) & 9 ~ \ \ & 9 ~ \ \ & \textit{13} ~ \ \ \\ 
       \toprule
      \end{tabular}
      \clearpage
\CountryData{ Angola }
      \rowcolors{1}{FAOblue!10}{white}
      \begin{tabular}{L{3.9cm} R{1cm} R{1cm} R{1cm}}
      \toprule
      \multicolumn{1}{c}{} & \multicolumn{1}{c}{ 1992 } & \multicolumn{1}{c}{ 2002 } & \multicolumn{1}{c}{ 2014 } \\
      \midrule
	\multicolumn{4}{l}{\textcolor{FAOblue}{\textbf{\large{The setting}}}} \\ 
	 ~ Population, total (mln) & 11 ~ \ \ & 14.9 ~ \ \ & 22.1 ~ \ \ \\ 
	 ~ Population, rural (\% total population) & 6.6 ~ \ \ & 7.3 ~ \ \ & 8.5 ~ \ \ \\ 
	 ~ Govt expenditure on ag (\% total outlays) &  ~ \ \ & 0.7 ~ \ \ & \textit{1.4} ~ \ \ \\ 
	 ~ Area harvested (mln ha) & 2 ~ \ \ & 7 ~ \ \ & 18 ~ \ \ \\ 
	 ~ Cropping intensity ratio (\%) & 0 ~ \ \ & 0.1 ~ \ \ &  ~ \ \ \\ 
	 ~ Water resources (m\textsuperscript{3}/person/year) & \textit{13} ~ \ \ & \textit{10} ~ \ \ & \textit{7} ~ \ \ \\ 
	 ~ Area equipped for irrigation (1000 ha) &  ~ \ \ &  ~ \ \ & \textit{86} ~ \ \ \\ 
	 ~ Area irrigated (\%) &  ~ \ \ & \textit{29} ~ \ \ &  ~ \ \ \\ 
	 ~ Employment in agriculture (\%) & 5.1 ~ \ \ &  ~ \ \ &  ~ \ \ \\ 
	 ~ Employment in agriculture, female (\%) &  ~ \ \ &  ~ \ \ &  ~ \ \ \\ 
	 ~ Fertilizers, Nitrogen (nutrients per ha) &  ~ \ \ & 0 ~ \ \ & \textit{0.4} ~ \ \ \\ 
	 ~ Fertilizers, Phosphate (nutrients per ha) &  ~ \ \ & 0 ~ \ \ & \textit{0.2} ~ \ \ \\ 
	 ~ Fertilizers, Potash (nutrients per ha) &  ~ \ \ & 0 ~ \ \ & \textit{0.2} ~ \ \ \\ 
	 ~ Energy consump, power irrigation (mln kWh) &  ~ \ \ &  ~ \ \ &  ~ \ \ \\ 
	 ~ Agr value added per worker (constant US\$) & 0.3 ~ \ \ & 0.3 ~ \ \ & \textit{0.9} ~ \ \ \\ 
	\multicolumn{4}{l}{\textcolor{FAOblue}{\textbf{\large{Hunger dimensions}}}} \\ 
	 ~ Dietary energy supply (kcal/pc/day) & 1\,599 ~ \ \ & 1\,912 ~ \ \ & 2\,502 ~ \ \ \\ 
	 ~ Average dietary energy supply adequacy (\%) & 78 ~ \ \ & 93 ~ \ \ & 120 ~ \ \ \\ 
	 ~ Dietary en supp, cereals/roots/tubers (\%) & 61 ~ \ \ & 65 ~ \ \ & \textit{59} ~ \ \ \\ 
	 ~ Prevalence of undernourishment (\%) & 64.5 ~ \ \ & 45.8 ~ \ \ & 15.3 ~ \ \ \\ 
	 ~ GDP per capita (US\$, PPP) & 3\,656 ~ \ \ & 3\,758 ~ \ \ & \textit{7\,488} ~ \ \ \\ 
	 ~ Domestic food price volatility (index) &  ~ \ \ & 29.9 ~ \ \ & \textit{13.7} ~ \ \ \\ 
	 ~ Cereal import dependency ratio (\%) & 48.4 ~ \ \ & 54 ~ \ \ & \textit{56.7} ~ \ \ \\ 
	 ~ Underweight, children under-5 (\%) &  ~ \ \ & \textit{27.5} ~ \ \ & \textit{15.6} ~ \ \ \\ 
	 ~ Improved water source (\% pop) & 42.4 ~ \ \ & 46.8 ~ \ \ & \textit{54.3} ~ \ \ \\ 
	\multicolumn{4}{l}{\textcolor{FAOblue}{\textbf{\large{Food Supply}}}} \\ 
	 ~ Food production value, (2004-2006 mln I\$) & 871 ~ \ \ & 1\,693 ~ \ \ & \textit{4\,440} ~ \ \ \\ 
	 ~ Agriculture, value added (\% GDP) & 10 ~ \ \ & 9 ~ \ \ & \textit{10} ~ \ \ \\ 
	 ~ Food exports (mln US\$)  & 0 ~ \ \ & 1 ~ \ \ & \textit{19} ~ \ \ \\ 
	 ~ Food imports (mln US\$)  & 366 ~ \ \ & 504 ~ \ \ & \textit{2\,707} ~ \ \ \\ 
	\multicolumn{4}{l}{\textit{\normalsize{Production indices (2004-06=100)}}} \\ 
	 ~ Net food & 42 ~ \ \ & 81 ~ \ \ & \textit{213} ~ \ \ \\ 
	 ~ Net crop & 33 ~ \ \ & 74 ~ \ \ & \textit{236} ~ \ \ \\ 
	 ~ Cereal & 50 ~ \ \ & 94 ~ \ \ & \textit{204} ~ \ \ \\ 
	 ~ Vegetable oils & 70 ~ \ \ & 86 ~ \ \ & \textit{144} ~ \ \ \\ 
	 ~ Roots and tubers & 22 ~ \ \ & 74 ~ \ \ & \textit{191} ~ \ \ \\ 
	 ~ Fruit and vegetables & 48 ~ \ \ & 63 ~ \ \ & \textit{356} ~ \ \ \\ 
	 ~ Sugar & 71 ~ \ \ & 95 ~ \ \ & \textit{134} ~ \ \ \\ 
	 ~ Livestock & 72 ~ \ \ & 106 ~ \ \ & \textit{136} ~ \ \ \\ 
	 ~ Milk & 69 ~ \ \ & 100 ~ \ \ & \textit{83} ~ \ \ \\ 
	 ~ Meat & 71 ~ \ \ & 106 ~ \ \ & \textit{154} ~ \ \ \\ 
	 ~ Fish  & 51 ~ \ \ & 115 ~ \ \ & \textit{124} ~ \ \ \\ 
	\multicolumn{4}{l}{\textit{\normalsize{Net trade (min US\$)}}} \\ 
	 ~ Cereals & -83 ~ \ \ & -151 ~ \ \ & \textit{-746} ~ \ \ \\ 
	 ~ Fruit and vegetables & -22 ~ \ \ & -53 ~ \ \ & \textit{-209} ~ \ \ \\ 
	 ~ Meat & -108 ~ \ \ & -104 ~ \ \ & \textit{-809} ~ \ \ \\ 
	 ~ Dairy products & -26 ~ \ \ & -10 ~ \ \ & \textit{-82} ~ \ \ \\ 
	 ~ Fish &  ~ \ \ &  ~ \ \ &  ~ \ \ \\ 
	\multicolumn{4}{l}{\textcolor{FAOblue}{\textbf{\large{Environment}}}} \\ 
	 ~ Forest area (\%) & 49 ~ \ \ & 48 ~ \ \ & \textit{47} ~ \ \ \\ 
	 ~ Renewable water res withdrawn (\% of total) &  ~ \ \ & \textit{21} ~ \ \ & 21 ~ \ \ \\ 
	 ~ Terrestrial protect areas (\% total land area)  & 12 ~ \ \ & 12 ~ \ \ & \textit{12} ~ \ \ \\ 
	 ~ Organic area (\% total agricultural area) &  ~ \ \ &  ~ \ \ & \textit{0} ~ \ \ \\ 
	 ~ Water withdrawal by agriculture (\% of total) &  ~ \ \ & \textit{21} ~ \ \ & 21 ~ \ \ \\ 
	 ~ Biofuel production (thousand kt of oil eq.) & 1 ~ \ \ & \textit{0} ~ \ \ &  ~ \ \ \\ 
	 ~ Wood pellet prod. (min tonnes) &  ~ \ \ &  ~ \ \ &  ~ \ \ \\ 
	 ~ GHG emissions from ag (Co2 eq, gigagrams) & 79 ~ \ \ & 77 ~ \ \ & \textit{79} ~ \ \ \\ 
       \toprule
      \end{tabular}
      \clearpage
\CountryData{ Argentina }
      \rowcolors{1}{FAOblue!10}{white}
      \begin{tabular}{L{3.9cm} R{1cm} R{1cm} R{1cm}}
      \toprule
      \multicolumn{1}{c}{} & \multicolumn{1}{c}{ 1992 } & \multicolumn{1}{c}{ 2002 } & \multicolumn{1}{c}{ 2014 } \\
      \midrule
	\multicolumn{4}{l}{\textcolor{FAOblue}{\textbf{\large{The setting}}}} \\ 
	 ~ Population, total (mln) & 33.5 ~ \ \ & 37.6 ~ \ \ & 41.8 ~ \ \ \\ 
	 ~ Population, rural (\% total population) & 4.1 ~ \ \ & 3.5 ~ \ \ & 2.9 ~ \ \ \\ 
	 ~ Govt expenditure on ag (\% total outlays) &  ~ \ \ &  ~ \ \ &  ~ \ \ \\ 
	 ~ Area harvested (mln ha) & 26 ~ \ \ & 32 ~ \ \ & 51 ~ \ \ \\ 
	 ~ Cropping intensity ratio (\%) & 0.2 ~ \ \ & 0.2 ~ \ \ &  ~ \ \ \\ 
	 ~ Water resources (m\textsuperscript{3}/person/year) & \textit{26} ~ \ \ & \textit{23} ~ \ \ & \textit{21} ~ \ \ \\ 
	 ~ Area equipped for irrigation (1000 ha) &  ~ \ \ &  ~ \ \ & \textit{2\,360} ~ \ \ \\ 
	 ~ Area irrigated (\%) &  ~ \ \ &  ~ \ \ & \textit{91.7} ~ \ \ \\ 
	 ~ Employment in agriculture (\%) & 0.4 ~ \ \ & 1 ~ \ \ & \textit{0.6} ~ \ \ \\ 
	 ~ Employment in agriculture, female (\%) & 0.3 ~ \ \ & 0.5 ~ \ \ & \textit{0.3} ~ \ \ \\ 
	 ~ Fertilizers, Nitrogen (nutrients per ha) &  ~ \ \ & 4.2 ~ \ \ & \textit{6.5} ~ \ \ \\ 
	 ~ Fertilizers, Phosphate (nutrients per ha) &  ~ \ \ & 2.2 ~ \ \ & \textit{3.5} ~ \ \ \\ 
	 ~ Fertilizers, Potash (nutrients per ha) &  ~ \ \ & 0.2 ~ \ \ & \textit{0.3} ~ \ \ \\ 
	 ~ Energy consump, power irrigation (mln kWh) & 142 ~ \ \ & 885 ~ \ \ & \textit{885} ~ \ \ \\ 
	 ~ Agr value added per worker (constant US\$) & 7.5 ~ \ \ & 9.7 ~ \ \ & \textit{12.1} ~ \ \ \\ 
	\multicolumn{4}{l}{\textcolor{FAOblue}{\textbf{\large{Hunger dimensions}}}} \\ 
	 ~ Dietary energy supply (kcal/pc/day) & 3\,067 ~ \ \ & 3\,070 ~ \ \ & 3\,606 ~ \ \ \\ 
	 ~ Average dietary energy supply adequacy (\%) & 130 ~ \ \ & 128 ~ \ \ & 149 ~ \ \ \\ 
	 ~ Dietary en supp, cereals/roots/tubers (\%) & 34 ~ \ \ & 37 ~ \ \ & \textit{36} ~ \ \ \\ 
	 ~ Prevalence of undernourishment (\%) & <5.0 ~ \ \ & <5.0 ~ \ \ & <5.0 ~ \ \ \\ 
	 ~ GDP per capita (US\$, PPP) &  ~ \ \ &  ~ \ \ &  ~ \ \ \\ 
	 ~ Domestic food price volatility (index) &  ~ \ \ &  ~ \ \ &  ~ \ \ \\ 
	 ~ Cereal import dependency ratio (\%) & -111.1 ~ \ \ & -163.7 ~ \ \ & \textit{-168.5} ~ \ \ \\ 
	 ~ Underweight, children under-5 (\%) & \textit{1.7} ~ \ \ & \textit{2.3} ~ \ \ & \textit{2.3} ~ \ \ \\ 
	 ~ Improved water source (\% pop) & 94.4 ~ \ \ & 96.9 ~ \ \ & \textit{98.7} ~ \ \ \\ 
	\multicolumn{4}{l}{\textcolor{FAOblue}{\textbf{\large{Food Supply}}}} \\ 
	 ~ Food production value, (2004-2006 mln I\$) & 23\,113 ~ \ \ & 30\,257 ~ \ \ & \textit{42\,146} ~ \ \ \\ 
	 ~ Agriculture, value added (\% GDP) & 6 ~ \ \ & 11 ~ \ \ & \textit{7} ~ \ \ \\ 
	 ~ Food exports (mln US\$)  & 5\,160 ~ \ \ & 7\,762 ~ \ \ & \textit{27\,534} ~ \ \ \\ 
	 ~ Food imports (mln US\$)  & 650 ~ \ \ & 315 ~ \ \ & \textit{1\,084} ~ \ \ \\ 
	\multicolumn{4}{l}{\textit{\normalsize{Production indices (2004-06=100)}}} \\ 
	 ~ Net food & 65 ~ \ \ & 85 ~ \ \ & \textit{118} ~ \ \ \\ 
	 ~ Net crop & 58 ~ \ \ & 86 ~ \ \ & \textit{126} ~ \ \ \\ 
	 ~ Cereal & 70 ~ \ \ & 87 ~ \ \ & \textit{141} ~ \ \ \\ 
	 ~ Vegetable oils & 47 ~ \ \ & 86 ~ \ \ & \textit{129} ~ \ \ \\ 
	 ~ Roots and tubers & 82 ~ \ \ & 114 ~ \ \ & \textit{105} ~ \ \ \\ 
	 ~ Fruit and vegetables & 77 ~ \ \ & 90 ~ \ \ & \textit{102} ~ \ \ \\ 
	 ~ Sugar & 70 ~ \ \ & 80 ~ \ \ & \textit{99} ~ \ \ \\ 
	 ~ Livestock & 81 ~ \ \ & 83 ~ \ \ & \textit{111} ~ \ \ \\ 
	 ~ Milk & 72 ~ \ \ & 93 ~ \ \ & \textit{124} ~ \ \ \\ 
	 ~ Meat & 84 ~ \ \ & 80 ~ \ \ & \textit{107} ~ \ \ \\ 
	 ~ Fish  & 73 ~ \ \ & 93 ~ \ \ & \textit{86} ~ \ \ \\ 
	\multicolumn{4}{l}{\textit{\normalsize{Net trade (min US\$)}}} \\ 
	 ~ Cereals & 1\,579 ~ \ \ & 2\,271 ~ \ \ & \textit{10\,586} ~ \ \ \\ 
	 ~ Fruit and vegetables & 575 ~ \ \ & 823 ~ \ \ & \textit{2\,897} ~ \ \ \\ 
	 ~ Meat & 642 ~ \ \ & 553 ~ \ \ & \textit{1\,830} ~ \ \ \\ 
	 ~ Dairy products & -89 ~ \ \ & 288 ~ \ \ & \textit{1\,260} ~ \ \ \\ 
	 ~ Fish &  ~ \ \ &  ~ \ \ &  ~ \ \ \\ 
	\multicolumn{4}{l}{\textcolor{FAOblue}{\textbf{\large{Environment}}}} \\ 
	 ~ Forest area (\%) & 12 ~ \ \ & 11 ~ \ \ & \textit{11} ~ \ \ \\ 
	 ~ Renewable water res withdrawn (\% of total) &  ~ \ \ &  ~ \ \ & 74 ~ \ \ \\ 
	 ~ Terrestrial protect areas (\% total land area)  & 5 ~ \ \ & 5 ~ \ \ & \textit{7} ~ \ \ \\ 
	 ~ Organic area (\% total agricultural area) &  ~ \ \ & \textit{2} ~ \ \ & \textit{2} ~ \ \ \\ 
	 ~ Water withdrawal by agriculture (\% of total) &  ~ \ \ &  ~ \ \ & 74 ~ \ \ \\ 
	 ~ Biofuel production (thousand kt of oil eq.) & 76 ~ \ \ & 71 ~ \ \ & \textit{49\,086} ~ \ \ \\ 
	 ~ Wood pellet prod. (min tonnes) &  ~ \ \ &  ~ \ \ & \textit{10} ~ \ \ \\ 
	 ~ GHG emissions from ag (Co2 eq, gigagrams) & 175 ~ \ \ & 181 ~ \ \ & \textit{174} ~ \ \ \\ 
       \toprule
      \end{tabular}
      \clearpage
\CountryData{ Armenia }
      \rowcolors{1}{FAOblue!10}{white}
      \begin{tabular}{L{3.9cm} R{1cm} R{1cm} R{1cm}}
      \toprule
      \multicolumn{1}{c}{} & \multicolumn{1}{c}{ 1992 } & \multicolumn{1}{c}{ 2002 } & \multicolumn{1}{c}{ 2014 } \\
      \midrule
	\multicolumn{4}{l}{\textcolor{FAOblue}{\textbf{\large{The setting}}}} \\ 
	 ~ Population, total (mln) & 3.4 ~ \ \ & 3 ~ \ \ & 3 ~ \ \ \\ 
	 ~ Population, rural (\% total population) & 1.1 ~ \ \ & 1.1 ~ \ \ & 1.1 ~ \ \ \\ 
	 ~ Govt expenditure on ag (\% total outlays) &  ~ \ \ & 5.5 ~ \ \ & \textit{4.3} ~ \ \ \\ 
	 ~ Area harvested (mln ha) & 1 ~ \ \ & 1 ~ \ \ & 1 ~ \ \ \\ 
	 ~ Cropping intensity ratio (\%) & 0.5 ~ \ \ & 0.4 ~ \ \ &  ~ \ \ \\ 
	 ~ Water resources (m\textsuperscript{3}/person/year) & \textit{2} ~ \ \ & \textit{3} ~ \ \ & \textit{3} ~ \ \ \\ 
	 ~ Area equipped for irrigation (1000 ha) &  ~ \ \ &  ~ \ \ & \textit{274} ~ \ \ \\ 
	 ~ Area irrigated (\%) &  ~ \ \ &  ~ \ \ & \textit{64.4} ~ \ \ \\ 
	 ~ Employment in agriculture (\%) &  ~ \ \ & 45.3 ~ \ \ & \textit{38.9} ~ \ \ \\ 
	 ~ Employment in agriculture, female (\%) &  ~ \ \ & 42.6 ~ \ \ & \textit{45.5} ~ \ \ \\ 
	 ~ Fertilizers, Nitrogen (nutrients per ha) &  ~ \ \ & 11.3 ~ \ \ & \textit{8.5} ~ \ \ \\ 
	 ~ Fertilizers, Phosphate (nutrients per ha) &  ~ \ \ & 0 ~ \ \ & \textit{0.2} ~ \ \ \\ 
	 ~ Fertilizers, Potash (nutrients per ha) &  ~ \ \ & 0 ~ \ \ & \textit{0.2} ~ \ \ \\ 
	 ~ Energy consump, power irrigation (mln kWh) &  ~ \ \ &  ~ \ \ & \textit{63} ~ \ \ \\ 
	 ~ Agr value added per worker (constant US\$) & 2.2 ~ \ \ & 4.2 ~ \ \ & \textit{9.1} ~ \ \ \\ 
	\multicolumn{4}{l}{\textcolor{FAOblue}{\textbf{\large{Hunger dimensions}}}} \\ 
	 ~ Dietary energy supply (kcal/pc/day) & 2\,103 ~ \ \ & 2\,330 ~ \ \ & 2\,862 ~ \ \ \\ 
	 ~ Average dietary energy supply adequacy (\%) & 91 ~ \ \ & 98 ~ \ \ & 118 ~ \ \ \\ 
	 ~ Dietary en supp, cereals/roots/tubers (\%) & 69 ~ \ \ & 57 ~ \ \ & \textit{43} ~ \ \ \\ 
	 ~ Prevalence of undernourishment (\%) & 27.3 ~ \ \ & 22.3 ~ \ \ & 6.3 ~ \ \ \\ 
	 ~ GDP per capita (US\$, PPP) & 1\,973 ~ \ \ & 3\,654 ~ \ \ & \textit{7\,527} ~ \ \ \\ 
	 ~ Domestic food price volatility (index) &  ~ \ \ & 12.7 ~ \ \ & 11.9 ~ \ \ \\ 
	 ~ Cereal import dependency ratio (\%) & 61.2 ~ \ \ & 51.3 ~ \ \ & \textit{55.7} ~ \ \ \\ 
	 ~ Underweight, children under-5 (\%) &  ~ \ \ & \textit{4.2} ~ \ \ & \textit{5.3} ~ \ \ \\ 
	 ~ Improved water source (\% pop) & 90.7 ~ \ \ & 93.7 ~ \ \ & \textit{99.8} ~ \ \ \\ 
	\multicolumn{4}{l}{\textcolor{FAOblue}{\textbf{\large{Food Supply}}}} \\ 
	 ~ Food production value, (2004-2006 mln I\$) & 622 ~ \ \ & 606 ~ \ \ & \textit{1\,087} ~ \ \ \\ 
	 ~ Agriculture, value added (\% GDP) & 31 ~ \ \ & 26 ~ \ \ & \textit{22} ~ \ \ \\ 
	 ~ Food exports (mln US\$)  & 0 ~ \ \ & 7 ~ \ \ & \textit{59} ~ \ \ \\ 
	 ~ Food imports (mln US\$)  & 147 ~ \ \ & 140 ~ \ \ & \textit{589} ~ \ \ \\ 
	\multicolumn{4}{l}{\textit{\normalsize{Production indices (2004-06=100)}}} \\ 
	 ~ Net food & 74 ~ \ \ & 72 ~ \ \ & \textit{129} ~ \ \ \\ 
	 ~ Net crop & 66 ~ \ \ & 62 ~ \ \ & \textit{132} ~ \ \ \\ 
	 ~ Cereal & 76 ~ \ \ & 115 ~ \ \ & \textit{161} ~ \ \ \\ 
	 ~ Vegetable oils &  ~ \ \ &  ~ \ \ &  ~ \ \ \\ 
	 ~ Roots and tubers & 44 ~ \ \ & 52 ~ \ \ & \textit{117} ~ \ \ \\ 
	 ~ Fruit and vegetables & 69 ~ \ \ & 57 ~ \ \ & \textit{129} ~ \ \ \\ 
	 ~ Sugar & 6 ~ \ \ & 6 ~ \ \ & \textit{488} ~ \ \ \\ 
	 ~ Livestock & 82 ~ \ \ & 83 ~ \ \ & \textit{124} ~ \ \ \\ 
	 ~ Milk & 66 ~ \ \ & 83 ~ \ \ & \textit{112} ~ \ \ \\ 
	 ~ Meat & 107 ~ \ \ & 82 ~ \ \ & \textit{142} ~ \ \ \\ 
	 ~ Fish  & 390 ~ \ \ & 130 ~ \ \ & \textit{1\,435} ~ \ \ \\ 
	\multicolumn{4}{l}{\textit{\normalsize{Net trade (min US\$)}}} \\ 
	 ~ Cereals & \textit{-66} ~ \ \ & -61 ~ \ \ & \textit{-182} ~ \ \ \\ 
	 ~ Fruit and vegetables & -23 ~ \ \ & -3 ~ \ \ & \textit{-27} ~ \ \ \\ 
	 ~ Meat & \textit{-12} ~ \ \ & -19 ~ \ \ & \textit{-89} ~ \ \ \\ 
	 ~ Dairy products & \textit{-40} ~ \ \ & -8 ~ \ \ & \textit{-34} ~ \ \ \\ 
	 ~ Fish &  ~ \ \ &  ~ \ \ &  ~ \ \ \\ 
	\multicolumn{4}{l}{\textcolor{FAOblue}{\textbf{\large{Environment}}}} \\ 
	 ~ Forest area (\%) & 12 ~ \ \ & 10 ~ \ \ & \textit{9} ~ \ \ \\ 
	 ~ Renewable water res withdrawn (\% of total) &  ~ \ \ &  ~ \ \ & 66 ~ \ \ \\ 
	 ~ Terrestrial protect areas (\% total land area)  & 7 ~ \ \ & 8 ~ \ \ & \textit{8} ~ \ \ \\ 
	 ~ Organic area (\% total agricultural area) &  ~ \ \ & \textit{0} ~ \ \ & \textit{0} ~ \ \ \\ 
	 ~ Water withdrawal by agriculture (\% of total) &  ~ \ \ &  ~ \ \ & 66 ~ \ \ \\ 
	 ~ Biofuel production (thousand kt of oil eq.) &  ~ \ \ &  ~ \ \ &  ~ \ \ \\ 
	 ~ Wood pellet prod. (min tonnes) &  ~ \ \ &  ~ \ \ & \textit{0} ~ \ \ \\ 
	 ~ GHG emissions from ag (Co2 eq, gigagrams) & 1 ~ \ \ & 2 ~ \ \ & \textit{2} ~ \ \ \\ 
       \toprule
      \end{tabular}
      \clearpage
\CountryData{ Australia }
      \rowcolors{1}{FAOblue!10}{white}
      \begin{tabular}{L{3.9cm} R{1cm} R{1cm} R{1cm}}
      \toprule
      \multicolumn{1}{c}{} & \multicolumn{1}{c}{ 1992 } & \multicolumn{1}{c}{ 2002 } & \multicolumn{1}{c}{ 2014 } \\
      \midrule
	\multicolumn{4}{l}{\textcolor{FAOblue}{\textbf{\large{The setting}}}} \\ 
	 ~ Population, total (mln) & 17.5 ~ \ \ & 19.7 ~ \ \ & 23.6 ~ \ \ \\ 
	 ~ Population, rural (\% total population) & 2.5 ~ \ \ & 2.4 ~ \ \ & 2.4 ~ \ \ \\ 
	 ~ Govt expenditure on ag (\% total outlays) &  ~ \ \ &  ~ \ \ &  ~ \ \ \\ 
	 ~ Area harvested (mln ha) & 25 ~ \ \ & 31 ~ \ \ & 36 ~ \ \ \\ 
	 ~ Cropping intensity ratio (\%) & 0.1 ~ \ \ & 0.1 ~ \ \ &  ~ \ \ \\ 
	 ~ Water resources (m\textsuperscript{3}/person/year) & \textit{28} ~ \ \ & \textit{25} ~ \ \ & \textit{21} ~ \ \ \\ 
	 ~ Area equipped for irrigation (1000 ha) &  ~ \ \ &  ~ \ \ & \textit{2\,550} ~ \ \ \\ 
	 ~ Area irrigated (\%) &  ~ \ \ &  ~ \ \ &  ~ \ \ \\ 
	 ~ Employment in agriculture (\%) & 5.3 ~ \ \ & 4.4 ~ \ \ & \textit{3.3} ~ \ \ \\ 
	 ~ Employment in agriculture, female (\%) & 3.8 ~ \ \ & 3 ~ \ \ & \textit{2.2} ~ \ \ \\ 
	 ~ Fertilizers, Nitrogen (nutrients per ha) &  ~ \ \ & 2.2 ~ \ \ & \textit{2.7} ~ \ \ \\ 
	 ~ Fertilizers, Phosphate (nutrients per ha) &  ~ \ \ & 2.4 ~ \ \ & \textit{2} ~ \ \ \\ 
	 ~ Fertilizers, Potash (nutrients per ha) &  ~ \ \ & 0.5 ~ \ \ & \textit{0.5} ~ \ \ \\ 
	 ~ Energy consump, power irrigation (mln kWh) & 284 ~ \ \ & 284 ~ \ \ & \textit{1\,379} ~ \ \ \\ 
	 ~ Agr value added per worker (constant US\$) & 27.6 ~ \ \ & 44.2 ~ \ \ & \textit{49.7} ~ \ \ \\ 
	\multicolumn{4}{l}{\textcolor{FAOblue}{\textbf{\large{Hunger dimensions}}}} \\ 
	 ~ Dietary energy supply (kcal/pc/day) &  ~ \ \ &  ~ \ \ &  ~ \ \ \\ 
	 ~ Average dietary energy supply adequacy (\%) & 124 ~ \ \ & 124 ~ \ \ & 132 ~ \ \ \\ 
	 ~ Dietary en supp, cereals/roots/tubers (\%) & 25 ~ \ \ & 26 ~ \ \ & \textit{25} ~ \ \ \\ 
	 ~ Prevalence of undernourishment (\%) & <5.0 ~ \ \ & <5.0 ~ \ \ & <5.0 ~ \ \ \\ 
	 ~ GDP per capita (US\$, PPP) & 27\,895 ~ \ \ & 36\,375 ~ \ \ & \textit{42\,834} ~ \ \ \\ 
	 ~ Domestic food price volatility (index) &  ~ \ \ &  ~ \ \ &  ~ \ \ \\ 
	 ~ Cereal import dependency ratio (\%) & -139.5 ~ \ \ & -111.7 ~ \ \ & \textit{-144.9} ~ \ \ \\ 
	 ~ Underweight, children under-5 (\%) & \textit{0} ~ \ \ & \textit{0} ~ \ \ & \textit{0.2} ~ \ \ \\ 
	 ~ Improved water source (\% pop) & 100 ~ \ \ & 100 ~ \ \ & \textit{100} ~ \ \ \\ 
	\multicolumn{4}{l}{\textcolor{FAOblue}{\textbf{\large{Food Supply}}}} \\ 
	 ~ Food production value, (2004-2006 mln I\$) & 16\,617 ~ \ \ & 19\,670 ~ \ \ & \textit{25\,035} ~ \ \ \\ 
	 ~ Agriculture, value added (\% GDP) &  ~ \ \ & 4 ~ \ \ & \textit{2} ~ \ \ \\ 
	 ~ Food exports (mln US\$)  & 6\,568 ~ \ \ & 11\,031 ~ \ \ & \textit{27\,285} ~ \ \ \\ 
	 ~ Food imports (mln US\$)  & 1\,142 ~ \ \ & 2\,060 ~ \ \ & \textit{8\,036} ~ \ \ \\ 
	\multicolumn{4}{l}{\textit{\normalsize{Production indices (2004-06=100)}}} \\ 
	 ~ Net food & 79 ~ \ \ & 94 ~ \ \ & \textit{120} ~ \ \ \\ 
	 ~ Net crop & 76 ~ \ \ & 81 ~ \ \ & \textit{129} ~ \ \ \\ 
	 ~ Cereal & 82 ~ \ \ & 62 ~ \ \ & \textit{117} ~ \ \ \\ 
	 ~ Vegetable oils & 42 ~ \ \ & 89 ~ \ \ & \textit{293} ~ \ \ \\ 
	 ~ Roots and tubers & 89 ~ \ \ & 104 ~ \ \ & \textit{102} ~ \ \ \\ 
	 ~ Fruit and vegetables & 66 ~ \ \ & 94 ~ \ \ & \textit{95} ~ \ \ \\ 
	 ~ Sugar & 55 ~ \ \ & 84 ~ \ \ & \textit{73} ~ \ \ \\ 
	 ~ Livestock & 88 ~ \ \ & 106 ~ \ \ & \textit{106} ~ \ \ \\ 
	 ~ Milk & 69 ~ \ \ & 112 ~ \ \ & \textit{94} ~ \ \ \\ 
	 ~ Meat & 85 ~ \ \ & 102 ~ \ \ & \textit{112} ~ \ \ \\ 
	 ~ Fish  & 93 ~ \ \ & 88 ~ \ \ & \textit{87} ~ \ \ \\ 
	\multicolumn{4}{l}{\textit{\normalsize{Net trade (min US\$)}}} \\ 
	 ~ Cereals & 1\,741 ~ \ \ & 2\,999 ~ \ \ & \textit{9\,024} ~ \ \ \\ 
	 ~ Fruit and vegetables & 211 ~ \ \ & 385 ~ \ \ & \textit{162} ~ \ \ \\ 
	 ~ Meat & 2\,596 ~ \ \ & 3\,112 ~ \ \ & \textit{6\,751} ~ \ \ \\ 
	 ~ Dairy products & 518 ~ \ \ & 1\,388 ~ \ \ & \textit{1\,584} ~ \ \ \\ 
	 ~ Fish &  ~ \ \ &  ~ \ \ &  ~ \ \ \\ 
	\multicolumn{4}{l}{\textcolor{FAOblue}{\textbf{\large{Environment}}}} \\ 
	 ~ Forest area (\%) & 20 ~ \ \ & 20 ~ \ \ & \textit{19} ~ \ \ \\ 
	 ~ Renewable water res withdrawn (\% of total) &  ~ \ \ &  ~ \ \ & 66 ~ \ \ \\ 
	 ~ Terrestrial protect areas (\% total land area)  & 8 ~ \ \ & 10 ~ \ \ & \textit{13} ~ \ \ \\ 
	 ~ Organic area (\% total agricultural area) &  ~ \ \ & \textit{3} ~ \ \ & \textit{3} ~ \ \ \\ 
	 ~ Water withdrawal by agriculture (\% of total) &  ~ \ \ &  ~ \ \ & 66 ~ \ \ \\ 
	 ~ Biofuel production (thousand kt of oil eq.) & 170 ~ \ \ & 264 ~ \ \ & \textit{1\,908} ~ \ \ \\ 
	 ~ Wood pellet prod. (min tonnes) &  ~ \ \ &  ~ \ \ & \textit{2} ~ \ \ \\ 
	 ~ GHG emissions from ag (Co2 eq, gigagrams) & 175 ~ \ \ & 239 ~ \ \ & \textit{227} ~ \ \ \\ 
       \toprule
      \end{tabular}
      \clearpage
\CountryData{ Austria }
      \rowcolors{1}{FAOblue!10}{white}
      \begin{tabular}{L{3.9cm} R{1cm} R{1cm} R{1cm}}
      \toprule
      \multicolumn{1}{c}{} & \multicolumn{1}{c}{ 1992 } & \multicolumn{1}{c}{ 2002 } & \multicolumn{1}{c}{ 2014 } \\
      \midrule
	\multicolumn{4}{l}{\textcolor{FAOblue}{\textbf{\large{The setting}}}} \\ 
	 ~ Population, total (mln) & 7.8 ~ \ \ & 8.1 ~ \ \ & 8.5 ~ \ \ \\ 
	 ~ Population, rural (\% total population) & 2.7 ~ \ \ & 2.8 ~ \ \ & 2.7 ~ \ \ \\ 
	 ~ Govt expenditure on ag (\% total outlays) &  ~ \ \ &  ~ \ \ &  ~ \ \ \\ 
	 ~ Area harvested (mln ha) & 4 ~ \ \ & 5 ~ \ \ & 5 ~ \ \ \\ 
	 ~ Cropping intensity ratio (\%) & 1.2 ~ \ \ & 1.4 ~ \ \ &  ~ \ \ \\ 
	 ~ Water resources (m\textsuperscript{3}/person/year) & \textit{10} ~ \ \ & \textit{10} ~ \ \ & \textit{9} ~ \ \ \\ 
	 ~ Area equipped for irrigation (1000 ha) &  ~ \ \ &  ~ \ \ & \textit{117} ~ \ \ \\ 
	 ~ Area irrigated (\%) &  ~ \ \ &  ~ \ \ & \textit{37.1} ~ \ \ \\ 
	 ~ Employment in agriculture (\%) & 7.1 ~ \ \ & 5.6 ~ \ \ & \textit{4.9} ~ \ \ \\ 
	 ~ Employment in agriculture, female (\%) & 8 ~ \ \ & 6 ~ \ \ & \textit{4.5} ~ \ \ \\ 
	 ~ Fertilizers, Nitrogen (nutrients per ha) &  ~ \ \ & 50.7 ~ \ \ & \textit{27.2} ~ \ \ \\ 
	 ~ Fertilizers, Phosphate (nutrients per ha) &  ~ \ \ & 23 ~ \ \ & \textit{7.3} ~ \ \ \\ 
	 ~ Fertilizers, Potash (nutrients per ha) &  ~ \ \ & 22.4 ~ \ \ & \textit{10} ~ \ \ \\ 
	 ~ Energy consump, power irrigation (mln kWh) & 4 ~ \ \ & 4 ~ \ \ & \textit{4} ~ \ \ \\ 
	 ~ Agr value added per worker (constant US\$) & 13.4 ~ \ \ & 20.6 ~ \ \ & \textit{32.7} ~ \ \ \\ 
	\multicolumn{4}{l}{\textcolor{FAOblue}{\textbf{\large{Hunger dimensions}}}} \\ 
	 ~ Dietary energy supply (kcal/pc/day) &  ~ \ \ &  ~ \ \ &  ~ \ \ \\ 
	 ~ Average dietary energy supply adequacy (\%) & 140 ~ \ \ & 149 ~ \ \ & 150 ~ \ \ \\ 
	 ~ Dietary en supp, cereals/roots/tubers (\%) & 24 ~ \ \ & 28 ~ \ \ & \textit{26} ~ \ \ \\ 
	 ~ Prevalence of undernourishment (\%) & <5.0 ~ \ \ & <5.0 ~ \ \ & <5.0 ~ \ \ \\ 
	 ~ GDP per capita (US\$, PPP) & 32\,113 ~ \ \ & 39\,370 ~ \ \ & \textit{44\,056} ~ \ \ \\ 
	 ~ Domestic food price volatility (index) &  ~ \ \ & 5.5 ~ \ \ & 5.9 ~ \ \ \\ 
	 ~ Cereal import dependency ratio (\%) & -13.2 ~ \ \ & -6.8 ~ \ \ & \textit{6.3} ~ \ \ \\ 
	 ~ Underweight, children under-5 (\%) &  ~ \ \ &  ~ \ \ &  ~ \ \ \\ 
	 ~ Improved water source (\% pop) & 100 ~ \ \ & 100 ~ \ \ & \textit{100} ~ \ \ \\ 
	\multicolumn{4}{l}{\textcolor{FAOblue}{\textbf{\large{Food Supply}}}} \\ 
	 ~ Food production value, (2004-2006 mln I\$) & 3\,875 ~ \ \ & 4\,159 ~ \ \ & \textit{4\,225} ~ \ \ \\ 
	 ~ Agriculture, value added (\% GDP) & 3 ~ \ \ & 2 ~ \ \ & \textit{1} ~ \ \ \\ 
	 ~ Food exports (mln US\$)  & 1\,034 ~ \ \ & 2\,897 ~ \ \ & \textit{8\,396} ~ \ \ \\ 
	 ~ Food imports (mln US\$)  & 1\,964 ~ \ \ & 3\,496 ~ \ \ & \textit{9\,461} ~ \ \ \\ 
	\multicolumn{4}{l}{\textit{\normalsize{Production indices (2004-06=100)}}} \\ 
	 ~ Net food & 94 ~ \ \ & 100 ~ \ \ & \textit{102} ~ \ \ \\ 
	 ~ Net crop & 80 ~ \ \ & 95 ~ \ \ & \textit{96} ~ \ \ \\ 
	 ~ Cereal & 78 ~ \ \ & 94 ~ \ \ & \textit{96} ~ \ \ \\ 
	 ~ Vegetable oils & 100 ~ \ \ & 88 ~ \ \ & \textit{119} ~ \ \ \\ 
	 ~ Roots and tubers & 103 ~ \ \ & 97 ~ \ \ & \textit{85} ~ \ \ \\ 
	 ~ Fruit and vegetables & 72 ~ \ \ & 96 ~ \ \ & \textit{94} ~ \ \ \\ 
	 ~ Sugar & 92 ~ \ \ & 108 ~ \ \ & \textit{123} ~ \ \ \\ 
	 ~ Livestock & 102 ~ \ \ & 102 ~ \ \ & \textit{104} ~ \ \ \\ 
	 ~ Milk & 105 ~ \ \ & 105 ~ \ \ & \textit{109} ~ \ \ \\ 
	 ~ Meat & 100 ~ \ \ & 99 ~ \ \ & \textit{100} ~ \ \ \\ 
	 ~ Fish  & 130 ~ \ \ & 97 ~ \ \ & \textit{129} ~ \ \ \\ 
	\multicolumn{4}{l}{\textit{\normalsize{Net trade (min US\$)}}} \\ 
	 ~ Cereals & -10 ~ \ \ & -16 ~ \ \ & \textit{46} ~ \ \ \\ 
	 ~ Fruit and vegetables & -776 ~ \ \ & -573 ~ \ \ & \textit{-1\,349} ~ \ \ \\ 
	 ~ Meat & 60 ~ \ \ & 115 ~ \ \ & \textit{506} ~ \ \ \\ 
	 ~ Dairy products & 19 ~ \ \ & 243 ~ \ \ & \textit{511} ~ \ \ \\ 
	 ~ Fish &  ~ \ \ &  ~ \ \ &  ~ \ \ \\ 
	\multicolumn{4}{l}{\textcolor{FAOblue}{\textbf{\large{Environment}}}} \\ 
	 ~ Forest area (\%) & 46 ~ \ \ & 47 ~ \ \ & \textit{47} ~ \ \ \\ 
	 ~ Renewable water res withdrawn (\% of total) &  ~ \ \ & 3 ~ \ \ & 3 ~ \ \ \\ 
	 ~ Terrestrial protect areas (\% total land area)  & 21 ~ \ \ & 23 ~ \ \ & \textit{24} ~ \ \ \\ 
	 ~ Organic area (\% total agricultural area) &  ~ \ \ & \textit{15} ~ \ \ & \textit{17} ~ \ \ \\ 
	 ~ Water withdrawal by agriculture (\% of total) &  ~ \ \ & 3 ~ \ \ & 3 ~ \ \ \\ 
	 ~ Biofuel production (thousand kt of oil eq.) &  ~ \ \ & 36 ~ \ \ & \textit{2\,476} ~ \ \ \\ 
	 ~ Wood pellet prod. (min tonnes) &  ~ \ \ &  ~ \ \ & \textit{962} ~ \ \ \\ 
	 ~ GHG emissions from ag (Co2 eq, gigagrams) & -4 ~ \ \ & -10 ~ \ \ & \textit{11} ~ \ \ \\ 
       \toprule
      \end{tabular}
      \clearpage
\CountryData{ Azerbaijan }
      \rowcolors{1}{FAOblue!10}{white}
      \begin{tabular}{L{3.9cm} R{1cm} R{1cm} R{1cm}}
      \toprule
      \multicolumn{1}{c}{} & \multicolumn{1}{c}{ 1992 } & \multicolumn{1}{c}{ 2002 } & \multicolumn{1}{c}{ 2014 } \\
      \midrule
	\multicolumn{4}{l}{\textcolor{FAOblue}{\textbf{\large{The setting}}}} \\ 
	 ~ Population, total (mln) & 7.5 ~ \ \ & 8.3 ~ \ \ & 9.5 ~ \ \ \\ 
	 ~ Population, rural (\% total population) & 3.5 ~ \ \ & 4 ~ \ \ & 4.3 ~ \ \ \\ 
	 ~ Govt expenditure on ag (\% total outlays) &  ~ \ \ & 4.9 ~ \ \ & \textit{2.9} ~ \ \ \\ 
	 ~ Area harvested (mln ha) & 1 ~ \ \ & 2 ~ \ \ & 3 ~ \ \ \\ 
	 ~ Cropping intensity ratio (\%) & 0.3 ~ \ \ & 0.4 ~ \ \ &  ~ \ \ \\ 
	 ~ Water resources (m\textsuperscript{3}/person/year) & \textit{5} ~ \ \ & \textit{4} ~ \ \ & \textit{4} ~ \ \ \\ 
	 ~ Area equipped for irrigation (1000 ha) &  ~ \ \ &  ~ \ \ & \textit{1\,428} ~ \ \ \\ 
	 ~ Area irrigated (\%) &  ~ \ \ &  ~ \ \ & \textit{95.3} ~ \ \ \\ 
	 ~ Employment in agriculture (\%) & 34.7 ~ \ \ & 40.2 ~ \ \ & \textit{37.7} ~ \ \ \\ 
	 ~ Employment in agriculture, female (\%) &  ~ \ \ & 39.4 ~ \ \ & \textit{43.9} ~ \ \ \\ 
	 ~ Fertilizers, Nitrogen (nutrients per ha) &  ~ \ \ & 3.3 ~ \ \ & \textit{5.7} ~ \ \ \\ 
	 ~ Fertilizers, Phosphate (nutrients per ha) &  ~ \ \ & 0.3 ~ \ \ & \textit{0.9} ~ \ \ \\ 
	 ~ Fertilizers, Potash (nutrients per ha) &  ~ \ \ & 0.4 ~ \ \ & \textit{0.6} ~ \ \ \\ 
	 ~ Energy consump, power irrigation (mln kWh) & \textit{366} ~ \ \ & 366 ~ \ \ & \textit{1\,464} ~ \ \ \\ 
	 ~ Agr value added per worker (constant US\$) & 1 ~ \ \ & 1 ~ \ \ & \textit{1.5} ~ \ \ \\ 
	\multicolumn{4}{l}{\textcolor{FAOblue}{\textbf{\large{Hunger dimensions}}}} \\ 
	 ~ Dietary energy supply (kcal/pc/day) & 2\,235 ~ \ \ & 2\,586 ~ \ \ & 3\,030 ~ \ \ \\ 
	 ~ Average dietary energy supply adequacy (\%) & 99 ~ \ \ & 110 ~ \ \ & 126 ~ \ \ \\ 
	 ~ Dietary en supp, cereals/roots/tubers (\%) & 71 ~ \ \ & 68 ~ \ \ & \textit{63} ~ \ \ \\ 
	 ~ Prevalence of undernourishment (\%) & 23.6 ~ \ \ & 12.8 ~ \ \ & <5.0 ~ \ \ \\ 
	 ~ GDP per capita (US\$, PPP) & 6\,346 ~ \ \ & 5\,338 ~ \ \ & \textit{16\,593} ~ \ \ \\ 
	 ~ Domestic food price volatility (index) &  ~ \ \ &  ~ \ \ &  ~ \ \ \\ 
	 ~ Cereal import dependency ratio (\%) & 43.3 ~ \ \ & 27.9 ~ \ \ & \textit{37.7} ~ \ \ \\ 
	 ~ Underweight, children under-5 (\%) &  ~ \ \ & \textit{5.9} ~ \ \ & \textit{8.4} ~ \ \ \\ 
	 ~ Improved water source (\% pop) & 69.6 ~ \ \ & 75.6 ~ \ \ & \textit{80.2} ~ \ \ \\ 
	\multicolumn{4}{l}{\textcolor{FAOblue}{\textbf{\large{Food Supply}}}} \\ 
	 ~ Food production value, (2004-2006 mln I\$) & 1\,463 ~ \ \ & 1\,627 ~ \ \ & \textit{2\,578} ~ \ \ \\ 
	 ~ Agriculture, value added (\% GDP) & 29 ~ \ \ & 15 ~ \ \ & \textit{6} ~ \ \ \\ 
	 ~ Food exports (mln US\$)  & 28 ~ \ \ & 35 ~ \ \ & \textit{679} ~ \ \ \\ 
	 ~ Food imports (mln US\$)  & 262 ~ \ \ & 191 ~ \ \ & \textit{1\,018} ~ \ \ \\ 
	\multicolumn{4}{l}{\textit{\normalsize{Production indices (2004-06=100)}}} \\ 
	 ~ Net food & 79 ~ \ \ & 88 ~ \ \ & \textit{140} ~ \ \ \\ 
	 ~ Net crop & 96 ~ \ \ & 84 ~ \ \ & \textit{120} ~ \ \ \\ 
	 ~ Cereal & 64 ~ \ \ & 107 ~ \ \ & \textit{138} ~ \ \ \\ 
	 ~ Vegetable oils & 122 ~ \ \ & 59 ~ \ \ & \textit{68} ~ \ \ \\ 
	 ~ Roots and tubers & 17 ~ \ \ & 69 ~ \ \ & \textit{97} ~ \ \ \\ 
	 ~ Fruit and vegetables & 102 ~ \ \ & 85 ~ \ \ & \textit{132} ~ \ \ \\ 
	 ~ Sugar & \textit{32} ~ \ \ & 133 ~ \ \ & \textit{216} ~ \ \ \\ 
	 ~ Livestock & 72 ~ \ \ & 86 ~ \ \ & \textit{166} ~ \ \ \\ 
	 ~ Milk & 68 ~ \ \ & 89 ~ \ \ & \textit{145} ~ \ \ \\ 
	 ~ Meat & 71 ~ \ \ & 85 ~ \ \ & \textit{189} ~ \ \ \\ 
	 ~ Fish  & 404 ~ \ \ & 142 ~ \ \ & \textit{14} ~ \ \ \\ 
	\multicolumn{4}{l}{\textit{\normalsize{Net trade (min US\$)}}} \\ 
	 ~ Cereals & \textit{-43} ~ \ \ & -85 ~ \ \ & \textit{-432} ~ \ \ \\ 
	 ~ Fruit and vegetables & 14 ~ \ \ & 14 ~ \ \ & \textit{206} ~ \ \ \\ 
	 ~ Meat & \textit{-35} ~ \ \ & -16 ~ \ \ & \textit{-10} ~ \ \ \\ 
	 ~ Dairy products & \textit{-40} ~ \ \ & -16 ~ \ \ & \textit{-53} ~ \ \ \\ 
	 ~ Fish &  ~ \ \ &  ~ \ \ &  ~ \ \ \\ 
	\multicolumn{4}{l}{\textcolor{FAOblue}{\textbf{\large{Environment}}}} \\ 
	 ~ Forest area (\%) & 11 ~ \ \ & 11 ~ \ \ & \textit{11} ~ \ \ \\ 
	 ~ Renewable water res withdrawn (\% of total) &  ~ \ \ & \textit{76} ~ \ \ & 76 ~ \ \ \\ 
	 ~ Terrestrial protect areas (\% total land area)  & 6 ~ \ \ & 7 ~ \ \ & \textit{7} ~ \ \ \\ 
	 ~ Organic area (\% total agricultural area) &  ~ \ \ & \textit{0} ~ \ \ & \textit{0} ~ \ \ \\ 
	 ~ Water withdrawal by agriculture (\% of total) &  ~ \ \ & \textit{76} ~ \ \ & 76 ~ \ \ \\ 
	 ~ Biofuel production (thousand kt of oil eq.) &  ~ \ \ &  ~ \ \ &  ~ \ \ \\ 
	 ~ Wood pellet prod. (min tonnes) &  ~ \ \ &  ~ \ \ & \textit{0} ~ \ \ \\ 
	 ~ GHG emissions from ag (Co2 eq, gigagrams) & 5 ~ \ \ & 5 ~ \ \ & \textit{7} ~ \ \ \\ 
       \toprule
      \end{tabular}
      \clearpage
\CountryData{ Bahrain }
      \rowcolors{1}{FAOblue!10}{white}
      \begin{tabular}{L{3.9cm} R{1cm} R{1cm} R{1cm}}
      \toprule
      \multicolumn{1}{c}{} & \multicolumn{1}{c}{ 1992 } & \multicolumn{1}{c}{ 2002 } & \multicolumn{1}{c}{ 2014 } \\
      \midrule
	\multicolumn{4}{l}{\textcolor{FAOblue}{\textbf{\large{The setting}}}} \\ 
	 ~ Population, total (mln) & 0.5 ~ \ \ & 0.7 ~ \ \ & 1.3 ~ \ \ \\ 
	 ~ Population, rural (\% total population) & 0.1 ~ \ \ & 0.1 ~ \ \ & 0.1 ~ \ \ \\ 
	 ~ Govt expenditure on ag (\% total outlays) &  ~ \ \ &  ~ \ \ &  ~ \ \ \\ 
	 ~ Area harvested (mln ha) & 0 ~ \ \ & 1 ~ \ \ & 0 ~ \ \ \\ 
	 ~ Cropping intensity ratio (\%) & 55.6 ~ \ \ & 92 ~ \ \ &  ~ \ \ \\ 
	 ~ Water resources (m\textsuperscript{3}/person/year) & \textit{0} ~ \ \ & \textit{0} ~ \ \ & \textit{0} ~ \ \ \\ 
	 ~ Area equipped for irrigation (1000 ha) &  ~ \ \ &  ~ \ \ & \textit{4} ~ \ \ \\ 
	 ~ Area irrigated (\%) &  ~ \ \ & \textit{100} ~ \ \ &  ~ \ \ \\ 
	 ~ Employment in agriculture (\%) & \textit{2.4} ~ \ \ & \textit{0.8} ~ \ \ & \textit{1.1} ~ \ \ \\ 
	 ~ Employment in agriculture, female (\%) & \textit{0} ~ \ \ & \textit{0.1} ~ \ \ & \textit{0} ~ \ \ \\ 
	 ~ Fertilizers, Nitrogen (nutrients per ha) &  ~ \ \ & 1\,929.5 ~ \ \ & \textit{75} ~ \ \ \\ 
	 ~ Fertilizers, Phosphate (nutrients per ha) &  ~ \ \ & 10.5 ~ \ \ & \textit{69.2} ~ \ \ \\ 
	 ~ Fertilizers, Potash (nutrients per ha) &  ~ \ \ & 8.8 ~ \ \ & \textit{8.7} ~ \ \ \\ 
	 ~ Energy consump, power irrigation (mln kWh) & \textit{2} ~ \ \ & 2 ~ \ \ & \textit{2} ~ \ \ \\ 
	 ~ Agr value added per worker (constant US\$) &  ~ \ \ &  ~ \ \ &  ~ \ \ \\ 
	\multicolumn{4}{l}{\textcolor{FAOblue}{\textbf{\large{Hunger dimensions}}}} \\ 
	 ~ Dietary energy supply (kcal/pc/day) &  ~ \ \ &  ~ \ \ &  ~ \ \ \\ 
	 ~ Average dietary energy supply adequacy (\%) &  ~ \ \ &  ~ \ \ &  ~ \ \ \\ 
	 ~ Dietary en supp, cereals/roots/tubers (\%) &  ~ \ \ &  ~ \ \ &  ~ \ \ \\ 
	 ~ Prevalence of undernourishment (\%) &  ~ \ \ &  ~ \ \ &  ~ \ \ \\ 
	 ~ GDP per capita (US\$, PPP) & 39\,854 ~ \ \ & 43\,654 ~ \ \ & \textit{42\,444} ~ \ \ \\ 
	 ~ Domestic food price volatility (index) &  ~ \ \ & 17 ~ \ \ & 18.5 ~ \ \ \\ 
	 ~ Cereal import dependency ratio (\%) &  ~ \ \ &  ~ \ \ &  ~ \ \ \\ 
	 ~ Underweight, children under-5 (\%) & \textit{7.6} ~ \ \ & \textit{7.6} ~ \ \ &  ~ \ \ \\ 
	 ~ Improved water source (\% pop) & 94.9 ~ \ \ & 100 ~ \ \ & \textit{100} ~ \ \ \\ 
	\multicolumn{4}{l}{\textcolor{FAOblue}{\textbf{\large{Food Supply}}}} \\ 
	 ~ Food production value, (2004-2006 mln I\$) & 23 ~ \ \ & 23 ~ \ \ & \textit{44} ~ \ \ \\ 
	 ~ Agriculture, value added (\% GDP) & 1 ~ \ \ & \textit{1} ~ \ \ &  ~ \ \ \\ 
	 ~ Food exports (mln US\$)  & 11 ~ \ \ & 29 ~ \ \ & \textit{147} ~ \ \ \\ 
	 ~ Food imports (mln US\$)  & 265 ~ \ \ & 444 ~ \ \ & \textit{877} ~ \ \ \\ 
	\multicolumn{4}{l}{\textit{\normalsize{Production indices (2004-06=100)}}} \\ 
	 ~ Net food & 108 ~ \ \ & 106 ~ \ \ & \textit{207} ~ \ \ \\ 
	 ~ Net crop & 74 ~ \ \ & 93 ~ \ \ & \textit{123} ~ \ \ \\ 
	 ~ Cereal &  ~ \ \ &  ~ \ \ &  ~ \ \ \\ 
	 ~ Vegetable oils &  ~ \ \ &  ~ \ \ &  ~ \ \ \\ 
	 ~ Roots and tubers & 634 ~ \ \ & 106 ~ \ \ & \textit{1\,000} ~ \ \ \\ 
	 ~ Fruit and vegetables & 75 ~ \ \ & 92 ~ \ \ & \textit{125} ~ \ \ \\ 
	 ~ Sugar &  ~ \ \ &  ~ \ \ &  ~ \ \ \\ 
	 ~ Livestock & 161 ~ \ \ & 129 ~ \ \ & \textit{338} ~ \ \ \\ 
	 ~ Milk & 146 ~ \ \ & 156 ~ \ \ & \textit{83} ~ \ \ \\ 
	 ~ Meat & 220 ~ \ \ & 131 ~ \ \ & \textit{1\,007} ~ \ \ \\ 
	 ~ Fish  & 57 ~ \ \ & 80 ~ \ \ & \textit{107} ~ \ \ \\ 
	\multicolumn{4}{l}{\textit{\normalsize{Net trade (min US\$)}}} \\ 
	 ~ Cereals & -45 ~ \ \ & -55 ~ \ \ & \textit{-133} ~ \ \ \\ 
	 ~ Fruit and vegetables & -75 ~ \ \ & -138 ~ \ \ & \textit{-174} ~ \ \ \\ 
	 ~ Meat & -42 ~ \ \ & -53 ~ \ \ & \textit{-178} ~ \ \ \\ 
	 ~ Dairy products & -35 ~ \ \ & -65 ~ \ \ & \textit{-145} ~ \ \ \\ 
	 ~ Fish &  ~ \ \ &  ~ \ \ &  ~ \ \ \\ 
	\multicolumn{4}{l}{\textcolor{FAOblue}{\textbf{\large{Environment}}}} \\ 
	 ~ Forest area (\%) & 0 ~ \ \ & 1 ~ \ \ & \textit{1} ~ \ \ \\ 
	 ~ Renewable water res withdrawn (\% of total) &  ~ \ \ & \textit{44} ~ \ \ & 44 ~ \ \ \\ 
	 ~ Terrestrial protect areas (\% total land area)  & 1 ~ \ \ & 1 ~ \ \ & \textit{3} ~ \ \ \\ 
	 ~ Organic area (\% total agricultural area) &  ~ \ \ &  ~ \ \ &  ~ \ \ \\ 
	 ~ Water withdrawal by agriculture (\% of total) &  ~ \ \ & \textit{44} ~ \ \ & 44 ~ \ \ \\ 
	 ~ Biofuel production (thousand kt of oil eq.) &  ~ \ \ &  ~ \ \ &  ~ \ \ \\ 
	 ~ Wood pellet prod. (min tonnes) &  ~ \ \ &  ~ \ \ &  ~ \ \ \\ 
	 ~ GHG emissions from ag (Co2 eq, gigagrams) & 0 ~ \ \ & 0 ~ \ \ & \textit{0} ~ \ \ \\ 
       \toprule
      \end{tabular}
      \clearpage
\CountryData{ Bangladesh }
      \rowcolors{1}{FAOblue!10}{white}
      \begin{tabular}{L{3.9cm} R{1cm} R{1cm} R{1cm}}
      \toprule
      \multicolumn{1}{c}{} & \multicolumn{1}{c}{ 1992 } & \multicolumn{1}{c}{ 2002 } & \multicolumn{1}{c}{ 2014 } \\
      \midrule
	\multicolumn{4}{l}{\textcolor{FAOblue}{\textbf{\large{The setting}}}} \\ 
	 ~ Population, total (mln) & 112.4 ~ \ \ & 137 ~ \ \ & 158.5 ~ \ \ \\ 
	 ~ Population, rural (\% total population) & 89.3 ~ \ \ & 103.6 ~ \ \ & 111.2 ~ \ \ \\ 
	 ~ Govt expenditure on ag (\% total outlays) &  ~ \ \ & 3.2 ~ \ \ & \textit{15.2} ~ \ \ \\ 
	 ~ Area harvested (mln ha) & 29 ~ \ \ & 39 ~ \ \ & 55 ~ \ \ \\ 
	 ~ Cropping intensity ratio (\%) & 3 ~ \ \ & 4.2 ~ \ \ &  ~ \ \ \\ 
	 ~ Water resources (m\textsuperscript{3}/person/year) & \textit{11} ~ \ \ & \textit{9} ~ \ \ & \textit{8} ~ \ \ \\ 
	 ~ Area equipped for irrigation (1000 ha) &  ~ \ \ &  ~ \ \ & \textit{5\,300} ~ \ \ \\ 
	 ~ Area irrigated (\%) & 88.3 ~ \ \ &  ~ \ \ &  ~ \ \ \\ 
	 ~ Employment in agriculture (\%) & \textit{66.4} ~ \ \ & \textit{48.1} ~ \ \ & \textit{48.1} ~ \ \ \\ 
	 ~ Employment in agriculture, female (\%) & \textit{84.9} ~ \ \ & \textit{68.1} ~ \ \ & \textit{68.1} ~ \ \ \\ 
	 ~ Fertilizers, Nitrogen (nutrients per ha) &  ~ \ \ & 115.4 ~ \ \ & \textit{124} ~ \ \ \\ 
	 ~ Fertilizers, Phosphate (nutrients per ha) &  ~ \ \ & 34.4 ~ \ \ & \textit{63.3} ~ \ \ \\ 
	 ~ Fertilizers, Potash (nutrients per ha) &  ~ \ \ & 16.7 ~ \ \ & \textit{47.1} ~ \ \ \\ 
	 ~ Energy consump, power irrigation (mln kWh) & \textit{0} ~ \ \ & 0 ~ \ \ & \textit{0} ~ \ \ \\ 
	 ~ Agr value added per worker (constant US\$) & 0.3 ~ \ \ & 0.3 ~ \ \ & \textit{0.5} ~ \ \ \\ 
	\multicolumn{4}{l}{\textcolor{FAOblue}{\textbf{\large{Hunger dimensions}}}} \\ 
	 ~ Dietary energy supply (kcal/pc/day) & 2\,078 ~ \ \ & 2\,331 ~ \ \ & 2\,470 ~ \ \ \\ 
	 ~ Average dietary energy supply adequacy (\%) & 97 ~ \ \ & 105 ~ \ \ & 108 ~ \ \ \\ 
	 ~ Dietary en supp, cereals/roots/tubers (\%) & 85 ~ \ \ & 84 ~ \ \ & \textit{81} ~ \ \ \\ 
	 ~ Prevalence of undernourishment (\%) & 33.2 ~ \ \ & 19.1 ~ \ \ & 16.9 ~ \ \ \\ 
	 ~ GDP per capita (US\$, PPP) & 1\,284 ~ \ \ & 1\,705 ~ \ \ & \textit{2\,853} ~ \ \ \\ 
	 ~ Domestic food price volatility (index) &  ~ \ \ & 4.8 ~ \ \ & 4.5 ~ \ \ \\ 
	 ~ Cereal import dependency ratio (\%) & 6.8 ~ \ \ & 10.8 ~ \ \ & \textit{10.8} ~ \ \ \\ 
	 ~ Underweight, children under-5 (\%) & 63.7 ~ \ \ & 43.1 ~ \ \ & \textit{31.9} ~ \ \ \\ 
	 ~ Improved water source (\% pop) & 69.7 ~ \ \ & 77.5 ~ \ \ & \textit{84.8} ~ \ \ \\ 
	\multicolumn{4}{l}{\textcolor{FAOblue}{\textbf{\large{Food Supply}}}} \\ 
	 ~ Food production value, (2004-2006 mln I\$) & 10\,645 ~ \ \ & 14\,413 ~ \ \ & \textit{21\,639} ~ \ \ \\ 
	 ~ Agriculture, value added (\% GDP) & 29 ~ \ \ & 23 ~ \ \ & \textit{16} ~ \ \ \\ 
	 ~ Food exports (mln US\$)  & 8 ~ \ \ & 16 ~ \ \ & \textit{71} ~ \ \ \\ 
	 ~ Food imports (mln US\$)  & 618 ~ \ \ & 1\,159 ~ \ \ & \textit{4\,712} ~ \ \ \\ 
	\multicolumn{4}{l}{\textit{\normalsize{Production indices (2004-06=100)}}} \\ 
	 ~ Net food & 67 ~ \ \ & 90 ~ \ \ & \textit{135} ~ \ \ \\ 
	 ~ Net crop & 68 ~ \ \ & 91 ~ \ \ & \textit{138} ~ \ \ \\ 
	 ~ Cereal & 70 ~ \ \ & 97 ~ \ \ & \textit{134} ~ \ \ \\ 
	 ~ Vegetable oils & 130 ~ \ \ & 96 ~ \ \ & \textit{133} ~ \ \ \\ 
	 ~ Roots and tubers & 32 ~ \ \ & 64 ~ \ \ & \textit{184} ~ \ \ \\ 
	 ~ Fruit and vegetables & 50 ~ \ \ & 58 ~ \ \ & \textit{157} ~ \ \ \\ 
	 ~ Sugar & 119 ~ \ \ & 105 ~ \ \ & \textit{73} ~ \ \ \\ 
	 ~ Livestock & 61 ~ \ \ & 88 ~ \ \ & \textit{131} ~ \ \ \\ 
	 ~ Milk & 65 ~ \ \ & 86 ~ \ \ & \textit{136} ~ \ \ \\ 
	 ~ Meat & 65 ~ \ \ & 92 ~ \ \ & \textit{120} ~ \ \ \\ 
	 ~ Fish  & 42 ~ \ \ & 85 ~ \ \ & \textit{154} ~ \ \ \\ 
	\multicolumn{4}{l}{\textit{\normalsize{Net trade (min US\$)}}} \\ 
	 ~ Cereals & -192 ~ \ \ & -362 ~ \ \ & \textit{-800} ~ \ \ \\ 
	 ~ Fruit and vegetables & -51 ~ \ \ & -151 ~ \ \ & \textit{-619} ~ \ \ \\ 
	 ~ Meat & 0 ~ \ \ & 0 ~ \ \ & \textit{-3} ~ \ \ \\ 
	 ~ Dairy products & -71 ~ \ \ & -70 ~ \ \ & \textit{-249} ~ \ \ \\ 
	 ~ Fish &  ~ \ \ &  ~ \ \ &  ~ \ \ \\ 
	\multicolumn{4}{l}{\textcolor{FAOblue}{\textbf{\large{Environment}}}} \\ 
	 ~ Forest area (\%) & 11 ~ \ \ & 11 ~ \ \ & \textit{11} ~ \ \ \\ 
	 ~ Renewable water res withdrawn (\% of total) &  ~ \ \ &  ~ \ \ & 88 ~ \ \ \\ 
	 ~ Terrestrial protect areas (\% total land area)  & 2 ~ \ \ & 2 ~ \ \ & \textit{5} ~ \ \ \\ 
	 ~ Organic area (\% total agricultural area) &  ~ \ \ &  ~ \ \ & \textit{0} ~ \ \ \\ 
	 ~ Water withdrawal by agriculture (\% of total) &  ~ \ \ &  ~ \ \ & 88 ~ \ \ \\ 
	 ~ Biofuel production (thousand kt of oil eq.) & 233 ~ \ \ & 216 ~ \ \ & \textit{235} ~ \ \ \\ 
	 ~ Wood pellet prod. (min tonnes) &  ~ \ \ &  ~ \ \ &  ~ \ \ \\ 
	 ~ GHG emissions from ag (Co2 eq, gigagrams) & 92 ~ \ \ & 99 ~ \ \ & \textit{106} ~ \ \ \\ 
       \toprule
      \end{tabular}
      \clearpage
\CountryData{ Barbados }
      \rowcolors{1}{FAOblue!10}{white}
      \begin{tabular}{L{3.9cm} R{1cm} R{1cm} R{1cm}}
      \toprule
      \multicolumn{1}{c}{} & \multicolumn{1}{c}{ 1992 } & \multicolumn{1}{c}{ 2002 } & \multicolumn{1}{c}{ 2014 } \\
      \midrule
	\multicolumn{4}{l}{\textcolor{FAOblue}{\textbf{\large{The setting}}}} \\ 
	 ~ Population, total (mln) & 0.3 ~ \ \ & 0.3 ~ \ \ & 0.3 ~ \ \ \\ 
	 ~ Population, rural (\% total population) & 0.2 ~ \ \ & 0.2 ~ \ \ & 0.2 ~ \ \ \\ 
	 ~ Govt expenditure on ag (\% total outlays) &  ~ \ \ & 2.5 ~ \ \ & \textit{1.7} ~ \ \ \\ 
	 ~ Area harvested (mln ha) & 1 ~ \ \ & 1 ~ \ \ & 0 ~ \ \ \\ 
	 ~ Cropping intensity ratio (\%) & 30.2 ~ \ \ & 30.9 ~ \ \ &  ~ \ \ \\ 
	 ~ Water resources (m\textsuperscript{3}/person/year) & \textit{0} ~ \ \ & \textit{0} ~ \ \ & \textit{0} ~ \ \ \\ 
	 ~ Area equipped for irrigation (1000 ha) &  ~ \ \ &  ~ \ \ & \textit{5} ~ \ \ \\ 
	 ~ Area irrigated (\%) &  ~ \ \ &  ~ \ \ &  ~ \ \ \\ 
	 ~ Employment in agriculture (\%) & 6.2 ~ \ \ & 4 ~ \ \ & \textit{2.8} ~ \ \ \\ 
	 ~ Employment in agriculture, female (\%) & 4.2 ~ \ \ & 3.1 ~ \ \ & \textit{2.2} ~ \ \ \\ 
	 ~ Fertilizers, Nitrogen (nutrients per ha) &  ~ \ \ & 34.5 ~ \ \ & \textit{81.3} ~ \ \ \\ 
	 ~ Fertilizers, Phosphate (nutrients per ha) &  ~ \ \ & 2.9 ~ \ \ & \textit{47.9} ~ \ \ \\ 
	 ~ Fertilizers, Potash (nutrients per ha) &  ~ \ \ & 0.3 ~ \ \ & \textit{7.2} ~ \ \ \\ 
	 ~ Energy consump, power irrigation (mln kWh) &  ~ \ \ &  ~ \ \ &  ~ \ \ \\ 
	 ~ Agr value added per worker (constant US\$) & 8.3 ~ \ \ & 10.5 ~ \ \ & \textit{12.8} ~ \ \ \\ 
	\multicolumn{4}{l}{\textcolor{FAOblue}{\textbf{\large{Hunger dimensions}}}} \\ 
	 ~ Dietary energy supply (kcal/pc/day) & 2\,968 ~ \ \ & 2\,865 ~ \ \ & 3\,071 ~ \ \ \\ 
	 ~ Average dietary energy supply adequacy (\%) & 121 ~ \ \ & 116 ~ \ \ & 123 ~ \ \ \\ 
	 ~ Dietary en supp, cereals/roots/tubers (\%) & 34 ~ \ \ & 34 ~ \ \ & \textit{32} ~ \ \ \\ 
	 ~ Prevalence of undernourishment (\%) & <5.0 ~ \ \ & 5.4 ~ \ \ & <5.0 ~ \ \ \\ 
	 ~ GDP per capita (US\$, PPP) & 11\,941 ~ \ \ & 14\,310 ~ \ \ & \textit{15\,299} ~ \ \ \\ 
	 ~ Domestic food price volatility (index) &  ~ \ \ & 4.8 ~ \ \ & 5.4 ~ \ \ \\ 
	 ~ Cereal import dependency ratio (\%) & 97.2 ~ \ \ & 100 ~ \ \ & \textit{100} ~ \ \ \\ 
	 ~ Underweight, children under-5 (\%) &  ~ \ \ &  ~ \ \ & \textit{3.5} ~ \ \ \\ 
	 ~ Improved water source (\% pop) & 96.1 ~ \ \ & 99.8 ~ \ \ & \textit{99.8} ~ \ \ \\ 
	\multicolumn{4}{l}{\textcolor{FAOblue}{\textbf{\large{Food Supply}}}} \\ 
	 ~ Food production value, (2004-2006 mln I\$) & 47 ~ \ \ & 45 ~ \ \ & \textit{47} ~ \ \ \\ 
	 ~ Agriculture, value added (\% GDP) & 4 ~ \ \ & 2 ~ \ \ & \textit{1} ~ \ \ \\ 
	 ~ Food exports (mln US\$)  & 46 ~ \ \ & 49 ~ \ \ & \textit{47} ~ \ \ \\ 
	 ~ Food imports (mln US\$)  & 84 ~ \ \ & 133 ~ \ \ & \textit{242} ~ \ \ \\ 
	\multicolumn{4}{l}{\textit{\normalsize{Production indices (2004-06=100)}}} \\ 
	 ~ Net food & 97 ~ \ \ & 92 ~ \ \ & \textit{97} ~ \ \ \\ 
	 ~ Net crop & 119 ~ \ \ & 106 ~ \ \ & \textit{84} ~ \ \ \\ 
	 ~ Cereal & 693 ~ \ \ & 84 ~ \ \ & \textit{131} ~ \ \ \\ 
	 ~ Vegetable oils & 110 ~ \ \ & 89 ~ \ \ & \textit{110} ~ \ \ \\ 
	 ~ Roots and tubers & 211 ~ \ \ & 69 ~ \ \ & \textit{38} ~ \ \ \\ 
	 ~ Fruit and vegetables & 71 ~ \ \ & 105 ~ \ \ & \textit{102} ~ \ \ \\ 
	 ~ Sugar & 137 ~ \ \ & 109 ~ \ \ & \textit{73} ~ \ \ \\ 
	 ~ Livestock & 83 ~ \ \ & 83 ~ \ \ & \textit{106} ~ \ \ \\ 
	 ~ Milk & 141 ~ \ \ & 121 ~ \ \ & \textit{101} ~ \ \ \\ 
	 ~ Meat & 80 ~ \ \ & 80 ~ \ \ & \textit{105} ~ \ \ \\ 
	 ~ Fish  & 170 ~ \ \ & 120 ~ \ \ & \textit{142} ~ \ \ \\ 
	\multicolumn{4}{l}{\textit{\normalsize{Net trade (min US\$)}}} \\ 
	 ~ Cereals & -17 ~ \ \ & -23 ~ \ \ & \textit{-42} ~ \ \ \\ 
	 ~ Fruit and vegetables & -13 ~ \ \ & -29 ~ \ \ & \textit{-43} ~ \ \ \\ 
	 ~ Meat & -13 ~ \ \ & -15 ~ \ \ & \textit{-26} ~ \ \ \\ 
	 ~ Dairy products & -7 ~ \ \ & -12 ~ \ \ & \textit{-22} ~ \ \ \\ 
	 ~ Fish &  ~ \ \ &  ~ \ \ &  ~ \ \ \\ 
	\multicolumn{4}{l}{\textcolor{FAOblue}{\textbf{\large{Environment}}}} \\ 
	 ~ Forest area (\%) & 19 ~ \ \ & 19 ~ \ \ & \textit{19} ~ \ \ \\ 
	 ~ Renewable water res withdrawn (\% of total) &  ~ \ \ & \textit{54} ~ \ \ & 54 ~ \ \ \\ 
	 ~ Terrestrial protect areas (\% total land area)  & 0 ~ \ \ & 0 ~ \ \ & \textit{0} ~ \ \ \\ 
	 ~ Organic area (\% total agricultural area) &  ~ \ \ &  ~ \ \ &  ~ \ \ \\ 
	 ~ Water withdrawal by agriculture (\% of total) &  ~ \ \ & \textit{54} ~ \ \ & 54 ~ \ \ \\ 
	 ~ Biofuel production (thousand kt of oil eq.) & 1 ~ \ \ & 1 ~ \ \ & \textit{1} ~ \ \ \\ 
	 ~ Wood pellet prod. (min tonnes) &  ~ \ \ &  ~ \ \ &  ~ \ \ \\ 
	 ~ GHG emissions from ag (Co2 eq, gigagrams) & 0 ~ \ \ & 0 ~ \ \ & \textit{0} ~ \ \ \\ 
       \toprule
      \end{tabular}
      \clearpage
\CountryData{ Belarus }
      \rowcolors{1}{FAOblue!10}{white}
      \begin{tabular}{L{3.9cm} R{1cm} R{1cm} R{1cm}}
      \toprule
      \multicolumn{1}{c}{} & \multicolumn{1}{c}{ 1992 } & \multicolumn{1}{c}{ 2002 } & \multicolumn{1}{c}{ 2014 } \\
      \midrule
	\multicolumn{4}{l}{\textcolor{FAOblue}{\textbf{\large{The setting}}}} \\ 
	 ~ Population, total (mln) & 10.3 ~ \ \ & 9.9 ~ \ \ & 9.3 ~ \ \ \\ 
	 ~ Population, rural (\% total population) & 3.4 ~ \ \ & 2.9 ~ \ \ & 2.2 ~ \ \ \\ 
	 ~ Govt expenditure on ag (\% total outlays) &  ~ \ \ & 10.2 ~ \ \ & \textit{9.8} ~ \ \ \\ 
	 ~ Area harvested (mln ha) & 9 ~ \ \ & 7 ~ \ \ & 7 ~ \ \ \\ 
	 ~ Cropping intensity ratio (\%) & 1 ~ \ \ & 0.8 ~ \ \ &  ~ \ \ \\ 
	 ~ Water resources (m\textsuperscript{3}/person/year) & \textit{6} ~ \ \ & \textit{6} ~ \ \ & \textit{6} ~ \ \ \\ 
	 ~ Area equipped for irrigation (1000 ha) &  ~ \ \ &  ~ \ \ & \textit{114} ~ \ \ \\ 
	 ~ Area irrigated (\%) &  ~ \ \ &  ~ \ \ & \textit{26.8} ~ \ \ \\ 
	 ~ Employment in agriculture (\%) & 22.3 ~ \ \ &  ~ \ \ & \textit{10.5} ~ \ \ \\ 
	 ~ Employment in agriculture, female (\%) &  ~ \ \ &  ~ \ \ & \textit{7.8} ~ \ \ \\ 
	 ~ Fertilizers, Nitrogen (nutrients per ha) &  ~ \ \ & 27.4 ~ \ \ & \textit{63.3} ~ \ \ \\ 
	 ~ Fertilizers, Phosphate (nutrients per ha) &  ~ \ \ & 8.1 ~ \ \ & \textit{25.2} ~ \ \ \\ 
	 ~ Fertilizers, Potash (nutrients per ha) &  ~ \ \ & 49.1 ~ \ \ & \textit{81.8} ~ \ \ \\ 
	 ~ Energy consump, power irrigation (mln kWh) & \textit{253} ~ \ \ & 253 ~ \ \ & \textit{253} ~ \ \ \\ 
	 ~ Agr value added per worker (constant US\$) & 2.6 ~ \ \ & 3.5 ~ \ \ & \textit{9.1} ~ \ \ \\ 
	\multicolumn{4}{l}{\textcolor{FAOblue}{\textbf{\large{Hunger dimensions}}}} \\ 
	 ~ Dietary energy supply (kcal/pc/day) &  ~ \ \ &  ~ \ \ &  ~ \ \ \\ 
	 ~ Average dietary energy supply adequacy (\%) & 132 ~ \ \ & 119 ~ \ \ & 133 ~ \ \ \\ 
	 ~ Dietary en supp, cereals/roots/tubers (\%) & 48 ~ \ \ & 44 ~ \ \ & \textit{38} ~ \ \ \\ 
	 ~ Prevalence of undernourishment (\%) & <5.0 ~ \ \ & <5.0 ~ \ \ & <5.0 ~ \ \ \\ 
	 ~ GDP per capita (US\$, PPP) & 7\,201 ~ \ \ & 8\,144 ~ \ \ & \textit{17\,055} ~ \ \ \\ 
	 ~ Domestic food price volatility (index) &  ~ \ \ & 5.8 ~ \ \ & \textit{6} ~ \ \ \\ 
	 ~ Cereal import dependency ratio (\%) & 19.3 ~ \ \ & 10.4 ~ \ \ & \textit{1.4} ~ \ \ \\ 
	 ~ Underweight, children under-5 (\%) &  ~ \ \ & \textit{1.3} ~ \ \ & \textit{1.3} ~ \ \ \\ 
	 ~ Improved water source (\% pop) & 99.5 ~ \ \ & 99.6 ~ \ \ & \textit{99.6} ~ \ \ \\ 
	\multicolumn{4}{l}{\textcolor{FAOblue}{\textbf{\large{Food Supply}}}} \\ 
	 ~ Food production value, (2004-2006 mln I\$) & 5\,090 ~ \ \ & 3\,857 ~ \ \ & \textit{5\,746} ~ \ \ \\ 
	 ~ Agriculture, value added (\% GDP) & 24 ~ \ \ & 12 ~ \ \ & \textit{9} ~ \ \ \\ 
	 ~ Food exports (mln US\$)  & 184 ~ \ \ & 521 ~ \ \ & \textit{4\,120} ~ \ \ \\ 
	 ~ Food imports (mln US\$)  & 819 ~ \ \ & 592 ~ \ \ & \textit{2\,070} ~ \ \ \\ 
	\multicolumn{4}{l}{\textit{\normalsize{Production indices (2004-06=100)}}} \\ 
	 ~ Net food & 112 ~ \ \ & 85 ~ \ \ & \textit{126} ~ \ \ \\ 
	 ~ Net crop & 78 ~ \ \ & 79 ~ \ \ & \textit{96} ~ \ \ \\ 
	 ~ Cereal & 106 ~ \ \ & 91 ~ \ \ & \textit{130} ~ \ \ \\ 
	 ~ Vegetable oils & 33 ~ \ \ & 52 ~ \ \ & \textit{407} ~ \ \ \\ 
	 ~ Roots and tubers & 84 ~ \ \ & 77 ~ \ \ & \textit{64} ~ \ \ \\ 
	 ~ Fruit and vegetables & 55 ~ \ \ & 76 ~ \ \ & \textit{85} ~ \ \ \\ 
	 ~ Sugar & 33 ~ \ \ & 34 ~ \ \ & \textit{129} ~ \ \ \\ 
	 ~ Livestock & 126 ~ \ \ & 88 ~ \ \ & \textit{136} ~ \ \ \\ 
	 ~ Milk & 106 ~ \ \ & 86 ~ \ \ & \textit{119} ~ \ \ \\ 
	 ~ Meat & 155 ~ \ \ & 91 ~ \ \ & \textit{161} ~ \ \ \\ 
	 ~ Fish  & 149 ~ \ \ & 113 ~ \ \ & \textit{213} ~ \ \ \\ 
	\multicolumn{4}{l}{\textit{\normalsize{Net trade (min US\$)}}} \\ 
	 ~ Cereals & -545 ~ \ \ & -105 ~ \ \ & \textit{-136} ~ \ \ \\ 
	 ~ Fruit and vegetables & -117 ~ \ \ & -65 ~ \ \ & \textit{-274} ~ \ \ \\ 
	 ~ Meat & 102 ~ \ \ & 59 ~ \ \ & \textit{957} ~ \ \ \\ 
	 ~ Dairy products & 26 ~ \ \ & 117 ~ \ \ & \textit{1\,749} ~ \ \ \\ 
	 ~ Fish & \textit{-31} ~ \ \ & -81 ~ \ \ & \textit{-172} ~ \ \ \\ 
	\multicolumn{4}{l}{\textcolor{FAOblue}{\textbf{\large{Environment}}}} \\ 
	 ~ Forest area (\%) & 39 ~ \ \ & 41 ~ \ \ & \textit{43} ~ \ \ \\ 
	 ~ Renewable water res withdrawn (\% of total) &  ~ \ \ & \textit{19} ~ \ \ & 19 ~ \ \ \\ 
	 ~ Terrestrial protect areas (\% total land area)  & 7 ~ \ \ & 7 ~ \ \ & \textit{8} ~ \ \ \\ 
	 ~ Organic area (\% total agricultural area) &  ~ \ \ &  ~ \ \ &  ~ \ \ \\ 
	 ~ Water withdrawal by agriculture (\% of total) &  ~ \ \ & \textit{19} ~ \ \ & 19 ~ \ \ \\ 
	 ~ Biofuel production (thousand kt of oil eq.) &  ~ \ \ & 8 ~ \ \ & \textit{1\,042} ~ \ \ \\ 
	 ~ Wood pellet prod. (min tonnes) &  ~ \ \ &  ~ \ \ & \textit{1} ~ \ \ \\ 
	 ~ GHG emissions from ag (Co2 eq, gigagrams) & 49 ~ \ \ & 3 ~ \ \ & \textit{-8} ~ \ \ \\ 
       \toprule
      \end{tabular}
      \clearpage
\CountryData{ Belgium }
      \rowcolors{1}{FAOblue!10}{white}
      \begin{tabular}{L{3.9cm} R{1cm} R{1cm} R{1cm}}
      \toprule
      \multicolumn{1}{c}{} & \multicolumn{1}{c}{ 1992 } & \multicolumn{1}{c}{ 2002 } & \multicolumn{1}{c}{ 2014 } \\
      \midrule
	\multicolumn{4}{l}{\textcolor{FAOblue}{\textbf{\large{The setting}}}} \\ 
	 ~ Population, total (mln) &  ~ \ \ & 10.3 ~ \ \ & 11.1 ~ \ \ \\ 
	 ~ Population, rural (\% total population) &  ~ \ \ & 0.3 ~ \ \ & 0.3 ~ \ \ \\ 
	 ~ Govt expenditure on ag (\% total outlays) &  ~ \ \ &  ~ \ \ &  ~ \ \ \\ 
	 ~ Area harvested (mln ha) &  ~ \ \ & 7 ~ \ \ & 4 ~ \ \ \\ 
	 ~ Cropping intensity ratio (\%) &  ~ \ \ & 4.7 ~ \ \ &  ~ \ \ \\ 
	 ~ Water resources (m\textsuperscript{3}/person/year) & \textit{2} ~ \ \ & \textit{2} ~ \ \ & \textit{2} ~ \ \ \\ 
	 ~ Area equipped for irrigation (1000 ha) &  ~ \ \ &  ~ \ \ & \textit{23} ~ \ \ \\ 
	 ~ Area irrigated (\%) &  ~ \ \ &  ~ \ \ & \textit{24.3} ~ \ \ \\ 
	 ~ Employment in agriculture (\%) & 2.9 ~ \ \ & 1.8 ~ \ \ & \textit{1.2} ~ \ \ \\ 
	 ~ Employment in agriculture, female (\%) & 2.5 ~ \ \ & 1.1 ~ \ \ & \textit{0.7} ~ \ \ \\ 
	 ~ Fertilizers, Nitrogen (nutrients per ha) &  ~ \ \ & 128.1 ~ \ \ & \textit{143.2} ~ \ \ \\ 
	 ~ Fertilizers, Phosphate (nutrients per ha) &  ~ \ \ & 37.8 ~ \ \ & \textit{17.1} ~ \ \ \\ 
	 ~ Fertilizers, Potash (nutrients per ha) &  ~ \ \ & 30.2 ~ \ \ & \textit{16.8} ~ \ \ \\ 
	 ~ Energy consump, power irrigation (mln kWh) &  ~ \ \ &  ~ \ \ &  ~ \ \ \\ 
	 ~ Agr value added per worker (constant US\$) &  ~ \ \ & 44.5 ~ \ \ & \textit{65} ~ \ \ \\ 
	\multicolumn{4}{l}{\textcolor{FAOblue}{\textbf{\large{Hunger dimensions}}}} \\ 
	 ~ Dietary energy supply (kcal/pc/day) &  ~ \ \ &  ~ \ \ &  ~ \ \ \\ 
	 ~ Average dietary energy supply adequacy (\%) &  ~ \ \ & 148 ~ \ \ & 151 ~ \ \ \\ 
	 ~ Dietary en supp, cereals/roots/tubers (\%) &  ~ \ \ & 27 ~ \ \ & \textit{30} ~ \ \ \\ 
	 ~ Prevalence of undernourishment (\%) & <5.0 ~ \ \ & <5.0 ~ \ \ & <5.0 ~ \ \ \\ 
	 ~ GDP per capita (US\$, PPP) & 31\,596 ~ \ \ & 38\,036 ~ \ \ & \textit{40\,609} ~ \ \ \\ 
	 ~ Domestic food price volatility (index) &  ~ \ \ & 6.2 ~ \ \ & 6 ~ \ \ \\ 
	 ~ Cereal import dependency ratio (\%) &  ~ \ \ & 54.2 ~ \ \ & \textit{57} ~ \ \ \\ 
	 ~ Underweight, children under-5 (\%) &  ~ \ \ &  ~ \ \ &  ~ \ \ \\ 
	 ~ Improved water source (\% pop) & 100 ~ \ \ & 100 ~ \ \ & \textit{100} ~ \ \ \\ 
	\multicolumn{4}{l}{\textcolor{FAOblue}{\textbf{\large{Food Supply}}}} \\ 
	 ~ Food production value, (2004-2006 mln I\$) &  ~ \ \ & 5\,732 ~ \ \ & \textit{5\,683} ~ \ \ \\ 
	 ~ Agriculture, value added (\% GDP) & \textit{1} ~ \ \ & 1 ~ \ \ & \textit{1} ~ \ \ \\ 
	 ~ Food exports (mln US\$)  &  ~ \ \ & 14\,591 ~ \ \ & \textit{31\,389} ~ \ \ \\ 
	 ~ Food imports (mln US\$)  &  ~ \ \ & 11\,674 ~ \ \ & \textit{26\,957} ~ \ \ \\ 
	\multicolumn{4}{l}{\textit{\normalsize{Production indices (2004-06=100)}}} \\ 
	 ~ Net food &  ~ \ \ & 103 ~ \ \ & \textit{102} ~ \ \ \\ 
	 ~ Net crop &  ~ \ \ & 98 ~ \ \ & \textit{95} ~ \ \ \\ 
	 ~ Cereal &  ~ \ \ & 94 ~ \ \ & \textit{110} ~ \ \ \\ 
	 ~ Vegetable oils &  ~ \ \ & 84 ~ \ \ & \textit{189} ~ \ \ \\ 
	 ~ Roots and tubers &  ~ \ \ & 102 ~ \ \ & \textit{122} ~ \ \ \\ 
	 ~ Fruit and vegetables &  ~ \ \ & 94 ~ \ \ & \textit{102} ~ \ \ \\ 
	 ~ Sugar &  ~ \ \ & 110 ~ \ \ & \textit{74} ~ \ \ \\ 
	 ~ Livestock &  ~ \ \ & 109 ~ \ \ & \textit{106} ~ \ \ \\ 
	 ~ Milk &  ~ \ \ & 117 ~ \ \ & \textit{117} ~ \ \ \\ 
	 ~ Meat &  ~ \ \ & 106 ~ \ \ & \textit{103} ~ \ \ \\ 
	 ~ Fish  & 151 ~ \ \ & 122 ~ \ \ & \textit{103} ~ \ \ \\ 
	\multicolumn{4}{l}{\textit{\normalsize{Net trade (min US\$)}}} \\ 
	 ~ Cereals &  ~ \ \ & 301 ~ \ \ & \textit{-1} ~ \ \ \\ 
	 ~ Fruit and vegetables &  ~ \ \ & 930 ~ \ \ & \textit{1\,859} ~ \ \ \\ 
	 ~ Meat &  ~ \ \ & 1\,536 ~ \ \ & \textit{2\,586} ~ \ \ \\ 
	 ~ Dairy products &  ~ \ \ & -192 ~ \ \ & \textit{-46} ~ \ \ \\ 
	 ~ Fish &  ~ \ \ &  ~ \ \ &  ~ \ \ \\ 
	\multicolumn{4}{l}{\textcolor{FAOblue}{\textbf{\large{Environment}}}} \\ 
	 ~ Forest area (\%) &  ~ \ \ & 22 ~ \ \ & \textit{22} ~ \ \ \\ 
	 ~ Renewable water res withdrawn (\% of total) &  ~ \ \ &  ~ \ \ & 1 ~ \ \ \\ 
	 ~ Terrestrial protect areas (\% total land area)  & 3 ~ \ \ & 11 ~ \ \ & \textit{23} ~ \ \ \\ 
	 ~ Organic area (\% total agricultural area) &  ~ \ \ & \textit{2} ~ \ \ & \textit{4} ~ \ \ \\ 
	 ~ Water withdrawal by agriculture (\% of total) &  ~ \ \ &  ~ \ \ & 1 ~ \ \ \\ 
	 ~ Biofuel production (thousand kt of oil eq.) &  ~ \ \ & 5 ~ \ \ & \textit{8\,506} ~ \ \ \\ 
	 ~ Wood pellet prod. (min tonnes) &  ~ \ \ &  ~ \ \ & \textit{390} ~ \ \ \\ 
	 ~ GHG emissions from ag (Co2 eq, gigagrams) &  ~ \ \ & 8 ~ \ \ & \textit{8} ~ \ \ \\ 
       \toprule
      \end{tabular}
      \clearpage
\CountryData{ Belize }
      \rowcolors{1}{FAOblue!10}{white}
      \begin{tabular}{L{3.9cm} R{1cm} R{1cm} R{1cm}}
      \toprule
      \multicolumn{1}{c}{} & \multicolumn{1}{c}{ 1992 } & \multicolumn{1}{c}{ 2002 } & \multicolumn{1}{c}{ 2014 } \\
      \midrule
	\multicolumn{4}{l}{\textcolor{FAOblue}{\textbf{\large{The setting}}}} \\ 
	 ~ Population, total (mln) & 0.2 ~ \ \ & 0.3 ~ \ \ & 0.3 ~ \ \ \\ 
	 ~ Population, rural (\% total population) & 0.1 ~ \ \ & 0.1 ~ \ \ & 0.2 ~ \ \ \\ 
	 ~ Govt expenditure on ag (\% total outlays) &  ~ \ \ &  ~ \ \ &  ~ \ \ \\ 
	 ~ Area harvested (mln ha) & 1 ~ \ \ & 1 ~ \ \ & 1 ~ \ \ \\ 
	 ~ Cropping intensity ratio (\%) & 8.2 ~ \ \ & 7.6 ~ \ \ &  ~ \ \ \\ 
	 ~ Water resources (m\textsuperscript{3}/person/year) & \textit{110} ~ \ \ & \textit{84} ~ \ \ & \textit{65} ~ \ \ \\ 
	 ~ Area equipped for irrigation (1000 ha) &  ~ \ \ &  ~ \ \ & \textit{4} ~ \ \ \\ 
	 ~ Area irrigated (\%) &  ~ \ \ & \textit{100} ~ \ \ & \textit{100} ~ \ \ \\ 
	 ~ Employment in agriculture (\%) & \textit{27.2} ~ \ \ & \textit{19.5} ~ \ \ & \textit{19.5} ~ \ \ \\ 
	 ~ Employment in agriculture, female (\%) & \textit{6.1} ~ \ \ & \textit{3.3} ~ \ \ & \textit{3.3} ~ \ \ \\ 
	 ~ Fertilizers, Nitrogen (nutrients per ha) &  ~ \ \ & 79.9 ~ \ \ & \textit{24} ~ \ \ \\ 
	 ~ Fertilizers, Phosphate (nutrients per ha) &  ~ \ \ & 64.7 ~ \ \ & \textit{26.9} ~ \ \ \\ 
	 ~ Fertilizers, Potash (nutrients per ha) &  ~ \ \ & 24.8 ~ \ \ & \textit{8} ~ \ \ \\ 
	 ~ Energy consump, power irrigation (mln kWh) &  ~ \ \ &  ~ \ \ &  ~ \ \ \\ 
	 ~ Agr value added per worker (constant US\$) & 3.2 ~ \ \ & 3.6 ~ \ \ & \textit{4.3} ~ \ \ \\ 
	\multicolumn{4}{l}{\textcolor{FAOblue}{\textbf{\large{Hunger dimensions}}}} \\ 
	 ~ Dietary energy supply (kcal/pc/day) & 2\,640 ~ \ \ & 2\,781 ~ \ \ & 2\,773 ~ \ \ \\ 
	 ~ Average dietary energy supply adequacy (\%) & 124 ~ \ \ & 128 ~ \ \ & 123 ~ \ \ \\ 
	 ~ Dietary en supp, cereals/roots/tubers (\%) & 39 ~ \ \ & 39 ~ \ \ & \textit{38} ~ \ \ \\ 
	 ~ Prevalence of undernourishment (\%) & 8.5 ~ \ \ & 5.3 ~ \ \ & 6.3 ~ \ \ \\ 
	 ~ GDP per capita (US\$, PPP) & 6\,154 ~ \ \ & 7\,548 ~ \ \ & \textit{8\,215} ~ \ \ \\ 
	 ~ Domestic food price volatility (index) &  ~ \ \ &  ~ \ \ & \textit{27.9} ~ \ \ \\ 
	 ~ Cereal import dependency ratio (\%) & 34.7 ~ \ \ & 35.5 ~ \ \ & \textit{11.1} ~ \ \ \\ 
	 ~ Underweight, children under-5 (\%) & 5.4 ~ \ \ &  ~ \ \ & \textit{6.2} ~ \ \ \\ 
	 ~ Improved water source (\% pop) & 75.3 ~ \ \ & 87.5 ~ \ \ & \textit{99.3} ~ \ \ \\ 
	\multicolumn{4}{l}{\textcolor{FAOblue}{\textbf{\large{Food Supply}}}} \\ 
	 ~ Food production value, (2004-2006 mln I\$) & 103 ~ \ \ & 142 ~ \ \ & \textit{178} ~ \ \ \\ 
	 ~ Agriculture, value added (\% GDP) & 18 ~ \ \ & 15 ~ \ \ & \textit{15} ~ \ \ \\ 
	 ~ Food exports (mln US\$)  & 96 ~ \ \ & 107 ~ \ \ & \textit{205} ~ \ \ \\ 
	 ~ Food imports (mln US\$)  & 42 ~ \ \ & 38 ~ \ \ & \textit{87} ~ \ \ \\ 
	\multicolumn{4}{l}{\textit{\normalsize{Production indices (2004-06=100)}}} \\ 
	 ~ Net food & 59 ~ \ \ & 81 ~ \ \ & \textit{102} ~ \ \ \\ 
	 ~ Net crop & 60 ~ \ \ & 80 ~ \ \ & \textit{103} ~ \ \ \\ 
	 ~ Cereal & 69 ~ \ \ & 110 ~ \ \ & \textit{208} ~ \ \ \\ 
	 ~ Vegetable oils & 238 ~ \ \ & 157 ~ \ \ & \textit{466} ~ \ \ \\ 
	 ~ Roots and tubers & 207 ~ \ \ & 179 ~ \ \ & \textit{102} ~ \ \ \\ 
	 ~ Fruit and vegetables & 43 ~ \ \ & 67 ~ \ \ & \textit{93} ~ \ \ \\ 
	 ~ Sugar & 102 ~ \ \ & 105 ~ \ \ & \textit{98} ~ \ \ \\ 
	 ~ Livestock & 58 ~ \ \ & 99 ~ \ \ & \textit{114} ~ \ \ \\ 
	 ~ Milk & 33 ~ \ \ & 102 ~ \ \ & \textit{144} ~ \ \ \\ 
	 ~ Meat & 58 ~ \ \ & 98 ~ \ \ & \textit{112} ~ \ \ \\ 
	 ~ Fish  & 2 ~ \ \ & 69 ~ \ \ & \textit{31} ~ \ \ \\ 
	\multicolumn{4}{l}{\textit{\normalsize{Net trade (min US\$)}}} \\ 
	 ~ Cereals & -8 ~ \ \ & -11 ~ \ \ & \textit{-18} ~ \ \ \\ 
	 ~ Fruit and vegetables & 42 ~ \ \ & 66 ~ \ \ & \textit{132} ~ \ \ \\ 
	 ~ Meat & -4 ~ \ \ & -3 ~ \ \ & \textit{-7} ~ \ \ \\ 
	 ~ Dairy products & -9 ~ \ \ & -8 ~ \ \ & \textit{-17} ~ \ \ \\ 
	 ~ Fish &  ~ \ \ &  ~ \ \ &  ~ \ \ \\ 
	\multicolumn{4}{l}{\textcolor{FAOblue}{\textbf{\large{Environment}}}} \\ 
	 ~ Forest area (\%) & 69 ~ \ \ & 64 ~ \ \ & \textit{60} ~ \ \ \\ 
	 ~ Renewable water res withdrawn (\% of total) &  ~ \ \ & \textit{68} ~ \ \ & 68 ~ \ \ \\ 
	 ~ Terrestrial protect areas (\% total land area)  & 16 ~ \ \ & 26 ~ \ \ & \textit{37} ~ \ \ \\ 
	 ~ Organic area (\% total agricultural area) &  ~ \ \ & \textit{1} ~ \ \ & \textit{1} ~ \ \ \\ 
	 ~ Water withdrawal by agriculture (\% of total) &  ~ \ \ & \textit{68} ~ \ \ & 68 ~ \ \ \\ 
	 ~ Biofuel production (thousand kt of oil eq.) & 3 ~ \ \ & 3 ~ \ \ & \textit{2} ~ \ \ \\ 
	 ~ Wood pellet prod. (min tonnes) &  ~ \ \ &  ~ \ \ &  ~ \ \ \\ 
	 ~ GHG emissions from ag (Co2 eq, gigagrams) & 5 ~ \ \ & 5 ~ \ \ & \textit{5} ~ \ \ \\ 
       \toprule
      \end{tabular}
      \clearpage
\CountryData{ Benin }
      \rowcolors{1}{FAOblue!10}{white}
      \begin{tabular}{L{3.9cm} R{1cm} R{1cm} R{1cm}}
      \toprule
      \multicolumn{1}{c}{} & \multicolumn{1}{c}{ 1992 } & \multicolumn{1}{c}{ 2002 } & \multicolumn{1}{c}{ 2014 } \\
      \midrule
	\multicolumn{4}{l}{\textcolor{FAOblue}{\textbf{\large{The setting}}}} \\ 
	 ~ Population, total (mln) & 5.4 ~ \ \ & 7.4 ~ \ \ & 10.6 ~ \ \ \\ 
	 ~ Population, rural (\% total population) & 3.5 ~ \ \ & 4.5 ~ \ \ & 5.6 ~ \ \ \\ 
	 ~ Govt expenditure on ag (\% total outlays) &  ~ \ \ &  ~ \ \ &  ~ \ \ \\ 
	 ~ Area harvested (mln ha) & 2 ~ \ \ & 4 ~ \ \ & 7 ~ \ \ \\ 
	 ~ Cropping intensity ratio (\%) & 1 ~ \ \ & 1.3 ~ \ \ &  ~ \ \ \\ 
	 ~ Water resources (m\textsuperscript{3}/person/year) & \textit{5} ~ \ \ & \textit{3} ~ \ \ & \textit{3} ~ \ \ \\ 
	 ~ Area equipped for irrigation (1000 ha) &  ~ \ \ &  ~ \ \ & \textit{23} ~ \ \ \\ 
	 ~ Area irrigated (\%) &  ~ \ \ &  ~ \ \ & \textit{74.7} ~ \ \ \\ 
	 ~ Employment in agriculture (\%) &  ~ \ \ & \textit{42.7} ~ \ \ &  ~ \ \ \\ 
	 ~ Employment in agriculture, female (\%) &  ~ \ \ & \textit{32.7} ~ \ \ &  ~ \ \ \\ 
	 ~ Fertilizers, Nitrogen (nutrients per ha) &  ~ \ \ & 7 ~ \ \ & \textit{7.8} ~ \ \ \\ 
	 ~ Fertilizers, Phosphate (nutrients per ha) &  ~ \ \ & 3.1 ~ \ \ & \textit{3.9} ~ \ \ \\ 
	 ~ Fertilizers, Potash (nutrients per ha) &  ~ \ \ & 2.3 ~ \ \ & \textit{2.5} ~ \ \ \\ 
	 ~ Energy consump, power irrigation (mln kWh) & \textit{14} ~ \ \ & 14 ~ \ \ & \textit{14} ~ \ \ \\ 
	 ~ Agr value added per worker (constant US\$) & 0.6 ~ \ \ & 0.9 ~ \ \ & \textit{1.2} ~ \ \ \\ 
	\multicolumn{4}{l}{\textcolor{FAOblue}{\textbf{\large{Hunger dimensions}}}} \\ 
	 ~ Dietary energy supply (kcal/pc/day) & 2\,165 ~ \ \ & 2\,351 ~ \ \ & 2\,792 ~ \ \ \\ 
	 ~ Average dietary energy supply adequacy (\%) & 102 ~ \ \ & 110 ~ \ \ & 127 ~ \ \ \\ 
	 ~ Dietary en supp, cereals/roots/tubers (\%) & 74 ~ \ \ & 71 ~ \ \ & \textit{72} ~ \ \ \\ 
	 ~ Prevalence of undernourishment (\%) & 27.7 ~ \ \ & 20.6 ~ \ \ & 8.1 ~ \ \ \\ 
	 ~ GDP per capita (US\$, PPP) & 1\,359 ~ \ \ & 1\,586 ~ \ \ & \textit{1\,733} ~ \ \ \\ 
	 ~ Domestic food price volatility (index) &  ~ \ \ & 20.4 ~ \ \ & 21.8 ~ \ \ \\ 
	 ~ Cereal import dependency ratio (\%) & 37 ~ \ \ & 17.5 ~ \ \ & \textit{22.2} ~ \ \ \\ 
	 ~ Underweight, children under-5 (\%) &  ~ \ \ & \textit{21.5} ~ \ \ & \textit{20.2} ~ \ \ \\ 
	 ~ Improved water source (\% pop) & 59 ~ \ \ & 67.9 ~ \ \ & \textit{76.1} ~ \ \ \\ 
	\multicolumn{4}{l}{\textcolor{FAOblue}{\textbf{\large{Food Supply}}}} \\ 
	 ~ Food production value, (2004-2006 mln I\$) & 756 ~ \ \ & 1\,333 ~ \ \ & \textit{2\,185} ~ \ \ \\ 
	 ~ Agriculture, value added (\% GDP) & 34 ~ \ \ & 34 ~ \ \ & \textit{37} ~ \ \ \\ 
	 ~ Food exports (mln US\$)  & 11 ~ \ \ & 46 ~ \ \ & \textit{373} ~ \ \ \\ 
	 ~ Food imports (mln US\$)  & 170 ~ \ \ & 154 ~ \ \ & \textit{1\,612} ~ \ \ \\ 
	\multicolumn{4}{l}{\textit{\normalsize{Production indices (2004-06=100)}}} \\ 
	 ~ Net food & 52 ~ \ \ & 92 ~ \ \ & \textit{152} ~ \ \ \\ 
	 ~ Net crop & 51 ~ \ \ & 98 ~ \ \ & \textit{149} ~ \ \ \\ 
	 ~ Cereal & 55 ~ \ \ & 88 ~ \ \ & \textit{165} ~ \ \ \\ 
	 ~ Vegetable oils & 48 ~ \ \ & 109 ~ \ \ & \textit{123} ~ \ \ \\ 
	 ~ Roots and tubers & 44 ~ \ \ & 85 ~ \ \ & \textit{145} ~ \ \ \\ 
	 ~ Fruit and vegetables & 65 ~ \ \ & 93 ~ \ \ & \textit{168} ~ \ \ \\ 
	 ~ Sugar & 67 ~ \ \ & 115 ~ \ \ & \textit{91} ~ \ \ \\ 
	 ~ Livestock & 71 ~ \ \ & 89 ~ \ \ & \textit{129} ~ \ \ \\ 
	 ~ Milk & 64 ~ \ \ & 92 ~ \ \ & \textit{120} ~ \ \ \\ 
	 ~ Meat & 72 ~ \ \ & 89 ~ \ \ & \textit{128} ~ \ \ \\ 
	 ~ Fish  & 86 ~ \ \ & 107 ~ \ \ & \textit{114} ~ \ \ \\ 
	\multicolumn{4}{l}{\textit{\normalsize{Net trade (min US\$)}}} \\ 
	 ~ Cereals & -126 ~ \ \ & -40 ~ \ \ & \textit{-625} ~ \ \ \\ 
	 ~ Fruit and vegetables & -1 ~ \ \ & 4 ~ \ \ & \textit{94} ~ \ \ \\ 
	 ~ Meat & -16 ~ \ \ & -40 ~ \ \ & \textit{-371} ~ \ \ \\ 
	 ~ Dairy products & -5 ~ \ \ & -19 ~ \ \ & \textit{-30} ~ \ \ \\ 
	 ~ Fish & -8 ~ \ \ & -5 ~ \ \ & \textit{-35} ~ \ \ \\ 
	\multicolumn{4}{l}{\textcolor{FAOblue}{\textbf{\large{Environment}}}} \\ 
	 ~ Forest area (\%) & 50 ~ \ \ & 44 ~ \ \ & \textit{40} ~ \ \ \\ 
	 ~ Renewable water res withdrawn (\% of total) &  ~ \ \ & \textit{45} ~ \ \ & 45 ~ \ \ \\ 
	 ~ Terrestrial protect areas (\% total land area)  & 24 ~ \ \ & 24 ~ \ \ & \textit{26} ~ \ \ \\ 
	 ~ Organic area (\% total agricultural area) &  ~ \ \ & \textit{0} ~ \ \ & \textit{0} ~ \ \ \\ 
	 ~ Water withdrawal by agriculture (\% of total) &  ~ \ \ & \textit{45} ~ \ \ & 45 ~ \ \ \\ 
	 ~ Biofuel production (thousand kt of oil eq.) & 0 ~ \ \ & 0 ~ \ \ & \textit{0} ~ \ \ \\ 
	 ~ Wood pellet prod. (min tonnes) &  ~ \ \ &  ~ \ \ &  ~ \ \ \\ 
	 ~ GHG emissions from ag (Co2 eq, gigagrams) & 19 ~ \ \ & 16 ~ \ \ & \textit{15} ~ \ \ \\ 
       \toprule
      \end{tabular}
      \clearpage
\CountryData{ Bhutan }
      \rowcolors{1}{FAOblue!10}{white}
      \begin{tabular}{L{3.9cm} R{1cm} R{1cm} R{1cm}}
      \toprule
      \multicolumn{1}{c}{} & \multicolumn{1}{c}{ 1992 } & \multicolumn{1}{c}{ 2002 } & \multicolumn{1}{c}{ 2014 } \\
      \midrule
	\multicolumn{4}{l}{\textcolor{FAOblue}{\textbf{\large{The setting}}}} \\ 
	 ~ Population, total (mln) & 0.5 ~ \ \ & 0.6 ~ \ \ & 0.8 ~ \ \ \\ 
	 ~ Population, rural (\% total population) & 0.4 ~ \ \ & 0.4 ~ \ \ & 0.5 ~ \ \ \\ 
	 ~ Govt expenditure on ag (\% total outlays) &  ~ \ \ & 9 ~ \ \ & \textit{11.6} ~ \ \ \\ 
	 ~ Area harvested (mln ha) & 0 ~ \ \ & 0 ~ \ \ & 0 ~ \ \ \\ 
	 ~ Cropping intensity ratio (\%) & 0.6 ~ \ \ & 0.6 ~ \ \ &  ~ \ \ \\ 
	 ~ Water resources (m\textsuperscript{3}/person/year) & \textit{150} ~ \ \ & \textit{127} ~ \ \ & \textit{103} ~ \ \ \\ 
	 ~ Area equipped for irrigation (1000 ha) &  ~ \ \ &  ~ \ \ & \textit{32} ~ \ \ \\ 
	 ~ Area irrigated (\%) &  ~ \ \ &  ~ \ \ & \textit{100} ~ \ \ \\ 
	 ~ Employment in agriculture (\%) &  ~ \ \ & \textit{43.6} ~ \ \ & \textit{62.2} ~ \ \ \\ 
	 ~ Employment in agriculture, female (\%) &  ~ \ \ & \textit{63} ~ \ \ & \textit{75.1} ~ \ \ \\ 
	 ~ Fertilizers, Nitrogen (nutrients per ha) &  ~ \ \ & 1.5 ~ \ \ & \textit{1.8} ~ \ \ \\ 
	 ~ Fertilizers, Phosphate (nutrients per ha) &  ~ \ \ & 0.3 ~ \ \ & \textit{0.7} ~ \ \ \\ 
	 ~ Fertilizers, Potash (nutrients per ha) &  ~ \ \ & 0.2 ~ \ \ & \textit{0.5} ~ \ \ \\ 
	 ~ Energy consump, power irrigation (mln kWh) & \textit{0} ~ \ \ & 0 ~ \ \ & \textit{0} ~ \ \ \\ 
	 ~ Agr value added per worker (constant US\$) & 0.9 ~ \ \ & 0.9 ~ \ \ & \textit{0.6} ~ \ \ \\ 
	\multicolumn{4}{l}{\textcolor{FAOblue}{\textbf{\large{Hunger dimensions}}}} \\ 
	 ~ Dietary energy supply (kcal/pc/day) &  ~ \ \ &  ~ \ \ &  ~ \ \ \\ 
	 ~ Average dietary energy supply adequacy (\%) &  ~ \ \ &  ~ \ \ &  ~ \ \ \\ 
	 ~ Dietary en supp, cereals/roots/tubers (\%) &  ~ \ \ &  ~ \ \ &  ~ \ \ \\ 
	 ~ Prevalence of undernourishment (\%) &  ~ \ \ &  ~ \ \ &  ~ \ \ \\ 
	 ~ GDP per capita (US\$, PPP) & 2\,464 ~ \ \ & 4\,062 ~ \ \ & \textit{7\,167} ~ \ \ \\ 
	 ~ Domestic food price volatility (index) &  ~ \ \ &  ~ \ \ & 6.4 ~ \ \ \\ 
	 ~ Cereal import dependency ratio (\%) &  ~ \ \ &  ~ \ \ &  ~ \ \ \\ 
	 ~ Underweight, children under-5 (\%) &  ~ \ \ & \textit{14.1} ~ \ \ & \textit{12.8} ~ \ \ \\ 
	 ~ Improved water source (\% pop) &  ~ \ \ & 87.6 ~ \ \ & \textit{98.1} ~ \ \ \\ 
	\multicolumn{4}{l}{\textcolor{FAOblue}{\textbf{\large{Food Supply}}}} \\ 
	 ~ Food production value, (2004-2006 mln I\$) & 101 ~ \ \ & 100 ~ \ \ & \textit{142} ~ \ \ \\ 
	 ~ Agriculture, value added (\% GDP) & 34 ~ \ \ & 26 ~ \ \ & \textit{17} ~ \ \ \\ 
	 ~ Food exports (mln US\$)  & 9 ~ \ \ & 7 ~ \ \ & \textit{33} ~ \ \ \\ 
	 ~ Food imports (mln US\$)  & 13 ~ \ \ & 12 ~ \ \ & \textit{124} ~ \ \ \\ 
	\multicolumn{4}{l}{\textit{\normalsize{Production indices (2004-06=100)}}} \\ 
	 ~ Net food & 69 ~ \ \ & 68 ~ \ \ & \textit{97} ~ \ \ \\ 
	 ~ Net crop & 64 ~ \ \ & 61 ~ \ \ & \textit{97} ~ \ \ \\ 
	 ~ Cereal & 72 ~ \ \ & 53 ~ \ \ & \textit{99} ~ \ \ \\ 
	 ~ Vegetable oils & 42 ~ \ \ & 86 ~ \ \ & \textit{16} ~ \ \ \\ 
	 ~ Roots and tubers & 69 ~ \ \ & 69 ~ \ \ & \textit{83} ~ \ \ \\ 
	 ~ Fruit and vegetables & 53 ~ \ \ & 72 ~ \ \ & \textit{92} ~ \ \ \\ 
	 ~ Sugar & 96 ~ \ \ & 99 ~ \ \ & \textit{117} ~ \ \ \\ 
	 ~ Livestock & 88 ~ \ \ & 96 ~ \ \ & \textit{96} ~ \ \ \\ 
	 ~ Milk & 91 ~ \ \ & 96 ~ \ \ & \textit{89} ~ \ \ \\ 
	 ~ Meat & 86 ~ \ \ & 96 ~ \ \ & \textit{102} ~ \ \ \\ 
	 ~ Fish  & 264 ~ \ \ & 129 ~ \ \ & \textit{104} ~ \ \ \\ 
	\multicolumn{4}{l}{\textit{\normalsize{Net trade (min US\$)}}} \\ 
	 ~ Cereals & -5 ~ \ \ & -3 ~ \ \ & \textit{-39} ~ \ \ \\ 
	 ~ Fruit and vegetables & 4 ~ \ \ & 3 ~ \ \ & \textit{5} ~ \ \ \\ 
	 ~ Meat &  ~ \ \ & 0 ~ \ \ & \textit{-14} ~ \ \ \\ 
	 ~ Dairy products & \textit{-1} ~ \ \ & -2 ~ \ \ & \textit{-18} ~ \ \ \\ 
	 ~ Fish &  ~ \ \ &  ~ \ \ &  ~ \ \ \\ 
	\multicolumn{4}{l}{\textcolor{FAOblue}{\textbf{\large{Environment}}}} \\ 
	 ~ Forest area (\%) & 65 ~ \ \ & 79 ~ \ \ & \textit{86} ~ \ \ \\ 
	 ~ Renewable water res withdrawn (\% of total) &  ~ \ \ &  ~ \ \ & 94 ~ \ \ \\ 
	 ~ Terrestrial protect areas (\% total land area)  & 14 ~ \ \ & 28 ~ \ \ & \textit{28} ~ \ \ \\ 
	 ~ Organic area (\% total agricultural area) &  ~ \ \ & \textit{0} ~ \ \ & \textit{1} ~ \ \ \\ 
	 ~ Water withdrawal by agriculture (\% of total) &  ~ \ \ &  ~ \ \ & 94 ~ \ \ \\ 
	 ~ Biofuel production (thousand kt of oil eq.) &  ~ \ \ &  ~ \ \ &  ~ \ \ \\ 
	 ~ Wood pellet prod. (min tonnes) &  ~ \ \ &  ~ \ \ &  ~ \ \ \\ 
	 ~ GHG emissions from ag (Co2 eq, gigagrams) & -6 ~ \ \ & -8 ~ \ \ & \textit{-8} ~ \ \ \\ 
       \toprule
      \end{tabular}
      \clearpage
\CountryData{ Bolivia }
      \rowcolors{1}{FAOblue!10}{white}
      \begin{tabular}{L{3.9cm} R{1cm} R{1cm} R{1cm}}
      \toprule
      \multicolumn{1}{c}{} & \multicolumn{1}{c}{ 1992 } & \multicolumn{1}{c}{ 2002 } & \multicolumn{1}{c}{ 2014 } \\
      \midrule
	\multicolumn{4}{l}{\textcolor{FAOblue}{\textbf{\large{The setting}}}} \\ 
	 ~ Population, total (mln) & 7.1 ~ \ \ & 8.8 ~ \ \ & 10.8 ~ \ \ \\ 
	 ~ Population, rural (\% total population) & 3 ~ \ \ & 3.3 ~ \ \ & 3.5 ~ \ \ \\ 
	 ~ Govt expenditure on ag (\% total outlays) &  ~ \ \ &  ~ \ \ &  ~ \ \ \\ 
	 ~ Area harvested (mln ha) & 3 ~ \ \ & 5 ~ \ \ & 8 ~ \ \ \\ 
	 ~ Cropping intensity ratio (\%) & 0.1 ~ \ \ & 0.1 ~ \ \ &  ~ \ \ \\ 
	 ~ Water resources (m\textsuperscript{3}/person/year) & \textit{79} ~ \ \ & \textit{64} ~ \ \ & \textit{54} ~ \ \ \\ 
	 ~ Area equipped for irrigation (1000 ha) &  ~ \ \ &  ~ \ \ & \textit{300} ~ \ \ \\ 
	 ~ Area irrigated (\%) &  ~ \ \ &  ~ \ \ & \textit{100} ~ \ \ \\ 
	 ~ Employment in agriculture (\%) & 2.1 ~ \ \ & 39.6 ~ \ \ & \textit{32.1} ~ \ \ \\ 
	 ~ Employment in agriculture, female (\%) & 1.3 ~ \ \ & 36.2 ~ \ \ & \textit{32.9} ~ \ \ \\ 
	 ~ Fertilizers, Nitrogen (nutrients per ha) &  ~ \ \ & 0.2 ~ \ \ & \textit{0.7} ~ \ \ \\ 
	 ~ Fertilizers, Phosphate (nutrients per ha) &  ~ \ \ & 0.2 ~ \ \ & \textit{0.3} ~ \ \ \\ 
	 ~ Fertilizers, Potash (nutrients per ha) &  ~ \ \ & 0 ~ \ \ & \textit{0.1} ~ \ \ \\ 
	 ~ Energy consump, power irrigation (mln kWh) &  ~ \ \ &  ~ \ \ & \textit{46} ~ \ \ \\ 
	 ~ Agr value added per worker (constant US\$) & 0.6 ~ \ \ & 0.6 ~ \ \ & \textit{0.7} ~ \ \ \\ 
	\multicolumn{4}{l}{\textcolor{FAOblue}{\textbf{\large{Hunger dimensions}}}} \\ 
	 ~ Dietary energy supply (kcal/pc/day) & 2\,016 ~ \ \ & 2\,098 ~ \ \ & 2\,277 ~ \ \ \\ 
	 ~ Average dietary energy supply adequacy (\%) & 95 ~ \ \ & 97 ~ \ \ & 103 ~ \ \ \\ 
	 ~ Dietary en supp, cereals/roots/tubers (\%) & 53 ~ \ \ & 51 ~ \ \ & \textit{52} ~ \ \ \\ 
	 ~ Prevalence of undernourishment (\%) & 35.9 ~ \ \ & 31 ~ \ \ & 16.6 ~ \ \ \\ 
	 ~ GDP per capita (US\$, PPP) & 3\,817 ~ \ \ & 4\,335 ~ \ \ & \textit{5\,934} ~ \ \ \\ 
	 ~ Domestic food price volatility (index) &  ~ \ \ & 4.4 ~ \ \ & 12.2 ~ \ \ \\ 
	 ~ Cereal import dependency ratio (\%) & 24 ~ \ \ & 28.4 ~ \ \ & \textit{18.7} ~ \ \ \\ 
	 ~ Underweight, children under-5 (\%) & 10.5 ~ \ \ & \textit{5.9} ~ \ \ & \textit{4.5} ~ \ \ \\ 
	 ~ Improved water source (\% pop) & 71.1 ~ \ \ & 80.7 ~ \ \ & \textit{88.1} ~ \ \ \\ 
	\multicolumn{4}{l}{\textcolor{FAOblue}{\textbf{\large{Food Supply}}}} \\ 
	 ~ Food production value, (2004-2006 mln I\$) & 1\,494 ~ \ \ & 2\,322 ~ \ \ & \textit{3\,547} ~ \ \ \\ 
	 ~ Agriculture, value added (\% GDP) & 16 ~ \ \ & 15 ~ \ \ & \textit{13} ~ \ \ \\ 
	 ~ Food exports (mln US\$)  & 76 ~ \ \ & 213 ~ \ \ & \textit{1\,004} ~ \ \ \\ 
	 ~ Food imports (mln US\$)  & 121 ~ \ \ & 213 ~ \ \ & \textit{494} ~ \ \ \\ 
	\multicolumn{4}{l}{\textit{\normalsize{Production indices (2004-06=100)}}} \\ 
	 ~ Net food & 57 ~ \ \ & 89 ~ \ \ & \textit{136} ~ \ \ \\ 
	 ~ Net crop & 56 ~ \ \ & 92 ~ \ \ & \textit{133} ~ \ \ \\ 
	 ~ Cereal & 59 ~ \ \ & 79 ~ \ \ & \textit{136} ~ \ \ \\ 
	 ~ Vegetable oils & 23 ~ \ \ & 90 ~ \ \ & \textit{156} ~ \ \ \\ 
	 ~ Roots and tubers & 89 ~ \ \ & 97 ~ \ \ & \textit{127} ~ \ \ \\ 
	 ~ Fruit and vegetables & 90 ~ \ \ & 119 ~ \ \ & \textit{108} ~ \ \ \\ 
	 ~ Sugar & 53 ~ \ \ & 85 ~ \ \ & \textit{144} ~ \ \ \\ 
	 ~ Livestock & 59 ~ \ \ & 86 ~ \ \ & \textit{134} ~ \ \ \\ 
	 ~ Milk & 53 ~ \ \ & 100 ~ \ \ & \textit{177} ~ \ \ \\ 
	 ~ Meat & 57 ~ \ \ & 85 ~ \ \ & \textit{131} ~ \ \ \\ 
	 ~ Fish  & 77 ~ \ \ & 100 ~ \ \ & \textit{122} ~ \ \ \\ 
	\multicolumn{4}{l}{\textit{\normalsize{Net trade (min US\$)}}} \\ 
	 ~ Cereals & -68 ~ \ \ & -88 ~ \ \ & \textit{-116} ~ \ \ \\ 
	 ~ Fruit and vegetables & 9 ~ \ \ & 30 ~ \ \ & \textit{194} ~ \ \ \\ 
	 ~ Meat & -3 ~ \ \ & 0 ~ \ \ & \textit{2} ~ \ \ \\ 
	 ~ Dairy products & -9 ~ \ \ & -6 ~ \ \ & \textit{0} ~ \ \ \\ 
	 ~ Fish &  ~ \ \ &  ~ \ \ &  ~ \ \ \\ 
	\multicolumn{4}{l}{\textcolor{FAOblue}{\textbf{\large{Environment}}}} \\ 
	 ~ Forest area (\%) & 57 ~ \ \ & 55 ~ \ \ & \textit{52} ~ \ \ \\ 
	 ~ Renewable water res withdrawn (\% of total) &  ~ \ \ &  ~ \ \ & 92 ~ \ \ \\ 
	 ~ Terrestrial protect areas (\% total land area)  & 9 ~ \ \ & 18 ~ \ \ & \textit{21} ~ \ \ \\ 
	 ~ Organic area (\% total agricultural area) &  ~ \ \ & \textit{0} ~ \ \ & \textit{0} ~ \ \ \\ 
	 ~ Water withdrawal by agriculture (\% of total) &  ~ \ \ &  ~ \ \ & 92 ~ \ \ \\ 
	 ~ Biofuel production (thousand kt of oil eq.) & 13 ~ \ \ & 23 ~ \ \ & \textit{19} ~ \ \ \\ 
	 ~ Wood pellet prod. (min tonnes) &  ~ \ \ &  ~ \ \ &  ~ \ \ \\ 
	 ~ GHG emissions from ag (Co2 eq, gigagrams) & 99 ~ \ \ & 98 ~ \ \ & \textit{115} ~ \ \ \\ 
       \toprule
      \end{tabular}
      \clearpage
\CountryData{ Bosnia and Herzegovina }
      \rowcolors{1}{FAOblue!10}{white}
      \begin{tabular}{L{3.9cm} R{1cm} R{1cm} R{1cm}}
      \toprule
      \multicolumn{1}{c}{} & \multicolumn{1}{c}{ 1992 } & \multicolumn{1}{c}{ 2002 } & \multicolumn{1}{c}{ 2014 } \\
      \midrule
	\multicolumn{4}{l}{\textcolor{FAOblue}{\textbf{\large{The setting}}}} \\ 
	 ~ Population, total (mln) & 4.1 ~ \ \ & 3.9 ~ \ \ & 3.8 ~ \ \ \\ 
	 ~ Population, rural (\% total population) & 2.5 ~ \ \ & 2.2 ~ \ \ & 1.9 ~ \ \ \\ 
	 ~ Govt expenditure on ag (\% total outlays) &  ~ \ \ &  ~ \ \ &  ~ \ \ \\ 
	 ~ Area harvested (mln ha) & 1 ~ \ \ & 1 ~ \ \ & 1 ~ \ \ \\ 
	 ~ Cropping intensity ratio (\%) & 0.5 ~ \ \ & 0.6 ~ \ \ &  ~ \ \ \\ 
	 ~ Water resources (m\textsuperscript{3}/person/year) & \textit{10} ~ \ \ & \textit{10} ~ \ \ & \textit{10} ~ \ \ \\ 
	 ~ Area equipped for irrigation (1000 ha) &  ~ \ \ &  ~ \ \ & \textit{3} ~ \ \ \\ 
	 ~ Area irrigated (\%) &  ~ \ \ &  ~ \ \ &  ~ \ \ \\ 
	 ~ Employment in agriculture (\%) &  ~ \ \ &  ~ \ \ & \textit{20.5} ~ \ \ \\ 
	 ~ Employment in agriculture, female (\%) &  ~ \ \ &  ~ \ \ & \textit{22.7} ~ \ \ \\ 
	 ~ Fertilizers, Nitrogen (nutrients per ha) &  ~ \ \ & 8.8 ~ \ \ & \textit{37.2} ~ \ \ \\ 
	 ~ Fertilizers, Phosphate (nutrients per ha) &  ~ \ \ & 3.3 ~ \ \ & \textit{4.5} ~ \ \ \\ 
	 ~ Fertilizers, Potash (nutrients per ha) &  ~ \ \ & 3.3 ~ \ \ & \textit{4.5} ~ \ \ \\ 
	 ~ Energy consump, power irrigation (mln kWh) &  ~ \ \ &  ~ \ \ &  ~ \ \ \\ 
	 ~ Agr value added per worker (constant US\$) &  ~ \ \ & \textit{13.8} ~ \ \ & \textit{13.8} ~ \ \ \\ 
	\multicolumn{4}{l}{\textcolor{FAOblue}{\textbf{\large{Hunger dimensions}}}} \\ 
	 ~ Dietary energy supply (kcal/pc/day) &  ~ \ \ &  ~ \ \ &  ~ \ \ \\ 
	 ~ Average dietary energy supply adequacy (\%) & 99 ~ \ \ & 116 ~ \ \ & 125 ~ \ \ \\ 
	 ~ Dietary en supp, cereals/roots/tubers (\%) & 65 ~ \ \ & 58 ~ \ \ & \textit{50} ~ \ \ \\ 
	 ~ Prevalence of undernourishment (\%) & <5.0 ~ \ \ & <5.0 ~ \ \ & <5.0 ~ \ \ \\ 
	 ~ GDP per capita (US\$, PPP) & \textit{1\,976} ~ \ \ & 6\,660 ~ \ \ & \textit{9\,387} ~ \ \ \\ 
	 ~ Domestic food price volatility (index) &  ~ \ \ & \textit{9.4} ~ \ \ & \textit{6.3} ~ \ \ \\ 
	 ~ Cereal import dependency ratio (\%) & 1.2 ~ \ \ & 32.7 ~ \ \ & \textit{34.8} ~ \ \ \\ 
	 ~ Underweight, children under-5 (\%) &  ~ \ \ & \textit{4.2} ~ \ \ & \textit{1.5} ~ \ \ \\ 
	 ~ Improved water source (\% pop) & 97.2 ~ \ \ & 97.9 ~ \ \ & \textit{99.6} ~ \ \ \\ 
	\multicolumn{4}{l}{\textcolor{FAOblue}{\textbf{\large{Food Supply}}}} \\ 
	 ~ Food production value, (2004-2006 mln I\$) & 760 ~ \ \ & 601 ~ \ \ & \textit{954} ~ \ \ \\ 
	 ~ Agriculture, value added (\% GDP) & \textit{21} ~ \ \ & 10 ~ \ \ & \textit{8} ~ \ \ \\ 
	 ~ Food exports (mln US\$)  & 35 ~ \ \ & 27 ~ \ \ & \textit{318} ~ \ \ \\ 
	 ~ Food imports (mln US\$)  & 60 ~ \ \ & 521 ~ \ \ & \textit{1\,275} ~ \ \ \\ 
	\multicolumn{4}{l}{\textit{\normalsize{Production indices (2004-06=100)}}} \\ 
	 ~ Net food & 93 ~ \ \ & 74 ~ \ \ & \textit{117} ~ \ \ \\ 
	 ~ Net crop & 80 ~ \ \ & 76 ~ \ \ & \textit{111} ~ \ \ \\ 
	 ~ Cereal & 79 ~ \ \ & 95 ~ \ \ & \textit{89} ~ \ \ \\ 
	 ~ Vegetable oils & 125 ~ \ \ & 93 ~ \ \ & \textit{90} ~ \ \ \\ 
	 ~ Roots and tubers & 73 ~ \ \ & 92 ~ \ \ & \textit{85} ~ \ \ \\ 
	 ~ Fruit and vegetables & 69 ~ \ \ & 59 ~ \ \ & \textit{134} ~ \ \ \\ 
	 ~ Sugar & 1\,405\,714 ~ \ \ & \textit{1\,714} ~ \ \ & \textit{103} ~ \ \ \\ 
	 ~ Livestock & 96 ~ \ \ & 70 ~ \ \ & \textit{119} ~ \ \ \\ 
	 ~ Milk & 71 ~ \ \ & 81 ~ \ \ & \textit{106} ~ \ \ \\ 
	 ~ Meat & 195 ~ \ \ & 29 ~ \ \ & \textit{163} ~ \ \ \\ 
	 ~ Fish  & 3 ~ \ \ & 67 ~ \ \ & \textit{44} ~ \ \ \\ 
	\multicolumn{4}{l}{\textit{\normalsize{Net trade (min US\$)}}} \\ 
	 ~ Cereals & 1 ~ \ \ & -85 ~ \ \ & \textit{-261} ~ \ \ \\ 
	 ~ Fruit and vegetables & 9 ~ \ \ & -29 ~ \ \ & \textit{-104} ~ \ \ \\ 
	 ~ Meat & -18 ~ \ \ & -84 ~ \ \ & \textit{-113} ~ \ \ \\ 
	 ~ Dairy products & -2 ~ \ \ & -60 ~ \ \ & \textit{-57} ~ \ \ \\ 
	 ~ Fish &  ~ \ \ &  ~ \ \ &  ~ \ \ \\ 
	\multicolumn{4}{l}{\textcolor{FAOblue}{\textbf{\large{Environment}}}} \\ 
	 ~ Forest area (\%) & 43 ~ \ \ & 43 ~ \ \ & \textit{43} ~ \ \ \\ 
	 ~ Renewable water res withdrawn (\% of total) &  ~ \ \ &  ~ \ \ & 0 ~ \ \ \\ 
	 ~ Terrestrial protect areas (\% total land area)  & 1 ~ \ \ & 1 ~ \ \ & \textit{1} ~ \ \ \\ 
	 ~ Organic area (\% total agricultural area) &  ~ \ \ & \textit{0} ~ \ \ & \textit{0} ~ \ \ \\ 
	 ~ Water withdrawal by agriculture (\% of total) &  ~ \ \ &  ~ \ \ & 0 ~ \ \ \\ 
	 ~ Biofuel production (thousand kt of oil eq.) &  ~ \ \ &  ~ \ \ &  ~ \ \ \\ 
	 ~ Wood pellet prod. (min tonnes) &  ~ \ \ &  ~ \ \ & \textit{184} ~ \ \ \\ 
	 ~ GHG emissions from ag (Co2 eq, gigagrams) & 2 ~ \ \ & 2 ~ \ \ & \textit{3} ~ \ \ \\ 
       \toprule
      \end{tabular}
      \clearpage
\CountryData{ Botswana }
      \rowcolors{1}{FAOblue!10}{white}
      \begin{tabular}{L{3.9cm} R{1cm} R{1cm} R{1cm}}
      \toprule
      \multicolumn{1}{c}{} & \multicolumn{1}{c}{ 1992 } & \multicolumn{1}{c}{ 2002 } & \multicolumn{1}{c}{ 2014 } \\
      \midrule
	\multicolumn{4}{l}{\textcolor{FAOblue}{\textbf{\large{The setting}}}} \\ 
	 ~ Population, total (mln) & 1.5 ~ \ \ & 1.8 ~ \ \ & 2 ~ \ \ \\ 
	 ~ Population, rural (\% total population) & 0.8 ~ \ \ & 0.8 ~ \ \ & 0.7 ~ \ \ \\ 
	 ~ Govt expenditure on ag (\% total outlays) &  ~ \ \ & 4.1 ~ \ \ & \textit{2.7} ~ \ \ \\ 
	 ~ Area harvested (mln ha) & 0 ~ \ \ & 0 ~ \ \ & 0 ~ \ \ \\ 
	 ~ Cropping intensity ratio (\%) & 0 ~ \ \ & 0 ~ \ \ &  ~ \ \ \\ 
	 ~ Water resources (m\textsuperscript{3}/person/year) & \textit{8} ~ \ \ & \textit{7} ~ \ \ & \textit{6} ~ \ \ \\ 
	 ~ Area equipped for irrigation (1000 ha) &  ~ \ \ &  ~ \ \ & \textit{2} ~ \ \ \\ 
	 ~ Area irrigated (\%) & 100 ~ \ \ &  ~ \ \ &  ~ \ \ \\ 
	 ~ Employment in agriculture (\%) &  ~ \ \ & \textit{21.2} ~ \ \ & \textit{29.9} ~ \ \ \\ 
	 ~ Employment in agriculture, female (\%) &  ~ \ \ & \textit{12.9} ~ \ \ & \textit{24.3} ~ \ \ \\ 
	 ~ Fertilizers, Nitrogen (nutrients per ha) &  ~ \ \ &  ~ \ \ & \textit{0.6} ~ \ \ \\ 
	 ~ Fertilizers, Phosphate (nutrients per ha) &  ~ \ \ &  ~ \ \ & \textit{0} ~ \ \ \\ 
	 ~ Fertilizers, Potash (nutrients per ha) &  ~ \ \ &  ~ \ \ & \textit{0} ~ \ \ \\ 
	 ~ Energy consump, power irrigation (mln kWh) & 3 ~ \ \ & 3 ~ \ \ & \textit{3} ~ \ \ \\ 
	 ~ Agr value added per worker (constant US\$) & 0.9 ~ \ \ & 0.6 ~ \ \ & \textit{0.8} ~ \ \ \\ 
	\multicolumn{4}{l}{\textcolor{FAOblue}{\textbf{\large{Hunger dimensions}}}} \\ 
	 ~ Dietary energy supply (kcal/pc/day) & 2\,164 ~ \ \ & 2\,126 ~ \ \ & 2\,326 ~ \ \ \\ 
	 ~ Average dietary energy supply adequacy (\%) & 98 ~ \ \ & 92 ~ \ \ & 99 ~ \ \ \\ 
	 ~ Dietary en supp, cereals/roots/tubers (\%) & 48 ~ \ \ & 50 ~ \ \ & \textit{49} ~ \ \ \\ 
	 ~ Prevalence of undernourishment (\%) & 26.7 ~ \ \ & 35.5 ~ \ \ & 24.8 ~ \ \ \\ 
	 ~ GDP per capita (US\$, PPP) & 8\,456 ~ \ \ & 10\,602 ~ \ \ & \textit{15\,247} ~ \ \ \\ 
	 ~ Domestic food price volatility (index) &  ~ \ \ & 3.8 ~ \ \ & 3.6 ~ \ \ \\ 
	 ~ Cereal import dependency ratio (\%) & 76.1 ~ \ \ & 86 ~ \ \ & \textit{80.8} ~ \ \ \\ 
	 ~ Underweight, children under-5 (\%) &  ~ \ \ & \textit{10.7} ~ \ \ & \textit{11.2} ~ \ \ \\ 
	 ~ Improved water source (\% pop) & 92.8 ~ \ \ & 95.3 ~ \ \ & \textit{96.8} ~ \ \ \\ 
	\multicolumn{4}{l}{\textcolor{FAOblue}{\textbf{\large{Food Supply}}}} \\ 
	 ~ Food production value, (2004-2006 mln I\$) & 226 ~ \ \ & 204 ~ \ \ & \textit{293} ~ \ \ \\ 
	 ~ Agriculture, value added (\% GDP) & 5 ~ \ \ & 3 ~ \ \ & \textit{3} ~ \ \ \\ 
	 ~ Food exports (mln US\$)  & 80 ~ \ \ & 49 ~ \ \ & \textit{106} ~ \ \ \\ 
	 ~ Food imports (mln US\$)  & 243 ~ \ \ & 234 ~ \ \ & \textit{564} ~ \ \ \\ 
	\multicolumn{4}{l}{\textit{\normalsize{Production indices (2004-06=100)}}} \\ 
	 ~ Net food & 101 ~ \ \ & 91 ~ \ \ & \textit{130} ~ \ \ \\ 
	 ~ Net crop & 73 ~ \ \ & 103 ~ \ \ & \textit{96} ~ \ \ \\ 
	 ~ Cereal & 60 ~ \ \ & 98 ~ \ \ & \textit{124} ~ \ \ \\ 
	 ~ Vegetable oils & 18 ~ \ \ & 96 ~ \ \ & \textit{170} ~ \ \ \\ 
	 ~ Roots and tubers & 73 ~ \ \ & 95 ~ \ \ & \textit{101} ~ \ \ \\ 
	 ~ Fruit and vegetables & 77 ~ \ \ & 93 ~ \ \ & \textit{131} ~ \ \ \\ 
	 ~ Sugar &  ~ \ \ &  ~ \ \ &  ~ \ \ \\ 
	 ~ Livestock & 106 ~ \ \ & 88 ~ \ \ & \textit{137} ~ \ \ \\ 
	 ~ Milk & 92 ~ \ \ & 97 ~ \ \ & \textit{113} ~ \ \ \\ 
	 ~ Meat & 111 ~ \ \ & 86 ~ \ \ & \textit{143} ~ \ \ \\ 
	 ~ Fish  &  ~ \ \ &  ~ \ \ &  ~ \ \ \\ 
	\multicolumn{4}{l}{\textit{\normalsize{Net trade (min US\$)}}} \\ 
	 ~ Cereals & -54 ~ \ \ & -80 ~ \ \ & \textit{-156} ~ \ \ \\ 
	 ~ Fruit and vegetables & -47 ~ \ \ & -48 ~ \ \ & \textit{-115} ~ \ \ \\ 
	 ~ Meat & 60 ~ \ \ & 37 ~ \ \ & \textit{47} ~ \ \ \\ 
	 ~ Dairy products & -41 ~ \ \ & -32 ~ \ \ & \textit{-60} ~ \ \ \\ 
	 ~ Fish &  ~ \ \ &  ~ \ \ &  ~ \ \ \\ 
	\multicolumn{4}{l}{\textcolor{FAOblue}{\textbf{\large{Environment}}}} \\ 
	 ~ Forest area (\%) & 24 ~ \ \ & 22 ~ \ \ & \textit{20} ~ \ \ \\ 
	 ~ Renewable water res withdrawn (\% of total) &  ~ \ \ & \textit{41} ~ \ \ & 41 ~ \ \ \\ 
	 ~ Terrestrial protect areas (\% total land area)  & 31 ~ \ \ & 31 ~ \ \ & \textit{37} ~ \ \ \\ 
	 ~ Organic area (\% total agricultural area) &  ~ \ \ &  ~ \ \ &  ~ \ \ \\ 
	 ~ Water withdrawal by agriculture (\% of total) &  ~ \ \ & \textit{41} ~ \ \ & 41 ~ \ \ \\ 
	 ~ Biofuel production (thousand kt of oil eq.) &  ~ \ \ &  ~ \ \ &  ~ \ \ \\ 
	 ~ Wood pellet prod. (min tonnes) &  ~ \ \ &  ~ \ \ &  ~ \ \ \\ 
	 ~ GHG emissions from ag (Co2 eq, gigagrams) & 18 ~ \ \ & 26 ~ \ \ & \textit{18} ~ \ \ \\ 
       \toprule
      \end{tabular}
      \clearpage
\CountryData{ Brazil }
      \rowcolors{1}{FAOblue!10}{white}
      \begin{tabular}{L{3.9cm} R{1cm} R{1cm} R{1cm}}
      \toprule
      \multicolumn{1}{c}{} & \multicolumn{1}{c}{ 1992 } & \multicolumn{1}{c}{ 2002 } & \multicolumn{1}{c}{ 2014 } \\
      \midrule
	\multicolumn{4}{l}{\textcolor{FAOblue}{\textbf{\large{The setting}}}} \\ 
	 ~ Population, total (mln) & 154.6 ~ \ \ & 179.4 ~ \ \ & 202 ~ \ \ \\ 
	 ~ Population, rural (\% total population) & 38 ~ \ \ & 32.5 ~ \ \ & 29.4 ~ \ \ \\ 
	 ~ Govt expenditure on ag (\% total outlays) &  ~ \ \ &  ~ \ \ &  ~ \ \ \\ 
	 ~ Area harvested (mln ha) & 271 ~ \ \ & 364 ~ \ \ & 739 ~ \ \ \\ 
	 ~ Cropping intensity ratio (\%) & 1.1 ~ \ \ & 1.4 ~ \ \ &  ~ \ \ \\ 
	 ~ Water resources (m\textsuperscript{3}/person/year) & \textit{55} ~ \ \ & \textit{48} ~ \ \ & \textit{43} ~ \ \ \\ 
	 ~ Area equipped for irrigation (1000 ha) &  ~ \ \ &  ~ \ \ & \textit{5\,400} ~ \ \ \\ 
	 ~ Area irrigated (\%) &  ~ \ \ &  ~ \ \ & \textit{96.8} ~ \ \ \\ 
	 ~ Employment in agriculture (\%) & 28.3 ~ \ \ & 20.6 ~ \ \ & \textit{15.3} ~ \ \ \\ 
	 ~ Employment in agriculture, female (\%) & 24.7 ~ \ \ & 16.5 ~ \ \ & \textit{11} ~ \ \ \\ 
	 ~ Fertilizers, Nitrogen (nutrients per ha) &  ~ \ \ & 6.9 ~ \ \ & \textit{15.4} ~ \ \ \\ 
	 ~ Fertilizers, Phosphate (nutrients per ha) &  ~ \ \ & 9.9 ~ \ \ & \textit{15.8} ~ \ \ \\ 
	 ~ Fertilizers, Potash (nutrients per ha) &  ~ \ \ & 11.1 ~ \ \ & \textit{16.7} ~ \ \ \\ 
	 ~ Energy consump, power irrigation (mln kWh) & 44 ~ \ \ & 2\,565 ~ \ \ & \textit{6\,034} ~ \ \ \\ 
	 ~ Agr value added per worker (constant US\$) & 1.9 ~ \ \ & 3.1 ~ \ \ & \textit{5.6} ~ \ \ \\ 
	\multicolumn{4}{l}{\textcolor{FAOblue}{\textbf{\large{Hunger dimensions}}}} \\ 
	 ~ Dietary energy supply (kcal/pc/day) & 2\,774 ~ \ \ & 2\,963 ~ \ \ & 3\,276 ~ \ \ \\ 
	 ~ Average dietary energy supply adequacy (\%) & 119 ~ \ \ & 125 ~ \ \ & 134 ~ \ \ \\ 
	 ~ Dietary en supp, cereals/roots/tubers (\%) & 38 ~ \ \ & 36 ~ \ \ & \textit{33} ~ \ \ \\ 
	 ~ Prevalence of undernourishment (\%) & 14.3 ~ \ \ & 9.6 ~ \ \ & <5.0 ~ \ \ \\ 
	 ~ GDP per capita (US\$, PPP) & 9\,777 ~ \ \ & 11\,144 ~ \ \ & \textit{14\,555} ~ \ \ \\ 
	 ~ Domestic food price volatility (index) &  ~ \ \ & 6.2 ~ \ \ & 4.4 ~ \ \ \\ 
	 ~ Cereal import dependency ratio (\%) & 16.1 ~ \ \ & 8.4 ~ \ \ & \textit{-3} ~ \ \ \\ 
	 ~ Underweight, children under-5 (\%) &  ~ \ \ & 3.7 ~ \ \ & \textit{2.2} ~ \ \ \\ 
	 ~ Improved water source (\% pop) & 89.6 ~ \ \ & 94.2 ~ \ \ & \textit{97.5} ~ \ \ \\ 
	\multicolumn{4}{l}{\textcolor{FAOblue}{\textbf{\large{Food Supply}}}} \\ 
	 ~ Food production value, (2004-2006 mln I\$) & 58\,126 ~ \ \ & 88\,652 ~ \ \ & \textit{140\,046} ~ \ \ \\ 
	 ~ Agriculture, value added (\% GDP) & 8 ~ \ \ & 7 ~ \ \ & \textit{6} ~ \ \ \\ 
	 ~ Food exports (mln US\$)  & 4\,890 ~ \ \ & 11\,605 ~ \ \ & \textit{59\,994} ~ \ \ \\ 
	 ~ Food imports (mln US\$)  & 1\,850 ~ \ \ & 2\,634 ~ \ \ & \textit{8\,276} ~ \ \ \\ 
	\multicolumn{4}{l}{\textit{\normalsize{Production indices (2004-06=100)}}} \\ 
	 ~ Net food & 57 ~ \ \ & 87 ~ \ \ & \textit{137} ~ \ \ \\ 
	 ~ Net crop & 64 ~ \ \ & 88 ~ \ \ & \textit{140} ~ \ \ \\ 
	 ~ Cereal & 74 ~ \ \ & 84 ~ \ \ & \textit{157} ~ \ \ \\ 
	 ~ Vegetable oils & 39 ~ \ \ & 83 ~ \ \ & \textit{157} ~ \ \ \\ 
	 ~ Roots and tubers & 85 ~ \ \ & 92 ~ \ \ & \textit{90} ~ \ \ \\ 
	 ~ Fruit and vegetables & 81 ~ \ \ & 97 ~ \ \ & \textit{108} ~ \ \ \\ 
	 ~ Sugar & 62 ~ \ \ & 83 ~ \ \ & \textit{175} ~ \ \ \\ 
	 ~ Livestock & 52 ~ \ \ & 86 ~ \ \ & \textit{128} ~ \ \ \\ 
	 ~ Milk & 65 ~ \ \ & 88 ~ \ \ & \textit{135} ~ \ \ \\ 
	 ~ Meat & 48 ~ \ \ & 85 ~ \ \ & \textit{125} ~ \ \ \\ 
	 ~ Fish  & 65 ~ \ \ & 98 ~ \ \ & \textit{121} ~ \ \ \\ 
	\multicolumn{4}{l}{\textit{\normalsize{Net trade (min US\$)}}} \\ 
	 ~ Cereals & -1\,015 ~ \ \ & -978 ~ \ \ & \textit{3\,439} ~ \ \ \\ 
	 ~ Fruit and vegetables & 1\,136 ~ \ \ & 1\,140 ~ \ \ & \textit{1\,491} ~ \ \ \\ 
	 ~ Meat & 1\,089 ~ \ \ & 3\,050 ~ \ \ & \textit{14\,937} ~ \ \ \\ 
	 ~ Dairy products & -67 ~ \ \ & -207 ~ \ \ & \textit{-536} ~ \ \ \\ 
	 ~ Fish &  ~ \ \ &  ~ \ \ &  ~ \ \ \\ 
	\multicolumn{4}{l}{\textcolor{FAOblue}{\textbf{\large{Environment}}}} \\ 
	 ~ Forest area (\%) & 68 ~ \ \ & 65 ~ \ \ & \textit{62} ~ \ \ \\ 
	 ~ Renewable water res withdrawn (\% of total) &  ~ \ \ &  ~ \ \ & 60 ~ \ \ \\ 
	 ~ Terrestrial protect areas (\% total land area)  & 10 ~ \ \ & 20 ~ \ \ & \textit{26} ~ \ \ \\ 
	 ~ Organic area (\% total agricultural area) &  ~ \ \ & \textit{0} ~ \ \ & \textit{0} ~ \ \ \\ 
	 ~ Water withdrawal by agriculture (\% of total) &  ~ \ \ &  ~ \ \ & 60 ~ \ \ \\ 
	 ~ Biofuel production (thousand kt of oil eq.) & 1\,132 ~ \ \ & 1\,422 ~ \ \ & \textit{2\,238} ~ \ \ \\ 
	 ~ Wood pellet prod. (min tonnes) &  ~ \ \ &  ~ \ \ & \textit{62} ~ \ \ \\ 
	 ~ GHG emissions from ag (Co2 eq, gigagrams) & 1\,356 ~ \ \ & 1\,571 ~ \ \ & \textit{1\,255} ~ \ \ \\ 
       \toprule
      \end{tabular}
      \clearpage
\CountryData{ Brunei Darussalam }
      \rowcolors{1}{FAOblue!10}{white}
      \begin{tabular}{L{3.9cm} R{1cm} R{1cm} R{1cm}}
      \toprule
      \multicolumn{1}{c}{} & \multicolumn{1}{c}{ 1992 } & \multicolumn{1}{c}{ 2002 } & \multicolumn{1}{c}{ 2014 } \\
      \midrule
	\multicolumn{4}{l}{\textcolor{FAOblue}{\textbf{\large{The setting}}}} \\ 
	 ~ Population, total (mln) & 0.3 ~ \ \ & 0.3 ~ \ \ & 0.4 ~ \ \ \\ 
	 ~ Population, rural (\% total population) & 0.1 ~ \ \ & 0.1 ~ \ \ & 0.1 ~ \ \ \\ 
	 ~ Govt expenditure on ag (\% total outlays) &  ~ \ \ & 0.5 ~ \ \ & \textit{0.6} ~ \ \ \\ 
	 ~ Area harvested (mln ha) & 0 ~ \ \ & 0 ~ \ \ & 0 ~ \ \ \\ 
	 ~ Cropping intensity ratio (\%) & 18.4 ~ \ \ & 18.1 ~ \ \ &  ~ \ \ \\ 
	 ~ Water resources (m\textsuperscript{3}/person/year) & \textit{30} ~ \ \ & \textit{24} ~ \ \ & \textit{20} ~ \ \ \\ 
	 ~ Area equipped for irrigation (1000 ha) &  ~ \ \ &  ~ \ \ & \textit{1} ~ \ \ \\ 
	 ~ Area irrigated (\%) & \textit{63} ~ \ \ & \textit{63} ~ \ \ &  ~ \ \ \\ 
	 ~ Employment in agriculture (\%) & \textit{2} ~ \ \ & \textit{1.4} ~ \ \ &  ~ \ \ \\ 
	 ~ Employment in agriculture, female (\%) & \textit{1.6} ~ \ \ & \textit{0.3} ~ \ \ &  ~ \ \ \\ 
	 ~ Fertilizers, Nitrogen (nutrients per ha) &  ~ \ \ & 1.7 ~ \ \ & \textit{21.9} ~ \ \ \\ 
	 ~ Fertilizers, Phosphate (nutrients per ha) &  ~ \ \ & 1 ~ \ \ & \textit{6} ~ \ \ \\ 
	 ~ Fertilizers, Potash (nutrients per ha) &  ~ \ \ & 58.4 ~ \ \ & \textit{6} ~ \ \ \\ 
	 ~ Energy consump, power irrigation (mln kWh) & \textit{0} ~ \ \ & 0 ~ \ \ & \textit{0} ~ \ \ \\ 
	 ~ Agr value added per worker (constant US\$) & 21.3 ~ \ \ & 71.4 ~ \ \ & \textit{83.9} ~ \ \ \\ 
	\multicolumn{4}{l}{\textcolor{FAOblue}{\textbf{\large{Hunger dimensions}}}} \\ 
	 ~ Dietary energy supply (kcal/pc/day) & 2\,780 ~ \ \ & 2\,933 ~ \ \ & 3\,067 ~ \ \ \\ 
	 ~ Average dietary energy supply adequacy (\%) & 122 ~ \ \ & 126 ~ \ \ & 129 ~ \ \ \\ 
	 ~ Dietary en supp, cereals/roots/tubers (\%) & 50 ~ \ \ & 48 ~ \ \ & \textit{46} ~ \ \ \\ 
	 ~ Prevalence of undernourishment (\%) & <5.0 ~ \ \ & <5.0 ~ \ \ & <5.0 ~ \ \ \\ 
	 ~ GDP per capita (US\$, PPP) & 78\,688 ~ \ \ & 76\,130 ~ \ \ & \textit{69\,474} ~ \ \ \\ 
	 ~ Domestic food price volatility (index) &  ~ \ \ & 8.4 ~ \ \ & 4.7 ~ \ \ \\ 
	 ~ Cereal import dependency ratio (\%) & 98.7 ~ \ \ & 100 ~ \ \ & \textit{98.3} ~ \ \ \\ 
	 ~ Underweight, children under-5 (\%) &  ~ \ \ &  ~ \ \ &  ~ \ \ \\ 
	 ~ Improved water source (\% pop) &  ~ \ \ &  ~ \ \ &  ~ \ \ \\ 
	\multicolumn{4}{l}{\textcolor{FAOblue}{\textbf{\large{Food Supply}}}} \\ 
	 ~ Food production value, (2004-2006 mln I\$) & 12 ~ \ \ & 32 ~ \ \ & \textit{50} ~ \ \ \\ 
	 ~ Agriculture, value added (\% GDP) & 1 ~ \ \ & 1 ~ \ \ & \textit{1} ~ \ \ \\ 
	 ~ Food exports (mln US\$)  & 5 ~ \ \ & 1 ~ \ \ & \textit{2} ~ \ \ \\ 
	 ~ Food imports (mln US\$)  & 124 ~ \ \ & 163 ~ \ \ & \textit{327} ~ \ \ \\ 
	\multicolumn{4}{l}{\textit{\normalsize{Production indices (2004-06=100)}}} \\ 
	 ~ Net food & 41 ~ \ \ & 107 ~ \ \ & \textit{167} ~ \ \ \\ 
	 ~ Net crop & 55 ~ \ \ & 90 ~ \ \ & \textit{104} ~ \ \ \\ 
	 ~ Cereal & 97 ~ \ \ & 46 ~ \ \ & \textit{238} ~ \ \ \\ 
	 ~ Vegetable oils & 39 ~ \ \ & 60 ~ \ \ & \textit{87} ~ \ \ \\ 
	 ~ Roots and tubers & 61 ~ \ \ & 84 ~ \ \ & \textit{135} ~ \ \ \\ 
	 ~ Fruit and vegetables & 52 ~ \ \ & 92 ~ \ \ & \textit{97} ~ \ \ \\ 
	 ~ Sugar &  ~ \ \ &  ~ \ \ &  ~ \ \ \\ 
	 ~ Livestock & 38 ~ \ \ & 111 ~ \ \ & \textit{183} ~ \ \ \\ 
	 ~ Milk & 42 ~ \ \ & 141 ~ \ \ & \textit{69} ~ \ \ \\ 
	 ~ Meat & 33 ~ \ \ & 115 ~ \ \ & \textit{201} ~ \ \ \\ 
	 ~ Fish  & 60 ~ \ \ & 78 ~ \ \ & \textit{169} ~ \ \ \\ 
	\multicolumn{4}{l}{\textit{\normalsize{Net trade (min US\$)}}} \\ 
	 ~ Cereals & -32 ~ \ \ & -39 ~ \ \ & \textit{-127} ~ \ \ \\ 
	 ~ Fruit and vegetables & -32 ~ \ \ & -42 ~ \ \ & \textit{-47} ~ \ \ \\ 
	 ~ Meat & -21 ~ \ \ & -7 ~ \ \ & \textit{-33} ~ \ \ \\ 
	 ~ Dairy products & -10 ~ \ \ & -15 ~ \ \ & \textit{-19} ~ \ \ \\ 
	 ~ Fish &  ~ \ \ &  ~ \ \ &  ~ \ \ \\ 
	\multicolumn{4}{l}{\textcolor{FAOblue}{\textbf{\large{Environment}}}} \\ 
	 ~ Forest area (\%) & 78 ~ \ \ & 75 ~ \ \ & \textit{71} ~ \ \ \\ 
	 ~ Renewable water res withdrawn (\% of total) &  ~ \ \ &  ~ \ \ &  ~ \ \ \\ 
	 ~ Terrestrial protect areas (\% total land area)  & 44 ~ \ \ & 44 ~ \ \ & \textit{44} ~ \ \ \\ 
	 ~ Organic area (\% total agricultural area) &  ~ \ \ &  ~ \ \ &  ~ \ \ \\ 
	 ~ Water withdrawal by agriculture (\% of total) &  ~ \ \ &  ~ \ \ &  ~ \ \ \\ 
	 ~ Biofuel production (thousand kt of oil eq.) &  ~ \ \ &  ~ \ \ &  ~ \ \ \\ 
	 ~ Wood pellet prod. (min tonnes) &  ~ \ \ &  ~ \ \ &  ~ \ \ \\ 
	 ~ GHG emissions from ag (Co2 eq, gigagrams) & 2 ~ \ \ & 2 ~ \ \ & \textit{2} ~ \ \ \\ 
       \toprule
      \end{tabular}
      \clearpage
\CountryData{ Bulgaria }
      \rowcolors{1}{FAOblue!10}{white}
      \begin{tabular}{L{3.9cm} R{1cm} R{1cm} R{1cm}}
      \toprule
      \multicolumn{1}{c}{} & \multicolumn{1}{c}{ 1992 } & \multicolumn{1}{c}{ 2002 } & \multicolumn{1}{c}{ 2014 } \\
      \midrule
	\multicolumn{4}{l}{\textcolor{FAOblue}{\textbf{\large{The setting}}}} \\ 
	 ~ Population, total (mln) & 8.7 ~ \ \ & 7.9 ~ \ \ & 7.2 ~ \ \ \\ 
	 ~ Population, rural (\% total population) & 2.9 ~ \ \ & 2.4 ~ \ \ & 1.8 ~ \ \ \\ 
	 ~ Govt expenditure on ag (\% total outlays) &  ~ \ \ & 3.5 ~ \ \ & \textit{0.6} ~ \ \ \\ 
	 ~ Area harvested (mln ha) & 7 ~ \ \ & 7 ~ \ \ & 8 ~ \ \ \\ 
	 ~ Cropping intensity ratio (\%) & 1.1 ~ \ \ & 1.3 ~ \ \ &  ~ \ \ \\ 
	 ~ Water resources (m\textsuperscript{3}/person/year) & \textit{2} ~ \ \ & \textit{3} ~ \ \ & \textit{3} ~ \ \ \\ 
	 ~ Area equipped for irrigation (1000 ha) &  ~ \ \ &  ~ \ \ & \textit{102} ~ \ \ \\ 
	 ~ Area irrigated (\%) &  ~ \ \ &  ~ \ \ & \textit{69.5} ~ \ \ \\ 
	 ~ Employment in agriculture (\%) & 21.2 ~ \ \ & 10.7 ~ \ \ & \textit{6.4} ~ \ \ \\ 
	 ~ Employment in agriculture, female (\%) &  ~ \ \ & 8.1 ~ \ \ & \textit{4.3} ~ \ \ \\ 
	 ~ Fertilizers, Nitrogen (nutrients per ha) &  ~ \ \ & 53.2 ~ \ \ & \textit{62} ~ \ \ \\ 
	 ~ Fertilizers, Phosphate (nutrients per ha) &  ~ \ \ & 17 ~ \ \ & \textit{13.5} ~ \ \ \\ 
	 ~ Fertilizers, Potash (nutrients per ha) &  ~ \ \ & 1.4 ~ \ \ & \textit{3.3} ~ \ \ \\ 
	 ~ Energy consump, power irrigation (mln kWh) &  ~ \ \ &  ~ \ \ & \textit{46} ~ \ \ \\ 
	 ~ Agr value added per worker (constant US\$) & 6.3 ~ \ \ & 11 ~ \ \ & \textit{16.6} ~ \ \ \\ 
	\multicolumn{4}{l}{\textcolor{FAOblue}{\textbf{\large{Hunger dimensions}}}} \\ 
	 ~ Dietary energy supply (kcal/pc/day) &  ~ \ \ &  ~ \ \ &  ~ \ \ \\ 
	 ~ Average dietary energy supply adequacy (\%) & 121 ~ \ \ & 110 ~ \ \ & 116 ~ \ \ \\ 
	 ~ Dietary en supp, cereals/roots/tubers (\%) & 42 ~ \ \ & 42 ~ \ \ & \textit{43} ~ \ \ \\ 
	 ~ Prevalence of undernourishment (\%) & <5.0 ~ \ \ & <5.0 ~ \ \ & <5.0 ~ \ \ \\ 
	 ~ GDP per capita (US\$, PPP) & 8\,089 ~ \ \ & 10\,220 ~ \ \ & \textit{15\,695} ~ \ \ \\ 
	 ~ Domestic food price volatility (index) &  ~ \ \ & 13.5 ~ \ \ & 5.9 ~ \ \ \\ 
	 ~ Cereal import dependency ratio (\%) & -0.1 ~ \ \ & -20.7 ~ \ \ & \textit{-92.2} ~ \ \ \\ 
	 ~ Underweight, children under-5 (\%) &  ~ \ \ & \textit{1.6} ~ \ \ &  ~ \ \ \\ 
	 ~ Improved water source (\% pop) & 99.9 ~ \ \ & 99.7 ~ \ \ & \textit{99.5} ~ \ \ \\ 
	\multicolumn{4}{l}{\textcolor{FAOblue}{\textbf{\large{Food Supply}}}} \\ 
	 ~ Food production value, (2004-2006 mln I\$) & 3\,853 ~ \ \ & 2\,712 ~ \ \ & \textit{3\,143} ~ \ \ \\ 
	 ~ Agriculture, value added (\% GDP) & 13 ~ \ \ & 11 ~ \ \ & \textit{5} ~ \ \ \\ 
	 ~ Food exports (mln US\$)  & 483 ~ \ \ & 537 ~ \ \ & \textit{3\,224} ~ \ \ \\ 
	 ~ Food imports (mln US\$)  & 181 ~ \ \ & 340 ~ \ \ & \textit{2\,108} ~ \ \ \\ 
	\multicolumn{4}{l}{\textit{\normalsize{Production indices (2004-06=100)}}} \\ 
	 ~ Net food & 148 ~ \ \ & 104 ~ \ \ & \textit{120} ~ \ \ \\ 
	 ~ Net crop & 127 ~ \ \ & 106 ~ \ \ & \textit{129} ~ \ \ \\ 
	 ~ Cereal & 103 ~ \ \ & 108 ~ \ \ & \textit{136} ~ \ \ \\ 
	 ~ Vegetable oils & 56 ~ \ \ & 60 ~ \ \ & \textit{205} ~ \ \ \\ 
	 ~ Roots and tubers & 124 ~ \ \ & 142 ~ \ \ & \textit{35} ~ \ \ \\ 
	 ~ Fruit and vegetables & 244 ~ \ \ & 124 ~ \ \ & \textit{99} ~ \ \ \\ 
	 ~ Sugar & 1\,170 ~ \ \ & 198 ~ \ \ & \textit{63} ~ \ \ \\ 
	 ~ Livestock & 197 ~ \ \ & 107 ~ \ \ & \textit{91} ~ \ \ \\ 
	 ~ Milk & 122 ~ \ \ & 98 ~ \ \ & \textit{85} ~ \ \ \\ 
	 ~ Meat & 317 ~ \ \ & 124 ~ \ \ & \textit{100} ~ \ \ \\ 
	 ~ Fish  & 320 ~ \ \ & 172 ~ \ \ & \textit{217} ~ \ \ \\ 
	\multicolumn{4}{l}{\textit{\normalsize{Net trade (min US\$)}}} \\ 
	 ~ Cereals & 89 ~ \ \ & 162 ~ \ \ & \textit{1\,095} ~ \ \ \\ 
	 ~ Fruit and vegetables & 65 ~ \ \ & 21 ~ \ \ & \textit{-87} ~ \ \ \\ 
	 ~ Meat & 74 ~ \ \ & -2 ~ \ \ & \textit{-296} ~ \ \ \\ 
	 ~ Dairy products & 45 ~ \ \ & 17 ~ \ \ & \textit{-79} ~ \ \ \\ 
	 ~ Fish & 3 ~ \ \ & -10 ~ \ \ & \textit{-43} ~ \ \ \\ 
	\multicolumn{4}{l}{\textcolor{FAOblue}{\textbf{\large{Environment}}}} \\ 
	 ~ Forest area (\%) & 30 ~ \ \ & 32 ~ \ \ & \textit{37} ~ \ \ \\ 
	 ~ Renewable water res withdrawn (\% of total) &  ~ \ \ &  ~ \ \ & 16 ~ \ \ \\ 
	 ~ Terrestrial protect areas (\% total land area)  & 3 ~ \ \ & 9 ~ \ \ & \textit{37} ~ \ \ \\ 
	 ~ Organic area (\% total agricultural area) &  ~ \ \ & \textit{0} ~ \ \ & \textit{1} ~ \ \ \\ 
	 ~ Water withdrawal by agriculture (\% of total) &  ~ \ \ &  ~ \ \ & 16 ~ \ \ \\ 
	 ~ Biofuel production (thousand kt of oil eq.) &  ~ \ \ & 1 ~ \ \ & \textit{516} ~ \ \ \\ 
	 ~ Wood pellet prod. (min tonnes) &  ~ \ \ &  ~ \ \ & \textit{120} ~ \ \ \\ 
	 ~ GHG emissions from ag (Co2 eq, gigagrams) & -3 ~ \ \ & -8 ~ \ \ & \textit{-9} ~ \ \ \\ 
       \toprule
      \end{tabular}
      \clearpage
\CountryData{ Burkina Faso }
      \rowcolors{1}{FAOblue!10}{white}
      \begin{tabular}{L{3.9cm} R{1cm} R{1cm} R{1cm}}
      \toprule
      \multicolumn{1}{c}{} & \multicolumn{1}{c}{ 1992 } & \multicolumn{1}{c}{ 2002 } & \multicolumn{1}{c}{ 2014 } \\
      \midrule
	\multicolumn{4}{l}{\textcolor{FAOblue}{\textbf{\large{The setting}}}} \\ 
	 ~ Population, total (mln) & 9.3 ~ \ \ & 12.3 ~ \ \ & 17.4 ~ \ \ \\ 
	 ~ Population, rural (\% total population) & 8 ~ \ \ & 9.9 ~ \ \ & 12.4 ~ \ \ \\ 
	 ~ Govt expenditure on ag (\% total outlays) &  ~ \ \ &  ~ \ \ &  ~ \ \ \\ 
	 ~ Area harvested (mln ha) & 3 ~ \ \ & 3 ~ \ \ & 5 ~ \ \ \\ 
	 ~ Cropping intensity ratio (\%) & 0.3 ~ \ \ & 0.3 ~ \ \ &  ~ \ \ \\ 
	 ~ Water resources (m\textsuperscript{3}/person/year) & \textit{1} ~ \ \ & \textit{1} ~ \ \ & \textit{1} ~ \ \ \\ 
	 ~ Area equipped for irrigation (1000 ha) &  ~ \ \ &  ~ \ \ & \textit{55} ~ \ \ \\ 
	 ~ Area irrigated (\%) &  ~ \ \ &  ~ \ \ & \textit{85} ~ \ \ \\ 
	 ~ Employment in agriculture (\%) & \textit{88.8} ~ \ \ & \textit{84.8} ~ \ \ & \textit{84.8} ~ \ \ \\ 
	 ~ Employment in agriculture, female (\%) &  ~ \ \ & \textit{87.2} ~ \ \ & \textit{87.2} ~ \ \ \\ 
	 ~ Fertilizers, Nitrogen (nutrients per ha) &  ~ \ \ & 0.2 ~ \ \ & \textit{2.3} ~ \ \ \\ 
	 ~ Fertilizers, Phosphate (nutrients per ha) &  ~ \ \ & 0 ~ \ \ & \textit{1.5} ~ \ \ \\ 
	 ~ Fertilizers, Potash (nutrients per ha) &  ~ \ \ & 0 ~ \ \ & \textit{1.6} ~ \ \ \\ 
	 ~ Energy consump, power irrigation (mln kWh) & 9 ~ \ \ & 9 ~ \ \ & \textit{10} ~ \ \ \\ 
	 ~ Agr value added per worker (constant US\$) & 0.3 ~ \ \ & 0.2 ~ \ \ & \textit{0.2} ~ \ \ \\ 
	\multicolumn{4}{l}{\textcolor{FAOblue}{\textbf{\large{Hunger dimensions}}}} \\ 
	 ~ Dietary energy supply (kcal/pc/day) & 2\,280 ~ \ \ & 2\,413 ~ \ \ & 2\,707 ~ \ \ \\ 
	 ~ Average dietary energy supply adequacy (\%) & 106 ~ \ \ & 111 ~ \ \ & 123 ~ \ \ \\ 
	 ~ Dietary en supp, cereals/roots/tubers (\%) & 67 ~ \ \ & 64 ~ \ \ & \textit{65} ~ \ \ \\ 
	 ~ Prevalence of undernourishment (\%) & 24.5 ~ \ \ & 28 ~ \ \ & 20.7 ~ \ \ \\ 
	 ~ GDP per capita (US\$, PPP) & 878 ~ \ \ & 1\,134 ~ \ \ & \textit{1\,630} ~ \ \ \\ 
	 ~ Domestic food price volatility (index) &  ~ \ \ & 21.2 ~ \ \ & 11.8 ~ \ \ \\ 
	 ~ Cereal import dependency ratio (\%) & 6.9 ~ \ \ & 7.3 ~ \ \ & \textit{9.8} ~ \ \ \\ 
	 ~ Underweight, children under-5 (\%) & \textit{29.6} ~ \ \ & \textit{35.2} ~ \ \ & \textit{26.2} ~ \ \ \\ 
	 ~ Improved water source (\% pop) & 45.4 ~ \ \ & 63.6 ~ \ \ & \textit{81.7} ~ \ \ \\ 
	\multicolumn{4}{l}{\textcolor{FAOblue}{\textbf{\large{Food Supply}}}} \\ 
	 ~ Food production value, (2004-2006 mln I\$) & 1\,153 ~ \ \ & 1\,614 ~ \ \ & \textit{2\,246} ~ \ \ \\ 
	 ~ Agriculture, value added (\% GDP) & 30 ~ \ \ & 27 ~ \ \ & \textit{23} ~ \ \ \\ 
	 ~ Food exports (mln US\$)  & 21 ~ \ \ & 44 ~ \ \ & \textit{176} ~ \ \ \\ 
	 ~ Food imports (mln US\$)  & 79 ~ \ \ & 99 ~ \ \ & \textit{367} ~ \ \ \\ 
	\multicolumn{4}{l}{\textit{\normalsize{Production indices (2004-06=100)}}} \\ 
	 ~ Net food & 64 ~ \ \ & 89 ~ \ \ & \textit{124} ~ \ \ \\ 
	 ~ Net crop & 58 ~ \ \ & 88 ~ \ \ & \textit{142} ~ \ \ \\ 
	 ~ Cereal & 73 ~ \ \ & 91 ~ \ \ & \textit{143} ~ \ \ \\ 
	 ~ Vegetable oils & 52 ~ \ \ & 109 ~ \ \ & \textit{199} ~ \ \ \\ 
	 ~ Roots and tubers & 124 ~ \ \ & 59 ~ \ \ & \textit{223} ~ \ \ \\ 
	 ~ Fruit and vegetables & 70 ~ \ \ & 104 ~ \ \ & \textit{80} ~ \ \ \\ 
	 ~ Sugar & 89 ~ \ \ & 100 ~ \ \ & \textit{107} ~ \ \ \\ 
	 ~ Livestock & 58 ~ \ \ & 81 ~ \ \ & \textit{85} ~ \ \ \\ 
	 ~ Milk & 63 ~ \ \ & 82 ~ \ \ & \textit{123} ~ \ \ \\ 
	 ~ Meat & 56 ~ \ \ & 80 ~ \ \ & \textit{75} ~ \ \ \\ 
	 ~ Fish  & 81 ~ \ \ & 92 ~ \ \ & \textit{224} ~ \ \ \\ 
	\multicolumn{4}{l}{\textit{\normalsize{Net trade (min US\$)}}} \\ 
	 ~ Cereals & -51 ~ \ \ & -55 ~ \ \ & \textit{-207} ~ \ \ \\ 
	 ~ Fruit and vegetables & -10 ~ \ \ & -4 ~ \ \ & \textit{34} ~ \ \ \\ 
	 ~ Meat & 0 ~ \ \ & 0 ~ \ \ & \textit{-1} ~ \ \ \\ 
	 ~ Dairy products & -13 ~ \ \ & -4 ~ \ \ & \textit{-19} ~ \ \ \\ 
	 ~ Fish & -4 ~ \ \ & -1 ~ \ \ & \textit{-10} ~ \ \ \\ 
	\multicolumn{4}{l}{\textcolor{FAOblue}{\textbf{\large{Environment}}}} \\ 
	 ~ Forest area (\%) & 25 ~ \ \ & 22 ~ \ \ & \textit{20} ~ \ \ \\ 
	 ~ Renewable water res withdrawn (\% of total) &  ~ \ \ & \textit{51} ~ \ \ & 51 ~ \ \ \\ 
	 ~ Terrestrial protect areas (\% total land area)  & 14 ~ \ \ & 14 ~ \ \ & \textit{15} ~ \ \ \\ 
	 ~ Organic area (\% total agricultural area) &  ~ \ \ & \textit{0} ~ \ \ & \textit{0} ~ \ \ \\ 
	 ~ Water withdrawal by agriculture (\% of total) &  ~ \ \ & \textit{51} ~ \ \ & 51 ~ \ \ \\ 
	 ~ Biofuel production (thousand kt of oil eq.) & 1 ~ \ \ & 1 ~ \ \ & \textit{1} ~ \ \ \\ 
	 ~ Wood pellet prod. (min tonnes) &  ~ \ \ &  ~ \ \ &  ~ \ \ \\ 
	 ~ GHG emissions from ag (Co2 eq, gigagrams) & 23 ~ \ \ & 26 ~ \ \ & \textit{31} ~ \ \ \\ 
       \toprule
      \end{tabular}
      \clearpage
\CountryData{ Burundi }
      \rowcolors{1}{FAOblue!10}{white}
      \begin{tabular}{L{3.9cm} R{1cm} R{1cm} R{1cm}}
      \toprule
      \multicolumn{1}{c}{} & \multicolumn{1}{c}{ 1992 } & \multicolumn{1}{c}{ 2002 } & \multicolumn{1}{c}{ 2014 } \\
      \midrule
	\multicolumn{4}{l}{\textcolor{FAOblue}{\textbf{\large{The setting}}}} \\ 
	 ~ Population, total (mln) & 5.9 ~ \ \ & 7 ~ \ \ & 10.5 ~ \ \ \\ 
	 ~ Population, rural (\% total population) & 5.5 ~ \ \ & 6.4 ~ \ \ & 9.2 ~ \ \ \\ 
	 ~ Govt expenditure on ag (\% total outlays) &  ~ \ \ &  ~ \ \ &  ~ \ \ \\ 
	 ~ Area harvested (mln ha) & 2 ~ \ \ & 2 ~ \ \ & 3 ~ \ \ \\ 
	 ~ Cropping intensity ratio (\%) & 0.8 ~ \ \ & 0.9 ~ \ \ &  ~ \ \ \\ 
	 ~ Water resources (m\textsuperscript{3}/person/year) & \textit{2} ~ \ \ & \textit{2} ~ \ \ & \textit{1} ~ \ \ \\ 
	 ~ Area equipped for irrigation (1000 ha) &  ~ \ \ &  ~ \ \ & \textit{23} ~ \ \ \\ 
	 ~ Area irrigated (\%) &  ~ \ \ &  ~ \ \ &  ~ \ \ \\ 
	 ~ Employment in agriculture (\%) &  ~ \ \ & \textit{92.2} ~ \ \ &  ~ \ \ \\ 
	 ~ Employment in agriculture, female (\%) &  ~ \ \ & \textit{96.6} ~ \ \ &  ~ \ \ \\ 
	 ~ Fertilizers, Nitrogen (nutrients per ha) &  ~ \ \ & 0.3 ~ \ \ & \textit{1.5} ~ \ \ \\ 
	 ~ Fertilizers, Phosphate (nutrients per ha) &  ~ \ \ & 0.4 ~ \ \ & \textit{1.4} ~ \ \ \\ 
	 ~ Fertilizers, Potash (nutrients per ha) &  ~ \ \ & 0 ~ \ \ & \textit{0.3} ~ \ \ \\ 
	 ~ Energy consump, power irrigation (mln kWh) &  ~ \ \ &  ~ \ \ &  ~ \ \ \\ 
	 ~ Agr value added per worker (constant US\$) & 0.2 ~ \ \ & 0.2 ~ \ \ & \textit{0.1} ~ \ \ \\ 
	\multicolumn{4}{l}{\textcolor{FAOblue}{\textbf{\large{Hunger dimensions}}}} \\ 
	 ~ Dietary energy supply (kcal/pc/day) &  ~ \ \ &  ~ \ \ &  ~ \ \ \\ 
	 ~ Average dietary energy supply adequacy (\%) &  ~ \ \ &  ~ \ \ &  ~ \ \ \\ 
	 ~ Dietary en supp, cereals/roots/tubers (\%) &  ~ \ \ &  ~ \ \ &  ~ \ \ \\ 
	 ~ Prevalence of undernourishment (\%) &  ~ \ \ &  ~ \ \ &  ~ \ \ \\ 
	 ~ GDP per capita (US\$, PPP) & 1\,062 ~ \ \ & 731 ~ \ \ & \textit{747} ~ \ \ \\ 
	 ~ Domestic food price volatility (index) &  ~ \ \ & 14.8 ~ \ \ & 8.3 ~ \ \ \\ 
	 ~ Cereal import dependency ratio (\%) & 11 ~ \ \ & 16.2 ~ \ \ & \textit{30.3} ~ \ \ \\ 
	 ~ Underweight, children under-5 (\%) &  ~ \ \ & \textit{35.2} ~ \ \ & \textit{29.1} ~ \ \ \\ 
	 ~ Improved water source (\% pop) & 69.4 ~ \ \ & 72.4 ~ \ \ & \textit{75.3} ~ \ \ \\ 
	\multicolumn{4}{l}{\textcolor{FAOblue}{\textbf{\large{Food Supply}}}} \\ 
	 ~ Food production value, (2004-2006 mln I\$) & 1\,015 ~ \ \ & 948 ~ \ \ & \textit{1\,368} ~ \ \ \\ 
	 ~ Agriculture, value added (\% GDP) & 54 ~ \ \ & 46 ~ \ \ & \textit{40} ~ \ \ \\ 
	 ~ Food exports (mln US\$)  & 4 ~ \ \ & 2 ~ \ \ & \textit{0} ~ \ \ \\ 
	 ~ Food imports (mln US\$)  & 20 ~ \ \ & 29 ~ \ \ & \textit{173} ~ \ \ \\ 
	\multicolumn{4}{l}{\textit{\normalsize{Production indices (2004-06=100)}}} \\ 
	 ~ Net food & 107 ~ \ \ & 100 ~ \ \ & \textit{145} ~ \ \ \\ 
	 ~ Net crop & 94 ~ \ \ & 106 ~ \ \ & \textit{129} ~ \ \ \\ 
	 ~ Cereal & 94 ~ \ \ & 98 ~ \ \ & \textit{80} ~ \ \ \\ 
	 ~ Vegetable oils & 77 ~ \ \ & 51 ~ \ \ & \textit{234} ~ \ \ \\ 
	 ~ Roots and tubers & 101 ~ \ \ & 110 ~ \ \ & \textit{230} ~ \ \ \\ 
	 ~ Fruit and vegetables & 98 ~ \ \ & 97 ~ \ \ & \textit{139} ~ \ \ \\ 
	 ~ Sugar & 124 ~ \ \ & 104 ~ \ \ & \textit{183} ~ \ \ \\ 
	 ~ Livestock & 111 ~ \ \ & 85 ~ \ \ & \textit{125} ~ \ \ \\ 
	 ~ Milk & 177 ~ \ \ & 105 ~ \ \ & \textit{229} ~ \ \ \\ 
	 ~ Meat & 99 ~ \ \ & 82 ~ \ \ & \textit{106} ~ \ \ \\ 
	 ~ Fish  & 179 ~ \ \ & 82 ~ \ \ & \textit{100} ~ \ \ \\ 
	\multicolumn{4}{l}{\textit{\normalsize{Net trade (min US\$)}}} \\ 
	 ~ Cereals & -12 ~ \ \ & -12 ~ \ \ & \textit{-84} ~ \ \ \\ 
	 ~ Fruit and vegetables & 0 ~ \ \ & -4 ~ \ \ & \textit{-5} ~ \ \ \\ 
	 ~ Meat &  ~ \ \ & 0 ~ \ \ & \textit{0} ~ \ \ \\ 
	 ~ Dairy products &  ~ \ \ & \textit{-1} ~ \ \ & \textit{-3} ~ \ \ \\ 
	 ~ Fish & 0 ~ \ \ & 0 ~ \ \ & \textit{-2} ~ \ \ \\ 
	\multicolumn{4}{l}{\textcolor{FAOblue}{\textbf{\large{Environment}}}} \\ 
	 ~ Forest area (\%) & 11 ~ \ \ & 7 ~ \ \ & \textit{7} ~ \ \ \\ 
	 ~ Renewable water res withdrawn (\% of total) &  ~ \ \ & \textit{77} ~ \ \ & 77 ~ \ \ \\ 
	 ~ Terrestrial protect areas (\% total land area)  & 4 ~ \ \ & 5 ~ \ \ & \textit{5} ~ \ \ \\ 
	 ~ Organic area (\% total agricultural area) &  ~ \ \ & \textit{0} ~ \ \ & \textit{0} ~ \ \ \\ 
	 ~ Water withdrawal by agriculture (\% of total) &  ~ \ \ & \textit{77} ~ \ \ & 77 ~ \ \ \\ 
	 ~ Biofuel production (thousand kt of oil eq.) & 0 ~ \ \ & 0 ~ \ \ & \textit{0} ~ \ \ \\ 
	 ~ Wood pellet prod. (min tonnes) &  ~ \ \ &  ~ \ \ &  ~ \ \ \\ 
	 ~ GHG emissions from ag (Co2 eq, gigagrams) & 7 ~ \ \ & 6 ~ \ \ & \textit{6} ~ \ \ \\ 
       \toprule
      \end{tabular}
      \clearpage
\CountryData{ Cabo Verde }
      \rowcolors{1}{FAOblue!10}{white}
      \begin{tabular}{L{3.9cm} R{1cm} R{1cm} R{1cm}}
      \toprule
      \multicolumn{1}{c}{} & \multicolumn{1}{c}{ 1992 } & \multicolumn{1}{c}{ 2002 } & \multicolumn{1}{c}{ 2014 } \\
      \midrule
	\multicolumn{4}{l}{\textcolor{FAOblue}{\textbf{\large{The setting}}}} \\ 
	 ~ Population, total (mln) & 0.4 ~ \ \ & 0.5 ~ \ \ & 0.5 ~ \ \ \\ 
	 ~ Population, rural (\% total population) & 0.2 ~ \ \ & 0.2 ~ \ \ & 0.2 ~ \ \ \\ 
	 ~ Govt expenditure on ag (\% total outlays) &  ~ \ \ & \textit{4.9} ~ \ \ & \textit{3.3} ~ \ \ \\ 
	 ~ Area harvested (mln ha) & 0 ~ \ \ & 0 ~ \ \ & 0 ~ \ \ \\ 
	 ~ Cropping intensity ratio (\%) & 5.3 ~ \ \ & 4.4 ~ \ \ &  ~ \ \ \\ 
	 ~ Water resources (m\textsuperscript{3}/person/year) & \textit{1} ~ \ \ & \textit{1} ~ \ \ & \textit{1} ~ \ \ \\ 
	 ~ Area equipped for irrigation (1000 ha) &  ~ \ \ &  ~ \ \ & \textit{4} ~ \ \ \\ 
	 ~ Area irrigated (\%) &  ~ \ \ & \textit{65.5} ~ \ \ &  ~ \ \ \\ 
	 ~ Employment in agriculture (\%) &  ~ \ \ &  ~ \ \ &  ~ \ \ \\ 
	 ~ Employment in agriculture, female (\%) &  ~ \ \ &  ~ \ \ &  ~ \ \ \\ 
	 ~ Fertilizers, Nitrogen (nutrients per ha) &  ~ \ \ &  ~ \ \ &  ~ \ \ \\ 
	 ~ Fertilizers, Phosphate (nutrients per ha) &  ~ \ \ &  ~ \ \ &  ~ \ \ \\ 
	 ~ Fertilizers, Potash (nutrients per ha) &  ~ \ \ &  ~ \ \ &  ~ \ \ \\ 
	 ~ Energy consump, power irrigation (mln kWh) &  ~ \ \ & 0 ~ \ \ & \textit{0} ~ \ \ \\ 
	 ~ Agr value added per worker (constant US\$) & 0.9 ~ \ \ & 2.4 ~ \ \ & \textit{2.6} ~ \ \ \\ 
	\multicolumn{4}{l}{\textcolor{FAOblue}{\textbf{\large{Hunger dimensions}}}} \\ 
	 ~ Dietary energy supply (kcal/pc/day) & 2\,410 ~ \ \ & 2\,371 ~ \ \ & 2\,821 ~ \ \ \\ 
	 ~ Average dietary energy supply adequacy (\%) & 111 ~ \ \ & 105 ~ \ \ & 118 ~ \ \ \\ 
	 ~ Dietary en supp, cereals/roots/tubers (\%) & 55 ~ \ \ & 50 ~ \ \ & \textit{46} ~ \ \ \\ 
	 ~ Prevalence of undernourishment (\%) & 15.8 ~ \ \ & 19.4 ~ \ \ & 10 ~ \ \ \\ 
	 ~ GDP per capita (US\$, PPP) & 1\,728 ~ \ \ & 3\,910 ~ \ \ & \textit{6\,210} ~ \ \ \\ 
	 ~ Domestic food price volatility (index) &  ~ \ \ & \textit{9.8} ~ \ \ & 5.4 ~ \ \ \\ 
	 ~ Cereal import dependency ratio (\%) & 90.2 ~ \ \ & 87.4 ~ \ \ & \textit{94} ~ \ \ \\ 
	 ~ Underweight, children under-5 (\%) & \textit{11.8} ~ \ \ &  ~ \ \ &  ~ \ \ \\ 
	 ~ Improved water source (\% pop) & 80.4 ~ \ \ & 83.8 ~ \ \ & \textit{89.3} ~ \ \ \\ 
	\multicolumn{4}{l}{\textcolor{FAOblue}{\textbf{\large{Food Supply}}}} \\ 
	 ~ Food production value, (2004-2006 mln I\$) & 27 ~ \ \ & 41 ~ \ \ & \textit{39} ~ \ \ \\ 
	 ~ Agriculture, value added (\% GDP) & 12 ~ \ \ & 10 ~ \ \ & \textit{8} ~ \ \ \\ 
	 ~ Food exports (mln US\$)  & 1 ~ \ \ & 0 ~ \ \ & \textit{0} ~ \ \ \\ 
	 ~ Food imports (mln US\$)  & 63 ~ \ \ & 58 ~ \ \ & \textit{184} ~ \ \ \\ 
	\multicolumn{4}{l}{\textit{\normalsize{Production indices (2004-06=100)}}} \\ 
	 ~ Net food & 62 ~ \ \ & 93 ~ \ \ & \textit{89} ~ \ \ \\ 
	 ~ Net crop & 49 ~ \ \ & 98 ~ \ \ & \textit{100} ~ \ \ \\ 
	 ~ Cereal & 173 ~ \ \ & 86 ~ \ \ & \textit{98} ~ \ \ \\ 
	 ~ Vegetable oils & 79 ~ \ \ & 98 ~ \ \ & \textit{129} ~ \ \ \\ 
	 ~ Roots and tubers & 49 ~ \ \ & 98 ~ \ \ & \textit{147} ~ \ \ \\ 
	 ~ Fruit and vegetables & 41 ~ \ \ & 98 ~ \ \ & \textit{94} ~ \ \ \\ 
	 ~ Sugar & 66 ~ \ \ & 98 ~ \ \ & \textit{104} ~ \ \ \\ 
	 ~ Livestock & 82 ~ \ \ & 87 ~ \ \ & \textit{77} ~ \ \ \\ 
	 ~ Milk & 47 ~ \ \ & 67 ~ \ \ & \textit{76} ~ \ \ \\ 
	 ~ Meat & 94 ~ \ \ & 92 ~ \ \ & \textit{72} ~ \ \ \\ 
	 ~ Fish  &  ~ \ \ &  ~ \ \ &  ~ \ \ \\ 
	\multicolumn{4}{l}{\textit{\normalsize{Net trade (min US\$)}}} \\ 
	 ~ Cereals & -25 ~ \ \ & -15 ~ \ \ & \textit{-50} ~ \ \ \\ 
	 ~ Fruit and vegetables & -9 ~ \ \ & -11 ~ \ \ & \textit{-29} ~ \ \ \\ 
	 ~ Meat & -3 ~ \ \ & -5 ~ \ \ & \textit{-26} ~ \ \ \\ 
	 ~ Dairy products & -6 ~ \ \ & -10 ~ \ \ & \textit{-32} ~ \ \ \\ 
	 ~ Fish & 3 ~ \ \ & 0 ~ \ \ & \textit{40} ~ \ \ \\ 
	\multicolumn{4}{l}{\textcolor{FAOblue}{\textbf{\large{Environment}}}} \\ 
	 ~ Forest area (\%) & 16 ~ \ \ & 21 ~ \ \ & \textit{21} ~ \ \ \\ 
	 ~ Renewable water res withdrawn (\% of total) &  ~ \ \ & \textit{91} ~ \ \ & 91 ~ \ \ \\ 
	 ~ Terrestrial protect areas (\% total land area)  & 2 ~ \ \ & 2 ~ \ \ & \textit{2} ~ \ \ \\ 
	 ~ Organic area (\% total agricultural area) &  ~ \ \ &  ~ \ \ &  ~ \ \ \\ 
	 ~ Water withdrawal by agriculture (\% of total) &  ~ \ \ & \textit{91} ~ \ \ & 91 ~ \ \ \\ 
	 ~ Biofuel production (thousand kt of oil eq.) &  ~ \ \ &  ~ \ \ &  ~ \ \ \\ 
	 ~ Wood pellet prod. (min tonnes) &  ~ \ \ &  ~ \ \ &  ~ \ \ \\ 
	 ~ GHG emissions from ag (Co2 eq, gigagrams) & 0 ~ \ \ & 0 ~ \ \ & \textit{0} ~ \ \ \\ 
       \toprule
      \end{tabular}
      \clearpage
\CountryData{ Cambodia }
      \rowcolors{1}{FAOblue!10}{white}
      \begin{tabular}{L{3.9cm} R{1cm} R{1cm} R{1cm}}
      \toprule
      \multicolumn{1}{c}{} & \multicolumn{1}{c}{ 1992 } & \multicolumn{1}{c}{ 2002 } & \multicolumn{1}{c}{ 2014 } \\
      \midrule
	\multicolumn{4}{l}{\textcolor{FAOblue}{\textbf{\large{The setting}}}} \\ 
	 ~ Population, total (mln) & 9.7 ~ \ \ & 12.7 ~ \ \ & 15.4 ~ \ \ \\ 
	 ~ Population, rural (\% total population) & 8.1 ~ \ \ & 10.3 ~ \ \ & 12.2 ~ \ \ \\ 
	 ~ Govt expenditure on ag (\% total outlays) &  ~ \ \ & 1.3 ~ \ \ & \textit{0.8} ~ \ \ \\ 
	 ~ Area harvested (mln ha) & 2 ~ \ \ & 4 ~ \ \ & 10 ~ \ \ \\ 
	 ~ Cropping intensity ratio (\%) & 0.5 ~ \ \ & 0.8 ~ \ \ &  ~ \ \ \\ 
	 ~ Water resources (m\textsuperscript{3}/person/year) & \textit{47} ~ \ \ & \textit{37} ~ \ \ & \textit{31} ~ \ \ \\ 
	 ~ Area equipped for irrigation (1000 ha) &  ~ \ \ &  ~ \ \ & \textit{354} ~ \ \ \\ 
	 ~ Area irrigated (\%) &  ~ \ \ &  ~ \ \ & \textit{89.7} ~ \ \ \\ 
	 ~ Employment in agriculture (\%) &  ~ \ \ & \textit{70.2} ~ \ \ & \textit{51} ~ \ \ \\ 
	 ~ Employment in agriculture, female (\%) &  ~ \ \ & \textit{69.7} ~ \ \ & \textit{52.8} ~ \ \ \\ 
	 ~ Fertilizers, Nitrogen (nutrients per ha) &  ~ \ \ & 1.6 ~ \ \ & \textit{7.6} ~ \ \ \\ 
	 ~ Fertilizers, Phosphate (nutrients per ha) &  ~ \ \ & 2.6 ~ \ \ & \textit{3.6} ~ \ \ \\ 
	 ~ Fertilizers, Potash (nutrients per ha) &  ~ \ \ & 0.2 ~ \ \ & \textit{0.6} ~ \ \ \\ 
	 ~ Energy consump, power irrigation (mln kWh) & \textit{0} ~ \ \ & 0 ~ \ \ & \textit{0} ~ \ \ \\ 
	 ~ Agr value added per worker (constant US\$) & \textit{0.4} ~ \ \ & 0.4 ~ \ \ & \textit{0.5} ~ \ \ \\ 
	\multicolumn{4}{l}{\textcolor{FAOblue}{\textbf{\large{Hunger dimensions}}}} \\ 
	 ~ Dietary energy supply (kcal/pc/day) & 2\,024 ~ \ \ & 2\,233 ~ \ \ & 2\,488 ~ \ \ \\ 
	 ~ Average dietary energy supply adequacy (\%) & 100 ~ \ \ & 104 ~ \ \ & 112 ~ \ \ \\ 
	 ~ Dietary en supp, cereals/roots/tubers (\%) & 84 ~ \ \ & 75 ~ \ \ & \textit{72} ~ \ \ \\ 
	 ~ Prevalence of undernourishment (\%) & 29.7 ~ \ \ & 25.9 ~ \ \ & 15 ~ \ \ \\ 
	 ~ GDP per capita (US\$, PPP) & \textit{1\,091} ~ \ \ & 1\,516 ~ \ \ & \textit{2\,944} ~ \ \ \\ 
	 ~ Domestic food price volatility (index) &  ~ \ \ & 11.5 ~ \ \ & 4.7 ~ \ \ \\ 
	 ~ Cereal import dependency ratio (\%) & 3.7 ~ \ \ & 3 ~ \ \ & \textit{-1.4} ~ \ \ \\ 
	 ~ Underweight, children under-5 (\%) &  ~ \ \ & \textit{28.4} ~ \ \ & \textit{29} ~ \ \ \\ 
	 ~ Improved water source (\% pop) & 21.6 ~ \ \ & 46.4 ~ \ \ & \textit{71.3} ~ \ \ \\ 
	\multicolumn{4}{l}{\textcolor{FAOblue}{\textbf{\large{Food Supply}}}} \\ 
	 ~ Food production value, (2004-2006 mln I\$) & 1\,119 ~ \ \ & 1\,720 ~ \ \ & \textit{4\,292} ~ \ \ \\ 
	 ~ Agriculture, value added (\% GDP) & \textit{50} ~ \ \ & 33 ~ \ \ & \textit{34} ~ \ \ \\ 
	 ~ Food exports (mln US\$)  & 3 ~ \ \ & 4 ~ \ \ & \textit{177} ~ \ \ \\ 
	 ~ Food imports (mln US\$)  & 30 ~ \ \ & 108 ~ \ \ & \textit{399} ~ \ \ \\ 
	\multicolumn{4}{l}{\textit{\normalsize{Production indices (2004-06=100)}}} \\ 
	 ~ Net food & 47 ~ \ \ & 73 ~ \ \ & \textit{181} ~ \ \ \\ 
	 ~ Net crop & 44 ~ \ \ & 69 ~ \ \ & \textit{200} ~ \ \ \\ 
	 ~ Cereal & 40 ~ \ \ & 70 ~ \ \ & \textit{179} ~ \ \ \\ 
	 ~ Vegetable oils & 20 ~ \ \ & 26 ~ \ \ & \textit{66} ~ \ \ \\ 
	 ~ Roots and tubers & 20 ~ \ \ & 17 ~ \ \ & \textit{734} ~ \ \ \\ 
	 ~ Fruit and vegetables & 83 ~ \ \ & 99 ~ \ \ & \textit{123} ~ \ \ \\ 
	 ~ Sugar & 110 ~ \ \ & 166 ~ \ \ & \textit{476} ~ \ \ \\ 
	 ~ Livestock & 66 ~ \ \ & 89 ~ \ \ & \textit{90} ~ \ \ \\ 
	 ~ Milk & 82 ~ \ \ & 94 ~ \ \ & \textit{108} ~ \ \ \\ 
	 ~ Meat & 65 ~ \ \ & 89 ~ \ \ & \textit{88} ~ \ \ \\ 
	 ~ Fish  & 27 ~ \ \ & 101 ~ \ \ & \textit{175} ~ \ \ \\ 
	\multicolumn{4}{l}{\textit{\normalsize{Net trade (min US\$)}}} \\ 
	 ~ Cereals & -22 ~ \ \ & -20 ~ \ \ & \textit{55} ~ \ \ \\ 
	 ~ Fruit and vegetables & 0 ~ \ \ & -4 ~ \ \ & \textit{-2} ~ \ \ \\ 
	 ~ Meat & 0 ~ \ \ & 0 ~ \ \ & \textit{-2} ~ \ \ \\ 
	 ~ Dairy products & -3 ~ \ \ & -4 ~ \ \ & \textit{-9} ~ \ \ \\ 
	 ~ Fish & 16 ~ \ \ & 36 ~ \ \ & \textit{38} ~ \ \ \\ 
	\multicolumn{4}{l}{\textcolor{FAOblue}{\textbf{\large{Environment}}}} \\ 
	 ~ Forest area (\%) & 72 ~ \ \ & 64 ~ \ \ & \textit{56} ~ \ \ \\ 
	 ~ Renewable water res withdrawn (\% of total) &  ~ \ \ &  ~ \ \ & 94 ~ \ \ \\ 
	 ~ Terrestrial protect areas (\% total land area)  & 0 ~ \ \ & 23 ~ \ \ & \textit{26} ~ \ \ \\ 
	 ~ Organic area (\% total agricultural area) &  ~ \ \ & \textit{0} ~ \ \ & \textit{0} ~ \ \ \\ 
	 ~ Water withdrawal by agriculture (\% of total) &  ~ \ \ &  ~ \ \ & 94 ~ \ \ \\ 
	 ~ Biofuel production (thousand kt of oil eq.) &  ~ \ \ &  ~ \ \ &  ~ \ \ \\ 
	 ~ Wood pellet prod. (min tonnes) &  ~ \ \ &  ~ \ \ &  ~ \ \ \\ 
	 ~ GHG emissions from ag (Co2 eq, gigagrams) & 41 ~ \ \ & 46 ~ \ \ & \textit{42} ~ \ \ \\ 
       \toprule
      \end{tabular}
      \clearpage
\CountryData{ Cameroon }
      \rowcolors{1}{FAOblue!10}{white}
      \begin{tabular}{L{3.9cm} R{1cm} R{1cm} R{1cm}}
      \toprule
      \multicolumn{1}{c}{} & \multicolumn{1}{c}{ 1992 } & \multicolumn{1}{c}{ 2002 } & \multicolumn{1}{c}{ 2014 } \\
      \midrule
	\multicolumn{4}{l}{\textcolor{FAOblue}{\textbf{\large{The setting}}}} \\ 
	 ~ Population, total (mln) & 12.8 ~ \ \ & 16.8 ~ \ \ & 22.8 ~ \ \ \\ 
	 ~ Population, rural (\% total population) & 7.6 ~ \ \ & 8.9 ~ \ \ & 10.5 ~ \ \ \\ 
	 ~ Govt expenditure on ag (\% total outlays) &  ~ \ \ &  ~ \ \ &  ~ \ \ \\ 
	 ~ Area harvested (mln ha) & 2 ~ \ \ & 4 ~ \ \ & 7 ~ \ \ \\ 
	 ~ Cropping intensity ratio (\%) & 0.3 ~ \ \ & 0.4 ~ \ \ &  ~ \ \ \\ 
	 ~ Water resources (m\textsuperscript{3}/person/year) & \textit{21} ~ \ \ & \textit{16} ~ \ \ & \textit{13} ~ \ \ \\ 
	 ~ Area equipped for irrigation (1000 ha) &  ~ \ \ &  ~ \ \ & \textit{29} ~ \ \ \\ 
	 ~ Area irrigated (\%) &  ~ \ \ &  ~ \ \ &  ~ \ \ \\ 
	 ~ Employment in agriculture (\%) &  ~ \ \ & \textit{55.7} ~ \ \ & \textit{53.3} ~ \ \ \\ 
	 ~ Employment in agriculture, female (\%) &  ~ \ \ & \textit{64.7} ~ \ \ & \textit{58.1} ~ \ \ \\ 
	 ~ Fertilizers, Nitrogen (nutrients per ha) &  ~ \ \ & 2.3 ~ \ \ & \textit{2.9} ~ \ \ \\ 
	 ~ Fertilizers, Phosphate (nutrients per ha) &  ~ \ \ & 1.1 ~ \ \ & \textit{1.1} ~ \ \ \\ 
	 ~ Fertilizers, Potash (nutrients per ha) &  ~ \ \ & 3 ~ \ \ & \textit{2.4} ~ \ \ \\ 
	 ~ Energy consump, power irrigation (mln kWh) &  ~ \ \ & 13 ~ \ \ & \textit{13} ~ \ \ \\ 
	 ~ Agr value added per worker (constant US\$) & 0.5 ~ \ \ & 0.8 ~ \ \ & \textit{1.2} ~ \ \ \\ 
	\multicolumn{4}{l}{\textcolor{FAOblue}{\textbf{\large{Hunger dimensions}}}} \\ 
	 ~ Dietary energy supply (kcal/pc/day) & 2\,050 ~ \ \ & 2\,200 ~ \ \ & 2\,596 ~ \ \ \\ 
	 ~ Average dietary energy supply adequacy (\%) & 95 ~ \ \ & 100 ~ \ \ & 117 ~ \ \ \\ 
	 ~ Dietary en supp, cereals/roots/tubers (\%) & 58 ~ \ \ & 56 ~ \ \ & \textit{54} ~ \ \ \\ 
	 ~ Prevalence of undernourishment (\%) & 37.5 ~ \ \ & 29 ~ \ \ & 10.2 ~ \ \ \\ 
	 ~ GDP per capita (US\$, PPP) & 2\,434 ~ \ \ & 2\,480 ~ \ \ & \textit{2\,739} ~ \ \ \\ 
	 ~ Domestic food price volatility (index) &  ~ \ \ & 5.9 ~ \ \ & 10 ~ \ \ \\ 
	 ~ Cereal import dependency ratio (\%) & 27.9 ~ \ \ & 31.8 ~ \ \ & \textit{25.8} ~ \ \ \\ 
	 ~ Underweight, children under-5 (\%) & \textit{18} ~ \ \ & \textit{15.1} ~ \ \ & \textit{15.1} ~ \ \ \\ 
	 ~ Improved water source (\% pop) & 53.4 ~ \ \ & 63.8 ~ \ \ & \textit{74.1} ~ \ \ \\ 
	\multicolumn{4}{l}{\textcolor{FAOblue}{\textbf{\large{Food Supply}}}} \\ 
	 ~ Food production value, (2004-2006 mln I\$) & 1\,946 ~ \ \ & 2\,676 ~ \ \ & \textit{5\,279} ~ \ \ \\ 
	 ~ Agriculture, value added (\% GDP) & 27 ~ \ \ & 22 ~ \ \ & \textit{23} ~ \ \ \\ 
	 ~ Food exports (mln US\$)  & 162 ~ \ \ & 302 ~ \ \ & \textit{602} ~ \ \ \\ 
	 ~ Food imports (mln US\$)  & 179 ~ \ \ & 298 ~ \ \ & \textit{934} ~ \ \ \\ 
	\multicolumn{4}{l}{\textit{\normalsize{Production indices (2004-06=100)}}} \\ 
	 ~ Net food & 59 ~ \ \ & 81 ~ \ \ & \textit{160} ~ \ \ \\ 
	 ~ Net crop & 56 ~ \ \ & 80 ~ \ \ & \textit{161} ~ \ \ \\ 
	 ~ Cereal & 53 ~ \ \ & 77 ~ \ \ & \textit{163} ~ \ \ \\ 
	 ~ Vegetable oils & 51 ~ \ \ & 74 ~ \ \ & \textit{141} ~ \ \ \\ 
	 ~ Roots and tubers & 51 ~ \ \ & 85 ~ \ \ & \textit{157} ~ \ \ \\ 
	 ~ Fruit and vegetables & 58 ~ \ \ & 78 ~ \ \ & \textit{182} ~ \ \ \\ 
	 ~ Sugar & 100 ~ \ \ & 104 ~ \ \ & \textit{89} ~ \ \ \\ 
	 ~ Livestock & 74 ~ \ \ & 88 ~ \ \ & \textit{124} ~ \ \ \\ 
	 ~ Milk & 79 ~ \ \ & 86 ~ \ \ & \textit{114} ~ \ \ \\ 
	 ~ Meat & 73 ~ \ \ & 88 ~ \ \ & \textit{126} ~ \ \ \\ 
	 ~ Fish  & 53 ~ \ \ & 96 ~ \ \ & \textit{113} ~ \ \ \\ 
	\multicolumn{4}{l}{\textit{\normalsize{Net trade (min US\$)}}} \\ 
	 ~ Cereals & -115 ~ \ \ & -172 ~ \ \ & \textit{-642} ~ \ \ \\ 
	 ~ Fruit and vegetables & 39 ~ \ \ & 42 ~ \ \ & \textit{65} ~ \ \ \\ 
	 ~ Meat & -7 ~ \ \ & -16 ~ \ \ & \textit{-14} ~ \ \ \\ 
	 ~ Dairy products & -11 ~ \ \ & -20 ~ \ \ & \textit{-54} ~ \ \ \\ 
	 ~ Fish & -24 ~ \ \ & -24 ~ \ \ & \textit{-247} ~ \ \ \\ 
	\multicolumn{4}{l}{\textcolor{FAOblue}{\textbf{\large{Environment}}}} \\ 
	 ~ Forest area (\%) & 51 ~ \ \ & 46 ~ \ \ & \textit{41} ~ \ \ \\ 
	 ~ Renewable water res withdrawn (\% of total) &  ~ \ \ & \textit{76} ~ \ \ & 76 ~ \ \ \\ 
	 ~ Terrestrial protect areas (\% total land area)  & 7 ~ \ \ & 9 ~ \ \ & \textit{11} ~ \ \ \\ 
	 ~ Organic area (\% total agricultural area) &  ~ \ \ &  ~ \ \ & \textit{0} ~ \ \ \\ 
	 ~ Water withdrawal by agriculture (\% of total) &  ~ \ \ & \textit{76} ~ \ \ & 76 ~ \ \ \\ 
	 ~ Biofuel production (thousand kt of oil eq.) & 2 ~ \ \ & 5 ~ \ \ & \textit{7} ~ \ \ \\ 
	 ~ Wood pellet prod. (min tonnes) &  ~ \ \ &  ~ \ \ &  ~ \ \ \\ 
	 ~ GHG emissions from ag (Co2 eq, gigagrams) & 123 ~ \ \ & 123 ~ \ \ & \textit{121} ~ \ \ \\ 
       \toprule
      \end{tabular}
      \clearpage
\CountryData{ Canada }
      \rowcolors{1}{FAOblue!10}{white}
      \begin{tabular}{L{3.9cm} R{1cm} R{1cm} R{1cm}}
      \toprule
      \multicolumn{1}{c}{} & \multicolumn{1}{c}{ 1992 } & \multicolumn{1}{c}{ 2002 } & \multicolumn{1}{c}{ 2014 } \\
      \midrule
	\multicolumn{4}{l}{\textcolor{FAOblue}{\textbf{\large{The setting}}}} \\ 
	 ~ Population, total (mln) & 28.3 ~ \ \ & 31.3 ~ \ \ & 35.5 ~ \ \ \\ 
	 ~ Population, rural (\% total population) & 6.6 ~ \ \ & 6.3 ~ \ \ & 6.8 ~ \ \ \\ 
	 ~ Govt expenditure on ag (\% total outlays) &  ~ \ \ & 1.9 ~ \ \ & \textit{0.6} ~ \ \ \\ 
	 ~ Area harvested (mln ha) & 50 ~ \ \ & 36 ~ \ \ & 66 ~ \ \ \\ 
	 ~ Cropping intensity ratio (\%) & 0.7 ~ \ \ & 0.5 ~ \ \ &  ~ \ \ \\ 
	 ~ Water resources (m\textsuperscript{3}/person/year) & \textit{101} ~ \ \ & \textit{92} ~ \ \ & \textit{82} ~ \ \ \\ 
	 ~ Area equipped for irrigation (1000 ha) &  ~ \ \ &  ~ \ \ & \textit{870} ~ \ \ \\ 
	 ~ Area irrigated (\%) &  ~ \ \ &  ~ \ \ & \textit{69.5} ~ \ \ \\ 
	 ~ Employment in agriculture (\%) & 4.1 ~ \ \ & 2.8 ~ \ \ & \textit{2.4} ~ \ \ \\ 
	 ~ Employment in agriculture, female (\%) & 2.4 ~ \ \ & 1.6 ~ \ \ & \textit{1.3} ~ \ \ \\ 
	 ~ Fertilizers, Nitrogen (nutrients per ha) &  ~ \ \ & 24.3 ~ \ \ & \textit{35.8} ~ \ \ \\ 
	 ~ Fertilizers, Phosphate (nutrients per ha) &  ~ \ \ & 9.7 ~ \ \ & \textit{11.1} ~ \ \ \\ 
	 ~ Fertilizers, Potash (nutrients per ha) &  ~ \ \ & 5 ~ \ \ & \textit{5.4} ~ \ \ \\ 
	 ~ Energy consump, power irrigation (mln kWh) & 18 ~ \ \ & 18 ~ \ \ & \textit{1\,329} ~ \ \ \\ 
	 ~ Agr value added per worker (constant US\$) &  ~ \ \ &  ~ \ \ &  ~ \ \ \\ 
	\multicolumn{4}{l}{\textcolor{FAOblue}{\textbf{\large{Hunger dimensions}}}} \\ 
	 ~ Dietary energy supply (kcal/pc/day) &  ~ \ \ &  ~ \ \ &  ~ \ \ \\ 
	 ~ Average dietary energy supply adequacy (\%) & 124 ~ \ \ & 141 ~ \ \ & 145 ~ \ \ \\ 
	 ~ Dietary en supp, cereals/roots/tubers (\%) & 26 ~ \ \ & 28 ~ \ \ & \textit{28} ~ \ \ \\ 
	 ~ Prevalence of undernourishment (\%) & <5.0 ~ \ \ & <5.0 ~ \ \ & <5.0 ~ \ \ \\ 
	 ~ GDP per capita (US\$, PPP) & 29\,933 ~ \ \ & 38\,214 ~ \ \ & \textit{41\,899} ~ \ \ \\ 
	 ~ Domestic food price volatility (index) &  ~ \ \ & 5.9 ~ \ \ & 7.1 ~ \ \ \\ 
	 ~ Cereal import dependency ratio (\%) & -100.6 ~ \ \ & -45.9 ~ \ \ & \textit{-81} ~ \ \ \\ 
	 ~ Underweight, children under-5 (\%) &  ~ \ \ &  ~ \ \ &  ~ \ \ \\ 
	 ~ Improved water source (\% pop) & 99.8 ~ \ \ & 99.8 ~ \ \ & \textit{99.8} ~ \ \ \\ 
	\multicolumn{4}{l}{\textcolor{FAOblue}{\textbf{\large{Food Supply}}}} \\ 
	 ~ Food production value, (2004-2006 mln I\$) & 17\,233 ~ \ \ & 19\,412 ~ \ \ & \textit{27\,181} ~ \ \ \\ 
	 ~ Agriculture, value added (\% GDP) &  ~ \ \ &  ~ \ \ & \textit{2} ~ \ \ \\ 
	 ~ Food exports (mln US\$)  & 8\,740 ~ \ \ & 13\,427 ~ \ \ & \textit{37\,701} ~ \ \ \\ 
	 ~ Food imports (mln US\$)  & 5\,621 ~ \ \ & 9\,297 ~ \ \ & \textit{22\,561} ~ \ \ \\ 
	\multicolumn{4}{l}{\textit{\normalsize{Production indices (2004-06=100)}}} \\ 
	 ~ Net food & 73 ~ \ \ & 82 ~ \ \ & \textit{115} ~ \ \ \\ 
	 ~ Net crop & 76 ~ \ \ & 71 ~ \ \ & \textit{142} ~ \ \ \\ 
	 ~ Cereal & 100 ~ \ \ & 70 ~ \ \ & \textit{136} ~ \ \ \\ 
	 ~ Vegetable oils & 45 ~ \ \ & 58 ~ \ \ & \textit{188} ~ \ \ \\ 
	 ~ Roots and tubers & 73 ~ \ \ & 95 ~ \ \ & \textit{96} ~ \ \ \\ 
	 ~ Fruit and vegetables & 79 ~ \ \ & 94 ~ \ \ & \textit{99} ~ \ \ \\ 
	 ~ Sugar & 105 ~ \ \ & 47 ~ \ \ & \textit{81} ~ \ \ \\ 
	 ~ Livestock & 71 ~ \ \ & 98 ~ \ \ & \textit{85} ~ \ \ \\ 
	 ~ Milk & 96 ~ \ \ & 101 ~ \ \ & \textit{106} ~ \ \ \\ 
	 ~ Meat & 64 ~ \ \ & 98 ~ \ \ & \textit{79} ~ \ \ \\ 
	 ~ Fish  & 106 ~ \ \ & 97 ~ \ \ & \textit{80} ~ \ \ \\ 
	\multicolumn{4}{l}{\textit{\normalsize{Net trade (min US\$)}}} \\ 
	 ~ Cereals & 4\,092 ~ \ \ & 2\,253 ~ \ \ & \textit{7\,473} ~ \ \ \\ 
	 ~ Fruit and vegetables & -2\,036 ~ \ \ & -1\,804 ~ \ \ & \textit{-3\,751} ~ \ \ \\ 
	 ~ Meat & 160 ~ \ \ & 1\,817 ~ \ \ & \textit{1\,565} ~ \ \ \\ 
	 ~ Dairy products & 6 ~ \ \ & -47 ~ \ \ & \textit{-220} ~ \ \ \\ 
	 ~ Fish & 1\,399 ~ \ \ & 1\,691 ~ \ \ & \textit{1\,541} ~ \ \ \\ 
	\multicolumn{4}{l}{\textcolor{FAOblue}{\textbf{\large{Environment}}}} \\ 
	 ~ Forest area (\%) & 34 ~ \ \ & 34 ~ \ \ & \textit{34} ~ \ \ \\ 
	 ~ Renewable water res withdrawn (\% of total) &  ~ \ \ &  ~ \ \ & 5 ~ \ \ \\ 
	 ~ Terrestrial protect areas (\% total land area)  & 5 ~ \ \ & 6 ~ \ \ & \textit{9} ~ \ \ \\ 
	 ~ Organic area (\% total agricultural area) &  ~ \ \ & \textit{1} ~ \ \ & \textit{1} ~ \ \ \\ 
	 ~ Water withdrawal by agriculture (\% of total) &  ~ \ \ &  ~ \ \ & 5 ~ \ \ \\ 
	 ~ Biofuel production (thousand kt of oil eq.) &  ~ \ \ & 182 ~ \ \ & \textit{225} ~ \ \ \\ 
	 ~ Wood pellet prod. (min tonnes) &  ~ \ \ &  ~ \ \ & \textit{1\,800} ~ \ \ \\ 
	 ~ GHG emissions from ag (Co2 eq, gigagrams) & 88 ~ \ \ & 347 ~ \ \ & \textit{200} ~ \ \ \\ 
       \toprule
      \end{tabular}
      \clearpage
\CountryData{ Central African Republic }
      \rowcolors{1}{FAOblue!10}{white}
      \begin{tabular}{L{3.9cm} R{1cm} R{1cm} R{1cm}}
      \toprule
      \multicolumn{1}{c}{} & \multicolumn{1}{c}{ 1992 } & \multicolumn{1}{c}{ 2002 } & \multicolumn{1}{c}{ 2014 } \\
      \midrule
	\multicolumn{4}{l}{\textcolor{FAOblue}{\textbf{\large{The setting}}}} \\ 
	 ~ Population, total (mln) & 3.1 ~ \ \ & 3.8 ~ \ \ & 4.7 ~ \ \ \\ 
	 ~ Population, rural (\% total population) & 1.9 ~ \ \ & 2.3 ~ \ \ & 2.8 ~ \ \ \\ 
	 ~ Govt expenditure on ag (\% total outlays) &  ~ \ \ &  ~ \ \ &  ~ \ \ \\ 
	 ~ Area harvested (mln ha) & 1 ~ \ \ & 1 ~ \ \ & 1 ~ \ \ \\ 
	 ~ Cropping intensity ratio (\%) & 0.2 ~ \ \ & 0.2 ~ \ \ &  ~ \ \ \\ 
	 ~ Water resources (m\textsuperscript{3}/person/year) & \textit{45} ~ \ \ & \textit{37} ~ \ \ & \textit{31} ~ \ \ \\ 
	 ~ Area equipped for irrigation (1000 ha) &  ~ \ \ &  ~ \ \ & \textit{1} ~ \ \ \\ 
	 ~ Area irrigated (\%) &  ~ \ \ &  ~ \ \ &  ~ \ \ \\ 
	 ~ Employment in agriculture (\%) &  ~ \ \ &  ~ \ \ &  ~ \ \ \\ 
	 ~ Employment in agriculture, female (\%) &  ~ \ \ &  ~ \ \ &  ~ \ \ \\ 
	 ~ Fertilizers, Nitrogen (nutrients per ha) &  ~ \ \ &  ~ \ \ &  ~ \ \ \\ 
	 ~ Fertilizers, Phosphate (nutrients per ha) &  ~ \ \ &  ~ \ \ &  ~ \ \ \\ 
	 ~ Fertilizers, Potash (nutrients per ha) &  ~ \ \ &  ~ \ \ &  ~ \ \ \\ 
	 ~ Energy consump, power irrigation (mln kWh) &  ~ \ \ &  ~ \ \ &  ~ \ \ \\ 
	 ~ Agr value added per worker (constant US\$) & 0.4 ~ \ \ & 0.6 ~ \ \ & \textit{0.9} ~ \ \ \\ 
	\multicolumn{4}{l}{\textcolor{FAOblue}{\textbf{\large{Hunger dimensions}}}} \\ 
	 ~ Dietary energy supply (kcal/pc/day) & 1\,897 ~ \ \ & 1\,983 ~ \ \ & 1\,926 ~ \ \ \\ 
	 ~ Average dietary energy supply adequacy (\%) & 88 ~ \ \ & 91 ~ \ \ & 87 ~ \ \ \\ 
	 ~ Dietary en supp, cereals/roots/tubers (\%) & 58 ~ \ \ & 55 ~ \ \ & \textit{56} ~ \ \ \\ 
	 ~ Prevalence of undernourishment (\%) & 47.6 ~ \ \ & 42.5 ~ \ \ & 44.5 ~ \ \ \\ 
	 ~ GDP per capita (US\$, PPP) & 693 ~ \ \ & 733 ~ \ \ & \textit{584} ~ \ \ \\ 
	 ~ Domestic food price volatility (index) &  ~ \ \ &  ~ \ \ &  ~ \ \ \\ 
	 ~ Cereal import dependency ratio (\%) & 29.8 ~ \ \ & 18.2 ~ \ \ & \textit{21.4} ~ \ \ \\ 
	 ~ Underweight, children under-5 (\%) & \textit{20.4} ~ \ \ & \textit{21.8} ~ \ \ & \textit{23.5} ~ \ \ \\ 
	 ~ Improved water source (\% pop) & 58.8 ~ \ \ & 63.4 ~ \ \ & \textit{68.2} ~ \ \ \\ 
	\multicolumn{4}{l}{\textcolor{FAOblue}{\textbf{\large{Food Supply}}}} \\ 
	 ~ Food production value, (2004-2006 mln I\$) & 524 ~ \ \ & 759 ~ \ \ & \textit{966} ~ \ \ \\ 
	 ~ Agriculture, value added (\% GDP) & 47 ~ \ \ & 54 ~ \ \ & \textit{54} ~ \ \ \\ 
	 ~ Food exports (mln US\$)  & 18 ~ \ \ & 16 ~ \ \ & \textit{16} ~ \ \ \\ 
	 ~ Food imports (mln US\$)  & 27 ~ \ \ & 16 ~ \ \ & \textit{59} ~ \ \ \\ 
	\multicolumn{4}{l}{\textit{\normalsize{Production indices (2004-06=100)}}} \\ 
	 ~ Net food & 66 ~ \ \ & 95 ~ \ \ & \textit{121} ~ \ \ \\ 
	 ~ Net crop & 77 ~ \ \ & 101 ~ \ \ & \textit{122} ~ \ \ \\ 
	 ~ Cereal & 39 ~ \ \ & 84 ~ \ \ & \textit{119} ~ \ \ \\ 
	 ~ Vegetable oils & 63 ~ \ \ & 95 ~ \ \ & \textit{101} ~ \ \ \\ 
	 ~ Roots and tubers & 81 ~ \ \ & 98 ~ \ \ & \textit{127} ~ \ \ \\ 
	 ~ Fruit and vegetables & 81 ~ \ \ & 99 ~ \ \ & \textit{112} ~ \ \ \\ 
	 ~ Sugar & 30 ~ \ \ & 99 ~ \ \ & \textit{121} ~ \ \ \\ 
	 ~ Livestock & 60 ~ \ \ & 94 ~ \ \ & \textit{124} ~ \ \ \\ 
	 ~ Milk & 71 ~ \ \ & 97 ~ \ \ & \textit{123} ~ \ \ \\ 
	 ~ Meat & 59 ~ \ \ & 93 ~ \ \ & \textit{126} ~ \ \ \\ 
	 ~ Fish  & 53 ~ \ \ & 76 ~ \ \ & \textit{120} ~ \ \ \\ 
	\multicolumn{4}{l}{\textit{\normalsize{Net trade (min US\$)}}} \\ 
	 ~ Cereals & -10 ~ \ \ & -8 ~ \ \ & \textit{-23} ~ \ \ \\ 
	 ~ Fruit and vegetables & -2 ~ \ \ & 0 ~ \ \ & \textit{-1} ~ \ \ \\ 
	 ~ Meat & -2 ~ \ \ & 0 ~ \ \ & \textit{-1} ~ \ \ \\ 
	 ~ Dairy products & -3 ~ \ \ & -1 ~ \ \ & \textit{-3} ~ \ \ \\ 
	 ~ Fish & -1 ~ \ \ & 0 ~ \ \ & \textit{-3} ~ \ \ \\ 
	\multicolumn{4}{l}{\textcolor{FAOblue}{\textbf{\large{Environment}}}} \\ 
	 ~ Forest area (\%) & 37 ~ \ \ & 37 ~ \ \ & \textit{36} ~ \ \ \\ 
	 ~ Renewable water res withdrawn (\% of total) &  ~ \ \ & \textit{1} ~ \ \ & 1 ~ \ \ \\ 
	 ~ Terrestrial protect areas (\% total land area)  & 18 ~ \ \ & 18 ~ \ \ & \textit{18} ~ \ \ \\ 
	 ~ Organic area (\% total agricultural area) &  ~ \ \ &  ~ \ \ &  ~ \ \ \\ 
	 ~ Water withdrawal by agriculture (\% of total) &  ~ \ \ & \textit{1} ~ \ \ & 1 ~ \ \ \\ 
	 ~ Biofuel production (thousand kt of oil eq.) &  ~ \ \ &  ~ \ \ &  ~ \ \ \\ 
	 ~ Wood pellet prod. (min tonnes) &  ~ \ \ &  ~ \ \ &  ~ \ \ \\ 
	 ~ GHG emissions from ag (Co2 eq, gigagrams) & 33 ~ \ \ & 30 ~ \ \ & \textit{30} ~ \ \ \\ 
       \toprule
      \end{tabular}
      \clearpage
\CountryData{ Chad }
      \rowcolors{1}{FAOblue!10}{white}
      \begin{tabular}{L{3.9cm} R{1cm} R{1cm} R{1cm}}
      \toprule
      \multicolumn{1}{c}{} & \multicolumn{1}{c}{ 1992 } & \multicolumn{1}{c}{ 2002 } & \multicolumn{1}{c}{ 2014 } \\
      \midrule
	\multicolumn{4}{l}{\textcolor{FAOblue}{\textbf{\large{The setting}}}} \\ 
	 ~ Population, total (mln) & 6.3 ~ \ \ & 9 ~ \ \ & 13.2 ~ \ \ \\ 
	 ~ Population, rural (\% total population) & 5 ~ \ \ & 7 ~ \ \ & 10.3 ~ \ \ \\ 
	 ~ Govt expenditure on ag (\% total outlays) &  ~ \ \ &  ~ \ \ &  ~ \ \ \\ 
	 ~ Area harvested (mln ha) & 1 ~ \ \ & 2 ~ \ \ & 3 ~ \ \ \\ 
	 ~ Cropping intensity ratio (\%) & 0 ~ \ \ & 0 ~ \ \ &  ~ \ \ \\ 
	 ~ Water resources (m\textsuperscript{3}/person/year) & \textit{7} ~ \ \ & \textit{5} ~ \ \ & \textit{4} ~ \ \ \\ 
	 ~ Area equipped for irrigation (1000 ha) &  ~ \ \ &  ~ \ \ & \textit{30} ~ \ \ \\ 
	 ~ Area irrigated (\%) &  ~ \ \ & 86.6 ~ \ \ &  ~ \ \ \\ 
	 ~ Employment in agriculture (\%) & \textit{83} ~ \ \ &  ~ \ \ &  ~ \ \ \\ 
	 ~ Employment in agriculture, female (\%) & \textit{85.9} ~ \ \ &  ~ \ \ &  ~ \ \ \\ 
	 ~ Fertilizers, Nitrogen (nutrients per ha) &  ~ \ \ &  ~ \ \ &  ~ \ \ \\ 
	 ~ Fertilizers, Phosphate (nutrients per ha) &  ~ \ \ &  ~ \ \ &  ~ \ \ \\ 
	 ~ Fertilizers, Potash (nutrients per ha) &  ~ \ \ &  ~ \ \ &  ~ \ \ \\ 
	 ~ Energy consump, power irrigation (mln kWh) & 8 ~ \ \ & 9 ~ \ \ & \textit{9} ~ \ \ \\ 
	 ~ Agr value added per worker (constant US\$) &  ~ \ \ & \textit{1.3} ~ \ \ & \textit{1.3} ~ \ \ \\ 
	\multicolumn{4}{l}{\textcolor{FAOblue}{\textbf{\large{Hunger dimensions}}}} \\ 
	 ~ Dietary energy supply (kcal/pc/day) & 1\,755 ~ \ \ & 2\,003 ~ \ \ & 2\,171 ~ \ \ \\ 
	 ~ Average dietary energy supply adequacy (\%) & 82 ~ \ \ & 94 ~ \ \ & 101 ~ \ \ \\ 
	 ~ Dietary en supp, cereals/roots/tubers (\%) & 64 ~ \ \ & 59 ~ \ \ & \textit{69} ~ \ \ \\ 
	 ~ Prevalence of undernourishment (\%) & 56.4 ~ \ \ & 40.2 ~ \ \ & 36.1 ~ \ \ \\ 
	 ~ GDP per capita (US\$, PPP) & 1\,223 ~ \ \ & 1\,120 ~ \ \ & \textit{2\,022} ~ \ \ \\ 
	 ~ Domestic food price volatility (index) &  ~ \ \ & 19.2 ~ \ \ & \textit{11.7} ~ \ \ \\ 
	 ~ Cereal import dependency ratio (\%) & 8.7 ~ \ \ & 4.8 ~ \ \ & \textit{7.7} ~ \ \ \\ 
	 ~ Underweight, children under-5 (\%) &  ~ \ \ & \textit{33.9} ~ \ \ & \textit{30.3} ~ \ \ \\ 
	 ~ Improved water source (\% pop) & 40.8 ~ \ \ & 45.7 ~ \ \ & \textit{50.7} ~ \ \ \\ 
	\multicolumn{4}{l}{\textcolor{FAOblue}{\textbf{\large{Food Supply}}}} \\ 
	 ~ Food production value, (2004-2006 mln I\$) & 846 ~ \ \ & 1\,146 ~ \ \ & \textit{1\,531} ~ \ \ \\ 
	 ~ Agriculture, value added (\% GDP) & 35 ~ \ \ & 39 ~ \ \ & \textit{52} ~ \ \ \\ 
	 ~ Food exports (mln US\$)  & 28 ~ \ \ & 50 ~ \ \ & \textit{39} ~ \ \ \\ 
	 ~ Food imports (mln US\$)  & 19 ~ \ \ & 43 ~ \ \ & \textit{154} ~ \ \ \\ 
	\multicolumn{4}{l}{\textit{\normalsize{Production indices (2004-06=100)}}} \\ 
	 ~ Net food & 66 ~ \ \ & 90 ~ \ \ & \textit{120} ~ \ \ \\ 
	 ~ Net crop & 61 ~ \ \ & 89 ~ \ \ & \textit{116} ~ \ \ \\ 
	 ~ Cereal & 58 ~ \ \ & 72 ~ \ \ & \textit{159} ~ \ \ \\ 
	 ~ Vegetable oils & 50 ~ \ \ & 100 ~ \ \ & \textit{90} ~ \ \ \\ 
	 ~ Roots and tubers & 79 ~ \ \ & 96 ~ \ \ & \textit{134} ~ \ \ \\ 
	 ~ Fruit and vegetables & 95 ~ \ \ & 91 ~ \ \ & \textit{105} ~ \ \ \\ 
	 ~ Sugar & 79 ~ \ \ & 96 ~ \ \ & \textit{105} ~ \ \ \\ 
	 ~ Livestock & 79 ~ \ \ & 94 ~ \ \ & \textit{115} ~ \ \ \\ 
	 ~ Milk & 67 ~ \ \ & 95 ~ \ \ & \textit{114} ~ \ \ \\ 
	 ~ Meat & 82 ~ \ \ & 94 ~ \ \ & \textit{116} ~ \ \ \\ 
	 ~ Fish  &  ~ \ \ &  ~ \ \ &  ~ \ \ \\ 
	\multicolumn{4}{l}{\textit{\normalsize{Net trade (min US\$)}}} \\ 
	 ~ Cereals & -13 ~ \ \ & -12 ~ \ \ & \textit{-63} ~ \ \ \\ 
	 ~ Fruit and vegetables & 0 ~ \ \ & 0 ~ \ \ & \textit{-4} ~ \ \ \\ 
	 ~ Meat & 0 ~ \ \ & 0 ~ \ \ & \textit{-1} ~ \ \ \\ 
	 ~ Dairy products & -2 ~ \ \ & -1 ~ \ \ & \textit{-5} ~ \ \ \\ 
	 ~ Fish & \textit{-1} ~ \ \ & -1 ~ \ \ & \textit{-1} ~ \ \ \\ 
	\multicolumn{4}{l}{\textcolor{FAOblue}{\textbf{\large{Environment}}}} \\ 
	 ~ Forest area (\%) & 10 ~ \ \ & 10 ~ \ \ & \textit{9} ~ \ \ \\ 
	 ~ Renewable water res withdrawn (\% of total) &  ~ \ \ & \textit{76} ~ \ \ & 76 ~ \ \ \\ 
	 ~ Terrestrial protect areas (\% total land area)  & 9 ~ \ \ & 9 ~ \ \ & \textit{17} ~ \ \ \\ 
	 ~ Organic area (\% total agricultural area) &  ~ \ \ &  ~ \ \ &  ~ \ \ \\ 
	 ~ Water withdrawal by agriculture (\% of total) &  ~ \ \ & \textit{76} ~ \ \ & 76 ~ \ \ \\ 
	 ~ Biofuel production (thousand kt of oil eq.) & 1 ~ \ \ & 1 ~ \ \ & \textit{1} ~ \ \ \\ 
	 ~ Wood pellet prod. (min tonnes) &  ~ \ \ &  ~ \ \ &  ~ \ \ \\ 
	 ~ GHG emissions from ag (Co2 eq, gigagrams) & 33 ~ \ \ & 36 ~ \ \ & \textit{36} ~ \ \ \\ 
       \toprule
      \end{tabular}
      \clearpage
\CountryData{ Chile }
      \rowcolors{1}{FAOblue!10}{white}
      \begin{tabular}{L{3.9cm} R{1cm} R{1cm} R{1cm}}
      \toprule
      \multicolumn{1}{c}{} & \multicolumn{1}{c}{ 1992 } & \multicolumn{1}{c}{ 2002 } & \multicolumn{1}{c}{ 2014 } \\
      \midrule
	\multicolumn{4}{l}{\textcolor{FAOblue}{\textbf{\large{The setting}}}} \\ 
	 ~ Population, total (mln) & 13.7 ~ \ \ & 15.8 ~ \ \ & 17.8 ~ \ \ \\ 
	 ~ Population, rural (\% total population) & 2.3 ~ \ \ & 2.1 ~ \ \ & 1.8 ~ \ \ \\ 
	 ~ Govt expenditure on ag (\% total outlays) &  ~ \ \ &  ~ \ \ &  ~ \ \ \\ 
	 ~ Area harvested (mln ha) & 4 ~ \ \ & 4 ~ \ \ & 4 ~ \ \ \\ 
	 ~ Cropping intensity ratio (\%) & 0.2 ~ \ \ & 0.3 ~ \ \ &  ~ \ \ \\ 
	 ~ Water resources (m\textsuperscript{3}/person/year) & \textit{66} ~ \ \ & \textit{58} ~ \ \ & \textit{52} ~ \ \ \\ 
	 ~ Area equipped for irrigation (1000 ha) &  ~ \ \ &  ~ \ \ & \textit{1\,110} ~ \ \ \\ 
	 ~ Area irrigated (\%) &  ~ \ \ &  ~ \ \ & \textit{98.6} ~ \ \ \\ 
	 ~ Employment in agriculture (\%) & 18 ~ \ \ & 13.5 ~ \ \ & \textit{10.3} ~ \ \ \\ 
	 ~ Employment in agriculture, female (\%) & 6 ~ \ \ & 4.7 ~ \ \ & \textit{5.1} ~ \ \ \\ 
	 ~ Fertilizers, Nitrogen (nutrients per ha) &  ~ \ \ & 15.8 ~ \ \ & \textit{21.7} ~ \ \ \\ 
	 ~ Fertilizers, Phosphate (nutrients per ha) &  ~ \ \ & 10 ~ \ \ & \textit{7.7} ~ \ \ \\ 
	 ~ Fertilizers, Potash (nutrients per ha) &  ~ \ \ & 6.8 ~ \ \ & \textit{0.9} ~ \ \ \\ 
	 ~ Energy consump, power irrigation (mln kWh) &  ~ \ \ &  ~ \ \ & \textit{667} ~ \ \ \\ 
	 ~ Agr value added per worker (constant US\$) & 3.6 ~ \ \ & 4.7 ~ \ \ & \textit{6.7} ~ \ \ \\ 
	\multicolumn{4}{l}{\textcolor{FAOblue}{\textbf{\large{Hunger dimensions}}}} \\ 
	 ~ Dietary energy supply (kcal/pc/day) & 2\,674 ~ \ \ & 2\,887 ~ \ \ & 3\,047 ~ \ \ \\ 
	 ~ Average dietary energy supply adequacy (\%) & 114 ~ \ \ & 121 ~ \ \ & 125 ~ \ \ \\ 
	 ~ Dietary en supp, cereals/roots/tubers (\%) & 48 ~ \ \ & 45 ~ \ \ & \textit{44} ~ \ \ \\ 
	 ~ Prevalence of undernourishment (\%) & 8.1 ~ \ \ & <5.0 ~ \ \ & <5.0 ~ \ \ \\ 
	 ~ GDP per capita (US\$, PPP) & 10\,751 ~ \ \ & 15\,084 ~ \ \ & \textit{21\,714} ~ \ \ \\ 
	 ~ Domestic food price volatility (index) &  ~ \ \ & 4.7 ~ \ \ & 7.4 ~ \ \ \\ 
	 ~ Cereal import dependency ratio (\%) & 21.5 ~ \ \ & 30.5 ~ \ \ & \textit{38.8} ~ \ \ \\ 
	 ~ Underweight, children under-5 (\%) & \textit{0.8} ~ \ \ & 0.7 ~ \ \ & \textit{0.5} ~ \ \ \\ 
	 ~ Improved water source (\% pop) & 91.3 ~ \ \ & 95.6 ~ \ \ & \textit{98.8} ~ \ \ \\ 
	\multicolumn{4}{l}{\textcolor{FAOblue}{\textbf{\large{Food Supply}}}} \\ 
	 ~ Food production value, (2004-2006 mln I\$) & 4\,502 ~ \ \ & 6\,464 ~ \ \ & \textit{8\,424} ~ \ \ \\ 
	 ~ Agriculture, value added (\% GDP) & 10 ~ \ \ & 6 ~ \ \ & \textit{3} ~ \ \ \\ 
	 ~ Food exports (mln US\$)  & 1\,319 ~ \ \ & 2\,589 ~ \ \ & \textit{8\,004} ~ \ \ \\ 
	 ~ Food imports (mln US\$)  & 463 ~ \ \ & 869 ~ \ \ & \textit{4\,022} ~ \ \ \\ 
	\multicolumn{4}{l}{\textit{\normalsize{Production indices (2004-06=100)}}} \\ 
	 ~ Net food & 63 ~ \ \ & 90 ~ \ \ & \textit{117} ~ \ \ \\ 
	 ~ Net crop & 67 ~ \ \ & 91 ~ \ \ & \textit{117} ~ \ \ \\ 
	 ~ Cereal & 77 ~ \ \ & 88 ~ \ \ & \textit{102} ~ \ \ \\ 
	 ~ Vegetable oils & 117 ~ \ \ & 42 ~ \ \ & \textit{335} ~ \ \ \\ 
	 ~ Roots and tubers & 82 ~ \ \ & 108 ~ \ \ & \textit{96} ~ \ \ \\ 
	 ~ Fruit and vegetables & 60 ~ \ \ & 89 ~ \ \ & \textit{119} ~ \ \ \\ 
	 ~ Sugar & 152 ~ \ \ & 151 ~ \ \ & \textit{80} ~ \ \ \\ 
	 ~ Livestock & 58 ~ \ \ & 88 ~ \ \ & \textit{117} ~ \ \ \\ 
	 ~ Milk & 67 ~ \ \ & 94 ~ \ \ & \textit{115} ~ \ \ \\ 
	 ~ Meat & 53 ~ \ \ & 85 ~ \ \ & \textit{115} ~ \ \ \\ 
	 ~ Fish  & 125 ~ \ \ & 93 ~ \ \ & \textit{54} ~ \ \ \\ 
	\multicolumn{4}{l}{\textit{\normalsize{Net trade (min US\$)}}} \\ 
	 ~ Cereals & -99 ~ \ \ & -111 ~ \ \ & \textit{-533} ~ \ \ \\ 
	 ~ Fruit and vegetables & 1\,092 ~ \ \ & 1\,898 ~ \ \ & \textit{5\,326} ~ \ \ \\ 
	 ~ Meat & -37 ~ \ \ & 111 ~ \ \ & \textit{-95} ~ \ \ \\ 
	 ~ Dairy products & -44 ~ \ \ & 17 ~ \ \ & \textit{26} ~ \ \ \\ 
	 ~ Fish & 1\,230 ~ \ \ & 1\,823 ~ \ \ & \textit{3\,911} ~ \ \ \\ 
	\multicolumn{4}{l}{\textcolor{FAOblue}{\textbf{\large{Environment}}}} \\ 
	 ~ Forest area (\%) & 21 ~ \ \ & 21 ~ \ \ & \textit{22} ~ \ \ \\ 
	 ~ Renewable water res withdrawn (\% of total) &  ~ \ \ &  ~ \ \ & 83 ~ \ \ \\ 
	 ~ Terrestrial protect areas (\% total land area)  & 16 ~ \ \ & 17 ~ \ \ & \textit{19} ~ \ \ \\ 
	 ~ Organic area (\% total agricultural area) &  ~ \ \ & \textit{0} ~ \ \ & \textit{0} ~ \ \ \\ 
	 ~ Water withdrawal by agriculture (\% of total) &  ~ \ \ &  ~ \ \ & 83 ~ \ \ \\ 
	 ~ Biofuel production (thousand kt of oil eq.) & 1 ~ \ \ & \textit{0} ~ \ \ & \textit{0} ~ \ \ \\ 
	 ~ Wood pellet prod. (min tonnes) &  ~ \ \ &  ~ \ \ & \textit{10} ~ \ \ \\ 
	 ~ GHG emissions from ag (Co2 eq, gigagrams) & -3 ~ \ \ & 5 ~ \ \ & \textit{5} ~ \ \ \\ 
       \toprule
      \end{tabular}
      \clearpage
\CountryData{ China }
      \rowcolors{1}{FAOblue!10}{white}
      \begin{tabular}{L{3.9cm} R{1cm} R{1cm} R{1cm}}
      \toprule
      \multicolumn{1}{c}{} & \multicolumn{1}{c}{ 1992 } & \multicolumn{1}{c}{ 2002 } & \multicolumn{1}{c}{ 2014 } \\
      \midrule
	\multicolumn{4}{l}{\textcolor{FAOblue}{\textbf{\large{The setting}}}} \\ 
	 ~ Population, total (mln) & 1\,225.8 ~ \ \ & 1\,324.9 ~ \ \ & 1\,425 ~ \ \ \\ 
	 ~ Population, rural (\% total population) & 867.5 ~ \ \ & 804.2 ~ \ \ & 641.6 ~ \ \ \\ 
	 ~ Govt expenditure on ag (\% total outlays) &  ~ \ \ & 1.4 ~ \ \ & \textit{1.9} ~ \ \ \\ 
	 ~ Area harvested (mln ha) & 404 ~ \ \ & 413 ~ \ \ & 553 ~ \ \ \\ 
	 ~ Cropping intensity ratio (\%) & 0.8 ~ \ \ & 0.8 ~ \ \ &  ~ \ \ \\ 
	 ~ Water resources (m\textsuperscript{3}/person/year) & \textit{2} ~ \ \ & \textit{2} ~ \ \ & \textit{2} ~ \ \ \\ 
	 ~ Area equipped for irrigation (1000 ha) &  ~ \ \ &  ~ \ \ & \textit{69\,390} ~ \ \ \\ 
	 ~ Area irrigated (\%) &  ~ \ \ &  ~ \ \ & \textit{86.1} ~ \ \ \\ 
	 ~ Employment in agriculture (\%) & 58.2 ~ \ \ & 49.8 ~ \ \ & \textit{36.7} ~ \ \ \\ 
	 ~ Employment in agriculture, female (\%) & 0.4 ~ \ \ & 0.2 ~ \ \ & \textit{0.2} ~ \ \ \\ 
	 ~ Fertilizers, Nitrogen (nutrients per ha) &  ~ \ \ & 55.7 ~ \ \ & \textit{87.3} ~ \ \ \\ 
	 ~ Fertilizers, Phosphate (nutrients per ha) &  ~ \ \ & 19 ~ \ \ & \textit{32.2} ~ \ \ \\ 
	 ~ Fertilizers, Potash (nutrients per ha) &  ~ \ \ & 9.8 ~ \ \ & \textit{13.6} ~ \ \ \\ 
	 ~ Energy consump, power irrigation (mln kWh) & 1\,159 ~ \ \ & 1\,337 ~ \ \ & \textit{7\,813} ~ \ \ \\ 
	 ~ Agr value added per worker (constant US\$) &  ~ \ \ &  ~ \ \ &  ~ \ \ \\ 
	\multicolumn{4}{l}{\textcolor{FAOblue}{\textbf{\large{Hunger dimensions}}}} \\ 
	 ~ Dietary energy supply (kcal/pc/day) & 2\,488 ~ \ \ & 2\,831 ~ \ \ & 3\,132 ~ \ \ \\ 
	 ~ Average dietary energy supply adequacy (\%) & 106 ~ \ \ & 116 ~ \ \ & 128 ~ \ \ \\ 
	 ~ Dietary en supp, cereals/roots/tubers (\%) & 68 ~ \ \ & 59 ~ \ \ & \textit{51} ~ \ \ \\ 
	 ~ Prevalence of undernourishment (\%) & 24.5 ~ \ \ & 15.9 ~ \ \ & 9.8 ~ \ \ \\ 
	 ~ GDP per capita (US\$, PPP) & 1\,956 ~ \ \ & 4\,375 ~ \ \ & \textit{11\,778} ~ \ \ \\ 
	 ~ Domestic food price volatility (index) &  ~ \ \ & 10 ~ \ \ & 8.1 ~ \ \ \\ 
	 ~ Cereal import dependency ratio (\%) & 1.9 ~ \ \ & -1.7 ~ \ \ & \textit{100} ~ \ \ \\ 
	 ~ Underweight, children under-5 (\%) & 14.2 ~ \ \ & 6.8 ~ \ \ & \textit{3.4} ~ \ \ \\ 
	 ~ Improved water source (\% pop) &  ~ \ \ &  ~ \ \ &  ~ \ \ \\ 
	\multicolumn{4}{l}{\textcolor{FAOblue}{\textbf{\large{Food Supply}}}} \\ 
	 ~ Food production value, (2004-2006 mln I\$) & 212\,398 ~ \ \ & 357\,560 ~ \ \ & \textit{518\,851} ~ \ \ \\ 
	 ~ Agriculture, value added (\% GDP) &  ~ \ \ & 12 ~ \ \ & \textit{9} ~ \ \ \\ 
	 ~ Food exports (mln US\$)  & 9\,543 ~ \ \ & 12\,116 ~ \ \ & \textit{36\,567} ~ \ \ \\ 
	 ~ Food imports (mln US\$)  & 10\,011 ~ \ \ & 15\,590 ~ \ \ & \textit{96\,838} ~ \ \ \\ 
	\multicolumn{4}{l}{\textit{\normalsize{Production indices (2004-06=100)}}} \\ 
	 ~ Net food &  ~ \ \ &  ~ \ \ &  ~ \ \ \\ 
	 ~ Net crop &  ~ \ \ &  ~ \ \ &  ~ \ \ \\ 
	 ~ Cereal & 96 ~ \ \ & 94 ~ \ \ & \textit{125} ~ \ \ \\ 
	 ~ Vegetable oils & 59 ~ \ \ & 98 ~ \ \ & \textit{111} ~ \ \ \\ 
	 ~ Roots and tubers & 77 ~ \ \ & 110 ~ \ \ & \textit{117} ~ \ \ \\ 
	 ~ Fruit and vegetables & 31 ~ \ \ & 87 ~ \ \ & \textit{145} ~ \ \ \\ 
	 ~ Sugar & 99 ~ \ \ & 109 ~ \ \ & \textit{141} ~ \ \ \\ 
	 ~ Livestock &  ~ \ \ &  ~ \ \ &  ~ \ \ \\ 
	 ~ Milk & 26 ~ \ \ & 55 ~ \ \ & \textit{126} ~ \ \ \\ 
	 ~ Meat & 51 ~ \ \ & 90 ~ \ \ & \textit{128} ~ \ \ \\ 
	 ~ Fish  & 39 ~ \ \ & 90 ~ \ \ & \textit{140} ~ \ \ \\ 
	\multicolumn{4}{l}{\textit{\normalsize{Net trade (min US\$)}}} \\ 
	 ~ Cereals & -1\,429 ~ \ \ & -176 ~ \ \ & \textit{-8\,335} ~ \ \ \\ 
	 ~ Fruit and vegetables & 1\,216 ~ \ \ & 2\,238 ~ \ \ & \textit{8\,588} ~ \ \ \\ 
	 ~ Meat & 798 ~ \ \ & -315 ~ \ \ & \textit{-5\,731} ~ \ \ \\ 
	 ~ Dairy products & -462 ~ \ \ & -616 ~ \ \ & \textit{-4\,748} ~ \ \ \\ 
	 ~ Fish & 72 ~ \ \ & 871 ~ \ \ & \textit{7\,824} ~ \ \ \\ 
	\multicolumn{4}{l}{\textcolor{FAOblue}{\textbf{\large{Environment}}}} \\ 
	 ~ Forest area (\%) &  ~ \ \ &  ~ \ \ &  ~ \ \ \\ 
	 ~ Renewable water res withdrawn (\% of total) &  ~ \ \ & \textit{65} ~ \ \ & \textit{65} ~ \ \ \\ 
	 ~ Terrestrial protect areas (\% total land area)  & 14 ~ \ \ & 17 ~ \ \ & \textit{17} ~ \ \ \\ 
	 ~ Organic area (\% total agricultural area) &  ~ \ \ & \textit{0} ~ \ \ & \textit{0} ~ \ \ \\ 
	 ~ Water withdrawal by agriculture (\% of total) &  ~ \ \ & \textit{65} ~ \ \ & \textit{65} ~ \ \ \\ 
	 ~ Biofuel production (thousand kt of oil eq.) & 179 ~ \ \ & 220 ~ \ \ & \textit{266} ~ \ \ \\ 
	 ~ Wood pellet prod. (min tonnes) &  ~ \ \ &  ~ \ \ & \textit{200} ~ \ \ \\ 
	 ~ GHG emissions from ag (Co2 eq, gigagrams) & 246 ~ \ \ & 291 ~ \ \ & \textit{544} ~ \ \ \\ 
       \toprule
      \end{tabular}
      \clearpage
\CountryData{ Colombia }
      \rowcolors{1}{FAOblue!10}{white}
      \begin{tabular}{L{3.9cm} R{1cm} R{1cm} R{1cm}}
      \toprule
      \multicolumn{1}{c}{} & \multicolumn{1}{c}{ 1992 } & \multicolumn{1}{c}{ 2002 } & \multicolumn{1}{c}{ 2014 } \\
      \midrule
	\multicolumn{4}{l}{\textcolor{FAOblue}{\textbf{\large{The setting}}}} \\ 
	 ~ Population, total (mln) & 34.6 ~ \ \ & 41.2 ~ \ \ & 48.9 ~ \ \ \\ 
	 ~ Population, rural (\% total population) & 10.6 ~ \ \ & 11.3 ~ \ \ & 11.7 ~ \ \ \\ 
	 ~ Govt expenditure on ag (\% total outlays) &  ~ \ \ &  ~ \ \ &  ~ \ \ \\ 
	 ~ Area harvested (mln ha) & 29 ~ \ \ & 38 ~ \ \ & 35 ~ \ \ \\ 
	 ~ Cropping intensity ratio (\%) & 0.6 ~ \ \ & 0.9 ~ \ \ &  ~ \ \ \\ 
	 ~ Water resources (m\textsuperscript{3}/person/year) & \textit{67} ~ \ \ & \textit{56} ~ \ \ & \textit{49} ~ \ \ \\ 
	 ~ Area equipped for irrigation (1000 ha) &  ~ \ \ &  ~ \ \ & \textit{1\,090} ~ \ \ \\ 
	 ~ Area irrigated (\%) &  ~ \ \ &  ~ \ \ & \textit{36.2} ~ \ \ \\ 
	 ~ Employment in agriculture (\%) & 1.4 ~ \ \ & 21.1 ~ \ \ & \textit{16.9} ~ \ \ \\ 
	 ~ Employment in agriculture, female (\%) & 0.8 ~ \ \ & 6.6 ~ \ \ & \textit{6.9} ~ \ \ \\ 
	 ~ Fertilizers, Nitrogen (nutrients per ha) &  ~ \ \ & 8.6 ~ \ \ & \textit{14.5} ~ \ \ \\ 
	 ~ Fertilizers, Phosphate (nutrients per ha) &  ~ \ \ & 2.8 ~ \ \ & \textit{6.9} ~ \ \ \\ 
	 ~ Fertilizers, Potash (nutrients per ha) &  ~ \ \ & 5 ~ \ \ & \textit{6.2} ~ \ \ \\ 
	 ~ Energy consump, power irrigation (mln kWh) & 94 ~ \ \ & 94 ~ \ \ & \textit{94} ~ \ \ \\ 
	 ~ Agr value added per worker (constant US\$) & 3.6 ~ \ \ & 2.9 ~ \ \ & \textit{3.8} ~ \ \ \\ 
	\multicolumn{4}{l}{\textcolor{FAOblue}{\textbf{\large{Hunger dimensions}}}} \\ 
	 ~ Dietary energy supply (kcal/pc/day) & 2\,630 ~ \ \ & 2\,798 ~ \ \ & 2\,829 ~ \ \ \\ 
	 ~ Average dietary energy supply adequacy (\%) & 117 ~ \ \ & 122 ~ \ \ & 121 ~ \ \ \\ 
	 ~ Dietary en supp, cereals/roots/tubers (\%) & 38 ~ \ \ & 36 ~ \ \ & \textit{34} ~ \ \ \\ 
	 ~ Prevalence of undernourishment (\%) & 13.8 ~ \ \ & 9 ~ \ \ & 9.5 ~ \ \ \\ 
	 ~ GDP per capita (US\$, PPP) & 8\,014 ~ \ \ & 8\,488 ~ \ \ & \textit{12\,025} ~ \ \ \\ 
	 ~ Domestic food price volatility (index) &  ~ \ \ & 4.5 ~ \ \ & 4.5 ~ \ \ \\ 
	 ~ Cereal import dependency ratio (\%) & 29.6 ~ \ \ & 53 ~ \ \ & \textit{63.3} ~ \ \ \\ 
	 ~ Underweight, children under-5 (\%) & \textit{6.3} ~ \ \ & \textit{5.1} ~ \ \ & \textit{3.4} ~ \ \ \\ 
	 ~ Improved water source (\% pop) & 88.8 ~ \ \ & 90.1 ~ \ \ & \textit{91.2} ~ \ \ \\ 
	\multicolumn{4}{l}{\textcolor{FAOblue}{\textbf{\large{Food Supply}}}} \\ 
	 ~ Food production value, (2004-2006 mln I\$) & 8\,197 ~ \ \ & 10\,575 ~ \ \ & \textit{13\,623} ~ \ \ \\ 
	 ~ Agriculture, value added (\% GDP) & 16 ~ \ \ & 9 ~ \ \ & \textit{6} ~ \ \ \\ 
	 ~ Food exports (mln US\$)  & 727 ~ \ \ & 1\,082 ~ \ \ & \textit{2\,717} ~ \ \ \\ 
	 ~ Food imports (mln US\$)  & 479 ~ \ \ & 1\,213 ~ \ \ & \textit{4\,360} ~ \ \ \\ 
	\multicolumn{4}{l}{\textit{\normalsize{Production indices (2004-06=100)}}} \\ 
	 ~ Net food & 70 ~ \ \ & 90 ~ \ \ & \textit{117} ~ \ \ \\ 
	 ~ Net crop & 84 ~ \ \ & 92 ~ \ \ & \textit{108} ~ \ \ \\ 
	 ~ Cereal & 86 ~ \ \ & 90 ~ \ \ & \textit{93} ~ \ \ \\ 
	 ~ Vegetable oils & 47 ~ \ \ & 79 ~ \ \ & \textit{139} ~ \ \ \\ 
	 ~ Roots and tubers & 96 ~ \ \ & 90 ~ \ \ & \textit{118} ~ \ \ \\ 
	 ~ Fruit and vegetables & 65 ~ \ \ & 92 ~ \ \ & \textit{121} ~ \ \ \\ 
	 ~ Sugar & 74 ~ \ \ & 96 ~ \ \ & \textit{89} ~ \ \ \\ 
	 ~ Livestock & 67 ~ \ \ & 89 ~ \ \ & \textit{120} ~ \ \ \\ 
	 ~ Milk & 67 ~ \ \ & 100 ~ \ \ & \textit{99} ~ \ \ \\ 
	 ~ Meat & 69 ~ \ \ & 83 ~ \ \ & \textit{130} ~ \ \ \\ 
	 ~ Fish  & 95 ~ \ \ & 103 ~ \ \ & \textit{85} ~ \ \ \\ 
	\multicolumn{4}{l}{\textit{\normalsize{Net trade (min US\$)}}} \\ 
	 ~ Cereals & -243 ~ \ \ & -509 ~ \ \ & \textit{-1\,966} ~ \ \ \\ 
	 ~ Fruit and vegetables & 414 ~ \ \ & 329 ~ \ \ & \textit{395} ~ \ \ \\ 
	 ~ Meat & 6 ~ \ \ & -30 ~ \ \ & \textit{-108} ~ \ \ \\ 
	 ~ Dairy products & -14 ~ \ \ & 24 ~ \ \ & \textit{-112} ~ \ \ \\ 
	 ~ Fish & 116 ~ \ \ & 89 ~ \ \ & \textit{-170} ~ \ \ \\ 
	\multicolumn{4}{l}{\textcolor{FAOblue}{\textbf{\large{Environment}}}} \\ 
	 ~ Forest area (\%) & 56 ~ \ \ & 55 ~ \ \ & \textit{54} ~ \ \ \\ 
	 ~ Renewable water res withdrawn (\% of total) &  ~ \ \ &  ~ \ \ & 54 ~ \ \ \\ 
	 ~ Terrestrial protect areas (\% total land area)  & 19 ~ \ \ & 20 ~ \ \ & \textit{21} ~ \ \ \\ 
	 ~ Organic area (\% total agricultural area) &  ~ \ \ & \textit{0} ~ \ \ & \textit{0} ~ \ \ \\ 
	 ~ Water withdrawal by agriculture (\% of total) &  ~ \ \ &  ~ \ \ & 54 ~ \ \ \\ 
	 ~ Biofuel production (thousand kt of oil eq.) & 64 ~ \ \ & 77 ~ \ \ & \textit{9\,285} ~ \ \ \\ 
	 ~ Wood pellet prod. (min tonnes) &  ~ \ \ &  ~ \ \ &  ~ \ \ \\ 
	 ~ GHG emissions from ag (Co2 eq, gigagrams) & 97 ~ \ \ & 99 ~ \ \ & \textit{98} ~ \ \ \\ 
       \toprule
      \end{tabular}
      \clearpage
\CountryData{ Comoros }
      \rowcolors{1}{FAOblue!10}{white}
      \begin{tabular}{L{3.9cm} R{1cm} R{1cm} R{1cm}}
      \toprule
      \multicolumn{1}{c}{} & \multicolumn{1}{c}{ 1992 } & \multicolumn{1}{c}{ 2002 } & \multicolumn{1}{c}{ 2014 } \\
      \midrule
	\multicolumn{4}{l}{\textcolor{FAOblue}{\textbf{\large{The setting}}}} \\ 
	 ~ Population, total (mln) & 0.4 ~ \ \ & 0.6 ~ \ \ & 0.8 ~ \ \ \\ 
	 ~ Population, rural (\% total population) & 0.3 ~ \ \ & 0.4 ~ \ \ & 0.5 ~ \ \ \\ 
	 ~ Govt expenditure on ag (\% total outlays) &  ~ \ \ &  ~ \ \ &  ~ \ \ \\ 
	 ~ Area harvested (mln ha) & 0 ~ \ \ & 0 ~ \ \ & 0 ~ \ \ \\ 
	 ~ Cropping intensity ratio (\%) & 1.1 ~ \ \ & 1 ~ \ \ &  ~ \ \ \\ 
	 ~ Water resources (m\textsuperscript{3}/person/year) & \textit{3} ~ \ \ & \textit{2} ~ \ \ & \textit{2} ~ \ \ \\ 
	 ~ Area equipped for irrigation (1000 ha) &  ~ \ \ &  ~ \ \ & \textit{0} ~ \ \ \\ 
	 ~ Area irrigated (\%) &  ~ \ \ &  ~ \ \ &  ~ \ \ \\ 
	 ~ Employment in agriculture (\%) &  ~ \ \ &  ~ \ \ &  ~ \ \ \\ 
	 ~ Employment in agriculture, female (\%) &  ~ \ \ &  ~ \ \ &  ~ \ \ \\ 
	 ~ Fertilizers, Nitrogen (nutrients per ha) &  ~ \ \ &  ~ \ \ &  ~ \ \ \\ 
	 ~ Fertilizers, Phosphate (nutrients per ha) &  ~ \ \ &  ~ \ \ &  ~ \ \ \\ 
	 ~ Fertilizers, Potash (nutrients per ha) &  ~ \ \ &  ~ \ \ &  ~ \ \ \\ 
	 ~ Energy consump, power irrigation (mln kWh) &  ~ \ \ &  ~ \ \ &  ~ \ \ \\ 
	 ~ Agr value added per worker (constant US\$) & 0.7 ~ \ \ & 0.8 ~ \ \ & \textit{0.8} ~ \ \ \\ 
	\multicolumn{4}{l}{\textcolor{FAOblue}{\textbf{\large{Hunger dimensions}}}} \\ 
	 ~ Dietary energy supply (kcal/pc/day) &  ~ \ \ &  ~ \ \ &  ~ \ \ \\ 
	 ~ Average dietary energy supply adequacy (\%) &  ~ \ \ &  ~ \ \ &  ~ \ \ \\ 
	 ~ Dietary en supp, cereals/roots/tubers (\%) &  ~ \ \ &  ~ \ \ &  ~ \ \ \\ 
	 ~ Prevalence of undernourishment (\%) &  ~ \ \ &  ~ \ \ &  ~ \ \ \\ 
	 ~ GDP per capita (US\$, PPP) & 1\,507 ~ \ \ & 1\,460 ~ \ \ & \textit{1\,400} ~ \ \ \\ 
	 ~ Domestic food price volatility (index) &  ~ \ \ &  ~ \ \ &  ~ \ \ \\ 
	 ~ Cereal import dependency ratio (\%) & 69.8 ~ \ \ & 70.8 ~ \ \ & \textit{71.2} ~ \ \ \\ 
	 ~ Underweight, children under-5 (\%) & 16.2 ~ \ \ & \textit{25} ~ \ \ & \textit{16.9} ~ \ \ \\ 
	 ~ Improved water source (\% pop) & 88.1 ~ \ \ & 93 ~ \ \ & \textit{95.1} ~ \ \ \\ 
	\multicolumn{4}{l}{\textcolor{FAOblue}{\textbf{\large{Food Supply}}}} \\ 
	 ~ Food production value, (2004-2006 mln I\$) & 53 ~ \ \ & 62 ~ \ \ & \textit{72} ~ \ \ \\ 
	 ~ Agriculture, value added (\% GDP) & 40 ~ \ \ & 41 ~ \ \ & \textit{37} ~ \ \ \\ 
	 ~ Food exports (mln US\$)  & 16 ~ \ \ & 9 ~ \ \ & \textit{27} ~ \ \ \\ 
	 ~ Food imports (mln US\$)  & 21 ~ \ \ & 23 ~ \ \ & \textit{81} ~ \ \ \\ 
	\multicolumn{4}{l}{\textit{\normalsize{Production indices (2004-06=100)}}} \\ 
	 ~ Net food & 84 ~ \ \ & 97 ~ \ \ & \textit{113} ~ \ \ \\ 
	 ~ Net crop & 82 ~ \ \ & 96 ~ \ \ & \textit{113} ~ \ \ \\ 
	 ~ Cereal & 81 ~ \ \ & 92 ~ \ \ & \textit{149} ~ \ \ \\ 
	 ~ Vegetable oils & 74 ~ \ \ & 88 ~ \ \ & \textit{110} ~ \ \ \\ 
	 ~ Roots and tubers & 84 ~ \ \ & 94 ~ \ \ & \textit{124} ~ \ \ \\ 
	 ~ Fruit and vegetables & 81 ~ \ \ & 91 ~ \ \ & \textit{101} ~ \ \ \\ 
	 ~ Sugar &  ~ \ \ &  ~ \ \ &  ~ \ \ \\ 
	 ~ Livestock & 96 ~ \ \ & 98 ~ \ \ & \textit{113} ~ \ \ \\ 
	 ~ Milk & 92 ~ \ \ & 77 ~ \ \ & \textit{117} ~ \ \ \\ 
	 ~ Meat & 99 ~ \ \ & 105 ~ \ \ & \textit{112} ~ \ \ \\ 
	 ~ Fish  &  ~ \ \ &  ~ \ \ &  ~ \ \ \\ 
	\multicolumn{4}{l}{\textit{\normalsize{Net trade (min US\$)}}} \\ 
	 ~ Cereals & \textit{-16} ~ \ \ & -13 ~ \ \ & \textit{-38} ~ \ \ \\ 
	 ~ Fruit and vegetables & -1 ~ \ \ & -2 ~ \ \ & \textit{-2} ~ \ \ \\ 
	 ~ Meat & \textit{-4} ~ \ \ & -4 ~ \ \ & \textit{-28} ~ \ \ \\ 
	 ~ Dairy products & \textit{-1} ~ \ \ & -2 ~ \ \ & \textit{-3} ~ \ \ \\ 
	 ~ Fish & -1 ~ \ \ & 0 ~ \ \ & \textit{-3} ~ \ \ \\ 
	\multicolumn{4}{l}{\textcolor{FAOblue}{\textbf{\large{Environment}}}} \\ 
	 ~ Forest area (\%) & 6 ~ \ \ & 4 ~ \ \ & \textit{1} ~ \ \ \\ 
	 ~ Renewable water res withdrawn (\% of total) &  ~ \ \ & \textit{47} ~ \ \ & 47 ~ \ \ \\ 
	 ~ Terrestrial protect areas (\% total land area)  &  ~ \ \ & \textit{1} ~ \ \ & \textit{10} ~ \ \ \\ 
	 ~ Organic area (\% total agricultural area) &  ~ \ \ &  ~ \ \ & \textit{2} ~ \ \ \\ 
	 ~ Water withdrawal by agriculture (\% of total) &  ~ \ \ & \textit{47} ~ \ \ & 47 ~ \ \ \\ 
	 ~ Biofuel production (thousand kt of oil eq.) &  ~ \ \ &  ~ \ \ &  ~ \ \ \\ 
	 ~ Wood pellet prod. (min tonnes) &  ~ \ \ &  ~ \ \ &  ~ \ \ \\ 
	 ~ GHG emissions from ag (Co2 eq, gigagrams) & 0 ~ \ \ & 1 ~ \ \ & \textit{0} ~ \ \ \\ 
       \toprule
      \end{tabular}
      \clearpage
\CountryData{ Congo }
      \rowcolors{1}{FAOblue!10}{white}
      \begin{tabular}{L{3.9cm} R{1cm} R{1cm} R{1cm}}
      \toprule
      \multicolumn{1}{c}{} & \multicolumn{1}{c}{ 1992 } & \multicolumn{1}{c}{ 2002 } & \multicolumn{1}{c}{ 2014 } \\
      \midrule
	\multicolumn{4}{l}{\textcolor{FAOblue}{\textbf{\large{The setting}}}} \\ 
	 ~ Population, total (mln) & 2.5 ~ \ \ & 3.3 ~ \ \ & 4.6 ~ \ \ \\ 
	 ~ Population, rural (\% total population) & 1.1 ~ \ \ & 1.3 ~ \ \ & 1.6 ~ \ \ \\ 
	 ~ Govt expenditure on ag (\% total outlays) &  ~ \ \ &  ~ \ \ &  ~ \ \ \\ 
	 ~ Area harvested (mln ha) & 1 ~ \ \ & 1 ~ \ \ & 1 ~ \ \ \\ 
	 ~ Cropping intensity ratio (\%) & 0.1 ~ \ \ & 0.1 ~ \ \ &  ~ \ \ \\ 
	 ~ Water resources (m\textsuperscript{3}/person/year) & \textit{323} ~ \ \ & \textit{247} ~ \ \ & \textit{187} ~ \ \ \\ 
	 ~ Area equipped for irrigation (1000 ha) &  ~ \ \ &  ~ \ \ & \textit{2} ~ \ \ \\ 
	 ~ Area irrigated (\%) &  ~ \ \ &  ~ \ \ &  ~ \ \ \\ 
	 ~ Employment in agriculture (\%) &  ~ \ \ & \textit{35.4} ~ \ \ & \textit{35.4} ~ \ \ \\ 
	 ~ Employment in agriculture, female (\%) &  ~ \ \ & \textit{39.3} ~ \ \ & \textit{39.3} ~ \ \ \\ 
	 ~ Fertilizers, Nitrogen (nutrients per ha) &  ~ \ \ & 0 ~ \ \ & \textit{0.3} ~ \ \ \\ 
	 ~ Fertilizers, Phosphate (nutrients per ha) &  ~ \ \ & 0 ~ \ \ & \textit{0} ~ \ \ \\ 
	 ~ Fertilizers, Potash (nutrients per ha) &  ~ \ \ & 0 ~ \ \ & \textit{0.1} ~ \ \ \\ 
	 ~ Energy consump, power irrigation (mln kWh) & \textit{0} ~ \ \ & 0 ~ \ \ & \textit{0} ~ \ \ \\ 
	 ~ Agr value added per worker (constant US\$) & 0.5 ~ \ \ & 0.5 ~ \ \ & \textit{0.8} ~ \ \ \\ 
	\multicolumn{4}{l}{\textcolor{FAOblue}{\textbf{\large{Hunger dimensions}}}} \\ 
	 ~ Dietary energy supply (kcal/pc/day) & 2\,003 ~ \ \ & 2\,242 ~ \ \ & 2\,132 ~ \ \ \\ 
	 ~ Average dietary energy supply adequacy (\%) & 91 ~ \ \ & 102 ~ \ \ & 97 ~ \ \ \\ 
	 ~ Dietary en supp, cereals/roots/tubers (\%) & 59 ~ \ \ & 60 ~ \ \ & \textit{61} ~ \ \ \\ 
	 ~ Prevalence of undernourishment (\%) & 43.3 ~ \ \ & 29.2 ~ \ \ & 29 ~ \ \ \\ 
	 ~ GDP per capita (US\$, PPP) & 5\,248 ~ \ \ & 4\,778 ~ \ \ & \textit{5\,680} ~ \ \ \\ 
	 ~ Domestic food price volatility (index) &  ~ \ \ & 19.3 ~ \ \ & 18.8 ~ \ \ \\ 
	 ~ Cereal import dependency ratio (\%) & 95.5 ~ \ \ & 92.9 ~ \ \ & \textit{92.9} ~ \ \ \\ 
	 ~ Underweight, children under-5 (\%) &  ~ \ \ & \textit{11.8} ~ \ \ & \textit{11.8} ~ \ \ \\ 
	 ~ Improved water source (\% pop) &  ~ \ \ & 70.1 ~ \ \ & \textit{75.3} ~ \ \ \\ 
	\multicolumn{4}{l}{\textcolor{FAOblue}{\textbf{\large{Food Supply}}}} \\ 
	 ~ Food production value, (2004-2006 mln I\$) & 201 ~ \ \ & 280 ~ \ \ & \textit{441} ~ \ \ \\ 
	 ~ Agriculture, value added (\% GDP) & 12 ~ \ \ & 6 ~ \ \ & \textit{4} ~ \ \ \\ 
	 ~ Food exports (mln US\$)  & 17 ~ \ \ & 18 ~ \ \ & \textit{17} ~ \ \ \\ 
	 ~ Food imports (mln US\$)  & 126 ~ \ \ & 147 ~ \ \ & \textit{623} ~ \ \ \\ 
	\multicolumn{4}{l}{\textit{\normalsize{Production indices (2004-06=100)}}} \\ 
	 ~ Net food & 62 ~ \ \ & 86 ~ \ \ & \textit{136} ~ \ \ \\ 
	 ~ Net crop & 64 ~ \ \ & 89 ~ \ \ & \textit{127} ~ \ \ \\ 
	 ~ Cereal & 27 ~ \ \ & 84 ~ \ \ & \textit{137} ~ \ \ \\ 
	 ~ Vegetable oils & 86 ~ \ \ & 95 ~ \ \ & \textit{134} ~ \ \ \\ 
	 ~ Roots and tubers & 63 ~ \ \ & 87 ~ \ \ & \textit{126} ~ \ \ \\ 
	 ~ Fruit and vegetables & 57 ~ \ \ & 94 ~ \ \ & \textit{127} ~ \ \ \\ 
	 ~ Sugar & 67 ~ \ \ & 77 ~ \ \ & \textit{109} ~ \ \ \\ 
	 ~ Livestock & 54 ~ \ \ & 75 ~ \ \ & \textit{164} ~ \ \ \\ 
	 ~ Milk & 73 ~ \ \ & 80 ~ \ \ & \textit{112} ~ \ \ \\ 
	 ~ Meat & 53 ~ \ \ & 75 ~ \ \ & \textit{165} ~ \ \ \\ 
	 ~ Fish  & 71 ~ \ \ & 91 ~ \ \ & \textit{129} ~ \ \ \\ 
	\multicolumn{4}{l}{\textit{\normalsize{Net trade (min US\$)}}} \\ 
	 ~ Cereals & -42 ~ \ \ & -60 ~ \ \ & \textit{-248} ~ \ \ \\ 
	 ~ Fruit and vegetables & -6 ~ \ \ & -13 ~ \ \ & \textit{-30} ~ \ \ \\ 
	 ~ Meat & -42 ~ \ \ & -19 ~ \ \ & \textit{-204} ~ \ \ \\ 
	 ~ Dairy products & -11 ~ \ \ & -15 ~ \ \ & \textit{-33} ~ \ \ \\ 
	 ~ Fish & -25 ~ \ \ & -9 ~ \ \ & \textit{-52} ~ \ \ \\ 
	\multicolumn{4}{l}{\textcolor{FAOblue}{\textbf{\large{Environment}}}} \\ 
	 ~ Forest area (\%) & 66 ~ \ \ & 66 ~ \ \ & \textit{66} ~ \ \ \\ 
	 ~ Renewable water res withdrawn (\% of total) &  ~ \ \ & 9 ~ \ \ & 9 ~ \ \ \\ 
	 ~ Terrestrial protect areas (\% total land area)  & 5 ~ \ \ & 9 ~ \ \ & \textit{30} ~ \ \ \\ 
	 ~ Organic area (\% total agricultural area) &  ~ \ \ &  ~ \ \ &  ~ \ \ \\ 
	 ~ Water withdrawal by agriculture (\% of total) &  ~ \ \ & 9 ~ \ \ & 9 ~ \ \ \\ 
	 ~ Biofuel production (thousand kt of oil eq.) & 1 ~ \ \ & 1 ~ \ \ & \textit{2} ~ \ \ \\ 
	 ~ Wood pellet prod. (min tonnes) &  ~ \ \ &  ~ \ \ &  ~ \ \ \\ 
	 ~ GHG emissions from ag (Co2 eq, gigagrams) & 12 ~ \ \ & 12 ~ \ \ & \textit{12} ~ \ \ \\ 
       \toprule
      \end{tabular}
      \clearpage
\CountryData{ Costa Rica }
      \rowcolors{1}{FAOblue!10}{white}
      \begin{tabular}{L{3.9cm} R{1cm} R{1cm} R{1cm}}
      \toprule
      \multicolumn{1}{c}{} & \multicolumn{1}{c}{ 1992 } & \multicolumn{1}{c}{ 2002 } & \multicolumn{1}{c}{ 2014 } \\
      \midrule
	\multicolumn{4}{l}{\textcolor{FAOblue}{\textbf{\large{The setting}}}} \\ 
	 ~ Population, total (mln) & 3.2 ~ \ \ & 4.1 ~ \ \ & 4.9 ~ \ \ \\ 
	 ~ Population, rural (\% total population) & 1.5 ~ \ \ & 1.6 ~ \ \ & 1.7 ~ \ \ \\ 
	 ~ Govt expenditure on ag (\% total outlays) &  ~ \ \ &  ~ \ \ &  ~ \ \ \\ 
	 ~ Area harvested (mln ha) & 3 ~ \ \ & 4 ~ \ \ & 4 ~ \ \ \\ 
	 ~ Cropping intensity ratio (\%) & 1.4 ~ \ \ & 2.1 ~ \ \ &  ~ \ \ \\ 
	 ~ Water resources (m\textsuperscript{3}/person/year) & \textit{34} ~ \ \ & \textit{27} ~ \ \ & \textit{23} ~ \ \ \\ 
	 ~ Area equipped for irrigation (1000 ha) &  ~ \ \ &  ~ \ \ & \textit{102} ~ \ \ \\ 
	 ~ Area irrigated (\%) &  ~ \ \ &  ~ \ \ & \textit{100} ~ \ \ \\ 
	 ~ Employment in agriculture (\%) & 24.1 ~ \ \ & 15.9 ~ \ \ & \textit{13.4} ~ \ \ \\ 
	 ~ Employment in agriculture, female (\%) & 5.4 ~ \ \ & 4.1 ~ \ \ & \textit{3.9} ~ \ \ \\ 
	 ~ Fertilizers, Nitrogen (nutrients per ha) &  ~ \ \ & 20.4 ~ \ \ & \textit{45.4} ~ \ \ \\ 
	 ~ Fertilizers, Phosphate (nutrients per ha) &  ~ \ \ & 12.6 ~ \ \ & \textit{11.8} ~ \ \ \\ 
	 ~ Fertilizers, Potash (nutrients per ha) &  ~ \ \ & 25.7 ~ \ \ & \textit{34.5} ~ \ \ \\ 
	 ~ Energy consump, power irrigation (mln kWh) & 0 ~ \ \ & 38 ~ \ \ & \textit{38} ~ \ \ \\ 
	 ~ Agr value added per worker (constant US\$) & 3.5 ~ \ \ & 4.4 ~ \ \ & \textit{6.6} ~ \ \ \\ 
	\multicolumn{4}{l}{\textcolor{FAOblue}{\textbf{\large{Hunger dimensions}}}} \\ 
	 ~ Dietary energy supply (kcal/pc/day) & 2\,729 ~ \ \ & 2\,770 ~ \ \ & 2\,897 ~ \ \ \\ 
	 ~ Average dietary energy supply adequacy (\%) & 119 ~ \ \ & 117 ~ \ \ & 119 ~ \ \ \\ 
	 ~ Dietary en supp, cereals/roots/tubers (\%) & 35 ~ \ \ & 35 ~ \ \ & \textit{34} ~ \ \ \\ 
	 ~ Prevalence of undernourishment (\%) & 5.4 ~ \ \ & 5.4 ~ \ \ & 5.5 ~ \ \ \\ 
	 ~ GDP per capita (US\$, PPP) & 7\,816 ~ \ \ & 9\,549 ~ \ \ & \textit{13\,431} ~ \ \ \\ 
	 ~ Domestic food price volatility (index) &  ~ \ \ & 6.3 ~ \ \ & 7.6 ~ \ \ \\ 
	 ~ Cereal import dependency ratio (\%) & 67.7 ~ \ \ & 84.8 ~ \ \ & \textit{82.4} ~ \ \ \\ 
	 ~ Underweight, children under-5 (\%) & 2 ~ \ \ & \textit{4.5} ~ \ \ & \textit{1.1} ~ \ \ \\ 
	 ~ Improved water source (\% pop) & 93.6 ~ \ \ & 95.3 ~ \ \ & \textit{96.6} ~ \ \ \\ 
	\multicolumn{4}{l}{\textcolor{FAOblue}{\textbf{\large{Food Supply}}}} \\ 
	 ~ Food production value, (2004-2006 mln I\$) & 1\,563 ~ \ \ & 2\,027 ~ \ \ & \textit{2\,941} ~ \ \ \\ 
	 ~ Agriculture, value added (\% GDP) & 13 ~ \ \ & 9 ~ \ \ & \textit{6} ~ \ \ \\ 
	 ~ Food exports (mln US\$)  & 705 ~ \ \ & 1\,218 ~ \ \ & \textit{3\,080} ~ \ \ \\ 
	 ~ Food imports (mln US\$)  & 125 ~ \ \ & 457 ~ \ \ & \textit{1\,306} ~ \ \ \\ 
	\multicolumn{4}{l}{\textit{\normalsize{Production indices (2004-06=100)}}} \\ 
	 ~ Net food & 66 ~ \ \ & 85 ~ \ \ & \textit{124} ~ \ \ \\ 
	 ~ Net crop & 69 ~ \ \ & 85 ~ \ \ & \textit{122} ~ \ \ \\ 
	 ~ Cereal & 113 ~ \ \ & 99 ~ \ \ & \textit{117} ~ \ \ \\ 
	 ~ Vegetable oils & 43 ~ \ \ & 72 ~ \ \ & \textit{169} ~ \ \ \\ 
	 ~ Roots and tubers & 58 ~ \ \ & 122 ~ \ \ & \textit{109} ~ \ \ \\ 
	 ~ Fruit and vegetables & 60 ~ \ \ & 81 ~ \ \ & \textit{126} ~ \ \ \\ 
	 ~ Sugar & 71 ~ \ \ & 89 ~ \ \ & \textit{114} ~ \ \ \\ 
	 ~ Livestock & 73 ~ \ \ & 91 ~ \ \ & \textit{123} ~ \ \ \\ 
	 ~ Milk & 67 ~ \ \ & 97 ~ \ \ & \textit{136} ~ \ \ \\ 
	 ~ Meat & 75 ~ \ \ & 87 ~ \ \ & \textit{116} ~ \ \ \\ 
	 ~ Fish  & 42 ~ \ \ & 118 ~ \ \ & \textit{125} ~ \ \ \\ 
	\multicolumn{4}{l}{\textit{\normalsize{Net trade (min US\$)}}} \\ 
	 ~ Cereals & -56 ~ \ \ & -117 ~ \ \ & \textit{-376} ~ \ \ \\ 
	 ~ Fruit and vegetables & 595 ~ \ \ & 816 ~ \ \ & \textit{1\,776} ~ \ \ \\ 
	 ~ Meat & 27 ~ \ \ & 39 ~ \ \ & \textit{22} ~ \ \ \\ 
	 ~ Dairy products & 5 ~ \ \ & 5 ~ \ \ & \textit{74} ~ \ \ \\ 
	 ~ Fish & 60 ~ \ \ & 106 ~ \ \ & \textit{93} ~ \ \ \\ 
	\multicolumn{4}{l}{\textcolor{FAOblue}{\textbf{\large{Environment}}}} \\ 
	 ~ Forest area (\%) & 49 ~ \ \ & 47 ~ \ \ & \textit{52} ~ \ \ \\ 
	 ~ Renewable water res withdrawn (\% of total) &  ~ \ \ &  ~ \ \ & 57 ~ \ \ \\ 
	 ~ Terrestrial protect areas (\% total land area)  & 20 ~ \ \ & 21 ~ \ \ & \textit{27} ~ \ \ \\ 
	 ~ Organic area (\% total agricultural area) &  ~ \ \ & \textit{1} ~ \ \ & \textit{0} ~ \ \ \\ 
	 ~ Water withdrawal by agriculture (\% of total) &  ~ \ \ &  ~ \ \ & 57 ~ \ \ \\ 
	 ~ Biofuel production (thousand kt of oil eq.) & 3 ~ \ \ & 6 ~ \ \ & \textit{16} ~ \ \ \\ 
	 ~ Wood pellet prod. (min tonnes) &  ~ \ \ &  ~ \ \ & \textit{30} ~ \ \ \\ 
	 ~ GHG emissions from ag (Co2 eq, gigagrams) & 11 ~ \ \ & -5 ~ \ \ & \textit{-4} ~ \ \ \\ 
       \toprule
      \end{tabular}
      \clearpage
\CountryData{ Croatia }
      \rowcolors{1}{FAOblue!10}{white}
      \begin{tabular}{L{3.9cm} R{1cm} R{1cm} R{1cm}}
      \toprule
      \multicolumn{1}{c}{} & \multicolumn{1}{c}{ 1992 } & \multicolumn{1}{c}{ 2002 } & \multicolumn{1}{c}{ 2014 } \\
      \midrule
	\multicolumn{4}{l}{\textcolor{FAOblue}{\textbf{\large{The setting}}}} \\ 
	 ~ Population, total (mln) & 4.8 ~ \ \ & 4.4 ~ \ \ & 4.3 ~ \ \ \\ 
	 ~ Population, rural (\% total population) & 2.2 ~ \ \ & 2 ~ \ \ & 1.8 ~ \ \ \\ 
	 ~ Govt expenditure on ag (\% total outlays) &  ~ \ \ & 2.6 ~ \ \ & \textit{3.9} ~ \ \ \\ 
	 ~ Area harvested (mln ha) & 2 ~ \ \ & 4 ~ \ \ & 3 ~ \ \ \\ 
	 ~ Cropping intensity ratio (\%) & 1 ~ \ \ & 3.2 ~ \ \ &  ~ \ \ \\ 
	 ~ Water resources (m\textsuperscript{3}/person/year) & \textit{22} ~ \ \ & \textit{24} ~ \ \ & \textit{25} ~ \ \ \\ 
	 ~ Area equipped for irrigation (1000 ha) &  ~ \ \ &  ~ \ \ & \textit{24} ~ \ \ \\ 
	 ~ Area irrigated (\%) &  ~ \ \ &  ~ \ \ &  ~ \ \ \\ 
	 ~ Employment in agriculture (\%) &  ~ \ \ & 15.2 ~ \ \ & \textit{13.7} ~ \ \ \\ 
	 ~ Employment in agriculture, female (\%) &  ~ \ \ & 15 ~ \ \ & \textit{13.6} ~ \ \ \\ 
	 ~ Fertilizers, Nitrogen (nutrients per ha) &  ~ \ \ & 109.9 ~ \ \ & \textit{94.5} ~ \ \ \\ 
	 ~ Fertilizers, Phosphate (nutrients per ha) &  ~ \ \ & 33.8 ~ \ \ & \textit{73.5} ~ \ \ \\ 
	 ~ Fertilizers, Potash (nutrients per ha) &  ~ \ \ & 43 ~ \ \ & \textit{36.6} ~ \ \ \\ 
	 ~ Energy consump, power irrigation (mln kWh) &  ~ \ \ & 1 ~ \ \ & \textit{3} ~ \ \ \\ 
	 ~ Agr value added per worker (constant US\$) & \textit{6.7} ~ \ \ & 13 ~ \ \ & \textit{22.9} ~ \ \ \\ 
	\multicolumn{4}{l}{\textcolor{FAOblue}{\textbf{\large{Hunger dimensions}}}} \\ 
	 ~ Dietary energy supply (kcal/pc/day) &  ~ \ \ &  ~ \ \ &  ~ \ \ \\ 
	 ~ Average dietary energy supply adequacy (\%) & 90 ~ \ \ & 113 ~ \ \ & 121 ~ \ \ \\ 
	 ~ Dietary en supp, cereals/roots/tubers (\%) & 39 ~ \ \ & 38 ~ \ \ & \textit{32} ~ \ \ \\ 
	 ~ Prevalence of undernourishment (\%) & <5.0 ~ \ \ & <5.0 ~ \ \ & <5.0 ~ \ \ \\ 
	 ~ GDP per capita (US\$, PPP) & \textit{12\,543} ~ \ \ & 16\,977 ~ \ \ & \textit{20\,049} ~ \ \ \\ 
	 ~ Domestic food price volatility (index) &  ~ \ \ & 9.3 ~ \ \ & 2.7 ~ \ \ \\ 
	 ~ Cereal import dependency ratio (\%) & -5.4 ~ \ \ & -1.9 ~ \ \ & \textit{-12.4} ~ \ \ \\ 
	 ~ Underweight, children under-5 (\%) & \textit{0.5} ~ \ \ & \textit{0.5} ~ \ \ &  ~ \ \ \\ 
	 ~ Improved water source (\% pop) & 98.4 ~ \ \ & 98.5 ~ \ \ & \textit{98.6} ~ \ \ \\ 
	\multicolumn{4}{l}{\textcolor{FAOblue}{\textbf{\large{Food Supply}}}} \\ 
	 ~ Food production value, (2004-2006 mln I\$) & 1\,155 ~ \ \ & 1\,252 ~ \ \ & \textit{1\,082} ~ \ \ \\ 
	 ~ Agriculture, value added (\% GDP) & \textit{7} ~ \ \ & 6 ~ \ \ & \textit{4} ~ \ \ \\ 
	 ~ Food exports (mln US\$)  & 377 ~ \ \ & 308 ~ \ \ & \textit{1\,104} ~ \ \ \\ 
	 ~ Food imports (mln US\$)  & 433 ~ \ \ & 679 ~ \ \ & \textit{1\,819} ~ \ \ \\ 
	\multicolumn{4}{l}{\textit{\normalsize{Production indices (2004-06=100)}}} \\ 
	 ~ Net food & 105 ~ \ \ & 114 ~ \ \ & \textit{98} ~ \ \ \\ 
	 ~ Net crop & 92 ~ \ \ & 129 ~ \ \ & \textit{98} ~ \ \ \\ 
	 ~ Cereal & 75 ~ \ \ & 118 ~ \ \ & \textit{100} ~ \ \ \\ 
	 ~ Vegetable oils & 52 ~ \ \ & 91 ~ \ \ & \textit{137} ~ \ \ \\ 
	 ~ Roots and tubers & 161 ~ \ \ & 269 ~ \ \ & \textit{62} ~ \ \ \\ 
	 ~ Fruit and vegetables & 133 ~ \ \ & 146 ~ \ \ & \textit{99} ~ \ \ \\ 
	 ~ Sugar & 38 ~ \ \ & 87 ~ \ \ & \textit{77} ~ \ \ \\ 
	 ~ Livestock & 106 ~ \ \ & 84 ~ \ \ & \textit{92} ~ \ \ \\ 
	 ~ Milk & 89 ~ \ \ & 89 ~ \ \ & \textit{90} ~ \ \ \\ 
	 ~ Meat & 133 ~ \ \ & 78 ~ \ \ & \textit{96} ~ \ \ \\ 
	 ~ Fish  & 71 ~ \ \ & 64 ~ \ \ & \textit{186} ~ \ \ \\ 
	\multicolumn{4}{l}{\textit{\normalsize{Net trade (min US\$)}}} \\ 
	 ~ Cereals & 43 ~ \ \ & -13 ~ \ \ & \textit{19} ~ \ \ \\ 
	 ~ Fruit and vegetables & -32 ~ \ \ & -148 ~ \ \ & \textit{-291} ~ \ \ \\ 
	 ~ Meat & -18 ~ \ \ & -47 ~ \ \ & \textit{-189} ~ \ \ \\ 
	 ~ Dairy products & -21 ~ \ \ & -32 ~ \ \ & \textit{-75} ~ \ \ \\ 
	 ~ Fish & 34 ~ \ \ & -4 ~ \ \ & \textit{42} ~ \ \ \\ 
	\multicolumn{4}{l}{\textcolor{FAOblue}{\textbf{\large{Environment}}}} \\ 
	 ~ Forest area (\%) & 33 ~ \ \ & 34 ~ \ \ & \textit{34} ~ \ \ \\ 
	 ~ Renewable water res withdrawn (\% of total) &  ~ \ \ &  ~ \ \ & 1 ~ \ \ \\ 
	 ~ Terrestrial protect areas (\% total land area)  & 8 ~ \ \ & 10 ~ \ \ & \textit{14} ~ \ \ \\ 
	 ~ Organic area (\% total agricultural area) &  ~ \ \ & \textit{0} ~ \ \ & \textit{2} ~ \ \ \\ 
	 ~ Water withdrawal by agriculture (\% of total) &  ~ \ \ &  ~ \ \ & 1 ~ \ \ \\ 
	 ~ Biofuel production (thousand kt of oil eq.) &  ~ \ \ & 2 ~ \ \ & \textit{2} ~ \ \ \\ 
	 ~ Wood pellet prod. (min tonnes) &  ~ \ \ &  ~ \ \ & \textit{190} ~ \ \ \\ 
	 ~ GHG emissions from ag (Co2 eq, gigagrams) & 3 ~ \ \ & -9 ~ \ \ & \textit{-9} ~ \ \ \\ 
       \toprule
      \end{tabular}
      \clearpage
\CountryData{ Cuba }
      \rowcolors{1}{FAOblue!10}{white}
      \begin{tabular}{L{3.9cm} R{1cm} R{1cm} R{1cm}}
      \toprule
      \multicolumn{1}{c}{} & \multicolumn{1}{c}{ 1992 } & \multicolumn{1}{c}{ 2002 } & \multicolumn{1}{c}{ 2014 } \\
      \midrule
	\multicolumn{4}{l}{\textcolor{FAOblue}{\textbf{\large{The setting}}}} \\ 
	 ~ Population, total (mln) & 10.8 ~ \ \ & 11.2 ~ \ \ & 11.3 ~ \ \ \\ 
	 ~ Population, rural (\% total population) & 2.8 ~ \ \ & 2.7 ~ \ \ & 2.8 ~ \ \ \\ 
	 ~ Govt expenditure on ag (\% total outlays) &  ~ \ \ &  ~ \ \ &  ~ \ \ \\ 
	 ~ Area harvested (mln ha) & 66 ~ \ \ & 35 ~ \ \ & 14 ~ \ \ \\ 
	 ~ Cropping intensity ratio (\%) & 9.8 ~ \ \ & 5.2 ~ \ \ &  ~ \ \ \\ 
	 ~ Water resources (m\textsuperscript{3}/person/year) & \textit{4} ~ \ \ & \textit{3} ~ \ \ & \textit{3} ~ \ \ \\ 
	 ~ Area equipped for irrigation (1000 ha) &  ~ \ \ &  ~ \ \ & \textit{870} ~ \ \ \\ 
	 ~ Area irrigated (\%) &  ~ \ \ & \textit{84.8} ~ \ \ &  ~ \ \ \\ 
	 ~ Employment in agriculture (\%) & 25.3 ~ \ \ & 21.7 ~ \ \ & \textit{19.7} ~ \ \ \\ 
	 ~ Employment in agriculture, female (\%) & \textit{13.2} ~ \ \ & 9.2 ~ \ \ & \textit{8.8} ~ \ \ \\ 
	 ~ Fertilizers, Nitrogen (nutrients per ha) &  ~ \ \ & 10.1 ~ \ \ & \textit{12} ~ \ \ \\ 
	 ~ Fertilizers, Phosphate (nutrients per ha) &  ~ \ \ & 3.1 ~ \ \ & \textit{5.3} ~ \ \ \\ 
	 ~ Fertilizers, Potash (nutrients per ha) &  ~ \ \ & 8.5 ~ \ \ & \textit{7.8} ~ \ \ \\ 
	 ~ Energy consump, power irrigation (mln kWh) &  ~ \ \ & 916 ~ \ \ & \textit{916} ~ \ \ \\ 
	 ~ Agr value added per worker (constant US\$) & 2.9 ~ \ \ & 3.3 ~ \ \ & \textit{4.2} ~ \ \ \\ 
	\multicolumn{4}{l}{\textcolor{FAOblue}{\textbf{\large{Hunger dimensions}}}} \\ 
	 ~ Dietary energy supply (kcal/pc/day) & 2\,519 ~ \ \ & 3\,159 ~ \ \ & 3\,459 ~ \ \ \\ 
	 ~ Average dietary energy supply adequacy (\%) & 105 ~ \ \ & 130 ~ \ \ & 140 ~ \ \ \\ 
	 ~ Dietary en supp, cereals/roots/tubers (\%) & 39 ~ \ \ & 48 ~ \ \ & \textit{44} ~ \ \ \\ 
	 ~ Prevalence of undernourishment (\%) & 9 ~ \ \ & <5.0 ~ \ \ & <5.0 ~ \ \ \\ 
	 ~ GDP per capita (US\$, PPP) & 10\,530 ~ \ \ & 11\,597 ~ \ \ & \textit{18\,796} ~ \ \ \\ 
	 ~ Domestic food price volatility (index) &  ~ \ \ &  ~ \ \ &  ~ \ \ \\ 
	 ~ Cereal import dependency ratio (\%) & 84.7 ~ \ \ & 69.3 ~ \ \ & \textit{75.5} ~ \ \ \\ 
	 ~ Underweight, children under-5 (\%) &  ~ \ \ & \textit{3.4} ~ \ \ &  ~ \ \ \\ 
	 ~ Improved water source (\% pop) & \textit{89.6} ~ \ \ & 91.3 ~ \ \ & \textit{94} ~ \ \ \\ 
	\multicolumn{4}{l}{\textcolor{FAOblue}{\textbf{\large{Food Supply}}}} \\ 
	 ~ Food production value, (2004-2006 mln I\$) & 3\,485 ~ \ \ & 3\,240 ~ \ \ & \textit{2\,821} ~ \ \ \\ 
	 ~ Agriculture, value added (\% GDP) & 13 ~ \ \ & 8 ~ \ \ & \textit{5} ~ \ \ \\ 
	 ~ Food exports (mln US\$)  & 1\,413 ~ \ \ & 492 ~ \ \ & \textit{499} ~ \ \ \\ 
	 ~ Food imports (mln US\$)  & 562 ~ \ \ & 665 ~ \ \ & \textit{1\,845} ~ \ \ \\ 
	\multicolumn{4}{l}{\textit{\normalsize{Production indices (2004-06=100)}}} \\ 
	 ~ Net food & 128 ~ \ \ & 119 ~ \ \ & \textit{103} ~ \ \ \\ 
	 ~ Net crop & 116 ~ \ \ & 118 ~ \ \ & \textit{90} ~ \ \ \\ 
	 ~ Cereal & 67 ~ \ \ & 139 ~ \ \ & \textit{145} ~ \ \ \\ 
	 ~ Vegetable oils & 34 ~ \ \ & 96 ~ \ \ & \textit{33} ~ \ \ \\ 
	 ~ Roots and tubers & 36 ~ \ \ & 94 ~ \ \ & \textit{101} ~ \ \ \\ 
	 ~ Fruit and vegetables & 39 ~ \ \ & 87 ~ \ \ & \textit{77} ~ \ \ \\ 
	 ~ Sugar & 428 ~ \ \ & 224 ~ \ \ & \textit{104} ~ \ \ \\ 
	 ~ Livestock & 121 ~ \ \ & 109 ~ \ \ & \textit{144} ~ \ \ \\ 
	 ~ Milk & 145 ~ \ \ & 138 ~ \ \ & \textit{138} ~ \ \ \\ 
	 ~ Meat & 112 ~ \ \ & 104 ~ \ \ & \textit{150} ~ \ \ \\ 
	 ~ Fish  & 203 ~ \ \ & 108 ~ \ \ & \textit{94} ~ \ \ \\ 
	\multicolumn{4}{l}{\textit{\normalsize{Net trade (min US\$)}}} \\ 
	 ~ Cereals &  ~ \ \ & -272 ~ \ \ & \textit{-828} ~ \ \ \\ 
	 ~ Fruit and vegetables & 8 ~ \ \ & -30 ~ \ \ & \textit{-68} ~ \ \ \\ 
	 ~ Meat & -57 ~ \ \ & -103 ~ \ \ & \textit{-379} ~ \ \ \\ 
	 ~ Dairy products & -92 ~ \ \ & -89 ~ \ \ & \textit{-182} ~ \ \ \\ 
	 ~ Fish & 99 ~ \ \ & 64 ~ \ \ & \textit{34} ~ \ \ \\ 
	\multicolumn{4}{l}{\textcolor{FAOblue}{\textbf{\large{Environment}}}} \\ 
	 ~ Forest area (\%) & 20 ~ \ \ & 24 ~ \ \ & \textit{28} ~ \ \ \\ 
	 ~ Renewable water res withdrawn (\% of total) &  ~ \ \ &  ~ \ \ & 65 ~ \ \ \\ 
	 ~ Terrestrial protect areas (\% total land area)  & 4 ~ \ \ & 6 ~ \ \ & \textit{12} ~ \ \ \\ 
	 ~ Organic area (\% total agricultural area) &  ~ \ \ & \textit{0} ~ \ \ & \textit{0} ~ \ \ \\ 
	 ~ Water withdrawal by agriculture (\% of total) &  ~ \ \ &  ~ \ \ & 65 ~ \ \ \\ 
	 ~ Biofuel production (thousand kt of oil eq.) & 184 ~ \ \ & 85 ~ \ \ & \textit{29} ~ \ \ \\ 
	 ~ Wood pellet prod. (min tonnes) &  ~ \ \ &  ~ \ \ &  ~ \ \ \\ 
	 ~ GHG emissions from ag (Co2 eq, gigagrams) & -10 ~ \ \ & -13 ~ \ \ & \textit{0} ~ \ \ \\ 
       \toprule
      \end{tabular}
      \clearpage
\CountryData{ Cyprus }
      \rowcolors{1}{FAOblue!10}{white}
      \begin{tabular}{L{3.9cm} R{1cm} R{1cm} R{1cm}}
      \toprule
      \multicolumn{1}{c}{} & \multicolumn{1}{c}{ 1992 } & \multicolumn{1}{c}{ 2002 } & \multicolumn{1}{c}{ 2014 } \\
      \midrule
	\multicolumn{4}{l}{\textcolor{FAOblue}{\textbf{\large{The setting}}}} \\ 
	 ~ Population, total (mln) & 0.8 ~ \ \ & 1 ~ \ \ & 1.2 ~ \ \ \\ 
	 ~ Population, rural (\% total population) & 0.3 ~ \ \ & 0.3 ~ \ \ & 0.3 ~ \ \ \\ 
	 ~ Govt expenditure on ag (\% total outlays) &  ~ \ \ &  ~ \ \ &  ~ \ \ \\ 
	 ~ Area harvested (mln ha) & 1 ~ \ \ & 1 ~ \ \ & 0 ~ \ \ \\ 
	 ~ Cropping intensity ratio (\%) & 3.8 ~ \ \ & 5.9 ~ \ \ &  ~ \ \ \\ 
	 ~ Water resources (m\textsuperscript{3}/person/year) & \textit{1} ~ \ \ & \textit{1} ~ \ \ & \textit{1} ~ \ \ \\ 
	 ~ Area equipped for irrigation (1000 ha) &  ~ \ \ &  ~ \ \ & \textit{46} ~ \ \ \\ 
	 ~ Area irrigated (\%) &  ~ \ \ & \textit{81.9} ~ \ \ & \textit{81.9} ~ \ \ \\ 
	 ~ Employment in agriculture (\%) & 12 ~ \ \ & 5.2 ~ \ \ & \textit{2.9} ~ \ \ \\ 
	 ~ Employment in agriculture, female (\%) & 13.1 ~ \ \ & 4.3 ~ \ \ & \textit{1.9} ~ \ \ \\ 
	 ~ Fertilizers, Nitrogen (nutrients per ha) &  ~ \ \ & 61.6 ~ \ \ & \textit{85.6} ~ \ \ \\ 
	 ~ Fertilizers, Phosphate (nutrients per ha) &  ~ \ \ & 40.6 ~ \ \ & \textit{43.5} ~ \ \ \\ 
	 ~ Fertilizers, Potash (nutrients per ha) &  ~ \ \ & 11.8 ~ \ \ & \textit{25.3} ~ \ \ \\ 
	 ~ Energy consump, power irrigation (mln kWh) & 67 ~ \ \ & 91 ~ \ \ & \textit{91} ~ \ \ \\ 
	 ~ Agr value added per worker (constant US\$) & 9.3 ~ \ \ & 13.1 ~ \ \ & \textit{11.4} ~ \ \ \\ 
	\multicolumn{4}{l}{\textcolor{FAOblue}{\textbf{\large{Hunger dimensions}}}} \\ 
	 ~ Dietary energy supply (kcal/pc/day) &  ~ \ \ &  ~ \ \ &  ~ \ \ \\ 
	 ~ Average dietary energy supply adequacy (\%) & 113 ~ \ \ & 105 ~ \ \ & 104 ~ \ \ \\ 
	 ~ Dietary en supp, cereals/roots/tubers (\%) & 30 ~ \ \ & 27 ~ \ \ & \textit{28} ~ \ \ \\ 
	 ~ Prevalence of undernourishment (\%) & <5.0 ~ \ \ & <5.0 ~ \ \ & <5.0 ~ \ \ \\ 
	 ~ GDP per capita (US\$, PPP) & 23\,214 ~ \ \ & 28\,942 ~ \ \ & \textit{27\,394} ~ \ \ \\ 
	 ~ Domestic food price volatility (index) &  ~ \ \ & 16.4 ~ \ \ & 12.7 ~ \ \ \\ 
	 ~ Cereal import dependency ratio (\%) & 77.3 ~ \ \ & 81.3 ~ \ \ & \textit{89.4} ~ \ \ \\ 
	 ~ Underweight, children under-5 (\%) &  ~ \ \ &  ~ \ \ &  ~ \ \ \\ 
	 ~ Improved water source (\% pop) & 100 ~ \ \ & 100 ~ \ \ & \textit{100} ~ \ \ \\ 
	\multicolumn{4}{l}{\textcolor{FAOblue}{\textbf{\large{Food Supply}}}} \\ 
	 ~ Food production value, (2004-2006 mln I\$) & 385 ~ \ \ & 435 ~ \ \ & \textit{328} ~ \ \ \\ 
	 ~ Agriculture, value added (\% GDP) & 6 ~ \ \ & 4 ~ \ \ & \textit{2} ~ \ \ \\ 
	 ~ Food exports (mln US\$)  & 142 ~ \ \ & 106 ~ \ \ & \textit{233} ~ \ \ \\ 
	 ~ Food imports (mln US\$)  & 214 ~ \ \ & 261 ~ \ \ & \textit{773} ~ \ \ \\ 
	\multicolumn{4}{l}{\textit{\normalsize{Production indices (2004-06=100)}}} \\ 
	 ~ Net food & 96 ~ \ \ & 108 ~ \ \ & \textit{81} ~ \ \ \\ 
	 ~ Net crop & 132 ~ \ \ & 109 ~ \ \ & \textit{68} ~ \ \ \\ 
	 ~ Cereal & 243 ~ \ \ & 188 ~ \ \ & \textit{101} ~ \ \ \\ 
	 ~ Vegetable oils & 92 ~ \ \ & 131 ~ \ \ & \textit{42} ~ \ \ \\ 
	 ~ Roots and tubers & 141 ~ \ \ & 107 ~ \ \ & \textit{81} ~ \ \ \\ 
	 ~ Fruit and vegetables & 127 ~ \ \ & 99 ~ \ \ & \textit{66} ~ \ \ \\ 
	 ~ Sugar &  ~ \ \ &  ~ \ \ &  ~ \ \ \\ 
	 ~ Livestock & 73 ~ \ \ & 109 ~ \ \ & \textit{90} ~ \ \ \\ 
	 ~ Milk & 75 ~ \ \ & 102 ~ \ \ & \textit{105} ~ \ \ \\ 
	 ~ Meat & 71 ~ \ \ & 110 ~ \ \ & \textit{84} ~ \ \ \\ 
	 ~ Fish  & 16 ~ \ \ & 241 ~ \ \ & \textit{11} ~ \ \ \\ 
	\multicolumn{4}{l}{\textit{\normalsize{Net trade (min US\$)}}} \\ 
	 ~ Cereals & -74 ~ \ \ & -104 ~ \ \ & \textit{-240} ~ \ \ \\ 
	 ~ Fruit and vegetables & 94 ~ \ \ & 32 ~ \ \ & \textit{-17} ~ \ \ \\ 
	 ~ Meat & -26 ~ \ \ & -11 ~ \ \ & \textit{-86} ~ \ \ \\ 
	 ~ Dairy products & -12 ~ \ \ & -5 ~ \ \ & \textit{7} ~ \ \ \\ 
	 ~ Fish & -30 ~ \ \ & -28 ~ \ \ & \textit{-70} ~ \ \ \\ 
	\multicolumn{4}{l}{\textcolor{FAOblue}{\textbf{\large{Environment}}}} \\ 
	 ~ Forest area (\%) & 18 ~ \ \ & 19 ~ \ \ & \textit{19} ~ \ \ \\ 
	 ~ Renewable water res withdrawn (\% of total) &  ~ \ \ &  ~ \ \ & 86 ~ \ \ \\ 
	 ~ Terrestrial protect areas (\% total land area)  & 10 ~ \ \ & 11 ~ \ \ & \textit{41} ~ \ \ \\ 
	 ~ Organic area (\% total agricultural area) &  ~ \ \ & \textit{1} ~ \ \ & \textit{3} ~ \ \ \\ 
	 ~ Water withdrawal by agriculture (\% of total) &  ~ \ \ &  ~ \ \ & 86 ~ \ \ \\ 
	 ~ Biofuel production (thousand kt of oil eq.) &  ~ \ \ & 0 ~ \ \ & \textit{162} ~ \ \ \\ 
	 ~ Wood pellet prod. (min tonnes) &  ~ \ \ &  ~ \ \ & \textit{0} ~ \ \ \\ 
	 ~ GHG emissions from ag (Co2 eq, gigagrams) & 0 ~ \ \ & 0 ~ \ \ & \textit{0} ~ \ \ \\ 
       \toprule
      \end{tabular}
      \clearpage
\CountryData{ Czech Republic }
      \rowcolors{1}{FAOblue!10}{white}
      \begin{tabular}{L{3.9cm} R{1cm} R{1cm} R{1cm}}
      \toprule
      \multicolumn{1}{c}{} & \multicolumn{1}{c}{ 1992 } & \multicolumn{1}{c}{ 2002 } & \multicolumn{1}{c}{ 2014 } \\
      \midrule
	\multicolumn{4}{l}{\textcolor{FAOblue}{\textbf{\large{The setting}}}} \\ 
	 ~ Population, total (mln) & \textit{10.3} ~ \ \ & 10.2 ~ \ \ & 10.7 ~ \ \ \\ 
	 ~ Population, rural (\% total population) & \textit{2.6} ~ \ \ & 2.7 ~ \ \ & 2.9 ~ \ \ \\ 
	 ~ Govt expenditure on ag (\% total outlays) &  ~ \ \ & 2.3 ~ \ \ & \textit{3.9} ~ \ \ \\ 
	 ~ Area harvested (mln ha) &  ~ \ \ & 7 ~ \ \ & 8 ~ \ \ \\ 
	 ~ Cropping intensity ratio (\%) &  ~ \ \ & 1.6 ~ \ \ &  ~ \ \ \\ 
	 ~ Water resources (m\textsuperscript{3}/person/year) & \textit{1} ~ \ \ & \textit{1} ~ \ \ & \textit{1} ~ \ \ \\ 
	 ~ Area equipped for irrigation (1000 ha) &  ~ \ \ &  ~ \ \ & \textit{32} ~ \ \ \\ 
	 ~ Area irrigated (\%) &  ~ \ \ &  ~ \ \ & \textit{51.7} ~ \ \ \\ 
	 ~ Employment in agriculture (\%) & \textit{6.6} ~ \ \ & 4.8 ~ \ \ & \textit{3.1} ~ \ \ \\ 
	 ~ Employment in agriculture, female (\%) & \textit{5.5} ~ \ \ & 3.4 ~ \ \ & \textit{1.9} ~ \ \ \\ 
	 ~ Fertilizers, Nitrogen (nutrients per ha) &  ~ \ \ & 44.9 ~ \ \ & \textit{55.9} ~ \ \ \\ 
	 ~ Fertilizers, Phosphate (nutrients per ha) &  ~ \ \ & 9.7 ~ \ \ & \textit{10.4} ~ \ \ \\ 
	 ~ Fertilizers, Potash (nutrients per ha) &  ~ \ \ & 7.2 ~ \ \ & \textit{12.8} ~ \ \ \\ 
	 ~ Energy consump, power irrigation (mln kWh) &  ~ \ \ &  ~ \ \ & \textit{31} ~ \ \ \\ 
	 ~ Agr value added per worker (constant US\$) & \textit{5.3} ~ \ \ & 5.9 ~ \ \ & \textit{7.2} ~ \ \ \\ 
	\multicolumn{4}{l}{\textcolor{FAOblue}{\textbf{\large{Hunger dimensions}}}} \\ 
	 ~ Dietary energy supply (kcal/pc/day) &  ~ \ \ &  ~ \ \ &  ~ \ \ \\ 
	 ~ Average dietary energy supply adequacy (\%) & 120 ~ \ \ & 126 ~ \ \ & 130 ~ \ \ \\ 
	 ~ Dietary en supp, cereals/roots/tubers (\%) & 29 ~ \ \ & 32 ~ \ \ & \textit{31} ~ \ \ \\ 
	 ~ Prevalence of undernourishment (\%) & <5.0 ~ \ \ & <5.0 ~ \ \ & <5.0 ~ \ \ \\ 
	 ~ GDP per capita (US\$, PPP) & 17\,470 ~ \ \ & 22\,126 ~ \ \ & \textit{28\,124} ~ \ \ \\ 
	 ~ Domestic food price volatility (index) &  ~ \ \ & 11.6 ~ \ \ & 10.7 ~ \ \ \\ 
	 ~ Cereal import dependency ratio (\%) & 2.3 ~ \ \ & -9.8 ~ \ \ & \textit{-44} ~ \ \ \\ 
	 ~ Underweight, children under-5 (\%) & \textit{0.9} ~ \ \ & \textit{2.1} ~ \ \ &  ~ \ \ \\ 
	 ~ Improved water source (\% pop) & 99.8 ~ \ \ & 99.8 ~ \ \ & \textit{99.8} ~ \ \ \\ 
	\multicolumn{4}{l}{\textcolor{FAOblue}{\textbf{\large{Food Supply}}}} \\ 
	 ~ Food production value, (2004-2006 mln I\$) & \textit{3\,961} ~ \ \ & 3\,628 ~ \ \ & \textit{3\,476} ~ \ \ \\ 
	 ~ Agriculture, value added (\% GDP) & \textit{4} ~ \ \ & 3 ~ \ \ & \textit{3} ~ \ \ \\ 
	 ~ Food exports (mln US\$)  & \textit{889} ~ \ \ & 872 ~ \ \ & \textit{5\,473} ~ \ \ \\ 
	 ~ Food imports (mln US\$)  & \textit{1\,072} ~ \ \ & 1\,349 ~ \ \ & \textit{6\,240} ~ \ \ \\ 
	\multicolumn{4}{l}{\textit{\normalsize{Production indices (2004-06=100)}}} \\ 
	 ~ Net food & \textit{105} ~ \ \ & 96 ~ \ \ & \textit{92} ~ \ \ \\ 
	 ~ Net crop & \textit{101} ~ \ \ & 94 ~ \ \ & \textit{102} ~ \ \ \\ 
	 ~ Cereal & \textit{85} ~ \ \ & 89 ~ \ \ & \textit{101} ~ \ \ \\ 
	 ~ Vegetable oils & \textit{71} ~ \ \ & 78 ~ \ \ & \textit{140} ~ \ \ \\ 
	 ~ Roots and tubers & \textit{143} ~ \ \ & 131 ~ \ \ & \textit{61} ~ \ \ \\ 
	 ~ Fruit and vegetables & \textit{177} ~ \ \ & 113 ~ \ \ & \textit{78} ~ \ \ \\ 
	 ~ Sugar & \textit{109} ~ \ \ & 113 ~ \ \ & \textit{110} ~ \ \ \\ 
	 ~ Livestock & \textit{113} ~ \ \ & 103 ~ \ \ & \textit{85} ~ \ \ \\ 
	 ~ Milk & \textit{114} ~ \ \ & 99 ~ \ \ & \textit{103} ~ \ \ \\ 
	 ~ Meat & \textit{115} ~ \ \ & 105 ~ \ \ & \textit{74} ~ \ \ \\ 
	 ~ Fish  & 0 ~ \ \ & 98 ~ \ \ & \textit{94} ~ \ \ \\ 
	\multicolumn{4}{l}{\textit{\normalsize{Net trade (min US\$)}}} \\ 
	 ~ Cereals & \textit{139} ~ \ \ & -28 ~ \ \ & \textit{531} ~ \ \ \\ 
	 ~ Fruit and vegetables & \textit{-344} ~ \ \ & -416 ~ \ \ & \textit{-1\,028} ~ \ \ \\ 
	 ~ Meat & \textit{-3} ~ \ \ & -27 ~ \ \ & \textit{-800} ~ \ \ \\ 
	 ~ Dairy products & \textit{171} ~ \ \ & 58 ~ \ \ & \textit{214} ~ \ \ \\ 
	 ~ Fish & \textit{-54} ~ \ \ & -55 ~ \ \ & \textit{-126} ~ \ \ \\ 
	\multicolumn{4}{l}{\textcolor{FAOblue}{\textbf{\large{Environment}}}} \\ 
	 ~ Forest area (\%) & \textit{34} ~ \ \ & 34 ~ \ \ & \textit{34} ~ \ \ \\ 
	 ~ Renewable water res withdrawn (\% of total) &  ~ \ \ &  ~ \ \ & 2 ~ \ \ \\ 
	 ~ Terrestrial protect areas (\% total land area)  & 15 ~ \ \ & 15 ~ \ \ & \textit{22} ~ \ \ \\ 
	 ~ Organic area (\% total agricultural area) &  ~ \ \ & \textit{6} ~ \ \ & \textit{11} ~ \ \ \\ 
	 ~ Water withdrawal by agriculture (\% of total) &  ~ \ \ &  ~ \ \ & 2 ~ \ \ \\ 
	 ~ Biofuel production (thousand kt of oil eq.) & 81 ~ \ \ & 2\,821 ~ \ \ & \textit{5\,382} ~ \ \ \\ 
	 ~ Wood pellet prod. (min tonnes) &  ~ \ \ &  ~ \ \ & \textit{165} ~ \ \ \\ 
	 ~ GHG emissions from ag (Co2 eq, gigagrams) & \textit{-5} ~ \ \ & -6 ~ \ \ & \textit{-6} ~ \ \ \\ 
       \toprule
      \end{tabular}
      \clearpage
\CountryData{ Côte d'Ivoire }
      \rowcolors{1}{FAOblue!10}{white}
      \begin{tabular}{L{3.9cm} R{1cm} R{1cm} R{1cm}}
      \toprule
      \multicolumn{1}{c}{} & \multicolumn{1}{c}{ 1992 } & \multicolumn{1}{c}{ 2002 } & \multicolumn{1}{c}{ 2014 } \\
      \midrule
	\multicolumn{4}{l}{\textcolor{FAOblue}{\textbf{\large{The setting}}}} \\ 
	 ~ Population, total (mln) & 13 ~ \ \ & 16.7 ~ \ \ & 20.8 ~ \ \ \\ 
	 ~ Population, rural (\% total population) & 7.8 ~ \ \ & 9.2 ~ \ \ & 9.7 ~ \ \ \\ 
	 ~ Govt expenditure on ag (\% total outlays) &  ~ \ \ &  ~ \ \ &  ~ \ \ \\ 
	 ~ Area harvested (mln ha) & 5 ~ \ \ & 7 ~ \ \ & 8 ~ \ \ \\ 
	 ~ Cropping intensity ratio (\%) & 0.3 ~ \ \ & 0.4 ~ \ \ &  ~ \ \ \\ 
	 ~ Water resources (m\textsuperscript{3}/person/year) & \textit{6} ~ \ \ & \textit{5} ~ \ \ & \textit{4} ~ \ \ \\ 
	 ~ Area equipped for irrigation (1000 ha) &  ~ \ \ &  ~ \ \ & \textit{73} ~ \ \ \\ 
	 ~ Area irrigated (\%) & \textit{92} ~ \ \ &  ~ \ \ &  ~ \ \ \\ 
	 ~ Employment in agriculture (\%) &  ~ \ \ &  ~ \ \ &  ~ \ \ \\ 
	 ~ Employment in agriculture, female (\%) &  ~ \ \ &  ~ \ \ &  ~ \ \ \\ 
	 ~ Fertilizers, Nitrogen (nutrients per ha) &  ~ \ \ & 1.5 ~ \ \ & \textit{1.6} ~ \ \ \\ 
	 ~ Fertilizers, Phosphate (nutrients per ha) &  ~ \ \ & 1.4 ~ \ \ & \textit{0.7} ~ \ \ \\ 
	 ~ Fertilizers, Potash (nutrients per ha) &  ~ \ \ & 1.5 ~ \ \ & \textit{1.2} ~ \ \ \\ 
	 ~ Energy consump, power irrigation (mln kWh) & \textit{87} ~ \ \ & 87 ~ \ \ & \textit{87} ~ \ \ \\ 
	 ~ Agr value added per worker (constant US\$) &  ~ \ \ & \textit{1.3} ~ \ \ & \textit{1.3} ~ \ \ \\ 
	\multicolumn{4}{l}{\textcolor{FAOblue}{\textbf{\large{Hunger dimensions}}}} \\ 
	 ~ Dietary energy supply (kcal/pc/day) & 2\,562 ~ \ \ & 2\,660 ~ \ \ & 2\,802 ~ \ \ \\ 
	 ~ Average dietary energy supply adequacy (\%) & 122 ~ \ \ & 125 ~ \ \ & 131 ~ \ \ \\ 
	 ~ Dietary en supp, cereals/roots/tubers (\%) & 67 ~ \ \ & 65 ~ \ \ & \textit{65} ~ \ \ \\ 
	 ~ Prevalence of undernourishment (\%) & 10.8 ~ \ \ & 16.7 ~ \ \ & 13.4 ~ \ \ \\ 
	 ~ GDP per capita (US\$, PPP) & 3\,019 ~ \ \ & 2\,907 ~ \ \ & \textit{3\,108} ~ \ \ \\ 
	 ~ Domestic food price volatility (index) &  ~ \ \ & 11.2 ~ \ \ & 8.8 ~ \ \ \\ 
	 ~ Cereal import dependency ratio (\%) & 39.4 ~ \ \ & 49.9 ~ \ \ & \textit{52.4} ~ \ \ \\ 
	 ~ Underweight, children under-5 (\%) & \textit{20.9} ~ \ \ & \textit{18.2} ~ \ \ & \textit{15.7} ~ \ \ \\ 
	 ~ Improved water source (\% pop) & 76.3 ~ \ \ & 77.9 ~ \ \ & \textit{80.2} ~ \ \ \\ 
	\multicolumn{4}{l}{\textcolor{FAOblue}{\textbf{\large{Food Supply}}}} \\ 
	 ~ Food production value, (2004-2006 mln I\$) & 3\,301 ~ \ \ & 4\,479 ~ \ \ & \textit{6\,009} ~ \ \ \\ 
	 ~ Agriculture, value added (\% GDP) & 34 ~ \ \ & 26 ~ \ \ & \textit{22} ~ \ \ \\ 
	 ~ Food exports (mln US\$)  & 1\,012 ~ \ \ & 2\,570 ~ \ \ & \textit{4\,193} ~ \ \ \\ 
	 ~ Food imports (mln US\$)  & 383 ~ \ \ & 403 ~ \ \ & \textit{1\,883} ~ \ \ \\ 
	\multicolumn{4}{l}{\textit{\normalsize{Production indices (2004-06=100)}}} \\ 
	 ~ Net food & 69 ~ \ \ & 94 ~ \ \ & \textit{126} ~ \ \ \\ 
	 ~ Net crop & 70 ~ \ \ & 95 ~ \ \ & \textit{124} ~ \ \ \\ 
	 ~ Cereal & 88 ~ \ \ & 94 ~ \ \ & \textit{224} ~ \ \ \\ 
	 ~ Vegetable oils & 99 ~ \ \ & 96 ~ \ \ & \textit{152} ~ \ \ \\ 
	 ~ Roots and tubers & 68 ~ \ \ & 91 ~ \ \ & \textit{110} ~ \ \ \\ 
	 ~ Fruit and vegetables & 82 ~ \ \ & 106 ~ \ \ & \textit{105} ~ \ \ \\ 
	 ~ Sugar & 94 ~ \ \ & 106 ~ \ \ & \textit{135} ~ \ \ \\ 
	 ~ Livestock & 85 ~ \ \ & 92 ~ \ \ & \textit{131} ~ \ \ \\ 
	 ~ Milk & 70 ~ \ \ & 94 ~ \ \ & \textit{109} ~ \ \ \\ 
	 ~ Meat & 88 ~ \ \ & 91 ~ \ \ & \textit{131} ~ \ \ \\ 
	 ~ Fish  & 170 ~ \ \ & 137 ~ \ \ & \textit{159} ~ \ \ \\ 
	\multicolumn{4}{l}{\textit{\normalsize{Net trade (min US\$)}}} \\ 
	 ~ Cereals & -163 ~ \ \ & -208 ~ \ \ & \textit{-1\,355} ~ \ \ \\ 
	 ~ Fruit and vegetables & 81 ~ \ \ & 154 ~ \ \ & \textit{457} ~ \ \ \\ 
	 ~ Meat & -33 ~ \ \ & -23 ~ \ \ & \textit{-76} ~ \ \ \\ 
	 ~ Dairy products & -55 ~ \ \ & -19 ~ \ \ & \textit{-89} ~ \ \ \\ 
	 ~ Fish & -17 ~ \ \ & -40 ~ \ \ & \textit{-242} ~ \ \ \\ 
	\multicolumn{4}{l}{\textcolor{FAOblue}{\textbf{\large{Environment}}}} \\ 
	 ~ Forest area (\%) & 32 ~ \ \ & 33 ~ \ \ & \textit{33} ~ \ \ \\ 
	 ~ Renewable water res withdrawn (\% of total) &  ~ \ \ & \textit{38} ~ \ \ & 38 ~ \ \ \\ 
	 ~ Terrestrial protect areas (\% total land area)  & 23 ~ \ \ & 23 ~ \ \ & \textit{23} ~ \ \ \\ 
	 ~ Organic area (\% total agricultural area) &  ~ \ \ &  ~ \ \ & \textit{0} ~ \ \ \\ 
	 ~ Water withdrawal by agriculture (\% of total) &  ~ \ \ & \textit{38} ~ \ \ & 38 ~ \ \ \\ 
	 ~ Biofuel production (thousand kt of oil eq.) & 4 ~ \ \ & 4 ~ \ \ & \textit{5} ~ \ \ \\ 
	 ~ Wood pellet prod. (min tonnes) &  ~ \ \ &  ~ \ \ &  ~ \ \ \\ 
	 ~ GHG emissions from ag (Co2 eq, gigagrams) & -1 ~ \ \ & -3 ~ \ \ & \textit{10} ~ \ \ \\ 
       \toprule
      \end{tabular}
      \clearpage
\CountryData{ DR Congo }
      \rowcolors{1}{FAOblue!10}{white}
      \begin{tabular}{L{3.9cm} R{1cm} R{1cm} R{1cm}}
      \toprule
      \multicolumn{1}{c}{} & \multicolumn{1}{c}{ 1992 } & \multicolumn{1}{c}{ 2002 } & \multicolumn{1}{c}{ 2014 } \\
      \midrule
	\multicolumn{4}{l}{\textcolor{FAOblue}{\textbf{\large{The setting}}}} \\ 
	 ~ Population, total (mln) & 37.7 ~ \ \ & 49.5 ~ \ \ & 69.4 ~ \ \ \\ 
	 ~ Population, rural (\% total population) & 27.2 ~ \ \ & 34.7 ~ \ \ & 44.4 ~ \ \ \\ 
	 ~ Govt expenditure on ag (\% total outlays) &  ~ \ \ &  ~ \ \ &  ~ \ \ \\ 
	 ~ Area harvested (mln ha) & 21 ~ \ \ & 16 ~ \ \ & 18 ~ \ \ \\ 
	 ~ Cropping intensity ratio (\%) & 0.8 ~ \ \ & 0.6 ~ \ \ &  ~ \ \ \\ 
	 ~ Water resources (m\textsuperscript{3}/person/year) & \textit{33} ~ \ \ & \textit{25} ~ \ \ & \textit{19} ~ \ \ \\ 
	 ~ Area equipped for irrigation (1000 ha) &  ~ \ \ &  ~ \ \ & \textit{11} ~ \ \ \\ 
	 ~ Area irrigated (\%) & \textit{76.2} ~ \ \ & \textit{76.2} ~ \ \ &  ~ \ \ \\ 
	 ~ Employment in agriculture (\%) &  ~ \ \ &  ~ \ \ &  ~ \ \ \\ 
	 ~ Employment in agriculture, female (\%) &  ~ \ \ &  ~ \ \ &  ~ \ \ \\ 
	 ~ Fertilizers, Nitrogen (nutrients per ha) &  ~ \ \ & 0 ~ \ \ & \textit{0.1} ~ \ \ \\ 
	 ~ Fertilizers, Phosphate (nutrients per ha) &  ~ \ \ & 0 ~ \ \ & \textit{0.1} ~ \ \ \\ 
	 ~ Fertilizers, Potash (nutrients per ha) &  ~ \ \ & 0 ~ \ \ & \textit{0.1} ~ \ \ \\ 
	 ~ Energy consump, power irrigation (mln kWh) & \textit{0} ~ \ \ & 0 ~ \ \ & \textit{0} ~ \ \ \\ 
	 ~ Agr value added per worker (constant US\$) & 0.3 ~ \ \ & 0.2 ~ \ \ & \textit{0.2} ~ \ \ \\ 
	\multicolumn{4}{l}{\textcolor{FAOblue}{\textbf{\large{Hunger dimensions}}}} \\ 
	 ~ Dietary energy supply (kcal/pc/day) &  ~ \ \ &  ~ \ \ &  ~ \ \ \\ 
	 ~ Average dietary energy supply adequacy (\%) &  ~ \ \ &  ~ \ \ &  ~ \ \ \\ 
	 ~ Dietary en supp, cereals/roots/tubers (\%) &  ~ \ \ &  ~ \ \ &  ~ \ \ \\ 
	 ~ Prevalence of undernourishment (\%) &  ~ \ \ &  ~ \ \ &  ~ \ \ \\ 
	 ~ GDP per capita (US\$, PPP) & 1\,043 ~ \ \ & 549 ~ \ \ & \textit{783} ~ \ \ \\ 
	 ~ Domestic food price volatility (index) &  ~ \ \ &  ~ \ \ &  ~ \ \ \\ 
	 ~ Cereal import dependency ratio (\%) & 18.9 ~ \ \ & 28.1 ~ \ \ & \textit{33.7} ~ \ \ \\ 
	 ~ Underweight, children under-5 (\%) & \textit{30.7} ~ \ \ & \textit{33.6} ~ \ \ & \textit{23.4} ~ \ \ \\ 
	 ~ Improved water source (\% pop) & 43.2 ~ \ \ & 44.3 ~ \ \ & \textit{46.5} ~ \ \ \\ 
	\multicolumn{4}{l}{\textcolor{FAOblue}{\textbf{\large{Food Supply}}}} \\ 
	 ~ Food production value, (2004-2006 mln I\$) & 4\,520 ~ \ \ & 3\,551 ~ \ \ & \textit{4\,236} ~ \ \ \\ 
	 ~ Agriculture, value added (\% GDP) & 49 ~ \ \ & 27 ~ \ \ & \textit{21} ~ \ \ \\ 
	 ~ Food exports (mln US\$)  & 5 ~ \ \ & 4 ~ \ \ & \textit{32} ~ \ \ \\ 
	 ~ Food imports (mln US\$)  & 192 ~ \ \ & 254 ~ \ \ & \textit{950} ~ \ \ \\ 
	\multicolumn{4}{l}{\textit{\normalsize{Production indices (2004-06=100)}}} \\ 
	 ~ Net food & 125 ~ \ \ & 98 ~ \ \ & \textit{117} ~ \ \ \\ 
	 ~ Net crop & 130 ~ \ \ & 98 ~ \ \ & \textit{116} ~ \ \ \\ 
	 ~ Cereal & 103 ~ \ \ & 99 ~ \ \ & \textit{117} ~ \ \ \\ 
	 ~ Vegetable oils & 118 ~ \ \ & 95 ~ \ \ & \textit{146} ~ \ \ \\ 
	 ~ Roots and tubers & 128 ~ \ \ & 99 ~ \ \ & \textit{112} ~ \ \ \\ 
	 ~ Fruit and vegetables & 140 ~ \ \ & 98 ~ \ \ & \textit{116} ~ \ \ \\ 
	 ~ Sugar & 112 ~ \ \ & 105 ~ \ \ & \textit{131} ~ \ \ \\ 
	 ~ Livestock & 100 ~ \ \ & 98 ~ \ \ & \textit{121} ~ \ \ \\ 
	 ~ Milk & 143 ~ \ \ & 89 ~ \ \ & \textit{170} ~ \ \ \\ 
	 ~ Meat & 100 ~ \ \ & 98 ~ \ \ & \textit{121} ~ \ \ \\ 
	 ~ Fish  & 79 ~ \ \ & 101 ~ \ \ & \textit{97} ~ \ \ \\ 
	\multicolumn{4}{l}{\textit{\normalsize{Net trade (min US\$)}}} \\ 
	 ~ Cereals & -110 ~ \ \ & -106 ~ \ \ & \textit{-418} ~ \ \ \\ 
	 ~ Fruit and vegetables & -10 ~ \ \ & -14 ~ \ \ & \textit{-63} ~ \ \ \\ 
	 ~ Meat & -50 ~ \ \ & -35 ~ \ \ & \textit{-173} ~ \ \ \\ 
	 ~ Dairy products & -8 ~ \ \ & -25 ~ \ \ & \textit{-49} ~ \ \ \\ 
	 ~ Fish & -39 ~ \ \ & -38 ~ \ \ & \textit{-175} ~ \ \ \\ 
	\multicolumn{4}{l}{\textcolor{FAOblue}{\textbf{\large{Environment}}}} \\ 
	 ~ Forest area (\%) & 70 ~ \ \ & 69 ~ \ \ & \textit{68} ~ \ \ \\ 
	 ~ Renewable water res withdrawn (\% of total) &  ~ \ \ & \textit{10} ~ \ \ & 10 ~ \ \ \\ 
	 ~ Terrestrial protect areas (\% total land area)  & 10 ~ \ \ & 10 ~ \ \ & \textit{12} ~ \ \ \\ 
	 ~ Organic area (\% total agricultural area) &  ~ \ \ &  ~ \ \ & \textit{0} ~ \ \ \\ 
	 ~ Water withdrawal by agriculture (\% of total) &  ~ \ \ & \textit{10} ~ \ \ & 10 ~ \ \ \\ 
	 ~ Biofuel production (thousand kt of oil eq.) & 2 ~ \ \ & \textit{2} ~ \ \ &  ~ \ \ \\ 
	 ~ Wood pellet prod. (min tonnes) &  ~ \ \ &  ~ \ \ &  ~ \ \ \\ 
	 ~ GHG emissions from ag (Co2 eq, gigagrams) & 187 ~ \ \ & 181 ~ \ \ & \textit{182} ~ \ \ \\ 
       \toprule
      \end{tabular}
      \clearpage
\CountryData{ Denmark }
      \rowcolors{1}{FAOblue!10}{white}
      \begin{tabular}{L{3.9cm} R{1cm} R{1cm} R{1cm}}
      \toprule
      \multicolumn{1}{c}{} & \multicolumn{1}{c}{ 1992 } & \multicolumn{1}{c}{ 2002 } & \multicolumn{1}{c}{ 2014 } \\
      \midrule
	\multicolumn{4}{l}{\textcolor{FAOblue}{\textbf{\large{The setting}}}} \\ 
	 ~ Population, total (mln) & 5.2 ~ \ \ & 5.4 ~ \ \ & 5.6 ~ \ \ \\ 
	 ~ Population, rural (\% total population) & 0.8 ~ \ \ & 0.8 ~ \ \ & 0.7 ~ \ \ \\ 
	 ~ Govt expenditure on ag (\% total outlays) &  ~ \ \ &  ~ \ \ &  ~ \ \ \\ 
	 ~ Area harvested (mln ha) & 7 ~ \ \ & 9 ~ \ \ & 9 ~ \ \ \\ 
	 ~ Cropping intensity ratio (\%) & 2.5 ~ \ \ & 3.3 ~ \ \ &  ~ \ \ \\ 
	 ~ Water resources (m\textsuperscript{3}/person/year) & \textit{1} ~ \ \ & \textit{1} ~ \ \ & \textit{1} ~ \ \ \\ 
	 ~ Area equipped for irrigation (1000 ha) &  ~ \ \ &  ~ \ \ & \textit{435} ~ \ \ \\ 
	 ~ Area irrigated (\%) &  ~ \ \ &  ~ \ \ & \textit{58.4} ~ \ \ \\ 
	 ~ Employment in agriculture (\%) & 5.1 ~ \ \ & 3 ~ \ \ & \textit{2.6} ~ \ \ \\ 
	 ~ Employment in agriculture, female (\%) & 2.7 ~ \ \ & 1.4 ~ \ \ & \textit{1.1} ~ \ \ \\ 
	 ~ Fertilizers, Nitrogen (nutrients per ha) &  ~ \ \ & 59.4 ~ \ \ & \textit{67.2} ~ \ \ \\ 
	 ~ Fertilizers, Phosphate (nutrients per ha) &  ~ \ \ & 2 ~ \ \ & \textit{12.7} ~ \ \ \\ 
	 ~ Fertilizers, Potash (nutrients per ha) &  ~ \ \ & 21.9 ~ \ \ & \textit{23.7} ~ \ \ \\ 
	 ~ Energy consump, power irrigation (mln kWh) &  ~ \ \ &  ~ \ \ &  ~ \ \ \\ 
	 ~ Agr value added per worker (constant US\$) & 13.7 ~ \ \ & 31.1 ~ \ \ & \textit{39.5} ~ \ \ \\ 
	\multicolumn{4}{l}{\textcolor{FAOblue}{\textbf{\large{Hunger dimensions}}}} \\ 
	 ~ Dietary energy supply (kcal/pc/day) &  ~ \ \ &  ~ \ \ &  ~ \ \ \\ 
	 ~ Average dietary energy supply adequacy (\%) & 129 ~ \ \ & 132 ~ \ \ & 132 ~ \ \ \\ 
	 ~ Dietary en supp, cereals/roots/tubers (\%) & 27 ~ \ \ & 28 ~ \ \ & \textit{29} ~ \ \ \\ 
	 ~ Prevalence of undernourishment (\%) & <5.0 ~ \ \ & <5.0 ~ \ \ & <5.0 ~ \ \ \\ 
	 ~ GDP per capita (US\$, PPP) & 34\,152 ~ \ \ & 41\,947 ~ \ \ & \textit{42\,483} ~ \ \ \\ 
	 ~ Domestic food price volatility (index) &  ~ \ \ & 5.4 ~ \ \ & 6 ~ \ \ \\ 
	 ~ Cereal import dependency ratio (\%) & -33.1 ~ \ \ & -11.6 ~ \ \ & \textit{-13.7} ~ \ \ \\ 
	 ~ Underweight, children under-5 (\%) &  ~ \ \ &  ~ \ \ &  ~ \ \ \\ 
	 ~ Improved water source (\% pop) & 100 ~ \ \ & 100 ~ \ \ & \textit{100} ~ \ \ \\ 
	\multicolumn{4}{l}{\textcolor{FAOblue}{\textbf{\large{Food Supply}}}} \\ 
	 ~ Food production value, (2004-2006 mln I\$) & 5\,311 ~ \ \ & 5\,794 ~ \ \ & \textit{5\,980} ~ \ \ \\ 
	 ~ Agriculture, value added (\% GDP) & 3 ~ \ \ & 2 ~ \ \ & \textit{1} ~ \ \ \\ 
	 ~ Food exports (mln US\$)  & 7\,322 ~ \ \ & 7\,420 ~ \ \ & \textit{13\,771} ~ \ \ \\ 
	 ~ Food imports (mln US\$)  & 1\,987 ~ \ \ & 3\,008 ~ \ \ & \textit{6\,842} ~ \ \ \\ 
	\multicolumn{4}{l}{\textit{\normalsize{Production indices (2004-06=100)}}} \\ 
	 ~ Net food & 91 ~ \ \ & 99 ~ \ \ & \textit{103} ~ \ \ \\ 
	 ~ Net crop & 91 ~ \ \ & 96 ~ \ \ & \textit{104} ~ \ \ \\ 
	 ~ Cereal & 76 ~ \ \ & 96 ~ \ \ & \textit{99} ~ \ \ \\ 
	 ~ Vegetable oils & 98 ~ \ \ & 52 ~ \ \ & \textit{166} ~ \ \ \\ 
	 ~ Roots and tubers & 115 ~ \ \ & 99 ~ \ \ & \textit{105} ~ \ \ \\ 
	 ~ Fruit and vegetables & 124 ~ \ \ & 83 ~ \ \ & \textit{108} ~ \ \ \\ 
	 ~ Sugar & 113 ~ \ \ & 128 ~ \ \ & \textit{87} ~ \ \ \\ 
	 ~ Livestock & 87 ~ \ \ & 99 ~ \ \ & \textit{99} ~ \ \ \\ 
	 ~ Milk & 96 ~ \ \ & 100 ~ \ \ & \textit{111} ~ \ \ \\ 
	 ~ Meat & 83 ~ \ \ & 99 ~ \ \ & \textit{94} ~ \ \ \\ 
	 ~ Fish  & 201 ~ \ \ & 148 ~ \ \ & \textit{70} ~ \ \ \\ 
	\multicolumn{4}{l}{\textit{\normalsize{Net trade (min US\$)}}} \\ 
	 ~ Cereals & 595 ~ \ \ & 408 ~ \ \ & \textit{477} ~ \ \ \\ 
	 ~ Fruit and vegetables & -315 ~ \ \ & -524 ~ \ \ & \textit{-1\,084} ~ \ \ \\ 
	 ~ Meat & 3\,609 ~ \ \ & 3\,035 ~ \ \ & \textit{3\,840} ~ \ \ \\ 
	 ~ Dairy products & 1\,147 ~ \ \ & 1\,048 ~ \ \ & \textit{1\,910} ~ \ \ \\ 
	 ~ Fish & 1\,120 ~ \ \ & 1\,067 ~ \ \ & \textit{1\,029} ~ \ \ \\ 
	\multicolumn{4}{l}{\textcolor{FAOblue}{\textbf{\large{Environment}}}} \\ 
	 ~ Forest area (\%) & 11 ~ \ \ & 12 ~ \ \ & \textit{13} ~ \ \ \\ 
	 ~ Renewable water res withdrawn (\% of total) &  ~ \ \ &  ~ \ \ & 36 ~ \ \ \\ 
	 ~ Terrestrial protect areas (\% total land area)  & 4 ~ \ \ & 5 ~ \ \ & \textit{18} ~ \ \ \\ 
	 ~ Organic area (\% total agricultural area) &  ~ \ \ & \textit{5} ~ \ \ & \textit{7} ~ \ \ \\ 
	 ~ Water withdrawal by agriculture (\% of total) &  ~ \ \ &  ~ \ \ & 36 ~ \ \ \\ 
	 ~ Biofuel production (thousand kt of oil eq.) & 15 ~ \ \ & 1\,112 ~ \ \ & \textit{2\,102} ~ \ \ \\ 
	 ~ Wood pellet prod. (min tonnes) &  ~ \ \ &  ~ \ \ & \textit{92} ~ \ \ \\ 
	 ~ GHG emissions from ag (Co2 eq, gigagrams) & 12 ~ \ \ & 4 ~ \ \ & \textit{11} ~ \ \ \\ 
       \toprule
      \end{tabular}
      \clearpage
\CountryData{ Djibouti }
      \rowcolors{1}{FAOblue!10}{white}
      \begin{tabular}{L{3.9cm} R{1cm} R{1cm} R{1cm}}
      \toprule
      \multicolumn{1}{c}{} & \multicolumn{1}{c}{ 1992 } & \multicolumn{1}{c}{ 2002 } & \multicolumn{1}{c}{ 2014 } \\
      \midrule
	\multicolumn{4}{l}{\textcolor{FAOblue}{\textbf{\large{The setting}}}} \\ 
	 ~ Population, total (mln) & 0.6 ~ \ \ & 0.7 ~ \ \ & 0.9 ~ \ \ \\ 
	 ~ Population, rural (\% total population) & 0.2 ~ \ \ & 0.2 ~ \ \ & 0.2 ~ \ \ \\ 
	 ~ Govt expenditure on ag (\% total outlays) &  ~ \ \ &  ~ \ \ &  ~ \ \ \\ 
	 ~ Area harvested (mln ha) & 0 ~ \ \ & 0 ~ \ \ & 0 ~ \ \ \\ 
	 ~ Cropping intensity ratio (\%) & 0 ~ \ \ & 0 ~ \ \ &  ~ \ \ \\ 
	 ~ Water resources (m\textsuperscript{3}/person/year) & \textit{0} ~ \ \ & \textit{0} ~ \ \ & \textit{0} ~ \ \ \\ 
	 ~ Area equipped for irrigation (1000 ha) &  ~ \ \ &  ~ \ \ & \textit{1} ~ \ \ \\ 
	 ~ Area irrigated (\%) &  ~ \ \ & \textit{38.3} ~ \ \ &  ~ \ \ \\ 
	 ~ Employment in agriculture (\%) &  ~ \ \ &  ~ \ \ &  ~ \ \ \\ 
	 ~ Employment in agriculture, female (\%) &  ~ \ \ &  ~ \ \ &  ~ \ \ \\ 
	 ~ Fertilizers, Nitrogen (nutrients per ha) &  ~ \ \ &  ~ \ \ &  ~ \ \ \\ 
	 ~ Fertilizers, Phosphate (nutrients per ha) &  ~ \ \ &  ~ \ \ &  ~ \ \ \\ 
	 ~ Fertilizers, Potash (nutrients per ha) &  ~ \ \ &  ~ \ \ &  ~ \ \ \\ 
	 ~ Energy consump, power irrigation (mln kWh) &  ~ \ \ &  ~ \ \ &  ~ \ \ \\ 
	 ~ Agr value added per worker (constant US\$) & 0.1 ~ \ \ & 0.1 ~ \ \ & \textit{0.1} ~ \ \ \\ 
	\multicolumn{4}{l}{\textcolor{FAOblue}{\textbf{\large{Hunger dimensions}}}} \\ 
	 ~ Dietary energy supply (kcal/pc/day) & 1\,529 ~ \ \ & 2\,044 ~ \ \ & 2\,662 ~ \ \ \\ 
	 ~ Average dietary energy supply adequacy (\%) & 69 ~ \ \ & 89 ~ \ \ & 114 ~ \ \ \\ 
	 ~ Dietary en supp, cereals/roots/tubers (\%) & 56 ~ \ \ & 58 ~ \ \ & \textit{56} ~ \ \ \\ 
	 ~ Prevalence of undernourishment (\%) & 76.8 ~ \ \ & 46.5 ~ \ \ & 17.3 ~ \ \ \\ 
	 ~ GDP per capita (US\$, PPP) & 2\,736 ~ \ \ & 2\,114 ~ \ \ & \textit{2\,902} ~ \ \ \\ 
	 ~ Domestic food price volatility (index) &  ~ \ \ &  ~ \ \ &  ~ \ \ \\ 
	 ~ Cereal import dependency ratio (\%) & 100 ~ \ \ & 100 ~ \ \ & \textit{100} ~ \ \ \\ 
	 ~ Underweight, children under-5 (\%) &  ~ \ \ & 25.4 ~ \ \ & \textit{29.8} ~ \ \ \\ 
	 ~ Improved water source (\% pop) & 76.7 ~ \ \ & 84.2 ~ \ \ & \textit{92.1} ~ \ \ \\ 
	\multicolumn{4}{l}{\textcolor{FAOblue}{\textbf{\large{Food Supply}}}} \\ 
	 ~ Food production value, (2004-2006 mln I\$) & 40 ~ \ \ & 53 ~ \ \ & \textit{73} ~ \ \ \\ 
	 ~ Agriculture, value added (\% GDP) & 3 ~ \ \ & 4 ~ \ \ & \textit{4} ~ \ \ \\ 
	 ~ Food exports (mln US\$)  & 4 ~ \ \ & 24 ~ \ \ & \textit{66} ~ \ \ \\ 
	 ~ Food imports (mln US\$)  & 46 ~ \ \ & 135 ~ \ \ & \textit{760} ~ \ \ \\ 
	\multicolumn{4}{l}{\textit{\normalsize{Production indices (2004-06=100)}}} \\ 
	 ~ Net food & 74 ~ \ \ & 98 ~ \ \ & \textit{134} ~ \ \ \\ 
	 ~ Net crop & 86 ~ \ \ & 109 ~ \ \ & \textit{137} ~ \ \ \\ 
	 ~ Cereal & 69 ~ \ \ & 100 ~ \ \ & \textit{161} ~ \ \ \\ 
	 ~ Vegetable oils &  ~ \ \ &  ~ \ \ &  ~ \ \ \\ 
	 ~ Roots and tubers &  ~ \ \ &  ~ \ \ &  ~ \ \ \\ 
	 ~ Fruit and vegetables & 81 ~ \ \ & 104 ~ \ \ & \textit{134} ~ \ \ \\ 
	 ~ Sugar & 88 ~ \ \ & 100 ~ \ \ & \textit{100} ~ \ \ \\ 
	 ~ Livestock & 72 ~ \ \ & 96 ~ \ \ & \textit{134} ~ \ \ \\ 
	 ~ Milk & 81 ~ \ \ & 82 ~ \ \ & \textit{107} ~ \ \ \\ 
	 ~ Meat & 71 ~ \ \ & 98 ~ \ \ & \textit{137} ~ \ \ \\ 
	 ~ Fish  &  ~ \ \ &  ~ \ \ &  ~ \ \ \\ 
	\multicolumn{4}{l}{\textit{\normalsize{Net trade (min US\$)}}} \\ 
	 ~ Cereals & -16 ~ \ \ & -50 ~ \ \ & \textit{-265} ~ \ \ \\ 
	 ~ Fruit and vegetables & -7 ~ \ \ & -14 ~ \ \ & \textit{-26} ~ \ \ \\ 
	 ~ Meat & -3 ~ \ \ & -1 ~ \ \ & \textit{-19} ~ \ \ \\ 
	 ~ Dairy products & -7 ~ \ \ & -19 ~ \ \ & \textit{-17} ~ \ \ \\ 
	 ~ Fish & -1 ~ \ \ & -1 ~ \ \ & \textit{-3} ~ \ \ \\ 
	\multicolumn{4}{l}{\textcolor{FAOblue}{\textbf{\large{Environment}}}} \\ 
	 ~ Forest area (\%) & 0 ~ \ \ & 0 ~ \ \ & \textit{0} ~ \ \ \\ 
	 ~ Renewable water res withdrawn (\% of total) &  ~ \ \ & \textit{16} ~ \ \ & 16 ~ \ \ \\ 
	 ~ Terrestrial protect areas (\% total land area)  & 0 ~ \ \ & 0 ~ \ \ & \textit{0} ~ \ \ \\ 
	 ~ Organic area (\% total agricultural area) &  ~ \ \ &  ~ \ \ &  ~ \ \ \\ 
	 ~ Water withdrawal by agriculture (\% of total) &  ~ \ \ & \textit{16} ~ \ \ & 16 ~ \ \ \\ 
	 ~ Biofuel production (thousand kt of oil eq.) &  ~ \ \ &  ~ \ \ &  ~ \ \ \\ 
	 ~ Wood pellet prod. (min tonnes) &  ~ \ \ &  ~ \ \ &  ~ \ \ \\ 
	 ~ GHG emissions from ag (Co2 eq, gigagrams) & 1 ~ \ \ & 1 ~ \ \ & \textit{1} ~ \ \ \\ 
       \toprule
      \end{tabular}
      \clearpage
\CountryData{ Dominican Republic }
      \rowcolors{1}{FAOblue!10}{white}
      \begin{tabular}{L{3.9cm} R{1cm} R{1cm} R{1cm}}
      \toprule
      \multicolumn{1}{c}{} & \multicolumn{1}{c}{ 1992 } & \multicolumn{1}{c}{ 2002 } & \multicolumn{1}{c}{ 2014 } \\
      \midrule
	\multicolumn{4}{l}{\textcolor{FAOblue}{\textbf{\large{The setting}}}} \\ 
	 ~ Population, total (mln) & 7.5 ~ \ \ & 8.9 ~ \ \ & 10.5 ~ \ \ \\ 
	 ~ Population, rural (\% total population) & 3.3 ~ \ \ & 3.3 ~ \ \ & 3 ~ \ \ \\ 
	 ~ Govt expenditure on ag (\% total outlays) &  ~ \ \ &  ~ \ \ &  ~ \ \ \\ 
	 ~ Area harvested (mln ha) & 7 ~ \ \ & 5 ~ \ \ & 5 ~ \ \ \\ 
	 ~ Cropping intensity ratio (\%) & 2.7 ~ \ \ & 2 ~ \ \ &  ~ \ \ \\ 
	 ~ Water resources (m\textsuperscript{3}/person/year) & \textit{3} ~ \ \ & \textit{2} ~ \ \ & \textit{2} ~ \ \ \\ 
	 ~ Area equipped for irrigation (1000 ha) &  ~ \ \ &  ~ \ \ & \textit{307} ~ \ \ \\ 
	 ~ Area irrigated (\%) &  ~ \ \ & \textit{71} ~ \ \ &  ~ \ \ \\ 
	 ~ Employment in agriculture (\%) & 18.7 ~ \ \ & 15.9 ~ \ \ & \textit{14.5} ~ \ \ \\ 
	 ~ Employment in agriculture, female (\%) & 2.7 ~ \ \ & 1.9 ~ \ \ & \textit{2.5} ~ \ \ \\ 
	 ~ Fertilizers, Nitrogen (nutrients per ha) &  ~ \ \ & 20.4 ~ \ \ & \textit{20.2} ~ \ \ \\ 
	 ~ Fertilizers, Phosphate (nutrients per ha) &  ~ \ \ & 1.2 ~ \ \ & \textit{6} ~ \ \ \\ 
	 ~ Fertilizers, Potash (nutrients per ha) &  ~ \ \ & 4.6 ~ \ \ & \textit{3.7} ~ \ \ \\ 
	 ~ Energy consump, power irrigation (mln kWh) &  ~ \ \ & 0 ~ \ \ & \textit{0} ~ \ \ \\ 
	 ~ Agr value added per worker (constant US\$) & 2.9 ~ \ \ & 4.2 ~ \ \ & \textit{7.8} ~ \ \ \\ 
	\multicolumn{4}{l}{\textcolor{FAOblue}{\textbf{\large{Hunger dimensions}}}} \\ 
	 ~ Dietary energy supply (kcal/pc/day) & 2\,168 ~ \ \ & 2\,228 ~ \ \ & 2\,609 ~ \ \ \\ 
	 ~ Average dietary energy supply adequacy (\%) & 96 ~ \ \ & 96 ~ \ \ & 111 ~ \ \ \\ 
	 ~ Dietary en supp, cereals/roots/tubers (\%) & 34 ~ \ \ & 30 ~ \ \ & \textit{32} ~ \ \ \\ 
	 ~ Prevalence of undernourishment (\%) & 32.1 ~ \ \ & 27.4 ~ \ \ & 12.5 ~ \ \ \\ 
	 ~ GDP per capita (US\$, PPP) & 5\,663 ~ \ \ & 8\,316 ~ \ \ & \textit{11\,796} ~ \ \ \\ 
	 ~ Domestic food price volatility (index) &  ~ \ \ & 4.9 ~ \ \ & 5.2 ~ \ \ \\ 
	 ~ Cereal import dependency ratio (\%) & 68.8 ~ \ \ & 73.3 ~ \ \ & \textit{73.9} ~ \ \ \\ 
	 ~ Underweight, children under-5 (\%) & \textit{8.4} ~ \ \ & 4.2 ~ \ \ & \textit{4} ~ \ \ \\ 
	 ~ Improved water source (\% pop) & 88.1 ~ \ \ & 85.1 ~ \ \ & \textit{80.9} ~ \ \ \\ 
	\multicolumn{4}{l}{\textcolor{FAOblue}{\textbf{\large{Food Supply}}}} \\ 
	 ~ Food production value, (2004-2006 mln I\$) & 1\,596 ~ \ \ & 2\,024 ~ \ \ & \textit{2\,891} ~ \ \ \\ 
	 ~ Agriculture, value added (\% GDP) & 13 ~ \ \ & 7 ~ \ \ & \textit{6} ~ \ \ \\ 
	 ~ Food exports (mln US\$)  & 257 ~ \ \ & 313 ~ \ \ & \textit{793} ~ \ \ \\ 
	 ~ Food imports (mln US\$)  & 265 ~ \ \ & 511 ~ \ \ & \textit{1\,457} ~ \ \ \\ 
	\multicolumn{4}{l}{\textit{\normalsize{Production indices (2004-06=100)}}} \\ 
	 ~ Net food & 75 ~ \ \ & 95 ~ \ \ & \textit{136} ~ \ \ \\ 
	 ~ Net crop & 85 ~ \ \ & 103 ~ \ \ & \textit{148} ~ \ \ \\ 
	 ~ Cereal & 91 ~ \ \ & 113 ~ \ \ & \textit{128} ~ \ \ \\ 
	 ~ Vegetable oils & 70 ~ \ \ & 93 ~ \ \ & \textit{174} ~ \ \ \\ 
	 ~ Roots and tubers & 89 ~ \ \ & 126 ~ \ \ & \textit{118} ~ \ \ \\ 
	 ~ Fruit and vegetables & 62 ~ \ \ & 99 ~ \ \ & \textit{174} ~ \ \ \\ 
	 ~ Sugar & 136 ~ \ \ & 102 ~ \ \ & \textit{94} ~ \ \ \\ 
	 ~ Livestock & 65 ~ \ \ & 84 ~ \ \ & \textit{114} ~ \ \ \\ 
	 ~ Milk & 81 ~ \ \ & 123 ~ \ \ & \textit{148} ~ \ \ \\ 
	 ~ Meat & 63 ~ \ \ & 76 ~ \ \ & \textit{107} ~ \ \ \\ 
	 ~ Fish  &  ~ \ \ &  ~ \ \ &  ~ \ \ \\ 
	\multicolumn{4}{l}{\textit{\normalsize{Net trade (min US\$)}}} \\ 
	 ~ Cereals & -120 ~ \ \ & -216 ~ \ \ & \textit{-695} ~ \ \ \\ 
	 ~ Fruit and vegetables & 34 ~ \ \ & 54 ~ \ \ & \textit{220} ~ \ \ \\ 
	 ~ Meat & 17 ~ \ \ & -12 ~ \ \ & \textit{-46} ~ \ \ \\ 
	 ~ Dairy products & -48 ~ \ \ & -28 ~ \ \ & \textit{-77} ~ \ \ \\ 
	 ~ Fish &  ~ \ \ &  ~ \ \ &  ~ \ \ \\ 
	\multicolumn{4}{l}{\textcolor{FAOblue}{\textbf{\large{Environment}}}} \\ 
	 ~ Forest area (\%) & 41 ~ \ \ & 41 ~ \ \ & \textit{41} ~ \ \ \\ 
	 ~ Renewable water res withdrawn (\% of total) &  ~ \ \ &  ~ \ \ & 80 ~ \ \ \\ 
	 ~ Terrestrial protect areas (\% total land area)  & 22 ~ \ \ & 22 ~ \ \ & \textit{19} ~ \ \ \\ 
	 ~ Organic area (\% total agricultural area) &  ~ \ \ & \textit{3} ~ \ \ & \textit{8} ~ \ \ \\ 
	 ~ Water withdrawal by agriculture (\% of total) &  ~ \ \ &  ~ \ \ & 80 ~ \ \ \\ 
	 ~ Biofuel production (thousand kt of oil eq.) & 15 ~ \ \ & 25 ~ \ \ & \textit{22} ~ \ \ \\ 
	 ~ Wood pellet prod. (min tonnes) &  ~ \ \ &  ~ \ \ &  ~ \ \ \\ 
	 ~ GHG emissions from ag (Co2 eq, gigagrams) & 6 ~ \ \ & 6 ~ \ \ & \textit{8} ~ \ \ \\ 
       \toprule
      \end{tabular}
      \clearpage
\CountryData{ Ecuador }
      \rowcolors{1}{FAOblue!10}{white}
      \begin{tabular}{L{3.9cm} R{1cm} R{1cm} R{1cm}}
      \toprule
      \multicolumn{1}{c}{} & \multicolumn{1}{c}{ 1992 } & \multicolumn{1}{c}{ 2002 } & \multicolumn{1}{c}{ 2014 } \\
      \midrule
	\multicolumn{4}{l}{\textcolor{FAOblue}{\textbf{\large{The setting}}}} \\ 
	 ~ Population, total (mln) & 10.6 ~ \ \ & 13 ~ \ \ & 16 ~ \ \ \\ 
	 ~ Population, rural (\% total population) & 4.6 ~ \ \ & 5 ~ \ \ & 4.9 ~ \ \ \\ 
	 ~ Govt expenditure on ag (\% total outlays) &  ~ \ \ &  ~ \ \ & \textit{68.6} ~ \ \ \\ 
	 ~ Area harvested (mln ha) & 7 ~ \ \ & 7 ~ \ \ & 7 ~ \ \ \\ 
	 ~ Cropping intensity ratio (\%) & 0.9 ~ \ \ & 0.9 ~ \ \ &  ~ \ \ \\ 
	 ~ Water resources (m\textsuperscript{3}/person/year) & \textit{42} ~ \ \ & \textit{34} ~ \ \ & \textit{29} ~ \ \ \\ 
	 ~ Area equipped for irrigation (1000 ha) &  ~ \ \ &  ~ \ \ & \textit{1\,500} ~ \ \ \\ 
	 ~ Area irrigated (\%) &  ~ \ \ &  ~ \ \ & \textit{62.8} ~ \ \ \\ 
	 ~ Employment in agriculture (\%) & 6.6 ~ \ \ & \textit{31.5} ~ \ \ & \textit{27.8} ~ \ \ \\ 
	 ~ Employment in agriculture, female (\%) & 1.9 ~ \ \ & \textit{26.7} ~ \ \ & \textit{21.2} ~ \ \ \\ 
	 ~ Fertilizers, Nitrogen (nutrients per ha) &  ~ \ \ & 17.7 ~ \ \ & \textit{22.2} ~ \ \ \\ 
	 ~ Fertilizers, Phosphate (nutrients per ha) &  ~ \ \ & 4.9 ~ \ \ & \textit{4.4} ~ \ \ \\ 
	 ~ Fertilizers, Potash (nutrients per ha) &  ~ \ \ & 7.9 ~ \ \ & \textit{11.2} ~ \ \ \\ 
	 ~ Energy consump, power irrigation (mln kWh) & 0 ~ \ \ & 411 ~ \ \ & \textit{411} ~ \ \ \\ 
	 ~ Agr value added per worker (constant US\$) & 2.1 ~ \ \ & 2.7 ~ \ \ & \textit{4.2} ~ \ \ \\ 
	\multicolumn{4}{l}{\textcolor{FAOblue}{\textbf{\large{Hunger dimensions}}}} \\ 
	 ~ Dietary energy supply (kcal/pc/day) & 2\,252 ~ \ \ & 2\,275 ~ \ \ & 2\,498 ~ \ \ \\ 
	 ~ Average dietary energy supply adequacy (\%) & 105 ~ \ \ & 104 ~ \ \ & 112 ~ \ \ \\ 
	 ~ Dietary en supp, cereals/roots/tubers (\%) & 39 ~ \ \ & 36 ~ \ \ & \textit{34} ~ \ \ \\ 
	 ~ Prevalence of undernourishment (\%) & 19.4 ~ \ \ & 19.3 ~ \ \ & 11.1 ~ \ \ \\ 
	 ~ GDP per capita (US\$, PPP) & 7\,672 ~ \ \ & 7\,753 ~ \ \ & \textit{10\,541} ~ \ \ \\ 
	 ~ Domestic food price volatility (index) &  ~ \ \ & 7.5 ~ \ \ & 5.7 ~ \ \ \\ 
	 ~ Cereal import dependency ratio (\%) & 21.5 ~ \ \ & 32.2 ~ \ \ & \textit{36.4} ~ \ \ \\ 
	 ~ Underweight, children under-5 (\%) &  ~ \ \ & \textit{6.2} ~ \ \ & \textit{6.4} ~ \ \ \\ 
	 ~ Improved water source (\% pop) & 75.1 ~ \ \ & 80.7 ~ \ \ & \textit{86.4} ~ \ \ \\ 
	\multicolumn{4}{l}{\textcolor{FAOblue}{\textbf{\large{Food Supply}}}} \\ 
	 ~ Food production value, (2004-2006 mln I\$) & 3\,863 ~ \ \ & 5\,591 ~ \ \ & \textit{6\,972} ~ \ \ \\ 
	 ~ Agriculture, value added (\% GDP) & 20 ~ \ \ & 12 ~ \ \ & \textit{9} ~ \ \ \\ 
	 ~ Food exports (mln US\$)  & 793 ~ \ \ & 1\,346 ~ \ \ & \textit{3\,422} ~ \ \ \\ 
	 ~ Food imports (mln US\$)  & 151 ~ \ \ & 425 ~ \ \ & \textit{1\,262} ~ \ \ \\ 
	\multicolumn{4}{l}{\textit{\normalsize{Production indices (2004-06=100)}}} \\ 
	 ~ Net food & 64 ~ \ \ & 93 ~ \ \ & \textit{115} ~ \ \ \\ 
	 ~ Net crop & 76 ~ \ \ & 92 ~ \ \ & \textit{104} ~ \ \ \\ 
	 ~ Cereal & 66 ~ \ \ & 87 ~ \ \ & \textit{116} ~ \ \ \\ 
	 ~ Vegetable oils & 63 ~ \ \ & 93 ~ \ \ & \textit{119} ~ \ \ \\ 
	 ~ Roots and tubers & 127 ~ \ \ & 129 ~ \ \ & \textit{89} ~ \ \ \\ 
	 ~ Fruit and vegetables & 69 ~ \ \ & 93 ~ \ \ & \textit{101} ~ \ \ \\ 
	 ~ Sugar & 104 ~ \ \ & 79 ~ \ \ & \textit{108} ~ \ \ \\ 
	 ~ Livestock & 55 ~ \ \ & 93 ~ \ \ & \textit{131} ~ \ \ \\ 
	 ~ Milk & 61 ~ \ \ & 93 ~ \ \ & \textit{129} ~ \ \ \\ 
	 ~ Meat & 48 ~ \ \ & 93 ~ \ \ & \textit{130} ~ \ \ \\ 
	 ~ Fish  & 60 ~ \ \ & 70 ~ \ \ & \textit{150} ~ \ \ \\ 
	\multicolumn{4}{l}{\textit{\normalsize{Net trade (min US\$)}}} \\ 
	 ~ Cereals & -65 ~ \ \ & -142 ~ \ \ & \textit{-487} ~ \ \ \\ 
	 ~ Fruit and vegetables & 693 ~ \ \ & 1\,033 ~ \ \ & \textit{2\,303} ~ \ \ \\ 
	 ~ Meat & 1 ~ \ \ & -3 ~ \ \ & \textit{-30} ~ \ \ \\ 
	 ~ Dairy products & 0 ~ \ \ & -6 ~ \ \ & \textit{28} ~ \ \ \\ 
	 ~ Fish & 611 ~ \ \ & 675 ~ \ \ & \textit{2\,687} ~ \ \ \\ 
	\multicolumn{4}{l}{\textcolor{FAOblue}{\textbf{\large{Environment}}}} \\ 
	 ~ Forest area (\%) & 48 ~ \ \ & 46 ~ \ \ & \textit{38} ~ \ \ \\ 
	 ~ Renewable water res withdrawn (\% of total) &  ~ \ \ & \textit{81} ~ \ \ & 81 ~ \ \ \\ 
	 ~ Terrestrial protect areas (\% total land area)  & 22 ~ \ \ & 25 ~ \ \ & \textit{24} ~ \ \ \\ 
	 ~ Organic area (\% total agricultural area) &  ~ \ \ &  ~ \ \ & \textit{1} ~ \ \ \\ 
	 ~ Water withdrawal by agriculture (\% of total) &  ~ \ \ & \textit{81} ~ \ \ & 81 ~ \ \ \\ 
	 ~ Biofuel production (thousand kt of oil eq.) & 8 ~ \ \ & 12 ~ \ \ & \textit{13} ~ \ \ \\ 
	 ~ Wood pellet prod. (min tonnes) &  ~ \ \ &  ~ \ \ &  ~ \ \ \\ 
	 ~ GHG emissions from ag (Co2 eq, gigagrams) & 91 ~ \ \ & 96 ~ \ \ & \textit{97} ~ \ \ \\ 
       \toprule
      \end{tabular}
      \clearpage
\CountryData{ Egypt }
      \rowcolors{1}{FAOblue!10}{white}
      \begin{tabular}{L{3.9cm} R{1cm} R{1cm} R{1cm}}
      \toprule
      \multicolumn{1}{c}{} & \multicolumn{1}{c}{ 1992 } & \multicolumn{1}{c}{ 2002 } & \multicolumn{1}{c}{ 2014 } \\
      \midrule
	\multicolumn{4}{l}{\textcolor{FAOblue}{\textbf{\large{The setting}}}} \\ 
	 ~ Population, total (mln) & 58.4 ~ \ \ & 68.3 ~ \ \ & 83.4 ~ \ \ \\ 
	 ~ Population, rural (\% total population) & 33.1 ~ \ \ & 39 ~ \ \ & 46.7 ~ \ \ \\ 
	 ~ Govt expenditure on ag (\% total outlays) &  ~ \ \ & 5.6 ~ \ \ & \textit{1.9} ~ \ \ \\ 
	 ~ Area harvested (mln ha) & 15 ~ \ \ & 20 ~ \ \ & 24 ~ \ \ \\ 
	 ~ Cropping intensity ratio (\%) & 5 ~ \ \ & 5.9 ~ \ \ &  ~ \ \ \\ 
	 ~ Water resources (m\textsuperscript{3}/person/year) & \textit{1} ~ \ \ & \textit{1} ~ \ \ & \textit{1} ~ \ \ \\ 
	 ~ Area equipped for irrigation (1000 ha) &  ~ \ \ &  ~ \ \ & \textit{3\,650} ~ \ \ \\ 
	 ~ Area irrigated (\%) &  ~ \ \ & 100 ~ \ \ &  ~ \ \ \\ 
	 ~ Employment in agriculture (\%) & 38.4 ~ \ \ & 27.5 ~ \ \ & \textit{29.2} ~ \ \ \\ 
	 ~ Employment in agriculture, female (\%) & 52.4 ~ \ \ & 27.6 ~ \ \ & \textit{43.3} ~ \ \ \\ 
	 ~ Fertilizers, Nitrogen (nutrients per ha) &  ~ \ \ & 312.5 ~ \ \ & \textit{333.9} ~ \ \ \\ 
	 ~ Fertilizers, Phosphate (nutrients per ha) &  ~ \ \ & 41.5 ~ \ \ & \textit{103.3} ~ \ \ \\ 
	 ~ Fertilizers, Potash (nutrients per ha) &  ~ \ \ & 16.9 ~ \ \ & \textit{8.9} ~ \ \ \\ 
	 ~ Energy consump, power irrigation (mln kWh) & 165 ~ \ \ & 948 ~ \ \ & \textit{948} ~ \ \ \\ 
	 ~ Agr value added per worker (constant US\$) & 1.3 ~ \ \ & 1.8 ~ \ \ & \textit{2.5} ~ \ \ \\ 
	\multicolumn{4}{l}{\textcolor{FAOblue}{\textbf{\large{Hunger dimensions}}}} \\ 
	 ~ Dietary energy supply (kcal/pc/day) & 3\,181 ~ \ \ & 3\,407 ~ \ \ & 3\,547 ~ \ \ \\ 
	 ~ Average dietary energy supply adequacy (\%) & 143 ~ \ \ & 148 ~ \ \ & 152 ~ \ \ \\ 
	 ~ Dietary en supp, cereals/roots/tubers (\%) & 68 ~ \ \ & 65 ~ \ \ & \textit{65} ~ \ \ \\ 
	 ~ Prevalence of undernourishment (\%) & <5.0 ~ \ \ & <5.0 ~ \ \ & <5.0 ~ \ \ \\ 
	 ~ GDP per capita (US\$, PPP) & 6\,137 ~ \ \ & 8\,017 ~ \ \ & \textit{10\,734} ~ \ \ \\ 
	 ~ Domestic food price volatility (index) &  ~ \ \ & 3.2 ~ \ \ & 9.8 ~ \ \ \\ 
	 ~ Cereal import dependency ratio (\%) & 34.9 ~ \ \ & 32.8 ~ \ \ & \textit{44.2} ~ \ \ \\ 
	 ~ Underweight, children under-5 (\%) & \textit{10.8} ~ \ \ & \textit{5.4} ~ \ \ & \textit{6.8} ~ \ \ \\ 
	 ~ Improved water source (\% pop) & 93.6 ~ \ \ & 96.8 ~ \ \ & \textit{99.3} ~ \ \ \\ 
	\multicolumn{4}{l}{\textcolor{FAOblue}{\textbf{\large{Food Supply}}}} \\ 
	 ~ Food production value, (2004-2006 mln I\$) & 10\,809 ~ \ \ & 15\,975 ~ \ \ & \textit{21\,720} ~ \ \ \\ 
	 ~ Agriculture, value added (\% GDP) & 17 ~ \ \ & 16 ~ \ \ & \textit{15} ~ \ \ \\ 
	 ~ Food exports (mln US\$)  & 301 ~ \ \ & 376 ~ \ \ & \textit{3\,523} ~ \ \ \\ 
	 ~ Food imports (mln US\$)  & 1\,964 ~ \ \ & 2\,697 ~ \ \ & \textit{13\,416} ~ \ \ \\ 
	\multicolumn{4}{l}{\textit{\normalsize{Production indices (2004-06=100)}}} \\ 
	 ~ Net food & 59 ~ \ \ & 87 ~ \ \ & \textit{119} ~ \ \ \\ 
	 ~ Net crop & 62 ~ \ \ & 89 ~ \ \ & \textit{114} ~ \ \ \\ 
	 ~ Cereal & 65 ~ \ \ & 93 ~ \ \ & \textit{101} ~ \ \ \\ 
	 ~ Vegetable oils & 51 ~ \ \ & 92 ~ \ \ & \textit{108} ~ \ \ \\ 
	 ~ Roots and tubers & 61 ~ \ \ & 71 ~ \ \ & \textit{168} ~ \ \ \\ 
	 ~ Fruit and vegetables & 56 ~ \ \ & 86 ~ \ \ & \textit{119} ~ \ \ \\ 
	 ~ Sugar & 61 ~ \ \ & 97 ~ \ \ & \textit{141} ~ \ \ \\ 
	 ~ Livestock & 56 ~ \ \ & 84 ~ \ \ & \textit{131} ~ \ \ \\ 
	 ~ Milk & 53 ~ \ \ & 87 ~ \ \ & \textit{121} ~ \ \ \\ 
	 ~ Meat & 57 ~ \ \ & 79 ~ \ \ & \textit{136} ~ \ \ \\ 
	 ~ Fish  & 38 ~ \ \ & 88 ~ \ \ & \textit{160} ~ \ \ \\ 
	\multicolumn{4}{l}{\textit{\normalsize{Net trade (min US\$)}}} \\ 
	 ~ Cereals & -885 ~ \ \ & -1\,329 ~ \ \ & \textit{-5\,791} ~ \ \ \\ 
	 ~ Fruit and vegetables & -26 ~ \ \ & -93 ~ \ \ & \textit{762} ~ \ \ \\ 
	 ~ Meat & -116 ~ \ \ & -236 ~ \ \ & \textit{-1\,154} ~ \ \ \\ 
	 ~ Dairy products & -153 ~ \ \ & -121 ~ \ \ & \textit{-400} ~ \ \ \\ 
	 ~ Fish & -70 ~ \ \ & -105 ~ \ \ & \textit{-764} ~ \ \ \\ 
	\multicolumn{4}{l}{\textcolor{FAOblue}{\textbf{\large{Environment}}}} \\ 
	 ~ Forest area (\%) & 0 ~ \ \ & 0 ~ \ \ & \textit{0} ~ \ \ \\ 
	 ~ Renewable water res withdrawn (\% of total) &  ~ \ \ & \textit{86} ~ \ \ & 86 ~ \ \ \\ 
	 ~ Terrestrial protect areas (\% total land area)  & 2 ~ \ \ & 5 ~ \ \ & \textit{11} ~ \ \ \\ 
	 ~ Organic area (\% total agricultural area) &  ~ \ \ & \textit{1} ~ \ \ & \textit{2} ~ \ \ \\ 
	 ~ Water withdrawal by agriculture (\% of total) &  ~ \ \ & \textit{86} ~ \ \ & 86 ~ \ \ \\ 
	 ~ Biofuel production (thousand kt of oil eq.) & 48 ~ \ \ & 56 ~ \ \ & \textit{58} ~ \ \ \\ 
	 ~ Wood pellet prod. (min tonnes) &  ~ \ \ &  ~ \ \ & \textit{20} ~ \ \ \\ 
	 ~ GHG emissions from ag (Co2 eq, gigagrams) & 19 ~ \ \ & 25 ~ \ \ & \textit{29} ~ \ \ \\ 
       \toprule
      \end{tabular}
      \clearpage
\CountryData{ El Salvador }
      \rowcolors{1}{FAOblue!10}{white}
      \begin{tabular}{L{3.9cm} R{1cm} R{1cm} R{1cm}}
      \toprule
      \multicolumn{1}{c}{} & \multicolumn{1}{c}{ 1992 } & \multicolumn{1}{c}{ 2002 } & \multicolumn{1}{c}{ 2014 } \\
      \midrule
	\multicolumn{4}{l}{\textcolor{FAOblue}{\textbf{\large{The setting}}}} \\ 
	 ~ Population, total (mln) & 5.5 ~ \ \ & 6 ~ \ \ & 6.4 ~ \ \ \\ 
	 ~ Population, rural (\% total population) & 2.7 ~ \ \ & 2.4 ~ \ \ & 2.2 ~ \ \ \\ 
	 ~ Govt expenditure on ag (\% total outlays) &  ~ \ \ & 1.9 ~ \ \ & \textit{1.8} ~ \ \ \\ 
	 ~ Area harvested (mln ha) & 4 ~ \ \ & 5 ~ \ \ & 7 ~ \ \ \\ 
	 ~ Cropping intensity ratio (\%) & 3.1 ~ \ \ & 2.9 ~ \ \ &  ~ \ \ \\ 
	 ~ Water resources (m\textsuperscript{3}/person/year) & \textit{5} ~ \ \ & \textit{4} ~ \ \ & \textit{4} ~ \ \ \\ 
	 ~ Area equipped for irrigation (1000 ha) &  ~ \ \ &  ~ \ \ & \textit{45} ~ \ \ \\ 
	 ~ Area irrigated (\%) &  ~ \ \ &  ~ \ \ & \textit{74.8} ~ \ \ \\ 
	 ~ Employment in agriculture (\%) & 35.8 ~ \ \ & 19.7 ~ \ \ & \textit{21} ~ \ \ \\ 
	 ~ Employment in agriculture, female (\%) & 14.8 ~ \ \ & 2.7 ~ \ \ & \textit{5} ~ \ \ \\ 
	 ~ Fertilizers, Nitrogen (nutrients per ha) &  ~ \ \ & 22.2 ~ \ \ & \textit{47} ~ \ \ \\ 
	 ~ Fertilizers, Phosphate (nutrients per ha) &  ~ \ \ & 9.4 ~ \ \ & \textit{17.9} ~ \ \ \\ 
	 ~ Fertilizers, Potash (nutrients per ha) &  ~ \ \ & 0.2 ~ \ \ & \textit{12.7} ~ \ \ \\ 
	 ~ Energy consump, power irrigation (mln kWh) & \textit{11} ~ \ \ & 11 ~ \ \ & \textit{11} ~ \ \ \\ 
	 ~ Agr value added per worker (constant US\$) & 2.2 ~ \ \ & 2.4 ~ \ \ & \textit{3.5} ~ \ \ \\ 
	\multicolumn{4}{l}{\textcolor{FAOblue}{\textbf{\large{Hunger dimensions}}}} \\ 
	 ~ Dietary energy supply (kcal/pc/day) & 2\,400 ~ \ \ & 2\,623 ~ \ \ & 2\,568 ~ \ \ \\ 
	 ~ Average dietary energy supply adequacy (\%) & 112 ~ \ \ & 120 ~ \ \ & 113 ~ \ \ \\ 
	 ~ Dietary en supp, cereals/roots/tubers (\%) & 57 ~ \ \ & 51 ~ \ \ & \textit{49} ~ \ \ \\ 
	 ~ Prevalence of undernourishment (\%) & 15.1 ~ \ \ & 9.3 ~ \ \ & 12.6 ~ \ \ \\ 
	 ~ GDP per capita (US\$, PPP) & 4\,811 ~ \ \ & 6\,469 ~ \ \ & \textit{7\,515} ~ \ \ \\ 
	 ~ Domestic food price volatility (index) &  ~ \ \ & 13.6 ~ \ \ & 3 ~ \ \ \\ 
	 ~ Cereal import dependency ratio (\%) & 27.9 ~ \ \ & 46.5 ~ \ \ & \textit{41.8} ~ \ \ \\ 
	 ~ Underweight, children under-5 (\%) & \textit{7.2} ~ \ \ & \textit{6.1} ~ \ \ & \textit{6.6} ~ \ \ \\ 
	 ~ Improved water source (\% pop) & 76.4 ~ \ \ & 84.9 ~ \ \ & \textit{90.1} ~ \ \ \\ 
	\multicolumn{4}{l}{\textcolor{FAOblue}{\textbf{\large{Food Supply}}}} \\ 
	 ~ Food production value, (2004-2006 mln I\$) & 719 ~ \ \ & 782 ~ \ \ & \textit{950} ~ \ \ \\ 
	 ~ Agriculture, value added (\% GDP) & 15 ~ \ \ & 9 ~ \ \ & \textit{11} ~ \ \ \\ 
	 ~ Food exports (mln US\$)  & 107 ~ \ \ & 237 ~ \ \ & \textit{679} ~ \ \ \\ 
	 ~ Food imports (mln US\$)  & 200 ~ \ \ & 578 ~ \ \ & \textit{1\,297} ~ \ \ \\ 
	\multicolumn{4}{l}{\textit{\normalsize{Production indices (2004-06=100)}}} \\ 
	 ~ Net food & 85 ~ \ \ & 93 ~ \ \ & \textit{113} ~ \ \ \\ 
	 ~ Net crop & 113 ~ \ \ & 93 ~ \ \ & \textit{113} ~ \ \ \\ 
	 ~ Cereal & 116 ~ \ \ & 92 ~ \ \ & \textit{118} ~ \ \ \\ 
	 ~ Vegetable oils & 170 ~ \ \ & 89 ~ \ \ & \textit{123} ~ \ \ \\ 
	 ~ Roots and tubers & 31 ~ \ \ & 171 ~ \ \ & \textit{141} ~ \ \ \\ 
	 ~ Fruit and vegetables & 86 ~ \ \ & 70 ~ \ \ & \textit{105} ~ \ \ \\ 
	 ~ Sugar & 92 ~ \ \ & 93 ~ \ \ & \textit{147} ~ \ \ \\ 
	 ~ Livestock & 67 ~ \ \ & 93 ~ \ \ & \textit{107} ~ \ \ \\ 
	 ~ Milk & 80 ~ \ \ & 93 ~ \ \ & \textit{113} ~ \ \ \\ 
	 ~ Meat & 55 ~ \ \ & 92 ~ \ \ & \textit{104} ~ \ \ \\ 
	 ~ Fish  & 28 ~ \ \ & 82 ~ \ \ & \textit{127} ~ \ \ \\ 
	\multicolumn{4}{l}{\textit{\normalsize{Net trade (min US\$)}}} \\ 
	 ~ Cereals & -49 ~ \ \ & -96 ~ \ \ & \textit{-208} ~ \ \ \\ 
	 ~ Fruit and vegetables & -17 ~ \ \ & -95 ~ \ \ & \textit{-107} ~ \ \ \\ 
	 ~ Meat & -9 ~ \ \ & -42 ~ \ \ & \textit{-115} ~ \ \ \\ 
	 ~ Dairy products & -32 ~ \ \ & -75 ~ \ \ & \textit{-135} ~ \ \ \\ 
	 ~ Fish & 19 ~ \ \ & 15 ~ \ \ & \textit{63} ~ \ \ \\ 
	\multicolumn{4}{l}{\textcolor{FAOblue}{\textbf{\large{Environment}}}} \\ 
	 ~ Forest area (\%) & 18 ~ \ \ & 16 ~ \ \ & \textit{13} ~ \ \ \\ 
	 ~ Renewable water res withdrawn (\% of total) &  ~ \ \ & \textit{68} ~ \ \ & 68 ~ \ \ \\ 
	 ~ Terrestrial protect areas (\% total land area)  & 0 ~ \ \ & 0 ~ \ \ & \textit{8} ~ \ \ \\ 
	 ~ Organic area (\% total agricultural area) &  ~ \ \ & \textit{0} ~ \ \ & \textit{0} ~ \ \ \\ 
	 ~ Water withdrawal by agriculture (\% of total) &  ~ \ \ & \textit{68} ~ \ \ & 68 ~ \ \ \\ 
	 ~ Biofuel production (thousand kt of oil eq.) & 10 ~ \ \ & 20 ~ \ \ & \textit{22} ~ \ \ \\ 
	 ~ Wood pellet prod. (min tonnes) &  ~ \ \ &  ~ \ \ &  ~ \ \ \\ 
	 ~ GHG emissions from ag (Co2 eq, gigagrams) & 4 ~ \ \ & 4 ~ \ \ & \textit{4} ~ \ \ \\ 
       \toprule
      \end{tabular}
      \clearpage
\CountryData{ Eritrea }
      \rowcolors{1}{FAOblue!10}{white}
      \begin{tabular}{L{3.9cm} R{1cm} R{1cm} R{1cm}}
      \toprule
      \multicolumn{1}{c}{} & \multicolumn{1}{c}{ 1992 } & \multicolumn{1}{c}{ 2002 } & \multicolumn{1}{c}{ 2014 } \\
      \midrule
	\multicolumn{4}{l}{\textcolor{FAOblue}{\textbf{\large{The setting}}}} \\ 
	 ~ Population, total (mln) & \textit{3.4} ~ \ \ & 4.3 ~ \ \ & 6.5 ~ \ \ \\ 
	 ~ Population, rural (\% total population) & \textit{2.8} ~ \ \ & 3.5 ~ \ \ & 5.1 ~ \ \ \\ 
	 ~ Govt expenditure on ag (\% total outlays) &  ~ \ \ &  ~ \ \ &  ~ \ \ \\ 
	 ~ Area harvested (mln ha) &  ~ \ \ & 0 ~ \ \ & 0 ~ \ \ \\ 
	 ~ Cropping intensity ratio (\%) &  ~ \ \ & 0 ~ \ \ &  ~ \ \ \\ 
	 ~ Water resources (m\textsuperscript{3}/person/year) & \textit{2} ~ \ \ & \textit{2} ~ \ \ & \textit{1} ~ \ \ \\ 
	 ~ Area equipped for irrigation (1000 ha) &  ~ \ \ &  ~ \ \ & \textit{21} ~ \ \ \\ 
	 ~ Area irrigated (\%) & \textit{62.5} ~ \ \ &  ~ \ \ &  ~ \ \ \\ 
	 ~ Employment in agriculture (\%) &  ~ \ \ &  ~ \ \ &  ~ \ \ \\ 
	 ~ Employment in agriculture, female (\%) &  ~ \ \ &  ~ \ \ &  ~ \ \ \\ 
	 ~ Fertilizers, Nitrogen (nutrients per ha) &  ~ \ \ & 0.4 ~ \ \ & \textit{0.1} ~ \ \ \\ 
	 ~ Fertilizers, Phosphate (nutrients per ha) &  ~ \ \ & 0.1 ~ \ \ & \textit{0} ~ \ \ \\ 
	 ~ Fertilizers, Potash (nutrients per ha) &  ~ \ \ & 0 ~ \ \ & \textit{0} ~ \ \ \\ 
	 ~ Energy consump, power irrigation (mln kWh) & \textit{0} ~ \ \ & 0 ~ \ \ & \textit{0} ~ \ \ \\ 
	 ~ Agr value added per worker (constant US\$) & \textit{0.2} ~ \ \ & 0.1 ~ \ \ & \textit{0.1} ~ \ \ \\ 
	\multicolumn{4}{l}{\textcolor{FAOblue}{\textbf{\large{Hunger dimensions}}}} \\ 
	 ~ Dietary energy supply (kcal/pc/day) &  ~ \ \ &  ~ \ \ &  ~ \ \ \\ 
	 ~ Average dietary energy supply adequacy (\%) &  ~ \ \ &  ~ \ \ &  ~ \ \ \\ 
	 ~ Dietary en supp, cereals/roots/tubers (\%) &  ~ \ \ &  ~ \ \ &  ~ \ \ \\ 
	 ~ Prevalence of undernourishment (\%) &  ~ \ \ &  ~ \ \ &  ~ \ \ \\ 
	 ~ GDP per capita (US\$, PPP) & 1\,040 ~ \ \ & 1\,491 ~ \ \ & \textit{1\,157} ~ \ \ \\ 
	 ~ Domestic food price volatility (index) &  ~ \ \ &  ~ \ \ &  ~ \ \ \\ 
	 ~ Cereal import dependency ratio (\%) & 58.4 ~ \ \ & 71.4 ~ \ \ & \textit{50.7} ~ \ \ \\ 
	 ~ Underweight, children under-5 (\%) & \textit{38.3} ~ \ \ & 34.5 ~ \ \ & \textit{38.8} ~ \ \ \\ 
	 ~ Improved water source (\% pop) & 42.7 ~ \ \ & 56.8 ~ \ \ & \textit{60.2} ~ \ \ \\ 
	\multicolumn{4}{l}{\textcolor{FAOblue}{\textbf{\large{Food Supply}}}} \\ 
	 ~ Food production value, (2004-2006 mln I\$) & \textit{171} ~ \ \ & 163 ~ \ \ & \textit{243} ~ \ \ \\ 
	 ~ Agriculture, value added (\% GDP) & 31 ~ \ \ & 15 ~ \ \ & \textit{15} ~ \ \ \\ 
	 ~ Food exports (mln US\$)  & \textit{4} ~ \ \ & 0 ~ \ \ & \textit{0} ~ \ \ \\ 
	 ~ Food imports (mln US\$)  & \textit{43} ~ \ \ & 58 ~ \ \ & \textit{97} ~ \ \ \\ 
	\multicolumn{4}{l}{\textit{\normalsize{Production indices (2004-06=100)}}} \\ 
	 ~ Net food & \textit{78} ~ \ \ & 74 ~ \ \ & \textit{111} ~ \ \ \\ 
	 ~ Net crop & \textit{87} ~ \ \ & 55 ~ \ \ & \textit{94} ~ \ \ \\ 
	 ~ Cereal & \textit{43} ~ \ \ & 17 ~ \ \ & \textit{103} ~ \ \ \\ 
	 ~ Vegetable oils & \textit{125} ~ \ \ & 41 ~ \ \ & \textit{35} ~ \ \ \\ 
	 ~ Roots and tubers & \textit{112} ~ \ \ & 88 ~ \ \ & \textit{60} ~ \ \ \\ 
	 ~ Fruit and vegetables & \textit{94} ~ \ \ & 82 ~ \ \ & \textit{142} ~ \ \ \\ 
	 ~ Sugar &  ~ \ \ &  ~ \ \ &  ~ \ \ \\ 
	 ~ Livestock & \textit{71} ~ \ \ & 89 ~ \ \ & \textit{123} ~ \ \ \\ 
	 ~ Milk & \textit{50} ~ \ \ & 82 ~ \ \ & \textit{114} ~ \ \ \\ 
	 ~ Meat & \textit{78} ~ \ \ & 93 ~ \ \ & \textit{129} ~ \ \ \\ 
	 ~ Fish  & 0 ~ \ \ & 117 ~ \ \ & \textit{59} ~ \ \ \\ 
	\multicolumn{4}{l}{\textit{\normalsize{Net trade (min US\$)}}} \\ 
	 ~ Cereals & \textit{-32} ~ \ \ & -39 ~ \ \ & \textit{-51} ~ \ \ \\ 
	 ~ Fruit and vegetables & \textit{-2} ~ \ \ & -8 ~ \ \ & \textit{-4} ~ \ \ \\ 
	 ~ Meat &  ~ \ \ &  ~ \ \ &  ~ \ \ \\ 
	 ~ Dairy products & \textit{-3} ~ \ \ & -2 ~ \ \ & \textit{0} ~ \ \ \\ 
	 ~ Fish & \textit{0} ~ \ \ & 1 ~ \ \ & \textit{0} ~ \ \ \\ 
	\multicolumn{4}{l}{\textcolor{FAOblue}{\textbf{\large{Environment}}}} \\ 
	 ~ Forest area (\%) & \textit{16} ~ \ \ & 16 ~ \ \ & \textit{15} ~ \ \ \\ 
	 ~ Renewable water res withdrawn (\% of total) &  ~ \ \ & \textit{94} ~ \ \ & 94 ~ \ \ \\ 
	 ~ Terrestrial protect areas (\% total land area)  & 5 ~ \ \ & 5 ~ \ \ & \textit{5} ~ \ \ \\ 
	 ~ Organic area (\% total agricultural area) &  ~ \ \ &  ~ \ \ &  ~ \ \ \\ 
	 ~ Water withdrawal by agriculture (\% of total) &  ~ \ \ & \textit{94} ~ \ \ & 94 ~ \ \ \\ 
	 ~ Biofuel production (thousand kt of oil eq.) & \textit{1} ~ \ \ & 1 ~ \ \ & \textit{1} ~ \ \ \\ 
	 ~ Wood pellet prod. (min tonnes) &  ~ \ \ &  ~ \ \ &  ~ \ \ \\ 
	 ~ GHG emissions from ag (Co2 eq, gigagrams) & \textit{3} ~ \ \ & 4 ~ \ \ & \textit{5} ~ \ \ \\ 
       \toprule
      \end{tabular}
      \clearpage
\CountryData{ Estonia }
      \rowcolors{1}{FAOblue!10}{white}
      \begin{tabular}{L{3.9cm} R{1cm} R{1cm} R{1cm}}
      \toprule
      \multicolumn{1}{c}{} & \multicolumn{1}{c}{ 1992 } & \multicolumn{1}{c}{ 2002 } & \multicolumn{1}{c}{ 2014 } \\
      \midrule
	\multicolumn{4}{l}{\textcolor{FAOblue}{\textbf{\large{The setting}}}} \\ 
	 ~ Population, total (mln) & 1.5 ~ \ \ & 1.3 ~ \ \ & 1.3 ~ \ \ \\ 
	 ~ Population, rural (\% total population) & 0.4 ~ \ \ & 0.4 ~ \ \ & 0.4 ~ \ \ \\ 
	 ~ Govt expenditure on ag (\% total outlays) &  ~ \ \ &  ~ \ \ &  ~ \ \ \\ 
	 ~ Area harvested (mln ha) & 1 ~ \ \ & 1 ~ \ \ & 1 ~ \ \ \\ 
	 ~ Cropping intensity ratio (\%) & 0.5 ~ \ \ & 0.8 ~ \ \ &  ~ \ \ \\ 
	 ~ Water resources (m\textsuperscript{3}/person/year) & \textit{9} ~ \ \ & \textit{10} ~ \ \ & \textit{10} ~ \ \ \\ 
	 ~ Area equipped for irrigation (1000 ha) &  ~ \ \ &  ~ \ \ & \textit{4} ~ \ \ \\ 
	 ~ Area irrigated (\%) &  ~ \ \ &  ~ \ \ & \textit{71.2} ~ \ \ \\ 
	 ~ Employment in agriculture (\%) & 18.1 ~ \ \ & 6.9 ~ \ \ & \textit{4.7} ~ \ \ \\ 
	 ~ Employment in agriculture, female (\%) & 13.2 ~ \ \ & 4.2 ~ \ \ & \textit{2.6} ~ \ \ \\ 
	 ~ Fertilizers, Nitrogen (nutrients per ha) &  ~ \ \ & 23.9 ~ \ \ & \textit{34.5} ~ \ \ \\ 
	 ~ Fertilizers, Phosphate (nutrients per ha) &  ~ \ \ & 5.8 ~ \ \ & \textit{7.1} ~ \ \ \\ 
	 ~ Fertilizers, Potash (nutrients per ha) &  ~ \ \ & 9 ~ \ \ & \textit{11} ~ \ \ \\ 
	 ~ Energy consump, power irrigation (mln kWh) & \textit{7} ~ \ \ & 7 ~ \ \ & \textit{7} ~ \ \ \\ 
	 ~ Agr value added per worker (constant US\$) & \textit{3.3} ~ \ \ & 6.1 ~ \ \ & \textit{12.5} ~ \ \ \\ 
	\multicolumn{4}{l}{\textcolor{FAOblue}{\textbf{\large{Hunger dimensions}}}} \\ 
	 ~ Dietary energy supply (kcal/pc/day) &  ~ \ \ &  ~ \ \ &  ~ \ \ \\ 
	 ~ Average dietary energy supply adequacy (\%) & 105 ~ \ \ & 121 ~ \ \ & 131 ~ \ \ \\ 
	 ~ Dietary en supp, cereals/roots/tubers (\%) & 41 ~ \ \ & 35 ~ \ \ & \textit{37} ~ \ \ \\ 
	 ~ Prevalence of undernourishment (\%) & <5.0 ~ \ \ & <5.0 ~ \ \ & <5.0 ~ \ \ \\ 
	 ~ GDP per capita (US\$, PPP) & \textit{10\,462} ~ \ \ & 16\,972 ~ \ \ & \textit{25\,254} ~ \ \ \\ 
	 ~ Domestic food price volatility (index) &  ~ \ \ & 7.3 ~ \ \ & 7.4 ~ \ \ \\ 
	 ~ Cereal import dependency ratio (\%) & 5.4 ~ \ \ & 26.1 ~ \ \ & \textit{-19.8} ~ \ \ \\ 
	 ~ Underweight, children under-5 (\%) &  ~ \ \ &  ~ \ \ &  ~ \ \ \\ 
	 ~ Improved water source (\% pop) & 99.2 ~ \ \ & 99.1 ~ \ \ & \textit{99.1} ~ \ \ \\ 
	\multicolumn{4}{l}{\textcolor{FAOblue}{\textbf{\large{Food Supply}}}} \\ 
	 ~ Food production value, (2004-2006 mln I\$) & 638 ~ \ \ & 396 ~ \ \ & \textit{589} ~ \ \ \\ 
	 ~ Agriculture, value added (\% GDP) & \textit{6} ~ \ \ & 4 ~ \ \ & \textit{4} ~ \ \ \\ 
	 ~ Food exports (mln US\$)  & 49 ~ \ \ & 328 ~ \ \ & \textit{977} ~ \ \ \\ 
	 ~ Food imports (mln US\$)  & 32 ~ \ \ & 432 ~ \ \ & \textit{959} ~ \ \ \\ 
	\multicolumn{4}{l}{\textit{\normalsize{Production indices (2004-06=100)}}} \\ 
	 ~ Net food & 145 ~ \ \ & 90 ~ \ \ & \textit{134} ~ \ \ \\ 
	 ~ Net crop & 131 ~ \ \ & 93 ~ \ \ & \textit{148} ~ \ \ \\ 
	 ~ Cereal & 82 ~ \ \ & 76 ~ \ \ & \textit{155} ~ \ \ \\ 
	 ~ Vegetable oils & 3 ~ \ \ & 81 ~ \ \ & \textit{221} ~ \ \ \\ 
	 ~ Roots and tubers & 376 ~ \ \ & 120 ~ \ \ & \textit{77} ~ \ \ \\ 
	 ~ Fruit and vegetables & 170 ~ \ \ & 135 ~ \ \ & \textit{104} ~ \ \ \\ 
	 ~ Sugar &  ~ \ \ &  ~ \ \ &  ~ \ \ \\ 
	 ~ Livestock & 157 ~ \ \ & 96 ~ \ \ & \textit{120} ~ \ \ \\ 
	 ~ Milk & 137 ~ \ \ & 91 ~ \ \ & \textit{115} ~ \ \ \\ 
	 ~ Meat & 184 ~ \ \ & 102 ~ \ \ & \textit{132} ~ \ \ \\ 
	 ~ Fish  & 142 ~ \ \ & 111 ~ \ \ & \textit{77} ~ \ \ \\ 
	\multicolumn{4}{l}{\textit{\normalsize{Net trade (min US\$)}}} \\ 
	 ~ Cereals & -11 ~ \ \ & -38 ~ \ \ & \textit{72} ~ \ \ \\ 
	 ~ Fruit and vegetables & 1 ~ \ \ & -62 ~ \ \ & \textit{-170} ~ \ \ \\ 
	 ~ Meat & 4 ~ \ \ & -14 ~ \ \ & \textit{-45} ~ \ \ \\ 
	 ~ Dairy products & 29 ~ \ \ & 47 ~ \ \ & \textit{149} ~ \ \ \\ 
	 ~ Fish & 12 ~ \ \ & 68 ~ \ \ & \textit{81} ~ \ \ \\ 
	\multicolumn{4}{l}{\textcolor{FAOblue}{\textbf{\large{Environment}}}} \\ 
	 ~ Forest area (\%) & 50 ~ \ \ & 53 ~ \ \ & \textit{52} ~ \ \ \\ 
	 ~ Renewable water res withdrawn (\% of total) &  ~ \ \ &  ~ \ \ & 0 ~ \ \ \\ 
	 ~ Terrestrial protect areas (\% total land area)  & 18 ~ \ \ & 20 ~ \ \ & \textit{21} ~ \ \ \\ 
	 ~ Organic area (\% total agricultural area) &  ~ \ \ & \textit{7} ~ \ \ & \textit{15} ~ \ \ \\ 
	 ~ Water withdrawal by agriculture (\% of total) &  ~ \ \ &  ~ \ \ & 0 ~ \ \ \\ 
	 ~ Biofuel production (thousand kt of oil eq.) & \textit{0} ~ \ \ & 11 ~ \ \ & \textit{8} ~ \ \ \\ 
	 ~ Wood pellet prod. (min tonnes) &  ~ \ \ &  ~ \ \ & \textit{590} ~ \ \ \\ 
	 ~ GHG emissions from ag (Co2 eq, gigagrams) & 9 ~ \ \ & 9 ~ \ \ & \textit{12} ~ \ \ \\ 
       \toprule
      \end{tabular}
      \clearpage
\CountryData{ Ethiopia }
      \rowcolors{1}{FAOblue!10}{white}
      \begin{tabular}{L{3.9cm} R{1cm} R{1cm} R{1cm}}
      \toprule
      \multicolumn{1}{c}{} & \multicolumn{1}{c}{ 1992 } & \multicolumn{1}{c}{ 2002 } & \multicolumn{1}{c}{ 2014 } \\
      \midrule
	\multicolumn{4}{l}{\textcolor{FAOblue}{\textbf{\large{The setting}}}} \\ 
	 ~ Population, total (mln) & \textit{57} ~ \ \ & 69.9 ~ \ \ & 96.5 ~ \ \ \\ 
	 ~ Population, rural (\% total population) & \textit{49.1} ~ \ \ & 59.4 ~ \ \ & 79.3 ~ \ \ \\ 
	 ~ Govt expenditure on ag (\% total outlays) &  ~ \ \ & 7.4 ~ \ \ & \textit{17.5} ~ \ \ \\ 
	 ~ Area harvested (mln ha) &  ~ \ \ & 9 ~ \ \ & 23 ~ \ \ \\ 
	 ~ Cropping intensity ratio (\%) &  ~ \ \ & 0.3 ~ \ \ &  ~ \ \ \\ 
	 ~ Water resources (m\textsuperscript{3}/person/year) & \textit{2} ~ \ \ & \textit{2} ~ \ \ & \textit{1} ~ \ \ \\ 
	 ~ Area equipped for irrigation (1000 ha) &  ~ \ \ &  ~ \ \ & \textit{290} ~ \ \ \\ 
	 ~ Area irrigated (\%) &  ~ \ \ &  ~ \ \ &  ~ \ \ \\ 
	 ~ Employment in agriculture (\%) & \textit{89.3} ~ \ \ & \textit{79.3} ~ \ \ & \textit{79.3} ~ \ \ \\ 
	 ~ Employment in agriculture, female (\%) &  ~ \ \ & \textit{74.8} ~ \ \ & \textit{74.8} ~ \ \ \\ 
	 ~ Fertilizers, Nitrogen (nutrients per ha) &  ~ \ \ & 3.2 ~ \ \ & \textit{4.9} ~ \ \ \\ 
	 ~ Fertilizers, Phosphate (nutrients per ha) &  ~ \ \ & 2.3 ~ \ \ & \textit{5.1} ~ \ \ \\ 
	 ~ Fertilizers, Potash (nutrients per ha) &  ~ \ \ & 0 ~ \ \ & \textit{0} ~ \ \ \\ 
	 ~ Energy consump, power irrigation (mln kWh) &  ~ \ \ & 15 ~ \ \ & \textit{15} ~ \ \ \\ 
	 ~ Agr value added per worker (constant US\$) & \textit{0.2} ~ \ \ & 0.2 ~ \ \ & \textit{0.3} ~ \ \ \\ 
	\multicolumn{4}{l}{\textcolor{FAOblue}{\textbf{\large{Hunger dimensions}}}} \\ 
	 ~ Dietary energy supply (kcal/pc/day) & 1\,508 ~ \ \ & 1\,853 ~ \ \ & 2\,164 ~ \ \ \\ 
	 ~ Average dietary energy supply adequacy (\%) & 71 ~ \ \ & 87 ~ \ \ & 98 ~ \ \ \\ 
	 ~ Dietary en supp, cereals/roots/tubers (\%) & 82 ~ \ \ & 81 ~ \ \ & \textit{76} ~ \ \ \\ 
	 ~ Prevalence of undernourishment (\%) & 75.2 ~ \ \ & 51.9 ~ \ \ & 33.1 ~ \ \ \\ 
	 ~ GDP per capita (US\$, PPP) & 516 ~ \ \ & 647 ~ \ \ & \textit{1\,336} ~ \ \ \\ 
	 ~ Domestic food price volatility (index) &  ~ \ \ & 8.3 ~ \ \ & 9 ~ \ \ \\ 
	 ~ Cereal import dependency ratio (\%) & 7.9 ~ \ \ & 12.1 ~ \ \ & \textit{10.7} ~ \ \ \\ 
	 ~ Underweight, children under-5 (\%) & 43.3 ~ \ \ & \textit{34.6} ~ \ \ & 25.2 ~ \ \ \\ 
	 ~ Improved water source (\% pop) & 13.6 ~ \ \ & 32.7 ~ \ \ & \textit{51.5} ~ \ \ \\ 
	\multicolumn{4}{l}{\textcolor{FAOblue}{\textbf{\large{Food Supply}}}} \\ 
	 ~ Food production value, (2004-2006 mln I\$) & \textit{3\,979} ~ \ \ & 5\,918 ~ \ \ & \textit{10\,294} ~ \ \ \\ 
	 ~ Agriculture, value added (\% GDP) & 66 ~ \ \ & 42 ~ \ \ & \textit{45} ~ \ \ \\ 
	 ~ Food exports (mln US\$)  & \textit{37} ~ \ \ & 125 ~ \ \ & \textit{1\,264} ~ \ \ \\ 
	 ~ Food imports (mln US\$)  & \textit{209} ~ \ \ & 191 ~ \ \ & \textit{1\,393} ~ \ \ \\ 
	\multicolumn{4}{l}{\textit{\normalsize{Production indices (2004-06=100)}}} \\ 
	 ~ Net food & \textit{57} ~ \ \ & 85 ~ \ \ & \textit{147} ~ \ \ \\ 
	 ~ Net crop & \textit{61} ~ \ \ & 79 ~ \ \ & \textit{157} ~ \ \ \\ 
	 ~ Cereal & \textit{55} ~ \ \ & 74 ~ \ \ & \textit{196} ~ \ \ \\ 
	 ~ Vegetable oils & \textit{36} ~ \ \ & 45 ~ \ \ & \textit{161} ~ \ \ \\ 
	 ~ Roots and tubers & \textit{69} ~ \ \ & 88 ~ \ \ & \textit{146} ~ \ \ \\ 
	 ~ Fruit and vegetables & \textit{46} ~ \ \ & 71 ~ \ \ & \textit{142} ~ \ \ \\ 
	 ~ Sugar & \textit{46} ~ \ \ & 85 ~ \ \ & \textit{105} ~ \ \ \\ 
	 ~ Livestock & \textit{55} ~ \ \ & 96 ~ \ \ & \textit{130} ~ \ \ \\ 
	 ~ Milk & \textit{36} ~ \ \ & 105 ~ \ \ & \textit{170} ~ \ \ \\ 
	 ~ Meat & \textit{62} ~ \ \ & 91 ~ \ \ & \textit{112} ~ \ \ \\ 
	 ~ Fish  & 47 ~ \ \ & 126 ~ \ \ & \textit{392} ~ \ \ \\ 
	\multicolumn{4}{l}{\textit{\normalsize{Net trade (min US\$)}}} \\ 
	 ~ Cereals & \textit{-154} ~ \ \ & -129 ~ \ \ & \textit{-700} ~ \ \ \\ 
	 ~ Fruit and vegetables & \textit{13} ~ \ \ & 36 ~ \ \ & \textit{435} ~ \ \ \\ 
	 ~ Meat & \textit{1} ~ \ \ & 2 ~ \ \ & \textit{74} ~ \ \ \\ 
	 ~ Dairy products & \textit{-2} ~ \ \ & -3 ~ \ \ & \textit{-10} ~ \ \ \\ 
	 ~ Fish & 0 ~ \ \ & 0 ~ \ \ & \textit{-2} ~ \ \ \\ 
	\multicolumn{4}{l}{\textcolor{FAOblue}{\textbf{\large{Environment}}}} \\ 
	 ~ Forest area (\%) & \textit{14} ~ \ \ & 13 ~ \ \ & \textit{12} ~ \ \ \\ 
	 ~ Renewable water res withdrawn (\% of total) &  ~ \ \ & 94 ~ \ \ & 94 ~ \ \ \\ 
	 ~ Terrestrial protect areas (\% total land area)  & 18 ~ \ \ & 18 ~ \ \ & \textit{18} ~ \ \ \\ 
	 ~ Organic area (\% total agricultural area) &  ~ \ \ &  ~ \ \ & \textit{0} ~ \ \ \\ 
	 ~ Water withdrawal by agriculture (\% of total) &  ~ \ \ & 94 ~ \ \ & 94 ~ \ \ \\ 
	 ~ Biofuel production (thousand kt of oil eq.) & \textit{61} ~ \ \ & 76 ~ \ \ & \textit{137} ~ \ \ \\ 
	 ~ Wood pellet prod. (min tonnes) &  ~ \ \ &  ~ \ \ &  ~ \ \ \\ 
	 ~ GHG emissions from ag (Co2 eq, gigagrams) & \textit{76} ~ \ \ & 91 ~ \ \ & \textit{115} ~ \ \ \\ 
       \toprule
      \end{tabular}
      \clearpage
\CountryData{ Fiji }
      \rowcolors{1}{FAOblue!10}{white}
      \begin{tabular}{L{3.9cm} R{1cm} R{1cm} R{1cm}}
      \toprule
      \multicolumn{1}{c}{} & \multicolumn{1}{c}{ 1992 } & \multicolumn{1}{c}{ 2002 } & \multicolumn{1}{c}{ 2014 } \\
      \midrule
	\multicolumn{4}{l}{\textcolor{FAOblue}{\textbf{\large{The setting}}}} \\ 
	 ~ Population, total (mln) & 0.7 ~ \ \ & 0.8 ~ \ \ & 0.9 ~ \ \ \\ 
	 ~ Population, rural (\% total population) & 0.4 ~ \ \ & 0.4 ~ \ \ & 0.4 ~ \ \ \\ 
	 ~ Govt expenditure on ag (\% total outlays) &  ~ \ \ & 0.7 ~ \ \ & \textit{1.2} ~ \ \ \\ 
	 ~ Area harvested (mln ha) & 4 ~ \ \ & 3 ~ \ \ & 2 ~ \ \ \\ 
	 ~ Cropping intensity ratio (\%) & 8.1 ~ \ \ & 7.5 ~ \ \ &  ~ \ \ \\ 
	 ~ Water resources (m\textsuperscript{3}/person/year) & \textit{38} ~ \ \ & \textit{35} ~ \ \ & \textit{32} ~ \ \ \\ 
	 ~ Area equipped for irrigation (1000 ha) &  ~ \ \ &  ~ \ \ & \textit{4} ~ \ \ \\ 
	 ~ Area irrigated (\%) &  ~ \ \ &  ~ \ \ &  ~ \ \ \\ 
	 ~ Employment in agriculture (\%) &  ~ \ \ &  ~ \ \ &  ~ \ \ \\ 
	 ~ Employment in agriculture, female (\%) &  ~ \ \ &  ~ \ \ &  ~ \ \ \\ 
	 ~ Fertilizers, Nitrogen (nutrients per ha) &  ~ \ \ & 21.9 ~ \ \ & \textit{9.5} ~ \ \ \\ 
	 ~ Fertilizers, Phosphate (nutrients per ha) &  ~ \ \ & 6.3 ~ \ \ & \textit{0.2} ~ \ \ \\ 
	 ~ Fertilizers, Potash (nutrients per ha) &  ~ \ \ & 0.2 ~ \ \ & \textit{3.1} ~ \ \ \\ 
	 ~ Energy consump, power irrigation (mln kWh) &  ~ \ \ &  ~ \ \ &  ~ \ \ \\ 
	 ~ Agr value added per worker (constant US\$) & 2.9 ~ \ \ & 2.8 ~ \ \ & \textit{2.9} ~ \ \ \\ 
	\multicolumn{4}{l}{\textcolor{FAOblue}{\textbf{\large{Hunger dimensions}}}} \\ 
	 ~ Dietary energy supply (kcal/pc/day) & 2\,772 ~ \ \ & 2\,905 ~ \ \ & 2\,930 ~ \ \ \\ 
	 ~ Average dietary energy supply adequacy (\%) & 122 ~ \ \ & 124 ~ \ \ & 124 ~ \ \ \\ 
	 ~ Dietary en supp, cereals/roots/tubers (\%) & 46 ~ \ \ & 49 ~ \ \ & \textit{48} ~ \ \ \\ 
	 ~ Prevalence of undernourishment (\%) & 6.1 ~ \ \ & <5.0 ~ \ \ & <5.0 ~ \ \ \\ 
	 ~ GDP per capita (US\$, PPP) & 5\,751 ~ \ \ & 6\,745 ~ \ \ & \textit{7\,502} ~ \ \ \\ 
	 ~ Domestic food price volatility (index) &  ~ \ \ & 7.1 ~ \ \ & 8.3 ~ \ \ \\ 
	 ~ Cereal import dependency ratio (\%) & 87.1 ~ \ \ & 93.4 ~ \ \ & \textit{93.1} ~ \ \ \\ 
	 ~ Underweight, children under-5 (\%) & \textit{6.9} ~ \ \ & \textit{5.3} ~ \ \ &  ~ \ \ \\ 
	 ~ Improved water source (\% pop) & 86.3 ~ \ \ & 92.3 ~ \ \ & \textit{96.3} ~ \ \ \\ 
	\multicolumn{4}{l}{\textcolor{FAOblue}{\textbf{\large{Food Supply}}}} \\ 
	 ~ Food production value, (2004-2006 mln I\$) & 226 ~ \ \ & 222 ~ \ \ & \textit{196} ~ \ \ \\ 
	 ~ Agriculture, value added (\% GDP) & 20 ~ \ \ & 15 ~ \ \ & \textit{12} ~ \ \ \\ 
	 ~ Food exports (mln US\$)  & 180 ~ \ \ & 147 ~ \ \ & \textit{201} ~ \ \ \\ 
	 ~ Food imports (mln US\$)  & 85 ~ \ \ & 108 ~ \ \ & \textit{287} ~ \ \ \\ 
	\multicolumn{4}{l}{\textit{\normalsize{Production indices (2004-06=100)}}} \\ 
	 ~ Net food & 99 ~ \ \ & 97 ~ \ \ & \textit{86} ~ \ \ \\ 
	 ~ Net crop & 102 ~ \ \ & 97 ~ \ \ & \textit{76} ~ \ \ \\ 
	 ~ Cereal & 158 ~ \ \ & 90 ~ \ \ & \textit{38} ~ \ \ \\ 
	 ~ Vegetable oils & 136 ~ \ \ & 95 ~ \ \ & \textit{127} ~ \ \ \\ 
	 ~ Roots and tubers & 14 ~ \ \ & 62 ~ \ \ & \textit{126} ~ \ \ \\ 
	 ~ Fruit and vegetables & 53 ~ \ \ & 83 ~ \ \ & \textit{83} ~ \ \ \\ 
	 ~ Sugar & 118 ~ \ \ & 107 ~ \ \ & \textit{53} ~ \ \ \\ 
	 ~ Livestock & 93 ~ \ \ & 97 ~ \ \ & \textit{110} ~ \ \ \\ 
	 ~ Milk & 117 ~ \ \ & 105 ~ \ \ & \textit{112} ~ \ \ \\ 
	 ~ Meat & 85 ~ \ \ & 95 ~ \ \ & \textit{104} ~ \ \ \\ 
	 ~ Fish  & 52 ~ \ \ & 83 ~ \ \ & \textit{91} ~ \ \ \\ 
	\multicolumn{4}{l}{\textit{\normalsize{Net trade (min US\$)}}} \\ 
	 ~ Cereals & -24 ~ \ \ & -26 ~ \ \ & \textit{-42} ~ \ \ \\ 
	 ~ Fruit and vegetables & -11 ~ \ \ & -10 ~ \ \ & \textit{-25} ~ \ \ \\ 
	 ~ Meat & -12 ~ \ \ & -14 ~ \ \ & \textit{-29} ~ \ \ \\ 
	 ~ Dairy products & -10 ~ \ \ & -7 ~ \ \ & \textit{-37} ~ \ \ \\ 
	 ~ Fish & 1 ~ \ \ & 21 ~ \ \ & \textit{173} ~ \ \ \\ 
	\multicolumn{4}{l}{\textcolor{FAOblue}{\textbf{\large{Environment}}}} \\ 
	 ~ Forest area (\%) & 52 ~ \ \ & 54 ~ \ \ & \textit{56} ~ \ \ \\ 
	 ~ Renewable water res withdrawn (\% of total) &  ~ \ \ & \textit{59} ~ \ \ & 59 ~ \ \ \\ 
	 ~ Terrestrial protect areas (\% total land area)  & 1 ~ \ \ & 1 ~ \ \ & \textit{4} ~ \ \ \\ 
	 ~ Organic area (\% total agricultural area) &  ~ \ \ & \textit{0} ~ \ \ & \textit{1} ~ \ \ \\ 
	 ~ Water withdrawal by agriculture (\% of total) &  ~ \ \ & \textit{59} ~ \ \ & 59 ~ \ \ \\ 
	 ~ Biofuel production (thousand kt of oil eq.) & 11 ~ \ \ & 8 ~ \ \ & \textit{4} ~ \ \ \\ 
	 ~ Wood pellet prod. (min tonnes) &  ~ \ \ &  ~ \ \ &  ~ \ \ \\ 
	 ~ GHG emissions from ag (Co2 eq, gigagrams) & 1 ~ \ \ & 0 ~ \ \ & \textit{0} ~ \ \ \\ 
       \toprule
      \end{tabular}
      \clearpage
\CountryData{ Finland }
      \rowcolors{1}{FAOblue!10}{white}
      \begin{tabular}{L{3.9cm} R{1cm} R{1cm} R{1cm}}
      \toprule
      \multicolumn{1}{c}{} & \multicolumn{1}{c}{ 1992 } & \multicolumn{1}{c}{ 2002 } & \multicolumn{1}{c}{ 2014 } \\
      \midrule
	\multicolumn{4}{l}{\textcolor{FAOblue}{\textbf{\large{The setting}}}} \\ 
	 ~ Population, total (mln) & 5 ~ \ \ & 5.2 ~ \ \ & 5.4 ~ \ \ \\ 
	 ~ Population, rural (\% total population) & 1 ~ \ \ & 0.9 ~ \ \ & 0.9 ~ \ \ \\ 
	 ~ Govt expenditure on ag (\% total outlays) &  ~ \ \ &  ~ \ \ &  ~ \ \ \\ 
	 ~ Area harvested (mln ha) & 3 ~ \ \ & 4 ~ \ \ & 4 ~ \ \ \\ 
	 ~ Cropping intensity ratio (\%) & 1.1 ~ \ \ & 1.8 ~ \ \ &  ~ \ \ \\ 
	 ~ Water resources (m\textsuperscript{3}/person/year) & \textit{22} ~ \ \ & \textit{21} ~ \ \ & \textit{20} ~ \ \ \\ 
	 ~ Area equipped for irrigation (1000 ha) &  ~ \ \ &  ~ \ \ & \textit{69} ~ \ \ \\ 
	 ~ Area irrigated (\%) &  ~ \ \ &  ~ \ \ & \textit{21.9} ~ \ \ \\ 
	 ~ Employment in agriculture (\%) & 8.9 ~ \ \ & 5.3 ~ \ \ & \textit{4.1} ~ \ \ \\ 
	 ~ Employment in agriculture, female (\%) & 6.3 ~ \ \ & 3.7 ~ \ \ & \textit{2.3} ~ \ \ \\ 
	 ~ Fertilizers, Nitrogen (nutrients per ha) &  ~ \ \ & 80.1 ~ \ \ & \textit{84.1} ~ \ \ \\ 
	 ~ Fertilizers, Phosphate (nutrients per ha) &  ~ \ \ & 21.6 ~ \ \ & \textit{31.8} ~ \ \ \\ 
	 ~ Fertilizers, Potash (nutrients per ha) &  ~ \ \ & 32.6 ~ \ \ & \textit{69.7} ~ \ \ \\ 
	 ~ Energy consump, power irrigation (mln kWh) &  ~ \ \ &  ~ \ \ & \textit{132} ~ \ \ \\ 
	 ~ Agr value added per worker (constant US\$) & 22.9 ~ \ \ & 35 ~ \ \ & \textit{61.6} ~ \ \ \\ 
	\multicolumn{4}{l}{\textcolor{FAOblue}{\textbf{\large{Hunger dimensions}}}} \\ 
	 ~ Dietary energy supply (kcal/pc/day) &  ~ \ \ &  ~ \ \ &  ~ \ \ \\ 
	 ~ Average dietary energy supply adequacy (\%) & 122 ~ \ \ & 123 ~ \ \ & 131 ~ \ \ \\ 
	 ~ Dietary en supp, cereals/roots/tubers (\%) & 29 ~ \ \ & 32 ~ \ \ & \textit{32} ~ \ \ \\ 
	 ~ Prevalence of undernourishment (\%) & <5.0 ~ \ \ & <5.0 ~ \ \ & <5.0 ~ \ \ \\ 
	 ~ GDP per capita (US\$, PPP) & 25\,726 ~ \ \ & 35\,834 ~ \ \ & \textit{38\,821} ~ \ \ \\ 
	 ~ Domestic food price volatility (index) &  ~ \ \ & 10.1 ~ \ \ & 6.2 ~ \ \ \\ 
	 ~ Cereal import dependency ratio (\%) & -38.6 ~ \ \ & -15.2 ~ \ \ & \textit{-21.3} ~ \ \ \\ 
	 ~ Underweight, children under-5 (\%) &  ~ \ \ &  ~ \ \ &  ~ \ \ \\ 
	 ~ Improved water source (\% pop) & 100 ~ \ \ & 100 ~ \ \ & \textit{100} ~ \ \ \\ 
	\multicolumn{4}{l}{\textcolor{FAOblue}{\textbf{\large{Food Supply}}}} \\ 
	 ~ Food production value, (2004-2006 mln I\$) & 1\,809 ~ \ \ & 1\,966 ~ \ \ & \textit{1\,876} ~ \ \ \\ 
	 ~ Agriculture, value added (\% GDP) & 5 ~ \ \ & 3 ~ \ \ & \textit{3} ~ \ \ \\ 
	 ~ Food exports (mln US\$)  & 492 ~ \ \ & 700 ~ \ \ & \textit{1\,485} ~ \ \ \\ 
	 ~ Food imports (mln US\$)  & 873 ~ \ \ & 1\,301 ~ \ \ & \textit{3\,483} ~ \ \ \\ 
	\multicolumn{4}{l}{\textit{\normalsize{Production indices (2004-06=100)}}} \\ 
	 ~ Net food & 93 ~ \ \ & 101 ~ \ \ & \textit{96} ~ \ \ \\ 
	 ~ Net crop & 77 ~ \ \ & 103 ~ \ \ & \textit{100} ~ \ \ \\ 
	 ~ Cereal & 65 ~ \ \ & 100 ~ \ \ & \textit{107} ~ \ \ \\ 
	 ~ Vegetable oils & 121 ~ \ \ & 94 ~ \ \ & \textit{73} ~ \ \ \\ 
	 ~ Roots and tubers & 95 ~ \ \ & 121 ~ \ \ & \textit{92} ~ \ \ \\ 
	 ~ Fruit and vegetables & 92 ~ \ \ & 102 ~ \ \ & \textit{116} ~ \ \ \\ 
	 ~ Sugar & 98 ~ \ \ & 100 ~ \ \ & \textit{45} ~ \ \ \\ 
	 ~ Livestock & 99 ~ \ \ & 100 ~ \ \ & \textit{98} ~ \ \ \\ 
	 ~ Milk & 102 ~ \ \ & 104 ~ \ \ & \textit{96} ~ \ \ \\ 
	 ~ Meat & 94 ~ \ \ & 95 ~ \ \ & \textit{99} ~ \ \ \\ 
	 ~ Fish  & 111 ~ \ \ & 103 ~ \ \ & \textit{119} ~ \ \ \\ 
	\multicolumn{4}{l}{\textit{\normalsize{Net trade (min US\$)}}} \\ 
	 ~ Cereals & 35 ~ \ \ & -47 ~ \ \ & \textit{-129} ~ \ \ \\ 
	 ~ Fruit and vegetables & -438 ~ \ \ & -431 ~ \ \ & \textit{-974} ~ \ \ \\ 
	 ~ Meat & 50 ~ \ \ & -6 ~ \ \ & \textit{-276} ~ \ \ \\ 
	 ~ Dairy products & 110 ~ \ \ & 164 ~ \ \ & \textit{178} ~ \ \ \\ 
	 ~ Fish & -109 ~ \ \ & -127 ~ \ \ & \textit{-387} ~ \ \ \\ 
	\multicolumn{4}{l}{\textcolor{FAOblue}{\textbf{\large{Environment}}}} \\ 
	 ~ Forest area (\%) & 72 ~ \ \ & 73 ~ \ \ & \textit{73} ~ \ \ \\ 
	 ~ Renewable water res withdrawn (\% of total) &  ~ \ \ & \textit{3} ~ \ \ & 3 ~ \ \ \\ 
	 ~ Terrestrial protect areas (\% total land area)  & 9 ~ \ \ & 9 ~ \ \ & \textit{15} ~ \ \ \\ 
	 ~ Organic area (\% total agricultural area) &  ~ \ \ & \textit{6} ~ \ \ & \textit{9} ~ \ \ \\ 
	 ~ Water withdrawal by agriculture (\% of total) &  ~ \ \ & \textit{3} ~ \ \ & 3 ~ \ \ \\ 
	 ~ Biofuel production (thousand kt of oil eq.) & 16 ~ \ \ & 83 ~ \ \ & \textit{7\,773} ~ \ \ \\ 
	 ~ Wood pellet prod. (min tonnes) &  ~ \ \ &  ~ \ \ & \textit{270} ~ \ \ \\ 
	 ~ GHG emissions from ag (Co2 eq, gigagrams) & -17 ~ \ \ & -10 ~ \ \ & \textit{12} ~ \ \ \\ 
       \toprule
      \end{tabular}
      \clearpage
\CountryData{ France }
      \rowcolors{1}{FAOblue!10}{white}
      \begin{tabular}{L{3.9cm} R{1cm} R{1cm} R{1cm}}
      \toprule
      \multicolumn{1}{c}{} & \multicolumn{1}{c}{ 1992 } & \multicolumn{1}{c}{ 2002 } & \multicolumn{1}{c}{ 2014 } \\
      \midrule
	\multicolumn{4}{l}{\textcolor{FAOblue}{\textbf{\large{The setting}}}} \\ 
	 ~ Population, total (mln) & 57.3 ~ \ \ & 60 ~ \ \ & 64.6 ~ \ \ \\ 
	 ~ Population, rural (\% total population) & 14.7 ~ \ \ & 12.7 ~ \ \ & 8.2 ~ \ \ \\ 
	 ~ Govt expenditure on ag (\% total outlays) &  ~ \ \ &  ~ \ \ &  ~ \ \ \\ 
	 ~ Area harvested (mln ha) & 61 ~ \ \ & 70 ~ \ \ & 68 ~ \ \ \\ 
	 ~ Cropping intensity ratio (\%) & 2 ~ \ \ & 2.3 ~ \ \ &  ~ \ \ \\ 
	 ~ Water resources (m\textsuperscript{3}/person/year) & \textit{4} ~ \ \ & \textit{3} ~ \ \ & \textit{3} ~ \ \ \\ 
	 ~ Area equipped for irrigation (1000 ha) &  ~ \ \ &  ~ \ \ & \textit{2\,600} ~ \ \ \\ 
	 ~ Area irrigated (\%) &  ~ \ \ &  ~ \ \ & \textit{57.2} ~ \ \ \\ 
	 ~ Employment in agriculture (\%) & 5.9 ~ \ \ & 4.1 ~ \ \ & \textit{2.9} ~ \ \ \\ 
	 ~ Employment in agriculture, female (\%) & 4.9 ~ \ \ & 2.8 ~ \ \ & \textit{1.8} ~ \ \ \\ 
	 ~ Fertilizers, Nitrogen (nutrients per ha) &  ~ \ \ & 74.2 ~ \ \ & \textit{66.4} ~ \ \ \\ 
	 ~ Fertilizers, Phosphate (nutrients per ha) &  ~ \ \ & 24 ~ \ \ & \textit{8.8} ~ \ \ \\ 
	 ~ Fertilizers, Potash (nutrients per ha) &  ~ \ \ & 32.5 ~ \ \ & \textit{11.7} ~ \ \ \\ 
	 ~ Energy consump, power irrigation (mln kWh) & 98 ~ \ \ & 4\,829 ~ \ \ & \textit{4\,875} ~ \ \ \\ 
	 ~ Agr value added per worker (constant US\$) & 26.7 ~ \ \ & 47.2 ~ \ \ & \textit{75.1} ~ \ \ \\ 
	\multicolumn{4}{l}{\textcolor{FAOblue}{\textbf{\large{Hunger dimensions}}}} \\ 
	 ~ Dietary energy supply (kcal/pc/day) &  ~ \ \ &  ~ \ \ &  ~ \ \ \\ 
	 ~ Average dietary energy supply adequacy (\%) & 142 ~ \ \ & 145 ~ \ \ & 141 ~ \ \ \\ 
	 ~ Dietary en supp, cereals/roots/tubers (\%) & 27 ~ \ \ & 28 ~ \ \ & \textit{29} ~ \ \ \\ 
	 ~ Prevalence of undernourishment (\%) & <5.0 ~ \ \ & <5.0 ~ \ \ & <5.0 ~ \ \ \\ 
	 ~ GDP per capita (US\$, PPP) & 30\,033 ~ \ \ & 35\,333 ~ \ \ & \textit{37\,217} ~ \ \ \\ 
	 ~ Domestic food price volatility (index) &  ~ \ \ & 4.2 ~ \ \ & 4.8 ~ \ \ \\ 
	 ~ Cereal import dependency ratio (\%) & -122.5 ~ \ \ & -87 ~ \ \ & \textit{-90.4} ~ \ \ \\ 
	 ~ Underweight, children under-5 (\%) &  ~ \ \ &  ~ \ \ &  ~ \ \ \\ 
	 ~ Improved water source (\% pop) & 100 ~ \ \ & 100 ~ \ \ & \textit{100} ~ \ \ \\ 
	\multicolumn{4}{l}{\textcolor{FAOblue}{\textbf{\large{Food Supply}}}} \\ 
	 ~ Food production value, (2004-2006 mln I\$) & 40\,881 ~ \ \ & 40\,585 ~ \ \ & \textit{38\,188} ~ \ \ \\ 
	 ~ Agriculture, value added (\% GDP) &  ~ \ \ & 2 ~ \ \ & \textit{2} ~ \ \ \\ 
	 ~ Food exports (mln US\$)  & 25\,837 ~ \ \ & 22\,803 ~ \ \ & \textit{46\,001} ~ \ \ \\ 
	 ~ Food imports (mln US\$)  & 17\,219 ~ \ \ & 16\,997 ~ \ \ & \textit{37\,016} ~ \ \ \\ 
	\multicolumn{4}{l}{\textit{\normalsize{Production indices (2004-06=100)}}} \\ 
	 ~ Net food & 106 ~ \ \ & 105 ~ \ \ & \textit{99} ~ \ \ \\ 
	 ~ Net crop & 102 ~ \ \ & 104 ~ \ \ & \textit{94} ~ \ \ \\ 
	 ~ Cereal & 92 ~ \ \ & 106 ~ \ \ & \textit{104} ~ \ \ \\ 
	 ~ Vegetable oils & 70 ~ \ \ & 86 ~ \ \ & \textit{104} ~ \ \ \\ 
	 ~ Roots and tubers & 99 ~ \ \ & 102 ~ \ \ & \textit{103} ~ \ \ \\ 
	 ~ Fruit and vegetables & 118 ~ \ \ & 104 ~ \ \ & \textit{80} ~ \ \ \\ 
	 ~ Sugar & 104 ~ \ \ & 109 ~ \ \ & \textit{110} ~ \ \ \\ 
	 ~ Livestock & 104 ~ \ \ & 106 ~ \ \ & \textit{100} ~ \ \ \\ 
	 ~ Milk & 104 ~ \ \ & 103 ~ \ \ & \textit{97} ~ \ \ \\ 
	 ~ Meat & 104 ~ \ \ & 109 ~ \ \ & \textit{102} ~ \ \ \\ 
	 ~ Fish  & 101 ~ \ \ & 107 ~ \ \ & \textit{84} ~ \ \ \\ 
	\multicolumn{4}{l}{\textit{\normalsize{Net trade (min US\$)}}} \\ 
	 ~ Cereals & 6\,339 ~ \ \ & 3\,566 ~ \ \ & \textit{9\,659} ~ \ \ \\ 
	 ~ Fruit and vegetables & -2\,660 ~ \ \ & -2\,081 ~ \ \ & \textit{-5\,751} ~ \ \ \\ 
	 ~ Meat & -328 ~ \ \ & 350 ~ \ \ & \textit{-1\,270} ~ \ \ \\ 
	 ~ Dairy products & 1\,971 ~ \ \ & 1\,861 ~ \ \ & \textit{4\,015} ~ \ \ \\ 
	 ~ Fish & -1\,979 ~ \ \ & -2\,118 ~ \ \ & \textit{-4\,281} ~ \ \ \\ 
	\multicolumn{4}{l}{\textcolor{FAOblue}{\textbf{\large{Environment}}}} \\ 
	 ~ Forest area (\%) & 27 ~ \ \ & 28 ~ \ \ & \textit{29} ~ \ \ \\ 
	 ~ Renewable water res withdrawn (\% of total) &  ~ \ \ &  ~ \ \ & 12 ~ \ \ \\ 
	 ~ Terrestrial protect areas (\% total land area)  & 10 ~ \ \ & 15 ~ \ \ & \textit{25} ~ \ \ \\ 
	 ~ Organic area (\% total agricultural area) &  ~ \ \ & \textit{2} ~ \ \ & \textit{4} ~ \ \ \\ 
	 ~ Water withdrawal by agriculture (\% of total) &  ~ \ \ &  ~ \ \ & 12 ~ \ \ \\ 
	 ~ Biofuel production (thousand kt of oil eq.) & 19 ~ \ \ & 8\,847 ~ \ \ & \textit{53\,997} ~ \ \ \\ 
	 ~ Wood pellet prod. (min tonnes) &  ~ \ \ &  ~ \ \ & \textit{890} ~ \ \ \\ 
	 ~ GHG emissions from ag (Co2 eq, gigagrams) & 54 ~ \ \ & -2 ~ \ \ & \textit{44} ~ \ \ \\ 
       \toprule
      \end{tabular}
      \clearpage
\CountryData{ Gabon }
      \rowcolors{1}{FAOblue!10}{white}
      \begin{tabular}{L{3.9cm} R{1cm} R{1cm} R{1cm}}
      \toprule
      \multicolumn{1}{c}{} & \multicolumn{1}{c}{ 1992 } & \multicolumn{1}{c}{ 2002 } & \multicolumn{1}{c}{ 2014 } \\
      \midrule
	\multicolumn{4}{l}{\textcolor{FAOblue}{\textbf{\large{The setting}}}} \\ 
	 ~ Population, total (mln) & 1 ~ \ \ & 1.3 ~ \ \ & 1.7 ~ \ \ \\ 
	 ~ Population, rural (\% total population) & 0.3 ~ \ \ & 0.2 ~ \ \ & 0.2 ~ \ \ \\ 
	 ~ Govt expenditure on ag (\% total outlays) &  ~ \ \ &  ~ \ \ &  ~ \ \ \\ 
	 ~ Area harvested (mln ha) & 0 ~ \ \ & 1 ~ \ \ & 1 ~ \ \ \\ 
	 ~ Cropping intensity ratio (\%) & 0.1 ~ \ \ & 0.1 ~ \ \ &  ~ \ \ \\ 
	 ~ Water resources (m\textsuperscript{3}/person/year) & \textit{162} ~ \ \ & \textit{126} ~ \ \ & \textit{99} ~ \ \ \\ 
	 ~ Area equipped for irrigation (1000 ha) &  ~ \ \ &  ~ \ \ & \textit{4} ~ \ \ \\ 
	 ~ Area irrigated (\%) &  ~ \ \ &  ~ \ \ &  ~ \ \ \\ 
	 ~ Employment in agriculture (\%) & \textit{41.6} ~ \ \ & \textit{24.2} ~ \ \ & \textit{24.2} ~ \ \ \\ 
	 ~ Employment in agriculture, female (\%) & \textit{60.8} ~ \ \ & \textit{33.7} ~ \ \ & \textit{33.7} ~ \ \ \\ 
	 ~ Fertilizers, Nitrogen (nutrients per ha) &  ~ \ \ & 0.1 ~ \ \ & \textit{0.4} ~ \ \ \\ 
	 ~ Fertilizers, Phosphate (nutrients per ha) &  ~ \ \ & 0 ~ \ \ & \textit{0.3} ~ \ \ \\ 
	 ~ Fertilizers, Potash (nutrients per ha) &  ~ \ \ & 0.2 ~ \ \ & \textit{0.4} ~ \ \ \\ 
	 ~ Energy consump, power irrigation (mln kWh) &  ~ \ \ &  ~ \ \ &  ~ \ \ \\ 
	 ~ Agr value added per worker (constant US\$) & 1.7 ~ \ \ & 2 ~ \ \ & \textit{2.6} ~ \ \ \\ 
	\multicolumn{4}{l}{\textcolor{FAOblue}{\textbf{\large{Hunger dimensions}}}} \\ 
	 ~ Dietary energy supply (kcal/pc/day) & 2\,530 ~ \ \ & 2\,668 ~ \ \ & 2\,807 ~ \ \ \\ 
	 ~ Average dietary energy supply adequacy (\%) & 116 ~ \ \ & 120 ~ \ \ & 125 ~ \ \ \\ 
	 ~ Dietary en supp, cereals/roots/tubers (\%) & 46 ~ \ \ & 50 ~ \ \ & \textit{51} ~ \ \ \\ 
	 ~ Prevalence of undernourishment (\%) & 9.5 ~ \ \ & <5.0 ~ \ \ & <5.0 ~ \ \ \\ 
	 ~ GDP per capita (US\$, PPP) & 18\,662 ~ \ \ & 16\,936 ~ \ \ & \textit{18\,646} ~ \ \ \\ 
	 ~ Domestic food price volatility (index) &  ~ \ \ & 9 ~ \ \ & 21 ~ \ \ \\ 
	 ~ Cereal import dependency ratio (\%) & 79.4 ~ \ \ & 83.5 ~ \ \ & \textit{81.9} ~ \ \ \\ 
	 ~ Underweight, children under-5 (\%) &  ~ \ \ & \textit{8.8} ~ \ \ & \textit{6.5} ~ \ \ \\ 
	 ~ Improved water source (\% pop) & \textit{80} ~ \ \ & 85.6 ~ \ \ & \textit{92.2} ~ \ \ \\ 
	\multicolumn{4}{l}{\textcolor{FAOblue}{\textbf{\large{Food Supply}}}} \\ 
	 ~ Food production value, (2004-2006 mln I\$) & 200 ~ \ \ & 214 ~ \ \ & \textit{260} ~ \ \ \\ 
	 ~ Agriculture, value added (\% GDP) & 8 ~ \ \ & 6 ~ \ \ & \textit{4} ~ \ \ \\ 
	 ~ Food exports (mln US\$)  & 6 ~ \ \ & 3 ~ \ \ & \textit{0} ~ \ \ \\ 
	 ~ Food imports (mln US\$)  & 130 ~ \ \ & 121 ~ \ \ & \textit{489} ~ \ \ \\ 
	\multicolumn{4}{l}{\textit{\normalsize{Production indices (2004-06=100)}}} \\ 
	 ~ Net food & 93 ~ \ \ & 99 ~ \ \ & \textit{121} ~ \ \ \\ 
	 ~ Net crop & 83 ~ \ \ & 97 ~ \ \ & \textit{123} ~ \ \ \\ 
	 ~ Cereal & 76 ~ \ \ & 73 ~ \ \ & \textit{138} ~ \ \ \\ 
	 ~ Vegetable oils & 92 ~ \ \ & 93 ~ \ \ & \textit{86} ~ \ \ \\ 
	 ~ Roots and tubers & 83 ~ \ \ & 100 ~ \ \ & \textit{127} ~ \ \ \\ 
	 ~ Fruit and vegetables & 91 ~ \ \ & 99 ~ \ \ & \textit{109} ~ \ \ \\ 
	 ~ Sugar & 83 ~ \ \ & 115 ~ \ \ & \textit{137} ~ \ \ \\ 
	 ~ Livestock & 91 ~ \ \ & 101 ~ \ \ & \textit{120} ~ \ \ \\ 
	 ~ Milk & 59 ~ \ \ & 76 ~ \ \ & \textit{108} ~ \ \ \\ 
	 ~ Meat & 92 ~ \ \ & 101 ~ \ \ & \textit{121} ~ \ \ \\ 
	 ~ Fish  & 55 ~ \ \ & 95 ~ \ \ & \textit{76} ~ \ \ \\ 
	\multicolumn{4}{l}{\textit{\normalsize{Net trade (min US\$)}}} \\ 
	 ~ Cereals & -42 ~ \ \ & -51 ~ \ \ & \textit{-137} ~ \ \ \\ 
	 ~ Fruit and vegetables & \textit{-11} ~ \ \ & -9 ~ \ \ & \textit{-36} ~ \ \ \\ 
	 ~ Meat & \textit{-45} ~ \ \ & -26 ~ \ \ & \textit{-170} ~ \ \ \\ 
	 ~ Dairy products & \textit{-14} ~ \ \ & -16 ~ \ \ & \textit{-44} ~ \ \ \\ 
	 ~ Fish & -11 ~ \ \ & 9 ~ \ \ & \textit{-41} ~ \ \ \\ 
	\multicolumn{4}{l}{\textcolor{FAOblue}{\textbf{\large{Environment}}}} \\ 
	 ~ Forest area (\%) & 85 ~ \ \ & 85 ~ \ \ & \textit{85} ~ \ \ \\ 
	 ~ Renewable water res withdrawn (\% of total) &  ~ \ \ & \textit{29} ~ \ \ & 29 ~ \ \ \\ 
	 ~ Terrestrial protect areas (\% total land area)  & 5 ~ \ \ & 15 ~ \ \ & \textit{20} ~ \ \ \\ 
	 ~ Organic area (\% total agricultural area) &  ~ \ \ &  ~ \ \ &  ~ \ \ \\ 
	 ~ Water withdrawal by agriculture (\% of total) &  ~ \ \ & \textit{29} ~ \ \ & 29 ~ \ \ \\ 
	 ~ Biofuel production (thousand kt of oil eq.) & 0 ~ \ \ & 0 ~ \ \ & \textit{1} ~ \ \ \\ 
	 ~ Wood pellet prod. (min tonnes) &  ~ \ \ &  ~ \ \ &  ~ \ \ \\ 
	 ~ GHG emissions from ag (Co2 eq, gigagrams) & 1 ~ \ \ & 1 ~ \ \ & \textit{1} ~ \ \ \\ 
       \toprule
      \end{tabular}
      \clearpage
\CountryData{ Gambia }
      \rowcolors{1}{FAOblue!10}{white}
      \begin{tabular}{L{3.9cm} R{1cm} R{1cm} R{1cm}}
      \toprule
      \multicolumn{1}{c}{} & \multicolumn{1}{c}{ 1992 } & \multicolumn{1}{c}{ 2002 } & \multicolumn{1}{c}{ 2014 } \\
      \midrule
	\multicolumn{4}{l}{\textcolor{FAOblue}{\textbf{\large{The setting}}}} \\ 
	 ~ Population, total (mln) & 1 ~ \ \ & 1.3 ~ \ \ & 1.9 ~ \ \ \\ 
	 ~ Population, rural (\% total population) & 0.6 ~ \ \ & 0.6 ~ \ \ & 0.8 ~ \ \ \\ 
	 ~ Govt expenditure on ag (\% total outlays) &  ~ \ \ &  ~ \ \ &  ~ \ \ \\ 
	 ~ Area harvested (mln ha) & 0 ~ \ \ & 0 ~ \ \ & 0 ~ \ \ \\ 
	 ~ Cropping intensity ratio (\%) & 0.2 ~ \ \ & 0.3 ~ \ \ &  ~ \ \ \\ 
	 ~ Water resources (m\textsuperscript{3}/person/year) & \textit{8} ~ \ \ & \textit{6} ~ \ \ & \textit{4} ~ \ \ \\ 
	 ~ Area equipped for irrigation (1000 ha) &  ~ \ \ &  ~ \ \ & \textit{5} ~ \ \ \\ 
	 ~ Area irrigated (\%) & \textit{64.7} ~ \ \ &  ~ \ \ &  ~ \ \ \\ 
	 ~ Employment in agriculture (\%) & \textit{64.7} ~ \ \ &  ~ \ \ &  ~ \ \ \\ 
	 ~ Employment in agriculture, female (\%) & \textit{76.5} ~ \ \ &  ~ \ \ &  ~ \ \ \\ 
	 ~ Fertilizers, Nitrogen (nutrients per ha) &  ~ \ \ & 0 ~ \ \ & \textit{1.9} ~ \ \ \\ 
	 ~ Fertilizers, Phosphate (nutrients per ha) &  ~ \ \ & 0 ~ \ \ & \textit{1.6} ~ \ \ \\ 
	 ~ Fertilizers, Potash (nutrients per ha) &  ~ \ \ & 0 ~ \ \ & \textit{1.2} ~ \ \ \\ 
	 ~ Energy consump, power irrigation (mln kWh) & 0 ~ \ \ & 0 ~ \ \ & \textit{0} ~ \ \ \\ 
	 ~ Agr value added per worker (constant US\$) & 0.3 ~ \ \ & 0.3 ~ \ \ & \textit{0.3} ~ \ \ \\ 
	\multicolumn{4}{l}{\textcolor{FAOblue}{\textbf{\large{Hunger dimensions}}}} \\ 
	 ~ Dietary energy supply (kcal/pc/day) & 2\,503 ~ \ \ & 2\,481 ~ \ \ & 2\,882 ~ \ \ \\ 
	 ~ Average dietary energy supply adequacy (\%) & 115 ~ \ \ & 113 ~ \ \ & 131 ~ \ \ \\ 
	 ~ Dietary en supp, cereals/roots/tubers (\%) & 57 ~ \ \ & 55 ~ \ \ & \textit{63} ~ \ \ \\ 
	 ~ Prevalence of undernourishment (\%) & 13.5 ~ \ \ & 13 ~ \ \ & 5.4 ~ \ \ \\ 
	 ~ GDP per capita (US\$, PPP) & 1\,513 ~ \ \ & 1\,506 ~ \ \ & \textit{1\,608} ~ \ \ \\ 
	 ~ Domestic food price volatility (index) &  ~ \ \ & 5.2 ~ \ \ & 2.7 ~ \ \ \\ 
	 ~ Cereal import dependency ratio (\%) & 50.3 ~ \ \ & 38.5 ~ \ \ & \textit{43.6} ~ \ \ \\ 
	 ~ Underweight, children under-5 (\%) &  ~ \ \ & \textit{15.4} ~ \ \ & \textit{17.4} ~ \ \ \\ 
	 ~ Improved water source (\% pop) & 77.2 ~ \ \ & 84.1 ~ \ \ & \textit{90.1} ~ \ \ \\ 
	\multicolumn{4}{l}{\textcolor{FAOblue}{\textbf{\large{Food Supply}}}} \\ 
	 ~ Food production value, (2004-2006 mln I\$) & 66 ~ \ \ & 82 ~ \ \ & \textit{116} ~ \ \ \\ 
	 ~ Agriculture, value added (\% GDP) &  ~ \ \ &  ~ \ \ &  ~ \ \ \\ 
	 ~ Food exports (mln US\$)  & 15 ~ \ \ & 16 ~ \ \ & \textit{23} ~ \ \ \\ 
	 ~ Food imports (mln US\$)  & 72 ~ \ \ & 60 ~ \ \ & \textit{98} ~ \ \ \\ 
	\multicolumn{4}{l}{\textit{\normalsize{Production indices (2004-06=100)}}} \\ 
	 ~ Net food & 55 ~ \ \ & 68 ~ \ \ & \textit{97} ~ \ \ \\ 
	 ~ Net crop & 50 ~ \ \ & 64 ~ \ \ & \textit{97} ~ \ \ \\ 
	 ~ Cereal & 46 ~ \ \ & 65 ~ \ \ & \textit{114} ~ \ \ \\ 
	 ~ Vegetable oils & 49 ~ \ \ & 61 ~ \ \ & \textit{77} ~ \ \ \\ 
	 ~ Roots and tubers & 101 ~ \ \ & 85 ~ \ \ & \textit{121} ~ \ \ \\ 
	 ~ Fruit and vegetables & 66 ~ \ \ & 86 ~ \ \ & \textit{124} ~ \ \ \\ 
	 ~ Sugar &  ~ \ \ &  ~ \ \ &  ~ \ \ \\ 
	 ~ Livestock & 81 ~ \ \ & 87 ~ \ \ & \textit{99} ~ \ \ \\ 
	 ~ Milk & 83 ~ \ \ & 93 ~ \ \ & \textit{110} ~ \ \ \\ 
	 ~ Meat & 80 ~ \ \ & 86 ~ \ \ & \textit{97} ~ \ \ \\ 
	 ~ Fish  & 51 ~ \ \ & 107 ~ \ \ & \textit{125} ~ \ \ \\ 
	\multicolumn{4}{l}{\textit{\normalsize{Net trade (min US\$)}}} \\ 
	 ~ Cereals & \textit{-34} ~ \ \ & -20 ~ \ \ & \textit{-58} ~ \ \ \\ 
	 ~ Fruit and vegetables & -9 ~ \ \ & -3 ~ \ \ & \textit{-4} ~ \ \ \\ 
	 ~ Meat & -1 ~ \ \ & -1 ~ \ \ & \textit{-2} ~ \ \ \\ 
	 ~ Dairy products &  ~ \ \ & -4 ~ \ \ & \textit{0} ~ \ \ \\ 
	 ~ Fish & 2 ~ \ \ & 1 ~ \ \ & \textit{1} ~ \ \ \\ 
	\multicolumn{4}{l}{\textcolor{FAOblue}{\textbf{\large{Environment}}}} \\ 
	 ~ Forest area (\%) & 44 ~ \ \ & 46 ~ \ \ & \textit{48} ~ \ \ \\ 
	 ~ Renewable water res withdrawn (\% of total) &  ~ \ \ & \textit{43} ~ \ \ & 43 ~ \ \ \\ 
	 ~ Terrestrial protect areas (\% total land area)  & 1 ~ \ \ & 2 ~ \ \ & \textit{5} ~ \ \ \\ 
	 ~ Organic area (\% total agricultural area) &  ~ \ \ &  ~ \ \ & \textit{0} ~ \ \ \\ 
	 ~ Water withdrawal by agriculture (\% of total) &  ~ \ \ & \textit{43} ~ \ \ & 43 ~ \ \ \\ 
	 ~ Biofuel production (thousand kt of oil eq.) &  ~ \ \ &  ~ \ \ &  ~ \ \ \\ 
	 ~ Wood pellet prod. (min tonnes) &  ~ \ \ &  ~ \ \ &  ~ \ \ \\ 
	 ~ GHG emissions from ag (Co2 eq, gigagrams) & 0 ~ \ \ & 0 ~ \ \ & \textit{1} ~ \ \ \\ 
       \toprule
      \end{tabular}
      \clearpage
\CountryData{ Georgia }
      \rowcolors{1}{FAOblue!10}{white}
      \begin{tabular}{L{3.9cm} R{1cm} R{1cm} R{1cm}}
      \toprule
      \multicolumn{1}{c}{} & \multicolumn{1}{c}{ 1992 } & \multicolumn{1}{c}{ 2002 } & \multicolumn{1}{c}{ 2014 } \\
      \midrule
	\multicolumn{4}{l}{\textcolor{FAOblue}{\textbf{\large{The setting}}}} \\ 
	 ~ Population, total (mln) & 5.3 ~ \ \ & 4.6 ~ \ \ & 4.3 ~ \ \ \\ 
	 ~ Population, rural (\% total population) & 2.4 ~ \ \ & 2.2 ~ \ \ & 2 ~ \ \ \\ 
	 ~ Govt expenditure on ag (\% total outlays) &  ~ \ \ & 2.1 ~ \ \ & \textit{0.5} ~ \ \ \\ 
	 ~ Area harvested (mln ha) & 1 ~ \ \ & 1 ~ \ \ & 0 ~ \ \ \\ 
	 ~ Cropping intensity ratio (\%) & 0.3 ~ \ \ & 0.2 ~ \ \ &  ~ \ \ \\ 
	 ~ Water resources (m\textsuperscript{3}/person/year) & \textit{12} ~ \ \ & \textit{14} ~ \ \ & \textit{15} ~ \ \ \\ 
	 ~ Area equipped for irrigation (1000 ha) &  ~ \ \ &  ~ \ \ & \textit{433} ~ \ \ \\ 
	 ~ Area irrigated (\%) &  ~ \ \ & \textit{63} ~ \ \ &  ~ \ \ \\ 
	 ~ Employment in agriculture (\%) &  ~ \ \ & 53.8 ~ \ \ & \textit{53.4} ~ \ \ \\ 
	 ~ Employment in agriculture, female (\%) &  ~ \ \ & 55.3 ~ \ \ & \textit{56.6} ~ \ \ \\ 
	 ~ Fertilizers, Nitrogen (nutrients per ha) &  ~ \ \ & 7.1 ~ \ \ & \textit{5.5} ~ \ \ \\ 
	 ~ Fertilizers, Phosphate (nutrients per ha) &  ~ \ \ & 0.9 ~ \ \ & \textit{1.5} ~ \ \ \\ 
	 ~ Fertilizers, Potash (nutrients per ha) &  ~ \ \ & 0.8 ~ \ \ & \textit{0.3} ~ \ \ \\ 
	 ~ Energy consump, power irrigation (mln kWh) &  ~ \ \ & 171 ~ \ \ & \textit{68} ~ \ \ \\ 
	 ~ Agr value added per worker (constant US\$) & 2.9 ~ \ \ & 1.9 ~ \ \ & \textit{2.9} ~ \ \ \\ 
	\multicolumn{4}{l}{\textcolor{FAOblue}{\textbf{\large{Hunger dimensions}}}} \\ 
	 ~ Dietary energy supply (kcal/pc/day) & 1\,665 ~ \ \ & 2\,803 ~ \ \ & 2\,852 ~ \ \ \\ 
	 ~ Average dietary energy supply adequacy (\%) & 68 ~ \ \ & 113 ~ \ \ & 115 ~ \ \ \\ 
	 ~ Dietary en supp, cereals/roots/tubers (\%) & 61 ~ \ \ & 59 ~ \ \ & \textit{58} ~ \ \ \\ 
	 ~ Prevalence of undernourishment (\%) & 80.8 ~ \ \ & 13.7 ~ \ \ & 8 ~ \ \ \\ 
	 ~ GDP per capita (US\$, PPP) & 3\,435 ~ \ \ & 3\,664 ~ \ \ & \textit{6\,930} ~ \ \ \\ 
	 ~ Domestic food price volatility (index) &  ~ \ \ &  ~ \ \ &  ~ \ \ \\ 
	 ~ Cereal import dependency ratio (\%) & 70.4 ~ \ \ & 53.9 ~ \ \ & \textit{68.6} ~ \ \ \\ 
	 ~ Underweight, children under-5 (\%) &  ~ \ \ & \textit{2.3} ~ \ \ & \textit{1.1} ~ \ \ \\ 
	 ~ Improved water source (\% pop) & 84.8 ~ \ \ & 90.8 ~ \ \ & \textit{98.7} ~ \ \ \\ 
	\multicolumn{4}{l}{\textcolor{FAOblue}{\textbf{\large{Food Supply}}}} \\ 
	 ~ Food production value, (2004-2006 mln I\$) & 935 ~ \ \ & 850 ~ \ \ & \textit{775} ~ \ \ \\ 
	 ~ Agriculture, value added (\% GDP) & 53 ~ \ \ & 21 ~ \ \ & \textit{9} ~ \ \ \\ 
	 ~ Food exports (mln US\$)  & 7 ~ \ \ & 36 ~ \ \ & \textit{260} ~ \ \ \\ 
	 ~ Food imports (mln US\$)  & 203 ~ \ \ & 182 ~ \ \ & \textit{961} ~ \ \ \\ 
	\multicolumn{4}{l}{\textit{\normalsize{Production indices (2004-06=100)}}} \\ 
	 ~ Net food & 105 ~ \ \ & 95 ~ \ \ & \textit{87} ~ \ \ \\ 
	 ~ Net crop & 139 ~ \ \ & 92 ~ \ \ & \textit{100} ~ \ \ \\ 
	 ~ Cereal & 89 ~ \ \ & 118 ~ \ \ & \textit{90} ~ \ \ \\ 
	 ~ Vegetable oils & 38 ~ \ \ & 63 ~ \ \ & \textit{33} ~ \ \ \\ 
	 ~ Roots and tubers & 81 ~ \ \ & 122 ~ \ \ & \textit{106} ~ \ \ \\ 
	 ~ Fruit and vegetables & 157 ~ \ \ & 76 ~ \ \ & \textit{94} ~ \ \ \\ 
	 ~ Sugar &  ~ \ \ &  ~ \ \ &  ~ \ \ \\ 
	 ~ Livestock & 78 ~ \ \ & 103 ~ \ \ & \textit{74} ~ \ \ \\ 
	 ~ Milk & 52 ~ \ \ & 101 ~ \ \ & \textit{85} ~ \ \ \\ 
	 ~ Meat & 106 ~ \ \ & 107 ~ \ \ & \textit{56} ~ \ \ \\ 
	 ~ Fish  & 318 ~ \ \ & 15 ~ \ \ & \textit{103} ~ \ \ \\ 
	\multicolumn{4}{l}{\textit{\normalsize{Net trade (min US\$)}}} \\ 
	 ~ Cereals & \textit{-209} ~ \ \ & -88 ~ \ \ & \textit{-272} ~ \ \ \\ 
	 ~ Fruit and vegetables & -17 ~ \ \ & 2 ~ \ \ & \textit{13} ~ \ \ \\ 
	 ~ Meat & -15 ~ \ \ & -14 ~ \ \ & \textit{-140} ~ \ \ \\ 
	 ~ Dairy products & -8 ~ \ \ & -2 ~ \ \ & \textit{-36} ~ \ \ \\ 
	 ~ Fish & \textit{-1} ~ \ \ & -1 ~ \ \ & \textit{-39} ~ \ \ \\ 
	\multicolumn{4}{l}{\textcolor{FAOblue}{\textbf{\large{Environment}}}} \\ 
	 ~ Forest area (\%) & 40 ~ \ \ & 40 ~ \ \ & \textit{39} ~ \ \ \\ 
	 ~ Renewable water res withdrawn (\% of total) &  ~ \ \ &  ~ \ \ & 58 ~ \ \ \\ 
	 ~ Terrestrial protect areas (\% total land area)  & 3 ~ \ \ & 4 ~ \ \ & \textit{4} ~ \ \ \\ 
	 ~ Organic area (\% total agricultural area) &  ~ \ \ & \textit{0} ~ \ \ & \textit{0} ~ \ \ \\ 
	 ~ Water withdrawal by agriculture (\% of total) &  ~ \ \ &  ~ \ \ & 58 ~ \ \ \\ 
	 ~ Biofuel production (thousand kt of oil eq.) &  ~ \ \ & 0 ~ \ \ & \textit{0} ~ \ \ \\ 
	 ~ Wood pellet prod. (min tonnes) &  ~ \ \ &  ~ \ \ &  ~ \ \ \\ 
	 ~ GHG emissions from ag (Co2 eq, gigagrams) & 3 ~ \ \ & -1 ~ \ \ & \textit{-1} ~ \ \ \\ 
       \toprule
      \end{tabular}
      \clearpage
\CountryData{ Germany }
      \rowcolors{1}{FAOblue!10}{white}
      \begin{tabular}{L{3.9cm} R{1cm} R{1cm} R{1cm}}
      \toprule
      \multicolumn{1}{c}{} & \multicolumn{1}{c}{ 1992 } & \multicolumn{1}{c}{ 2002 } & \multicolumn{1}{c}{ 2014 } \\
      \midrule
	\multicolumn{4}{l}{\textcolor{FAOblue}{\textbf{\large{The setting}}}} \\ 
	 ~ Population, total (mln) & 81.6 ~ \ \ & 83.7 ~ \ \ & 82.7 ~ \ \ \\ 
	 ~ Population, rural (\% total population) & 21.7 ~ \ \ & 22.4 ~ \ \ & 21.2 ~ \ \ \\ 
	 ~ Govt expenditure on ag (\% total outlays) &  ~ \ \ &  ~ \ \ &  ~ \ \ \\ 
	 ~ Area harvested (mln ha) & 35 ~ \ \ & 43 ~ \ \ & 48 ~ \ \ \\ 
	 ~ Cropping intensity ratio (\%) & 2.1 ~ \ \ & 2.6 ~ \ \ &  ~ \ \ \\ 
	 ~ Water resources (m\textsuperscript{3}/person/year) & \textit{2} ~ \ \ & \textit{2} ~ \ \ & \textit{2} ~ \ \ \\ 
	 ~ Area equipped for irrigation (1000 ha) &  ~ \ \ &  ~ \ \ & \textit{650} ~ \ \ \\ 
	 ~ Area irrigated (\%) &  ~ \ \ &  ~ \ \ & \textit{45.5} ~ \ \ \\ 
	 ~ Employment in agriculture (\%) & 3.7 ~ \ \ & 2.5 ~ \ \ & \textit{1.5} ~ \ \ \\ 
	 ~ Employment in agriculture, female (\%) & 3.6 ~ \ \ & 2 ~ \ \ & \textit{1.1} ~ \ \ \\ 
	 ~ Fertilizers, Nitrogen (nutrients per ha) &  ~ \ \ & 105.4 ~ \ \ & \textit{98.9} ~ \ \ \\ 
	 ~ Fertilizers, Phosphate (nutrients per ha) &  ~ \ \ & 19.3 ~ \ \ & \textit{17.1} ~ \ \ \\ 
	 ~ Fertilizers, Potash (nutrients per ha) &  ~ \ \ & 28.3 ~ \ \ & \textit{25.3} ~ \ \ \\ 
	 ~ Energy consump, power irrigation (mln kWh) & 4 ~ \ \ & 4 ~ \ \ & \textit{974} ~ \ \ \\ 
	 ~ Agr value added per worker (constant US\$) & 20.8 ~ \ \ & 22.2 ~ \ \ & \textit{35.2} ~ \ \ \\ 
	\multicolumn{4}{l}{\textcolor{FAOblue}{\textbf{\large{Hunger dimensions}}}} \\ 
	 ~ Dietary energy supply (kcal/pc/day) &  ~ \ \ &  ~ \ \ &  ~ \ \ \\ 
	 ~ Average dietary energy supply adequacy (\%) & 130 ~ \ \ & 133 ~ \ \ & 139 ~ \ \ \\ 
	 ~ Dietary en supp, cereals/roots/tubers (\%) & 26 ~ \ \ & 29 ~ \ \ & \textit{28} ~ \ \ \\ 
	 ~ Prevalence of undernourishment (\%) & <5.0 ~ \ \ & <5.0 ~ \ \ & <5.0 ~ \ \ \\ 
	 ~ GDP per capita (US\$, PPP) & 33\,222 ~ \ \ & 37\,458 ~ \ \ & \textit{42\,884} ~ \ \ \\ 
	 ~ Domestic food price volatility (index) &  ~ \ \ & 7.2 ~ \ \ & 5.6 ~ \ \ \\ 
	 ~ Cereal import dependency ratio (\%) & -14.2 ~ \ \ & -20.7 ~ \ \ & \textit{-9.2} ~ \ \ \\ 
	 ~ Underweight, children under-5 (\%) &  ~ \ \ & \textit{1.1} ~ \ \ & \textit{1.1} ~ \ \ \\ 
	 ~ Improved water source (\% pop) & 100 ~ \ \ & 100 ~ \ \ & \textit{100} ~ \ \ \\ 
	\multicolumn{4}{l}{\textcolor{FAOblue}{\textbf{\large{Food Supply}}}} \\ 
	 ~ Food production value, (2004-2006 mln I\$) & 30\,291 ~ \ \ & 30\,979 ~ \ \ & \textit{33\,635} ~ \ \ \\ 
	 ~ Agriculture, value added (\% GDP) & 1 ~ \ \ & 1 ~ \ \ & \textit{1} ~ \ \ \\ 
	 ~ Food exports (mln US\$)  & 16\,855 ~ \ \ & 17\,931 ~ \ \ & \textit{54\,646} ~ \ \ \\ 
	 ~ Food imports (mln US\$)  & 31\,421 ~ \ \ & 25\,592 ~ \ \ & \textit{63\,113} ~ \ \ \\ 
	\multicolumn{4}{l}{\textit{\normalsize{Production indices (2004-06=100)}}} \\ 
	 ~ Net food & 97 ~ \ \ & 99 ~ \ \ & \textit{107} ~ \ \ \\ 
	 ~ Net crop & 85 ~ \ \ & 96 ~ \ \ & \textit{98} ~ \ \ \\ 
	 ~ Cereal & 72 ~ \ \ & 92 ~ \ \ & \textit{102} ~ \ \ \\ 
	 ~ Vegetable oils & 54 ~ \ \ & 74 ~ \ \ & \textit{110} ~ \ \ \\ 
	 ~ Roots and tubers & 96 ~ \ \ & 99 ~ \ \ & \textit{83} ~ \ \ \\ 
	 ~ Fruit and vegetables & 119 ~ \ \ & 112 ~ \ \ & \textit{94} ~ \ \ \\ 
	 ~ Sugar & 111 ~ \ \ & 110 ~ \ \ & \textit{94} ~ \ \ \\ 
	 ~ Livestock & 99 ~ \ \ & 99 ~ \ \ & \textit{110} ~ \ \ \\ 
	 ~ Milk & 99 ~ \ \ & 99 ~ \ \ & \textit{110} ~ \ \ \\ 
	 ~ Meat & 98 ~ \ \ & 99 ~ \ \ & \textit{111} ~ \ \ \\ 
	 ~ Fish  & 97 ~ \ \ & 84 ~ \ \ & \textit{79} ~ \ \ \\ 
	\multicolumn{4}{l}{\textit{\normalsize{Net trade (min US\$)}}} \\ 
	 ~ Cereals & 881 ~ \ \ & 1\,522 ~ \ \ & \textit{3\,535} ~ \ \ \\ 
	 ~ Fruit and vegetables & -10\,947 ~ \ \ & -7\,617 ~ \ \ & \textit{-13\,940} ~ \ \ \\ 
	 ~ Meat & -3\,888 ~ \ \ & -724 ~ \ \ & \textit{3\,092} ~ \ \ \\ 
	 ~ Dairy products & 1\,412 ~ \ \ & 635 ~ \ \ & \textit{3\,109} ~ \ \ \\ 
	 ~ Fish & -1\,498 ~ \ \ & -1\,263 ~ \ \ & \textit{-2\,549} ~ \ \ \\ 
	\multicolumn{4}{l}{\textcolor{FAOblue}{\textbf{\large{Environment}}}} \\ 
	 ~ Forest area (\%) & 31 ~ \ \ & 32 ~ \ \ & \textit{32} ~ \ \ \\ 
	 ~ Renewable water res withdrawn (\% of total) &  ~ \ \ &  ~ \ \ & 0 ~ \ \ \\ 
	 ~ Terrestrial protect areas (\% total land area)  & 33 ~ \ \ & 40 ~ \ \ & \textit{48} ~ \ \ \\ 
	 ~ Organic area (\% total agricultural area) &  ~ \ \ & \textit{5} ~ \ \ & \textit{6} ~ \ \ \\ 
	 ~ Water withdrawal by agriculture (\% of total) &  ~ \ \ &  ~ \ \ & 0 ~ \ \ \\ 
	 ~ Biofuel production (thousand kt of oil eq.) & 24 ~ \ \ & 68 ~ \ \ & \textit{77\,610} ~ \ \ \\ 
	 ~ Wood pellet prod. (min tonnes) &  ~ \ \ &  ~ \ \ & \textit{2\,208} ~ \ \ \\ 
	 ~ GHG emissions from ag (Co2 eq, gigagrams) & 5 ~ \ \ & 9 ~ \ \ & \textit{-19} ~ \ \ \\ 
       \toprule
      \end{tabular}
      \clearpage
\CountryData{ Ghana }
      \rowcolors{1}{FAOblue!10}{white}
      \begin{tabular}{L{3.9cm} R{1cm} R{1cm} R{1cm}}
      \toprule
      \multicolumn{1}{c}{} & \multicolumn{1}{c}{ 1992 } & \multicolumn{1}{c}{ 2002 } & \multicolumn{1}{c}{ 2014 } \\
      \midrule
	\multicolumn{4}{l}{\textcolor{FAOblue}{\textbf{\large{The setting}}}} \\ 
	 ~ Population, total (mln) & 15.5 ~ \ \ & 19.8 ~ \ \ & 26.4 ~ \ \ \\ 
	 ~ Population, rural (\% total population) & 9.6 ~ \ \ & 10.8 ~ \ \ & 12.2 ~ \ \ \\ 
	 ~ Govt expenditure on ag (\% total outlays) &  ~ \ \ & \textit{1.6} ~ \ \ & \textit{1.5} ~ \ \ \\ 
	 ~ Area harvested (mln ha) & 9 ~ \ \ & 16 ~ \ \ & 23 ~ \ \ \\ 
	 ~ Cropping intensity ratio (\%) & 0.7 ~ \ \ & 1.1 ~ \ \ &  ~ \ \ \\ 
	 ~ Water resources (m\textsuperscript{3}/person/year) & \textit{4} ~ \ \ & \textit{3} ~ \ \ & \textit{2} ~ \ \ \\ 
	 ~ Area equipped for irrigation (1000 ha) &  ~ \ \ &  ~ \ \ & \textit{34} ~ \ \ \\ 
	 ~ Area irrigated (\%) &  ~ \ \ & \textit{90.3} ~ \ \ &  ~ \ \ \\ 
	 ~ Employment in agriculture (\%) & 62 ~ \ \ & \textit{55} ~ \ \ & \textit{41.5} ~ \ \ \\ 
	 ~ Employment in agriculture, female (\%) & 58.7 ~ \ \ & \textit{50.3} ~ \ \ & \textit{37.7} ~ \ \ \\ 
	 ~ Fertilizers, Nitrogen (nutrients per ha) &  ~ \ \ & 0.3 ~ \ \ & \textit{2.7} ~ \ \ \\ 
	 ~ Fertilizers, Phosphate (nutrients per ha) &  ~ \ \ & 0 ~ \ \ & \textit{4.5} ~ \ \ \\ 
	 ~ Fertilizers, Potash (nutrients per ha) &  ~ \ \ & 0.8 ~ \ \ & \textit{3.2} ~ \ \ \\ 
	 ~ Energy consump, power irrigation (mln kWh) & \textit{1} ~ \ \ & 15 ~ \ \ & \textit{15} ~ \ \ \\ 
	 ~ Agr value added per worker (constant US\$) &  ~ \ \ & \textit{0.8} ~ \ \ & \textit{0.8} ~ \ \ \\ 
	\multicolumn{4}{l}{\textcolor{FAOblue}{\textbf{\large{Hunger dimensions}}}} \\ 
	 ~ Dietary energy supply (kcal/pc/day) & 2\,164 ~ \ \ & 2\,576 ~ \ \ & 3\,295 ~ \ \ \\ 
	 ~ Average dietary energy supply adequacy (\%) & 99 ~ \ \ & 116 ~ \ \ & 147 ~ \ \ \\ 
	 ~ Dietary en supp, cereals/roots/tubers (\%) & 70 ~ \ \ & 69 ~ \ \ & \textit{66} ~ \ \ \\ 
	 ~ Prevalence of undernourishment (\%) & 36.9 ~ \ \ & 14.6 ~ \ \ & <5.0 ~ \ \ \\ 
	 ~ GDP per capita (US\$, PPP) & 1\,956 ~ \ \ & 2\,316 ~ \ \ & \textit{3\,864} ~ \ \ \\ 
	 ~ Domestic food price volatility (index) &  ~ \ \ & 12.4 ~ \ \ & 18.3 ~ \ \ \\ 
	 ~ Cereal import dependency ratio (\%) & 23.4 ~ \ \ & 29.1 ~ \ \ & \textit{26.1} ~ \ \ \\ 
	 ~ Underweight, children under-5 (\%) & \textit{25.1} ~ \ \ & \textit{18.8} ~ \ \ & \textit{13.4} ~ \ \ \\ 
	 ~ Improved water source (\% pop) & 57.8 ~ \ \ & 73.7 ~ \ \ & \textit{87.2} ~ \ \ \\ 
	\multicolumn{4}{l}{\textcolor{FAOblue}{\textbf{\large{Food Supply}}}} \\ 
	 ~ Food production value, (2004-2006 mln I\$) & 2\,871 ~ \ \ & 4\,752 ~ \ \ & \textit{7\,581} ~ \ \ \\ 
	 ~ Agriculture, value added (\% GDP) & 45 ~ \ \ & 39 ~ \ \ & \textit{22} ~ \ \ \\ 
	 ~ Food exports (mln US\$)  & 310 ~ \ \ & 629 ~ \ \ & \textit{2\,514} ~ \ \ \\ 
	 ~ Food imports (mln US\$)  & 233 ~ \ \ & 414 ~ \ \ & \textit{1\,396} ~ \ \ \\ 
	\multicolumn{4}{l}{\textit{\normalsize{Production indices (2004-06=100)}}} \\ 
	 ~ Net food & 54 ~ \ \ & 90 ~ \ \ & \textit{144} ~ \ \ \\ 
	 ~ Net crop & 53 ~ \ \ & 90 ~ \ \ & \textit{144} ~ \ \ \\ 
	 ~ Cereal & 65 ~ \ \ & 113 ~ \ \ & \textit{151} ~ \ \ \\ 
	 ~ Vegetable oils & 55 ~ \ \ & 104 ~ \ \ & \textit{98} ~ \ \ \\ 
	 ~ Roots and tubers & 60 ~ \ \ & 100 ~ \ \ & \textit{156} ~ \ \ \\ 
	 ~ Fruit and vegetables & 48 ~ \ \ & 90 ~ \ \ & \textit{152} ~ \ \ \\ 
	 ~ Sugar & 79 ~ \ \ & 100 ~ \ \ & \textit{107} ~ \ \ \\ 
	 ~ Livestock & 81 ~ \ \ & 91 ~ \ \ & \textit{136} ~ \ \ \\ 
	 ~ Milk & 64 ~ \ \ & 96 ~ \ \ & \textit{114} ~ \ \ \\ 
	 ~ Meat & 84 ~ \ \ & 91 ~ \ \ & \textit{135} ~ \ \ \\ 
	 ~ Fish  & 106 ~ \ \ & 95 ~ \ \ & \textit{83} ~ \ \ \\ 
	\multicolumn{4}{l}{\textit{\normalsize{Net trade (min US\$)}}} \\ 
	 ~ Cereals & -102 ~ \ \ & -180 ~ \ \ & \textit{-528} ~ \ \ \\ 
	 ~ Fruit and vegetables & 1 ~ \ \ & 12 ~ \ \ & \textit{96} ~ \ \ \\ 
	 ~ Meat &  ~ \ \ & -28 ~ \ \ & \textit{-253} ~ \ \ \\ 
	 ~ Dairy products & -15 ~ \ \ & -33 ~ \ \ & \textit{-76} ~ \ \ \\ 
	 ~ Fish & -6 ~ \ \ & -49 ~ \ \ & \textit{-214} ~ \ \ \\ 
	\multicolumn{4}{l}{\textcolor{FAOblue}{\textbf{\large{Environment}}}} \\ 
	 ~ Forest area (\%) & 32 ~ \ \ & 26 ~ \ \ & \textit{21} ~ \ \ \\ 
	 ~ Renewable water res withdrawn (\% of total) &  ~ \ \ & \textit{66} ~ \ \ & 66 ~ \ \ \\ 
	 ~ Terrestrial protect areas (\% total land area)  & 15 ~ \ \ & 15 ~ \ \ & \textit{15} ~ \ \ \\ 
	 ~ Organic area (\% total agricultural area) &  ~ \ \ & \textit{0} ~ \ \ & \textit{0} ~ \ \ \\ 
	 ~ Water withdrawal by agriculture (\% of total) &  ~ \ \ & \textit{66} ~ \ \ & 66 ~ \ \ \\ 
	 ~ Biofuel production (thousand kt of oil eq.) & 0 ~ \ \ & \textit{0} ~ \ \ &  ~ \ \ \\ 
	 ~ Wood pellet prod. (min tonnes) &  ~ \ \ &  ~ \ \ &  ~ \ \ \\ 
	 ~ GHG emissions from ag (Co2 eq, gigagrams) & 43 ~ \ \ & 39 ~ \ \ & \textit{40} ~ \ \ \\ 
       \toprule
      \end{tabular}
      \clearpage
\CountryData{ Greece }
      \rowcolors{1}{FAOblue!10}{white}
      \begin{tabular}{L{3.9cm} R{1cm} R{1cm} R{1cm}}
      \toprule
      \multicolumn{1}{c}{} & \multicolumn{1}{c}{ 1992 } & \multicolumn{1}{c}{ 2002 } & \multicolumn{1}{c}{ 2014 } \\
      \midrule
	\multicolumn{4}{l}{\textcolor{FAOblue}{\textbf{\large{The setting}}}} \\ 
	 ~ Population, total (mln) & 10.3 ~ \ \ & 11 ~ \ \ & 11.1 ~ \ \ \\ 
	 ~ Population, rural (\% total population) & 4.2 ~ \ \ & 4.4 ~ \ \ & 4.2 ~ \ \ \\ 
	 ~ Govt expenditure on ag (\% total outlays) &  ~ \ \ &  ~ \ \ &  ~ \ \ \\ 
	 ~ Area harvested (mln ha) & 5 ~ \ \ & 5 ~ \ \ & 5 ~ \ \ \\ 
	 ~ Cropping intensity ratio (\%) & 0.5 ~ \ \ & 0.6 ~ \ \ &  ~ \ \ \\ 
	 ~ Water resources (m\textsuperscript{3}/person/year) & \textit{7} ~ \ \ & \textit{6} ~ \ \ & \textit{6} ~ \ \ \\ 
	 ~ Area equipped for irrigation (1000 ha) &  ~ \ \ &  ~ \ \ & \textit{1\,555} ~ \ \ \\ 
	 ~ Area irrigated (\%) &  ~ \ \ &  ~ \ \ & \textit{82.3} ~ \ \ \\ 
	 ~ Employment in agriculture (\%) & 21.9 ~ \ \ & 15.5 ~ \ \ & \textit{13} ~ \ \ \\ 
	 ~ Employment in agriculture, female (\%) & 26.3 ~ \ \ & 17.2 ~ \ \ & \textit{12.9} ~ \ \ \\ 
	 ~ Fertilizers, Nitrogen (nutrients per ha) &  ~ \ \ & 32.2 ~ \ \ & \textit{17.2} ~ \ \ \\ 
	 ~ Fertilizers, Phosphate (nutrients per ha) &  ~ \ \ & 14.3 ~ \ \ & \textit{11.1} ~ \ \ \\ 
	 ~ Fertilizers, Potash (nutrients per ha) &  ~ \ \ & 3.8 ~ \ \ & \textit{3.1} ~ \ \ \\ 
	 ~ Energy consump, power irrigation (mln kWh) &  ~ \ \ &  ~ \ \ &  ~ \ \ \\ 
	 ~ Agr value added per worker (constant US\$) & \textit{11.6} ~ \ \ & 12.9 ~ \ \ & \textit{13.3} ~ \ \ \\ 
	\multicolumn{4}{l}{\textcolor{FAOblue}{\textbf{\large{Hunger dimensions}}}} \\ 
	 ~ Dietary energy supply (kcal/pc/day) &  ~ \ \ &  ~ \ \ &  ~ \ \ \\ 
	 ~ Average dietary energy supply adequacy (\%) & 141 ~ \ \ & 143 ~ \ \ & 135 ~ \ \ \\ 
	 ~ Dietary en supp, cereals/roots/tubers (\%) & 33 ~ \ \ & 32 ~ \ \ & \textit{30} ~ \ \ \\ 
	 ~ Prevalence of undernourishment (\%) & <5.0 ~ \ \ & <5.0 ~ \ \ & <5.0 ~ \ \ \\ 
	 ~ GDP per capita (US\$, PPP) & 21\,442 ~ \ \ & 26\,625 ~ \ \ & \textit{24\,305} ~ \ \ \\ 
	 ~ Domestic food price volatility (index) &  ~ \ \ & 15.1 ~ \ \ & 11.2 ~ \ \ \\ 
	 ~ Cereal import dependency ratio (\%) & -16.4 ~ \ \ & 22.9 ~ \ \ & \textit{16.7} ~ \ \ \\ 
	 ~ Underweight, children under-5 (\%) &  ~ \ \ &  ~ \ \ &  ~ \ \ \\ 
	 ~ Improved water source (\% pop) & 96.8 ~ \ \ & 99.5 ~ \ \ & \textit{99.8} ~ \ \ \\ 
	\multicolumn{4}{l}{\textcolor{FAOblue}{\textbf{\large{Food Supply}}}} \\ 
	 ~ Food production value, (2004-2006 mln I\$) & 7\,433 ~ \ \ & 7\,477 ~ \ \ & \textit{6\,481} ~ \ \ \\ 
	 ~ Agriculture, value added (\% GDP) & \textit{8} ~ \ \ & 6 ~ \ \ & \textit{4} ~ \ \ \\ 
	 ~ Food exports (mln US\$)  & 2\,373 ~ \ \ & 1\,742 ~ \ \ & \textit{4\,293} ~ \ \ \\ 
	 ~ Food imports (mln US\$)  & 2\,523 ~ \ \ & 2\,778 ~ \ \ & \textit{5\,644} ~ \ \ \\ 
	\multicolumn{4}{l}{\textit{\normalsize{Production indices (2004-06=100)}}} \\ 
	 ~ Net food & 101 ~ \ \ & 101 ~ \ \ & \textit{88} ~ \ \ \\ 
	 ~ Net crop & 98 ~ \ \ & 104 ~ \ \ & \textit{84} ~ \ \ \\ 
	 ~ Cereal & 100 ~ \ \ & 97 ~ \ \ & \textit{92} ~ \ \ \\ 
	 ~ Vegetable oils & 77 ~ \ \ & 108 ~ \ \ & \textit{87} ~ \ \ \\ 
	 ~ Roots and tubers & 118 ~ \ \ & 101 ~ \ \ & \textit{89} ~ \ \ \\ 
	 ~ Fruit and vegetables & 112 ~ \ \ & 100 ~ \ \ & \textit{89} ~ \ \ \\ 
	 ~ Sugar & 133 ~ \ \ & 126 ~ \ \ & \textit{15} ~ \ \ \\ 
	 ~ Livestock & 99 ~ \ \ & 95 ~ \ \ & \textit{91} ~ \ \ \\ 
	 ~ Milk & 91 ~ \ \ & 101 ~ \ \ & \textit{90} ~ \ \ \\ 
	 ~ Meat & 104 ~ \ \ & 91 ~ \ \ & \textit{91} ~ \ \ \\ 
	 ~ Fish  & 86 ~ \ \ & 92 ~ \ \ & \textit{104} ~ \ \ \\ 
	\multicolumn{4}{l}{\textit{\normalsize{Net trade (min US\$)}}} \\ 
	 ~ Cereals & 166 ~ \ \ & -286 ~ \ \ & \textit{-388} ~ \ \ \\ 
	 ~ Fruit and vegetables & 941 ~ \ \ & 577 ~ \ \ & \textit{1\,480} ~ \ \ \\ 
	 ~ Meat & -920 ~ \ \ & -711 ~ \ \ & \textit{-1\,399} ~ \ \ \\ 
	 ~ Dairy products & -450 ~ \ \ & -389 ~ \ \ & \textit{-471} ~ \ \ \\ 
	 ~ Fish & -72 ~ \ \ & -154 ~ \ \ & \textit{152} ~ \ \ \\ 
	\multicolumn{4}{l}{\textcolor{FAOblue}{\textbf{\large{Environment}}}} \\ 
	 ~ Forest area (\%) & 26 ~ \ \ & 28 ~ \ \ & \textit{31} ~ \ \ \\ 
	 ~ Renewable water res withdrawn (\% of total) &  ~ \ \ &  ~ \ \ & 89 ~ \ \ \\ 
	 ~ Terrestrial protect areas (\% total land area)  & 6 ~ \ \ & 10 ~ \ \ & \textit{35} ~ \ \ \\ 
	 ~ Organic area (\% total agricultural area) &  ~ \ \ & \textit{3} ~ \ \ & \textit{6} ~ \ \ \\ 
	 ~ Water withdrawal by agriculture (\% of total) &  ~ \ \ &  ~ \ \ & 89 ~ \ \ \\ 
	 ~ Biofuel production (thousand kt of oil eq.) &  ~ \ \ & 12 ~ \ \ & \textit{3\,361} ~ \ \ \\ 
	 ~ Wood pellet prod. (min tonnes) &  ~ \ \ &  ~ \ \ & \textit{35} ~ \ \ \\ 
	 ~ GHG emissions from ag (Co2 eq, gigagrams) & 9 ~ \ \ & 8 ~ \ \ & \textit{6} ~ \ \ \\ 
       \toprule
      \end{tabular}
      \clearpage
\CountryData{ Guatemala }
      \rowcolors{1}{FAOblue!10}{white}
      \begin{tabular}{L{3.9cm} R{1cm} R{1cm} R{1cm}}
      \toprule
      \multicolumn{1}{c}{} & \multicolumn{1}{c}{ 1992 } & \multicolumn{1}{c}{ 2002 } & \multicolumn{1}{c}{ 2014 } \\
      \midrule
	\multicolumn{4}{l}{\textcolor{FAOblue}{\textbf{\large{The setting}}}} \\ 
	 ~ Population, total (mln) & 9.3 ~ \ \ & 11.8 ~ \ \ & 15.9 ~ \ \ \\ 
	 ~ Population, rural (\% total population) & 5.4 ~ \ \ & 6.4 ~ \ \ & 7.8 ~ \ \ \\ 
	 ~ Govt expenditure on ag (\% total outlays) &  ~ \ \ & 2.8 ~ \ \ & \textit{2.1} ~ \ \ \\ 
	 ~ Area harvested (mln ha) & 11 ~ \ \ & 17 ~ \ \ & 26 ~ \ \ \\ 
	 ~ Cropping intensity ratio (\%) & 2.6 ~ \ \ & 3.9 ~ \ \ &  ~ \ \ \\ 
	 ~ Water resources (m\textsuperscript{3}/person/year) & \textit{13} ~ \ \ & \textit{11} ~ \ \ & \textit{8} ~ \ \ \\ 
	 ~ Area equipped for irrigation (1000 ha) &  ~ \ \ &  ~ \ \ & \textit{338} ~ \ \ \\ 
	 ~ Area irrigated (\%) &  ~ \ \ & \textit{100} ~ \ \ &  ~ \ \ \\ 
	 ~ Employment in agriculture (\%) & \textit{13.6} ~ \ \ & 38.7 ~ \ \ & \textit{32.3} ~ \ \ \\ 
	 ~ Employment in agriculture, female (\%) & \textit{2.6} ~ \ \ & 18.1 ~ \ \ & \textit{12.6} ~ \ \ \\ 
	 ~ Fertilizers, Nitrogen (nutrients per ha) &  ~ \ \ & 18.6 ~ \ \ & \textit{40.2} ~ \ \ \\ 
	 ~ Fertilizers, Phosphate (nutrients per ha) &  ~ \ \ & 10.2 ~ \ \ & \textit{11.4} ~ \ \ \\ 
	 ~ Fertilizers, Potash (nutrients per ha) &  ~ \ \ & 2.6 ~ \ \ & \textit{3.3} ~ \ \ \\ 
	 ~ Energy consump, power irrigation (mln kWh) &  ~ \ \ & 0 ~ \ \ & \textit{246} ~ \ \ \\ 
	 ~ Agr value added per worker (constant US\$) & 1.6 ~ \ \ & 1.8 ~ \ \ & \textit{2} ~ \ \ \\ 
	\multicolumn{4}{l}{\textcolor{FAOblue}{\textbf{\large{Hunger dimensions}}}} \\ 
	 ~ Dietary energy supply (kcal/pc/day) & 2\,312 ~ \ \ & 2\,308 ~ \ \ & 2\,435 ~ \ \ \\ 
	 ~ Average dietary energy supply adequacy (\%) & 113 ~ \ \ & 112 ~ \ \ & 116 ~ \ \ \\ 
	 ~ Dietary en supp, cereals/roots/tubers (\%) & 59 ~ \ \ & 51 ~ \ \ & \textit{48} ~ \ \ \\ 
	 ~ Prevalence of undernourishment (\%) & 15.4 ~ \ \ & 18.7 ~ \ \ & 15.6 ~ \ \ \\ 
	 ~ GDP per capita (US\$, PPP) & 5\,514 ~ \ \ & 6\,390 ~ \ \ & \textit{7\,062} ~ \ \ \\ 
	 ~ Domestic food price volatility (index) &  ~ \ \ & 4.7 ~ \ \ & 5.5 ~ \ \ \\ 
	 ~ Cereal import dependency ratio (\%) & 22.3 ~ \ \ & 48 ~ \ \ & \textit{44.3} ~ \ \ \\ 
	 ~ Underweight, children under-5 (\%) & \textit{21.7} ~ \ \ & 17.7 ~ \ \ & \textit{13} ~ \ \ \\ 
	 ~ Improved water source (\% pop) & 82.6 ~ \ \ & 88.6 ~ \ \ & \textit{93.8} ~ \ \ \\ 
	\multicolumn{4}{l}{\textcolor{FAOblue}{\textbf{\large{Food Supply}}}} \\ 
	 ~ Food production value, (2004-2006 mln I\$) & 1\,658 ~ \ \ & 2\,615 ~ \ \ & \textit{4\,330} ~ \ \ \\ 
	 ~ Agriculture, value added (\% GDP) &  ~ \ \ & 15 ~ \ \ & \textit{11} ~ \ \ \\ 
	 ~ Food exports (mln US\$)  & 473 ~ \ \ & 913 ~ \ \ & \textit{3\,101} ~ \ \ \\ 
	 ~ Food imports (mln US\$)  & 217 ~ \ \ & 658 ~ \ \ & \textit{1\,749} ~ \ \ \\ 
	\multicolumn{4}{l}{\textit{\normalsize{Production indices (2004-06=100)}}} \\ 
	 ~ Net food & 58 ~ \ \ & 91 ~ \ \ & \textit{151} ~ \ \ \\ 
	 ~ Net crop & 64 ~ \ \ & 90 ~ \ \ & \textit{154} ~ \ \ \\ 
	 ~ Cereal & 127 ~ \ \ & 97 ~ \ \ & \textit{150} ~ \ \ \\ 
	 ~ Vegetable oils & 31 ~ \ \ & 84 ~ \ \ & \textit{327} ~ \ \ \\ 
	 ~ Roots and tubers & 43 ~ \ \ & 59 ~ \ \ & \textit{123} ~ \ \ \\ 
	 ~ Fruit and vegetables & 47 ~ \ \ & 90 ~ \ \ & \textit{169} ~ \ \ \\ 
	 ~ Sugar & 61 ~ \ \ & 94 ~ \ \ & \textit{142} ~ \ \ \\ 
	 ~ Livestock & 55 ~ \ \ & 95 ~ \ \ & \textit{121} ~ \ \ \\ 
	 ~ Milk & 61 ~ \ \ & 85 ~ \ \ & \textit{120} ~ \ \ \\ 
	 ~ Meat & 58 ~ \ \ & 98 ~ \ \ & \textit{120} ~ \ \ \\ 
	 ~ Fish  & 31 ~ \ \ & 124 ~ \ \ & \textit{153} ~ \ \ \\ 
	\multicolumn{4}{l}{\textit{\normalsize{Net trade (min US\$)}}} \\ 
	 ~ Cereals & -57 ~ \ \ & -178 ~ \ \ & \textit{-589} ~ \ \ \\ 
	 ~ Fruit and vegetables & 145 ~ \ \ & 306 ~ \ \ & \textit{1\,003} ~ \ \ \\ 
	 ~ Meat & 8 ~ \ \ & -36 ~ \ \ & \textit{-88} ~ \ \ \\ 
	 ~ Dairy products & -30 ~ \ \ & -63 ~ \ \ & \textit{-162} ~ \ \ \\ 
	 ~ Fish & 17 ~ \ \ & 15 ~ \ \ & \textit{20} ~ \ \ \\ 
	\multicolumn{4}{l}{\textcolor{FAOblue}{\textbf{\large{Environment}}}} \\ 
	 ~ Forest area (\%) & 43 ~ \ \ & 38 ~ \ \ & \textit{33} ~ \ \ \\ 
	 ~ Renewable water res withdrawn (\% of total) &  ~ \ \ &  ~ \ \ & 57 ~ \ \ \\ 
	 ~ Terrestrial protect areas (\% total land area)  & 26 ~ \ \ & 29 ~ \ \ & \textit{31} ~ \ \ \\ 
	 ~ Organic area (\% total agricultural area) &  ~ \ \ & \textit{0} ~ \ \ & \textit{0} ~ \ \ \\ 
	 ~ Water withdrawal by agriculture (\% of total) &  ~ \ \ &  ~ \ \ & 57 ~ \ \ \\ 
	 ~ Biofuel production (thousand kt of oil eq.) & 29 ~ \ \ & 76 ~ \ \ & \textit{109} ~ \ \ \\ 
	 ~ Wood pellet prod. (min tonnes) &  ~ \ \ &  ~ \ \ &  ~ \ \ \\ 
	 ~ GHG emissions from ag (Co2 eq, gigagrams) & 21 ~ \ \ & 21 ~ \ \ & \textit{24} ~ \ \ \\ 
       \toprule
      \end{tabular}
      \clearpage
\CountryData{ Guinea }
      \rowcolors{1}{FAOblue!10}{white}
      \begin{tabular}{L{3.9cm} R{1cm} R{1cm} R{1cm}}
      \toprule
      \multicolumn{1}{c}{} & \multicolumn{1}{c}{ 1992 } & \multicolumn{1}{c}{ 2002 } & \multicolumn{1}{c}{ 2014 } \\
      \midrule
	\multicolumn{4}{l}{\textcolor{FAOblue}{\textbf{\large{The setting}}}} \\ 
	 ~ Population, total (mln) & 6.7 ~ \ \ & 9 ~ \ \ & 12 ~ \ \ \\ 
	 ~ Population, rural (\% total population) & 4.8 ~ \ \ & 6.2 ~ \ \ & 7.6 ~ \ \ \\ 
	 ~ Govt expenditure on ag (\% total outlays) &  ~ \ \ &  ~ \ \ &  ~ \ \ \\ 
	 ~ Area harvested (mln ha) & 1 ~ \ \ & 2 ~ \ \ & 3 ~ \ \ \\ 
	 ~ Cropping intensity ratio (\%) & 0.1 ~ \ \ & 0.1 ~ \ \ &  ~ \ \ \\ 
	 ~ Water resources (m\textsuperscript{3}/person/year) & \textit{32} ~ \ \ & \textit{25} ~ \ \ & \textit{19} ~ \ \ \\ 
	 ~ Area equipped for irrigation (1000 ha) &  ~ \ \ &  ~ \ \ & \textit{95} ~ \ \ \\ 
	 ~ Area irrigated (\%) &  ~ \ \ & \textit{100} ~ \ \ &  ~ \ \ \\ 
	 ~ Employment in agriculture (\%) & \textit{76} ~ \ \ &  ~ \ \ &  ~ \ \ \\ 
	 ~ Employment in agriculture, female (\%) & \textit{78.8} ~ \ \ &  ~ \ \ &  ~ \ \ \\ 
	 ~ Fertilizers, Nitrogen (nutrients per ha) &  ~ \ \ & 0.1 ~ \ \ & \textit{0.4} ~ \ \ \\ 
	 ~ Fertilizers, Phosphate (nutrients per ha) &  ~ \ \ & 0 ~ \ \ & \textit{0.1} ~ \ \ \\ 
	 ~ Fertilizers, Potash (nutrients per ha) &  ~ \ \ & 0 ~ \ \ & \textit{0.1} ~ \ \ \\ 
	 ~ Energy consump, power irrigation (mln kWh) & \textit{4} ~ \ \ & 1 ~ \ \ & \textit{1} ~ \ \ \\ 
	 ~ Agr value added per worker (constant US\$) & 0.1 ~ \ \ & 0.2 ~ \ \ & \textit{0.2} ~ \ \ \\ 
	\multicolumn{4}{l}{\textcolor{FAOblue}{\textbf{\large{Hunger dimensions}}}} \\ 
	 ~ Dietary energy supply (kcal/pc/day) & 2\,435 ~ \ \ & 2\,364 ~ \ \ & 2\,604 ~ \ \ \\ 
	 ~ Average dietary energy supply adequacy (\%) & 113 ~ \ \ & 109 ~ \ \ & 117 ~ \ \ \\ 
	 ~ Dietary en supp, cereals/roots/tubers (\%) & 62 ~ \ \ & 62 ~ \ \ & \textit{62} ~ \ \ \\ 
	 ~ Prevalence of undernourishment (\%) & 22.8 ~ \ \ & 25.5 ~ \ \ & 16.8 ~ \ \ \\ 
	 ~ GDP per capita (US\$, PPP) & 1\,068 ~ \ \ & 1\,200 ~ \ \ & \textit{1\,213} ~ \ \ \\ 
	 ~ Domestic food price volatility (index) &  ~ \ \ & 16.9 ~ \ \ & \textit{7.3} ~ \ \ \\ 
	 ~ Cereal import dependency ratio (\%) & 26.5 ~ \ \ & 22.5 ~ \ \ & \textit{13.8} ~ \ \ \\ 
	 ~ Underweight, children under-5 (\%) & \textit{21.2} ~ \ \ & \textit{22.5} ~ \ \ & \textit{16.3} ~ \ \ \\ 
	 ~ Improved water source (\% pop) & 54.5 ~ \ \ & 64.8 ~ \ \ & \textit{74.8} ~ \ \ \\ 
	\multicolumn{4}{l}{\textcolor{FAOblue}{\textbf{\large{Food Supply}}}} \\ 
	 ~ Food production value, (2004-2006 mln I\$) & 1\,078 ~ \ \ & 1\,440 ~ \ \ & \textit{1\,988} ~ \ \ \\ 
	 ~ Agriculture, value added (\% GDP) & 17 ~ \ \ & 23 ~ \ \ & \textit{20} ~ \ \ \\ 
	 ~ Food exports (mln US\$)  & 23 ~ \ \ & 23 ~ \ \ & \textit{58} ~ \ \ \\ 
	 ~ Food imports (mln US\$)  & 140 ~ \ \ & 129 ~ \ \ & \textit{490} ~ \ \ \\ 
	\multicolumn{4}{l}{\textit{\normalsize{Production indices (2004-06=100)}}} \\ 
	 ~ Net food & 67 ~ \ \ & 89 ~ \ \ & \textit{124} ~ \ \ \\ 
	 ~ Net crop & 68 ~ \ \ & 89 ~ \ \ & \textit{121} ~ \ \ \\ 
	 ~ Cereal & 52 ~ \ \ & 82 ~ \ \ & \textit{156} ~ \ \ \\ 
	 ~ Vegetable oils & 63 ~ \ \ & 90 ~ \ \ & \textit{97} ~ \ \ \\ 
	 ~ Roots and tubers & 74 ~ \ \ & 91 ~ \ \ & \textit{117} ~ \ \ \\ 
	 ~ Fruit and vegetables & 84 ~ \ \ & 97 ~ \ \ & \textit{102} ~ \ \ \\ 
	 ~ Sugar & 71 ~ \ \ & 96 ~ \ \ & \textit{107} ~ \ \ \\ 
	 ~ Livestock & 49 ~ \ \ & 84 ~ \ \ & \textit{132} ~ \ \ \\ 
	 ~ Milk & 51 ~ \ \ & 84 ~ \ \ & \textit{131} ~ \ \ \\ 
	 ~ Meat & 49 ~ \ \ & 84 ~ \ \ & \textit{132} ~ \ \ \\ 
	 ~ Fish  & 53 ~ \ \ & 90 ~ \ \ & \textit{122} ~ \ \ \\ 
	\multicolumn{4}{l}{\textit{\normalsize{Net trade (min US\$)}}} \\ 
	 ~ Cereals & -62 ~ \ \ & -62 ~ \ \ & \textit{-224} ~ \ \ \\ 
	 ~ Fruit and vegetables & -3 ~ \ \ & -9 ~ \ \ & \textit{-19} ~ \ \ \\ 
	 ~ Meat & \textit{-5} ~ \ \ & -1 ~ \ \ & \textit{-22} ~ \ \ \\ 
	 ~ Dairy products & \textit{-13} ~ \ \ & -6 ~ \ \ & \textit{-41} ~ \ \ \\ 
	 ~ Fish & -11 ~ \ \ & 2 ~ \ \ & \textit{4} ~ \ \ \\ 
	\multicolumn{4}{l}{\textcolor{FAOblue}{\textbf{\large{Environment}}}} \\ 
	 ~ Forest area (\%) & 29 ~ \ \ & 28 ~ \ \ & \textit{26} ~ \ \ \\ 
	 ~ Renewable water res withdrawn (\% of total) &  ~ \ \ & \textit{53} ~ \ \ & 53 ~ \ \ \\ 
	 ~ Terrestrial protect areas (\% total land area)  & 7 ~ \ \ & 7 ~ \ \ & \textit{28} ~ \ \ \\ 
	 ~ Organic area (\% total agricultural area) &  ~ \ \ &  ~ \ \ &  ~ \ \ \\ 
	 ~ Water withdrawal by agriculture (\% of total) &  ~ \ \ & \textit{53} ~ \ \ & 53 ~ \ \ \\ 
	 ~ Biofuel production (thousand kt of oil eq.) & 0 ~ \ \ & 1 ~ \ \ & \textit{1} ~ \ \ \\ 
	 ~ Wood pellet prod. (min tonnes) &  ~ \ \ &  ~ \ \ &  ~ \ \ \\ 
	 ~ GHG emissions from ag (Co2 eq, gigagrams) & 19 ~ \ \ & 22 ~ \ \ & \textit{25} ~ \ \ \\ 
       \toprule
      \end{tabular}
      \clearpage
\CountryData{ Guinea-Bissau }
      \rowcolors{1}{FAOblue!10}{white}
      \begin{tabular}{L{3.9cm} R{1cm} R{1cm} R{1cm}}
      \toprule
      \multicolumn{1}{c}{} & \multicolumn{1}{c}{ 1992 } & \multicolumn{1}{c}{ 2002 } & \multicolumn{1}{c}{ 2014 } \\
      \midrule
	\multicolumn{4}{l}{\textcolor{FAOblue}{\textbf{\large{The setting}}}} \\ 
	 ~ Population, total (mln) & 1.1 ~ \ \ & 1.3 ~ \ \ & 1.7 ~ \ \ \\ 
	 ~ Population, rural (\% total population) & 0.7 ~ \ \ & 0.8 ~ \ \ & 0.9 ~ \ \ \\ 
	 ~ Govt expenditure on ag (\% total outlays) &  ~ \ \ &  ~ \ \ &  ~ \ \ \\ 
	 ~ Area harvested (mln ha) & 0 ~ \ \ & 0 ~ \ \ & 0 ~ \ \ \\ 
	 ~ Cropping intensity ratio (\%) & 0.2 ~ \ \ & 0.2 ~ \ \ &  ~ \ \ \\ 
	 ~ Water resources (m\textsuperscript{3}/person/year) & \textit{29} ~ \ \ & \textit{23} ~ \ \ & \textit{18} ~ \ \ \\ 
	 ~ Area equipped for irrigation (1000 ha) &  ~ \ \ &  ~ \ \ & \textit{25} ~ \ \ \\ 
	 ~ Area irrigated (\%) &  ~ \ \ & \textit{100} ~ \ \ &  ~ \ \ \\ 
	 ~ Employment in agriculture (\%) &  ~ \ \ &  ~ \ \ &  ~ \ \ \\ 
	 ~ Employment in agriculture, female (\%) &  ~ \ \ &  ~ \ \ &  ~ \ \ \\ 
	 ~ Fertilizers, Nitrogen (nutrients per ha) &  ~ \ \ &  ~ \ \ &  ~ \ \ \\ 
	 ~ Fertilizers, Phosphate (nutrients per ha) &  ~ \ \ &  ~ \ \ &  ~ \ \ \\ 
	 ~ Fertilizers, Potash (nutrients per ha) &  ~ \ \ &  ~ \ \ &  ~ \ \ \\ 
	 ~ Energy consump, power irrigation (mln kWh) &  ~ \ \ &  ~ \ \ &  ~ \ \ \\ 
	 ~ Agr value added per worker (constant US\$) &  ~ \ \ & 0.6 ~ \ \ & \textit{0.6} ~ \ \ \\ 
	\multicolumn{4}{l}{\textcolor{FAOblue}{\textbf{\large{Hunger dimensions}}}} \\ 
	 ~ Dietary energy supply (kcal/pc/day) & 2\,333 ~ \ \ & 2\,254 ~ \ \ & 2\,371 ~ \ \ \\ 
	 ~ Average dietary energy supply adequacy (\%) & 108 ~ \ \ & 103 ~ \ \ & 107 ~ \ \ \\ 
	 ~ Dietary en supp, cereals/roots/tubers (\%) & 69 ~ \ \ & 70 ~ \ \ & \textit{64} ~ \ \ \\ 
	 ~ Prevalence of undernourishment (\%) & 21.8 ~ \ \ & 25.5 ~ \ \ & 22 ~ \ \ \\ 
	 ~ GDP per capita (US\$, PPP) & 1\,588 ~ \ \ & 1\,287 ~ \ \ & \textit{1\,362} ~ \ \ \\ 
	 ~ Domestic food price volatility (index) &  ~ \ \ &  ~ \ \ &  ~ \ \ \\ 
	 ~ Cereal import dependency ratio (\%) & 34.8 ~ \ \ & 43.2 ~ \ \ & \textit{31.4} ~ \ \ \\ 
	 ~ Underweight, children under-5 (\%) &  ~ \ \ & \textit{21.9} ~ \ \ & \textit{18.1} ~ \ \ \\ 
	 ~ Improved water source (\% pop) & 38.9 ~ \ \ & 55.3 ~ \ \ & \textit{73.6} ~ \ \ \\ 
	\multicolumn{4}{l}{\textcolor{FAOblue}{\textbf{\large{Food Supply}}}} \\ 
	 ~ Food production value, (2004-2006 mln I\$) & 151 ~ \ \ & 213 ~ \ \ & \textit{333} ~ \ \ \\ 
	 ~ Agriculture, value added (\% GDP) & 49 ~ \ \ & 43 ~ \ \ & \textit{44} ~ \ \ \\ 
	 ~ Food exports (mln US\$)  & 4 ~ \ \ & 44 ~ \ \ & \textit{113} ~ \ \ \\ 
	 ~ Food imports (mln US\$)  & 33 ~ \ \ & 36 ~ \ \ & \textit{71} ~ \ \ \\ 
	\multicolumn{4}{l}{\textit{\normalsize{Production indices (2004-06=100)}}} \\ 
	 ~ Net food & 64 ~ \ \ & 90 ~ \ \ & \textit{141} ~ \ \ \\ 
	 ~ Net crop & 61 ~ \ \ & 90 ~ \ \ & \textit{143} ~ \ \ \\ 
	 ~ Cereal & 94 ~ \ \ & 78 ~ \ \ & \textit{150} ~ \ \ \\ 
	 ~ Vegetable oils & 79 ~ \ \ & 90 ~ \ \ & \textit{155} ~ \ \ \\ 
	 ~ Roots and tubers & 67 ~ \ \ & 96 ~ \ \ & \textit{119} ~ \ \ \\ 
	 ~ Fruit and vegetables & 78 ~ \ \ & 96 ~ \ \ & \textit{125} ~ \ \ \\ 
	 ~ Sugar & 100 ~ \ \ & 100 ~ \ \ & \textit{115} ~ \ \ \\ 
	 ~ Livestock & 76 ~ \ \ & 92 ~ \ \ & \textit{129} ~ \ \ \\ 
	 ~ Milk & 89 ~ \ \ & 96 ~ \ \ & \textit{125} ~ \ \ \\ 
	 ~ Meat & 74 ~ \ \ & 91 ~ \ \ & \textit{131} ~ \ \ \\ 
	 ~ Fish  &  ~ \ \ &  ~ \ \ &  ~ \ \ \\ 
	\multicolumn{4}{l}{\textit{\normalsize{Net trade (min US\$)}}} \\ 
	 ~ Cereals & -23 ~ \ \ & -26 ~ \ \ & \textit{-37} ~ \ \ \\ 
	 ~ Fruit and vegetables & 2 ~ \ \ & 43 ~ \ \ & \textit{108} ~ \ \ \\ 
	 ~ Meat & -2 ~ \ \ & 0 ~ \ \ & \textit{-2} ~ \ \ \\ 
	 ~ Dairy products & -2 ~ \ \ & -1 ~ \ \ & \textit{-4} ~ \ \ \\ 
	 ~ Fish & 0 ~ \ \ & 5 ~ \ \ & \textit{2} ~ \ \ \\ 
	\multicolumn{4}{l}{\textcolor{FAOblue}{\textbf{\large{Environment}}}} \\ 
	 ~ Forest area (\%) & 78 ~ \ \ & 75 ~ \ \ & \textit{71} ~ \ \ \\ 
	 ~ Renewable water res withdrawn (\% of total) &  ~ \ \ & \textit{82} ~ \ \ & 82 ~ \ \ \\ 
	 ~ Terrestrial protect areas (\% total land area)  & 8 ~ \ \ & 16 ~ \ \ & \textit{16} ~ \ \ \\ 
	 ~ Organic area (\% total agricultural area) &  ~ \ \ &  ~ \ \ & \textit{0} ~ \ \ \\ 
	 ~ Water withdrawal by agriculture (\% of total) &  ~ \ \ & \textit{82} ~ \ \ & 82 ~ \ \ \\ 
	 ~ Biofuel production (thousand kt of oil eq.) &  ~ \ \ &  ~ \ \ &  ~ \ \ \\ 
	 ~ Wood pellet prod. (min tonnes) &  ~ \ \ &  ~ \ \ &  ~ \ \ \\ 
	 ~ GHG emissions from ag (Co2 eq, gigagrams) & 3 ~ \ \ & 3 ~ \ \ & \textit{4} ~ \ \ \\ 
       \toprule
      \end{tabular}
      \clearpage
\CountryData{ Guyana }
      \rowcolors{1}{FAOblue!10}{white}
      \begin{tabular}{L{3.9cm} R{1cm} R{1cm} R{1cm}}
      \toprule
      \multicolumn{1}{c}{} & \multicolumn{1}{c}{ 1992 } & \multicolumn{1}{c}{ 2002 } & \multicolumn{1}{c}{ 2014 } \\
      \midrule
	\multicolumn{4}{l}{\textcolor{FAOblue}{\textbf{\large{The setting}}}} \\ 
	 ~ Population, total (mln) & 0.7 ~ \ \ & 0.8 ~ \ \ & 0.8 ~ \ \ \\ 
	 ~ Population, rural (\% total population) & 0.5 ~ \ \ & 0.5 ~ \ \ & 0.6 ~ \ \ \\ 
	 ~ Govt expenditure on ag (\% total outlays) &  ~ \ \ &  ~ \ \ &  ~ \ \ \\ 
	 ~ Area harvested (mln ha) & 3 ~ \ \ & 4 ~ \ \ & 2 ~ \ \ \\ 
	 ~ Cropping intensity ratio (\%) & 1.8 ~ \ \ & 2.2 ~ \ \ &  ~ \ \ \\ 
	 ~ Water resources (m\textsuperscript{3}/person/year) & \textit{374} ~ \ \ & \textit{359} ~ \ \ & \textit{339} ~ \ \ \\ 
	 ~ Area equipped for irrigation (1000 ha) &  ~ \ \ &  ~ \ \ & \textit{143} ~ \ \ \\ 
	 ~ Area irrigated (\%) &  ~ \ \ &  ~ \ \ & \textit{89.2} ~ \ \ \\ 
	 ~ Employment in agriculture (\%) &  ~ \ \ & 21.4 ~ \ \ &  ~ \ \ \\ 
	 ~ Employment in agriculture, female (\%) &  ~ \ \ & 7.1 ~ \ \ &  ~ \ \ \\ 
	 ~ Fertilizers, Nitrogen (nutrients per ha) &  ~ \ \ & 8 ~ \ \ & \textit{5.1} ~ \ \ \\ 
	 ~ Fertilizers, Phosphate (nutrients per ha) &  ~ \ \ & 0.7 ~ \ \ & \textit{1} ~ \ \ \\ 
	 ~ Fertilizers, Potash (nutrients per ha) &  ~ \ \ & 0 ~ \ \ & \textit{0.3} ~ \ \ \\ 
	 ~ Energy consump, power irrigation (mln kWh) & 0 ~ \ \ & 0 ~ \ \ & \textit{0} ~ \ \ \\ 
	 ~ Agr value added per worker (constant US\$) & 3.7 ~ \ \ & 4.9 ~ \ \ & \textit{5.7} ~ \ \ \\ 
	\multicolumn{4}{l}{\textcolor{FAOblue}{\textbf{\large{Hunger dimensions}}}} \\ 
	 ~ Dietary energy supply (kcal/pc/day) & 2\,414 ~ \ \ & 2\,732 ~ \ \ & 2\,727 ~ \ \ \\ 
	 ~ Average dietary energy supply adequacy (\%) & 104 ~ \ \ & 118 ~ \ \ & 117 ~ \ \ \\ 
	 ~ Dietary en supp, cereals/roots/tubers (\%) & 56 ~ \ \ & 50 ~ \ \ & \textit{50} ~ \ \ \\ 
	 ~ Prevalence of undernourishment (\%) & 22 ~ \ \ & 9.3 ~ \ \ & 11.2 ~ \ \ \\ 
	 ~ GDP per capita (US\$, PPP) & 3\,703 ~ \ \ & 5\,197 ~ \ \ & \textit{6\,336} ~ \ \ \\ 
	 ~ Domestic food price volatility (index) &  ~ \ \ &  ~ \ \ &  ~ \ \ \\ 
	 ~ Cereal import dependency ratio (\%) & -28.4 ~ \ \ & -36.5 ~ \ \ & \textit{-21.2} ~ \ \ \\ 
	 ~ Underweight, children under-5 (\%) & \textit{16.1} ~ \ \ & \textit{11.9} ~ \ \ & \textit{11.1} ~ \ \ \\ 
	 ~ Improved water source (\% pop) & 78.1 ~ \ \ & 88.2 ~ \ \ & \textit{97.6} ~ \ \ \\ 
	\multicolumn{4}{l}{\textcolor{FAOblue}{\textbf{\large{Food Supply}}}} \\ 
	 ~ Food production value, (2004-2006 mln I\$) & 226 ~ \ \ & 320 ~ \ \ & \textit{410} ~ \ \ \\ 
	 ~ Agriculture, value added (\% GDP) & 41 ~ \ \ & 31 ~ \ \ & \textit{22} ~ \ \ \\ 
	 ~ Food exports (mln US\$)  & 168 ~ \ \ & 144 ~ \ \ & \textit{387} ~ \ \ \\ 
	 ~ Food imports (mln US\$)  & 37 ~ \ \ & 75 ~ \ \ & \textit{202} ~ \ \ \\ 
	\multicolumn{4}{l}{\textit{\normalsize{Production indices (2004-06=100)}}} \\ 
	 ~ Net food & 72 ~ \ \ & 101 ~ \ \ & \textit{130} ~ \ \ \\ 
	 ~ Net crop & 78 ~ \ \ & 105 ~ \ \ & \textit{129} ~ \ \ \\ 
	 ~ Cereal & 62 ~ \ \ & 96 ~ \ \ & \textit{179} ~ \ \ \\ 
	 ~ Vegetable oils & 60 ~ \ \ & 86 ~ \ \ & \textit{90} ~ \ \ \\ 
	 ~ Roots and tubers & 82 ~ \ \ & 99 ~ \ \ & \textit{75} ~ \ \ \\ 
	 ~ Fruit and vegetables & 93 ~ \ \ & 121 ~ \ \ & \textit{137} ~ \ \ \\ 
	 ~ Sugar & 95 ~ \ \ & 114 ~ \ \ & \textit{76} ~ \ \ \\ 
	 ~ Livestock & 48 ~ \ \ & 80 ~ \ \ & \textit{126} ~ \ \ \\ 
	 ~ Milk & 54 ~ \ \ & 91 ~ \ \ & \textit{125} ~ \ \ \\ 
	 ~ Meat & 47 ~ \ \ & 76 ~ \ \ & \textit{126} ~ \ \ \\ 
	 ~ Fish  & 75 ~ \ \ & 88 ~ \ \ & \textit{90} ~ \ \ \\ 
	\multicolumn{4}{l}{\textit{\normalsize{Net trade (min US\$)}}} \\ 
	 ~ Cereals & 21 ~ \ \ & 21 ~ \ \ & \textit{125} ~ \ \ \\ 
	 ~ Fruit and vegetables & -3 ~ \ \ & -6 ~ \ \ & \textit{-17} ~ \ \ \\ 
	 ~ Meat & -9 ~ \ \ & -4 ~ \ \ & \textit{-8} ~ \ \ \\ 
	 ~ Dairy products & -5 ~ \ \ & -15 ~ \ \ & \textit{-41} ~ \ \ \\ 
	 ~ Fish & 19 ~ \ \ & 52 ~ \ \ & \textit{74} ~ \ \ \\ 
	\multicolumn{4}{l}{\textcolor{FAOblue}{\textbf{\large{Environment}}}} \\ 
	 ~ Forest area (\%) & 77 ~ \ \ & 77 ~ \ \ & \textit{77} ~ \ \ \\ 
	 ~ Renewable water res withdrawn (\% of total) &  ~ \ \ &  ~ \ \ & 94 ~ \ \ \\ 
	 ~ Terrestrial protect areas (\% total land area)  & 3 ~ \ \ & 5 ~ \ \ & \textit{5} ~ \ \ \\ 
	 ~ Organic area (\% total agricultural area) &  ~ \ \ & \textit{0} ~ \ \ & \textit{0} ~ \ \ \\ 
	 ~ Water withdrawal by agriculture (\% of total) &  ~ \ \ &  ~ \ \ & 94 ~ \ \ \\ 
	 ~ Biofuel production (thousand kt of oil eq.) & 6 ~ \ \ & 9 ~ \ \ & \textit{7} ~ \ \ \\ 
	 ~ Wood pellet prod. (min tonnes) &  ~ \ \ &  ~ \ \ &  ~ \ \ \\ 
	 ~ GHG emissions from ag (Co2 eq, gigagrams) & 5 ~ \ \ & 5 ~ \ \ & \textit{6} ~ \ \ \\ 
       \toprule
      \end{tabular}
      \clearpage
\CountryData{ Haiti }
      \rowcolors{1}{FAOblue!10}{white}
      \begin{tabular}{L{3.9cm} R{1cm} R{1cm} R{1cm}}
      \toprule
      \multicolumn{1}{c}{} & \multicolumn{1}{c}{ 1992 } & \multicolumn{1}{c}{ 2002 } & \multicolumn{1}{c}{ 2014 } \\
      \midrule
	\multicolumn{4}{l}{\textcolor{FAOblue}{\textbf{\large{The setting}}}} \\ 
	 ~ Population, total (mln) & 7.4 ~ \ \ & 8.9 ~ \ \ & 10.5 ~ \ \ \\ 
	 ~ Population, rural (\% total population) & 5.1 ~ \ \ & 5.4 ~ \ \ & 4.5 ~ \ \ \\ 
	 ~ Govt expenditure on ag (\% total outlays) &  ~ \ \ &  ~ \ \ &  ~ \ \ \\ 
	 ~ Area harvested (mln ha) & 1 ~ \ \ & 2 ~ \ \ & 2 ~ \ \ \\ 
	 ~ Cropping intensity ratio (\%) & 0.9 ~ \ \ & 1.2 ~ \ \ &  ~ \ \ \\ 
	 ~ Water resources (m\textsuperscript{3}/person/year) & \textit{2} ~ \ \ & \textit{2} ~ \ \ & \textit{1} ~ \ \ \\ 
	 ~ Area equipped for irrigation (1000 ha) &  ~ \ \ &  ~ \ \ & \textit{97} ~ \ \ \\ 
	 ~ Area irrigated (\%) &  ~ \ \ &  ~ \ \ & \textit{82} ~ \ \ \\ 
	 ~ Employment in agriculture (\%) & \textit{65.6} ~ \ \ & \textit{50.5} ~ \ \ &  ~ \ \ \\ 
	 ~ Employment in agriculture, female (\%) & \textit{49.6} ~ \ \ & \textit{37.3} ~ \ \ &  ~ \ \ \\ 
	 ~ Fertilizers, Nitrogen (nutrients per ha) &  ~ \ \ &  ~ \ \ &  ~ \ \ \\ 
	 ~ Fertilizers, Phosphate (nutrients per ha) &  ~ \ \ &  ~ \ \ &  ~ \ \ \\ 
	 ~ Fertilizers, Potash (nutrients per ha) &  ~ \ \ &  ~ \ \ &  ~ \ \ \\ 
	 ~ Energy consump, power irrigation (mln kWh) & 0 ~ \ \ & 0 ~ \ \ & \textit{0} ~ \ \ \\ 
	 ~ Agr value added per worker (constant US\$) &  ~ \ \ &  ~ \ \ &  ~ \ \ \\ 
	\multicolumn{4}{l}{\textcolor{FAOblue}{\textbf{\large{Hunger dimensions}}}} \\ 
	 ~ Dietary energy supply (kcal/pc/day) & 1\,731 ~ \ \ & 1\,890 ~ \ \ & 2\,014 ~ \ \ \\ 
	 ~ Average dietary energy supply adequacy (\%) & 79 ~ \ \ & 84 ~ \ \ & 87 ~ \ \ \\ 
	 ~ Dietary en supp, cereals/roots/tubers (\%) & 57 ~ \ \ & 54 ~ \ \ & \textit{52} ~ \ \ \\ 
	 ~ Prevalence of undernourishment (\%) & 61.5 ~ \ \ & 56.1 ~ \ \ & 52.3 ~ \ \ \\ 
	 ~ GDP per capita (US\$, PPP) &  ~ \ \ & 1\,657 ~ \ \ & \textit{1\,648} ~ \ \ \\ 
	 ~ Domestic food price volatility (index) &  ~ \ \ & 4.6 ~ \ \ & 3.4 ~ \ \ \\ 
	 ~ Cereal import dependency ratio (\%) & 48.5 ~ \ \ & 63.1 ~ \ \ & \textit{51.8} ~ \ \ \\ 
	 ~ Underweight, children under-5 (\%) & \textit{24} ~ \ \ & \textit{13.9} ~ \ \ & \textit{11.6} ~ \ \ \\ 
	 ~ Improved water source (\% pop) & 61.7 ~ \ \ & 61.3 ~ \ \ & \textit{62.4} ~ \ \ \\ 
	\multicolumn{4}{l}{\textcolor{FAOblue}{\textbf{\large{Food Supply}}}} \\ 
	 ~ Food production value, (2004-2006 mln I\$) & 806 ~ \ \ & 848 ~ \ \ & \textit{1\,092} ~ \ \ \\ 
	 ~ Agriculture, value added (\% GDP) &  ~ \ \ &  ~ \ \ &  ~ \ \ \\ 
	 ~ Food exports (mln US\$)  & 6 ~ \ \ & 13 ~ \ \ & \textit{18} ~ \ \ \\ 
	 ~ Food imports (mln US\$)  & 220 ~ \ \ & 335 ~ \ \ & \textit{911} ~ \ \ \\ 
	\multicolumn{4}{l}{\textit{\normalsize{Production indices (2004-06=100)}}} \\ 
	 ~ Net food & 89 ~ \ \ & 94 ~ \ \ & \textit{121} ~ \ \ \\ 
	 ~ Net crop & 97 ~ \ \ & 92 ~ \ \ & \textit{122} ~ \ \ \\ 
	 ~ Cereal & 113 ~ \ \ & 94 ~ \ \ & \textit{153} ~ \ \ \\ 
	 ~ Vegetable oils & 98 ~ \ \ & 84 ~ \ \ & \textit{98} ~ \ \ \\ 
	 ~ Roots and tubers & 95 ~ \ \ & 89 ~ \ \ & \textit{181} ~ \ \ \\ 
	 ~ Fruit and vegetables & 90 ~ \ \ & 98 ~ \ \ & \textit{90} ~ \ \ \\ 
	 ~ Sugar & 135 ~ \ \ & 56 ~ \ \ & \textit{113} ~ \ \ \\ 
	 ~ Livestock & 66 ~ \ \ & 99 ~ \ \ & \textit{107} ~ \ \ \\ 
	 ~ Milk & 89 ~ \ \ & 96 ~ \ \ & \textit{132} ~ \ \ \\ 
	 ~ Meat & 63 ~ \ \ & 99 ~ \ \ & \textit{104} ~ \ \ \\ 
	 ~ Fish  & 45 ~ \ \ & 74 ~ \ \ & \textit{157} ~ \ \ \\ 
	\multicolumn{4}{l}{\textit{\normalsize{Net trade (min US\$)}}} \\ 
	 ~ Cereals & -112 ~ \ \ & -146 ~ \ \ &  ~ \ \ \\ 
	 ~ Fruit and vegetables & -7 ~ \ \ & -19 ~ \ \ & \textit{-35} ~ \ \ \\ 
	 ~ Meat & -1 ~ \ \ & -26 ~ \ \ & \textit{-103} ~ \ \ \\ 
	 ~ Dairy products &  ~ \ \ & -28 ~ \ \ & \textit{-76} ~ \ \ \\ 
	 ~ Fish & -2 ~ \ \ & -1 ~ \ \ & \textit{-24} ~ \ \ \\ 
	\multicolumn{4}{l}{\textcolor{FAOblue}{\textbf{\large{Environment}}}} \\ 
	 ~ Forest area (\%) & 4 ~ \ \ & 4 ~ \ \ & \textit{4} ~ \ \ \\ 
	 ~ Renewable water res withdrawn (\% of total) &  ~ \ \ &  ~ \ \ & 83 ~ \ \ \\ 
	 ~ Terrestrial protect areas (\% total land area)  & 0 ~ \ \ & 0 ~ \ \ & \textit{0} ~ \ \ \\ 
	 ~ Organic area (\% total agricultural area) &  ~ \ \ &  ~ \ \ & \textit{0} ~ \ \ \\ 
	 ~ Water withdrawal by agriculture (\% of total) &  ~ \ \ &  ~ \ \ & 83 ~ \ \ \\ 
	 ~ Biofuel production (thousand kt of oil eq.) & 3 ~ \ \ & 3 ~ \ \ & \textit{0} ~ \ \ \\ 
	 ~ Wood pellet prod. (min tonnes) &  ~ \ \ &  ~ \ \ &  ~ \ \ \\ 
	 ~ GHG emissions from ag (Co2 eq, gigagrams) & 3 ~ \ \ & 4 ~ \ \ & \textit{4} ~ \ \ \\ 
       \toprule
      \end{tabular}
      \clearpage
\CountryData{ Honduras }
      \rowcolors{1}{FAOblue!10}{white}
      \begin{tabular}{L{3.9cm} R{1cm} R{1cm} R{1cm}}
      \toprule
      \multicolumn{1}{c}{} & \multicolumn{1}{c}{ 1992 } & \multicolumn{1}{c}{ 2002 } & \multicolumn{1}{c}{ 2014 } \\
      \midrule
	\multicolumn{4}{l}{\textcolor{FAOblue}{\textbf{\large{The setting}}}} \\ 
	 ~ Population, total (mln) & 5.2 ~ \ \ & 6.5 ~ \ \ & 8.3 ~ \ \ \\ 
	 ~ Population, rural (\% total population) & 3 ~ \ \ & 3.5 ~ \ \ & 3.8 ~ \ \ \\ 
	 ~ Govt expenditure on ag (\% total outlays) &  ~ \ \ &  ~ \ \ &  ~ \ \ \\ 
	 ~ Area harvested (mln ha) & 3 ~ \ \ & 4 ~ \ \ & 6 ~ \ \ \\ 
	 ~ Cropping intensity ratio (\%) & 0.8 ~ \ \ & 1.2 ~ \ \ &  ~ \ \ \\ 
	 ~ Water resources (m\textsuperscript{3}/person/year) & \textit{17} ~ \ \ & \textit{14} ~ \ \ & \textit{11} ~ \ \ \\ 
	 ~ Area equipped for irrigation (1000 ha) &  ~ \ \ &  ~ \ \ & \textit{90} ~ \ \ \\ 
	 ~ Area irrigated (\%) &  ~ \ \ &  ~ \ \ & \textit{92.9} ~ \ \ \\ 
	 ~ Employment in agriculture (\%) & 38.2 ~ \ \ & 38.8 ~ \ \ & \textit{35.3} ~ \ \ \\ 
	 ~ Employment in agriculture, female (\%) & 6.2 ~ \ \ & 8.9 ~ \ \ & \textit{9.4} ~ \ \ \\ 
	 ~ Fertilizers, Nitrogen (nutrients per ha) &  ~ \ \ & 10.7 ~ \ \ & \textit{22.8} ~ \ \ \\ 
	 ~ Fertilizers, Phosphate (nutrients per ha) &  ~ \ \ & 0.4 ~ \ \ & \textit{1.6} ~ \ \ \\ 
	 ~ Fertilizers, Potash (nutrients per ha) &  ~ \ \ & 0.2 ~ \ \ & \textit{1.9} ~ \ \ \\ 
	 ~ Energy consump, power irrigation (mln kWh) & 0 ~ \ \ & 0 ~ \ \ & \textit{0} ~ \ \ \\ 
	 ~ Agr value added per worker (constant US\$) & 1.3 ~ \ \ & 1.6 ~ \ \ & \textit{2.6} ~ \ \ \\ 
	\multicolumn{4}{l}{\textcolor{FAOblue}{\textbf{\large{Hunger dimensions}}}} \\ 
	 ~ Dietary energy supply (kcal/pc/day) & 2\,309 ~ \ \ & 2\,470 ~ \ \ & 2\,723 ~ \ \ \\ 
	 ~ Average dietary energy supply adequacy (\%) & 111 ~ \ \ & 116 ~ \ \ & 122 ~ \ \ \\ 
	 ~ Dietary en supp, cereals/roots/tubers (\%) & 50 ~ \ \ & 48 ~ \ \ & \textit{47} ~ \ \ \\ 
	 ~ Prevalence of undernourishment (\%) & 22.7 ~ \ \ & 17.8 ~ \ \ & 12.3 ~ \ \ \\ 
	 ~ GDP per capita (US\$, PPP) & 3\,308 ~ \ \ & 3\,563 ~ \ \ & \textit{4\,445} ~ \ \ \\ 
	 ~ Domestic food price volatility (index) &  ~ \ \ & 5.4 ~ \ \ & 4.8 ~ \ \ \\ 
	 ~ Cereal import dependency ratio (\%) & 23 ~ \ \ & 49.2 ~ \ \ & \textit{56.5} ~ \ \ \\ 
	 ~ Underweight, children under-5 (\%) & 15.8 ~ \ \ & \textit{12.5} ~ \ \ & \textit{7.1} ~ \ \ \\ 
	 ~ Improved water source (\% pop) & 74.5 ~ \ \ & 82.4 ~ \ \ & \textit{89.6} ~ \ \ \\ 
	\multicolumn{4}{l}{\textcolor{FAOblue}{\textbf{\large{Food Supply}}}} \\ 
	 ~ Food production value, (2004-2006 mln I\$) & 1\,001 ~ \ \ & 1\,200 ~ \ \ & \textit{1\,755} ~ \ \ \\ 
	 ~ Agriculture, value added (\% GDP) & 20 ~ \ \ & 13 ~ \ \ & \textit{13} ~ \ \ \\ 
	 ~ Food exports (mln US\$)  & 368 ~ \ \ & 297 ~ \ \ & \textit{923} ~ \ \ \\ 
	 ~ Food imports (mln US\$)  & 78 ~ \ \ & 372 ~ \ \ & \textit{1\,083} ~ \ \ \\ 
	\multicolumn{4}{l}{\textit{\normalsize{Production indices (2004-06=100)}}} \\ 
	 ~ Net food & 68 ~ \ \ & 81 ~ \ \ & \textit{119} ~ \ \ \\ 
	 ~ Net crop & 71 ~ \ \ & 79 ~ \ \ & \textit{126} ~ \ \ \\ 
	 ~ Cereal & 124 ~ \ \ & 101 ~ \ \ & \textit{126} ~ \ \ \\ 
	 ~ Vegetable oils & 37 ~ \ \ & 58 ~ \ \ & \textit{189} ~ \ \ \\ 
	 ~ Roots and tubers & 60 ~ \ \ & 72 ~ \ \ & \textit{126} ~ \ \ \\ 
	 ~ Fruit and vegetables & 84 ~ \ \ & 77 ~ \ \ & \textit{114} ~ \ \ \\ 
	 ~ Sugar & 51 ~ \ \ & 66 ~ \ \ & \textit{110} ~ \ \ \\ 
	 ~ Livestock & 56 ~ \ \ & 85 ~ \ \ & \textit{109} ~ \ \ \\ 
	 ~ Milk & 58 ~ \ \ & 85 ~ \ \ & \textit{101} ~ \ \ \\ 
	 ~ Meat & 53 ~ \ \ & 84 ~ \ \ & \textit{114} ~ \ \ \\ 
	 ~ Fish  & 47 ~ \ \ & 50 ~ \ \ & \textit{126} ~ \ \ \\ 
	\multicolumn{4}{l}{\textit{\normalsize{Net trade (min US\$)}}} \\ 
	 ~ Cereals & -29 ~ \ \ & -138 ~ \ \ & \textit{-411} ~ \ \ \\ 
	 ~ Fruit and vegetables & 319 ~ \ \ & 153 ~ \ \ & \textit{263} ~ \ \ \\ 
	 ~ Meat & 18 ~ \ \ & -18 ~ \ \ & \textit{-38} ~ \ \ \\ 
	 ~ Dairy products & -11 ~ \ \ & -25 ~ \ \ & \textit{-41} ~ \ \ \\ 
	 ~ Fish & 30 ~ \ \ & 125 ~ \ \ & \textit{241} ~ \ \ \\ 
	\multicolumn{4}{l}{\textcolor{FAOblue}{\textbf{\large{Environment}}}} \\ 
	 ~ Forest area (\%) & 70 ~ \ \ & 55 ~ \ \ & \textit{44} ~ \ \ \\ 
	 ~ Renewable water res withdrawn (\% of total) &  ~ \ \ & \textit{73} ~ \ \ & 73 ~ \ \ \\ 
	 ~ Terrestrial protect areas (\% total land area)  & 15 ~ \ \ & 18 ~ \ \ & \textit{21} ~ \ \ \\ 
	 ~ Organic area (\% total agricultural area) &  ~ \ \ & \textit{0} ~ \ \ & \textit{1} ~ \ \ \\ 
	 ~ Water withdrawal by agriculture (\% of total) &  ~ \ \ & \textit{73} ~ \ \ & 73 ~ \ \ \\ 
	 ~ Biofuel production (thousand kt of oil eq.) & 5 ~ \ \ & 15 ~ \ \ & \textit{23} ~ \ \ \\ 
	 ~ Wood pellet prod. (min tonnes) &  ~ \ \ &  ~ \ \ & \textit{4} ~ \ \ \\ 
	 ~ GHG emissions from ag (Co2 eq, gigagrams) & 45 ~ \ \ & 33 ~ \ \ & \textit{34} ~ \ \ \\ 
       \toprule
      \end{tabular}
      \clearpage
\CountryData{ Hungary }
      \rowcolors{1}{FAOblue!10}{white}
      \begin{tabular}{L{3.9cm} R{1cm} R{1cm} R{1cm}}
      \toprule
      \multicolumn{1}{c}{} & \multicolumn{1}{c}{ 1992 } & \multicolumn{1}{c}{ 2002 } & \multicolumn{1}{c}{ 2014 } \\
      \midrule
	\multicolumn{4}{l}{\textcolor{FAOblue}{\textbf{\large{The setting}}}} \\ 
	 ~ Population, total (mln) & 10.4 ~ \ \ & 10.2 ~ \ \ & 9.9 ~ \ \ \\ 
	 ~ Population, rural (\% total population) & 3.6 ~ \ \ & 3.6 ~ \ \ & 2.9 ~ \ \ \\ 
	 ~ Govt expenditure on ag (\% total outlays) &  ~ \ \ &  ~ \ \ &  ~ \ \ \\ 
	 ~ Area harvested (mln ha) & 10 ~ \ \ & 12 ~ \ \ & 14 ~ \ \ \\ 
	 ~ Cropping intensity ratio (\%) & 1.6 ~ \ \ & 2 ~ \ \ &  ~ \ \ \\ 
	 ~ Water resources (m\textsuperscript{3}/person/year) & \textit{10} ~ \ \ & \textit{10} ~ \ \ & \textit{10} ~ \ \ \\ 
	 ~ Area equipped for irrigation (1000 ha) &  ~ \ \ &  ~ \ \ & \textit{172} ~ \ \ \\ 
	 ~ Area irrigated (\%) &  ~ \ \ &  ~ \ \ & \textit{62.2} ~ \ \ \\ 
	 ~ Employment in agriculture (\%) & 11.3 ~ \ \ & 6.2 ~ \ \ & \textit{5.2} ~ \ \ \\ 
	 ~ Employment in agriculture, female (\%) & \textit{4.7} ~ \ \ & 3.6 ~ \ \ & \textit{2.9} ~ \ \ \\ 
	 ~ Fertilizers, Nitrogen (nutrients per ha) &  ~ \ \ & 51.8 ~ \ \ & \textit{56.7} ~ \ \ \\ 
	 ~ Fertilizers, Phosphate (nutrients per ha) &  ~ \ \ & 10.6 ~ \ \ & \textit{11.6} ~ \ \ \\ 
	 ~ Fertilizers, Potash (nutrients per ha) &  ~ \ \ & 12.3 ~ \ \ & \textit{11.7} ~ \ \ \\ 
	 ~ Energy consump, power irrigation (mln kWh) & 7 ~ \ \ & 7 ~ \ \ & \textit{235} ~ \ \ \\ 
	 ~ Agr value added per worker (constant US\$) & \textit{5.6} ~ \ \ & 6.8 ~ \ \ & \textit{11.5} ~ \ \ \\ 
	\multicolumn{4}{l}{\textcolor{FAOblue}{\textbf{\large{Hunger dimensions}}}} \\ 
	 ~ Dietary energy supply (kcal/pc/day) &  ~ \ \ &  ~ \ \ &  ~ \ \ \\ 
	 ~ Average dietary energy supply adequacy (\%) & 128 ~ \ \ & 121 ~ \ \ & 111 ~ \ \ \\ 
	 ~ Dietary en supp, cereals/roots/tubers (\%) & 29 ~ \ \ & 28 ~ \ \ & \textit{30} ~ \ \ \\ 
	 ~ Prevalence of undernourishment (\%) & <5.0 ~ \ \ & <5.0 ~ \ \ & <5.0 ~ \ \ \\ 
	 ~ GDP per capita (US\$, PPP) & 14\,474 ~ \ \ & 19\,292 ~ \ \ & \textit{22\,707} ~ \ \ \\ 
	 ~ Domestic food price volatility (index) &  ~ \ \ & 7.1 ~ \ \ & 5.8 ~ \ \ \\ 
	 ~ Cereal import dependency ratio (\%) & -19.4 ~ \ \ & -38.3 ~ \ \ & \textit{-81.1} ~ \ \ \\ 
	 ~ Underweight, children under-5 (\%) &  ~ \ \ &  ~ \ \ &  ~ \ \ \\ 
	 ~ Improved water source (\% pop) & 96.3 ~ \ \ & 99.6 ~ \ \ & \textit{100} ~ \ \ \\ 
	\multicolumn{4}{l}{\textcolor{FAOblue}{\textbf{\large{Food Supply}}}} \\ 
	 ~ Food production value, (2004-2006 mln I\$) & 5\,741 ~ \ \ & 5\,328 ~ \ \ & \textit{5\,189} ~ \ \ \\ 
	 ~ Agriculture, value added (\% GDP) & \textit{8} ~ \ \ & 5 ~ \ \ & \textit{4} ~ \ \ \\ 
	 ~ Food exports (mln US\$)  & 2\,267 ~ \ \ & 2\,193 ~ \ \ & \textit{7\,988} ~ \ \ \\ 
	 ~ Food imports (mln US\$)  & 336 ~ \ \ & 735 ~ \ \ & \textit{3\,590} ~ \ \ \\ 
	\multicolumn{4}{l}{\textit{\normalsize{Production indices (2004-06=100)}}} \\ 
	 ~ Net food & 98 ~ \ \ & 91 ~ \ \ & \textit{89} ~ \ \ \\ 
	 ~ Net crop & 79 ~ \ \ & 80 ~ \ \ & \textit{87} ~ \ \ \\ 
	 ~ Cereal & 61 ~ \ \ & 74 ~ \ \ & \textit{86} ~ \ \ \\ 
	 ~ Vegetable oils & 57 ~ \ \ & 68 ~ \ \ & \textit{134} ~ \ \ \\ 
	 ~ Roots and tubers & 183 ~ \ \ & 116 ~ \ \ & \textit{66} ~ \ \ \\ 
	 ~ Fruit and vegetables & 103 ~ \ \ & 90 ~ \ \ & \textit{84} ~ \ \ \\ 
	 ~ Sugar & 93 ~ \ \ & 72 ~ \ \ & \textit{30} ~ \ \ \\ 
	 ~ Livestock & 135 ~ \ \ & 117 ~ \ \ & \textit{88} ~ \ \ \\ 
	 ~ Milk & 122 ~ \ \ & 113 ~ \ \ & \textit{93} ~ \ \ \\ 
	 ~ Meat & 143 ~ \ \ & 121 ~ \ \ & \textit{86} ~ \ \ \\ 
	 ~ Fish  & 108 ~ \ \ & 87 ~ \ \ & \textit{101} ~ \ \ \\ 
	\multicolumn{4}{l}{\textit{\normalsize{Net trade (min US\$)}}} \\ 
	 ~ Cereals & 500 ~ \ \ & 318 ~ \ \ & \textit{1\,593} ~ \ \ \\ 
	 ~ Fruit and vegetables & 356 ~ \ \ & 176 ~ \ \ & \textit{288} ~ \ \ \\ 
	 ~ Meat & 658 ~ \ \ & 557 ~ \ \ & \textit{760} ~ \ \ \\ 
	 ~ Dairy products & 32 ~ \ \ & 38 ~ \ \ & \textit{-39} ~ \ \ \\ 
	 ~ Fish & -25 ~ \ \ & -52 ~ \ \ & \textit{-60} ~ \ \ \\ 
	\multicolumn{4}{l}{\textcolor{FAOblue}{\textbf{\large{Environment}}}} \\ 
	 ~ Forest area (\%) & 20 ~ \ \ & 22 ~ \ \ & \textit{23} ~ \ \ \\ 
	 ~ Renewable water res withdrawn (\% of total) &  ~ \ \ &  ~ \ \ & 6 ~ \ \ \\ 
	 ~ Terrestrial protect areas (\% total land area)  & 5 ~ \ \ & 5 ~ \ \ & \textit{23} ~ \ \ \\ 
	 ~ Organic area (\% total agricultural area) &  ~ \ \ & \textit{2} ~ \ \ & \textit{2} ~ \ \ \\ 
	 ~ Water withdrawal by agriculture (\% of total) &  ~ \ \ &  ~ \ \ & 6 ~ \ \ \\ 
	 ~ Biofuel production (thousand kt of oil eq.) &  ~ \ \ & 16 ~ \ \ & \textit{3\,843} ~ \ \ \\ 
	 ~ Wood pellet prod. (min tonnes) &  ~ \ \ &  ~ \ \ & \textit{27} ~ \ \ \\ 
	 ~ GHG emissions from ag (Co2 eq, gigagrams) & 11 ~ \ \ & 11 ~ \ \ & \textit{10} ~ \ \ \\ 
       \toprule
      \end{tabular}
      \clearpage
\CountryData{ Iceland }
      \rowcolors{1}{FAOblue!10}{white}
      \begin{tabular}{L{3.9cm} R{1cm} R{1cm} R{1cm}}
      \toprule
      \multicolumn{1}{c}{} & \multicolumn{1}{c}{ 1992 } & \multicolumn{1}{c}{ 2002 } & \multicolumn{1}{c}{ 2014 } \\
      \midrule
	\multicolumn{4}{l}{\textcolor{FAOblue}{\textbf{\large{The setting}}}} \\ 
	 ~ Population, total (mln) & 0.3 ~ \ \ & 0.3 ~ \ \ & 0.3 ~ \ \ \\ 
	 ~ Population, rural (\% total population) & 0 ~ \ \ & 0 ~ \ \ & 0 ~ \ \ \\ 
	 ~ Govt expenditure on ag (\% total outlays) &  ~ \ \ &  ~ \ \ &  ~ \ \ \\ 
	 ~ Area harvested (mln ha) & 1 ~ \ \ & 3 ~ \ \ & 0 ~ \ \ \\ 
	 ~ Cropping intensity ratio (\%) & 0.5 ~ \ \ & 1.8 ~ \ \ &  ~ \ \ \\ 
	 ~ Water resources (m\textsuperscript{3}/person/year) & \textit{649} ~ \ \ & \textit{586} ~ \ \ & \textit{515} ~ \ \ \\ 
	 ~ Area equipped for irrigation (1000 ha) &  ~ \ \ &  ~ \ \ &  ~ \ \ \\ 
	 ~ Area irrigated (\%) &  ~ \ \ &  ~ \ \ &  ~ \ \ \\ 
	 ~ Employment in agriculture (\%) & 10.4 ~ \ \ & 7.2 ~ \ \ & \textit{5.5} ~ \ \ \\ 
	 ~ Employment in agriculture, female (\%) & 4.5 ~ \ \ & 3.5 ~ \ \ & \textit{2.1} ~ \ \ \\ 
	 ~ Fertilizers, Nitrogen (nutrients per ha) &  ~ \ \ & 5.9 ~ \ \ & \textit{5} ~ \ \ \\ 
	 ~ Fertilizers, Phosphate (nutrients per ha) &  ~ \ \ & 2.5 ~ \ \ & \textit{3.8} ~ \ \ \\ 
	 ~ Fertilizers, Potash (nutrients per ha) &  ~ \ \ & 1.6 ~ \ \ & \textit{2.6} ~ \ \ \\ 
	 ~ Energy consump, power irrigation (mln kWh) &  ~ \ \ &  ~ \ \ &  ~ \ \ \\ 
	 ~ Agr value added per worker (constant US\$) &  ~ \ \ & 63 ~ \ \ & \textit{72.6} ~ \ \ \\ 
	\multicolumn{4}{l}{\textcolor{FAOblue}{\textbf{\large{Hunger dimensions}}}} \\ 
	 ~ Dietary energy supply (kcal/pc/day) &  ~ \ \ &  ~ \ \ &  ~ \ \ \\ 
	 ~ Average dietary energy supply adequacy (\%) & 124 ~ \ \ & 127 ~ \ \ & 132 ~ \ \ \\ 
	 ~ Dietary en supp, cereals/roots/tubers (\%) & 27 ~ \ \ & 22 ~ \ \ & \textit{23} ~ \ \ \\ 
	 ~ Prevalence of undernourishment (\%) & <5.0 ~ \ \ & <5.0 ~ \ \ & <5.0 ~ \ \ \\ 
	 ~ GDP per capita (US\$, PPP) & 26\,970 ~ \ \ & 34\,243 ~ \ \ & \textit{40\,789} ~ \ \ \\ 
	 ~ Domestic food price volatility (index) &  ~ \ \ & 8.3 ~ \ \ & 5.4 ~ \ \ \\ 
	 ~ Cereal import dependency ratio (\%) &  ~ \ \ &  ~ \ \ &  ~ \ \ \\ 
	 ~ Underweight, children under-5 (\%) &  ~ \ \ &  ~ \ \ &  ~ \ \ \\ 
	 ~ Improved water source (\% pop) & 100 ~ \ \ & 100 ~ \ \ & \textit{100} ~ \ \ \\ 
	\multicolumn{4}{l}{\textcolor{FAOblue}{\textbf{\large{Food Supply}}}} \\ 
	 ~ Food production value, (2004-2006 mln I\$) & 83 ~ \ \ & 94 ~ \ \ & \textit{110} ~ \ \ \\ 
	 ~ Agriculture, value added (\% GDP) &  ~ \ \ & 9 ~ \ \ & \textit{8} ~ \ \ \\ 
	 ~ Food exports (mln US\$)  & 10 ~ \ \ & 13 ~ \ \ & \textit{50} ~ \ \ \\ 
	 ~ Food imports (mln US\$)  & 103 ~ \ \ & 134 ~ \ \ & \textit{292} ~ \ \ \\ 
	\multicolumn{4}{l}{\textit{\normalsize{Production indices (2004-06=100)}}} \\ 
	 ~ Net food & 87 ~ \ \ & 98 ~ \ \ & \textit{114} ~ \ \ \\ 
	 ~ Net crop & 64 ~ \ \ & 74 ~ \ \ & \textit{79} ~ \ \ \\ 
	 ~ Cereal &  ~ \ \ &  ~ \ \ &  ~ \ \ \\ 
	 ~ Vegetable oils &  ~ \ \ &  ~ \ \ &  ~ \ \ \\ 
	 ~ Roots and tubers & 82 ~ \ \ & 63 ~ \ \ & \textit{44} ~ \ \ \\ 
	 ~ Fruit and vegetables & 43 ~ \ \ & 87 ~ \ \ & \textit{120} ~ \ \ \\ 
	 ~ Sugar &  ~ \ \ &  ~ \ \ &  ~ \ \ \\ 
	 ~ Livestock & 89 ~ \ \ & 99 ~ \ \ & \textit{115} ~ \ \ \\ 
	 ~ Milk & 101 ~ \ \ & 98 ~ \ \ & \textit{109} ~ \ \ \\ 
	 ~ Meat & 79 ~ \ \ & 100 ~ \ \ & \textit{120} ~ \ \ \\ 
	 ~ Fish  & 100 ~ \ \ & 135 ~ \ \ & \textit{87} ~ \ \ \\ 
	\multicolumn{4}{l}{\textit{\normalsize{Net trade (min US\$)}}} \\ 
	 ~ Cereals & -28 ~ \ \ & -39 ~ \ \ & \textit{-80} ~ \ \ \\ 
	 ~ Fruit and vegetables & -36 ~ \ \ & -42 ~ \ \ & \textit{-96} ~ \ \ \\ 
	 ~ Meat & 5 ~ \ \ & 5 ~ \ \ & \textit{20} ~ \ \ \\ 
	 ~ Dairy products & 1 ~ \ \ & 0 ~ \ \ & \textit{5} ~ \ \ \\ 
	 ~ Fish & 1\,238 ~ \ \ & 1\,356 ~ \ \ & \textit{2\,093} ~ \ \ \\ 
	\multicolumn{4}{l}{\textcolor{FAOblue}{\textbf{\large{Environment}}}} \\ 
	 ~ Forest area (\%) & 0 ~ \ \ & 0 ~ \ \ & \textit{0} ~ \ \ \\ 
	 ~ Renewable water res withdrawn (\% of total) &  ~ \ \ & \textit{42} ~ \ \ & 42 ~ \ \ \\ 
	 ~ Terrestrial protect areas (\% total land area)  & 10 ~ \ \ & 10 ~ \ \ & \textit{20} ~ \ \ \\ 
	 ~ Organic area (\% total agricultural area) &  ~ \ \ & \textit{0} ~ \ \ & \textit{1} ~ \ \ \\ 
	 ~ Water withdrawal by agriculture (\% of total) &  ~ \ \ & \textit{42} ~ \ \ & 42 ~ \ \ \\ 
	 ~ Biofuel production (thousand kt of oil eq.) &  ~ \ \ & \textit{0} ~ \ \ & \textit{0} ~ \ \ \\ 
	 ~ Wood pellet prod. (min tonnes) &  ~ \ \ &  ~ \ \ & \textit{0} ~ \ \ \\ 
	 ~ GHG emissions from ag (Co2 eq, gigagrams) & 0 ~ \ \ & 0 ~ \ \ & \textit{0} ~ \ \ \\ 
       \toprule
      \end{tabular}
      \clearpage
\CountryData{ India }
      \rowcolors{1}{FAOblue!10}{white}
      \begin{tabular}{L{3.9cm} R{1cm} R{1cm} R{1cm}}
      \toprule
      \multicolumn{1}{c}{} & \multicolumn{1}{c}{ 1992 } & \multicolumn{1}{c}{ 2002 } & \multicolumn{1}{c}{ 2014 } \\
      \midrule
	\multicolumn{4}{l}{\textcolor{FAOblue}{\textbf{\large{The setting}}}} \\ 
	 ~ Population, total (mln) & 903.8 ~ \ \ & 1\,076.7 ~ \ \ & 1\,267.4 ~ \ \ \\ 
	 ~ Population, rural (\% total population) & 668.9 ~ \ \ & 772.6 ~ \ \ & 857.1 ~ \ \ \\ 
	 ~ Govt expenditure on ag (\% total outlays) &  ~ \ \ & 4.1 ~ \ \ & \textit{6.5} ~ \ \ \\ 
	 ~ Area harvested (mln ha) & 254 ~ \ \ & 297 ~ \ \ & 341 ~ \ \ \\ 
	 ~ Cropping intensity ratio (\%) & 1.4 ~ \ \ & 1.6 ~ \ \ &  ~ \ \ \\ 
	 ~ Water resources (m\textsuperscript{3}/person/year) & \textit{2} ~ \ \ & \textit{2} ~ \ \ & \textit{2} ~ \ \ \\ 
	 ~ Area equipped for irrigation (1000 ha) &  ~ \ \ &  ~ \ \ & \textit{66\,700} ~ \ \ \\ 
	 ~ Area irrigated (\%) &  ~ \ \ &  ~ \ \ & \textit{93.9} ~ \ \ \\ 
	 ~ Employment in agriculture (\%) & \textit{60.5} ~ \ \ & \textit{55.8} ~ \ \ & \textit{47.2} ~ \ \ \\ 
	 ~ Employment in agriculture, female (\%) & \textit{72.4} ~ \ \ & \textit{70.9} ~ \ \ & \textit{59.8} ~ \ \ \\ 
	 ~ Fertilizers, Nitrogen (nutrients per ha) &  ~ \ \ & 58 ~ \ \ & \textit{94} ~ \ \ \\ 
	 ~ Fertilizers, Phosphate (nutrients per ha) &  ~ \ \ & 22.3 ~ \ \ & \textit{37.5} ~ \ \ \\ 
	 ~ Fertilizers, Potash (nutrients per ha) &  ~ \ \ & 8.8 ~ \ \ & \textit{11.1} ~ \ \ \\ 
	 ~ Energy consump, power irrigation (mln kWh) & 170 ~ \ \ & 4\,881 ~ \ \ & \textit{4\,881} ~ \ \ \\ 
	 ~ Agr value added per worker (constant US\$) & 0.5 ~ \ \ & 0.5 ~ \ \ & \textit{0.7} ~ \ \ \\ 
	\multicolumn{4}{l}{\textcolor{FAOblue}{\textbf{\large{Hunger dimensions}}}} \\ 
	 ~ Dietary energy supply (kcal/pc/day) & 2\,299 ~ \ \ & 2\,300 ~ \ \ & 2\,462 ~ \ \ \\ 
	 ~ Average dietary energy supply adequacy (\%) & 106 ~ \ \ & 104 ~ \ \ & 108 ~ \ \ \\ 
	 ~ Dietary en supp, cereals/roots/tubers (\%) & 67 ~ \ \ & 62 ~ \ \ & \textit{58} ~ \ \ \\ 
	 ~ Prevalence of undernourishment (\%) & 22.2 ~ \ \ & 18.6 ~ \ \ & 15.3 ~ \ \ \\ 
	 ~ GDP per capita (US\$, PPP) & 1\,821 ~ \ \ & 2\,684 ~ \ \ & \textit{5\,244} ~ \ \ \\ 
	 ~ Domestic food price volatility (index) &  ~ \ \ & 5.8 ~ \ \ & 8.4 ~ \ \ \\ 
	 ~ Cereal import dependency ratio (\%) & -0.2 ~ \ \ & -4.4 ~ \ \ & \textit{-3.1} ~ \ \ \\ 
	 ~ Underweight, children under-5 (\%) & 50.7 ~ \ \ & \textit{40.3} ~ \ \ & \textit{43.5} ~ \ \ \\ 
	 ~ Improved water source (\% pop) & 72.4 ~ \ \ & 82.7 ~ \ \ & \textit{92.6} ~ \ \ \\ 
	\multicolumn{4}{l}{\textcolor{FAOblue}{\textbf{\large{Food Supply}}}} \\ 
	 ~ Food production value, (2004-2006 mln I\$) & 126\,543 ~ \ \ & 148\,929 ~ \ \ & \textit{236\,540} ~ \ \ \\ 
	 ~ Agriculture, value added (\% GDP) & 29 ~ \ \ & 21 ~ \ \ & \textit{18} ~ \ \ \\ 
	 ~ Food exports (mln US\$)  & 1\,302 ~ \ \ & 3\,796 ~ \ \ & \textit{20\,835} ~ \ \ \\ 
	 ~ Food imports (mln US\$)  & 933 ~ \ \ & 3\,051 ~ \ \ & \textit{16\,523} ~ \ \ \\ 
	\multicolumn{4}{l}{\textit{\normalsize{Production indices (2004-06=100)}}} \\ 
	 ~ Net food & 74 ~ \ \ & 87 ~ \ \ & \textit{139} ~ \ \ \\ 
	 ~ Net crop & 77 ~ \ \ & 85 ~ \ \ & \textit{142} ~ \ \ \\ 
	 ~ Cereal & 83 ~ \ \ & 85 ~ \ \ & \textit{123} ~ \ \ \\ 
	 ~ Vegetable oils & 91 ~ \ \ & 63 ~ \ \ & \textit{134} ~ \ \ \\ 
	 ~ Roots and tubers & 66 ~ \ \ & 87 ~ \ \ & \textit{152} ~ \ \ \\ 
	 ~ Fruit and vegetables & 65 ~ \ \ & 88 ~ \ \ & \textit{163} ~ \ \ \\ 
	 ~ Sugar & 101 ~ \ \ & 119 ~ \ \ & \textit{136} ~ \ \ \\ 
	 ~ Livestock & 64 ~ \ \ & 89 ~ \ \ & \textit{135} ~ \ \ \\ 
	 ~ Milk & 59 ~ \ \ & 89 ~ \ \ & \textit{141} ~ \ \ \\ 
	 ~ Meat & 81 ~ \ \ & 93 ~ \ \ & \textit{117} ~ \ \ \\ 
	 ~ Fish  & 64 ~ \ \ & 89 ~ \ \ & \textit{139} ~ \ \ \\ 
	\multicolumn{4}{l}{\textit{\normalsize{Net trade (min US\$)}}} \\ 
	 ~ Cereals & 27 ~ \ \ & 1\,688 ~ \ \ & \textit{9\,154} ~ \ \ \\ 
	 ~ Fruit and vegetables & 134 ~ \ \ & -126 ~ \ \ & \textit{-1\,551} ~ \ \ \\ 
	 ~ Meat & 97 ~ \ \ & 280 ~ \ \ & \textit{3\,145} ~ \ \ \\ 
	 ~ Dairy products & -15 ~ \ \ & 13 ~ \ \ & \textit{56} ~ \ \ \\ 
	 ~ Fish & 671 ~ \ \ & 1\,383 ~ \ \ & \textit{3\,326} ~ \ \ \\ 
	\multicolumn{4}{l}{\textcolor{FAOblue}{\textbf{\large{Environment}}}} \\ 
	 ~ Forest area (\%) & 22 ~ \ \ & 22 ~ \ \ & \textit{23} ~ \ \ \\ 
	 ~ Renewable water res withdrawn (\% of total) &  ~ \ \ &  ~ \ \ & 90 ~ \ \ \\ 
	 ~ Terrestrial protect areas (\% total land area)  & 5 ~ \ \ & 5 ~ \ \ & \textit{5} ~ \ \ \\ 
	 ~ Organic area (\% total agricultural area) &  ~ \ \ & \textit{0} ~ \ \ & \textit{0} ~ \ \ \\ 
	 ~ Water withdrawal by agriculture (\% of total) &  ~ \ \ &  ~ \ \ & 90 ~ \ \ \\ 
	 ~ Biofuel production (thousand kt of oil eq.) & 349 ~ \ \ & 2\,321 ~ \ \ & \textit{2\,505} ~ \ \ \\ 
	 ~ Wood pellet prod. (min tonnes) &  ~ \ \ &  ~ \ \ &  ~ \ \ \\ 
	 ~ GHG emissions from ag (Co2 eq, gigagrams) & 470 ~ \ \ & 378 ~ \ \ & \textit{532} ~ \ \ \\ 
       \toprule
      \end{tabular}
      \clearpage
\CountryData{ Indonesia }
      \rowcolors{1}{FAOblue!10}{white}
      \begin{tabular}{L{3.9cm} R{1cm} R{1cm} R{1cm}}
      \toprule
      \multicolumn{1}{c}{} & \multicolumn{1}{c}{ 1992 } & \multicolumn{1}{c}{ 2002 } & \multicolumn{1}{c}{ 2014 } \\
      \midrule
	\multicolumn{4}{l}{\textcolor{FAOblue}{\textbf{\large{The setting}}}} \\ 
	 ~ Population, total (mln) & 184.9 ~ \ \ & 215 ~ \ \ & 252.8 ~ \ \ \\ 
	 ~ Population, rural (\% total population) & 124.8 ~ \ \ & 121.4 ~ \ \ & 118.8 ~ \ \ \\ 
	 ~ Govt expenditure on ag (\% total outlays) &  ~ \ \ & 2.3 ~ \ \ & \textit{0.9} ~ \ \ \\ 
	 ~ Area harvested (mln ha) & 56 ~ \ \ & 61 ~ \ \ & 120 ~ \ \ \\ 
	 ~ Cropping intensity ratio (\%) & 1.4 ~ \ \ & 1.3 ~ \ \ &  ~ \ \ \\ 
	 ~ Water resources (m\textsuperscript{3}/person/year) & \textit{11} ~ \ \ & \textit{9} ~ \ \ & \textit{8} ~ \ \ \\ 
	 ~ Area equipped for irrigation (1000 ha) &  ~ \ \ &  ~ \ \ & \textit{6\,722} ~ \ \ \\ 
	 ~ Area irrigated (\%) &  ~ \ \ &  ~ \ \ &  ~ \ \ \\ 
	 ~ Employment in agriculture (\%) & 54.9 ~ \ \ & 44.3 ~ \ \ & \textit{35.1} ~ \ \ \\ 
	 ~ Employment in agriculture, female (\%) & 56.5 ~ \ \ & 45.4 ~ \ \ & \textit{34.5} ~ \ \ \\ 
	 ~ Fertilizers, Nitrogen (nutrients per ha) &  ~ \ \ & 41 ~ \ \ & \textit{52.2} ~ \ \ \\ 
	 ~ Fertilizers, Phosphate (nutrients per ha) &  ~ \ \ & 5.3 ~ \ \ & \textit{12.1} ~ \ \ \\ 
	 ~ Fertilizers, Potash (nutrients per ha) &  ~ \ \ & 5.4 ~ \ \ & \textit{16.8} ~ \ \ \\ 
	 ~ Energy consump, power irrigation (mln kWh) &  ~ \ \ & 0 ~ \ \ & \textit{0} ~ \ \ \\ 
	 ~ Agr value added per worker (constant US\$) & 0.6 ~ \ \ & 0.7 ~ \ \ & \textit{1} ~ \ \ \\ 
	\multicolumn{4}{l}{\textcolor{FAOblue}{\textbf{\large{Hunger dimensions}}}} \\ 
	 ~ Dietary energy supply (kcal/pc/day) & 2\,401 ~ \ \ & 2\,448 ~ \ \ & 2\,771 ~ \ \ \\ 
	 ~ Average dietary energy supply adequacy (\%) & 108 ~ \ \ & 108 ~ \ \ & 122 ~ \ \ \\ 
	 ~ Dietary en supp, cereals/roots/tubers (\%) & 73 ~ \ \ & 71 ~ \ \ & \textit{69} ~ \ \ \\ 
	 ~ Prevalence of undernourishment (\%) & 19 ~ \ \ & 18.7 ~ \ \ & 7.6 ~ \ \ \\ 
	 ~ GDP per capita (US\$, PPP) & 4\,846 ~ \ \ & 5\,843 ~ \ \ & \textit{9\,254} ~ \ \ \\ 
	 ~ Domestic food price volatility (index) &  ~ \ \ & 14.9 ~ \ \ & 10.7 ~ \ \ \\ 
	 ~ Cereal import dependency ratio (\%) & 7 ~ \ \ & 12.8 ~ \ \ & \textit{12.7} ~ \ \ \\ 
	 ~ Underweight, children under-5 (\%) & 29.8 ~ \ \ & 23 ~ \ \ & \textit{19.9} ~ \ \ \\ 
	 ~ Improved water source (\% pop) & 71.3 ~ \ \ & 79 ~ \ \ & \textit{84.9} ~ \ \ \\ 
	\multicolumn{4}{l}{\textcolor{FAOblue}{\textbf{\large{Food Supply}}}} \\ 
	 ~ Food production value, (2004-2006 mln I\$) & 30\,041 ~ \ \ & 37\,701 ~ \ \ & \textit{60\,205} ~ \ \ \\ 
	 ~ Agriculture, value added (\% GDP) & 19 ~ \ \ & 15 ~ \ \ & \textit{14} ~ \ \ \\ 
	 ~ Food exports (mln US\$)  & 1\,424 ~ \ \ & 4\,181 ~ \ \ & \textit{24\,711} ~ \ \ \\ 
	 ~ Food imports (mln US\$)  & 1\,399 ~ \ \ & 2\,680 ~ \ \ & \textit{11\,598} ~ \ \ \\ 
	\multicolumn{4}{l}{\textit{\normalsize{Production indices (2004-06=100)}}} \\ 
	 ~ Net food & 69 ~ \ \ & 86 ~ \ \ & \textit{137} ~ \ \ \\ 
	 ~ Net crop & 67 ~ \ \ & 85 ~ \ \ & \textit{137} ~ \ \ \\ 
	 ~ Cereal & 87 ~ \ \ & 94 ~ \ \ & \textit{134} ~ \ \ \\ 
	 ~ Vegetable oils & 29 ~ \ \ & 73 ~ \ \ & \textit{194} ~ \ \ \\ 
	 ~ Roots and tubers & 85 ~ \ \ & 87 ~ \ \ & \textit{121} ~ \ \ \\ 
	 ~ Fruit and vegetables & 45 ~ \ \ & 78 ~ \ \ & \textit{116} ~ \ \ \\ 
	 ~ Sugar & 113 ~ \ \ & 90 ~ \ \ & \textit{119} ~ \ \ \\ 
	 ~ Livestock & 71 ~ \ \ & 90 ~ \ \ & \textit{140} ~ \ \ \\ 
	 ~ Milk & 72 ~ \ \ & 89 ~ \ \ & \textit{156} ~ \ \ \\ 
	 ~ Meat & 76 ~ \ \ & 91 ~ \ \ & \textit{139} ~ \ \ \\ 
	 ~ Fish  & 58 ~ \ \ & 89 ~ \ \ & \textit{169} ~ \ \ \\ 
	\multicolumn{4}{l}{\textit{\normalsize{Net trade (min US\$)}}} \\ 
	 ~ Cereals & -555 ~ \ \ & -1\,125 ~ \ \ & \textit{-3\,708} ~ \ \ \\ 
	 ~ Fruit and vegetables & 204 ~ \ \ & -16 ~ \ \ & \textit{-732} ~ \ \ \\ 
	 ~ Meat & 5 ~ \ \ & -22 ~ \ \ & \textit{-158} ~ \ \ \\ 
	 ~ Dairy products & -109 ~ \ \ & -186 ~ \ \ & \textit{-1\,002} ~ \ \ \\ 
	 ~ Fish & 1\,122 ~ \ \ & 1\,414 ~ \ \ & \textit{3\,228} ~ \ \ \\ 
	\multicolumn{4}{l}{\textcolor{FAOblue}{\textbf{\large{Environment}}}} \\ 
	 ~ Forest area (\%) & 63 ~ \ \ & 55 ~ \ \ & \textit{51} ~ \ \ \\ 
	 ~ Renewable water res withdrawn (\% of total) &  ~ \ \ & \textit{82} ~ \ \ & 82 ~ \ \ \\ 
	 ~ Terrestrial protect areas (\% total land area)  & 11 ~ \ \ & 14 ~ \ \ & \textit{15} ~ \ \ \\ 
	 ~ Organic area (\% total agricultural area) &  ~ \ \ & \textit{0} ~ \ \ & \textit{0} ~ \ \ \\ 
	 ~ Water withdrawal by agriculture (\% of total) &  ~ \ \ & \textit{82} ~ \ \ & 82 ~ \ \ \\ 
	 ~ Biofuel production (thousand kt of oil eq.) & 59 ~ \ \ & 462 ~ \ \ & \textit{510} ~ \ \ \\ 
	 ~ Wood pellet prod. (min tonnes) &  ~ \ \ &  ~ \ \ & \textit{40} ~ \ \ \\ 
	 ~ GHG emissions from ag (Co2 eq, gigagrams) & 808 ~ \ \ & 1\,137 ~ \ \ & \textit{1\,383} ~ \ \ \\ 
       \toprule
      \end{tabular}
      \clearpage
\CountryData{ Iran (Islamic Republic of) }
      \rowcolors{1}{FAOblue!10}{white}
      \begin{tabular}{L{3.9cm} R{1cm} R{1cm} R{1cm}}
      \toprule
      \multicolumn{1}{c}{} & \multicolumn{1}{c}{ 1992 } & \multicolumn{1}{c}{ 2002 } & \multicolumn{1}{c}{ 2014 } \\
      \midrule
	\multicolumn{4}{l}{\textcolor{FAOblue}{\textbf{\large{The setting}}}} \\ 
	 ~ Population, total (mln) & 58.3 ~ \ \ & 67.7 ~ \ \ & 78.5 ~ \ \ \\ 
	 ~ Population, rural (\% total population) & 24.7 ~ \ \ & 23.4 ~ \ \ & 23.9 ~ \ \ \\ 
	 ~ Govt expenditure on ag (\% total outlays) &  ~ \ \ & 5.4 ~ \ \ & \textit{1.4} ~ \ \ \\ 
	 ~ Area harvested (mln ha) & 16 ~ \ \ & 20 ~ \ \ & 22 ~ \ \ \\ 
	 ~ Cropping intensity ratio (\%) & 0.2 ~ \ \ & 0.3 ~ \ \ &  ~ \ \ \\ 
	 ~ Water resources (m\textsuperscript{3}/person/year) & \textit{2} ~ \ \ & \textit{2} ~ \ \ & \textit{2} ~ \ \ \\ 
	 ~ Area equipped for irrigation (1000 ha) &  ~ \ \ &  ~ \ \ & \textit{9\,553} ~ \ \ \\ 
	 ~ Area irrigated (\%) &  ~ \ \ &  ~ \ \ & \textit{77.4} ~ \ \ \\ 
	 ~ Employment in agriculture (\%) &  ~ \ \ & \textit{24.7} ~ \ \ & \textit{21.2} ~ \ \ \\ 
	 ~ Employment in agriculture, female (\%) &  ~ \ \ & \textit{33.6} ~ \ \ & \textit{30.6} ~ \ \ \\ 
	 ~ Fertilizers, Nitrogen (nutrients per ha) &  ~ \ \ & 13 ~ \ \ & \textit{6.3} ~ \ \ \\ 
	 ~ Fertilizers, Phosphate (nutrients per ha) &  ~ \ \ & 5.2 ~ \ \ & \textit{2.7} ~ \ \ \\ 
	 ~ Fertilizers, Potash (nutrients per ha) &  ~ \ \ & 1.5 ~ \ \ & \textit{0.5} ~ \ \ \\ 
	 ~ Energy consump, power irrigation (mln kWh) & 2 ~ \ \ & 219 ~ \ \ & \textit{1\,688} ~ \ \ \\ 
	 ~ Agr value added per worker (constant US\$) & 2.5 ~ \ \ & 2.7 ~ \ \ & \textit{3.3} ~ \ \ \\ 
	\multicolumn{4}{l}{\textcolor{FAOblue}{\textbf{\large{Hunger dimensions}}}} \\ 
	 ~ Dietary energy supply (kcal/pc/day) & 3\,021 ~ \ \ & 3\,016 ~ \ \ & 3\,195 ~ \ \ \\ 
	 ~ Average dietary energy supply adequacy (\%) & 138 ~ \ \ & 127 ~ \ \ & 133 ~ \ \ \\ 
	 ~ Dietary en supp, cereals/roots/tubers (\%) & 62 ~ \ \ & 59 ~ \ \ & \textit{53} ~ \ \ \\ 
	 ~ Prevalence of undernourishment (\%) & <5.0 ~ \ \ & 6.1 ~ \ \ & 5.1 ~ \ \ \\ 
	 ~ GDP per capita (US\$, PPP) & 9\,848 ~ \ \ & 11\,600 ~ \ \ & \textit{15\,090} ~ \ \ \\ 
	 ~ Domestic food price volatility (index) &  ~ \ \ & 15.2 ~ \ \ & 13 ~ \ \ \\ 
	 ~ Cereal import dependency ratio (\%) & 24.8 ~ \ \ & 28.7 ~ \ \ & \textit{28.7} ~ \ \ \\ 
	 ~ Underweight, children under-5 (\%) & \textit{13.8} ~ \ \ & \textit{4.6} ~ \ \ &  ~ \ \ \\ 
	 ~ Improved water source (\% pop) & 92.4 ~ \ \ & 94.5 ~ \ \ & \textit{95.9} ~ \ \ \\ 
	\multicolumn{4}{l}{\textcolor{FAOblue}{\textbf{\large{Food Supply}}}} \\ 
	 ~ Food production value, (2004-2006 mln I\$) & 14\,031 ~ \ \ & 20\,284 ~ \ \ & \textit{25\,588} ~ \ \ \\ 
	 ~ Agriculture, value added (\% GDP) & 19 ~ \ \ & 12 ~ \ \ & \textit{10} ~ \ \ \\ 
	 ~ Food exports (mln US\$)  & 572 ~ \ \ & 1\,072 ~ \ \ & \textit{3\,970} ~ \ \ \\ 
	 ~ Food imports (mln US\$)  & 2\,094 ~ \ \ & 1\,715 ~ \ \ & \textit{9\,668} ~ \ \ \\ 
	\multicolumn{4}{l}{\textit{\normalsize{Production indices (2004-06=100)}}} \\ 
	 ~ Net food & 62 ~ \ \ & 90 ~ \ \ & \textit{113} ~ \ \ \\ 
	 ~ Net crop & 63 ~ \ \ & 92 ~ \ \ & \textit{117} ~ \ \ \\ 
	 ~ Cereal & 71 ~ \ \ & 91 ~ \ \ & \textit{103} ~ \ \ \\ 
	 ~ Vegetable oils & 38 ~ \ \ & 54 ~ \ \ & \textit{105} ~ \ \ \\ 
	 ~ Roots and tubers & 59 ~ \ \ & 83 ~ \ \ & \textit{124} ~ \ \ \\ 
	 ~ Fruit and vegetables & 57 ~ \ \ & 91 ~ \ \ & \textit{112} ~ \ \ \\ 
	 ~ Sugar & 77 ~ \ \ & 92 ~ \ \ & \textit{92} ~ \ \ \\ 
	 ~ Livestock & 59 ~ \ \ & 85 ~ \ \ & \textit{105} ~ \ \ \\ 
	 ~ Milk & 59 ~ \ \ & 86 ~ \ \ & \textit{105} ~ \ \ \\ 
	 ~ Meat & 60 ~ \ \ & 86 ~ \ \ & \textit{108} ~ \ \ \\ 
	 ~ Fish  & 63 ~ \ \ & 77 ~ \ \ & \textit{169} ~ \ \ \\ 
	\multicolumn{4}{l}{\textit{\normalsize{Net trade (min US\$)}}} \\ 
	 ~ Cereals & -1\,018 ~ \ \ & -892 ~ \ \ & \textit{-4\,387} ~ \ \ \\ 
	 ~ Fruit and vegetables & 435 ~ \ \ & 720 ~ \ \ & \textit{1\,305} ~ \ \ \\ 
	 ~ Meat & -222 ~ \ \ & -9 ~ \ \ & \textit{-506} ~ \ \ \\ 
	 ~ Dairy products & -149 ~ \ \ & -47 ~ \ \ & \textit{188} ~ \ \ \\ 
	 ~ Fish & -6 ~ \ \ & 21 ~ \ \ & \textit{178} ~ \ \ \\ 
	\multicolumn{4}{l}{\textcolor{FAOblue}{\textbf{\large{Environment}}}} \\ 
	 ~ Forest area (\%) & 7 ~ \ \ & 7 ~ \ \ & \textit{7} ~ \ \ \\ 
	 ~ Renewable water res withdrawn (\% of total) &  ~ \ \ & \textit{92} ~ \ \ & 92 ~ \ \ \\ 
	 ~ Terrestrial protect areas (\% total land area)  & 5 ~ \ \ & 7 ~ \ \ & \textit{7} ~ \ \ \\ 
	 ~ Organic area (\% total agricultural area) &  ~ \ \ &  ~ \ \ & \textit{0} ~ \ \ \\ 
	 ~ Water withdrawal by agriculture (\% of total) &  ~ \ \ & \textit{92} ~ \ \ & 92 ~ \ \ \\ 
	 ~ Biofuel production (thousand kt of oil eq.) & 6 ~ \ \ & 16 ~ \ \ & \textit{1} ~ \ \ \\ 
	 ~ Wood pellet prod. (min tonnes) &  ~ \ \ &  ~ \ \ &  ~ \ \ \\ 
	 ~ GHG emissions from ag (Co2 eq, gigagrams) & 39 ~ \ \ & 39 ~ \ \ & \textit{38} ~ \ \ \\ 
       \toprule
      \end{tabular}
      \clearpage
\CountryData{ Iraq }
      \rowcolors{1}{FAOblue!10}{white}
      \begin{tabular}{L{3.9cm} R{1cm} R{1cm} R{1cm}}
      \toprule
      \multicolumn{1}{c}{} & \multicolumn{1}{c}{ 1992 } & \multicolumn{1}{c}{ 2002 } & \multicolumn{1}{c}{ 2014 } \\
      \midrule
	\multicolumn{4}{l}{\textcolor{FAOblue}{\textbf{\large{The setting}}}} \\ 
	 ~ Population, total (mln) & 18.5 ~ \ \ & 25.2 ~ \ \ & 34.8 ~ \ \ \\ 
	 ~ Population, rural (\% total population) & 5.7 ~ \ \ & 8.2 ~ \ \ & 11.7 ~ \ \ \\ 
	 ~ Govt expenditure on ag (\% total outlays) &  ~ \ \ & \textit{0.8} ~ \ \ & \textit{0.4} ~ \ \ \\ 
	 ~ Area harvested (mln ha) & 4 ~ \ \ & 4 ~ \ \ & 5 ~ \ \ \\ 
	 ~ Cropping intensity ratio (\%) & 0.4 ~ \ \ & 0.5 ~ \ \ &  ~ \ \ \\ 
	 ~ Water resources (m\textsuperscript{3}/person/year) & \textit{5} ~ \ \ & \textit{3} ~ \ \ & \textit{3} ~ \ \ \\ 
	 ~ Area equipped for irrigation (1000 ha) &  ~ \ \ &  ~ \ \ & \textit{3\,525} ~ \ \ \\ 
	 ~ Area irrigated (\%) & \textit{54.9} ~ \ \ &  ~ \ \ &  ~ \ \ \\ 
	 ~ Employment in agriculture (\%) &  ~ \ \ & \textit{17} ~ \ \ & \textit{23.4} ~ \ \ \\ 
	 ~ Employment in agriculture, female (\%) &  ~ \ \ & \textit{32.6} ~ \ \ & \textit{50.7} ~ \ \ \\ 
	 ~ Fertilizers, Nitrogen (nutrients per ha) &  ~ \ \ & 0 ~ \ \ & \textit{18.5} ~ \ \ \\ 
	 ~ Fertilizers, Phosphate (nutrients per ha) &  ~ \ \ & 0 ~ \ \ & \textit{6.4} ~ \ \ \\ 
	 ~ Fertilizers, Potash (nutrients per ha) &  ~ \ \ & 0 ~ \ \ & \textit{0.4} ~ \ \ \\ 
	 ~ Energy consump, power irrigation (mln kWh) & \textit{19} ~ \ \ & 19 ~ \ \ & \textit{19} ~ \ \ \\ 
	 ~ Agr value added per worker (constant US\$) & 3.4 ~ \ \ & 6.1 ~ \ \ & \textit{7} ~ \ \ \\ 
	\multicolumn{4}{l}{\textcolor{FAOblue}{\textbf{\large{Hunger dimensions}}}} \\ 
	 ~ Dietary energy supply (kcal/pc/day) & 2\,288 ~ \ \ & 2\,267 ~ \ \ & 2\,549 ~ \ \ \\ 
	 ~ Average dietary energy supply adequacy (\%) & 109 ~ \ \ & 107 ~ \ \ & 118 ~ \ \ \\ 
	 ~ Dietary en supp, cereals/roots/tubers (\%) & 67 ~ \ \ & 63 ~ \ \ & \textit{62} ~ \ \ \\ 
	 ~ Prevalence of undernourishment (\%) & 13.5 ~ \ \ & 23.3 ~ \ \ & 23.2 ~ \ \ \\ 
	 ~ GDP per capita (US\$, PPP) & 5\,048 ~ \ \ & 10\,567 ~ \ \ & \textit{14\,472} ~ \ \ \\ 
	 ~ Domestic food price volatility (index) &  ~ \ \ &  ~ \ \ & 16.4 ~ \ \ \\ 
	 ~ Cereal import dependency ratio (\%) & 38.3 ~ \ \ & 51.2 ~ \ \ & \textit{56.8} ~ \ \ \\ 
	 ~ Underweight, children under-5 (\%) & \textit{10.4} ~ \ \ & \textit{8} ~ \ \ & \textit{8.5} ~ \ \ \\ 
	 ~ Improved water source (\% pop) & 78.1 ~ \ \ & 80.8 ~ \ \ & \textit{85.4} ~ \ \ \\ 
	\multicolumn{4}{l}{\textcolor{FAOblue}{\textbf{\large{Food Supply}}}} \\ 
	 ~ Food production value, (2004-2006 mln I\$) & 1\,979 ~ \ \ & 2\,976 ~ \ \ & \textit{2\,917} ~ \ \ \\ 
	 ~ Agriculture, value added (\% GDP) &  ~ \ \ &  ~ \ \ &  ~ \ \ \\ 
	 ~ Food exports (mln US\$)  & 7 ~ \ \ & 28 ~ \ \ & \textit{57} ~ \ \ \\ 
	 ~ Food imports (mln US\$)  & 1\,054 ~ \ \ & 1\,365 ~ \ \ & \textit{6\,333} ~ \ \ \\ 
	\multicolumn{4}{l}{\textit{\normalsize{Production indices (2004-06=100)}}} \\ 
	 ~ Net food & 85 ~ \ \ & 129 ~ \ \ & \textit{126} ~ \ \ \\ 
	 ~ Net crop & 84 ~ \ \ & 125 ~ \ \ & \textit{132} ~ \ \ \\ 
	 ~ Cereal & 75 ~ \ \ & 118 ~ \ \ & \textit{205} ~ \ \ \\ 
	 ~ Vegetable oils & 89 ~ \ \ & 106 ~ \ \ & \textit{60} ~ \ \ \\ 
	 ~ Roots and tubers & 26 ~ \ \ & 121 ~ \ \ & \textit{76} ~ \ \ \\ 
	 ~ Fruit and vegetables & 92 ~ \ \ & 125 ~ \ \ & \textit{116} ~ \ \ \\ 
	 ~ Sugar & 1\,584 ~ \ \ & 838 ~ \ \ & \textit{356} ~ \ \ \\ 
	 ~ Livestock & 78 ~ \ \ & 132 ~ \ \ & \textit{127} ~ \ \ \\ 
	 ~ Milk & 73 ~ \ \ & 189 ~ \ \ & \textit{97} ~ \ \ \\ 
	 ~ Meat & 80 ~ \ \ & 116 ~ \ \ & \textit{143} ~ \ \ \\ 
	 ~ Fish  & 47 ~ \ \ & 58 ~ \ \ & \textit{147} ~ \ \ \\ 
	\multicolumn{4}{l}{\textit{\normalsize{Net trade (min US\$)}}} \\ 
	 ~ Cereals & -466 ~ \ \ & -717 ~ \ \ & \textit{-2\,538} ~ \ \ \\ 
	 ~ Fruit and vegetables & -69 ~ \ \ & -30 ~ \ \ & \textit{-498} ~ \ \ \\ 
	 ~ Meat & -56 ~ \ \ & 0 ~ \ \ & \textit{-838} ~ \ \ \\ 
	 ~ Dairy products & -23 ~ \ \ & -181 ~ \ \ & \textit{-195} ~ \ \ \\ 
	 ~ Fish &  ~ \ \ & 0 ~ \ \ & \textit{-42} ~ \ \ \\ 
	\multicolumn{4}{l}{\textcolor{FAOblue}{\textbf{\large{Environment}}}} \\ 
	 ~ Forest area (\%) & 2 ~ \ \ & 2 ~ \ \ & \textit{2} ~ \ \ \\ 
	 ~ Renewable water res withdrawn (\% of total) &  ~ \ \ & \textit{79} ~ \ \ & 79 ~ \ \ \\ 
	 ~ Terrestrial protect areas (\% total land area)  & 0 ~ \ \ & 0 ~ \ \ & \textit{0} ~ \ \ \\ 
	 ~ Organic area (\% total agricultural area) &  ~ \ \ &  ~ \ \ &  ~ \ \ \\ 
	 ~ Water withdrawal by agriculture (\% of total) &  ~ \ \ & \textit{79} ~ \ \ & 79 ~ \ \ \\ 
	 ~ Biofuel production (thousand kt of oil eq.) & 0 ~ \ \ & \textit{0} ~ \ \ &  ~ \ \ \\ 
	 ~ Wood pellet prod. (min tonnes) &  ~ \ \ &  ~ \ \ &  ~ \ \ \\ 
	 ~ GHG emissions from ag (Co2 eq, gigagrams) & 5 ~ \ \ & 4 ~ \ \ & \textit{7} ~ \ \ \\ 
       \toprule
      \end{tabular}
      \clearpage
\CountryData{ Ireland }
      \rowcolors{1}{FAOblue!10}{white}
      \begin{tabular}{L{3.9cm} R{1cm} R{1cm} R{1cm}}
      \toprule
      \multicolumn{1}{c}{} & \multicolumn{1}{c}{ 1992 } & \multicolumn{1}{c}{ 2002 } & \multicolumn{1}{c}{ 2014 } \\
      \midrule
	\multicolumn{4}{l}{\textcolor{FAOblue}{\textbf{\large{The setting}}}} \\ 
	 ~ Population, total (mln) & 3.5 ~ \ \ & 3.9 ~ \ \ & 4.7 ~ \ \ \\ 
	 ~ Population, rural (\% total population) & 1.5 ~ \ \ & 1.6 ~ \ \ & 1.7 ~ \ \ \\ 
	 ~ Govt expenditure on ag (\% total outlays) &  ~ \ \ &  ~ \ \ &  ~ \ \ \\ 
	 ~ Area harvested (mln ha) & 2 ~ \ \ & 4 ~ \ \ & 2 ~ \ \ \\ 
	 ~ Cropping intensity ratio (\%) & 0.5 ~ \ \ & 0.8 ~ \ \ &  ~ \ \ \\ 
	 ~ Water resources (m\textsuperscript{3}/person/year) & \textit{15} ~ \ \ & \textit{13} ~ \ \ & \textit{11} ~ \ \ \\ 
	 ~ Area equipped for irrigation (1000 ha) &  ~ \ \ &  ~ \ \ &  ~ \ \ \\ 
	 ~ Area irrigated (\%) &  ~ \ \ & \textit{100} ~ \ \ &  ~ \ \ \\ 
	 ~ Employment in agriculture (\%) & 11.6 ~ \ \ & 5.9 ~ \ \ & \textit{4.7} ~ \ \ \\ 
	 ~ Employment in agriculture, female (\%) & 3.2 ~ \ \ & 1.4 ~ \ \ & \textit{1.2} ~ \ \ \\ 
	 ~ Fertilizers, Nitrogen (nutrients per ha) &  ~ \ \ & 107.1 ~ \ \ & \textit{61.8} ~ \ \ \\ 
	 ~ Fertilizers, Phosphate (nutrients per ha) &  ~ \ \ & 23 ~ \ \ & \textit{17} ~ \ \ \\ 
	 ~ Fertilizers, Potash (nutrients per ha) &  ~ \ \ & 30.6 ~ \ \ & \textit{21} ~ \ \ \\ 
	 ~ Energy consump, power irrigation (mln kWh) &  ~ \ \ &  ~ \ \ &  ~ \ \ \\ 
	 ~ Agr value added per worker (constant US\$) & \textit{14.2} ~ \ \ & 17 ~ \ \ & \textit{7.9} ~ \ \ \\ 
	\multicolumn{4}{l}{\textcolor{FAOblue}{\textbf{\large{Hunger dimensions}}}} \\ 
	 ~ Dietary energy supply (kcal/pc/day) &  ~ \ \ &  ~ \ \ &  ~ \ \ \\ 
	 ~ Average dietary energy supply adequacy (\%) & 146 ~ \ \ & 146 ~ \ \ & 146 ~ \ \ \\ 
	 ~ Dietary en supp, cereals/roots/tubers (\%) & 34 ~ \ \ & 31 ~ \ \ & \textit{34} ~ \ \ \\ 
	 ~ Prevalence of undernourishment (\%) & <5.0 ~ \ \ & <5.0 ~ \ \ & <5.0 ~ \ \ \\ 
	 ~ GDP per capita (US\$, PPP) & 23\,371 ~ \ \ & 44\,423 ~ \ \ & \textit{44\,647} ~ \ \ \\ 
	 ~ Domestic food price volatility (index) &  ~ \ \ & 3.8 ~ \ \ & 3.3 ~ \ \ \\ 
	 ~ Cereal import dependency ratio (\%) & 1.1 ~ \ \ & 23.5 ~ \ \ & \textit{32} ~ \ \ \\ 
	 ~ Underweight, children under-5 (\%) &  ~ \ \ &  ~ \ \ &  ~ \ \ \\ 
	 ~ Improved water source (\% pop) & 99.8 ~ \ \ & 99.8 ~ \ \ & \textit{99.9} ~ \ \ \\ 
	\multicolumn{4}{l}{\textcolor{FAOblue}{\textbf{\large{Food Supply}}}} \\ 
	 ~ Food production value, (2004-2006 mln I\$) & 4\,439 ~ \ \ & 4\,280 ~ \ \ & \textit{4\,317} ~ \ \ \\ 
	 ~ Agriculture, value added (\% GDP) & \textit{6} ~ \ \ & 2 ~ \ \ & \textit{2} ~ \ \ \\ 
	 ~ Food exports (mln US\$)  & 5\,860 ~ \ \ & 4\,822 ~ \ \ & \textit{9\,303} ~ \ \ \\ 
	 ~ Food imports (mln US\$)  & 1\,711 ~ \ \ & 2\,484 ~ \ \ & \textit{5\,985} ~ \ \ \\ 
	\multicolumn{4}{l}{\textit{\normalsize{Production indices (2004-06=100)}}} \\ 
	 ~ Net food & 102 ~ \ \ & 98 ~ \ \ & \textit{99} ~ \ \ \\ 
	 ~ Net crop & 95 ~ \ \ & 96 ~ \ \ & \textit{94} ~ \ \ \\ 
	 ~ Cereal & 91 ~ \ \ & 91 ~ \ \ & \textit{105} ~ \ \ \\ 
	 ~ Vegetable oils & 139 ~ \ \ & 52 ~ \ \ & \textit{379} ~ \ \ \\ 
	 ~ Roots and tubers & 134 ~ \ \ & 116 ~ \ \ & \textit{86} ~ \ \ \\ 
	 ~ Fruit and vegetables & 77 ~ \ \ & 94 ~ \ \ & \textit{98} ~ \ \ \\ 
	 ~ Sugar & 86 ~ \ \ & 80 ~ \ \ & \textit{86} ~ \ \ \\ 
	 ~ Livestock & 101 ~ \ \ & 98 ~ \ \ & \textit{100} ~ \ \ \\ 
	 ~ Milk & 99 ~ \ \ & 100 ~ \ \ & \textit{104} ~ \ \ \\ 
	 ~ Meat & 102 ~ \ \ & 97 ~ \ \ & \textit{96} ~ \ \ \\ 
	 ~ Fish  & 89 ~ \ \ & 111 ~ \ \ & \textit{90} ~ \ \ \\ 
	\multicolumn{4}{l}{\textit{\normalsize{Net trade (min US\$)}}} \\ 
	 ~ Cereals & 46 ~ \ \ & 309 ~ \ \ & \textit{591} ~ \ \ \\ 
	 ~ Fruit and vegetables & -290 ~ \ \ & -377 ~ \ \ & \textit{-995} ~ \ \ \\ 
	 ~ Meat & 1\,561 ~ \ \ & 1\,286 ~ \ \ & \textit{2\,787} ~ \ \ \\ 
	 ~ Dairy products & 1\,590 ~ \ \ & 632 ~ \ \ & \textit{1\,374} ~ \ \ \\ 
	 ~ Fish & 230 ~ \ \ & 272 ~ \ \ & \textit{389} ~ \ \ \\ 
	\multicolumn{4}{l}{\textcolor{FAOblue}{\textbf{\large{Environment}}}} \\ 
	 ~ Forest area (\%) & 7 ~ \ \ & 10 ~ \ \ & \textit{11} ~ \ \ \\ 
	 ~ Renewable water res withdrawn (\% of total) &  ~ \ \ & \textit{0} ~ \ \ & 0 ~ \ \ \\ 
	 ~ Terrestrial protect areas (\% total land area)  & 1 ~ \ \ & 1 ~ \ \ & \textit{14} ~ \ \ \\ 
	 ~ Organic area (\% total agricultural area) &  ~ \ \ & \textit{1} ~ \ \ & \textit{1} ~ \ \ \\ 
	 ~ Water withdrawal by agriculture (\% of total) &  ~ \ \ & \textit{0} ~ \ \ & 0 ~ \ \ \\ 
	 ~ Biofuel production (thousand kt of oil eq.) & 0 ~ \ \ & 6 ~ \ \ & \textit{1\,871} ~ \ \ \\ 
	 ~ Wood pellet prod. (min tonnes) &  ~ \ \ &  ~ \ \ & \textit{32} ~ \ \ \\ 
	 ~ GHG emissions from ag (Co2 eq, gigagrams) & 19 ~ \ \ & 21 ~ \ \ & \textit{19} ~ \ \ \\ 
       \toprule
      \end{tabular}
      \clearpage
\CountryData{ Israel }
      \rowcolors{1}{FAOblue!10}{white}
      \begin{tabular}{L{3.9cm} R{1cm} R{1cm} R{1cm}}
      \toprule
      \multicolumn{1}{c}{} & \multicolumn{1}{c}{ 1992 } & \multicolumn{1}{c}{ 2002 } & \multicolumn{1}{c}{ 2014 } \\
      \midrule
	\multicolumn{4}{l}{\textcolor{FAOblue}{\textbf{\large{The setting}}}} \\ 
	 ~ Population, total (mln) & 4.8 ~ \ \ & 6.2 ~ \ \ & 7.8 ~ \ \ \\ 
	 ~ Population, rural (\% total population) & 0.5 ~ \ \ & 0.5 ~ \ \ & 0.6 ~ \ \ \\ 
	 ~ Govt expenditure on ag (\% total outlays) &  ~ \ \ &  ~ \ \ &  ~ \ \ \\ 
	 ~ Area harvested (mln ha) & 2 ~ \ \ & 2 ~ \ \ & 1 ~ \ \ \\ 
	 ~ Cropping intensity ratio (\%) & 2.7 ~ \ \ & 3.1 ~ \ \ &  ~ \ \ \\ 
	 ~ Water resources (m\textsuperscript{3}/person/year) & \textit{0} ~ \ \ & \textit{0} ~ \ \ & \textit{0} ~ \ \ \\ 
	 ~ Area equipped for irrigation (1000 ha) &  ~ \ \ &  ~ \ \ & \textit{225} ~ \ \ \\ 
	 ~ Area irrigated (\%) &  ~ \ \ &  ~ \ \ & \textit{80.7} ~ \ \ \\ 
	 ~ Employment in agriculture (\%) & 3.5 ~ \ \ & 2 ~ \ \ & \textit{1.7} ~ \ \ \\ 
	 ~ Employment in agriculture, female (\%) & 1.9 ~ \ \ & 0.7 ~ \ \ & \textit{0.7} ~ \ \ \\ 
	 ~ Fertilizers, Nitrogen (nutrients per ha) &  ~ \ \ & 79.4 ~ \ \ & \textit{121.9} ~ \ \ \\ 
	 ~ Fertilizers, Phosphate (nutrients per ha) &  ~ \ \ & 20.6 ~ \ \ & \textit{22.7} ~ \ \ \\ 
	 ~ Fertilizers, Potash (nutrients per ha) &  ~ \ \ & 56.1 ~ \ \ & \textit{61} ~ \ \ \\ 
	 ~ Energy consump, power irrigation (mln kWh) & 251 ~ \ \ & 555 ~ \ \ & \textit{552} ~ \ \ \\ 
	 ~ Agr value added per worker (constant US\$) &  ~ \ \ &  ~ \ \ &  ~ \ \ \\ 
	\multicolumn{4}{l}{\textcolor{FAOblue}{\textbf{\large{Hunger dimensions}}}} \\ 
	 ~ Dietary energy supply (kcal/pc/day) &  ~ \ \ &  ~ \ \ &  ~ \ \ \\ 
	 ~ Average dietary energy supply adequacy (\%) & 150 ~ \ \ & 158 ~ \ \ & 159 ~ \ \ \\ 
	 ~ Dietary en supp, cereals/roots/tubers (\%) & 37 ~ \ \ & 34 ~ \ \ & \textit{35} ~ \ \ \\ 
	 ~ Prevalence of undernourishment (\%) & <5.0 ~ \ \ & <5.0 ~ \ \ & <5.0 ~ \ \ \\ 
	 ~ GDP per capita (US\$, PPP) & 17\,749 ~ \ \ & 24\,650 ~ \ \ & \textit{30\,927} ~ \ \ \\ 
	 ~ Domestic food price volatility (index) &  ~ \ \ & 8.4 ~ \ \ & 5.9 ~ \ \ \\ 
	 ~ Cereal import dependency ratio (\%) & 89.5 ~ \ \ & 92.5 ~ \ \ & \textit{93.3} ~ \ \ \\ 
	 ~ Underweight, children under-5 (\%) &  ~ \ \ &  ~ \ \ &  ~ \ \ \\ 
	 ~ Improved water source (\% pop) & 100 ~ \ \ & 100 ~ \ \ & \textit{100} ~ \ \ \\ 
	\multicolumn{4}{l}{\textcolor{FAOblue}{\textbf{\large{Food Supply}}}} \\ 
	 ~ Food production value, (2004-2006 mln I\$) & 1\,765 ~ \ \ & 2\,426 ~ \ \ & \textit{2\,836} ~ \ \ \\ 
	 ~ Agriculture, value added (\% GDP) &  ~ \ \ &  ~ \ \ &  ~ \ \ \\ 
	 ~ Food exports (mln US\$)  & 801 ~ \ \ & 712 ~ \ \ & \textit{1\,875} ~ \ \ \\ 
	 ~ Food imports (mln US\$)  & 1\,040 ~ \ \ & 1\,509 ~ \ \ & \textit{3\,810} ~ \ \ \\ 
	\multicolumn{4}{l}{\textit{\normalsize{Production indices (2004-06=100)}}} \\ 
	 ~ Net food & 69 ~ \ \ & 94 ~ \ \ & \textit{110} ~ \ \ \\ 
	 ~ Net crop & 80 ~ \ \ & 94 ~ \ \ & \textit{106} ~ \ \ \\ 
	 ~ Cereal & 128 ~ \ \ & 98 ~ \ \ & \textit{114} ~ \ \ \\ 
	 ~ Vegetable oils & 113 ~ \ \ & 100 ~ \ \ & \textit{105} ~ \ \ \\ 
	 ~ Roots and tubers & 38 ~ \ \ & 71 ~ \ \ & \textit{101} ~ \ \ \\ 
	 ~ Fruit and vegetables & 80 ~ \ \ & 95 ~ \ \ & \textit{106} ~ \ \ \\ 
	 ~ Sugar &  ~ \ \ &  ~ \ \ &  ~ \ \ \\ 
	 ~ Livestock & 62 ~ \ \ & 96 ~ \ \ & \textit{114} ~ \ \ \\ 
	 ~ Milk & 87 ~ \ \ & 104 ~ \ \ & \textit{119} ~ \ \ \\ 
	 ~ Meat & 47 ~ \ \ & 92 ~ \ \ & \textit{110} ~ \ \ \\ 
	 ~ Fish  & 68 ~ \ \ & 105 ~ \ \ & \textit{96} ~ \ \ \\ 
	\multicolumn{4}{l}{\textit{\normalsize{Net trade (min US\$)}}} \\ 
	 ~ Cereals & -344 ~ \ \ & -469 ~ \ \ & \textit{-1\,247} ~ \ \ \\ 
	 ~ Fruit and vegetables & 411 ~ \ \ & 231 ~ \ \ & \textit{861} ~ \ \ \\ 
	 ~ Meat & -44 ~ \ \ & -93 ~ \ \ & \textit{-391} ~ \ \ \\ 
	 ~ Dairy products & -6 ~ \ \ & -18 ~ \ \ & \textit{-39} ~ \ \ \\ 
	 ~ Fish & -84 ~ \ \ & -127 ~ \ \ & \textit{-359} ~ \ \ \\ 
	\multicolumn{4}{l}{\textcolor{FAOblue}{\textbf{\large{Environment}}}} \\ 
	 ~ Forest area (\%) & 6 ~ \ \ & 7 ~ \ \ & \textit{7} ~ \ \ \\ 
	 ~ Renewable water res withdrawn (\% of total) &  ~ \ \ & \textit{58} ~ \ \ & 58 ~ \ \ \\ 
	 ~ Terrestrial protect areas (\% total land area)  & 18 ~ \ \ & 18 ~ \ \ & \textit{17} ~ \ \ \\ 
	 ~ Organic area (\% total agricultural area) &  ~ \ \ &  ~ \ \ &  ~ \ \ \\ 
	 ~ Water withdrawal by agriculture (\% of total) &  ~ \ \ & \textit{58} ~ \ \ & 58 ~ \ \ \\ 
	 ~ Biofuel production (thousand kt of oil eq.) &  ~ \ \ &  ~ \ \ & \textit{0} ~ \ \ \\ 
	 ~ Wood pellet prod. (min tonnes) &  ~ \ \ &  ~ \ \ &  ~ \ \ \\ 
	 ~ GHG emissions from ag (Co2 eq, gigagrams) & 1 ~ \ \ & 1 ~ \ \ & \textit{1} ~ \ \ \\ 
       \toprule
      \end{tabular}
      \clearpage
\CountryData{ Italy }
      \rowcolors{1}{FAOblue!10}{white}
      \begin{tabular}{L{3.9cm} R{1cm} R{1cm} R{1cm}}
      \toprule
      \multicolumn{1}{c}{} & \multicolumn{1}{c}{ 1992 } & \multicolumn{1}{c}{ 2002 } & \multicolumn{1}{c}{ 2014 } \\
      \midrule
	\multicolumn{4}{l}{\textcolor{FAOblue}{\textbf{\large{The setting}}}} \\ 
	 ~ Population, total (mln) & 56.9 ~ \ \ & 57.5 ~ \ \ & 61.1 ~ \ \ \\ 
	 ~ Population, rural (\% total population) & 18.9 ~ \ \ & 18.8 ~ \ \ & 19 ~ \ \ \\ 
	 ~ Govt expenditure on ag (\% total outlays) &  ~ \ \ &  ~ \ \ &  ~ \ \ \\ 
	 ~ Area harvested (mln ha) & 21 ~ \ \ & 21 ~ \ \ & 16 ~ \ \ \\ 
	 ~ Cropping intensity ratio (\%) & 1.3 ~ \ \ & 1.4 ~ \ \ &  ~ \ \ \\ 
	 ~ Water resources (m\textsuperscript{3}/person/year) & \textit{3} ~ \ \ & \textit{3} ~ \ \ & \textit{3} ~ \ \ \\ 
	 ~ Area equipped for irrigation (1000 ha) &  ~ \ \ &  ~ \ \ & \textit{3\,950} ~ \ \ \\ 
	 ~ Area irrigated (\%) &  ~ \ \ &  ~ \ \ & \textit{67.5} ~ \ \ \\ 
	 ~ Employment in agriculture (\%) & 8 ~ \ \ & 4.9 ~ \ \ & \textit{3.7} ~ \ \ \\ 
	 ~ Employment in agriculture, female (\%) & 8.6 ~ \ \ & 3.9 ~ \ \ & \textit{2.6} ~ \ \ \\ 
	 ~ Fertilizers, Nitrogen (nutrients per ha) &  ~ \ \ & 55.3 ~ \ \ & \textit{49.9} ~ \ \ \\ 
	 ~ Fertilizers, Phosphate (nutrients per ha) &  ~ \ \ & 20.7 ~ \ \ & \textit{15.1} ~ \ \ \\ 
	 ~ Fertilizers, Potash (nutrients per ha) &  ~ \ \ & 16.8 ~ \ \ & \textit{13.2} ~ \ \ \\ 
	 ~ Energy consump, power irrigation (mln kWh) & 1\,149 ~ \ \ & 1\,149 ~ \ \ & \textit{2\,993} ~ \ \ \\ 
	 ~ Agr value added per worker (constant US\$) & 18.4 ~ \ \ & 31.7 ~ \ \ & \textit{50.5} ~ \ \ \\ 
	\multicolumn{4}{l}{\textcolor{FAOblue}{\textbf{\large{Hunger dimensions}}}} \\ 
	 ~ Dietary energy supply (kcal/pc/day) &  ~ \ \ &  ~ \ \ &  ~ \ \ \\ 
	 ~ Average dietary energy supply adequacy (\%) & 140 ~ \ \ & 145 ~ \ \ & 140 ~ \ \ \\ 
	 ~ Dietary en supp, cereals/roots/tubers (\%) & 34 ~ \ \ & 34 ~ \ \ & \textit{34} ~ \ \ \\ 
	 ~ Prevalence of undernourishment (\%) & <5.0 ~ \ \ & <5.0 ~ \ \ & <5.0 ~ \ \ \\ 
	 ~ GDP per capita (US\$, PPP) & 31\,436 ~ \ \ & 36\,729 ~ \ \ & \textit{33\,924} ~ \ \ \\ 
	 ~ Domestic food price volatility (index) &  ~ \ \ & 3.7 ~ \ \ & 5 ~ \ \ \\ 
	 ~ Cereal import dependency ratio (\%) & 13.6 ~ \ \ & 22.2 ~ \ \ & \textit{26.6} ~ \ \ \\ 
	 ~ Underweight, children under-5 (\%) &  ~ \ \ &  ~ \ \ &  ~ \ \ \\ 
	 ~ Improved water source (\% pop) & 100 ~ \ \ & 100 ~ \ \ & \textit{100} ~ \ \ \\ 
	\multicolumn{4}{l}{\textcolor{FAOblue}{\textbf{\large{Food Supply}}}} \\ 
	 ~ Food production value, (2004-2006 mln I\$) & 31\,330 ~ \ \ & 29\,320 ~ \ \ & \textit{29\,303} ~ \ \ \\ 
	 ~ Agriculture, value added (\% GDP) & 3 ~ \ \ & 3 ~ \ \ & \textit{2} ~ \ \ \\ 
	 ~ Food exports (mln US\$)  & 9\,809 ~ \ \ & 11\,881 ~ \ \ & \textit{27\,468} ~ \ \ \\ 
	 ~ Food imports (mln US\$)  & 17\,224 ~ \ \ & 14\,845 ~ \ \ & \textit{31\,717} ~ \ \ \\ 
	\multicolumn{4}{l}{\textit{\normalsize{Production indices (2004-06=100)}}} \\ 
	 ~ Net food & 101 ~ \ \ & 95 ~ \ \ & \textit{95} ~ \ \ \\ 
	 ~ Net crop & 100 ~ \ \ & 91 ~ \ \ & \textit{89} ~ \ \ \\ 
	 ~ Cereal & 91 ~ \ \ & 98 ~ \ \ & \textit{86} ~ \ \ \\ 
	 ~ Vegetable oils & 67 ~ \ \ & 85 ~ \ \ & \textit{78} ~ \ \ \\ 
	 ~ Roots and tubers & 138 ~ \ \ & 105 ~ \ \ & \textit{72} ~ \ \ \\ 
	 ~ Fruit and vegetables & 106 ~ \ \ & 88 ~ \ \ & \textit{96} ~ \ \ \\ 
	 ~ Sugar & 160 ~ \ \ & 139 ~ \ \ & \textit{24} ~ \ \ \\ 
	 ~ Livestock & 101 ~ \ \ & 105 ~ \ \ & \textit{93} ~ \ \ \\ 
	 ~ Milk & 101 ~ \ \ & 105 ~ \ \ & \textit{93} ~ \ \ \\ 
	 ~ Meat & 101 ~ \ \ & 104 ~ \ \ & \textit{92} ~ \ \ \\ 
	 ~ Fish  & 123 ~ \ \ & 99 ~ \ \ & \textit{74} ~ \ \ \\ 
	\multicolumn{4}{l}{\textit{\normalsize{Net trade (min US\$)}}} \\ 
	 ~ Cereals & -367 ~ \ \ & 433 ~ \ \ & \textit{403} ~ \ \ \\ 
	 ~ Fruit and vegetables & 1\,645 ~ \ \ & 1\,627 ~ \ \ & \textit{3\,252} ~ \ \ \\ 
	 ~ Meat & -3\,538 ~ \ \ & -1\,900 ~ \ \ & \textit{-3\,134} ~ \ \ \\ 
	 ~ Dairy products & -2\,237 ~ \ \ & -1\,280 ~ \ \ & \textit{-1\,447} ~ \ \ \\ 
	 ~ Fish & -2\,385 ~ \ \ & -2\,482 ~ \ \ & \textit{-4\,830} ~ \ \ \\ 
	\multicolumn{4}{l}{\textcolor{FAOblue}{\textbf{\large{Environment}}}} \\ 
	 ~ Forest area (\%) & 26 ~ \ \ & 29 ~ \ \ & \textit{32} ~ \ \ \\ 
	 ~ Renewable water res withdrawn (\% of total) &  ~ \ \ & \textit{44} ~ \ \ & 44 ~ \ \ \\ 
	 ~ Terrestrial protect areas (\% total land area)  & 7 ~ \ \ & 15 ~ \ \ & \textit{22} ~ \ \ \\ 
	 ~ Organic area (\% total agricultural area) &  ~ \ \ & \textit{7} ~ \ \ & \textit{9} ~ \ \ \\ 
	 ~ Water withdrawal by agriculture (\% of total) &  ~ \ \ & \textit{44} ~ \ \ & 44 ~ \ \ \\ 
	 ~ Biofuel production (thousand kt of oil eq.) & 0 ~ \ \ & 9 ~ \ \ & \textit{21\,619} ~ \ \ \\ 
	 ~ Wood pellet prod. (min tonnes) &  ~ \ \ &  ~ \ \ & \textit{300} ~ \ \ \\ 
	 ~ GHG emissions from ag (Co2 eq, gigagrams) & 7 ~ \ \ & 1 ~ \ \ & \textit{-1} ~ \ \ \\ 
       \toprule
      \end{tabular}
      \clearpage
\CountryData{ Jamaica }
      \rowcolors{1}{FAOblue!10}{white}
      \begin{tabular}{L{3.9cm} R{1cm} R{1cm} R{1cm}}
      \toprule
      \multicolumn{1}{c}{} & \multicolumn{1}{c}{ 1992 } & \multicolumn{1}{c}{ 2002 } & \multicolumn{1}{c}{ 2014 } \\
      \midrule
	\multicolumn{4}{l}{\textcolor{FAOblue}{\textbf{\large{The setting}}}} \\ 
	 ~ Population, total (mln) & 2.4 ~ \ \ & 2.6 ~ \ \ & 2.8 ~ \ \ \\ 
	 ~ Population, rural (\% total population) & 1.2 ~ \ \ & 1.3 ~ \ \ & 1.3 ~ \ \ \\ 
	 ~ Govt expenditure on ag (\% total outlays) &  ~ \ \ & 1.9 ~ \ \ & \textit{1.6} ~ \ \ \\ 
	 ~ Area harvested (mln ha) & 3 ~ \ \ & 2 ~ \ \ & 1 ~ \ \ \\ 
	 ~ Cropping intensity ratio (\%) & 5.3 ~ \ \ & 4.1 ~ \ \ &  ~ \ \ \\ 
	 ~ Water resources (m\textsuperscript{3}/person/year) & \textit{4} ~ \ \ & \textit{4} ~ \ \ & \textit{3} ~ \ \ \\ 
	 ~ Area equipped for irrigation (1000 ha) &  ~ \ \ &  ~ \ \ & \textit{25} ~ \ \ \\ 
	 ~ Area irrigated (\%) &  ~ \ \ & \textit{100} ~ \ \ &  ~ \ \ \\ 
	 ~ Employment in agriculture (\%) & 27.3 ~ \ \ & 20.1 ~ \ \ & \textit{18.1} ~ \ \ \\ 
	 ~ Employment in agriculture, female (\%) & 15.9 ~ \ \ & 9.9 ~ \ \ & \textit{7.9} ~ \ \ \\ 
	 ~ Fertilizers, Nitrogen (nutrients per ha) &  ~ \ \ & 17.3 ~ \ \ & \textit{13.5} ~ \ \ \\ 
	 ~ Fertilizers, Phosphate (nutrients per ha) &  ~ \ \ & 8.8 ~ \ \ & \textit{8.2} ~ \ \ \\ 
	 ~ Fertilizers, Potash (nutrients per ha) &  ~ \ \ & 0.8 ~ \ \ & \textit{0.9} ~ \ \ \\ 
	 ~ Energy consump, power irrigation (mln kWh) &  ~ \ \ & 13 ~ \ \ & \textit{13} ~ \ \ \\ 
	 ~ Agr value added per worker (constant US\$) & 2.9 ~ \ \ & 2.8 ~ \ \ & \textit{3.7} ~ \ \ \\ 
	\multicolumn{4}{l}{\textcolor{FAOblue}{\textbf{\large{Hunger dimensions}}}} \\ 
	 ~ Dietary energy supply (kcal/pc/day) & 2\,637 ~ \ \ & 2\,801 ~ \ \ & 2\,793 ~ \ \ \\ 
	 ~ Average dietary energy supply adequacy (\%) & 114 ~ \ \ & 119 ~ \ \ & 116 ~ \ \ \\ 
	 ~ Dietary en supp, cereals/roots/tubers (\%) & 41 ~ \ \ & 37 ~ \ \ & \textit{39} ~ \ \ \\ 
	 ~ Prevalence of undernourishment (\%) & 10.5 ~ \ \ & 6.8 ~ \ \ & 8.7 ~ \ \ \\ 
	 ~ GDP per capita (US\$, PPP) & 7\,792 ~ \ \ & \textit{8\,197} ~ \ \ & \textit{8\,608} ~ \ \ \\ 
	 ~ Domestic food price volatility (index) &  ~ \ \ & 7.4 ~ \ \ & 7 ~ \ \ \\ 
	 ~ Cereal import dependency ratio (\%) & 99.3 ~ \ \ & 99.6 ~ \ \ & \textit{99.5} ~ \ \ \\ 
	 ~ Underweight, children under-5 (\%) & 7 ~ \ \ & 2.6 ~ \ \ & \textit{3.2} ~ \ \ \\ 
	 ~ Improved water source (\% pop) & 93.4 ~ \ \ & 93.4 ~ \ \ & \textit{93.1} ~ \ \ \\ 
	\multicolumn{4}{l}{\textcolor{FAOblue}{\textbf{\large{Food Supply}}}} \\ 
	 ~ Food production value, (2004-2006 mln I\$) & 523 ~ \ \ & 540 ~ \ \ & \textit{563} ~ \ \ \\ 
	 ~ Agriculture, value added (\% GDP) & \textit{9} ~ \ \ & 6 ~ \ \ & \textit{7} ~ \ \ \\ 
	 ~ Food exports (mln US\$)  & 182 ~ \ \ & 181 ~ \ \ & \textit{234} ~ \ \ \\ 
	 ~ Food imports (mln US\$)  & 222 ~ \ \ & 344 ~ \ \ & \textit{824} ~ \ \ \\ 
	\multicolumn{4}{l}{\textit{\normalsize{Production indices (2004-06=100)}}} \\ 
	 ~ Net food & 99 ~ \ \ & 102 ~ \ \ & \textit{106} ~ \ \ \\ 
	 ~ Net crop & 111 ~ \ \ & 105 ~ \ \ & \textit{108} ~ \ \ \\ 
	 ~ Cereal & 266 ~ \ \ & 97 ~ \ \ & \textit{166} ~ \ \ \\ 
	 ~ Vegetable oils & 56 ~ \ \ & 95 ~ \ \ & \textit{109} ~ \ \ \\ 
	 ~ Roots and tubers & 165 ~ \ \ & 116 ~ \ \ & \textit{124} ~ \ \ \\ 
	 ~ Fruit and vegetables & 99 ~ \ \ & 106 ~ \ \ & \textit{107} ~ \ \ \\ 
	 ~ Sugar & 148 ~ \ \ & 115 ~ \ \ & \textit{82} ~ \ \ \\ 
	 ~ Livestock & 82 ~ \ \ & 95 ~ \ \ & \textit{103} ~ \ \ \\ 
	 ~ Milk & 103 ~ \ \ & 101 ~ \ \ & \textit{110} ~ \ \ \\ 
	 ~ Meat & 74 ~ \ \ & 93 ~ \ \ & \textit{100} ~ \ \ \\ 
	 ~ Fish  & 107 ~ \ \ & 101 ~ \ \ & \textit{124} ~ \ \ \\ 
	\multicolumn{4}{l}{\textit{\normalsize{Net trade (min US\$)}}} \\ 
	 ~ Cereals & -84 ~ \ \ & -121 ~ \ \ & \textit{-269} ~ \ \ \\ 
	 ~ Fruit and vegetables & 62 ~ \ \ & 23 ~ \ \ & \textit{-35} ~ \ \ \\ 
	 ~ Meat & -37 ~ \ \ & -47 ~ \ \ & \textit{-99} ~ \ \ \\ 
	 ~ Dairy products & -18 ~ \ \ & -25 ~ \ \ & \textit{-38} ~ \ \ \\ 
	 ~ Fish & -14 ~ \ \ & -53 ~ \ \ & \textit{-99} ~ \ \ \\ 
	\multicolumn{4}{l}{\textcolor{FAOblue}{\textbf{\large{Environment}}}} \\ 
	 ~ Forest area (\%) & 32 ~ \ \ & 31 ~ \ \ & \textit{31} ~ \ \ \\ 
	 ~ Renewable water res withdrawn (\% of total) &  ~ \ \ &  ~ \ \ & 55 ~ \ \ \\ 
	 ~ Terrestrial protect areas (\% total land area)  & 10 ~ \ \ & 19 ~ \ \ & \textit{16} ~ \ \ \\ 
	 ~ Organic area (\% total agricultural area) &  ~ \ \ & \textit{0} ~ \ \ & \textit{0} ~ \ \ \\ 
	 ~ Water withdrawal by agriculture (\% of total) &  ~ \ \ &  ~ \ \ & 55 ~ \ \ \\ 
	 ~ Biofuel production (thousand kt of oil eq.) & 6 ~ \ \ & 4 ~ \ \ & \textit{2} ~ \ \ \\ 
	 ~ Wood pellet prod. (min tonnes) &  ~ \ \ &  ~ \ \ &  ~ \ \ \\ 
	 ~ GHG emissions from ag (Co2 eq, gigagrams) & 2 ~ \ \ & 2 ~ \ \ & \textit{1} ~ \ \ \\ 
       \toprule
      \end{tabular}
      \clearpage
\CountryData{ Japan }
      \rowcolors{1}{FAOblue!10}{white}
      \begin{tabular}{L{3.9cm} R{1cm} R{1cm} R{1cm}}
      \toprule
      \multicolumn{1}{c}{} & \multicolumn{1}{c}{ 1992 } & \multicolumn{1}{c}{ 2002 } & \multicolumn{1}{c}{ 2014 } \\
      \midrule
	\multicolumn{4}{l}{\textcolor{FAOblue}{\textbf{\large{The setting}}}} \\ 
	 ~ Population, total (mln) & 123.2 ~ \ \ & 126.2 ~ \ \ & 127 ~ \ \ \\ 
	 ~ Population, rural (\% total population) & 27.6 ~ \ \ & 23.2 ~ \ \ & 8.8 ~ \ \ \\ 
	 ~ Govt expenditure on ag (\% total outlays) &  ~ \ \ &  ~ \ \ &  ~ \ \ \\ 
	 ~ Area harvested (mln ha) & 14 ~ \ \ & 12 ~ \ \ & 12 ~ \ \ \\ 
	 ~ Cropping intensity ratio (\%) & 2.6 ~ \ \ & 2.6 ~ \ \ &  ~ \ \ \\ 
	 ~ Water resources (m\textsuperscript{3}/person/year) & \textit{3} ~ \ \ & \textit{3} ~ \ \ & \textit{3} ~ \ \ \\ 
	 ~ Area equipped for irrigation (1000 ha) &  ~ \ \ &  ~ \ \ & \textit{2\,469} ~ \ \ \\ 
	 ~ Area irrigated (\%) &  ~ \ \ &  ~ \ \ & \textit{92.9} ~ \ \ \\ 
	 ~ Employment in agriculture (\%) & 6.4 ~ \ \ & 4.7 ~ \ \ & \textit{3.7} ~ \ \ \\ 
	 ~ Employment in agriculture, female (\%) & 7.3 ~ \ \ & 4.9 ~ \ \ & \textit{3.7} ~ \ \ \\ 
	 ~ Fertilizers, Nitrogen (nutrients per ha) &  ~ \ \ & 111.5 ~ \ \ & \textit{95.1} ~ \ \ \\ 
	 ~ Fertilizers, Phosphate (nutrients per ha) &  ~ \ \ & 133.4 ~ \ \ & \textit{89.7} ~ \ \ \\ 
	 ~ Fertilizers, Potash (nutrients per ha) &  ~ \ \ & 64.5 ~ \ \ & \textit{57.1} ~ \ \ \\ 
	 ~ Energy consump, power irrigation (mln kWh) & 377 ~ \ \ & 647 ~ \ \ & \textit{1\,063} ~ \ \ \\ 
	 ~ Agr value added per worker (constant US\$) & 15.3 ~ \ \ & 28.1 ~ \ \ & \textit{46} ~ \ \ \\ 
	\multicolumn{4}{l}{\textcolor{FAOblue}{\textbf{\large{Hunger dimensions}}}} \\ 
	 ~ Dietary energy supply (kcal/pc/day) &  ~ \ \ &  ~ \ \ &  ~ \ \ \\ 
	 ~ Average dietary energy supply adequacy (\%) & 121 ~ \ \ & 118 ~ \ \ & 112 ~ \ \ \\ 
	 ~ Dietary en supp, cereals/roots/tubers (\%) & 42 ~ \ \ & 41 ~ \ \ & \textit{41} ~ \ \ \\ 
	 ~ Prevalence of undernourishment (\%) & <5.0 ~ \ \ & <5.0 ~ \ \ & <5.0 ~ \ \ \\ 
	 ~ GDP per capita (US\$, PPP) & 30\,609 ~ \ \ & 32\,248 ~ \ \ & \textit{35\,614} ~ \ \ \\ 
	 ~ Domestic food price volatility (index) &  ~ \ \ & 6.1 ~ \ \ & 5.6 ~ \ \ \\ 
	 ~ Cereal import dependency ratio (\%) & 78.9 ~ \ \ & 79.9 ~ \ \ & \textit{79.7} ~ \ \ \\ 
	 ~ Underweight, children under-5 (\%) &  ~ \ \ &  ~ \ \ &  ~ \ \ \\ 
	 ~ Improved water source (\% pop) & 100 ~ \ \ & 100 ~ \ \ & \textit{100} ~ \ \ \\ 
	\multicolumn{4}{l}{\textcolor{FAOblue}{\textbf{\large{Food Supply}}}} \\ 
	 ~ Food production value, (2004-2006 mln I\$) & 20\,619 ~ \ \ & 18\,559 ~ \ \ & \textit{17\,730} ~ \ \ \\ 
	 ~ Agriculture, value added (\% GDP) & 2 ~ \ \ & 1 ~ \ \ & \textit{1} ~ \ \ \\ 
	 ~ Food exports (mln US\$)  & 768 ~ \ \ & 881 ~ \ \ & \textit{1\,750} ~ \ \ \\ 
	 ~ Food imports (mln US\$)  & 20\,431 ~ \ \ & 23\,648 ~ \ \ & \textit{45\,107} ~ \ \ \\ 
	\multicolumn{4}{l}{\textit{\normalsize{Production indices (2004-06=100)}}} \\ 
	 ~ Net food & 114 ~ \ \ & 103 ~ \ \ & \textit{98} ~ \ \ \\ 
	 ~ Net crop & 121 ~ \ \ & 104 ~ \ \ & \textit{94} ~ \ \ \\ 
	 ~ Cereal & 119 ~ \ \ & 101 ~ \ \ & \textit{97} ~ \ \ \\ 
	 ~ Vegetable oils & 119 ~ \ \ & 128 ~ \ \ & \textit{93} ~ \ \ \\ 
	 ~ Roots and tubers & 125 ~ \ \ & 109 ~ \ \ & \textit{94} ~ \ \ \\ 
	 ~ Fruit and vegetables & 124 ~ \ \ & 107 ~ \ \ & \textit{93} ~ \ \ \\ 
	 ~ Sugar & 95 ~ \ \ & 98 ~ \ \ & \textit{83} ~ \ \ \\ 
	 ~ Livestock & 109 ~ \ \ & 101 ~ \ \ & \textit{101} ~ \ \ \\ 
	 ~ Milk & 104 ~ \ \ & 102 ~ \ \ & \textit{91} ~ \ \ \\ 
	 ~ Meat & 112 ~ \ \ & 100 ~ \ \ & \textit{106} ~ \ \ \\ 
	 ~ Fish  & 168 ~ \ \ & 102 ~ \ \ & \textit{84} ~ \ \ \\ 
	\multicolumn{4}{l}{\textit{\normalsize{Net trade (min US\$)}}} \\ 
	 ~ Cereals & -4\,524 ~ \ \ & -4\,534 ~ \ \ & \textit{-10\,064} ~ \ \ \\ 
	 ~ Fruit and vegetables & -4\,478 ~ \ \ & -5\,488 ~ \ \ & \textit{-9\,682} ~ \ \ \\ 
	 ~ Meat & -6\,456 ~ \ \ & -7\,755 ~ \ \ & \textit{-13\,206} ~ \ \ \\ 
	 ~ Dairy products & -547 ~ \ \ & -720 ~ \ \ & \textit{-1\,450} ~ \ \ \\ 
	 ~ Fish & -12\,039 ~ \ \ & -12\,857 ~ \ \ & \textit{-16\,174} ~ \ \ \\ 
	\multicolumn{4}{l}{\textcolor{FAOblue}{\textbf{\large{Environment}}}} \\ 
	 ~ Forest area (\%) & 68 ~ \ \ & 68 ~ \ \ & \textit{69} ~ \ \ \\ 
	 ~ Renewable water res withdrawn (\% of total) &  ~ \ \ & \textit{63} ~ \ \ & 63 ~ \ \ \\ 
	 ~ Terrestrial protect areas (\% total land area)  & 14 ~ \ \ & 16 ~ \ \ & \textit{17} ~ \ \ \\ 
	 ~ Organic area (\% total agricultural area) &  ~ \ \ &  ~ \ \ & \textit{0} ~ \ \ \\ 
	 ~ Water withdrawal by agriculture (\% of total) &  ~ \ \ & \textit{63} ~ \ \ & 63 ~ \ \ \\ 
	 ~ Biofuel production (thousand kt of oil eq.) & 6 ~ \ \ & 8 ~ \ \ & \textit{8} ~ \ \ \\ 
	 ~ Wood pellet prod. (min tonnes) &  ~ \ \ &  ~ \ \ & \textit{90} ~ \ \ \\ 
	 ~ GHG emissions from ag (Co2 eq, gigagrams) & -45 ~ \ \ & -73 ~ \ \ & \textit{-113} ~ \ \ \\ 
       \toprule
      \end{tabular}
      \clearpage
\CountryData{ Jordan }
      \rowcolors{1}{FAOblue!10}{white}
      \begin{tabular}{L{3.9cm} R{1cm} R{1cm} R{1cm}}
      \toprule
      \multicolumn{1}{c}{} & \multicolumn{1}{c}{ 1992 } & \multicolumn{1}{c}{ 2002 } & \multicolumn{1}{c}{ 2014 } \\
      \midrule
	\multicolumn{4}{l}{\textcolor{FAOblue}{\textbf{\large{The setting}}}} \\ 
	 ~ Population, total (mln) & 3.7 ~ \ \ & 4.9 ~ \ \ & 7.5 ~ \ \ \\ 
	 ~ Population, rural (\% total population) & 0.9 ~ \ \ & 1 ~ \ \ & 1.2 ~ \ \ \\ 
	 ~ Govt expenditure on ag (\% total outlays) &  ~ \ \ & 1.2 ~ \ \ & \textit{0.7} ~ \ \ \\ 
	 ~ Area harvested (mln ha) & 1 ~ \ \ & 1 ~ \ \ & 0 ~ \ \ \\ 
	 ~ Cropping intensity ratio (\%) & 0.8 ~ \ \ & 1.1 ~ \ \ &  ~ \ \ \\ 
	 ~ Water resources (m\textsuperscript{3}/person/year) & \textit{0} ~ \ \ & \textit{0} ~ \ \ & \textit{0} ~ \ \ \\ 
	 ~ Area equipped for irrigation (1000 ha) &  ~ \ \ &  ~ \ \ & \textit{96} ~ \ \ \\ 
	 ~ Area irrigated (\%) &  ~ \ \ &  ~ \ \ & \textit{91.3} ~ \ \ \\ 
	 ~ Employment in agriculture (\%) &  ~ \ \ & 3.9 ~ \ \ & \textit{2} ~ \ \ \\ 
	 ~ Employment in agriculture, female (\%) &  ~ \ \ & 2.5 ~ \ \ & \textit{0.9} ~ \ \ \\ 
	 ~ Fertilizers, Nitrogen (nutrients per ha) &  ~ \ \ & 86.7 ~ \ \ & \textit{29.7} ~ \ \ \\ 
	 ~ Fertilizers, Phosphate (nutrients per ha) &  ~ \ \ & 194.3 ~ \ \ & \textit{25} ~ \ \ \\ 
	 ~ Fertilizers, Potash (nutrients per ha) &  ~ \ \ & 21.6 ~ \ \ & \textit{204.5} ~ \ \ \\ 
	 ~ Energy consump, power irrigation (mln kWh) & 106 ~ \ \ & 106 ~ \ \ & \textit{157} ~ \ \ \\ 
	 ~ Agr value added per worker (constant US\$) & 3.5 ~ \ \ & 2.4 ~ \ \ & \textit{4.4} ~ \ \ \\ 
	\multicolumn{4}{l}{\textcolor{FAOblue}{\textbf{\large{Hunger dimensions}}}} \\ 
	 ~ Dietary energy supply (kcal/pc/day) & 2\,761 ~ \ \ & 2\,952 ~ \ \ & 3\,132 ~ \ \ \\ 
	 ~ Average dietary energy supply adequacy (\%) & 126 ~ \ \ & 132 ~ \ \ & 137 ~ \ \ \\ 
	 ~ Dietary en supp, cereals/roots/tubers (\%) & 49 ~ \ \ & 49 ~ \ \ & \textit{48} ~ \ \ \\ 
	 ~ Prevalence of undernourishment (\%) & 5.8 ~ \ \ & <5.0 ~ \ \ & <5.0 ~ \ \ \\ 
	 ~ GDP per capita (US\$, PPP) & 7\,242 ~ \ \ & 8\,160 ~ \ \ & \textit{11\,405} ~ \ \ \\ 
	 ~ Domestic food price volatility (index) &  ~ \ \ & 8.1 ~ \ \ & 6.1 ~ \ \ \\ 
	 ~ Cereal import dependency ratio (\%) & 92.2 ~ \ \ & 95.4 ~ \ \ & \textit{96.2} ~ \ \ \\ 
	 ~ Underweight, children under-5 (\%) & \textit{4.8} ~ \ \ & 3.6 ~ \ \ & \textit{3} ~ \ \ \\ 
	 ~ Improved water source (\% pop) & 96.8 ~ \ \ & 96.6 ~ \ \ & \textit{96.1} ~ \ \ \\ 
	\multicolumn{4}{l}{\textcolor{FAOblue}{\textbf{\large{Food Supply}}}} \\ 
	 ~ Food production value, (2004-2006 mln I\$) & 682 ~ \ \ & 891 ~ \ \ & \textit{1\,315} ~ \ \ \\ 
	 ~ Agriculture, value added (\% GDP) & 8 ~ \ \ & 3 ~ \ \ & \textit{3} ~ \ \ \\ 
	 ~ Food exports (mln US\$)  & 145 ~ \ \ & 251 ~ \ \ & \textit{1\,217} ~ \ \ \\ 
	 ~ Food imports (mln US\$)  & 635 ~ \ \ & 708 ~ \ \ & \textit{3\,016} ~ \ \ \\ 
	\multicolumn{4}{l}{\textit{\normalsize{Production indices (2004-06=100)}}} \\ 
	 ~ Net food & 71 ~ \ \ & 93 ~ \ \ & \textit{137} ~ \ \ \\ 
	 ~ Net crop & 75 ~ \ \ & 97 ~ \ \ & \textit{131} ~ \ \ \\ 
	 ~ Cereal & 230 ~ \ \ & 161 ~ \ \ & \textit{150} ~ \ \ \\ 
	 ~ Vegetable oils & 59 ~ \ \ & 129 ~ \ \ & \textit{91} ~ \ \ \\ 
	 ~ Roots and tubers & 26 ~ \ \ & 62 ~ \ \ & \textit{63} ~ \ \ \\ 
	 ~ Fruit and vegetables & 77 ~ \ \ & 90 ~ \ \ & \textit{144} ~ \ \ \\ 
	 ~ Sugar &  ~ \ \ &  ~ \ \ &  ~ \ \ \\ 
	 ~ Livestock & 68 ~ \ \ & 86 ~ \ \ & \textit{147} ~ \ \ \\ 
	 ~ Milk & 55 ~ \ \ & 71 ~ \ \ & \textit{105} ~ \ \ \\ 
	 ~ Meat & 64 ~ \ \ & 90 ~ \ \ & \textit{175} ~ \ \ \\ 
	 ~ Fish  & 39 ~ \ \ & 100 ~ \ \ & \textit{118} ~ \ \ \\ 
	\multicolumn{4}{l}{\textit{\normalsize{Net trade (min US\$)}}} \\ 
	 ~ Cereals & -225 ~ \ \ & -232 ~ \ \ & \textit{-961} ~ \ \ \\ 
	 ~ Fruit and vegetables & 4 ~ \ \ & 58 ~ \ \ & \textit{307} ~ \ \ \\ 
	 ~ Meat & -68 ~ \ \ & -51 ~ \ \ & \textit{-279} ~ \ \ \\ 
	 ~ Dairy products & -80 ~ \ \ & -17 ~ \ \ & \textit{-203} ~ \ \ \\ 
	 ~ Fish & -14 ~ \ \ & -27 ~ \ \ & \textit{-105} ~ \ \ \\ 
	\multicolumn{4}{l}{\textcolor{FAOblue}{\textbf{\large{Environment}}}} \\ 
	 ~ Forest area (\%) & 1 ~ \ \ & 1 ~ \ \ & \textit{1} ~ \ \ \\ 
	 ~ Renewable water res withdrawn (\% of total) &  ~ \ \ & \textit{65} ~ \ \ & 65 ~ \ \ \\ 
	 ~ Terrestrial protect areas (\% total land area)  & 1 ~ \ \ & 2 ~ \ \ & \textit{2} ~ \ \ \\ 
	 ~ Organic area (\% total agricultural area) &  ~ \ \ & \textit{0} ~ \ \ & \textit{0} ~ \ \ \\ 
	 ~ Water withdrawal by agriculture (\% of total) &  ~ \ \ & \textit{65} ~ \ \ & 65 ~ \ \ \\ 
	 ~ Biofuel production (thousand kt of oil eq.) &  ~ \ \ & 0 ~ \ \ & \textit{0} ~ \ \ \\ 
	 ~ Wood pellet prod. (min tonnes) &  ~ \ \ &  ~ \ \ &  ~ \ \ \\ 
	 ~ GHG emissions from ag (Co2 eq, gigagrams) & 1 ~ \ \ & 1 ~ \ \ & \textit{1} ~ \ \ \\ 
       \toprule
      \end{tabular}
      \clearpage
\CountryData{ Kazakhstan }
      \rowcolors{1}{FAOblue!10}{white}
      \begin{tabular}{L{3.9cm} R{1cm} R{1cm} R{1cm}}
      \toprule
      \multicolumn{1}{c}{} & \multicolumn{1}{c}{ 1992 } & \multicolumn{1}{c}{ 2002 } & \multicolumn{1}{c}{ 2014 } \\
      \midrule
	\multicolumn{4}{l}{\textcolor{FAOblue}{\textbf{\large{The setting}}}} \\ 
	 ~ Population, total (mln) & 16.1 ~ \ \ & 14.6 ~ \ \ & 16.6 ~ \ \ \\ 
	 ~ Population, rural (\% total population) & 7.1 ~ \ \ & 6.5 ~ \ \ & 7.8 ~ \ \ \\ 
	 ~ Govt expenditure on ag (\% total outlays) &  ~ \ \ &  ~ \ \ &  ~ \ \ \\ 
	 ~ Area harvested (mln ha) & 30 ~ \ \ & 16 ~ \ \ & 18 ~ \ \ \\ 
	 ~ Cropping intensity ratio (\%) & 0.1 ~ \ \ & 0.1 ~ \ \ &  ~ \ \ \\ 
	 ~ Water resources (m\textsuperscript{3}/person/year) & \textit{7} ~ \ \ & \textit{7} ~ \ \ & \textit{7} ~ \ \ \\ 
	 ~ Area equipped for irrigation (1000 ha) &  ~ \ \ &  ~ \ \ & \textit{2\,066} ~ \ \ \\ 
	 ~ Area irrigated (\%) &  ~ \ \ &  ~ \ \ & \textit{61.2} ~ \ \ \\ 
	 ~ Employment in agriculture (\%) &  ~ \ \ & 35.5 ~ \ \ & \textit{25.5} ~ \ \ \\ 
	 ~ Employment in agriculture, female (\%) &  ~ \ \ & 34.7 ~ \ \ & \textit{29.2} ~ \ \ \\ 
	 ~ Fertilizers, Nitrogen (nutrients per ha) &  ~ \ \ & 0.1 ~ \ \ & \textit{0.1} ~ \ \ \\ 
	 ~ Fertilizers, Phosphate (nutrients per ha) &  ~ \ \ & 0.1 ~ \ \ & \textit{0.1} ~ \ \ \\ 
	 ~ Fertilizers, Potash (nutrients per ha) &  ~ \ \ & 0 ~ \ \ & \textit{0} ~ \ \ \\ 
	 ~ Energy consump, power irrigation (mln kWh) & \textit{1\,325} ~ \ \ & 1\,325 ~ \ \ & \textit{98} ~ \ \ \\ 
	 ~ Agr value added per worker (constant US\$) & 3.3 ~ \ \ & 2.6 ~ \ \ & \textit{4} ~ \ \ \\ 
	\multicolumn{4}{l}{\textcolor{FAOblue}{\textbf{\large{Hunger dimensions}}}} \\ 
	 ~ Dietary energy supply (kcal/pc/day) & 2\,906 ~ \ \ & 2\,926 ~ \ \ & 3\,154 ~ \ \ \\ 
	 ~ Average dietary energy supply adequacy (\%) & 128 ~ \ \ & 125 ~ \ \ & 136 ~ \ \ \\ 
	 ~ Dietary en supp, cereals/roots/tubers (\%) & 60 ~ \ \ & 51 ~ \ \ & \textit{39} ~ \ \ \\ 
	 ~ Prevalence of undernourishment (\%) & <5.0 ~ \ \ & <5.0 ~ \ \ & <5.0 ~ \ \ \\ 
	 ~ GDP per capita (US\$, PPP) & 10\,668 ~ \ \ & 12\,116 ~ \ \ & \textit{22\,470} ~ \ \ \\ 
	 ~ Domestic food price volatility (index) &  ~ \ \ &  ~ \ \ &  ~ \ \ \\ 
	 ~ Cereal import dependency ratio (\%) & -27.4 ~ \ \ & -44.8 ~ \ \ & \textit{-50.6} ~ \ \ \\ 
	 ~ Underweight, children under-5 (\%) & \textit{6.7} ~ \ \ & \textit{3.8} ~ \ \ & \textit{3.7} ~ \ \ \\ 
	 ~ Improved water source (\% pop) & 94.1 ~ \ \ & 93.7 ~ \ \ & \textit{93.1} ~ \ \ \\ 
	\multicolumn{4}{l}{\textcolor{FAOblue}{\textbf{\large{Food Supply}}}} \\ 
	 ~ Food production value, (2004-2006 mln I\$) & 8\,400 ~ \ \ & 5\,616 ~ \ \ & \textit{7\,421} ~ \ \ \\ 
	 ~ Agriculture, value added (\% GDP) & 27 ~ \ \ & 9 ~ \ \ & \textit{5} ~ \ \ \\ 
	 ~ Food exports (mln US\$)  & 344 ~ \ \ & 424 ~ \ \ & \textit{2\,634} ~ \ \ \\ 
	 ~ Food imports (mln US\$)  & 532 ~ \ \ & 395 ~ \ \ & \textit{3\,307} ~ \ \ \\ 
	\multicolumn{4}{l}{\textit{\normalsize{Production indices (2004-06=100)}}} \\ 
	 ~ Net food & 142 ~ \ \ & 95 ~ \ \ & \textit{125} ~ \ \ \\ 
	 ~ Net crop & 144 ~ \ \ & 101 ~ \ \ & \textit{146} ~ \ \ \\ 
	 ~ Cereal & 214 ~ \ \ & 118 ~ \ \ & \textit{134} ~ \ \ \\ 
	 ~ Vegetable oils & 55 ~ \ \ & 68 ~ \ \ & \textit{281} ~ \ \ \\ 
	 ~ Roots and tubers & 90 ~ \ \ & 87 ~ \ \ & \textit{144} ~ \ \ \\ 
	 ~ Fruit and vegetables & 54 ~ \ \ & 86 ~ \ \ & \textit{161} ~ \ \ \\ 
	 ~ Sugar & 332 ~ \ \ & 107 ~ \ \ & \textit{19} ~ \ \ \\ 
	 ~ Livestock & 146 ~ \ \ & 87 ~ \ \ & \textit{111} ~ \ \ \\ 
	 ~ Milk & 111 ~ \ \ & 87 ~ \ \ & \textit{104} ~ \ \ \\ 
	 ~ Meat & 170 ~ \ \ & 88 ~ \ \ & \textit{113} ~ \ \ \\ 
	 ~ Fish  & 202 ~ \ \ & 75 ~ \ \ & \textit{105} ~ \ \ \\ 
	\multicolumn{4}{l}{\textit{\normalsize{Net trade (min US\$)}}} \\ 
	 ~ Cereals & 198 ~ \ \ & 333 ~ \ \ & \textit{2\,019} ~ \ \ \\ 
	 ~ Fruit and vegetables & -70 ~ \ \ & -6 ~ \ \ & \textit{-840} ~ \ \ \\ 
	 ~ Meat & -8 ~ \ \ & -31 ~ \ \ & \textit{-424} ~ \ \ \\ 
	 ~ Dairy products & -20 ~ \ \ & -34 ~ \ \ & \textit{-365} ~ \ \ \\ 
	 ~ Fish & \textit{5} ~ \ \ & 6 ~ \ \ & \textit{-20} ~ \ \ \\ 
	\multicolumn{4}{l}{\textcolor{FAOblue}{\textbf{\large{Environment}}}} \\ 
	 ~ Forest area (\%) & 1 ~ \ \ & 1 ~ \ \ & \textit{1} ~ \ \ \\ 
	 ~ Renewable water res withdrawn (\% of total) &  ~ \ \ &  ~ \ \ & 66 ~ \ \ \\ 
	 ~ Terrestrial protect areas (\% total land area)  & 2 ~ \ \ & 3 ~ \ \ & \textit{3} ~ \ \ \\ 
	 ~ Organic area (\% total agricultural area) &  ~ \ \ &  ~ \ \ & \textit{0} ~ \ \ \\ 
	 ~ Water withdrawal by agriculture (\% of total) &  ~ \ \ &  ~ \ \ & 66 ~ \ \ \\ 
	 ~ Biofuel production (thousand kt of oil eq.) &  ~ \ \ &  ~ \ \ &  ~ \ \ \\ 
	 ~ Wood pellet prod. (min tonnes) &  ~ \ \ &  ~ \ \ & \textit{0} ~ \ \ \\ 
	 ~ GHG emissions from ag (Co2 eq, gigagrams) & 33 ~ \ \ & 22 ~ \ \ & \textit{19} ~ \ \ \\ 
       \toprule
      \end{tabular}
      \clearpage
\CountryData{ Kenya }
      \rowcolors{1}{FAOblue!10}{white}
      \begin{tabular}{L{3.9cm} R{1cm} R{1cm} R{1cm}}
      \toprule
      \multicolumn{1}{c}{} & \multicolumn{1}{c}{ 1992 } & \multicolumn{1}{c}{ 2002 } & \multicolumn{1}{c}{ 2014 } \\
      \midrule
	\multicolumn{4}{l}{\textcolor{FAOblue}{\textbf{\large{The setting}}}} \\ 
	 ~ Population, total (mln) & 25 ~ \ \ & 33 ~ \ \ & 45.5 ~ \ \ \\ 
	 ~ Population, rural (\% total population) & 20.7 ~ \ \ & 26.2 ~ \ \ & 34.1 ~ \ \ \\ 
	 ~ Govt expenditure on ag (\% total outlays) &  ~ \ \ & 5 ~ \ \ & \textit{5.1} ~ \ \ \\ 
	 ~ Area harvested (mln ha) & 4 ~ \ \ & 5 ~ \ \ & 6 ~ \ \ \\ 
	 ~ Cropping intensity ratio (\%) & 0.2 ~ \ \ & 0.2 ~ \ \ &  ~ \ \ \\ 
	 ~ Water resources (m\textsuperscript{3}/person/year) & \textit{1} ~ \ \ & \textit{1} ~ \ \ & \textit{1} ~ \ \ \\ 
	 ~ Area equipped for irrigation (1000 ha) &  ~ \ \ &  ~ \ \ & \textit{103} ~ \ \ \\ 
	 ~ Area irrigated (\%) &  ~ \ \ & \textit{94.2} ~ \ \ &  ~ \ \ \\ 
	 ~ Employment in agriculture (\%) &  ~ \ \ & \textit{61.1} ~ \ \ & \textit{61.1} ~ \ \ \\ 
	 ~ Employment in agriculture, female (\%) &  ~ \ \ & \textit{68} ~ \ \ & \textit{68} ~ \ \ \\ 
	 ~ Fertilizers, Nitrogen (nutrients per ha) &  ~ \ \ & 2.3 ~ \ \ & \textit{4.1} ~ \ \ \\ 
	 ~ Fertilizers, Phosphate (nutrients per ha) &  ~ \ \ & 2.8 ~ \ \ & \textit{3.7} ~ \ \ \\ 
	 ~ Fertilizers, Potash (nutrients per ha) &  ~ \ \ & 0 ~ \ \ & \textit{1.2} ~ \ \ \\ 
	 ~ Energy consump, power irrigation (mln kWh) & 53 ~ \ \ & 53 ~ \ \ & \textit{154} ~ \ \ \\ 
	 ~ Agr value added per worker (constant US\$) & 0.4 ~ \ \ & 0.4 ~ \ \ & \textit{0.4} ~ \ \ \\ 
	\multicolumn{4}{l}{\textcolor{FAOblue}{\textbf{\large{Hunger dimensions}}}} \\ 
	 ~ Dietary energy supply (kcal/pc/day) & 1\,945 ~ \ \ & 2\,030 ~ \ \ & 2\,210 ~ \ \ \\ 
	 ~ Average dietary energy supply adequacy (\%) & 93 ~ \ \ & 94 ~ \ \ & 102 ~ \ \ \\ 
	 ~ Dietary en supp, cereals/roots/tubers (\%) & 55 ~ \ \ & 56 ~ \ \ & \textit{58} ~ \ \ \\ 
	 ~ Prevalence of undernourishment (\%) & 35.3 ~ \ \ & 33.1 ~ \ \ & 21.5 ~ \ \ \\ 
	 ~ GDP per capita (US\$, PPP) & 2\,239 ~ \ \ & 2\,120 ~ \ \ & \textit{2\,705} ~ \ \ \\ 
	 ~ Domestic food price volatility (index) &  ~ \ \ & 8.2 ~ \ \ & 6 ~ \ \ \\ 
	 ~ Cereal import dependency ratio (\%) & 8.8 ~ \ \ & 20.8 ~ \ \ & \textit{36.4} ~ \ \ \\ 
	 ~ Underweight, children under-5 (\%) & \textit{19.8} ~ \ \ & \textit{18.4} ~ \ \ & \textit{16.4} ~ \ \ \\ 
	 ~ Improved water source (\% pop) & 44.6 ~ \ \ & 53.5 ~ \ \ & \textit{61.7} ~ \ \ \\ 
	\multicolumn{4}{l}{\textcolor{FAOblue}{\textbf{\large{Food Supply}}}} \\ 
	 ~ Food production value, (2004-2006 mln I\$) & 3\,882 ~ \ \ & 4\,700 ~ \ \ & \textit{6\,840} ~ \ \ \\ 
	 ~ Agriculture, value added (\% GDP) & 29 ~ \ \ & 29 ~ \ \ & \textit{30} ~ \ \ \\ 
	 ~ Food exports (mln US\$)  & 241 ~ \ \ & 204 ~ \ \ & \textit{382} ~ \ \ \\ 
	 ~ Food imports (mln US\$)  & 305 ~ \ \ & 328 ~ \ \ & \textit{1\,216} ~ \ \ \\ 
	\multicolumn{4}{l}{\textit{\normalsize{Production indices (2004-06=100)}}} \\ 
	 ~ Net food & 70 ~ \ \ & 84 ~ \ \ & \textit{123} ~ \ \ \\ 
	 ~ Net crop & 77 ~ \ \ & 87 ~ \ \ & \textit{127} ~ \ \ \\ 
	 ~ Cereal & 84 ~ \ \ & 85 ~ \ \ & \textit{123} ~ \ \ \\ 
	 ~ Vegetable oils & 70 ~ \ \ & 90 ~ \ \ & \textit{111} ~ \ \ \\ 
	 ~ Roots and tubers & 41 ~ \ \ & 53 ~ \ \ & \textit{127} ~ \ \ \\ 
	 ~ Fruit and vegetables & 73 ~ \ \ & 90 ~ \ \ & \textit{130} ~ \ \ \\ 
	 ~ Sugar & 87 ~ \ \ & 94 ~ \ \ & \textit{123} ~ \ \ \\ 
	 ~ Livestock & 63 ~ \ \ & 83 ~ \ \ & \textit{118} ~ \ \ \\ 
	 ~ Milk & 58 ~ \ \ & 79 ~ \ \ & \textit{122} ~ \ \ \\ 
	 ~ Meat & 66 ~ \ \ & 84 ~ \ \ & \textit{115} ~ \ \ \\ 
	 ~ Fish  & 112 ~ \ \ & 100 ~ \ \ & \textit{128} ~ \ \ \\ 
	\multicolumn{4}{l}{\textit{\normalsize{Net trade (min US\$)}}} \\ 
	 ~ Cereals & -60 ~ \ \ & -89 ~ \ \ & \textit{-606} ~ \ \ \\ 
	 ~ Fruit and vegetables & 90 ~ \ \ & 133 ~ \ \ & \textit{102} ~ \ \ \\ 
	 ~ Meat & 0 ~ \ \ & 1 ~ \ \ & \textit{7} ~ \ \ \\ 
	 ~ Dairy products & -4 ~ \ \ & -2 ~ \ \ & \textit{-25} ~ \ \ \\ 
	 ~ Fish & 23 ~ \ \ & 53 ~ \ \ & \textit{51} ~ \ \ \\ 
	\multicolumn{4}{l}{\textcolor{FAOblue}{\textbf{\large{Environment}}}} \\ 
	 ~ Forest area (\%) & 6 ~ \ \ & 6 ~ \ \ & \textit{6} ~ \ \ \\ 
	 ~ Renewable water res withdrawn (\% of total) &  ~ \ \ & \textit{79} ~ \ \ & 79 ~ \ \ \\ 
	 ~ Terrestrial protect areas (\% total land area)  & 12 ~ \ \ & 12 ~ \ \ & \textit{12} ~ \ \ \\ 
	 ~ Organic area (\% total agricultural area) &  ~ \ \ & \textit{0} ~ \ \ & \textit{0} ~ \ \ \\ 
	 ~ Water withdrawal by agriculture (\% of total) &  ~ \ \ & \textit{79} ~ \ \ & 79 ~ \ \ \\ 
	 ~ Biofuel production (thousand kt of oil eq.) & 42 ~ \ \ & 55 ~ \ \ & \textit{54} ~ \ \ \\ 
	 ~ Wood pellet prod. (min tonnes) &  ~ \ \ &  ~ \ \ &  ~ \ \ \\ 
	 ~ GHG emissions from ag (Co2 eq, gigagrams) & 32 ~ \ \ & 33 ~ \ \ & \textit{51} ~ \ \ \\ 
       \toprule
      \end{tabular}
      \clearpage
\CountryData{ Kuwait }
      \rowcolors{1}{FAOblue!10}{white}
      \begin{tabular}{L{3.9cm} R{1cm} R{1cm} R{1cm}}
      \toprule
      \multicolumn{1}{c}{} & \multicolumn{1}{c}{ 1992 } & \multicolumn{1}{c}{ 2002 } & \multicolumn{1}{c}{ 2014 } \\
      \midrule
	\multicolumn{4}{l}{\textcolor{FAOblue}{\textbf{\large{The setting}}}} \\ 
	 ~ Population, total (mln) & 1.9 ~ \ \ & 2 ~ \ \ & 3.5 ~ \ \ \\ 
	 ~ Population, rural (\% total population) & 0 ~ \ \ & 0 ~ \ \ & 0.1 ~ \ \ \\ 
	 ~ Govt expenditure on ag (\% total outlays) &  ~ \ \ & 0 ~ \ \ & \textit{0} ~ \ \ \\ 
	 ~ Area harvested (mln ha) & 1 ~ \ \ & 1 ~ \ \ & 0 ~ \ \ \\ 
	 ~ Cropping intensity ratio (\%) & 9.6 ~ \ \ & 5.1 ~ \ \ &  ~ \ \ \\ 
	 ~ Water resources (m\textsuperscript{3}/person/year) & \textit{0} ~ \ \ & \textit{0} ~ \ \ & \textit{0} ~ \ \ \\ 
	 ~ Area equipped for irrigation (1000 ha) &  ~ \ \ &  ~ \ \ & \textit{10} ~ \ \ \\ 
	 ~ Area irrigated (\%) &  ~ \ \ &  ~ \ \ & \textit{100} ~ \ \ \\ 
	 ~ Employment in agriculture (\%) & \textit{2.1} ~ \ \ & \textit{2.7} ~ \ \ & \textit{2.7} ~ \ \ \\ 
	 ~ Employment in agriculture, female (\%) &  ~ \ \ & \textit{0} ~ \ \ & \textit{0} ~ \ \ \\ 
	 ~ Fertilizers, Nitrogen (nutrients per ha) &  ~ \ \ & 140.1 ~ \ \ & \textit{0} ~ \ \ \\ 
	 ~ Fertilizers, Phosphate (nutrients per ha) &  ~ \ \ & 0 ~ \ \ & \textit{2.8} ~ \ \ \\ 
	 ~ Fertilizers, Potash (nutrients per ha) &  ~ \ \ & 0 ~ \ \ & \textit{19.4} ~ \ \ \\ 
	 ~ Energy consump, power irrigation (mln kWh) & \textit{4} ~ \ \ & 4 ~ \ \ & \textit{4} ~ \ \ \\ 
	 ~ Agr value added per worker (constant US\$) &  ~ \ \ &  ~ \ \ &  ~ \ \ \\ 
	\multicolumn{4}{l}{\textcolor{FAOblue}{\textbf{\large{Hunger dimensions}}}} \\ 
	 ~ Dietary energy supply (kcal/pc/day) & 2\,190 ~ \ \ & 3\,482 ~ \ \ & 3\,304 ~ \ \ \\ 
	 ~ Average dietary energy supply adequacy (\%) & 93 ~ \ \ & 143 ~ \ \ & 135 ~ \ \ \\ 
	 ~ Dietary en supp, cereals/roots/tubers (\%) & 47 ~ \ \ & 41 ~ \ \ & \textit{43} ~ \ \ \\ 
	 ~ Prevalence of undernourishment (\%) & 43.6 ~ \ \ & <5.0 ~ \ \ & <5.0 ~ \ \ \\ 
	 ~ GDP per capita (US\$, PPP) & \textit{81\,836} ~ \ \ & 72\,249 ~ \ \ & \textit{82\,358} ~ \ \ \\ 
	 ~ Domestic food price volatility (index) &  ~ \ \ & 10 ~ \ \ & 3.7 ~ \ \ \\ 
	 ~ Cereal import dependency ratio (\%) & 99.5 ~ \ \ & 99.4 ~ \ \ & \textit{97.8} ~ \ \ \\ 
	 ~ Underweight, children under-5 (\%) & \textit{9.2} ~ \ \ & 2.1 ~ \ \ & \textit{2.2} ~ \ \ \\ 
	 ~ Improved water source (\% pop) & 99 ~ \ \ & 99 ~ \ \ & \textit{99} ~ \ \ \\ 
	\multicolumn{4}{l}{\textcolor{FAOblue}{\textbf{\large{Food Supply}}}} \\ 
	 ~ Food production value, (2004-2006 mln I\$) & 33 ~ \ \ & 154 ~ \ \ & \textit{303} ~ \ \ \\ 
	 ~ Agriculture, value added (\% GDP) &  ~ \ \ &  ~ \ \ & \textit{0} ~ \ \ \\ 
	 ~ Food exports (mln US\$)  & 16 ~ \ \ & 29 ~ \ \ & \textit{131} ~ \ \ \\ 
	 ~ Food imports (mln US\$)  & 849 ~ \ \ & 792 ~ \ \ & \textit{2\,236} ~ \ \ \\ 
	\multicolumn{4}{l}{\textit{\normalsize{Production indices (2004-06=100)}}} \\ 
	 ~ Net food & 19 ~ \ \ & 86 ~ \ \ & \textit{170} ~ \ \ \\ 
	 ~ Net crop & 21 ~ \ \ & 79 ~ \ \ & \textit{163} ~ \ \ \\ 
	 ~ Cereal & 46 ~ \ \ & 132 ~ \ \ & \textit{749} ~ \ \ \\ 
	 ~ Vegetable oils & 36 ~ \ \ & 27 ~ \ \ & \textit{155} ~ \ \ \\ 
	 ~ Roots and tubers & 14 ~ \ \ & 84 ~ \ \ & \textit{226} ~ \ \ \\ 
	 ~ Fruit and vegetables & 21 ~ \ \ & 78 ~ \ \ & \textit{156} ~ \ \ \\ 
	 ~ Sugar &  ~ \ \ &  ~ \ \ &  ~ \ \ \\ 
	 ~ Livestock & 17 ~ \ \ & 92 ~ \ \ & \textit{177} ~ \ \ \\ 
	 ~ Milk & 21 ~ \ \ & 78 ~ \ \ & \textit{142} ~ \ \ \\ 
	 ~ Meat & 20 ~ \ \ & 100 ~ \ \ & \textit{183} ~ \ \ \\ 
	 ~ Fish  & 140 ~ \ \ & 99 ~ \ \ & \textit{88} ~ \ \ \\ 
	\multicolumn{4}{l}{\textit{\normalsize{Net trade (min US\$)}}} \\ 
	 ~ Cereals & -116 ~ \ \ & -169 ~ \ \ & \textit{-585} ~ \ \ \\ 
	 ~ Fruit and vegetables & -217 ~ \ \ & -127 ~ \ \ & \textit{-306} ~ \ \ \\ 
	 ~ Meat & -102 ~ \ \ & -82 ~ \ \ & \textit{-430} ~ \ \ \\ 
	 ~ Dairy products & -90 ~ \ \ & -125 ~ \ \ & \textit{-222} ~ \ \ \\ 
	 ~ Fish & -13 ~ \ \ & -19 ~ \ \ & \textit{-112} ~ \ \ \\ 
	\multicolumn{4}{l}{\textcolor{FAOblue}{\textbf{\large{Environment}}}} \\ 
	 ~ Forest area (\%) & 0 ~ \ \ & 0 ~ \ \ & \textit{0} ~ \ \ \\ 
	 ~ Renewable water res withdrawn (\% of total) &  ~ \ \ & 54 ~ \ \ & 54 ~ \ \ \\ 
	 ~ Terrestrial protect areas (\% total land area)  & 2 ~ \ \ & 2 ~ \ \ & \textit{18} ~ \ \ \\ 
	 ~ Organic area (\% total agricultural area) &  ~ \ \ &  ~ \ \ &  ~ \ \ \\ 
	 ~ Water withdrawal by agriculture (\% of total) &  ~ \ \ & 54 ~ \ \ & 54 ~ \ \ \\ 
	 ~ Biofuel production (thousand kt of oil eq.) &  ~ \ \ &  ~ \ \ &  ~ \ \ \\ 
	 ~ Wood pellet prod. (min tonnes) &  ~ \ \ &  ~ \ \ &  ~ \ \ \\ 
	 ~ GHG emissions from ag (Co2 eq, gigagrams) & 0 ~ \ \ & 0 ~ \ \ & \textit{0} ~ \ \ \\ 
       \toprule
      \end{tabular}
      \clearpage
\CountryData{ Kyrgyzstan }
      \rowcolors{1}{FAOblue!10}{white}
      \begin{tabular}{L{3.9cm} R{1cm} R{1cm} R{1cm}}
      \toprule
      \multicolumn{1}{c}{} & \multicolumn{1}{c}{ 1992 } & \multicolumn{1}{c}{ 2002 } & \multicolumn{1}{c}{ 2014 } \\
      \midrule
	\multicolumn{4}{l}{\textcolor{FAOblue}{\textbf{\large{The setting}}}} \\ 
	 ~ Population, total (mln) & 4.5 ~ \ \ & 5 ~ \ \ & 5.6 ~ \ \ \\ 
	 ~ Population, rural (\% total population) & 2.8 ~ \ \ & 3.2 ~ \ \ & 3.6 ~ \ \ \\ 
	 ~ Govt expenditure on ag (\% total outlays) &  ~ \ \ & 4.8 ~ \ \ & \textit{1.5} ~ \ \ \\ 
	 ~ Area harvested (mln ha) & 2 ~ \ \ & 2 ~ \ \ & 2 ~ \ \ \\ 
	 ~ Cropping intensity ratio (\%) & 0.2 ~ \ \ & 0.2 ~ \ \ &  ~ \ \ \\ 
	 ~ Water resources (m\textsuperscript{3}/person/year) & \textit{5} ~ \ \ & \textit{5} ~ \ \ & \textit{4} ~ \ \ \\ 
	 ~ Area equipped for irrigation (1000 ha) &  ~ \ \ &  ~ \ \ & \textit{1\,023} ~ \ \ \\ 
	 ~ Area irrigated (\%) &  ~ \ \ &  ~ \ \ & \textit{99.8} ~ \ \ \\ 
	 ~ Employment in agriculture (\%) & 38.2 ~ \ \ & 49.1 ~ \ \ & \textit{34} ~ \ \ \\ 
	 ~ Employment in agriculture, female (\%) &  ~ \ \ & 48.7 ~ \ \ & \textit{35.4} ~ \ \ \\ 
	 ~ Fertilizers, Nitrogen (nutrients per ha) &  ~ \ \ & 0.9 ~ \ \ & \textit{2.4} ~ \ \ \\ 
	 ~ Fertilizers, Phosphate (nutrients per ha) &  ~ \ \ & 0 ~ \ \ & \textit{0.1} ~ \ \ \\ 
	 ~ Fertilizers, Potash (nutrients per ha) &  ~ \ \ & 0 ~ \ \ & \textit{0.1} ~ \ \ \\ 
	 ~ Energy consump, power irrigation (mln kWh) & \textit{89} ~ \ \ & 89 ~ \ \ & \textit{1} ~ \ \ \\ 
	 ~ Agr value added per worker (constant US\$) & 0.9 ~ \ \ & 1.3 ~ \ \ & \textit{1.4} ~ \ \ \\ 
	\multicolumn{4}{l}{\textcolor{FAOblue}{\textbf{\large{Hunger dimensions}}}} \\ 
	 ~ Dietary energy supply (kcal/pc/day) & 2\,405 ~ \ \ & 2\,518 ~ \ \ & 2\,855 ~ \ \ \\ 
	 ~ Average dietary energy supply adequacy (\%) & 107 ~ \ \ & 108 ~ \ \ & 122 ~ \ \ \\ 
	 ~ Dietary en supp, cereals/roots/tubers (\%) & 60 ~ \ \ & 60 ~ \ \ & \textit{54} ~ \ \ \\ 
	 ~ Prevalence of undernourishment (\%) & 15.9 ~ \ \ & 16.5 ~ \ \ & 6 ~ \ \ \\ 
	 ~ GDP per capita (US\$, PPP) & 2\,681 ~ \ \ & 2\,144 ~ \ \ & \textit{3\,110} ~ \ \ \\ 
	 ~ Domestic food price volatility (index) &  ~ \ \ &  ~ \ \ &  ~ \ \ \\ 
	 ~ Cereal import dependency ratio (\%) & 35 ~ \ \ & 6.9 ~ \ \ & \textit{23.4} ~ \ \ \\ 
	 ~ Underweight, children under-5 (\%) &  ~ \ \ & \textit{8.2} ~ \ \ & 2.8 ~ \ \ \\ 
	 ~ Improved water source (\% pop) & 72.7 ~ \ \ & 80.9 ~ \ \ & \textit{87.6} ~ \ \ \\ 
	\multicolumn{4}{l}{\textcolor{FAOblue}{\textbf{\large{Food Supply}}}} \\ 
	 ~ Food production value, (2004-2006 mln I\$) & 1\,127 ~ \ \ & 1\,386 ~ \ \ & \textit{1\,619} ~ \ \ \\ 
	 ~ Agriculture, value added (\% GDP) & 39 ~ \ \ & 38 ~ \ \ & \textit{18} ~ \ \ \\ 
	 ~ Food exports (mln US\$)  & 21 ~ \ \ & 31 ~ \ \ & \textit{203} ~ \ \ \\ 
	 ~ Food imports (mln US\$)  & 292 ~ \ \ & 53 ~ \ \ & \textit{614} ~ \ \ \\ 
	\multicolumn{4}{l}{\textit{\normalsize{Production indices (2004-06=100)}}} \\ 
	 ~ Net food & 78 ~ \ \ & 96 ~ \ \ & \textit{112} ~ \ \ \\ 
	 ~ Net crop & 65 ~ \ \ & 89 ~ \ \ & \textit{112} ~ \ \ \\ 
	 ~ Cereal & 92 ~ \ \ & 109 ~ \ \ & \textit{106} ~ \ \ \\ 
	 ~ Vegetable oils & 17 ~ \ \ & 81 ~ \ \ & \textit{63} ~ \ \ \\ 
	 ~ Roots and tubers & 34 ~ \ \ & 100 ~ \ \ & \textit{109} ~ \ \ \\ 
	 ~ Fruit and vegetables & 62 ~ \ \ & 71 ~ \ \ & \textit{132} ~ \ \ \\ 
	 ~ Sugar & 35 ~ \ \ & 135 ~ \ \ & \textit{51} ~ \ \ \\ 
	 ~ Livestock & 109 ~ \ \ & 103 ~ \ \ & \textit{113} ~ \ \ \\ 
	 ~ Milk & 80 ~ \ \ & 98 ~ \ \ & \textit{118} ~ \ \ \\ 
	 ~ Meat & 120 ~ \ \ & 108 ~ \ \ & \textit{107} ~ \ \ \\ 
	 ~ Fish  & 1\,945 ~ \ \ & 565 ~ \ \ & \textit{755} ~ \ \ \\ 
	\multicolumn{4}{l}{\textit{\normalsize{Net trade (min US\$)}}} \\ 
	 ~ Cereals & \textit{-26} ~ \ \ & -20 ~ \ \ & \textit{-185} ~ \ \ \\ 
	 ~ Fruit and vegetables & -2 ~ \ \ & 13 ~ \ \ & \textit{106} ~ \ \ \\ 
	 ~ Meat & \textit{4} ~ \ \ & -1 ~ \ \ & \textit{-73} ~ \ \ \\ 
	 ~ Dairy products & \textit{-5} ~ \ \ & 4 ~ \ \ & \textit{9} ~ \ \ \\ 
	 ~ Fish & \textit{-1} ~ \ \ & -1 ~ \ \ & \textit{-15} ~ \ \ \\ 
	\multicolumn{4}{l}{\textcolor{FAOblue}{\textbf{\large{Environment}}}} \\ 
	 ~ Forest area (\%) & 4 ~ \ \ & 4 ~ \ \ & \textit{5} ~ \ \ \\ 
	 ~ Renewable water res withdrawn (\% of total) &  ~ \ \ &  ~ \ \ & 93 ~ \ \ \\ 
	 ~ Terrestrial protect areas (\% total land area)  & 6 ~ \ \ & 7 ~ \ \ & \textit{6} ~ \ \ \\ 
	 ~ Organic area (\% total agricultural area) &  ~ \ \ & \textit{0} ~ \ \ & \textit{0} ~ \ \ \\ 
	 ~ Water withdrawal by agriculture (\% of total) &  ~ \ \ &  ~ \ \ & 93 ~ \ \ \\ 
	 ~ Biofuel production (thousand kt of oil eq.) &  ~ \ \ &  ~ \ \ &  ~ \ \ \\ 
	 ~ Wood pellet prod. (min tonnes) &  ~ \ \ &  ~ \ \ &  ~ \ \ \\ 
	 ~ GHG emissions from ag (Co2 eq, gigagrams) & 5 ~ \ \ & 1 ~ \ \ & \textit{-11} ~ \ \ \\ 
       \toprule
      \end{tabular}
      \clearpage
\CountryData{ Laos }
      \rowcolors{1}{FAOblue!10}{white}
      \begin{tabular}{L{3.9cm} R{1cm} R{1cm} R{1cm}}
      \toprule
      \multicolumn{1}{c}{} & \multicolumn{1}{c}{ 1992 } & \multicolumn{1}{c}{ 2002 } & \multicolumn{1}{c}{ 2014 } \\
      \midrule
	\multicolumn{4}{l}{\textcolor{FAOblue}{\textbf{\large{The setting}}}} \\ 
	 ~ Population, total (mln) & 4.5 ~ \ \ & 5.5 ~ \ \ & 6.9 ~ \ \ \\ 
	 ~ Population, rural (\% total population) & 3.8 ~ \ \ & 4.2 ~ \ \ & 4.3 ~ \ \ \\ 
	 ~ Govt expenditure on ag (\% total outlays) &  ~ \ \ &  ~ \ \ &  ~ \ \ \\ 
	 ~ Area harvested (mln ha) & 2 ~ \ \ & 3 ~ \ \ & 4 ~ \ \ \\ 
	 ~ Cropping intensity ratio (\%) & 0.9 ~ \ \ & 1.3 ~ \ \ &  ~ \ \ \\ 
	 ~ Water resources (m\textsuperscript{3}/person/year) & \textit{72} ~ \ \ & \textit{59} ~ \ \ & \textit{49} ~ \ \ \\ 
	 ~ Area equipped for irrigation (1000 ha) &  ~ \ \ &  ~ \ \ & \textit{310} ~ \ \ \\ 
	 ~ Area irrigated (\%) &  ~ \ \ & \textit{87.3} ~ \ \ & \textit{87.3} ~ \ \ \\ 
	 ~ Employment in agriculture (\%) & \textit{85.4} ~ \ \ & \textit{85.4} ~ \ \ &  ~ \ \ \\ 
	 ~ Employment in agriculture, female (\%) & \textit{89.3} ~ \ \ & \textit{89.3} ~ \ \ &  ~ \ \ \\ 
	 ~ Fertilizers, Nitrogen (nutrients per ha) &  ~ \ \ &  ~ \ \ &  ~ \ \ \\ 
	 ~ Fertilizers, Phosphate (nutrients per ha) &  ~ \ \ &  ~ \ \ &  ~ \ \ \\ 
	 ~ Fertilizers, Potash (nutrients per ha) &  ~ \ \ &  ~ \ \ &  ~ \ \ \\ 
	 ~ Energy consump, power irrigation (mln kWh) & \textit{0} ~ \ \ & 0 ~ \ \ & \textit{0} ~ \ \ \\ 
	 ~ Agr value added per worker (constant US\$) & 0.3 ~ \ \ & 0.4 ~ \ \ & \textit{0.5} ~ \ \ \\ 
	\multicolumn{4}{l}{\textcolor{FAOblue}{\textbf{\large{Hunger dimensions}}}} \\ 
	 ~ Dietary energy supply (kcal/pc/day) & 1\,968 ~ \ \ & 2\,137 ~ \ \ & 2\,399 ~ \ \ \\ 
	 ~ Average dietary energy supply adequacy (\%) & 90 ~ \ \ & 96 ~ \ \ & 104 ~ \ \ \\ 
	 ~ Dietary en supp, cereals/roots/tubers (\%) & 84 ~ \ \ & 76 ~ \ \ & \textit{72} ~ \ \ \\ 
	 ~ Prevalence of undernourishment (\%) & 43.6 ~ \ \ & 36.7 ~ \ \ & 18.9 ~ \ \ \\ 
	 ~ GDP per capita (US\$, PPP) & 1\,686 ~ \ \ & 2\,533 ~ \ \ & \textit{4\,667} ~ \ \ \\ 
	 ~ Domestic food price volatility (index) &  ~ \ \ & 11.7 ~ \ \ & \textit{3.6} ~ \ \ \\ 
	 ~ Cereal import dependency ratio (\%) & 1.6 ~ \ \ & 2.6 ~ \ \ & \textit{-5.1} ~ \ \ \\ 
	 ~ Underweight, children under-5 (\%) & \textit{35.9} ~ \ \ & \textit{36.4} ~ \ \ & \textit{26.5} ~ \ \ \\ 
	 ~ Improved water source (\% pop) & \textit{39.9} ~ \ \ & 50.1 ~ \ \ & \textit{71.5} ~ \ \ \\ 
	\multicolumn{4}{l}{\textcolor{FAOblue}{\textbf{\large{Food Supply}}}} \\ 
	 ~ Food production value, (2004-2006 mln I\$) & 599 ~ \ \ & 1\,099 ~ \ \ & \textit{1\,807} ~ \ \ \\ 
	 ~ Agriculture, value added (\% GDP) & 62 ~ \ \ & 43 ~ \ \ & \textit{27} ~ \ \ \\ 
	 ~ Food exports (mln US\$)  & 24 ~ \ \ & 6 ~ \ \ & \textit{39} ~ \ \ \\ 
	 ~ Food imports (mln US\$)  & 19 ~ \ \ & 45 ~ \ \ & \textit{199} ~ \ \ \\ 
	\multicolumn{4}{l}{\textit{\normalsize{Production indices (2004-06=100)}}} \\ 
	 ~ Net food & 51 ~ \ \ & 93 ~ \ \ & \textit{153} ~ \ \ \\ 
	 ~ Net crop & 52 ~ \ \ & 94 ~ \ \ & \textit{163} ~ \ \ \\ 
	 ~ Cereal & 55 ~ \ \ & 90 ~ \ \ & \textit{146} ~ \ \ \\ 
	 ~ Vegetable oils & 56 ~ \ \ & 65 ~ \ \ & \textit{198} ~ \ \ \\ 
	 ~ Roots and tubers & 82 ~ \ \ & 74 ~ \ \ & \textit{524} ~ \ \ \\ 
	 ~ Fruit and vegetables & 25 ~ \ \ & 107 ~ \ \ & \textit{153} ~ \ \ \\ 
	 ~ Sugar & 44 ~ \ \ & 104 ~ \ \ & \textit{555} ~ \ \ \\ 
	 ~ Livestock & 56 ~ \ \ & 88 ~ \ \ & \textit{126} ~ \ \ \\ 
	 ~ Milk & 85 ~ \ \ & 93 ~ \ \ & \textit{112} ~ \ \ \\ 
	 ~ Meat & 57 ~ \ \ & 88 ~ \ \ & \textit{126} ~ \ \ \\ 
	 ~ Fish  & 34 ~ \ \ & 104 ~ \ \ & \textit{159} ~ \ \ \\ 
	\multicolumn{4}{l}{\textit{\normalsize{Net trade (min US\$)}}} \\ 
	 ~ Cereals & -7 ~ \ \ & -18 ~ \ \ & \textit{-21} ~ \ \ \\ 
	 ~ Fruit and vegetables & 0 ~ \ \ & -1 ~ \ \ & \textit{-2} ~ \ \ \\ 
	 ~ Meat & 0 ~ \ \ & 0 ~ \ \ & \textit{0} ~ \ \ \\ 
	 ~ Dairy products &  ~ \ \ &  ~ \ \ &  ~ \ \ \\ 
	 ~ Fish & \textit{-2} ~ \ \ & -2 ~ \ \ & \textit{-8} ~ \ \ \\ 
	\multicolumn{4}{l}{\textcolor{FAOblue}{\textbf{\large{Environment}}}} \\ 
	 ~ Forest area (\%) & 74 ~ \ \ & 71 ~ \ \ & \textit{68} ~ \ \ \\ 
	 ~ Renewable water res withdrawn (\% of total) &  ~ \ \ & \textit{91} ~ \ \ & 91 ~ \ \ \\ 
	 ~ Terrestrial protect areas (\% total land area)  & 1 ~ \ \ & 17 ~ \ \ & \textit{17} ~ \ \ \\ 
	 ~ Organic area (\% total agricultural area) &  ~ \ \ &  ~ \ \ & \textit{0} ~ \ \ \\ 
	 ~ Water withdrawal by agriculture (\% of total) &  ~ \ \ & \textit{91} ~ \ \ & 91 ~ \ \ \\ 
	 ~ Biofuel production (thousand kt of oil eq.) &  ~ \ \ &  ~ \ \ &  ~ \ \ \\ 
	 ~ Wood pellet prod. (min tonnes) &  ~ \ \ &  ~ \ \ &  ~ \ \ \\ 
	 ~ GHG emissions from ag (Co2 eq, gigagrams) & 27 ~ \ \ & 27 ~ \ \ & \textit{33} ~ \ \ \\ 
       \toprule
      \end{tabular}
      \clearpage
\CountryData{ Latvia }
      \rowcolors{1}{FAOblue!10}{white}
      \begin{tabular}{L{3.9cm} R{1cm} R{1cm} R{1cm}}
      \toprule
      \multicolumn{1}{c}{} & \multicolumn{1}{c}{ 1992 } & \multicolumn{1}{c}{ 2002 } & \multicolumn{1}{c}{ 2014 } \\
      \midrule
	\multicolumn{4}{l}{\textcolor{FAOblue}{\textbf{\large{The setting}}}} \\ 
	 ~ Population, total (mln) & 2.6 ~ \ \ & 2.3 ~ \ \ & 2 ~ \ \ \\ 
	 ~ Population, rural (\% total population) & 0.8 ~ \ \ & 0.7 ~ \ \ & 0.7 ~ \ \ \\ 
	 ~ Govt expenditure on ag (\% total outlays) &  ~ \ \ & 8.3 ~ \ \ & \textit{12.5} ~ \ \ \\ 
	 ~ Area harvested (mln ha) & 1 ~ \ \ & 1 ~ \ \ & 2 ~ \ \ \\ 
	 ~ Cropping intensity ratio (\%) & 0.5 ~ \ \ & 0.6 ~ \ \ &  ~ \ \ \\ 
	 ~ Water resources (m\textsuperscript{3}/person/year) & \textit{14} ~ \ \ & \textit{15} ~ \ \ & \textit{17} ~ \ \ \\ 
	 ~ Area equipped for irrigation (1000 ha) &  ~ \ \ &  ~ \ \ & \textit{1} ~ \ \ \\ 
	 ~ Area irrigated (\%) &  ~ \ \ &  ~ \ \ & \textit{74.7} ~ \ \ \\ 
	 ~ Employment in agriculture (\%) &  ~ \ \ & 15.4 ~ \ \ & \textit{8.4} ~ \ \ \\ 
	 ~ Employment in agriculture, female (\%) &  ~ \ \ & 11.3 ~ \ \ & \textit{4.9} ~ \ \ \\ 
	 ~ Fertilizers, Nitrogen (nutrients per ha) &  ~ \ \ & 22.6 ~ \ \ & \textit{35.7} ~ \ \ \\ 
	 ~ Fertilizers, Phosphate (nutrients per ha) &  ~ \ \ & 2.9 ~ \ \ & \textit{10.7} ~ \ \ \\ 
	 ~ Fertilizers, Potash (nutrients per ha) &  ~ \ \ & 5.4 ~ \ \ & \textit{11.2} ~ \ \ \\ 
	 ~ Energy consump, power irrigation (mln kWh) & \textit{0} ~ \ \ & 0 ~ \ \ & \textit{0} ~ \ \ \\ 
	 ~ Agr value added per worker (constant US\$) & 2.7 ~ \ \ & 3.9 ~ \ \ & \textit{5.5} ~ \ \ \\ 
	\multicolumn{4}{l}{\textcolor{FAOblue}{\textbf{\large{Hunger dimensions}}}} \\ 
	 ~ Dietary energy supply (kcal/pc/day) &  ~ \ \ &  ~ \ \ &  ~ \ \ \\ 
	 ~ Average dietary energy supply adequacy (\%) & 135 ~ \ \ & 119 ~ \ \ & 134 ~ \ \ \\ 
	 ~ Dietary en supp, cereals/roots/tubers (\%) & 47 ~ \ \ & 38 ~ \ \ & \textit{33} ~ \ \ \\ 
	 ~ Prevalence of undernourishment (\%) & <5.0 ~ \ \ & <5.0 ~ \ \ & <5.0 ~ \ \ \\ 
	 ~ GDP per capita (US\$, PPP) & 8\,045 ~ \ \ & 13\,270 ~ \ \ & \textit{21\,833} ~ \ \ \\ 
	 ~ Domestic food price volatility (index) &  ~ \ \ & 8.4 ~ \ \ & 7.9 ~ \ \ \\ 
	 ~ Cereal import dependency ratio (\%) & -2.6 ~ \ \ & -5.8 ~ \ \ & \textit{-72.2} ~ \ \ \\ 
	 ~ Underweight, children under-5 (\%) &  ~ \ \ &  ~ \ \ &  ~ \ \ \\ 
	 ~ Improved water source (\% pop) & 98.4 ~ \ \ & 98.4 ~ \ \ & \textit{98.4} ~ \ \ \\ 
	\multicolumn{4}{l}{\textcolor{FAOblue}{\textbf{\large{Food Supply}}}} \\ 
	 ~ Food production value, (2004-2006 mln I\$) & 1\,266 ~ \ \ & 603 ~ \ \ & \textit{797} ~ \ \ \\ 
	 ~ Agriculture, value added (\% GDP) & 18 ~ \ \ & 5 ~ \ \ & \textit{4} ~ \ \ \\ 
	 ~ Food exports (mln US\$)  & 32 ~ \ \ & 129 ~ \ \ & \textit{1\,548} ~ \ \ \\ 
	 ~ Food imports (mln US\$)  & 38 ~ \ \ & 534 ~ \ \ & \textit{1\,474} ~ \ \ \\ 
	\multicolumn{4}{l}{\textit{\normalsize{Production indices (2004-06=100)}}} \\ 
	 ~ Net food & 192 ~ \ \ & 91 ~ \ \ & \textit{121} ~ \ \ \\ 
	 ~ Net crop & 106 ~ \ \ & 96 ~ \ \ & \textit{125} ~ \ \ \\ 
	 ~ Cereal & 85 ~ \ \ & 88 ~ \ \ & \textit{183} ~ \ \ \\ 
	 ~ Vegetable oils & 3 ~ \ \ & 28 ~ \ \ & \textit{239} ~ \ \ \\ 
	 ~ Roots and tubers & 175 ~ \ \ & 131 ~ \ \ & \textit{41} ~ \ \ \\ 
	 ~ Fruit and vegetables & 131 ~ \ \ & 103 ~ \ \ & \textit{62} ~ \ \ \\ 
	 ~ Sugar & 93 ~ \ \ & 125 ~ \ \ & \textit{2} ~ \ \ \\ 
	 ~ Livestock & 247 ~ \ \ & 91 ~ \ \ & \textit{117} ~ \ \ \\ 
	 ~ Milk & 184 ~ \ \ & 101 ~ \ \ & \textit{114} ~ \ \ \\ 
	 ~ Meat & 394 ~ \ \ & 71 ~ \ \ & \textit{123} ~ \ \ \\ 
	 ~ Fish  & 113 ~ \ \ & 85 ~ \ \ & \textit{84} ~ \ \ \\ 
	\multicolumn{4}{l}{\textit{\normalsize{Net trade (min US\$)}}} \\ 
	 ~ Cereals & -14 ~ \ \ & -21 ~ \ \ & \textit{369} ~ \ \ \\ 
	 ~ Fruit and vegetables & -1 ~ \ \ & -138 ~ \ \ & \textit{-178} ~ \ \ \\ 
	 ~ Meat & 0 ~ \ \ & -74 ~ \ \ & \textit{-96} ~ \ \ \\ 
	 ~ Dairy products & 12 ~ \ \ & 2 ~ \ \ & \textit{102} ~ \ \ \\ 
	 ~ Fish & 10 ~ \ \ & 75 ~ \ \ & \textit{53} ~ \ \ \\ 
	\multicolumn{4}{l}{\textcolor{FAOblue}{\textbf{\large{Environment}}}} \\ 
	 ~ Forest area (\%) & 51 ~ \ \ & 52 ~ \ \ & \textit{54} ~ \ \ \\ 
	 ~ Renewable water res withdrawn (\% of total) &  ~ \ \ & 13 ~ \ \ & 13 ~ \ \ \\ 
	 ~ Terrestrial protect areas (\% total land area)  & 6 ~ \ \ & 15 ~ \ \ & \textit{19} ~ \ \ \\ 
	 ~ Organic area (\% total agricultural area) &  ~ \ \ & \textit{7} ~ \ \ & \textit{11} ~ \ \ \\ 
	 ~ Water withdrawal by agriculture (\% of total) &  ~ \ \ & 13 ~ \ \ & 13 ~ \ \ \\ 
	 ~ Biofuel production (thousand kt of oil eq.) & \textit{3} ~ \ \ & 15 ~ \ \ & \textit{1\,195} ~ \ \ \\ 
	 ~ Wood pellet prod. (min tonnes) &  ~ \ \ &  ~ \ \ & \textit{1\,093} ~ \ \ \\ 
	 ~ GHG emissions from ag (Co2 eq, gigagrams) & 10 ~ \ \ & 1 ~ \ \ & \textit{-12} ~ \ \ \\ 
       \toprule
      \end{tabular}
      \clearpage
\CountryData{ Lebanon }
      \rowcolors{1}{FAOblue!10}{white}
      \begin{tabular}{L{3.9cm} R{1cm} R{1cm} R{1cm}}
      \toprule
      \multicolumn{1}{c}{} & \multicolumn{1}{c}{ 1992 } & \multicolumn{1}{c}{ 2002 } & \multicolumn{1}{c}{ 2014 } \\
      \midrule
	\multicolumn{4}{l}{\textcolor{FAOblue}{\textbf{\large{The setting}}}} \\ 
	 ~ Population, total (mln) & 2.8 ~ \ \ & 3.5 ~ \ \ & 5 ~ \ \ \\ 
	 ~ Population, rural (\% total population) & 0.5 ~ \ \ & 0.5 ~ \ \ & 0.6 ~ \ \ \\ 
	 ~ Govt expenditure on ag (\% total outlays) &  ~ \ \ &  ~ \ \ &  ~ \ \ \\ 
	 ~ Area harvested (mln ha) & 1 ~ \ \ & 1 ~ \ \ & 0 ~ \ \ \\ 
	 ~ Cropping intensity ratio (\%) & 2.2 ~ \ \ & 1.4 ~ \ \ &  ~ \ \ \\ 
	 ~ Water resources (m\textsuperscript{3}/person/year) & \textit{2} ~ \ \ & \textit{1} ~ \ \ & \textit{1} ~ \ \ \\ 
	 ~ Area equipped for irrigation (1000 ha) &  ~ \ \ &  ~ \ \ & \textit{104} ~ \ \ \\ 
	 ~ Area irrigated (\%) &  ~ \ \ & \textit{86.5} ~ \ \ &  ~ \ \ \\ 
	 ~ Employment in agriculture (\%) &  ~ \ \ &  ~ \ \ & \textit{6.3} ~ \ \ \\ 
	 ~ Employment in agriculture, female (\%) &  ~ \ \ &  ~ \ \ & \textit{5.7} ~ \ \ \\ 
	 ~ Fertilizers, Nitrogen (nutrients per ha) &  ~ \ \ & 31.9 ~ \ \ & \textit{20.9} ~ \ \ \\ 
	 ~ Fertilizers, Phosphate (nutrients per ha) &  ~ \ \ & 30.5 ~ \ \ & \textit{51.5} ~ \ \ \\ 
	 ~ Fertilizers, Potash (nutrients per ha) &  ~ \ \ & 13.2 ~ \ \ & \textit{7.5} ~ \ \ \\ 
	 ~ Energy consump, power irrigation (mln kWh) & \textit{82} ~ \ \ & 91 ~ \ \ & \textit{91} ~ \ \ \\ 
	 ~ Agr value added per worker (constant US\$) & \textit{11.3} ~ \ \ & 17.2 ~ \ \ & \textit{39.6} ~ \ \ \\ 
	\multicolumn{4}{l}{\textcolor{FAOblue}{\textbf{\large{Hunger dimensions}}}} \\ 
	 ~ Dietary energy supply (kcal/pc/day) & 3\,260 ~ \ \ & 3\,277 ~ \ \ & 3\,213 ~ \ \ \\ 
	 ~ Average dietary energy supply adequacy (\%) & 141 ~ \ \ & 138 ~ \ \ & 130 ~ \ \ \\ 
	 ~ Dietary en supp, cereals/roots/tubers (\%) & 37 ~ \ \ & 38 ~ \ \ & \textit{39} ~ \ \ \\ 
	 ~ Prevalence of undernourishment (\%) & <5.0 ~ \ \ & <5.0 ~ \ \ & <5.0 ~ \ \ \\ 
	 ~ GDP per capita (US\$, PPP) & 10\,700 ~ \ \ & 12\,199 ~ \ \ & \textit{16\,623} ~ \ \ \\ 
	 ~ Domestic food price volatility (index) &  ~ \ \ &  ~ \ \ &  ~ \ \ \\ 
	 ~ Cereal import dependency ratio (\%) & 89.5 ~ \ \ & 85.9 ~ \ \ & \textit{88.3} ~ \ \ \\ 
	 ~ Underweight, children under-5 (\%) &  ~ \ \ & \textit{4.2} ~ \ \ &  ~ \ \ \\ 
	 ~ Improved water source (\% pop) & 100 ~ \ \ & 100 ~ \ \ & \textit{100} ~ \ \ \\ 
	\multicolumn{4}{l}{\textcolor{FAOblue}{\textbf{\large{Food Supply}}}} \\ 
	 ~ Food production value, (2004-2006 mln I\$) & 1\,227 ~ \ \ & 1\,193 ~ \ \ & \textit{1\,163} ~ \ \ \\ 
	 ~ Agriculture, value added (\% GDP) & \textit{8} ~ \ \ & 7 ~ \ \ & \textit{7} ~ \ \ \\ 
	 ~ Food exports (mln US\$)  & 106 ~ \ \ & 130 ~ \ \ & \textit{478} ~ \ \ \\ 
	 ~ Food imports (mln US\$)  & 687 ~ \ \ & 928 ~ \ \ & \textit{2\,348} ~ \ \ \\ 
	\multicolumn{4}{l}{\textit{\normalsize{Production indices (2004-06=100)}}} \\ 
	 ~ Net food & 101 ~ \ \ & 99 ~ \ \ & \textit{96} ~ \ \ \\ 
	 ~ Net crop & 117 ~ \ \ & 100 ~ \ \ & \textit{99} ~ \ \ \\ 
	 ~ Cereal & 49 ~ \ \ & 79 ~ \ \ & \textit{101} ~ \ \ \\ 
	 ~ Vegetable oils & 76 ~ \ \ & 130 ~ \ \ & \textit{72} ~ \ \ \\ 
	 ~ Roots and tubers & 59 ~ \ \ & 84 ~ \ \ & \textit{88} ~ \ \ \\ 
	 ~ Fruit and vegetables & 148 ~ \ \ & 100 ~ \ \ & \textit{107} ~ \ \ \\ 
	 ~ Sugar & 315 ~ \ \ & \textit{119} ~ \ \ & \textit{2} ~ \ \ \\ 
	 ~ Livestock & 58 ~ \ \ & 95 ~ \ \ & \textit{93} ~ \ \ \\ 
	 ~ Milk & 78 ~ \ \ & 104 ~ \ \ & \textit{161} ~ \ \ \\ 
	 ~ Meat & 49 ~ \ \ & 91 ~ \ \ & \textit{75} ~ \ \ \\ 
	 ~ Fish  & 40 ~ \ \ & 103 ~ \ \ & \textit{110} ~ \ \ \\ 
	\multicolumn{4}{l}{\textit{\normalsize{Net trade (min US\$)}}} \\ 
	 ~ Cereals & -117 ~ \ \ & -166 ~ \ \ & \textit{-521} ~ \ \ \\ 
	 ~ Fruit and vegetables & -98 ~ \ \ & -82 ~ \ \ & \textit{-92} ~ \ \ \\ 
	 ~ Meat & -50 ~ \ \ & -55 ~ \ \ & \textit{-184} ~ \ \ \\ 
	 ~ Dairy products & -82 ~ \ \ & -152 ~ \ \ & \textit{-287} ~ \ \ \\ 
	 ~ Fish & -8 ~ \ \ & -46 ~ \ \ & \textit{-132} ~ \ \ \\ 
	\multicolumn{4}{l}{\textcolor{FAOblue}{\textbf{\large{Environment}}}} \\ 
	 ~ Forest area (\%) & 13 ~ \ \ & 13 ~ \ \ & \textit{13} ~ \ \ \\ 
	 ~ Renewable water res withdrawn (\% of total) &  ~ \ \ & \textit{60} ~ \ \ & 60 ~ \ \ \\ 
	 ~ Terrestrial protect areas (\% total land area)  & 0 ~ \ \ & 0 ~ \ \ & \textit{1} ~ \ \ \\ 
	 ~ Organic area (\% total agricultural area) &  ~ \ \ & \textit{0} ~ \ \ & \textit{0} ~ \ \ \\ 
	 ~ Water withdrawal by agriculture (\% of total) &  ~ \ \ & \textit{60} ~ \ \ & 60 ~ \ \ \\ 
	 ~ Biofuel production (thousand kt of oil eq.) &  ~ \ \ &  ~ \ \ &  ~ \ \ \\ 
	 ~ Wood pellet prod. (min tonnes) &  ~ \ \ &  ~ \ \ &  ~ \ \ \\ 
	 ~ GHG emissions from ag (Co2 eq, gigagrams) & 0 ~ \ \ & 1 ~ \ \ & \textit{1} ~ \ \ \\ 
       \toprule
      \end{tabular}
      \clearpage
\CountryData{ Lesotho }
      \rowcolors{1}{FAOblue!10}{white}
      \begin{tabular}{L{3.9cm} R{1cm} R{1cm} R{1cm}}
      \toprule
      \multicolumn{1}{c}{} & \multicolumn{1}{c}{ 1992 } & \multicolumn{1}{c}{ 2002 } & \multicolumn{1}{c}{ 2014 } \\
      \midrule
	\multicolumn{4}{l}{\textcolor{FAOblue}{\textbf{\large{The setting}}}} \\ 
	 ~ Population, total (mln) & 1.7 ~ \ \ & 1.9 ~ \ \ & 2.1 ~ \ \ \\ 
	 ~ Population, rural (\% total population) & 1.4 ~ \ \ & 1.5 ~ \ \ & 1.5 ~ \ \ \\ 
	 ~ Govt expenditure on ag (\% total outlays) &  ~ \ \ &  ~ \ \ & \textit{2.4} ~ \ \ \\ 
	 ~ Area harvested (mln ha) & 0 ~ \ \ & 0 ~ \ \ & 0 ~ \ \ \\ 
	 ~ Cropping intensity ratio (\%) & 0.1 ~ \ \ & 0.1 ~ \ \ &  ~ \ \ \\ 
	 ~ Water resources (m\textsuperscript{3}/person/year) & \textit{2} ~ \ \ & \textit{2} ~ \ \ & \textit{1} ~ \ \ \\ 
	 ~ Area equipped for irrigation (1000 ha) &  ~ \ \ &  ~ \ \ & \textit{3} ~ \ \ \\ 
	 ~ Area irrigated (\%) &  ~ \ \ & \textit{2.5} ~ \ \ &  ~ \ \ \\ 
	 ~ Employment in agriculture (\%) &  ~ \ \ & \textit{72.3} ~ \ \ &  ~ \ \ \\ 
	 ~ Employment in agriculture, female (\%) &  ~ \ \ & \textit{64.9} ~ \ \ &  ~ \ \ \\ 
	 ~ Fertilizers, Nitrogen (nutrients per ha) &  ~ \ \ &  ~ \ \ &  ~ \ \ \\ 
	 ~ Fertilizers, Phosphate (nutrients per ha) &  ~ \ \ &  ~ \ \ &  ~ \ \ \\ 
	 ~ Fertilizers, Potash (nutrients per ha) &  ~ \ \ &  ~ \ \ &  ~ \ \ \\ 
	 ~ Energy consump, power irrigation (mln kWh) &  ~ \ \ &  ~ \ \ &  ~ \ \ \\ 
	 ~ Agr value added per worker (constant US\$) & 0.4 ~ \ \ & 0.3 ~ \ \ & \textit{0.3} ~ \ \ \\ 
	\multicolumn{4}{l}{\textcolor{FAOblue}{\textbf{\large{Hunger dimensions}}}} \\ 
	 ~ Dietary energy supply (kcal/pc/day) & 2\,381 ~ \ \ & 2\,506 ~ \ \ & 2\,593 ~ \ \ \\ 
	 ~ Average dietary energy supply adequacy (\%) & 112 ~ \ \ & 116 ~ \ \ & 116 ~ \ \ \\ 
	 ~ Dietary en supp, cereals/roots/tubers (\%) & 79 ~ \ \ & 81 ~ \ \ & \textit{80} ~ \ \ \\ 
	 ~ Prevalence of undernourishment (\%) & 15.2 ~ \ \ & 11.9 ~ \ \ & 11.2 ~ \ \ \\ 
	 ~ GDP per capita (US\$, PPP) & 1\,401 ~ \ \ & 1\,688 ~ \ \ & \textit{2\,494} ~ \ \ \\ 
	 ~ Domestic food price volatility (index) &  ~ \ \ & 15.5 ~ \ \ & 6.4 ~ \ \ \\ 
	 ~ Cereal import dependency ratio (\%) & 77.2 ~ \ \ & 60.1 ~ \ \ & \textit{78.2} ~ \ \ \\ 
	 ~ Underweight, children under-5 (\%) & 13.8 ~ \ \ & \textit{16.6} ~ \ \ & \textit{13.5} ~ \ \ \\ 
	 ~ Improved water source (\% pop) & 77.7 ~ \ \ & 79.4 ~ \ \ & \textit{81.3} ~ \ \ \\ 
	\multicolumn{4}{l}{\textcolor{FAOblue}{\textbf{\large{Food Supply}}}} \\ 
	 ~ Food production value, (2004-2006 mln I\$) & 98 ~ \ \ & 117 ~ \ \ & \textit{132} ~ \ \ \\ 
	 ~ Agriculture, value added (\% GDP) & 19 ~ \ \ & 10 ~ \ \ & \textit{8} ~ \ \ \\ 
	 ~ Food exports (mln US\$)  & 4 ~ \ \ & 0 ~ \ \ & \textit{0} ~ \ \ \\ 
	 ~ Food imports (mln US\$)  & 121 ~ \ \ & 123 ~ \ \ & \textit{189} ~ \ \ \\ 
	\multicolumn{4}{l}{\textit{\normalsize{Production indices (2004-06=100)}}} \\ 
	 ~ Net food & 83 ~ \ \ & 99 ~ \ \ & \textit{112} ~ \ \ \\ 
	 ~ Net crop & 73 ~ \ \ & 108 ~ \ \ & \textit{113} ~ \ \ \\ 
	 ~ Cereal & 83 ~ \ \ & 133 ~ \ \ & \textit{93} ~ \ \ \\ 
	 ~ Vegetable oils &  ~ \ \ &  ~ \ \ &  ~ \ \ \\ 
	 ~ Roots and tubers & 56 ~ \ \ & 91 ~ \ \ & \textit{126} ~ \ \ \\ 
	 ~ Fruit and vegetables & 86 ~ \ \ & 85 ~ \ \ & \textit{95} ~ \ \ \\ 
	 ~ Sugar &  ~ \ \ &  ~ \ \ &  ~ \ \ \\ 
	 ~ Livestock & 95 ~ \ \ & 92 ~ \ \ & \textit{111} ~ \ \ \\ 
	 ~ Milk & 92 ~ \ \ & 105 ~ \ \ & \textit{138} ~ \ \ \\ 
	 ~ Meat & 91 ~ \ \ & 92 ~ \ \ & \textit{107} ~ \ \ \\ 
	 ~ Fish  & 71 ~ \ \ & 103 ~ \ \ & \textit{1\,179} ~ \ \ \\ 
	\multicolumn{4}{l}{\textit{\normalsize{Net trade (min US\$)}}} \\ 
	 ~ Cereals & -44 ~ \ \ & -53 ~ \ \ & \textit{-96} ~ \ \ \\ 
	 ~ Fruit and vegetables & -24 ~ \ \ & -26 ~ \ \ & \textit{-33} ~ \ \ \\ 
	 ~ Meat & -10 ~ \ \ & -11 ~ \ \ & \textit{-20} ~ \ \ \\ 
	 ~ Dairy products &  ~ \ \ &  ~ \ \ &  ~ \ \ \\ 
	 ~ Fish &  ~ \ \ & -7 ~ \ \ & \textit{-3} ~ \ \ \\ 
	\multicolumn{4}{l}{\textcolor{FAOblue}{\textbf{\large{Environment}}}} \\ 
	 ~ Forest area (\%) & 1 ~ \ \ & 1 ~ \ \ & \textit{1} ~ \ \ \\ 
	 ~ Renewable water res withdrawn (\% of total) &  ~ \ \ & \textit{9} ~ \ \ & 9 ~ \ \ \\ 
	 ~ Terrestrial protect areas (\% total land area)  & 0 ~ \ \ & 0 ~ \ \ & \textit{1} ~ \ \ \\ 
	 ~ Organic area (\% total agricultural area) &  ~ \ \ &  ~ \ \ & \textit{0} ~ \ \ \\ 
	 ~ Water withdrawal by agriculture (\% of total) &  ~ \ \ & \textit{9} ~ \ \ & 9 ~ \ \ \\ 
	 ~ Biofuel production (thousand kt of oil eq.) &  ~ \ \ &  ~ \ \ &  ~ \ \ \\ 
	 ~ Wood pellet prod. (min tonnes) &  ~ \ \ &  ~ \ \ &  ~ \ \ \\ 
	 ~ GHG emissions from ag (Co2 eq, gigagrams) & 1 ~ \ \ & 1 ~ \ \ & \textit{1} ~ \ \ \\ 
       \toprule
      \end{tabular}
      \clearpage
\CountryData{ Liberia }
      \rowcolors{1}{FAOblue!10}{white}
      \begin{tabular}{L{3.9cm} R{1cm} R{1cm} R{1cm}}
      \toprule
      \multicolumn{1}{c}{} & \multicolumn{1}{c}{ 1992 } & \multicolumn{1}{c}{ 2002 } & \multicolumn{1}{c}{ 2014 } \\
      \midrule
	\multicolumn{4}{l}{\textcolor{FAOblue}{\textbf{\large{The setting}}}} \\ 
	 ~ Population, total (mln) & 2 ~ \ \ & 3.1 ~ \ \ & 4.4 ~ \ \ \\ 
	 ~ Population, rural (\% total population) & 0.9 ~ \ \ & 1.7 ~ \ \ & 2.2 ~ \ \ \\ 
	 ~ Govt expenditure on ag (\% total outlays) &  ~ \ \ & \textit{1.3} ~ \ \ & \textit{2.4} ~ \ \ \\ 
	 ~ Area harvested (mln ha) & 0 ~ \ \ & 1 ~ \ \ & 1 ~ \ \ \\ 
	 ~ Cropping intensity ratio (\%) & 0.1 ~ \ \ & 0.2 ~ \ \ &  ~ \ \ \\ 
	 ~ Water resources (m\textsuperscript{3}/person/year) & \textit{116} ~ \ \ & \textit{74} ~ \ \ & \textit{54} ~ \ \ \\ 
	 ~ Area equipped for irrigation (1000 ha) &  ~ \ \ &  ~ \ \ & \textit{3} ~ \ \ \\ 
	 ~ Area irrigated (\%) &  ~ \ \ &  ~ \ \ &  ~ \ \ \\ 
	 ~ Employment in agriculture (\%) &  ~ \ \ &  ~ \ \ & \textit{48.9} ~ \ \ \\ 
	 ~ Employment in agriculture, female (\%) &  ~ \ \ &  ~ \ \ & \textit{48.3} ~ \ \ \\ 
	 ~ Fertilizers, Nitrogen (nutrients per ha) &  ~ \ \ &  ~ \ \ &  ~ \ \ \\ 
	 ~ Fertilizers, Phosphate (nutrients per ha) &  ~ \ \ &  ~ \ \ &  ~ \ \ \\ 
	 ~ Fertilizers, Potash (nutrients per ha) &  ~ \ \ &  ~ \ \ &  ~ \ \ \\ 
	 ~ Energy consump, power irrigation (mln kWh) &  ~ \ \ &  ~ \ \ &  ~ \ \ \\ 
	 ~ Agr value added per worker (constant US\$) &  ~ \ \ & 0.9 ~ \ \ & \textit{0.7} ~ \ \ \\ 
	\multicolumn{4}{l}{\textcolor{FAOblue}{\textbf{\large{Hunger dimensions}}}} \\ 
	 ~ Dietary energy supply (kcal/pc/day) & 2\,256 ~ \ \ & 2\,052 ~ \ \ & 2\,336 ~ \ \ \\ 
	 ~ Average dietary energy supply adequacy (\%) & 106 ~ \ \ & 96 ~ \ \ & 108 ~ \ \ \\ 
	 ~ Dietary en supp, cereals/roots/tubers (\%) & 63 ~ \ \ & 66 ~ \ \ & \textit{66} ~ \ \ \\ 
	 ~ Prevalence of undernourishment (\%) & 27.6 ~ \ \ & 39.6 ~ \ \ & 32.5 ~ \ \ \\ 
	 ~ GDP per capita (US\$, PPP) & 290 ~ \ \ & 718 ~ \ \ & \textit{850} ~ \ \ \\ 
	 ~ Domestic food price volatility (index) &  ~ \ \ &  ~ \ \ &  ~ \ \ \\ 
	 ~ Cereal import dependency ratio (\%) & 73 ~ \ \ & 73.2 ~ \ \ & \textit{61.1} ~ \ \ \\ 
	 ~ Underweight, children under-5 (\%) &  ~ \ \ & \textit{22.8} ~ \ \ & \textit{15.3} ~ \ \ \\ 
	 ~ Improved water source (\% pop) & \textit{58.6} ~ \ \ & 63.4 ~ \ \ & \textit{74.6} ~ \ \ \\ 
	\multicolumn{4}{l}{\textcolor{FAOblue}{\textbf{\large{Food Supply}}}} \\ 
	 ~ Food production value, (2004-2006 mln I\$) & 186 ~ \ \ & 256 ~ \ \ & \textit{323} ~ \ \ \\ 
	 ~ Agriculture, value added (\% GDP) & 51 ~ \ \ & 80 ~ \ \ & \textit{39} ~ \ \ \\ 
	 ~ Food exports (mln US\$)  & 2 ~ \ \ & 1 ~ \ \ & \textit{31} ~ \ \ \\ 
	 ~ Food imports (mln US\$)  & 77 ~ \ \ & 64 ~ \ \ & \textit{308} ~ \ \ \\ 
	\multicolumn{4}{l}{\textit{\normalsize{Production indices (2004-06=100)}}} \\ 
	 ~ Net food & 69 ~ \ \ & 95 ~ \ \ & \textit{120} ~ \ \ \\ 
	 ~ Net crop & 54 ~ \ \ & 98 ~ \ \ & \textit{95} ~ \ \ \\ 
	 ~ Cereal & 79 ~ \ \ & 76 ~ \ \ & \textit{166} ~ \ \ \\ 
	 ~ Vegetable oils & 81 ~ \ \ & 99 ~ \ \ & \textit{105} ~ \ \ \\ 
	 ~ Roots and tubers & 57 ~ \ \ & 99 ~ \ \ & \textit{101} ~ \ \ \\ 
	 ~ Fruit and vegetables & 67 ~ \ \ & 101 ~ \ \ & \textit{104} ~ \ \ \\ 
	 ~ Sugar & 88 ~ \ \ & 100 ~ \ \ & \textit{104} ~ \ \ \\ 
	 ~ Livestock & 80 ~ \ \ & 94 ~ \ \ & \textit{140} ~ \ \ \\ 
	 ~ Milk & 77 ~ \ \ & 85 ~ \ \ & \textit{91} ~ \ \ \\ 
	 ~ Meat & 79 ~ \ \ & 95 ~ \ \ & \textit{144} ~ \ \ \\ 
	 ~ Fish  & 77 ~ \ \ & 95 ~ \ \ & \textit{82} ~ \ \ \\ 
	\multicolumn{4}{l}{\textit{\normalsize{Net trade (min US\$)}}} \\ 
	 ~ Cereals & -54 ~ \ \ & -41 ~ \ \ & \textit{-167} ~ \ \ \\ 
	 ~ Fruit and vegetables & -4 ~ \ \ & -4 ~ \ \ & \textit{-11} ~ \ \ \\ 
	 ~ Meat & -6 ~ \ \ & -4 ~ \ \ & \textit{-33} ~ \ \ \\ 
	 ~ Dairy products & -2 ~ \ \ & -2 ~ \ \ & \textit{-8} ~ \ \ \\ 
	 ~ Fish & -2 ~ \ \ & -1 ~ \ \ & \textit{-10} ~ \ \ \\ 
	\multicolumn{4}{l}{\textcolor{FAOblue}{\textbf{\large{Environment}}}} \\ 
	 ~ Forest area (\%) & 51 ~ \ \ & 47 ~ \ \ & \textit{44} ~ \ \ \\ 
	 ~ Renewable water res withdrawn (\% of total) &  ~ \ \ & \textit{9} ~ \ \ & 9 ~ \ \ \\ 
	 ~ Terrestrial protect areas (\% total land area)  & 2 ~ \ \ & 2 ~ \ \ & \textit{3} ~ \ \ \\ 
	 ~ Organic area (\% total agricultural area) &  ~ \ \ &  ~ \ \ &  ~ \ \ \\ 
	 ~ Water withdrawal by agriculture (\% of total) &  ~ \ \ & \textit{9} ~ \ \ & 9 ~ \ \ \\ 
	 ~ Biofuel production (thousand kt of oil eq.) & 0 ~ \ \ & \textit{0} ~ \ \ &  ~ \ \ \\ 
	 ~ Wood pellet prod. (min tonnes) &  ~ \ \ &  ~ \ \ &  ~ \ \ \\ 
	 ~ GHG emissions from ag (Co2 eq, gigagrams) & 15 ~ \ \ & 15 ~ \ \ & \textit{16} ~ \ \ \\ 
       \toprule
      \end{tabular}
      \clearpage
\CountryData{ Libya }
      \rowcolors{1}{FAOblue!10}{white}
      \begin{tabular}{L{3.9cm} R{1cm} R{1cm} R{1cm}}
      \toprule
      \multicolumn{1}{c}{} & \multicolumn{1}{c}{ 1992 } & \multicolumn{1}{c}{ 2002 } & \multicolumn{1}{c}{ 2014 } \\
      \midrule
	\multicolumn{4}{l}{\textcolor{FAOblue}{\textbf{\large{The setting}}}} \\ 
	 ~ Population, total (mln) & 4.5 ~ \ \ & 5.3 ~ \ \ & 6.3 ~ \ \ \\ 
	 ~ Population, rural (\% total population) & 1.1 ~ \ \ & 1.3 ~ \ \ & 1.4 ~ \ \ \\ 
	 ~ Govt expenditure on ag (\% total outlays) &  ~ \ \ &  ~ \ \ &  ~ \ \ \\ 
	 ~ Area harvested (mln ha) & 1 ~ \ \ & 1 ~ \ \ & 0 ~ \ \ \\ 
	 ~ Cropping intensity ratio (\%) & 0 ~ \ \ & 0.1 ~ \ \ &  ~ \ \ \\ 
	 ~ Water resources (m\textsuperscript{3}/person/year) & \textit{0} ~ \ \ & \textit{0} ~ \ \ & \textit{0} ~ \ \ \\ 
	 ~ Area equipped for irrigation (1000 ha) &  ~ \ \ &  ~ \ \ & \textit{470} ~ \ \ \\ 
	 ~ Area irrigated (\%) &  ~ \ \ & \textit{67.2} ~ \ \ &  ~ \ \ \\ 
	 ~ Employment in agriculture (\%) &  ~ \ \ &  ~ \ \ &  ~ \ \ \\ 
	 ~ Employment in agriculture, female (\%) &  ~ \ \ &  ~ \ \ &  ~ \ \ \\ 
	 ~ Fertilizers, Nitrogen (nutrients per ha) &  ~ \ \ & 4.9 ~ \ \ & \textit{2.9} ~ \ \ \\ 
	 ~ Fertilizers, Phosphate (nutrients per ha) &  ~ \ \ & 2.5 ~ \ \ & \textit{0.5} ~ \ \ \\ 
	 ~ Fertilizers, Potash (nutrients per ha) &  ~ \ \ & 0.3 ~ \ \ & \textit{0} ~ \ \ \\ 
	 ~ Energy consump, power irrigation (mln kWh) &  ~ \ \ &  ~ \ \ &  ~ \ \ \\ 
	 ~ Agr value added per worker (constant US\$) &  ~ \ \ & \textit{13.4} ~ \ \ & \textit{13.4} ~ \ \ \\ 
	\multicolumn{4}{l}{\textcolor{FAOblue}{\textbf{\large{Hunger dimensions}}}} \\ 
	 ~ Dietary energy supply (kcal/pc/day) &  ~ \ \ &  ~ \ \ &  ~ \ \ \\ 
	 ~ Average dietary energy supply adequacy (\%) &  ~ \ \ &  ~ \ \ &  ~ \ \ \\ 
	 ~ Dietary en supp, cereals/roots/tubers (\%) &  ~ \ \ &  ~ \ \ &  ~ \ \ \\ 
	 ~ Prevalence of undernourishment (\%) &  ~ \ \ &  ~ \ \ &  ~ \ \ \\ 
	 ~ GDP per capita (US\$, PPP) &  ~ \ \ & 21\,134 ~ \ \ & \textit{20\,371} ~ \ \ \\ 
	 ~ Domestic food price volatility (index) &  ~ \ \ &  ~ \ \ &  ~ \ \ \\ 
	 ~ Cereal import dependency ratio (\%) & 91.5 ~ \ \ & 90.1 ~ \ \ & \textit{92} ~ \ \ \\ 
	 ~ Underweight, children under-5 (\%) & \textit{4.3} ~ \ \ & \textit{4.3} ~ \ \ & \textit{5.6} ~ \ \ \\ 
	 ~ Improved water source (\% pop) & 54.4 ~ \ \ & \textit{54.4} ~ \ \ &  ~ \ \ \\ 
	\multicolumn{4}{l}{\textcolor{FAOblue}{\textbf{\large{Food Supply}}}} \\ 
	 ~ Food production value, (2004-2006 mln I\$) & 757 ~ \ \ & 953 ~ \ \ & \textit{1\,103} ~ \ \ \\ 
	 ~ Agriculture, value added (\% GDP) &  ~ \ \ & 5 ~ \ \ & \textit{2} ~ \ \ \\ 
	 ~ Food exports (mln US\$)  & 17 ~ \ \ & 1 ~ \ \ & \textit{2} ~ \ \ \\ 
	 ~ Food imports (mln US\$)  & 900 ~ \ \ & 819 ~ \ \ & \textit{3\,287} ~ \ \ \\ 
	\multicolumn{4}{l}{\textit{\normalsize{Production indices (2004-06=100)}}} \\ 
	 ~ Net food & 76 ~ \ \ & 95 ~ \ \ & \textit{110} ~ \ \ \\ 
	 ~ Net crop & 78 ~ \ \ & 101 ~ \ \ & \textit{110} ~ \ \ \\ 
	 ~ Cereal & 106 ~ \ \ & 98 ~ \ \ & \textit{152} ~ \ \ \\ 
	 ~ Vegetable oils & 56 ~ \ \ & 84 ~ \ \ & \textit{76} ~ \ \ \\ 
	 ~ Roots and tubers & 57 ~ \ \ & 79 ~ \ \ & \textit{120} ~ \ \ \\ 
	 ~ Fruit and vegetables & 80 ~ \ \ & 108 ~ \ \ & \textit{119} ~ \ \ \\ 
	 ~ Sugar &  ~ \ \ &  ~ \ \ &  ~ \ \ \\ 
	 ~ Livestock & 71 ~ \ \ & 92 ~ \ \ & \textit{113} ~ \ \ \\ 
	 ~ Milk & 65 ~ \ \ & 90 ~ \ \ & \textit{97} ~ \ \ \\ 
	 ~ Meat & 75 ~ \ \ & 94 ~ \ \ & \textit{122} ~ \ \ \\ 
	 ~ Fish  & 77 ~ \ \ & 117 ~ \ \ & \textit{96} ~ \ \ \\ 
	\multicolumn{4}{l}{\textit{\normalsize{Net trade (min US\$)}}} \\ 
	 ~ Cereals & -375 ~ \ \ & -440 ~ \ \ & \textit{-1\,233} ~ \ \ \\ 
	 ~ Fruit and vegetables & -84 ~ \ \ & -88 ~ \ \ & \textit{-355} ~ \ \ \\ 
	 ~ Meat & -8 ~ \ \ & -21 ~ \ \ & \textit{-287} ~ \ \ \\ 
	 ~ Dairy products & -83 ~ \ \ & -98 ~ \ \ & \textit{-425} ~ \ \ \\ 
	 ~ Fish & -7 ~ \ \ & 1 ~ \ \ & \textit{-246} ~ \ \ \\ 
	\multicolumn{4}{l}{\textcolor{FAOblue}{\textbf{\large{Environment}}}} \\ 
	 ~ Forest area (\%) & 0 ~ \ \ & 0 ~ \ \ & \textit{0} ~ \ \ \\ 
	 ~ Renewable water res withdrawn (\% of total) &  ~ \ \ & \textit{83} ~ \ \ & 83 ~ \ \ \\ 
	 ~ Terrestrial protect areas (\% total land area)  & 0 ~ \ \ & 0 ~ \ \ & \textit{0} ~ \ \ \\ 
	 ~ Organic area (\% total agricultural area) &  ~ \ \ &  ~ \ \ &  ~ \ \ \\ 
	 ~ Water withdrawal by agriculture (\% of total) &  ~ \ \ & \textit{83} ~ \ \ & 83 ~ \ \ \\ 
	 ~ Biofuel production (thousand kt of oil eq.) &  ~ \ \ &  ~ \ \ &  ~ \ \ \\ 
	 ~ Wood pellet prod. (min tonnes) &  ~ \ \ &  ~ \ \ &  ~ \ \ \\ 
	 ~ GHG emissions from ag (Co2 eq, gigagrams) & 2 ~ \ \ & 2 ~ \ \ & \textit{3} ~ \ \ \\ 
       \toprule
      \end{tabular}
      \clearpage
\CountryData{ Lithuania }
      \rowcolors{1}{FAOblue!10}{white}
      \begin{tabular}{L{3.9cm} R{1cm} R{1cm} R{1cm}}
      \toprule
      \multicolumn{1}{c}{} & \multicolumn{1}{c}{ 1992 } & \multicolumn{1}{c}{ 2002 } & \multicolumn{1}{c}{ 2014 } \\
      \midrule
	\multicolumn{4}{l}{\textcolor{FAOblue}{\textbf{\large{The setting}}}} \\ 
	 ~ Population, total (mln) & 3.7 ~ \ \ & 3.4 ~ \ \ & 3 ~ \ \ \\ 
	 ~ Population, rural (\% total population) & 1.2 ~ \ \ & 1.1 ~ \ \ & 1 ~ \ \ \\ 
	 ~ Govt expenditure on ag (\% total outlays) &  ~ \ \ &  ~ \ \ &  ~ \ \ \\ 
	 ~ Area harvested (mln ha) & 2 ~ \ \ & 3 ~ \ \ & 4 ~ \ \ \\ 
	 ~ Cropping intensity ratio (\%) & 0.6 ~ \ \ & 0.9 ~ \ \ &  ~ \ \ \\ 
	 ~ Water resources (m\textsuperscript{3}/person/year) & \textit{7} ~ \ \ & \textit{7} ~ \ \ & \textit{8} ~ \ \ \\ 
	 ~ Area equipped for irrigation (1000 ha) &  ~ \ \ &  ~ \ \ & \textit{4} ~ \ \ \\ 
	 ~ Area irrigated (\%) &  ~ \ \ &  ~ \ \ & \textit{22.4} ~ \ \ \\ 
	 ~ Employment in agriculture (\%) &  ~ \ \ & 17.8 ~ \ \ & \textit{8.9} ~ \ \ \\ 
	 ~ Employment in agriculture, female (\%) &  ~ \ \ & 14.2 ~ \ \ & \textit{6.4} ~ \ \ \\ 
	 ~ Fertilizers, Nitrogen (nutrients per ha) &  ~ \ \ & 4.7 ~ \ \ & \textit{23.3} ~ \ \ \\ 
	 ~ Fertilizers, Phosphate (nutrients per ha) &  ~ \ \ & 23.4 ~ \ \ & \textit{33.8} ~ \ \ \\ 
	 ~ Fertilizers, Potash (nutrients per ha) &  ~ \ \ & 34.5 ~ \ \ & \textit{21.7} ~ \ \ \\ 
	 ~ Energy consump, power irrigation (mln kWh) & \textit{18} ~ \ \ & 18 ~ \ \ & \textit{18} ~ \ \ \\ 
	 ~ Agr value added per worker (constant US\$) & \textit{3.5} ~ \ \ & 5.7 ~ \ \ & \textit{9.4} ~ \ \ \\ 
	\multicolumn{4}{l}{\textcolor{FAOblue}{\textbf{\large{Hunger dimensions}}}} \\ 
	 ~ Dietary energy supply (kcal/pc/day) &  ~ \ \ &  ~ \ \ &  ~ \ \ \\ 
	 ~ Average dietary energy supply adequacy (\%) & 116 ~ \ \ & 131 ~ \ \ & 143 ~ \ \ \\ 
	 ~ Dietary en supp, cereals/roots/tubers (\%) & 51 ~ \ \ & 45 ~ \ \ & \textit{39} ~ \ \ \\ 
	 ~ Prevalence of undernourishment (\%) & <5.0 ~ \ \ & <5.0 ~ \ \ & <5.0 ~ \ \ \\ 
	 ~ GDP per capita (US\$, PPP) & 11\,587 ~ \ \ & 13\,781 ~ \ \ & \textit{24\,470} ~ \ \ \\ 
	 ~ Domestic food price volatility (index) &  ~ \ \ & 6.3 ~ \ \ & 5.5 ~ \ \ \\ 
	 ~ Cereal import dependency ratio (\%) & 6.3 ~ \ \ & -10.6 ~ \ \ & \textit{-74.7} ~ \ \ \\ 
	 ~ Underweight, children under-5 (\%) &  ~ \ \ &  ~ \ \ &  ~ \ \ \\ 
	 ~ Improved water source (\% pop) & 87.9 ~ \ \ & 91.8 ~ \ \ & \textit{95.9} ~ \ \ \\ 
	\multicolumn{4}{l}{\textcolor{FAOblue}{\textbf{\large{Food Supply}}}} \\ 
	 ~ Food production value, (2004-2006 mln I\$) & 2\,119 ~ \ \ & 1\,380 ~ \ \ & \textit{1\,799} ~ \ \ \\ 
	 ~ Agriculture, value added (\% GDP) & 14 ~ \ \ & 5 ~ \ \ & \textit{3} ~ \ \ \\ 
	 ~ Food exports (mln US\$)  & 159 ~ \ \ & 358 ~ \ \ & \textit{3\,785} ~ \ \ \\ 
	 ~ Food imports (mln US\$)  & 71 ~ \ \ & 364 ~ \ \ & \textit{2\,621} ~ \ \ \\ 
	\multicolumn{4}{l}{\textit{\normalsize{Production indices (2004-06=100)}}} \\ 
	 ~ Net food & 138 ~ \ \ & 90 ~ \ \ & \textit{117} ~ \ \ \\ 
	 ~ Net crop & 80 ~ \ \ & 113 ~ \ \ & \textit{156} ~ \ \ \\ 
	 ~ Cereal & 81 ~ \ \ & 99 ~ \ \ & \textit{193} ~ \ \ \\ 
	 ~ Vegetable oils & 6 ~ \ \ & 57 ~ \ \ & \textit{294} ~ \ \ \\ 
	 ~ Roots and tubers & 134 ~ \ \ & 222 ~ \ \ & \textit{58} ~ \ \ \\ 
	 ~ Fruit and vegetables & 79 ~ \ \ & 88 ~ \ \ & \textit{93} ~ \ \ \\ 
	 ~ Sugar & 77 ~ \ \ & 130 ~ \ \ & \textit{120} ~ \ \ \\ 
	 ~ Livestock & 161 ~ \ \ & 85 ~ \ \ & \textit{97} ~ \ \ \\ 
	 ~ Milk & 130 ~ \ \ & 95 ~ \ \ & \textit{92} ~ \ \ \\ 
	 ~ Meat & 209 ~ \ \ & 74 ~ \ \ & \textit{102} ~ \ \ \\ 
	 ~ Fish  & 125 ~ \ \ & 98 ~ \ \ & \textit{62} ~ \ \ \\ 
	\multicolumn{4}{l}{\textit{\normalsize{Net trade (min US\$)}}} \\ 
	 ~ Cereals & -37 ~ \ \ & -2 ~ \ \ & \textit{620} ~ \ \ \\ 
	 ~ Fruit and vegetables & -2 ~ \ \ & -73 ~ \ \ & \textit{-120} ~ \ \ \\ 
	 ~ Meat & 34 ~ \ \ & -11 ~ \ \ & \textit{48} ~ \ \ \\ 
	 ~ Dairy products & 83 ~ \ \ & 129 ~ \ \ & \textit{436} ~ \ \ \\ 
	 ~ Fish & 7 ~ \ \ & -10 ~ \ \ & \textit{23} ~ \ \ \\ 
	\multicolumn{4}{l}{\textcolor{FAOblue}{\textbf{\large{Environment}}}} \\ 
	 ~ Forest area (\%) & 31 ~ \ \ & 33 ~ \ \ & \textit{35} ~ \ \ \\ 
	 ~ Renewable water res withdrawn (\% of total) &  ~ \ \ &  ~ \ \ & 3 ~ \ \ \\ 
	 ~ Terrestrial protect areas (\% total land area)  & 11 ~ \ \ & 12 ~ \ \ & \textit{17} ~ \ \ \\ 
	 ~ Organic area (\% total agricultural area) &  ~ \ \ & \textit{2} ~ \ \ & \textit{6} ~ \ \ \\ 
	 ~ Water withdrawal by agriculture (\% of total) &  ~ \ \ &  ~ \ \ & 3 ~ \ \ \\ 
	 ~ Biofuel production (thousand kt of oil eq.) & 5 ~ \ \ & 11 ~ \ \ & \textit{2\,431} ~ \ \ \\ 
	 ~ Wood pellet prod. (min tonnes) &  ~ \ \ &  ~ \ \ & \textit{289} ~ \ \ \\ 
	 ~ GHG emissions from ag (Co2 eq, gigagrams) & 14 ~ \ \ & 6 ~ \ \ & \textit{9} ~ \ \ \\ 
       \toprule
      \end{tabular}
      \clearpage
\CountryData{ Luxembourg }
      \rowcolors{1}{FAOblue!10}{white}
      \begin{tabular}{L{3.9cm} R{1cm} R{1cm} R{1cm}}
      \toprule
      \multicolumn{1}{c}{} & \multicolumn{1}{c}{ 1992 } & \multicolumn{1}{c}{ 2002 } & \multicolumn{1}{c}{ 2014 } \\
      \midrule
	\multicolumn{4}{l}{\textcolor{FAOblue}{\textbf{\large{The setting}}}} \\ 
	 ~ Population, total (mln) &  ~ \ \ & 0.4 ~ \ \ & 0.5 ~ \ \ \\ 
	 ~ Population, rural (\% total population) &  ~ \ \ & 0.1 ~ \ \ & 0.1 ~ \ \ \\ 
	 ~ Govt expenditure on ag (\% total outlays) &  ~ \ \ &  ~ \ \ &  ~ \ \ \\ 
	 ~ Area harvested (mln ha) &  ~ \ \ & 0 ~ \ \ & 0 ~ \ \ \\ 
	 ~ Cropping intensity ratio (\%) &  ~ \ \ & 3.5 ~ \ \ &  ~ \ \ \\ 
	 ~ Water resources (m\textsuperscript{3}/person/year) & \textit{9} ~ \ \ & \textit{8} ~ \ \ & \textit{7} ~ \ \ \\ 
	 ~ Area equipped for irrigation (1000 ha) &  ~ \ \ &  ~ \ \ &  ~ \ \ \\ 
	 ~ Area irrigated (\%) &  ~ \ \ & 100 ~ \ \ &  ~ \ \ \\ 
	 ~ Employment in agriculture (\%) & 6.3 ~ \ \ & 2 ~ \ \ & \textit{1.3} ~ \ \ \\ 
	 ~ Employment in agriculture, female (\%) & 5.8 ~ \ \ & 1.1 ~ \ \ & \textit{0.9} ~ \ \ \\ 
	 ~ Fertilizers, Nitrogen (nutrients per ha) &  ~ \ \ & 203.5 ~ \ \ & \textit{170.9} ~ \ \ \\ 
	 ~ Fertilizers, Phosphate (nutrients per ha) &  ~ \ \ & 38 ~ \ \ & \textit{6} ~ \ \ \\ 
	 ~ Fertilizers, Potash (nutrients per ha) &  ~ \ \ & 39.9 ~ \ \ & \textit{15.3} ~ \ \ \\ 
	 ~ Energy consump, power irrigation (mln kWh) &  ~ \ \ &  ~ \ \ &  ~ \ \ \\ 
	 ~ Agr value added per worker (constant US\$) &  ~ \ \ & 69.7 ~ \ \ & \textit{36.3} ~ \ \ \\ 
	\multicolumn{4}{l}{\textcolor{FAOblue}{\textbf{\large{Hunger dimensions}}}} \\ 
	 ~ Dietary energy supply (kcal/pc/day) &  ~ \ \ &  ~ \ \ &  ~ \ \ \\ 
	 ~ Average dietary energy supply adequacy (\%) &  ~ \ \ & 140 ~ \ \ & 140 ~ \ \ \\ 
	 ~ Dietary en supp, cereals/roots/tubers (\%) &  ~ \ \ & 25 ~ \ \ & \textit{30} ~ \ \ \\ 
	 ~ Prevalence of undernourishment (\%) & <5.0 ~ \ \ & <5.0 ~ \ \ & <5.0 ~ \ \ \\ 
	 ~ GDP per capita (US\$, PPP) & 61\,310 ~ \ \ & 83\,876 ~ \ \ & \textit{88\,850} ~ \ \ \\ 
	 ~ Domestic food price volatility (index) &  ~ \ \ & 5.8 ~ \ \ & 8.9 ~ \ \ \\ 
	 ~ Cereal import dependency ratio (\%) &  ~ \ \ & 5.7 ~ \ \ & \textit{14.6} ~ \ \ \\ 
	 ~ Underweight, children under-5 (\%) &  ~ \ \ &  ~ \ \ &  ~ \ \ \\ 
	 ~ Improved water source (\% pop) & 100 ~ \ \ & 100 ~ \ \ & \textit{100} ~ \ \ \\ 
	\multicolumn{4}{l}{\textcolor{FAOblue}{\textbf{\large{Food Supply}}}} \\ 
	 ~ Food production value, (2004-2006 mln I\$) &  ~ \ \ & 187 ~ \ \ & \textit{166} ~ \ \ \\ 
	 ~ Agriculture, value added (\% GDP) & \textit{1} ~ \ \ & 1 ~ \ \ & \textit{0} ~ \ \ \\ 
	 ~ Food exports (mln US\$)  &  ~ \ \ & 347 ~ \ \ & \textit{918} ~ \ \ \\ 
	 ~ Food imports (mln US\$)  &  ~ \ \ & 646 ~ \ \ & \textit{1\,534} ~ \ \ \\ 
	\multicolumn{4}{l}{\textit{\normalsize{Production indices (2004-06=100)}}} \\ 
	 ~ Net food &  ~ \ \ & 101 ~ \ \ & \textit{90} ~ \ \ \\ 
	 ~ Net crop &  ~ \ \ & 106 ~ \ \ & \textit{86} ~ \ \ \\ 
	 ~ Cereal &  ~ \ \ & 101 ~ \ \ & \textit{104} ~ \ \ \\ 
	 ~ Vegetable oils &  ~ \ \ & 79 ~ \ \ & \textit{96} ~ \ \ \\ 
	 ~ Roots and tubers &  ~ \ \ & 108 ~ \ \ & \textit{87} ~ \ \ \\ 
	 ~ Fruit and vegetables &  ~ \ \ & 119 ~ \ \ & \textit{59} ~ \ \ \\ 
	 ~ Sugar &  ~ \ \ &  ~ \ \ &  ~ \ \ \\ 
	 ~ Livestock &  ~ \ \ & 100 ~ \ \ & \textit{96} ~ \ \ \\ 
	 ~ Milk &  ~ \ \ & 101 ~ \ \ & \textit{111} ~ \ \ \\ 
	 ~ Meat &  ~ \ \ & 100 ~ \ \ & \textit{76} ~ \ \ \\ 
	 ~ Fish  &  ~ \ \ &  ~ \ \ &  ~ \ \ \\ 
	\multicolumn{4}{l}{\textit{\normalsize{Net trade (min US\$)}}} \\ 
	 ~ Cereals &  ~ \ \ & -44 ~ \ \ & \textit{-96} ~ \ \ \\ 
	 ~ Fruit and vegetables &  ~ \ \ & -123 ~ \ \ & \textit{-201} ~ \ \ \\ 
	 ~ Meat &  ~ \ \ & -81 ~ \ \ & \textit{-177} ~ \ \ \\ 
	 ~ Dairy products &  ~ \ \ & 12 ~ \ \ & \textit{-12} ~ \ \ \\ 
	 ~ Fish &  ~ \ \ & -39 ~ \ \ & \textit{-94} ~ \ \ \\ 
	\multicolumn{4}{l}{\textcolor{FAOblue}{\textbf{\large{Environment}}}} \\ 
	 ~ Forest area (\%) &  ~ \ \ & 34 ~ \ \ & \textit{33} ~ \ \ \\ 
	 ~ Renewable water res withdrawn (\% of total) &  ~ \ \ & \textit{0} ~ \ \ & 0 ~ \ \ \\ 
	 ~ Terrestrial protect areas (\% total land area)  & 12 ~ \ \ & 20 ~ \ \ & \textit{40} ~ \ \ \\ 
	 ~ Organic area (\% total agricultural area) &  ~ \ \ & \textit{2} ~ \ \ & \textit{3} ~ \ \ \\ 
	 ~ Water withdrawal by agriculture (\% of total) &  ~ \ \ & \textit{0} ~ \ \ & 0 ~ \ \ \\ 
	 ~ Biofuel production (thousand kt of oil eq.) & 0 ~ \ \ & 0 ~ \ \ & \textit{12} ~ \ \ \\ 
	 ~ Wood pellet prod. (min tonnes) &  ~ \ \ &  ~ \ \ & \textit{45} ~ \ \ \\ 
	 ~ GHG emissions from ag (Co2 eq, gigagrams) &  ~ \ \ & 1 ~ \ \ & \textit{1} ~ \ \ \\ 
       \toprule
      \end{tabular}
      \clearpage
\CountryData{ Macedonia }
      \rowcolors{1}{FAOblue!10}{white}
      \begin{tabular}{L{3.9cm} R{1cm} R{1cm} R{1cm}}
      \toprule
      \multicolumn{1}{c}{} & \multicolumn{1}{c}{ 1992 } & \multicolumn{1}{c}{ 2002 } & \multicolumn{1}{c}{ 2014 } \\
      \midrule
	\multicolumn{4}{l}{\textcolor{FAOblue}{\textbf{\large{The setting}}}} \\ 
	 ~ Population, total (mln) & 2 ~ \ \ & 2.1 ~ \ \ & 2.1 ~ \ \ \\ 
	 ~ Population, rural (\% total population) & 0.8 ~ \ \ & 0.8 ~ \ \ & 0.9 ~ \ \ \\ 
	 ~ Govt expenditure on ag (\% total outlays) &  ~ \ \ &  ~ \ \ &  ~ \ \ \\ 
	 ~ Area harvested (mln ha) & 1 ~ \ \ & 1 ~ \ \ & 1 ~ \ \ \\ 
	 ~ Cropping intensity ratio (\%) & 0.5 ~ \ \ & 0.4 ~ \ \ &  ~ \ \ \\ 
	 ~ Water resources (m\textsuperscript{3}/person/year) & \textit{3} ~ \ \ & \textit{3} ~ \ \ & \textit{3} ~ \ \ \\ 
	 ~ Area equipped for irrigation (1000 ha) &  ~ \ \ &  ~ \ \ & \textit{128} ~ \ \ \\ 
	 ~ Area irrigated (\%) &  ~ \ \ &  ~ \ \ &  ~ \ \ \\ 
	 ~ Employment in agriculture (\%) &  ~ \ \ & 23.9 ~ \ \ & \textit{17.3} ~ \ \ \\ 
	 ~ Employment in agriculture, female (\%) &  ~ \ \ & 24.7 ~ \ \ & \textit{17.6} ~ \ \ \\ 
	 ~ Fertilizers, Nitrogen (nutrients per ha) &  ~ \ \ & 6.7 ~ \ \ & \textit{12.9} ~ \ \ \\ 
	 ~ Fertilizers, Phosphate (nutrients per ha) &  ~ \ \ & 0 ~ \ \ & \textit{3.1} ~ \ \ \\ 
	 ~ Fertilizers, Potash (nutrients per ha) &  ~ \ \ & 4.8 ~ \ \ & \textit{2.7} ~ \ \ \\ 
	 ~ Energy consump, power irrigation (mln kWh) &  ~ \ \ &  ~ \ \ &  ~ \ \ \\ 
	 ~ Agr value added per worker (constant US\$) & 4.3 ~ \ \ & 5.8 ~ \ \ & \textit{11.8} ~ \ \ \\ 
	\multicolumn{4}{l}{\textcolor{FAOblue}{\textbf{\large{Hunger dimensions}}}} \\ 
	 ~ Dietary energy supply (kcal/pc/day) &  ~ \ \ &  ~ \ \ &  ~ \ \ \\ 
	 ~ Average dietary energy supply adequacy (\%) & 95 ~ \ \ & 108 ~ \ \ & 118 ~ \ \ \\ 
	 ~ Dietary en supp, cereals/roots/tubers (\%) & 50 ~ \ \ & 40 ~ \ \ & \textit{37} ~ \ \ \\ 
	 ~ Prevalence of undernourishment (\%) & <5.0 ~ \ \ & <5.0 ~ \ \ & <5.0 ~ \ \ \\ 
	 ~ GDP per capita (US\$, PPP) & 8\,666 ~ \ \ & 8\,342 ~ \ \ & \textit{11\,609} ~ \ \ \\ 
	 ~ Domestic food price volatility (index) &  ~ \ \ & 8 ~ \ \ & \textit{7.9} ~ \ \ \\ 
	 ~ Cereal import dependency ratio (\%) & 19.5 ~ \ \ & 29 ~ \ \ & \textit{25.4} ~ \ \ \\ 
	 ~ Underweight, children under-5 (\%) &  ~ \ \ & \textit{1.8} ~ \ \ & \textit{1.3} ~ \ \ \\ 
	 ~ Improved water source (\% pop) & 99.3 ~ \ \ & 99.3 ~ \ \ & \textit{99.4} ~ \ \ \\ 
	\multicolumn{4}{l}{\textcolor{FAOblue}{\textbf{\large{Food Supply}}}} \\ 
	 ~ Food production value, (2004-2006 mln I\$) & 661 ~ \ \ & 513 ~ \ \ & \textit{764} ~ \ \ \\ 
	 ~ Agriculture, value added (\% GDP) & 17 ~ \ \ & 12 ~ \ \ & \textit{10} ~ \ \ \\ 
	 ~ Food exports (mln US\$)  & 148 ~ \ \ & 70 ~ \ \ & \textit{367} ~ \ \ \\ 
	 ~ Food imports (mln US\$)  & 127 ~ \ \ & 239 ~ \ \ & \textit{666} ~ \ \ \\ 
	\multicolumn{4}{l}{\textit{\normalsize{Production indices (2004-06=100)}}} \\ 
	 ~ Net food & 102 ~ \ \ & 79 ~ \ \ & \textit{118} ~ \ \ \\ 
	 ~ Net crop & 102 ~ \ \ & 78 ~ \ \ & \textit{115} ~ \ \ \\ 
	 ~ Cereal & 100 ~ \ \ & 84 ~ \ \ & \textit{89} ~ \ \ \\ 
	 ~ Vegetable oils & 144 ~ \ \ & 104 ~ \ \ & \textit{101} ~ \ \ \\ 
	 ~ Roots and tubers & 68 ~ \ \ & 94 ~ \ \ & \textit{100} ~ \ \ \\ 
	 ~ Fruit and vegetables & 103 ~ \ \ & 73 ~ \ \ & \textit{123} ~ \ \ \\ 
	 ~ Sugar & 136 ~ \ \ & 97 ~ \ \ & \textit{21} ~ \ \ \\ 
	 ~ Livestock & 98 ~ \ \ & 91 ~ \ \ & \textit{116} ~ \ \ \\ 
	 ~ Milk & 67 ~ \ \ & 89 ~ \ \ & \textit{149} ~ \ \ \\ 
	 ~ Meat & 134 ~ \ \ & 90 ~ \ \ & \textit{82} ~ \ \ \\ 
	 ~ Fish  & 109 ~ \ \ & 90 ~ \ \ & \textit{166} ~ \ \ \\ 
	\multicolumn{4}{l}{\textit{\normalsize{Net trade (min US\$)}}} \\ 
	 ~ Cereals & -7 ~ \ \ & -44 ~ \ \ & \textit{-38} ~ \ \ \\ 
	 ~ Fruit and vegetables & 59 ~ \ \ & 8 ~ \ \ & \textit{108} ~ \ \ \\ 
	 ~ Meat & -1 ~ \ \ & -60 ~ \ \ & \textit{-132} ~ \ \ \\ 
	 ~ Dairy products & -12 ~ \ \ & -10 ~ \ \ & \textit{-42} ~ \ \ \\ 
	 ~ Fish & -4 ~ \ \ & -10 ~ \ \ & \textit{-23} ~ \ \ \\ 
	\multicolumn{4}{l}{\textcolor{FAOblue}{\textbf{\large{Environment}}}} \\ 
	 ~ Forest area (\%) & 36 ~ \ \ & 38 ~ \ \ & \textit{40} ~ \ \ \\ 
	 ~ Renewable water res withdrawn (\% of total) &  ~ \ \ &  ~ \ \ & 12 ~ \ \ \\ 
	 ~ Terrestrial protect areas (\% total land area)  & 4 ~ \ \ & 5 ~ \ \ & \textit{7} ~ \ \ \\ 
	 ~ Organic area (\% total agricultural area) &  ~ \ \ & \textit{0} ~ \ \ & \textit{0} ~ \ \ \\ 
	 ~ Water withdrawal by agriculture (\% of total) &  ~ \ \ &  ~ \ \ & 12 ~ \ \ \\ 
	 ~ Biofuel production (thousand kt of oil eq.) &  ~ \ \ &  ~ \ \ & \textit{54} ~ \ \ \\ 
	 ~ Wood pellet prod. (min tonnes) &  ~ \ \ &  ~ \ \ & \textit{0} ~ \ \ \\ 
	 ~ GHG emissions from ag (Co2 eq, gigagrams) & 2 ~ \ \ & 2 ~ \ \ & \textit{1} ~ \ \ \\ 
       \toprule
      \end{tabular}
      \clearpage
\CountryData{ Madagascar }
      \rowcolors{1}{FAOblue!10}{white}
      \begin{tabular}{L{3.9cm} R{1cm} R{1cm} R{1cm}}
      \toprule
      \multicolumn{1}{c}{} & \multicolumn{1}{c}{ 1992 } & \multicolumn{1}{c}{ 2002 } & \multicolumn{1}{c}{ 2014 } \\
      \midrule
	\multicolumn{4}{l}{\textcolor{FAOblue}{\textbf{\large{The setting}}}} \\ 
	 ~ Population, total (mln) & 12.3 ~ \ \ & 16.7 ~ \ \ & 23.6 ~ \ \ \\ 
	 ~ Population, rural (\% total population) & 9.2 ~ \ \ & 12.1 ~ \ \ & 15.4 ~ \ \ \\ 
	 ~ Govt expenditure on ag (\% total outlays) &  ~ \ \ &  ~ \ \ &  ~ \ \ \\ 
	 ~ Area harvested (mln ha) & 3 ~ \ \ & 3 ~ \ \ & 4 ~ \ \ \\ 
	 ~ Cropping intensity ratio (\%) & 0.1 ~ \ \ & 0.1 ~ \ \ &  ~ \ \ \\ 
	 ~ Water resources (m\textsuperscript{3}/person/year) & \textit{27} ~ \ \ & \textit{20} ~ \ \ & \textit{15} ~ \ \ \\ 
	 ~ Area equipped for irrigation (1000 ha) &  ~ \ \ &  ~ \ \ & \textit{1\,086} ~ \ \ \\ 
	 ~ Area irrigated (\%) &  ~ \ \ & \textit{50.6} ~ \ \ &  ~ \ \ \\ 
	 ~ Employment in agriculture (\%) &  ~ \ \ & \textit{80.4} ~ \ \ & \textit{80.4} ~ \ \ \\ 
	 ~ Employment in agriculture, female (\%) &  ~ \ \ & \textit{81.1} ~ \ \ & \textit{81.1} ~ \ \ \\ 
	 ~ Fertilizers, Nitrogen (nutrients per ha) &  ~ \ \ & 0.1 ~ \ \ & \textit{0.1} ~ \ \ \\ 
	 ~ Fertilizers, Phosphate (nutrients per ha) &  ~ \ \ & 0 ~ \ \ & \textit{0.1} ~ \ \ \\ 
	 ~ Fertilizers, Potash (nutrients per ha) &  ~ \ \ & 0 ~ \ \ & \textit{0.1} ~ \ \ \\ 
	 ~ Energy consump, power irrigation (mln kWh) &  ~ \ \ & 6 ~ \ \ & \textit{6} ~ \ \ \\ 
	 ~ Agr value added per worker (constant US\$) & 0.2 ~ \ \ & 0.2 ~ \ \ & \textit{0.2} ~ \ \ \\ 
	\multicolumn{4}{l}{\textcolor{FAOblue}{\textbf{\large{Hunger dimensions}}}} \\ 
	 ~ Dietary energy supply (kcal/pc/day) & 2\,097 ~ \ \ & 1\,972 ~ \ \ & 2\,059 ~ \ \ \\ 
	 ~ Average dietary energy supply adequacy (\%) & 101 ~ \ \ & 94 ~ \ \ & 97 ~ \ \ \\ 
	 ~ Dietary en supp, cereals/roots/tubers (\%) & 74 ~ \ \ & 78 ~ \ \ & \textit{80} ~ \ \ \\ 
	 ~ Prevalence of undernourishment (\%) & 29.1 ~ \ \ & 36.9 ~ \ \ & 32.8 ~ \ \ \\ 
	 ~ GDP per capita (US\$, PPP) & 1\,482 ~ \ \ & 1\,260 ~ \ \ & \textit{1\,369} ~ \ \ \\ 
	 ~ Domestic food price volatility (index) &  ~ \ \ & 13.9 ~ \ \ & 3.5 ~ \ \ \\ 
	 ~ Cereal import dependency ratio (\%) & 5 ~ \ \ & 12 ~ \ \ & \textit{8.7} ~ \ \ \\ 
	 ~ Underweight, children under-5 (\%) & 35.5 ~ \ \ & \textit{36.8} ~ \ \ &  ~ \ \ \\ 
	 ~ Improved water source (\% pop) & 30.8 ~ \ \ & 39.8 ~ \ \ & \textit{49.6} ~ \ \ \\ 
	\multicolumn{4}{l}{\textcolor{FAOblue}{\textbf{\large{Food Supply}}}} \\ 
	 ~ Food production value, (2004-2006 mln I\$) & 2\,297 ~ \ \ & 2\,365 ~ \ \ & \textit{3\,312} ~ \ \ \\ 
	 ~ Agriculture, value added (\% GDP) & 29 ~ \ \ & 32 ~ \ \ & \textit{26} ~ \ \ \\ 
	 ~ Food exports (mln US\$)  & 104 ~ \ \ & 183 ~ \ \ & \textit{274} ~ \ \ \\ 
	 ~ Food imports (mln US\$)  & 54 ~ \ \ & 66 ~ \ \ & \textit{340} ~ \ \ \\ 
	\multicolumn{4}{l}{\textit{\normalsize{Production indices (2004-06=100)}}} \\ 
	 ~ Net food & 84 ~ \ \ & 86 ~ \ \ & \textit{121} ~ \ \ \\ 
	 ~ Net crop & 79 ~ \ \ & 84 ~ \ \ & \textit{117} ~ \ \ \\ 
	 ~ Cereal & 71 ~ \ \ & 76 ~ \ \ & \textit{109} ~ \ \ \\ 
	 ~ Vegetable oils & 76 ~ \ \ & 93 ~ \ \ & \textit{95} ~ \ \ \\ 
	 ~ Roots and tubers & 81 ~ \ \ & 89 ~ \ \ & \textit{118} ~ \ \ \\ 
	 ~ Fruit and vegetables & 83 ~ \ \ & 94 ~ \ \ & \textit{122} ~ \ \ \\ 
	 ~ Sugar & 77 ~ \ \ & 91 ~ \ \ & \textit{132} ~ \ \ \\ 
	 ~ Livestock & 101 ~ \ \ & 88 ~ \ \ & \textit{125} ~ \ \ \\ 
	 ~ Milk & 97 ~ \ \ & 94 ~ \ \ & \textit{113} ~ \ \ \\ 
	 ~ Meat & 104 ~ \ \ & 85 ~ \ \ & \textit{130} ~ \ \ \\ 
	 ~ Fish  & 74 ~ \ \ & 97 ~ \ \ & \textit{79} ~ \ \ \\ 
	\multicolumn{4}{l}{\textit{\normalsize{Net trade (min US\$)}}} \\ 
	 ~ Cereals & -31 ~ \ \ & -27 ~ \ \ & \textit{-178} ~ \ \ \\ 
	 ~ Fruit and vegetables & 19 ~ \ \ & 19 ~ \ \ & \textit{42} ~ \ \ \\ 
	 ~ Meat & 3 ~ \ \ & 0 ~ \ \ & \textit{-1} ~ \ \ \\ 
	 ~ Dairy products & -4 ~ \ \ & -6 ~ \ \ & \textit{-13} ~ \ \ \\ 
	 ~ Fish & 44 ~ \ \ & 139 ~ \ \ & \textit{103} ~ \ \ \\ 
	\multicolumn{4}{l}{\textcolor{FAOblue}{\textbf{\large{Environment}}}} \\ 
	 ~ Forest area (\%) & 23 ~ \ \ & 22 ~ \ \ & \textit{21} ~ \ \ \\ 
	 ~ Renewable water res withdrawn (\% of total) &  ~ \ \ & \textit{98} ~ \ \ & 98 ~ \ \ \\ 
	 ~ Terrestrial protect areas (\% total land area)  & 2 ~ \ \ & 3 ~ \ \ & \textit{5} ~ \ \ \\ 
	 ~ Organic area (\% total agricultural area) &  ~ \ \ & \textit{0} ~ \ \ & \textit{0} ~ \ \ \\ 
	 ~ Water withdrawal by agriculture (\% of total) &  ~ \ \ & \textit{98} ~ \ \ & 98 ~ \ \ \\ 
	 ~ Biofuel production (thousand kt of oil eq.) & 2 ~ \ \ & 1 ~ \ \ & \textit{2} ~ \ \ \\ 
	 ~ Wood pellet prod. (min tonnes) &  ~ \ \ &  ~ \ \ &  ~ \ \ \\ 
	 ~ GHG emissions from ag (Co2 eq, gigagrams) & 57 ~ \ \ & 42 ~ \ \ & \textit{53} ~ \ \ \\ 
       \toprule
      \end{tabular}
      \clearpage
\CountryData{ Malawi }
      \rowcolors{1}{FAOblue!10}{white}
      \begin{tabular}{L{3.9cm} R{1cm} R{1cm} R{1cm}}
      \toprule
      \multicolumn{1}{c}{} & \multicolumn{1}{c}{ 1992 } & \multicolumn{1}{c}{ 2002 } & \multicolumn{1}{c}{ 2014 } \\
      \midrule
	\multicolumn{4}{l}{\textcolor{FAOblue}{\textbf{\large{The setting}}}} \\ 
	 ~ Population, total (mln) & 9.8 ~ \ \ & 11.9 ~ \ \ & 16.8 ~ \ \ \\ 
	 ~ Population, rural (\% total population) & 8.6 ~ \ \ & 10.2 ~ \ \ & 14.1 ~ \ \ \\ 
	 ~ Govt expenditure on ag (\% total outlays) &  ~ \ \ &  ~ \ \ &  ~ \ \ \\ 
	 ~ Area harvested (mln ha) & 2 ~ \ \ & 3 ~ \ \ & 9 ~ \ \ \\ 
	 ~ Cropping intensity ratio (\%) & 0.4 ~ \ \ & 0.6 ~ \ \ &  ~ \ \ \\ 
	 ~ Water resources (m\textsuperscript{3}/person/year) & \textit{2} ~ \ \ & \textit{1} ~ \ \ & \textit{1} ~ \ \ \\ 
	 ~ Area equipped for irrigation (1000 ha) &  ~ \ \ &  ~ \ \ & \textit{74} ~ \ \ \\ 
	 ~ Area irrigated (\%) & 96 ~ \ \ &  ~ \ \ &  ~ \ \ \\ 
	 ~ Employment in agriculture (\%) &  ~ \ \ &  ~ \ \ &  ~ \ \ \\ 
	 ~ Employment in agriculture, female (\%) &  ~ \ \ &  ~ \ \ &  ~ \ \ \\ 
	 ~ Fertilizers, Nitrogen (nutrients per ha) &  ~ \ \ & 12.4 ~ \ \ & \textit{19.8} ~ \ \ \\ 
	 ~ Fertilizers, Phosphate (nutrients per ha) &  ~ \ \ & 3.7 ~ \ \ & \textit{3.9} ~ \ \ \\ 
	 ~ Fertilizers, Potash (nutrients per ha) &  ~ \ \ & 1.4 ~ \ \ & \textit{2.4} ~ \ \ \\ 
	 ~ Energy consump, power irrigation (mln kWh) & 30 ~ \ \ & 30 ~ \ \ & \textit{117} ~ \ \ \\ 
	 ~ Agr value added per worker (constant US\$) & 0.1 ~ \ \ & 0.2 ~ \ \ & \textit{0.2} ~ \ \ \\ 
	\multicolumn{4}{l}{\textcolor{FAOblue}{\textbf{\large{Hunger dimensions}}}} \\ 
	 ~ Dietary energy supply (kcal/pc/day) & 1\,864 ~ \ \ & 2\,191 ~ \ \ & 2\,355 ~ \ \ \\ 
	 ~ Average dietary energy supply adequacy (\%) & 89 ~ \ \ & 105 ~ \ \ & 111 ~ \ \ \\ 
	 ~ Dietary en supp, cereals/roots/tubers (\%) & 73 ~ \ \ & 73 ~ \ \ & \textit{71} ~ \ \ \\ 
	 ~ Prevalence of undernourishment (\%) & 45.7 ~ \ \ & 26.6 ~ \ \ & 20.8 ~ \ \ \\ 
	 ~ GDP per capita (US\$, PPP) & 527 ~ \ \ & 580 ~ \ \ & \textit{755} ~ \ \ \\ 
	 ~ Domestic food price volatility (index) &  ~ \ \ & 18.3 ~ \ \ & \textit{23.6} ~ \ \ \\ 
	 ~ Cereal import dependency ratio (\%) & 21.8 ~ \ \ & 10.8 ~ \ \ & \textit{1.6} ~ \ \ \\ 
	 ~ Underweight, children under-5 (\%) & 24.4 ~ \ \ & \textit{18.4} ~ \ \ & 16.7 ~ \ \ \\ 
	 ~ Improved water source (\% pop) & 46.3 ~ \ \ & 66.3 ~ \ \ & \textit{85} ~ \ \ \\ 
	\multicolumn{4}{l}{\textcolor{FAOblue}{\textbf{\large{Food Supply}}}} \\ 
	 ~ Food production value, (2004-2006 mln I\$) & 613 ~ \ \ & 1\,315 ~ \ \ & \textit{3\,105} ~ \ \ \\ 
	 ~ Agriculture, value added (\% GDP) & 39 ~ \ \ & 37 ~ \ \ & \textit{27} ~ \ \ \\ 
	 ~ Food exports (mln US\$)  & 22 ~ \ \ & 51 ~ \ \ & \textit{207} ~ \ \ \\ 
	 ~ Food imports (mln US\$)  & 162 ~ \ \ & 157 ~ \ \ & \textit{194} ~ \ \ \\ 
	\multicolumn{4}{l}{\textit{\normalsize{Production indices (2004-06=100)}}} \\ 
	 ~ Net food & 39 ~ \ \ & 83 ~ \ \ & \textit{196} ~ \ \ \\ 
	 ~ Net crop & 45 ~ \ \ & 82 ~ \ \ & \textit{182} ~ \ \ \\ 
	 ~ Cereal & 34 ~ \ \ & 90 ~ \ \ & \textit{203} ~ \ \ \\ 
	 ~ Vegetable oils & 16 ~ \ \ & 86 ~ \ \ & \textit{245} ~ \ \ \\ 
	 ~ Roots and tubers & 12 ~ \ \ & 66 ~ \ \ & \textit{214} ~ \ \ \\ 
	 ~ Fruit and vegetables & 55 ~ \ \ & 88 ~ \ \ & \textit{121} ~ \ \ \\ 
	 ~ Sugar & 82 ~ \ \ & 112 ~ \ \ & \textit{125} ~ \ \ \\ 
	 ~ Livestock & 72 ~ \ \ & 86 ~ \ \ & \textit{253} ~ \ \ \\ 
	 ~ Milk & 128 ~ \ \ & 49 ~ \ \ & \textit{318} ~ \ \ \\ 
	 ~ Meat & 68 ~ \ \ & 87 ~ \ \ & \textit{266} ~ \ \ \\ 
	 ~ Fish  & 109 ~ \ \ & 66 ~ \ \ & \textit{181} ~ \ \ \\ 
	\multicolumn{4}{l}{\textit{\normalsize{Net trade (min US\$)}}} \\ 
	 ~ Cereals & -128 ~ \ \ & -117 ~ \ \ & \textit{-94} ~ \ \ \\ 
	 ~ Fruit and vegetables & -5 ~ \ \ & 6 ~ \ \ & \textit{44} ~ \ \ \\ 
	 ~ Meat & -1 ~ \ \ & -1 ~ \ \ & \textit{-1} ~ \ \ \\ 
	 ~ Dairy products & -6 ~ \ \ & -7 ~ \ \ & \textit{-8} ~ \ \ \\ 
	 ~ Fish & -1 ~ \ \ & 0 ~ \ \ & \textit{-1} ~ \ \ \\ 
	\multicolumn{4}{l}{\textcolor{FAOblue}{\textbf{\large{Environment}}}} \\ 
	 ~ Forest area (\%) & 41 ~ \ \ & 37 ~ \ \ & \textit{34} ~ \ \ \\ 
	 ~ Renewable water res withdrawn (\% of total) &  ~ \ \ & \textit{86} ~ \ \ & 86 ~ \ \ \\ 
	 ~ Terrestrial protect areas (\% total land area)  & 15 ~ \ \ & 15 ~ \ \ & \textit{18} ~ \ \ \\ 
	 ~ Organic area (\% total agricultural area) &  ~ \ \ & \textit{0} ~ \ \ & \textit{0} ~ \ \ \\ 
	 ~ Water withdrawal by agriculture (\% of total) &  ~ \ \ & \textit{86} ~ \ \ & 86 ~ \ \ \\ 
	 ~ Biofuel production (thousand kt of oil eq.) & 5 ~ \ \ & 7 ~ \ \ & \textit{7} ~ \ \ \\ 
	 ~ Wood pellet prod. (min tonnes) &  ~ \ \ &  ~ \ \ &  ~ \ \ \\ 
	 ~ GHG emissions from ag (Co2 eq, gigagrams) & 8 ~ \ \ & 9 ~ \ \ & \textit{11} ~ \ \ \\ 
       \toprule
      \end{tabular}
      \clearpage
\CountryData{ Malaysia }
      \rowcolors{1}{FAOblue!10}{white}
      \begin{tabular}{L{3.9cm} R{1cm} R{1cm} R{1cm}}
      \toprule
      \multicolumn{1}{c}{} & \multicolumn{1}{c}{ 1992 } & \multicolumn{1}{c}{ 2002 } & \multicolumn{1}{c}{ 2014 } \\
      \midrule
	\multicolumn{4}{l}{\textcolor{FAOblue}{\textbf{\large{The setting}}}} \\ 
	 ~ Population, total (mln) & 19.2 ~ \ \ & 24.4 ~ \ \ & 30.2 ~ \ \ \\ 
	 ~ Population, rural (\% total population) & 9.3 ~ \ \ & 8.7 ~ \ \ & 7.6 ~ \ \ \\ 
	 ~ Govt expenditure on ag (\% total outlays) &  ~ \ \ & 2.7 ~ \ \ & \textit{2.6} ~ \ \ \\ 
	 ~ Area harvested (mln ha) & 33 ~ \ \ & 60 ~ \ \ & 100 ~ \ \ \\ 
	 ~ Cropping intensity ratio (\%) & 4.8 ~ \ \ & 8.5 ~ \ \ &  ~ \ \ \\ 
	 ~ Water resources (m\textsuperscript{3}/person/year) & \textit{29} ~ \ \ & \textit{23} ~ \ \ & \textit{20} ~ \ \ \\ 
	 ~ Area equipped for irrigation (1000 ha) &  ~ \ \ &  ~ \ \ & \textit{380} ~ \ \ \\ 
	 ~ Area irrigated (\%) & \textit{100} ~ \ \ &  ~ \ \ &  ~ \ \ \\ 
	 ~ Employment in agriculture (\%) & 21.8 ~ \ \ & 14.9 ~ \ \ & \textit{12.6} ~ \ \ \\ 
	 ~ Employment in agriculture, female (\%) & 20.1 ~ \ \ & 11.6 ~ \ \ & \textit{8.2} ~ \ \ \\ 
	 ~ Fertilizers, Nitrogen (nutrients per ha) &  ~ \ \ & 64.4 ~ \ \ & \textit{72} ~ \ \ \\ 
	 ~ Fertilizers, Phosphate (nutrients per ha) &  ~ \ \ & 4.8 ~ \ \ & \textit{28.8} ~ \ \ \\ 
	 ~ Fertilizers, Potash (nutrients per ha) &  ~ \ \ & 86 ~ \ \ & \textit{94.7} ~ \ \ \\ 
	 ~ Energy consump, power irrigation (mln kWh) & 0 ~ \ \ & 0 ~ \ \ & \textit{17} ~ \ \ \\ 
	 ~ Agr value added per worker (constant US\$) & 5.2 ~ \ \ & 5.7 ~ \ \ & \textit{9.7} ~ \ \ \\ 
	\multicolumn{4}{l}{\textcolor{FAOblue}{\textbf{\large{Hunger dimensions}}}} \\ 
	 ~ Dietary energy supply (kcal/pc/day) & 2\,740 ~ \ \ & 2\,798 ~ \ \ & 3\,018 ~ \ \ \\ 
	 ~ Average dietary energy supply adequacy (\%) & 122 ~ \ \ & 122 ~ \ \ & 128 ~ \ \ \\ 
	 ~ Dietary en supp, cereals/roots/tubers (\%) & 45 ~ \ \ & 45 ~ \ \ & \textit{45} ~ \ \ \\ 
	 ~ Prevalence of undernourishment (\%) & <5.0 ~ \ \ & <5.0 ~ \ \ & <5.0 ~ \ \ \\ 
	 ~ GDP per capita (US\$, PPP) & 11\,491 ~ \ \ & 15\,950 ~ \ \ & \textit{22\,589} ~ \ \ \\ 
	 ~ Domestic food price volatility (index) &  ~ \ \ & 4.9 ~ \ \ & 4.3 ~ \ \ \\ 
	 ~ Cereal import dependency ratio (\%) & 70.6 ~ \ \ & 76.2 ~ \ \ & \textit{76} ~ \ \ \\ 
	 ~ Underweight, children under-5 (\%) & 22.6 ~ \ \ & \textit{16.7} ~ \ \ & \textit{12.9} ~ \ \ \\ 
	 ~ Improved water source (\% pop) & 89.9 ~ \ \ & 97.9 ~ \ \ & \textit{99.6} ~ \ \ \\ 
	\multicolumn{4}{l}{\textcolor{FAOblue}{\textbf{\large{Food Supply}}}} \\ 
	 ~ Food production value, (2004-2006 mln I\$) & 6\,460 ~ \ \ & 9\,425 ~ \ \ & \textit{14\,311} ~ \ \ \\ 
	 ~ Agriculture, value added (\% GDP) & 15 ~ \ \ & 9 ~ \ \ & \textit{9} ~ \ \ \\ 
	 ~ Food exports (mln US\$)  & 3\,543 ~ \ \ & 5\,643 ~ \ \ & \textit{21\,754} ~ \ \ \\ 
	 ~ Food imports (mln US\$)  & 1\,982 ~ \ \ & 3\,145 ~ \ \ & \textit{12\,373} ~ \ \ \\ 
	\multicolumn{4}{l}{\textit{\normalsize{Production indices (2004-06=100)}}} \\ 
	 ~ Net food & 58 ~ \ \ & 84 ~ \ \ & \textit{128} ~ \ \ \\ 
	 ~ Net crop & 60 ~ \ \ & 82 ~ \ \ & \textit{118} ~ \ \ \\ 
	 ~ Cereal & 88 ~ \ \ & 97 ~ \ \ & \textit{117} ~ \ \ \\ 
	 ~ Vegetable oils & 43 ~ \ \ & 80 ~ \ \ & \textit{128} ~ \ \ \\ 
	 ~ Roots and tubers & 343 ~ \ \ & 128 ~ \ \ & \textit{118} ~ \ \ \\ 
	 ~ Fruit and vegetables & 86 ~ \ \ & 97 ~ \ \ & \textit{113} ~ \ \ \\ 
	 ~ Sugar & 273 ~ \ \ & 207 ~ \ \ & \textit{44} ~ \ \ \\ 
	 ~ Livestock & 74 ~ \ \ & 88 ~ \ \ & \textit{139} ~ \ \ \\ 
	 ~ Milk & 82 ~ \ \ & 85 ~ \ \ & \textit{174} ~ \ \ \\ 
	 ~ Meat & 73 ~ \ \ & 87 ~ \ \ & \textit{136} ~ \ \ \\ 
	 ~ Fish  & 76 ~ \ \ & 99 ~ \ \ & \textit{121} ~ \ \ \\ 
	\multicolumn{4}{l}{\textit{\normalsize{Net trade (min US\$)}}} \\ 
	 ~ Cereals & -544 ~ \ \ & -589 ~ \ \ & \textit{-1\,571} ~ \ \ \\ 
	 ~ Fruit and vegetables & -112 ~ \ \ & -250 ~ \ \ & \textit{-995} ~ \ \ \\ 
	 ~ Meat & -81 ~ \ \ & -202 ~ \ \ & \textit{-614} ~ \ \ \\ 
	 ~ Dairy products & -204 ~ \ \ & -254 ~ \ \ & \textit{-534} ~ \ \ \\ 
	 ~ Fish & 50 ~ \ \ & -9 ~ \ \ & \textit{-223} ~ \ \ \\ 
	\multicolumn{4}{l}{\textcolor{FAOblue}{\textbf{\large{Environment}}}} \\ 
	 ~ Forest area (\%) & 68 ~ \ \ & 65 ~ \ \ & \textit{62} ~ \ \ \\ 
	 ~ Renewable water res withdrawn (\% of total) &  ~ \ \ & \textit{22} ~ \ \ & 22 ~ \ \ \\ 
	 ~ Terrestrial protect areas (\% total land area)  & 18 ~ \ \ & 18 ~ \ \ & \textit{18} ~ \ \ \\ 
	 ~ Organic area (\% total agricultural area) &  ~ \ \ & \textit{0} ~ \ \ & \textit{0} ~ \ \ \\ 
	 ~ Water withdrawal by agriculture (\% of total) &  ~ \ \ & \textit{22} ~ \ \ & 22 ~ \ \ \\ 
	 ~ Biofuel production (thousand kt of oil eq.) & 3 ~ \ \ & 234 ~ \ \ & \textit{230} ~ \ \ \\ 
	 ~ Wood pellet prod. (min tonnes) &  ~ \ \ &  ~ \ \ & \textit{85} ~ \ \ \\ 
	 ~ GHG emissions from ag (Co2 eq, gigagrams) & -227 ~ \ \ & 197 ~ \ \ & \textit{160} ~ \ \ \\ 
       \toprule
      \end{tabular}
      \clearpage
\CountryData{ Maldives }
      \rowcolors{1}{FAOblue!10}{white}
      \begin{tabular}{L{3.9cm} R{1cm} R{1cm} R{1cm}}
      \toprule
      \multicolumn{1}{c}{} & \multicolumn{1}{c}{ 1992 } & \multicolumn{1}{c}{ 2002 } & \multicolumn{1}{c}{ 2014 } \\
      \midrule
	\multicolumn{4}{l}{\textcolor{FAOblue}{\textbf{\large{The setting}}}} \\ 
	 ~ Population, total (mln) & 0.2 ~ \ \ & 0.3 ~ \ \ & 0.4 ~ \ \ \\ 
	 ~ Population, rural (\% total population) & 0.2 ~ \ \ & 0.2 ~ \ \ & 0.2 ~ \ \ \\ 
	 ~ Govt expenditure on ag (\% total outlays) &  ~ \ \ & 0.6 ~ \ \ & \textit{1.3} ~ \ \ \\ 
	 ~ Area harvested (mln ha) & 0 ~ \ \ & 0 ~ \ \ & 0 ~ \ \ \\ 
	 ~ Cropping intensity ratio (\%) & 18.9 ~ \ \ & 25 ~ \ \ &  ~ \ \ \\ 
	 ~ Water resources (m\textsuperscript{3}/person/year) & \textit{0} ~ \ \ & \textit{0} ~ \ \ & \textit{0} ~ \ \ \\ 
	 ~ Area equipped for irrigation (1000 ha) &  ~ \ \ &  ~ \ \ &  ~ \ \ \\ 
	 ~ Area irrigated (\%) &  ~ \ \ &  ~ \ \ &  ~ \ \ \\ 
	 ~ Employment in agriculture (\%) & \textit{22.2} ~ \ \ & \textit{17.3} ~ \ \ & \textit{11.5} ~ \ \ \\ 
	 ~ Employment in agriculture, female (\%) & \textit{9} ~ \ \ & \textit{9} ~ \ \ & \textit{7.1} ~ \ \ \\ 
	 ~ Fertilizers, Nitrogen (nutrients per ha) &  ~ \ \ & 0.8 ~ \ \ & \textit{21.4} ~ \ \ \\ 
	 ~ Fertilizers, Phosphate (nutrients per ha) &  ~ \ \ & 0.4 ~ \ \ & \textit{18.7} ~ \ \ \\ 
	 ~ Fertilizers, Potash (nutrients per ha) &  ~ \ \ & 0.3 ~ \ \ & \textit{19} ~ \ \ \\ 
	 ~ Energy consump, power irrigation (mln kWh) &  ~ \ \ &  ~ \ \ &  ~ \ \ \\ 
	 ~ Agr value added per worker (constant US\$) &  ~ \ \ & 3.2 ~ \ \ & \textit{3.2} ~ \ \ \\ 
	\multicolumn{4}{l}{\textcolor{FAOblue}{\textbf{\large{Hunger dimensions}}}} \\ 
	 ~ Dietary energy supply (kcal/pc/day) & 2\,383 ~ \ \ & 2\,451 ~ \ \ & 2\,879 ~ \ \ \\ 
	 ~ Average dietary energy supply adequacy (\%) & 117 ~ \ \ & 114 ~ \ \ & 128 ~ \ \ \\ 
	 ~ Dietary en supp, cereals/roots/tubers (\%) & 48 ~ \ \ & 44 ~ \ \ & \textit{41} ~ \ \ \\ 
	 ~ Prevalence of undernourishment (\%) & 11.4 ~ \ \ & 12.3 ~ \ \ & 5.9 ~ \ \ \\ 
	 ~ GDP per capita (US\$, PPP) &  ~ \ \ & 6\,856 ~ \ \ & \textit{11\,283} ~ \ \ \\ 
	 ~ Domestic food price volatility (index) &  ~ \ \ & 26.5 ~ \ \ & \textit{14.2} ~ \ \ \\ 
	 ~ Cereal import dependency ratio (\%) & 100 ~ \ \ & 100 ~ \ \ & \textit{100} ~ \ \ \\ 
	 ~ Underweight, children under-5 (\%) & \textit{39} ~ \ \ & \textit{25.7} ~ \ \ & \textit{17.8} ~ \ \ \\ 
	 ~ Improved water source (\% pop) & 93.2 ~ \ \ & 95.9 ~ \ \ & \textit{98.6} ~ \ \ \\ 
	\multicolumn{4}{l}{\textcolor{FAOblue}{\textbf{\large{Food Supply}}}} \\ 
	 ~ Food production value, (2004-2006 mln I\$) & 9 ~ \ \ & 9 ~ \ \ & \textit{7} ~ \ \ \\ 
	 ~ Agriculture, value added (\% GDP) & \textit{11} ~ \ \ & 7 ~ \ \ & \textit{4} ~ \ \ \\ 
	 ~ Food exports (mln US\$)  & 0 ~ \ \ & 0 ~ \ \ & \textit{0} ~ \ \ \\ 
	 ~ Food imports (mln US\$)  & 22 ~ \ \ & 70 ~ \ \ & \textit{256} ~ \ \ \\ 
	\multicolumn{4}{l}{\textit{\normalsize{Production indices (2004-06=100)}}} \\ 
	 ~ Net food & 76 ~ \ \ & 78 ~ \ \ & \textit{64} ~ \ \ \\ 
	 ~ Net crop & 72 ~ \ \ & 75 ~ \ \ & \textit{59} ~ \ \ \\ 
	 ~ Cereal & 7 ~ \ \ & 83 ~ \ \ & \textit{146} ~ \ \ \\ 
	 ~ Vegetable oils & 74 ~ \ \ & 126 ~ \ \ & \textit{3} ~ \ \ \\ 
	 ~ Roots and tubers & 209 ~ \ \ & 121 ~ \ \ & \textit{70} ~ \ \ \\ 
	 ~ Fruit and vegetables & 65 ~ \ \ & 52 ~ \ \ & \textit{36} ~ \ \ \\ 
	 ~ Sugar &  ~ \ \ &  ~ \ \ &  ~ \ \ \\ 
	 ~ Livestock & 111 ~ \ \ & 109 ~ \ \ & \textit{108} ~ \ \ \\ 
	 ~ Milk &  ~ \ \ &  ~ \ \ &  ~ \ \ \\ 
	 ~ Meat & 111 ~ \ \ & 109 ~ \ \ & \textit{108} ~ \ \ \\ 
	 ~ Fish  &  ~ \ \ &  ~ \ \ &  ~ \ \ \\ 
	\multicolumn{4}{l}{\textit{\normalsize{Net trade (min US\$)}}} \\ 
	 ~ Cereals &  ~ \ \ & \textit{-19} ~ \ \ & \textit{-43} ~ \ \ \\ 
	 ~ Fruit and vegetables &  ~ \ \ &  ~ \ \ &  ~ \ \ \\ 
	 ~ Meat &  ~ \ \ & -8 ~ \ \ & \textit{-40} ~ \ \ \\ 
	 ~ Dairy products &  ~ \ \ & -10 ~ \ \ & \textit{-32} ~ \ \ \\ 
	 ~ Fish & 32 ~ \ \ & 53 ~ \ \ & \textit{135} ~ \ \ \\ 
	\multicolumn{4}{l}{\textcolor{FAOblue}{\textbf{\large{Environment}}}} \\ 
	 ~ Forest area (\%) & 3 ~ \ \ & 3 ~ \ \ & \textit{3} ~ \ \ \\ 
	 ~ Renewable water res withdrawn (\% of total) &  ~ \ \ &  ~ \ \ & 0 ~ \ \ \\ 
	 ~ Terrestrial protect areas (\% total land area)  &  ~ \ \ &  ~ \ \ &  ~ \ \ \\ 
	 ~ Organic area (\% total agricultural area) &  ~ \ \ &  ~ \ \ &  ~ \ \ \\ 
	 ~ Water withdrawal by agriculture (\% of total) &  ~ \ \ &  ~ \ \ & 0 ~ \ \ \\ 
	 ~ Biofuel production (thousand kt of oil eq.) &  ~ \ \ &  ~ \ \ &  ~ \ \ \\ 
	 ~ Wood pellet prod. (min tonnes) &  ~ \ \ &  ~ \ \ &  ~ \ \ \\ 
	 ~ GHG emissions from ag (Co2 eq, gigagrams) & 0 ~ \ \ & 0 ~ \ \ & \textit{0} ~ \ \ \\ 
       \toprule
      \end{tabular}
      \clearpage
\CountryData{ Mali }
      \rowcolors{1}{FAOblue!10}{white}
      \begin{tabular}{L{3.9cm} R{1cm} R{1cm} R{1cm}}
      \toprule
      \multicolumn{1}{c}{} & \multicolumn{1}{c}{ 1992 } & \multicolumn{1}{c}{ 2002 } & \multicolumn{1}{c}{ 2014 } \\
      \midrule
	\multicolumn{4}{l}{\textcolor{FAOblue}{\textbf{\large{The setting}}}} \\ 
	 ~ Population, total (mln) & 8.3 ~ \ \ & 10.9 ~ \ \ & 15.8 ~ \ \ \\ 
	 ~ Population, rural (\% total population) & 6.3 ~ \ \ & 7.7 ~ \ \ & 10 ~ \ \ \\ 
	 ~ Govt expenditure on ag (\% total outlays) &  ~ \ \ &  ~ \ \ &  ~ \ \ \\ 
	 ~ Area harvested (mln ha) & 2 ~ \ \ & 3 ~ \ \ & 6 ~ \ \ \\ 
	 ~ Cropping intensity ratio (\%) & 0.1 ~ \ \ & 0.1 ~ \ \ &  ~ \ \ \\ 
	 ~ Water resources (m\textsuperscript{3}/person/year) & \textit{14} ~ \ \ & \textit{11} ~ \ \ & \textit{8} ~ \ \ \\ 
	 ~ Area equipped for irrigation (1000 ha) &  ~ \ \ &  ~ \ \ & \textit{378} ~ \ \ \\ 
	 ~ Area irrigated (\%) &  ~ \ \ & \textit{74.6} ~ \ \ &  ~ \ \ \\ 
	 ~ Employment in agriculture (\%) &  ~ \ \ & \textit{41.5} ~ \ \ & \textit{66} ~ \ \ \\ 
	 ~ Employment in agriculture, female (\%) &  ~ \ \ & \textit{29.9} ~ \ \ & \textit{63.9} ~ \ \ \\ 
	 ~ Fertilizers, Nitrogen (nutrients per ha) &  ~ \ \ & 0 ~ \ \ & \textit{3} ~ \ \ \\ 
	 ~ Fertilizers, Phosphate (nutrients per ha) &  ~ \ \ & 0 ~ \ \ & \textit{0.8} ~ \ \ \\ 
	 ~ Fertilizers, Potash (nutrients per ha) &  ~ \ \ & 0 ~ \ \ & \textit{0.5} ~ \ \ \\ 
	 ~ Energy consump, power irrigation (mln kWh) & 0 ~ \ \ & 0 ~ \ \ & \textit{0} ~ \ \ \\ 
	 ~ Agr value added per worker (constant US\$) & 0.6 ~ \ \ & 0.6 ~ \ \ & \textit{0.8} ~ \ \ \\ 
	\multicolumn{4}{l}{\textcolor{FAOblue}{\textbf{\large{Hunger dimensions}}}} \\ 
	 ~ Dietary energy supply (kcal/pc/day) & 2\,357 ~ \ \ & 2\,507 ~ \ \ & 2\,829 ~ \ \ \\ 
	 ~ Average dietary energy supply adequacy (\%) & 113 ~ \ \ & 120 ~ \ \ & 136 ~ \ \ \\ 
	 ~ Dietary en supp, cereals/roots/tubers (\%) & 70 ~ \ \ & 67 ~ \ \ & \textit{68} ~ \ \ \\ 
	 ~ Prevalence of undernourishment (\%) & 17.3 ~ \ \ & 11.9 ~ \ \ & <5.0 ~ \ \ \\ 
	 ~ GDP per capita (US\$, PPP) & 1\,159 ~ \ \ & 1\,397 ~ \ \ & \textit{1\,589} ~ \ \ \\ 
	 ~ Domestic food price volatility (index) &  ~ \ \ & 8.7 ~ \ \ & 9.4 ~ \ \ \\ 
	 ~ Cereal import dependency ratio (\%) & 3.9 ~ \ \ & 10.2 ~ \ \ & \textit{4.7} ~ \ \ \\ 
	 ~ Underweight, children under-5 (\%) &  ~ \ \ & \textit{30.1} ~ \ \ & \textit{27.9} ~ \ \ \\ 
	 ~ Improved water source (\% pop) & 31.5 ~ \ \ & 49.1 ~ \ \ & \textit{67.2} ~ \ \ \\ 
	\multicolumn{4}{l}{\textcolor{FAOblue}{\textbf{\large{Food Supply}}}} \\ 
	 ~ Food production value, (2004-2006 mln I\$) & 1\,224 ~ \ \ & 1\,752 ~ \ \ & \textit{3\,288} ~ \ \ \\ 
	 ~ Agriculture, value added (\% GDP) & 46 ~ \ \ & 35 ~ \ \ & \textit{42} ~ \ \ \\ 
	 ~ Food exports (mln US\$)  & 119 ~ \ \ & 53 ~ \ \ & \textit{150} ~ \ \ \\ 
	 ~ Food imports (mln US\$)  & 90 ~ \ \ & 140 ~ \ \ & \textit{382} ~ \ \ \\ 
	\multicolumn{4}{l}{\textit{\normalsize{Production indices (2004-06=100)}}} \\ 
	 ~ Net food & 56 ~ \ \ & 80 ~ \ \ & \textit{151} ~ \ \ \\ 
	 ~ Net crop & 59 ~ \ \ & 84 ~ \ \ & \textit{139} ~ \ \ \\ 
	 ~ Cereal & 52 ~ \ \ & 76 ~ \ \ & \textit{183} ~ \ \ \\ 
	 ~ Vegetable oils & 55 ~ \ \ & 83 ~ \ \ & \textit{119} ~ \ \ \\ 
	 ~ Roots and tubers & 18 ~ \ \ & 83 ~ \ \ & \textit{143} ~ \ \ \\ 
	 ~ Fruit and vegetables & 60 ~ \ \ & 90 ~ \ \ & \textit{117} ~ \ \ \\ 
	 ~ Sugar & 81 ~ \ \ & 99 ~ \ \ & \textit{107} ~ \ \ \\ 
	 ~ Livestock & 57 ~ \ \ & 79 ~ \ \ & \textit{147} ~ \ \ \\ 
	 ~ Milk & 39 ~ \ \ & 57 ~ \ \ & \textit{137} ~ \ \ \\ 
	 ~ Meat & 65 ~ \ \ & 90 ~ \ \ & \textit{151} ~ \ \ \\ 
	 ~ Fish  & 68 ~ \ \ & 100 ~ \ \ & \textit{101} ~ \ \ \\ 
	\multicolumn{4}{l}{\textit{\normalsize{Net trade (min US\$)}}} \\ 
	 ~ Cereals & -22 ~ \ \ & -65 ~ \ \ & \textit{-191} ~ \ \ \\ 
	 ~ Fruit and vegetables & -10 ~ \ \ & -3 ~ \ \ & \textit{-1} ~ \ \ \\ 
	 ~ Meat & -1 ~ \ \ & 0 ~ \ \ & \textit{-1} ~ \ \ \\ 
	 ~ Dairy products & -24 ~ \ \ & -11 ~ \ \ & \textit{-29} ~ \ \ \\ 
	 ~ Fish & -2 ~ \ \ & 0 ~ \ \ & \textit{-11} ~ \ \ \\ 
	\multicolumn{4}{l}{\textcolor{FAOblue}{\textbf{\large{Environment}}}} \\ 
	 ~ Forest area (\%) & 11 ~ \ \ & 11 ~ \ \ & \textit{10} ~ \ \ \\ 
	 ~ Renewable water res withdrawn (\% of total) &  ~ \ \ &  ~ \ \ & 98 ~ \ \ \\ 
	 ~ Terrestrial protect areas (\% total land area)  & 2 ~ \ \ & 2 ~ \ \ & \textit{6} ~ \ \ \\ 
	 ~ Organic area (\% total agricultural area) &  ~ \ \ & \textit{0} ~ \ \ & \textit{0} ~ \ \ \\ 
	 ~ Water withdrawal by agriculture (\% of total) &  ~ \ \ &  ~ \ \ & 98 ~ \ \ \\ 
	 ~ Biofuel production (thousand kt of oil eq.) & 1 ~ \ \ & 1 ~ \ \ & \textit{1} ~ \ \ \\ 
	 ~ Wood pellet prod. (min tonnes) &  ~ \ \ &  ~ \ \ &  ~ \ \ \\ 
	 ~ GHG emissions from ag (Co2 eq, gigagrams) & 23 ~ \ \ & 27 ~ \ \ & \textit{34} ~ \ \ \\ 
       \toprule
      \end{tabular}
      \clearpage
\CountryData{ Malta }
      \rowcolors{1}{FAOblue!10}{white}
      \begin{tabular}{L{3.9cm} R{1cm} R{1cm} R{1cm}}
      \toprule
      \multicolumn{1}{c}{} & \multicolumn{1}{c}{ 1992 } & \multicolumn{1}{c}{ 2002 } & \multicolumn{1}{c}{ 2014 } \\
      \midrule
	\multicolumn{4}{l}{\textcolor{FAOblue}{\textbf{\large{The setting}}}} \\ 
	 ~ Population, total (mln) & 0.4 ~ \ \ & 0.4 ~ \ \ & 0.4 ~ \ \ \\ 
	 ~ Population, rural (\% total population) & 0 ~ \ \ & 0 ~ \ \ & 0 ~ \ \ \\ 
	 ~ Govt expenditure on ag (\% total outlays) &  ~ \ \ &  ~ \ \ &  ~ \ \ \\ 
	 ~ Area harvested (mln ha) & 0 ~ \ \ & 0 ~ \ \ & 0 ~ \ \ \\ 
	 ~ Cropping intensity ratio (\%) & 24.8 ~ \ \ & 38.5 ~ \ \ &  ~ \ \ \\ 
	 ~ Water resources (m\textsuperscript{3}/person/year) & \textit{0} ~ \ \ & \textit{0} ~ \ \ & \textit{0} ~ \ \ \\ 
	 ~ Area equipped for irrigation (1000 ha) &  ~ \ \ &  ~ \ \ & \textit{4} ~ \ \ \\ 
	 ~ Area irrigated (\%) &  ~ \ \ &  ~ \ \ & \textit{87.8} ~ \ \ \\ 
	 ~ Employment in agriculture (\%) &  ~ \ \ & 1.9 ~ \ \ & \textit{1} ~ \ \ \\ 
	 ~ Employment in agriculture, female (\%) &  ~ \ \ & 0 ~ \ \ & \textit{0.5} ~ \ \ \\ 
	 ~ Fertilizers, Nitrogen (nutrients per ha) &  ~ \ \ & 59.8 ~ \ \ & \textit{191.2} ~ \ \ \\ 
	 ~ Fertilizers, Phosphate (nutrients per ha) &  ~ \ \ & 15.6 ~ \ \ & \textit{0} ~ \ \ \\ 
	 ~ Fertilizers, Potash (nutrients per ha) &  ~ \ \ & 17.5 ~ \ \ & \textit{0.3} ~ \ \ \\ 
	 ~ Energy consump, power irrigation (mln kWh) & 1 ~ \ \ & 1 ~ \ \ & \textit{1} ~ \ \ \\ 
	 ~ Agr value added per worker (constant US\$) & 35 ~ \ \ & 69.3 ~ \ \ & \textit{56.2} ~ \ \ \\ 
	\multicolumn{4}{l}{\textcolor{FAOblue}{\textbf{\large{Hunger dimensions}}}} \\ 
	 ~ Dietary energy supply (kcal/pc/day) &  ~ \ \ &  ~ \ \ &  ~ \ \ \\ 
	 ~ Average dietary energy supply adequacy (\%) & 127 ~ \ \ & 133 ~ \ \ & 135 ~ \ \ \\ 
	 ~ Dietary en supp, cereals/roots/tubers (\%) & 35 ~ \ \ & 37 ~ \ \ & \textit{35} ~ \ \ \\ 
	 ~ Prevalence of undernourishment (\%) & <5.0 ~ \ \ & <5.0 ~ \ \ & <5.0 ~ \ \ \\ 
	 ~ GDP per capita (US\$, PPP) & 18\,100 ~ \ \ & 25\,191 ~ \ \ & \textit{28\,822} ~ \ \ \\ 
	 ~ Domestic food price volatility (index) &  ~ \ \ & 12.5 ~ \ \ & 8.6 ~ \ \ \\ 
	 ~ Cereal import dependency ratio (\%) & 95.4 ~ \ \ & 93.9 ~ \ \ & \textit{89.3} ~ \ \ \\ 
	 ~ Underweight, children under-5 (\%) &  ~ \ \ &  ~ \ \ &  ~ \ \ \\ 
	 ~ Improved water source (\% pop) & 99.9 ~ \ \ & 100 ~ \ \ & \textit{100} ~ \ \ \\ 
	\multicolumn{4}{l}{\textcolor{FAOblue}{\textbf{\large{Food Supply}}}} \\ 
	 ~ Food production value, (2004-2006 mln I\$) & 67 ~ \ \ & 82 ~ \ \ & \textit{72} ~ \ \ \\ 
	 ~ Agriculture, value added (\% GDP) & 3 ~ \ \ & 3 ~ \ \ & \textit{2} ~ \ \ \\ 
	 ~ Food exports (mln US\$)  & 27 ~ \ \ & 48 ~ \ \ & \textit{121} ~ \ \ \\ 
	 ~ Food imports (mln US\$)  & 177 ~ \ \ & 214 ~ \ \ & \textit{470} ~ \ \ \\ 
	\multicolumn{4}{l}{\textit{\normalsize{Production indices (2004-06=100)}}} \\ 
	 ~ Net food & 84 ~ \ \ & 105 ~ \ \ & \textit{92} ~ \ \ \\ 
	 ~ Net crop & 85 ~ \ \ & 90 ~ \ \ & \textit{99} ~ \ \ \\ 
	 ~ Cereal & 54 ~ \ \ & 92 ~ \ \ & \textit{134} ~ \ \ \\ 
	 ~ Vegetable oils & 375 ~ \ \ & 75 ~ \ \ & \textit{125} ~ \ \ \\ 
	 ~ Roots and tubers & 117 ~ \ \ & 135 ~ \ \ & \textit{61} ~ \ \ \\ 
	 ~ Fruit and vegetables & 82 ~ \ \ & 84 ~ \ \ & \textit{100} ~ \ \ \\ 
	 ~ Sugar &  ~ \ \ &  ~ \ \ &  ~ \ \ \\ 
	 ~ Livestock & 84 ~ \ \ & 115 ~ \ \ & \textit{86} ~ \ \ \\ 
	 ~ Milk & 63 ~ \ \ & 112 ~ \ \ & \textit{99} ~ \ \ \\ 
	 ~ Meat & 90 ~ \ \ & 123 ~ \ \ & \textit{82} ~ \ \ \\ 
	 ~ Fish  & 44 ~ \ \ & 89 ~ \ \ & \textit{255} ~ \ \ \\ 
	\multicolumn{4}{l}{\textit{\normalsize{Net trade (min US\$)}}} \\ 
	 ~ Cereals & -36 ~ \ \ & -49 ~ \ \ & \textit{-78} ~ \ \ \\ 
	 ~ Fruit and vegetables & -23 ~ \ \ & -39 ~ \ \ & \textit{-78} ~ \ \ \\ 
	 ~ Meat & -35 ~ \ \ & -32 ~ \ \ & \textit{-96} ~ \ \ \\ 
	 ~ Dairy products & -25 ~ \ \ & -25 ~ \ \ & \textit{-53} ~ \ \ \\ 
	 ~ Fish & -19 ~ \ \ & 21 ~ \ \ & \textit{15} ~ \ \ \\ 
	\multicolumn{4}{l}{\textcolor{FAOblue}{\textbf{\large{Environment}}}} \\ 
	 ~ Forest area (\%) & 1 ~ \ \ & 1 ~ \ \ & \textit{1} ~ \ \ \\ 
	 ~ Renewable water res withdrawn (\% of total) &  ~ \ \ & 35 ~ \ \ & 35 ~ \ \ \\ 
	 ~ Terrestrial protect areas (\% total land area)  & 0 ~ \ \ & 16 ~ \ \ & \textit{22} ~ \ \ \\ 
	 ~ Organic area (\% total agricultural area) &  ~ \ \ & \textit{0} ~ \ \ & \textit{0} ~ \ \ \\ 
	 ~ Water withdrawal by agriculture (\% of total) &  ~ \ \ & 35 ~ \ \ & 35 ~ \ \ \\ 
	 ~ Biofuel production (thousand kt of oil eq.) &  ~ \ \ &  ~ \ \ &  ~ \ \ \\ 
	 ~ Wood pellet prod. (min tonnes) &  ~ \ \ &  ~ \ \ & \textit{0} ~ \ \ \\ 
	 ~ GHG emissions from ag (Co2 eq, gigagrams) & 0 ~ \ \ & 0 ~ \ \ & \textit{0} ~ \ \ \\ 
       \toprule
      \end{tabular}
      \clearpage
\CountryData{ Mauritania }
      \rowcolors{1}{FAOblue!10}{white}
      \begin{tabular}{L{3.9cm} R{1cm} R{1cm} R{1cm}}
      \toprule
      \multicolumn{1}{c}{} & \multicolumn{1}{c}{ 1992 } & \multicolumn{1}{c}{ 2002 } & \multicolumn{1}{c}{ 2014 } \\
      \midrule
	\multicolumn{4}{l}{\textcolor{FAOblue}{\textbf{\large{The setting}}}} \\ 
	 ~ Population, total (mln) & 2.1 ~ \ \ & 2.9 ~ \ \ & 4 ~ \ \ \\ 
	 ~ Population, rural (\% total population) & 1.3 ~ \ \ & 1.7 ~ \ \ & 2.3 ~ \ \ \\ 
	 ~ Govt expenditure on ag (\% total outlays) &  ~ \ \ &  ~ \ \ &  ~ \ \ \\ 
	 ~ Area harvested (mln ha) & 0 ~ \ \ & 0 ~ \ \ & 0 ~ \ \ \\ 
	 ~ Cropping intensity ratio (\%) & 0 ~ \ \ & 0 ~ \ \ &  ~ \ \ \\ 
	 ~ Water resources (m\textsuperscript{3}/person/year) & \textit{5} ~ \ \ & \textit{4} ~ \ \ & \textit{3} ~ \ \ \\ 
	 ~ Area equipped for irrigation (1000 ha) &  ~ \ \ &  ~ \ \ & \textit{45} ~ \ \ \\ 
	 ~ Area irrigated (\%) &  ~ \ \ & \textit{50.7} ~ \ \ &  ~ \ \ \\ 
	 ~ Employment in agriculture (\%) &  ~ \ \ &  ~ \ \ &  ~ \ \ \\ 
	 ~ Employment in agriculture, female (\%) &  ~ \ \ &  ~ \ \ &  ~ \ \ \\ 
	 ~ Fertilizers, Nitrogen (nutrients per ha) &  ~ \ \ &  ~ \ \ &  ~ \ \ \\ 
	 ~ Fertilizers, Phosphate (nutrients per ha) &  ~ \ \ &  ~ \ \ &  ~ \ \ \\ 
	 ~ Fertilizers, Potash (nutrients per ha) &  ~ \ \ &  ~ \ \ &  ~ \ \ \\ 
	 ~ Energy consump, power irrigation (mln kWh) &  ~ \ \ &  ~ \ \ &  ~ \ \ \\ 
	 ~ Agr value added per worker (constant US\$) & 1.1 ~ \ \ & 0.6 ~ \ \ & \textit{0.7} ~ \ \ \\ 
	\multicolumn{4}{l}{\textcolor{FAOblue}{\textbf{\large{Hunger dimensions}}}} \\ 
	 ~ Dietary energy supply (kcal/pc/day) & 2\,537 ~ \ \ & 2\,657 ~ \ \ & 2\,945 ~ \ \ \\ 
	 ~ Average dietary energy supply adequacy (\%) & 118 ~ \ \ & 122 ~ \ \ & 133 ~ \ \ \\ 
	 ~ Dietary en supp, cereals/roots/tubers (\%) & 56 ~ \ \ & 51 ~ \ \ & \textit{51} ~ \ \ \\ 
	 ~ Prevalence of undernourishment (\%) & 14.4 ~ \ \ & 11.1 ~ \ \ & 5.8 ~ \ \ \\ 
	 ~ GDP per capita (US\$, PPP) & 2\,411 ~ \ \ & 2\,385 ~ \ \ & \textit{2\,945} ~ \ \ \\ 
	 ~ Domestic food price volatility (index) &  ~ \ \ & 4.8 ~ \ \ & 3.1 ~ \ \ \\ 
	 ~ Cereal import dependency ratio (\%) & 71.9 ~ \ \ & 73.2 ~ \ \ & \textit{74} ~ \ \ \\ 
	 ~ Underweight, children under-5 (\%) & \textit{43.3} ~ \ \ & \textit{30.4} ~ \ \ & \textit{19.5} ~ \ \ \\ 
	 ~ Improved water source (\% pop) & 32.3 ~ \ \ & 42.5 ~ \ \ & \textit{49.6} ~ \ \ \\ 
	\multicolumn{4}{l}{\textcolor{FAOblue}{\textbf{\large{Food Supply}}}} \\ 
	 ~ Food production value, (2004-2006 mln I\$) & 329 ~ \ \ & 412 ~ \ \ & \textit{521} ~ \ \ \\ 
	 ~ Agriculture, value added (\% GDP) & 37 ~ \ \ & 26 ~ \ \ & \textit{15} ~ \ \ \\ 
	 ~ Food exports (mln US\$)  & 48 ~ \ \ & 13 ~ \ \ & \textit{19} ~ \ \ \\ 
	 ~ Food imports (mln US\$)  & 120 ~ \ \ & 171 ~ \ \ & \textit{409} ~ \ \ \\ 
	\multicolumn{4}{l}{\textit{\normalsize{Production indices (2004-06=100)}}} \\ 
	 ~ Net food & 76 ~ \ \ & 95 ~ \ \ & \textit{120} ~ \ \ \\ 
	 ~ Net crop & 66 ~ \ \ & 94 ~ \ \ & \textit{165} ~ \ \ \\ 
	 ~ Cereal & 69 ~ \ \ & 83 ~ \ \ & \textit{215} ~ \ \ \\ 
	 ~ Vegetable oils & 138 ~ \ \ & 87 ~ \ \ & \textit{103} ~ \ \ \\ 
	 ~ Roots and tubers & 78 ~ \ \ & 99 ~ \ \ & \textit{103} ~ \ \ \\ 
	 ~ Fruit and vegetables & 54 ~ \ \ & 108 ~ \ \ & \textit{89} ~ \ \ \\ 
	 ~ Sugar &  ~ \ \ &  ~ \ \ &  ~ \ \ \\ 
	 ~ Livestock & 77 ~ \ \ & 95 ~ \ \ & \textit{112} ~ \ \ \\ 
	 ~ Milk & 73 ~ \ \ & 97 ~ \ \ & \textit{110} ~ \ \ \\ 
	 ~ Meat & 79 ~ \ \ & 93 ~ \ \ & \textit{114} ~ \ \ \\ 
	 ~ Fish  &  ~ \ \ &  ~ \ \ &  ~ \ \ \\ 
	\multicolumn{4}{l}{\textit{\normalsize{Net trade (min US\$)}}} \\ 
	 ~ Cereals & -49 ~ \ \ & -61 ~ \ \ & \textit{-219} ~ \ \ \\ 
	 ~ Fruit and vegetables & -7 ~ \ \ & -13 ~ \ \ & \textit{-29} ~ \ \ \\ 
	 ~ Meat & -1 ~ \ \ & -5 ~ \ \ & \textit{-12} ~ \ \ \\ 
	 ~ Dairy products &  ~ \ \ & -20 ~ \ \ & \textit{-66} ~ \ \ \\ 
	 ~ Fish & 101 ~ \ \ & 96 ~ \ \ & \textit{340} ~ \ \ \\ 
	\multicolumn{4}{l}{\textcolor{FAOblue}{\textbf{\large{Environment}}}} \\ 
	 ~ Forest area (\%) & 0 ~ \ \ & 0 ~ \ \ & \textit{0} ~ \ \ \\ 
	 ~ Renewable water res withdrawn (\% of total) &  ~ \ \ & \textit{91} ~ \ \ & 91 ~ \ \ \\ 
	 ~ Terrestrial protect areas (\% total land area)  & 1 ~ \ \ & 1 ~ \ \ & \textit{1} ~ \ \ \\ 
	 ~ Organic area (\% total agricultural area) &  ~ \ \ &  ~ \ \ &  ~ \ \ \\ 
	 ~ Water withdrawal by agriculture (\% of total) &  ~ \ \ & \textit{91} ~ \ \ & 91 ~ \ \ \\ 
	 ~ Biofuel production (thousand kt of oil eq.) &  ~ \ \ &  ~ \ \ &  ~ \ \ \\ 
	 ~ Wood pellet prod. (min tonnes) &  ~ \ \ &  ~ \ \ &  ~ \ \ \\ 
	 ~ GHG emissions from ag (Co2 eq, gigagrams) & 6 ~ \ \ & 8 ~ \ \ & \textit{8} ~ \ \ \\ 
       \toprule
      \end{tabular}
      \clearpage
\CountryData{ Mauritius }
      \rowcolors{1}{FAOblue!10}{white}
      \begin{tabular}{L{3.9cm} R{1cm} R{1cm} R{1cm}}
      \toprule
      \multicolumn{1}{c}{} & \multicolumn{1}{c}{ 1992 } & \multicolumn{1}{c}{ 2002 } & \multicolumn{1}{c}{ 2014 } \\
      \midrule
	\multicolumn{4}{l}{\textcolor{FAOblue}{\textbf{\large{The setting}}}} \\ 
	 ~ Population, total (mln) & 1.1 ~ \ \ & 1.2 ~ \ \ & 1.2 ~ \ \ \\ 
	 ~ Population, rural (\% total population) & 0.6 ~ \ \ & 0.7 ~ \ \ & 0.7 ~ \ \ \\ 
	 ~ Govt expenditure on ag (\% total outlays) &  ~ \ \ &  ~ \ \ &  ~ \ \ \\ 
	 ~ Area harvested (mln ha) & 6 ~ \ \ & 5 ~ \ \ & 4 ~ \ \ \\ 
	 ~ Cropping intensity ratio (\%) & 53 ~ \ \ & 48.7 ~ \ \ &  ~ \ \ \\ 
	 ~ Water resources (m\textsuperscript{3}/person/year) & \textit{3} ~ \ \ & \textit{2} ~ \ \ & \textit{2} ~ \ \ \\ 
	 ~ Area equipped for irrigation (1000 ha) &  ~ \ \ &  ~ \ \ & \textit{19} ~ \ \ \\ 
	 ~ Area irrigated (\%) &  ~ \ \ & 98 ~ \ \ &  ~ \ \ \\ 
	 ~ Employment in agriculture (\%) & 14.3 ~ \ \ & 10.1 ~ \ \ & \textit{7.8} ~ \ \ \\ 
	 ~ Employment in agriculture, female (\%) & 12.7 ~ \ \ & 8.3 ~ \ \ & \textit{6.5} ~ \ \ \\ 
	 ~ Fertilizers, Nitrogen (nutrients per ha) &  ~ \ \ & 110.3 ~ \ \ & \textit{104.8} ~ \ \ \\ 
	 ~ Fertilizers, Phosphate (nutrients per ha) &  ~ \ \ & 40.1 ~ \ \ & \textit{25.9} ~ \ \ \\ 
	 ~ Fertilizers, Potash (nutrients per ha) &  ~ \ \ & 133 ~ \ \ & \textit{65.1} ~ \ \ \\ 
	 ~ Energy consump, power irrigation (mln kWh) & \textit{39} ~ \ \ & 45 ~ \ \ & \textit{45} ~ \ \ \\ 
	 ~ Agr value added per worker (constant US\$) & 4.7 ~ \ \ & 5.1 ~ \ \ & \textit{8.6} ~ \ \ \\ 
	\multicolumn{4}{l}{\textcolor{FAOblue}{\textbf{\large{Hunger dimensions}}}} \\ 
	 ~ Dietary energy supply (kcal/pc/day) & 2\,798 ~ \ \ & 2\,955 ~ \ \ & 3\,121 ~ \ \ \\ 
	 ~ Average dietary energy supply adequacy (\%) & 119 ~ \ \ & 124 ~ \ \ & 128 ~ \ \ \\ 
	 ~ Dietary en supp, cereals/roots/tubers (\%) & 48 ~ \ \ & 46 ~ \ \ & \textit{49} ~ \ \ \\ 
	 ~ Prevalence of undernourishment (\%) & 8 ~ \ \ & 6.4 ~ \ \ & <5.0 ~ \ \ \\ 
	 ~ GDP per capita (US\$, PPP) & 8\,222 ~ \ \ & 11\,716 ~ \ \ & \textit{17\,146} ~ \ \ \\ 
	 ~ Domestic food price volatility (index) &  ~ \ \ & 7.4 ~ \ \ & 11.7 ~ \ \ \\ 
	 ~ Cereal import dependency ratio (\%) & 99.1 ~ \ \ & 100 ~ \ \ & \textit{100} ~ \ \ \\ 
	 ~ Underweight, children under-5 (\%) & \textit{13} ~ \ \ & \textit{13} ~ \ \ &  ~ \ \ \\ 
	 ~ Improved water source (\% pop) & 99.2 ~ \ \ & 99.3 ~ \ \ & \textit{99.8} ~ \ \ \\ 
	\multicolumn{4}{l}{\textcolor{FAOblue}{\textbf{\large{Food Supply}}}} \\ 
	 ~ Food production value, (2004-2006 mln I\$) & 243 ~ \ \ & 244 ~ \ \ & \textit{241} ~ \ \ \\ 
	 ~ Agriculture, value added (\% GDP) & 12 ~ \ \ & 6 ~ \ \ & \textit{3} ~ \ \ \\ 
	 ~ Food exports (mln US\$)  & 384 ~ \ \ & 307 ~ \ \ & \textit{320} ~ \ \ \\ 
	 ~ Food imports (mln US\$)  & 181 ~ \ \ & 251 ~ \ \ & \textit{698} ~ \ \ \\ 
	\multicolumn{4}{l}{\textit{\normalsize{Production indices (2004-06=100)}}} \\ 
	 ~ Net food & 95 ~ \ \ & 95 ~ \ \ & \textit{94} ~ \ \ \\ 
	 ~ Net crop & 112 ~ \ \ & 97 ~ \ \ & \textit{82} ~ \ \ \\ 
	 ~ Cereal & 468 ~ \ \ & 69 ~ \ \ & \textit{430} ~ \ \ \\ 
	 ~ Vegetable oils & 253 ~ \ \ & 76 ~ \ \ & \textit{104} ~ \ \ \\ 
	 ~ Roots and tubers & 164 ~ \ \ & 118 ~ \ \ & \textit{144} ~ \ \ \\ 
	 ~ Fruit and vegetables & 62 ~ \ \ & 91 ~ \ \ & \textit{110} ~ \ \ \\ 
	 ~ Sugar & 116 ~ \ \ & 97 ~ \ \ & \textit{76} ~ \ \ \\ 
	 ~ Livestock & 51 ~ \ \ & 91 ~ \ \ & \textit{130} ~ \ \ \\ 
	 ~ Milk & 263 ~ \ \ & 100 ~ \ \ & \textit{125} ~ \ \ \\ 
	 ~ Meat & 50 ~ \ \ & 87 ~ \ \ & \textit{139} ~ \ \ \\ 
	 ~ Fish  & 192 ~ \ \ & 109 ~ \ \ & \textit{79} ~ \ \ \\ 
	\multicolumn{4}{l}{\textit{\normalsize{Net trade (min US\$)}}} \\ 
	 ~ Cereals & -49 ~ \ \ & -65 ~ \ \ & \textit{-174} ~ \ \ \\ 
	 ~ Fruit and vegetables & -23 ~ \ \ & -35 ~ \ \ & \textit{-82} ~ \ \ \\ 
	 ~ Meat & -21 ~ \ \ & -27 ~ \ \ & \textit{-72} ~ \ \ \\ 
	 ~ Dairy products & -37 ~ \ \ & -44 ~ \ \ & \textit{-116} ~ \ \ \\ 
	 ~ Fish & -11 ~ \ \ & 7 ~ \ \ & \textit{129} ~ \ \ \\ 
	\multicolumn{4}{l}{\textcolor{FAOblue}{\textbf{\large{Environment}}}} \\ 
	 ~ Forest area (\%) & 19 ~ \ \ & 18 ~ \ \ & \textit{17} ~ \ \ \\ 
	 ~ Renewable water res withdrawn (\% of total) &  ~ \ \ & \textit{68} ~ \ \ & 68 ~ \ \ \\ 
	 ~ Terrestrial protect areas (\% total land area)  & 2 ~ \ \ & 4 ~ \ \ & \textit{4} ~ \ \ \\ 
	 ~ Organic area (\% total agricultural area) &  ~ \ \ &  ~ \ \ & \textit{0} ~ \ \ \\ 
	 ~ Water withdrawal by agriculture (\% of total) &  ~ \ \ & \textit{68} ~ \ \ & 68 ~ \ \ \\ 
	 ~ Biofuel production (thousand kt of oil eq.) & 13 ~ \ \ & 12 ~ \ \ & \textit{11} ~ \ \ \\ 
	 ~ Wood pellet prod. (min tonnes) &  ~ \ \ &  ~ \ \ & \textit{0} ~ \ \ \\ 
	 ~ GHG emissions from ag (Co2 eq, gigagrams) & 0 ~ \ \ & 0 ~ \ \ & \textit{0} ~ \ \ \\ 
       \toprule
      \end{tabular}
      \clearpage
\CountryData{ Mexico }
      \rowcolors{1}{FAOblue!10}{white}
      \begin{tabular}{L{3.9cm} R{1cm} R{1cm} R{1cm}}
      \toprule
      \multicolumn{1}{c}{} & \multicolumn{1}{c}{ 1992 } & \multicolumn{1}{c}{ 2002 } & \multicolumn{1}{c}{ 2014 } \\
      \midrule
	\multicolumn{4}{l}{\textcolor{FAOblue}{\textbf{\large{The setting}}}} \\ 
	 ~ Population, total (mln) & 89.8 ~ \ \ & 106.7 ~ \ \ & 123.8 ~ \ \ \\ 
	 ~ Population, rural (\% total population) & 24.9 ~ \ \ & 26.3 ~ \ \ & 26 ~ \ \ \\ 
	 ~ Govt expenditure on ag (\% total outlays) &  ~ \ \ & \textit{3.4} ~ \ \ & \textit{3} ~ \ \ \\ 
	 ~ Area harvested (mln ha) & 42 ~ \ \ & 46 ~ \ \ & 61 ~ \ \ \\ 
	 ~ Cropping intensity ratio (\%) & 0.4 ~ \ \ & 0.4 ~ \ \ &  ~ \ \ \\ 
	 ~ Water resources (m\textsuperscript{3}/person/year) & \textit{5} ~ \ \ & \textit{4} ~ \ \ & \textit{4} ~ \ \ \\ 
	 ~ Area equipped for irrigation (1000 ha) &  ~ \ \ &  ~ \ \ & \textit{6\,500} ~ \ \ \\ 
	 ~ Area irrigated (\%) &  ~ \ \ &  ~ \ \ & \textit{86.1} ~ \ \ \\ 
	 ~ Employment in agriculture (\%) & \textit{23.8} ~ \ \ & 17.9 ~ \ \ & \textit{13.4} ~ \ \ \\ 
	 ~ Employment in agriculture, female (\%) & \textit{10.3} ~ \ \ & 6.5 ~ \ \ & \textit{3.6} ~ \ \ \\ 
	 ~ Fertilizers, Nitrogen (nutrients per ha) &  ~ \ \ & 8.3 ~ \ \ & \textit{12.7} ~ \ \ \\ 
	 ~ Fertilizers, Phosphate (nutrients per ha) &  ~ \ \ & 4 ~ \ \ & \textit{1} ~ \ \ \\ 
	 ~ Fertilizers, Potash (nutrients per ha) &  ~ \ \ & 1.9 ~ \ \ & \textit{1.9} ~ \ \ \\ 
	 ~ Energy consump, power irrigation (mln kWh) & 132 ~ \ \ & 985 ~ \ \ & \textit{985} ~ \ \ \\ 
	 ~ Agr value added per worker (constant US\$) & 2.7 ~ \ \ & 3.2 ~ \ \ & \textit{4.2} ~ \ \ \\ 
	\multicolumn{4}{l}{\textcolor{FAOblue}{\textbf{\large{Hunger dimensions}}}} \\ 
	 ~ Dietary energy supply (kcal/pc/day) & 2\,999 ~ \ \ & 3\,099 ~ \ \ & 3\,081 ~ \ \ \\ 
	 ~ Average dietary energy supply adequacy (\%) & 132 ~ \ \ & 134 ~ \ \ & 130 ~ \ \ \\ 
	 ~ Dietary en supp, cereals/roots/tubers (\%) & 47 ~ \ \ & 46 ~ \ \ & \textit{44} ~ \ \ \\ 
	 ~ Prevalence of undernourishment (\%) & 6.8 ~ \ \ & <5.0 ~ \ \ & <5.0 ~ \ \ \\ 
	 ~ GDP per capita (US\$, PPP) & 12\,926 ~ \ \ & 14\,243 ~ \ \ & \textit{16\,291} ~ \ \ \\ 
	 ~ Domestic food price volatility (index) &  ~ \ \ & 8 ~ \ \ & 4.7 ~ \ \ \\ 
	 ~ Cereal import dependency ratio (\%) & 21.6 ~ \ \ & 35.3 ~ \ \ & \textit{30.7} ~ \ \ \\ 
	 ~ Underweight, children under-5 (\%) &  ~ \ \ & \textit{6} ~ \ \ & \textit{2.8} ~ \ \ \\ 
	 ~ Improved water source (\% pop) & 83.7 ~ \ \ & 89.7 ~ \ \ & \textit{94.9} ~ \ \ \\ 
	\multicolumn{4}{l}{\textcolor{FAOblue}{\textbf{\large{Food Supply}}}} \\ 
	 ~ Food production value, (2004-2006 mln I\$) & 21\,064 ~ \ \ & 28\,251 ~ \ \ & \textit{35\,142} ~ \ \ \\ 
	 ~ Agriculture, value added (\% GDP) & 7 ~ \ \ & 4 ~ \ \ & \textit{3} ~ \ \ \\ 
	 ~ Food exports (mln US\$)  & 2\,178 ~ \ \ & 5\,257 ~ \ \ & \textit{16\,230} ~ \ \ \\ 
	 ~ Food imports (mln US\$)  & 4\,709 ~ \ \ & 9\,353 ~ \ \ & \textit{21\,503} ~ \ \ \\ 
	\multicolumn{4}{l}{\textit{\normalsize{Production indices (2004-06=100)}}} \\ 
	 ~ Net food & 69 ~ \ \ & 92 ~ \ \ & \textit{115} ~ \ \ \\ 
	 ~ Net crop & 72 ~ \ \ & 93 ~ \ \ & \textit{117} ~ \ \ \\ 
	 ~ Cereal & 87 ~ \ \ & 92 ~ \ \ & \textit{107} ~ \ \ \\ 
	 ~ Vegetable oils & 87 ~ \ \ & 71 ~ \ \ & \textit{129} ~ \ \ \\ 
	 ~ Roots and tubers & 73 ~ \ \ & 92 ~ \ \ & \textit{106} ~ \ \ \\ 
	 ~ Fruit and vegetables & 62 ~ \ \ & 92 ~ \ \ & \textit{122} ~ \ \ \\ 
	 ~ Sugar & 83 ~ \ \ & 91 ~ \ \ & \textit{122} ~ \ \ \\ 
	 ~ Livestock & 65 ~ \ \ & 91 ~ \ \ & \textit{113} ~ \ \ \\ 
	 ~ Milk & 70 ~ \ \ & 97 ~ \ \ & \textit{110} ~ \ \ \\ 
	 ~ Meat & 64 ~ \ \ & 89 ~ \ \ & \textit{113} ~ \ \ \\ 
	 ~ Fish  & 82 ~ \ \ & 105 ~ \ \ & \textit{124} ~ \ \ \\ 
	\multicolumn{4}{l}{\textit{\normalsize{Net trade (min US\$)}}} \\ 
	 ~ Cereals & -1\,179 ~ \ \ & -2\,141 ~ \ \ & \textit{-4\,463} ~ \ \ \\ 
	 ~ Fruit and vegetables & 1\,166 ~ \ \ & 2\,236 ~ \ \ & \textit{6\,934} ~ \ \ \\ 
	 ~ Meat & -828 ~ \ \ & -1\,837 ~ \ \ & \textit{-2\,427} ~ \ \ \\ 
	 ~ Dairy products & -599 ~ \ \ & -606 ~ \ \ & \textit{-1\,434} ~ \ \ \\ 
	 ~ Fish & 243 ~ \ \ & 419 ~ \ \ & \textit{416} ~ \ \ \\ 
	\multicolumn{4}{l}{\textcolor{FAOblue}{\textbf{\large{Environment}}}} \\ 
	 ~ Forest area (\%) & 36 ~ \ \ & 34 ~ \ \ & \textit{33} ~ \ \ \\ 
	 ~ Renewable water res withdrawn (\% of total) &  ~ \ \ &  ~ \ \ & 77 ~ \ \ \\ 
	 ~ Terrestrial protect areas (\% total land area)  & 2 ~ \ \ & 9 ~ \ \ & \textit{13} ~ \ \ \\ 
	 ~ Organic area (\% total agricultural area) &  ~ \ \ & \textit{0} ~ \ \ & \textit{0} ~ \ \ \\ 
	 ~ Water withdrawal by agriculture (\% of total) &  ~ \ \ &  ~ \ \ & 77 ~ \ \ \\ 
	 ~ Biofuel production (thousand kt of oil eq.) & 157 ~ \ \ & 212 ~ \ \ & \textit{219} ~ \ \ \\ 
	 ~ Wood pellet prod. (min tonnes) &  ~ \ \ &  ~ \ \ & \textit{4} ~ \ \ \\ 
	 ~ GHG emissions from ag (Co2 eq, gigagrams) & 108 ~ \ \ & 104 ~ \ \ & \textit{108} ~ \ \ \\ 
       \toprule
      \end{tabular}
      \clearpage
\CountryData{ Mongolia }
      \rowcolors{1}{FAOblue!10}{white}
      \begin{tabular}{L{3.9cm} R{1cm} R{1cm} R{1cm}}
      \toprule
      \multicolumn{1}{c}{} & \multicolumn{1}{c}{ 1992 } & \multicolumn{1}{c}{ 2002 } & \multicolumn{1}{c}{ 2014 } \\
      \midrule
	\multicolumn{4}{l}{\textcolor{FAOblue}{\textbf{\large{The setting}}}} \\ 
	 ~ Population, total (mln) & 2.2 ~ \ \ & 2.4 ~ \ \ & 2.9 ~ \ \ \\ 
	 ~ Population, rural (\% total population) & 1 ~ \ \ & 1 ~ \ \ & 0.8 ~ \ \ \\ 
	 ~ Govt expenditure on ag (\% total outlays) &  ~ \ \ & 2.5 ~ \ \ & \textit{5.9} ~ \ \ \\ 
	 ~ Area harvested (mln ha) & 1 ~ \ \ & 0 ~ \ \ & 0 ~ \ \ \\ 
	 ~ Cropping intensity ratio (\%) & 0 ~ \ \ & 0 ~ \ \ &  ~ \ \ \\ 
	 ~ Water resources (m\textsuperscript{3}/person/year) & \textit{15} ~ \ \ & \textit{14} ~ \ \ & \textit{12} ~ \ \ \\ 
	 ~ Area equipped for irrigation (1000 ha) &  ~ \ \ &  ~ \ \ & \textit{84} ~ \ \ \\ 
	 ~ Area irrigated (\%) & \textit{74.6} ~ \ \ &  ~ \ \ &  ~ \ \ \\ 
	 ~ Employment in agriculture (\%) & \textit{46.1} ~ \ \ & 44.9 ~ \ \ & \textit{32.6} ~ \ \ \\ 
	 ~ Employment in agriculture, female (\%) & \textit{45} ~ \ \ & 42.4 ~ \ \ & \textit{32.2} ~ \ \ \\ 
	 ~ Fertilizers, Nitrogen (nutrients per ha) &  ~ \ \ & 0 ~ \ \ & \textit{0.1} ~ \ \ \\ 
	 ~ Fertilizers, Phosphate (nutrients per ha) &  ~ \ \ & 0 ~ \ \ & \textit{0} ~ \ \ \\ 
	 ~ Fertilizers, Potash (nutrients per ha) &  ~ \ \ & 0 ~ \ \ & \textit{0} ~ \ \ \\ 
	 ~ Energy consump, power irrigation (mln kWh) & \textit{94} ~ \ \ & 94 ~ \ \ & \textit{94} ~ \ \ \\ 
	 ~ Agr value added per worker (constant US\$) & 2.1 ~ \ \ & 1.6 ~ \ \ & \textit{3.5} ~ \ \ \\ 
	\multicolumn{4}{l}{\textcolor{FAOblue}{\textbf{\large{Hunger dimensions}}}} \\ 
	 ~ Dietary energy supply (kcal/pc/day) & 1\,962 ~ \ \ & 2\,222 ~ \ \ & 2\,539 ~ \ \ \\ 
	 ~ Average dietary energy supply adequacy (\%) & 89 ~ \ \ & 95 ~ \ \ & 108 ~ \ \ \\ 
	 ~ Dietary en supp, cereals/roots/tubers (\%) & 44 ~ \ \ & 50 ~ \ \ & \textit{47} ~ \ \ \\ 
	 ~ Prevalence of undernourishment (\%) & 37.1 ~ \ \ & 35 ~ \ \ & 21.5 ~ \ \ \\ 
	 ~ GDP per capita (US\$, PPP) & 3\,471 ~ \ \ & 4\,138 ~ \ \ & \textit{9\,132} ~ \ \ \\ 
	 ~ Domestic food price volatility (index) &  ~ \ \ & 19.5 ~ \ \ & \textit{16.7} ~ \ \ \\ 
	 ~ Cereal import dependency ratio (\%) & 16.4 ~ \ \ & 63.7 ~ \ \ & \textit{35.1} ~ \ \ \\ 
	 ~ Underweight, children under-5 (\%) & 10.8 ~ \ \ & \textit{5.3} ~ \ \ & \textit{1.6} ~ \ \ \\ 
	 ~ Improved water source (\% pop) & 62.3 ~ \ \ & 71.5 ~ \ \ & \textit{84.6} ~ \ \ \\ 
	\multicolumn{4}{l}{\textcolor{FAOblue}{\textbf{\large{Food Supply}}}} \\ 
	 ~ Food production value, (2004-2006 mln I\$) & 777 ~ \ \ & 636 ~ \ \ & \textit{891} ~ \ \ \\ 
	 ~ Agriculture, value added (\% GDP) & 27 ~ \ \ & 21 ~ \ \ & \textit{16} ~ \ \ \\ 
	 ~ Food exports (mln US\$)  & 30 ~ \ \ & 24 ~ \ \ & \textit{42} ~ \ \ \\ 
	 ~ Food imports (mln US\$)  & 37 ~ \ \ & 102 ~ \ \ & \textit{388} ~ \ \ \\ 
	\multicolumn{4}{l}{\textit{\normalsize{Production indices (2004-06=100)}}} \\ 
	 ~ Net food & 126 ~ \ \ & 103 ~ \ \ & \textit{145} ~ \ \ \\ 
	 ~ Net crop & 149 ~ \ \ & 71 ~ \ \ & \textit{247} ~ \ \ \\ 
	 ~ Cereal & 446 ~ \ \ & 97 ~ \ \ & \textit{389} ~ \ \ \\ 
	 ~ Vegetable oils &  ~ \ \ &  ~ \ \ & \textit{109} ~ \ \ \\ 
	 ~ Roots and tubers & 67 ~ \ \ & 56 ~ \ \ & \textit{223} ~ \ \ \\ 
	 ~ Fruit and vegetables & 14 ~ \ \ & 63 ~ \ \ & \textit{177} ~ \ \ \\ 
	 ~ Sugar &  ~ \ \ &  ~ \ \ &  ~ \ \ \\ 
	 ~ Livestock & 127 ~ \ \ & 106 ~ \ \ & \textit{136} ~ \ \ \\ 
	 ~ Milk & 77 ~ \ \ & 88 ~ \ \ & \textit{134} ~ \ \ \\ 
	 ~ Meat & 141 ~ \ \ & 111 ~ \ \ & \textit{139} ~ \ \ \\ 
	 ~ Fish  &  ~ \ \ &  ~ \ \ &  ~ \ \ \\ 
	\multicolumn{4}{l}{\textit{\normalsize{Net trade (min US\$)}}} \\ 
	 ~ Cereals & -14 ~ \ \ & -51 ~ \ \ & \textit{-87} ~ \ \ \\ 
	 ~ Fruit and vegetables & \textit{-6} ~ \ \ & -11 ~ \ \ & \textit{-51} ~ \ \ \\ 
	 ~ Meat & 21 ~ \ \ & 20 ~ \ \ & \textit{-3} ~ \ \ \\ 
	 ~ Dairy products & -2 ~ \ \ & -4 ~ \ \ & \textit{-9} ~ \ \ \\ 
	 ~ Fish & \textit{0} ~ \ \ & 0 ~ \ \ & \textit{-4} ~ \ \ \\ 
	\multicolumn{4}{l}{\textcolor{FAOblue}{\textbf{\large{Environment}}}} \\ 
	 ~ Forest area (\%) & 8 ~ \ \ & 7 ~ \ \ & \textit{7} ~ \ \ \\ 
	 ~ Renewable water res withdrawn (\% of total) &  ~ \ \ &  ~ \ \ & 44 ~ \ \ \\ 
	 ~ Terrestrial protect areas (\% total land area)  & 6 ~ \ \ & 13 ~ \ \ & \textit{14} ~ \ \ \\ 
	 ~ Organic area (\% total agricultural area) &  ~ \ \ &  ~ \ \ &  ~ \ \ \\ 
	 ~ Water withdrawal by agriculture (\% of total) &  ~ \ \ &  ~ \ \ & 44 ~ \ \ \\ 
	 ~ Biofuel production (thousand kt of oil eq.) &  ~ \ \ &  ~ \ \ &  ~ \ \ \\ 
	 ~ Wood pellet prod. (min tonnes) &  ~ \ \ &  ~ \ \ &  ~ \ \ \\ 
	 ~ GHG emissions from ag (Co2 eq, gigagrams) & 41 ~ \ \ & 39 ~ \ \ & \textit{44} ~ \ \ \\ 
       \toprule
      \end{tabular}
      \clearpage
\CountryData{ Montenegro }
      \rowcolors{1}{FAOblue!10}{white}
      \begin{tabular}{L{3.9cm} R{1cm} R{1cm} R{1cm}}
      \toprule
      \multicolumn{1}{c}{} & \multicolumn{1}{c}{ 1992 } & \multicolumn{1}{c}{ 2002 } & \multicolumn{1}{c}{ 2014 } \\
      \midrule
	\multicolumn{4}{l}{\textcolor{FAOblue}{\textbf{\large{The setting}}}} \\ 
	 ~ Population, total (mln) &  ~ \ \ &  ~ \ \ & 0.6 ~ \ \ \\ 
	 ~ Population, rural (\% total population) &  ~ \ \ &  ~ \ \ & 0.2 ~ \ \ \\ 
	 ~ Govt expenditure on ag (\% total outlays) &  ~ \ \ &  ~ \ \ &  ~ \ \ \\ 
	 ~ Area harvested (mln ha) &  ~ \ \ &  ~ \ \ & 0 ~ \ \ \\ 
	 ~ Cropping intensity ratio (\%) &  ~ \ \ &  ~ \ \ &  ~ \ \ \\ 
	 ~ Water resources (m\textsuperscript{3}/person/year) &  ~ \ \ &  ~ \ \ &  ~ \ \ \\ 
	 ~ Area equipped for irrigation (1000 ha) &  ~ \ \ &  ~ \ \ & \textit{2} ~ \ \ \\ 
	 ~ Area irrigated (\%) &  ~ \ \ &  ~ \ \ &  ~ \ \ \\ 
	 ~ Employment in agriculture (\%) &  ~ \ \ & \textit{8.6} ~ \ \ & \textit{5.7} ~ \ \ \\ 
	 ~ Employment in agriculture, female (\%) &  ~ \ \ & \textit{8.9} ~ \ \ & \textit{4.8} ~ \ \ \\ 
	 ~ Fertilizers, Nitrogen (nutrients per ha) &  ~ \ \ &  ~ \ \ & \textit{2.4} ~ \ \ \\ 
	 ~ Fertilizers, Phosphate (nutrients per ha) &  ~ \ \ &  ~ \ \ & \textit{0.9} ~ \ \ \\ 
	 ~ Fertilizers, Potash (nutrients per ha) &  ~ \ \ &  ~ \ \ & \textit{0.8} ~ \ \ \\ 
	 ~ Energy consump, power irrigation (mln kWh) &  ~ \ \ &  ~ \ \ & \textit{5} ~ \ \ \\ 
	 ~ Agr value added per worker (constant US\$) &  ~ \ \ &  ~ \ \ & \textit{6.9} ~ \ \ \\ 
	\multicolumn{4}{l}{\textcolor{FAOblue}{\textbf{\large{Hunger dimensions}}}} \\ 
	 ~ Dietary energy supply (kcal/pc/day) &  ~ \ \ &  ~ \ \ &  ~ \ \ \\ 
	 ~ Average dietary energy supply adequacy (\%) &  ~ \ \ & \textit{145} ~ \ \ & 147 ~ \ \ \\ 
	 ~ Dietary en supp, cereals/roots/tubers (\%) &  ~ \ \ & \textit{40} ~ \ \ & \textit{35} ~ \ \ \\ 
	 ~ Prevalence of undernourishment (\%) & <5.0 ~ \ \ & <5.0 ~ \ \ & <5.0 ~ \ \ \\ 
	 ~ GDP per capita (US\$, PPP) &  ~ \ \ & 9\,978 ~ \ \ & \textit{14\,152} ~ \ \ \\ 
	 ~ Domestic food price volatility (index) &  ~ \ \ & \textit{4.5} ~ \ \ & \textit{9.1} ~ \ \ \\ 
	 ~ Cereal import dependency ratio (\%) &  ~ \ \ & \textit{91} ~ \ \ & \textit{89.1} ~ \ \ \\ 
	 ~ Underweight, children under-5 (\%) &  ~ \ \ & \textit{2.2} ~ \ \ & \textit{1} ~ \ \ \\ 
	 ~ Improved water source (\% pop) & 97.5 ~ \ \ & 97.9 ~ \ \ & \textit{98} ~ \ \ \\ 
	\multicolumn{4}{l}{\textcolor{FAOblue}{\textbf{\large{Food Supply}}}} \\ 
	 ~ Food production value, (2004-2006 mln I\$) &  ~ \ \ &  ~ \ \ & \textit{167} ~ \ \ \\ 
	 ~ Agriculture, value added (\% GDP) &  ~ \ \ & 12 ~ \ \ & \textit{10} ~ \ \ \\ 
	 ~ Food exports (mln US\$)  &  ~ \ \ &  ~ \ \ & \textit{38} ~ \ \ \\ 
	 ~ Food imports (mln US\$)  &  ~ \ \ &  ~ \ \ & \textit{422} ~ \ \ \\ 
	\multicolumn{4}{l}{\textit{\normalsize{Production indices (2004-06=100)}}} \\ 
	 ~ Net food &  ~ \ \ &  ~ \ \ & \textit{95} ~ \ \ \\ 
	 ~ Net crop &  ~ \ \ &  ~ \ \ & \textit{102} ~ \ \ \\ 
	 ~ Cereal &  ~ \ \ &  ~ \ \ & \textit{96} ~ \ \ \\ 
	 ~ Vegetable oils &  ~ \ \ &  ~ \ \ & \textit{154} ~ \ \ \\ 
	 ~ Roots and tubers &  ~ \ \ &  ~ \ \ & \textit{85} ~ \ \ \\ 
	 ~ Fruit and vegetables &  ~ \ \ &  ~ \ \ & \textit{106} ~ \ \ \\ 
	 ~ Sugar &  ~ \ \ &  ~ \ \ &  ~ \ \ \\ 
	 ~ Livestock &  ~ \ \ &  ~ \ \ & \textit{88} ~ \ \ \\ 
	 ~ Milk &  ~ \ \ &  ~ \ \ & \textit{100} ~ \ \ \\ 
	 ~ Meat &  ~ \ \ &  ~ \ \ & \textit{40} ~ \ \ \\ 
	 ~ Fish  & 0 ~ \ \ & 0 ~ \ \ & \textit{382} ~ \ \ \\ 
	\multicolumn{4}{l}{\textit{\normalsize{Net trade (min US\$)}}} \\ 
	 ~ Cereals &  ~ \ \ &  ~ \ \ & \textit{-70} ~ \ \ \\ 
	 ~ Fruit and vegetables &  ~ \ \ &  ~ \ \ & \textit{-39} ~ \ \ \\ 
	 ~ Meat &  ~ \ \ &  ~ \ \ & \textit{-100} ~ \ \ \\ 
	 ~ Dairy products &  ~ \ \ &  ~ \ \ & \textit{-52} ~ \ \ \\ 
	 ~ Fish &  ~ \ \ &  ~ \ \ & \textit{-14} ~ \ \ \\ 
	\multicolumn{4}{l}{\textcolor{FAOblue}{\textbf{\large{Environment}}}} \\ 
	 ~ Forest area (\%) &  ~ \ \ &  ~ \ \ & \textit{40} ~ \ \ \\ 
	 ~ Renewable water res withdrawn (\% of total) &  ~ \ \ &  ~ \ \ & 1 ~ \ \ \\ 
	 ~ Terrestrial protect areas (\% total land area)  & 13 ~ \ \ & 13 ~ \ \ & \textit{15} ~ \ \ \\ 
	 ~ Organic area (\% total agricultural area) &  ~ \ \ &  ~ \ \ & \textit{1} ~ \ \ \\ 
	 ~ Water withdrawal by agriculture (\% of total) &  ~ \ \ &  ~ \ \ & 1 ~ \ \ \\ 
	 ~ Biofuel production (thousand kt of oil eq.) &  ~ \ \ &  ~ \ \ &  ~ \ \ \\ 
	 ~ Wood pellet prod. (min tonnes) &  ~ \ \ &  ~ \ \ & \textit{3} ~ \ \ \\ 
	 ~ GHG emissions from ag (Co2 eq, gigagrams) &  ~ \ \ &  ~ \ \ & \textit{0} ~ \ \ \\ 
       \toprule
      \end{tabular}
      \clearpage
\CountryData{ Morocco }
      \rowcolors{1}{FAOblue!10}{white}
      \begin{tabular}{L{3.9cm} R{1cm} R{1cm} R{1cm}}
      \toprule
      \multicolumn{1}{c}{} & \multicolumn{1}{c}{ 1992 } & \multicolumn{1}{c}{ 2002 } & \multicolumn{1}{c}{ 2014 } \\
      \midrule
	\multicolumn{4}{l}{\textcolor{FAOblue}{\textbf{\large{The setting}}}} \\ 
	 ~ Population, total (mln) & 25.6 ~ \ \ & 29.3 ~ \ \ & 33.5 ~ \ \ \\ 
	 ~ Population, rural (\% total population) & 12.8 ~ \ \ & 13.5 ~ \ \ & 14 ~ \ \ \\ 
	 ~ Govt expenditure on ag (\% total outlays) &  ~ \ \ &  ~ \ \ &  ~ \ \ \\ 
	 ~ Area harvested (mln ha) & 5 ~ \ \ & 5 ~ \ \ & 10 ~ \ \ \\ 
	 ~ Cropping intensity ratio (\%) & 0.2 ~ \ \ & 0.2 ~ \ \ &  ~ \ \ \\ 
	 ~ Water resources (m\textsuperscript{3}/person/year) & \textit{1} ~ \ \ & \textit{1} ~ \ \ & \textit{1} ~ \ \ \\ 
	 ~ Area equipped for irrigation (1000 ha) &  ~ \ \ &  ~ \ \ & \textit{1\,458} ~ \ \ \\ 
	 ~ Area irrigated (\%) &  ~ \ \ & \textit{97.5} ~ \ \ &  ~ \ \ \\ 
	 ~ Employment in agriculture (\%) & 3.6 ~ \ \ & 44.4 ~ \ \ & \textit{39.2} ~ \ \ \\ 
	 ~ Employment in agriculture, female (\%) & 2.8 ~ \ \ & 57.1 ~ \ \ & \textit{59.2} ~ \ \ \\ 
	 ~ Fertilizers, Nitrogen (nutrients per ha) &  ~ \ \ & 8.2 ~ \ \ & \textit{5.6} ~ \ \ \\ 
	 ~ Fertilizers, Phosphate (nutrients per ha) &  ~ \ \ & 7.7 ~ \ \ & \textit{0.8} ~ \ \ \\ 
	 ~ Fertilizers, Potash (nutrients per ha) &  ~ \ \ & 2.3 ~ \ \ & \textit{1} ~ \ \ \\ 
	 ~ Energy consump, power irrigation (mln kWh) & 258 ~ \ \ & 258 ~ \ \ & \textit{998} ~ \ \ \\ 
	 ~ Agr value added per worker (constant US\$) & 1.6 ~ \ \ & 2.2 ~ \ \ & \textit{4.6} ~ \ \ \\ 
	\multicolumn{4}{l}{\textcolor{FAOblue}{\textbf{\large{Hunger dimensions}}}} \\ 
	 ~ Dietary energy supply (kcal/pc/day) & 2\,950 ~ \ \ & 3\,159 ~ \ \ & 3\,343 ~ \ \ \\ 
	 ~ Average dietary energy supply adequacy (\%) & 132 ~ \ \ & 136 ~ \ \ & 142 ~ \ \ \\ 
	 ~ Dietary en supp, cereals/roots/tubers (\%) & 64 ~ \ \ & 64 ~ \ \ & \textit{61} ~ \ \ \\ 
	 ~ Prevalence of undernourishment (\%) & 6.7 ~ \ \ & 6.5 ~ \ \ & <5.0 ~ \ \ \\ 
	 ~ GDP per capita (US\$, PPP) & 3\,930 ~ \ \ & 4\,815 ~ \ \ & \textit{6\,967} ~ \ \ \\ 
	 ~ Domestic food price volatility (index) &  ~ \ \ & 10.1 ~ \ \ & 4.9 ~ \ \ \\ 
	 ~ Cereal import dependency ratio (\%) & 37.7 ~ \ \ & 42.5 ~ \ \ & \textit{36.4} ~ \ \ \\ 
	 ~ Underweight, children under-5 (\%) & 8.1 ~ \ \ & \textit{9.9} ~ \ \ & \textit{3.1} ~ \ \ \\ 
	 ~ Improved water source (\% pop) & 74.1 ~ \ \ & 79.1 ~ \ \ & \textit{83.6} ~ \ \ \\ 
	\multicolumn{4}{l}{\textcolor{FAOblue}{\textbf{\large{Food Supply}}}} \\ 
	 ~ Food production value, (2004-2006 mln I\$) & 4\,306 ~ \ \ & 5\,735 ~ \ \ & \textit{9\,139} ~ \ \ \\ 
	 ~ Agriculture, value added (\% GDP) & 18 ~ \ \ & 17 ~ \ \ & \textit{17} ~ \ \ \\ 
	 ~ Food exports (mln US\$)  & 509 ~ \ \ & 705 ~ \ \ & \textit{1\,870} ~ \ \ \\ 
	 ~ Food imports (mln US\$)  & 809 ~ \ \ & 1\,370 ~ \ \ & \textit{4\,322} ~ \ \ \\ 
	\multicolumn{4}{l}{\textit{\normalsize{Production indices (2004-06=100)}}} \\ 
	 ~ Net food & 63 ~ \ \ & 84 ~ \ \ & \textit{134} ~ \ \ \\ 
	 ~ Net crop & 58 ~ \ \ & 80 ~ \ \ & \textit{129} ~ \ \ \\ 
	 ~ Cereal & 36 ~ \ \ & 69 ~ \ \ & \textit{136} ~ \ \ \\ 
	 ~ Vegetable oils & 74 ~ \ \ & 74 ~ \ \ & \textit{179} ~ \ \ \\ 
	 ~ Roots and tubers & 59 ~ \ \ & 87 ~ \ \ & \textit{131} ~ \ \ \\ 
	 ~ Fruit and vegetables & 58 ~ \ \ & 79 ~ \ \ & \textit{119} ~ \ \ \\ 
	 ~ Sugar & 95 ~ \ \ & 101 ~ \ \ & \textit{71} ~ \ \ \\ 
	 ~ Livestock & 73 ~ \ \ & 93 ~ \ \ & \textit{147} ~ \ \ \\ 
	 ~ Milk & 68 ~ \ \ & 87 ~ \ \ & \textit{160} ~ \ \ \\ 
	 ~ Meat & 74 ~ \ \ & 95 ~ \ \ & \textit{150} ~ \ \ \\ 
	 ~ Fish  & 59 ~ \ \ & 102 ~ \ \ & \textit{133} ~ \ \ \\ 
	\multicolumn{4}{l}{\textit{\normalsize{Net trade (min US\$)}}} \\ 
	 ~ Cereals & -413 ~ \ \ & -718 ~ \ \ & \textit{-2\,228} ~ \ \ \\ 
	 ~ Fruit and vegetables & 438 ~ \ \ & 529 ~ \ \ & \textit{1\,245} ~ \ \ \\ 
	 ~ Meat & -11 ~ \ \ & -4 ~ \ \ & \textit{-27} ~ \ \ \\ 
	 ~ Dairy products & -90 ~ \ \ & -56 ~ \ \ & \textit{-148} ~ \ \ \\ 
	 ~ Fish & 551 ~ \ \ & 927 ~ \ \ & \textit{1\,582} ~ \ \ \\ 
	\multicolumn{4}{l}{\textcolor{FAOblue}{\textbf{\large{Environment}}}} \\ 
	 ~ Forest area (\%) & 11 ~ \ \ & 11 ~ \ \ & \textit{12} ~ \ \ \\ 
	 ~ Renewable water res withdrawn (\% of total) &  ~ \ \ &  ~ \ \ & 88 ~ \ \ \\ 
	 ~ Terrestrial protect areas (\% total land area)  & 1 ~ \ \ & 2 ~ \ \ & \textit{22} ~ \ \ \\ 
	 ~ Organic area (\% total agricultural area) &  ~ \ \ & \textit{0} ~ \ \ & \textit{0} ~ \ \ \\ 
	 ~ Water withdrawal by agriculture (\% of total) &  ~ \ \ &  ~ \ \ & 88 ~ \ \ \\ 
	 ~ Biofuel production (thousand kt of oil eq.) & 2 ~ \ \ & 3 ~ \ \ & \textit{2} ~ \ \ \\ 
	 ~ Wood pellet prod. (min tonnes) &  ~ \ \ &  ~ \ \ &  ~ \ \ \\ 
	 ~ GHG emissions from ag (Co2 eq, gigagrams) & 2 ~ \ \ & 3 ~ \ \ & \textit{14} ~ \ \ \\ 
       \toprule
      \end{tabular}
      \clearpage
\CountryData{ Mozambique }
      \rowcolors{1}{FAOblue!10}{white}
      \begin{tabular}{L{3.9cm} R{1cm} R{1cm} R{1cm}}
      \toprule
      \multicolumn{1}{c}{} & \multicolumn{1}{c}{ 1992 } & \multicolumn{1}{c}{ 2002 } & \multicolumn{1}{c}{ 2014 } \\
      \midrule
	\multicolumn{4}{l}{\textcolor{FAOblue}{\textbf{\large{The setting}}}} \\ 
	 ~ Population, total (mln) & 14.3 ~ \ \ & 19.3 ~ \ \ & 26.5 ~ \ \ \\ 
	 ~ Population, rural (\% total population) & 11 ~ \ \ & 13.6 ~ \ \ & 18 ~ \ \ \\ 
	 ~ Govt expenditure on ag (\% total outlays) &  ~ \ \ &  ~ \ \ &  ~ \ \ \\ 
	 ~ Area harvested (mln ha) & 3 ~ \ \ & 6 ~ \ \ & 11 ~ \ \ \\ 
	 ~ Cropping intensity ratio (\%) & 0.1 ~ \ \ & 0.1 ~ \ \ &  ~ \ \ \\ 
	 ~ Water resources (m\textsuperscript{3}/person/year) & \textit{15} ~ \ \ & \textit{11} ~ \ \ & \textit{8} ~ \ \ \\ 
	 ~ Area equipped for irrigation (1000 ha) &  ~ \ \ &  ~ \ \ & \textit{118} ~ \ \ \\ 
	 ~ Area irrigated (\%) &  ~ \ \ & \textit{33.9} ~ \ \ &  ~ \ \ \\ 
	 ~ Employment in agriculture (\%) &  ~ \ \ & \textit{80.5} ~ \ \ &  ~ \ \ \\ 
	 ~ Employment in agriculture, female (\%) &  ~ \ \ & \textit{89.9} ~ \ \ &  ~ \ \ \\ 
	 ~ Fertilizers, Nitrogen (nutrients per ha) &  ~ \ \ & 0.4 ~ \ \ & \textit{0.5} ~ \ \ \\ 
	 ~ Fertilizers, Phosphate (nutrients per ha) &  ~ \ \ & 0 ~ \ \ & \textit{0.1} ~ \ \ \\ 
	 ~ Fertilizers, Potash (nutrients per ha) &  ~ \ \ & 0.1 ~ \ \ & \textit{0.1} ~ \ \ \\ 
	 ~ Energy consump, power irrigation (mln kWh) &  ~ \ \ &  ~ \ \ &  ~ \ \ \\ 
	 ~ Agr value added per worker (constant US\$) & 0.1 ~ \ \ & 0.2 ~ \ \ & \textit{0.3} ~ \ \ \\ 
	\multicolumn{4}{l}{\textcolor{FAOblue}{\textbf{\large{Hunger dimensions}}}} \\ 
	 ~ Dietary energy supply (kcal/pc/day) & 1\,700 ~ \ \ & 2\,040 ~ \ \ & 2\,306 ~ \ \ \\ 
	 ~ Average dietary energy supply adequacy (\%) & 81 ~ \ \ & 97 ~ \ \ & 109 ~ \ \ \\ 
	 ~ Dietary en supp, cereals/roots/tubers (\%) & 75 ~ \ \ & 76 ~ \ \ & \textit{72} ~ \ \ \\ 
	 ~ Prevalence of undernourishment (\%) & 58.8 ~ \ \ & 41.3 ~ \ \ & 26.2 ~ \ \ \\ 
	 ~ GDP per capita (US\$, PPP) & 434 ~ \ \ & 671 ~ \ \ & \textit{1\,070} ~ \ \ \\ 
	 ~ Domestic food price volatility (index) &  ~ \ \ & 6.6 ~ \ \ & \textit{6.7} ~ \ \ \\ 
	 ~ Cereal import dependency ratio (\%) & 61.3 ~ \ \ & 32.2 ~ \ \ & \textit{27.3} ~ \ \ \\ 
	 ~ Underweight, children under-5 (\%) & \textit{23.9} ~ \ \ & \textit{21.2} ~ \ \ & \textit{15.6} ~ \ \ \\ 
	 ~ Improved water source (\% pop) & 34.6 ~ \ \ & 42.4 ~ \ \ & \textit{49.2} ~ \ \ \\ 
	\multicolumn{4}{l}{\textcolor{FAOblue}{\textbf{\large{Food Supply}}}} \\ 
	 ~ Food production value, (2004-2006 mln I\$) & 847 ~ \ \ & 1\,660 ~ \ \ & \textit{2\,828} ~ \ \ \\ 
	 ~ Agriculture, value added (\% GDP) & 35 ~ \ \ & 28 ~ \ \ & \textit{29} ~ \ \ \\ 
	 ~ Food exports (mln US\$)  & 41 ~ \ \ & 35 ~ \ \ & \textit{297} ~ \ \ \\ 
	 ~ Food imports (mln US\$)  & 270 ~ \ \ & 246 ~ \ \ & \textit{632} ~ \ \ \\ 
	\multicolumn{4}{l}{\textit{\normalsize{Production indices (2004-06=100)}}} \\ 
	 ~ Net food & 48 ~ \ \ & 95 ~ \ \ & \textit{162} ~ \ \ \\ 
	 ~ Net crop & 45 ~ \ \ & 90 ~ \ \ & \textit{164} ~ \ \ \\ 
	 ~ Cereal & 17 ~ \ \ & 97 ~ \ \ & \textit{173} ~ \ \ \\ 
	 ~ Vegetable oils & 73 ~ \ \ & 93 ~ \ \ & \textit{188} ~ \ \ \\ 
	 ~ Roots and tubers & 55 ~ \ \ & 103 ~ \ \ & \textit{176} ~ \ \ \\ 
	 ~ Fruit and vegetables & 60 ~ \ \ & 74 ~ \ \ & \textit{192} ~ \ \ \\ 
	 ~ Sugar & 8 ~ \ \ & 77 ~ \ \ & \textit{184} ~ \ \ \\ 
	 ~ Livestock & 47 ~ \ \ & 104 ~ \ \ & \textit{129} ~ \ \ \\ 
	 ~ Milk & 87 ~ \ \ & 96 ~ \ \ & \textit{103} ~ \ \ \\ 
	 ~ Meat & 44 ~ \ \ & 107 ~ \ \ & \textit{122} ~ \ \ \\ 
	 ~ Fish  & 32 ~ \ \ & 39 ~ \ \ & \textit{222} ~ \ \ \\ 
	\multicolumn{4}{l}{\textit{\normalsize{Net trade (min US\$)}}} \\ 
	 ~ Cereals & -190 ~ \ \ & -131 ~ \ \ & \textit{-370} ~ \ \ \\ 
	 ~ Fruit and vegetables & -3 ~ \ \ & 10 ~ \ \ & \textit{36} ~ \ \ \\ 
	 ~ Meat & -3 ~ \ \ & -12 ~ \ \ & \textit{-29} ~ \ \ \\ 
	 ~ Dairy products & -15 ~ \ \ & -6 ~ \ \ & \textit{-44} ~ \ \ \\ 
	 ~ Fish & 56 ~ \ \ & 112 ~ \ \ & \textit{-23} ~ \ \ \\ 
	\multicolumn{4}{l}{\textcolor{FAOblue}{\textbf{\large{Environment}}}} \\ 
	 ~ Forest area (\%) & 55 ~ \ \ & 52 ~ \ \ & \textit{49} ~ \ \ \\ 
	 ~ Renewable water res withdrawn (\% of total) &  ~ \ \ & \textit{78} ~ \ \ & 78 ~ \ \ \\ 
	 ~ Terrestrial protect areas (\% total land area)  & 15 ~ \ \ & 16 ~ \ \ & \textit{18} ~ \ \ \\ 
	 ~ Organic area (\% total agricultural area) &  ~ \ \ & \textit{0} ~ \ \ & \textit{0} ~ \ \ \\ 
	 ~ Water withdrawal by agriculture (\% of total) &  ~ \ \ & \textit{78} ~ \ \ & 78 ~ \ \ \\ 
	 ~ Biofuel production (thousand kt of oil eq.) & 1 ~ \ \ & 4 ~ \ \ & \textit{7} ~ \ \ \\ 
	 ~ Wood pellet prod. (min tonnes) &  ~ \ \ &  ~ \ \ &  ~ \ \ \\ 
	 ~ GHG emissions from ag (Co2 eq, gigagrams) & 51 ~ \ \ & 50 ~ \ \ & \textit{49} ~ \ \ \\ 
       \toprule
      \end{tabular}
      \clearpage
\CountryData{ Myanmar }
      \rowcolors{1}{FAOblue!10}{white}
      \begin{tabular}{L{3.9cm} R{1cm} R{1cm} R{1cm}}
      \toprule
      \multicolumn{1}{c}{} & \multicolumn{1}{c}{ 1992 } & \multicolumn{1}{c}{ 2002 } & \multicolumn{1}{c}{ 2014 } \\
      \midrule
	\multicolumn{4}{l}{\textcolor{FAOblue}{\textbf{\large{The setting}}}} \\ 
	 ~ Population, total (mln) & 43.4 ~ \ \ & 49.3 ~ \ \ & 53.7 ~ \ \ \\ 
	 ~ Population, rural (\% total population) & 32.6 ~ \ \ & 35.5 ~ \ \ & 35.3 ~ \ \ \\ 
	 ~ Govt expenditure on ag (\% total outlays) &  ~ \ \ & 10.5 ~ \ \ & \textit{6.3} ~ \ \ \\ 
	 ~ Area harvested (mln ha) & 15 ~ \ \ & 23 ~ \ \ & 30 ~ \ \ \\ 
	 ~ Cropping intensity ratio (\%) & 1.5 ~ \ \ & 2.1 ~ \ \ &  ~ \ \ \\ 
	 ~ Water resources (m\textsuperscript{3}/person/year) & \textit{27} ~ \ \ & \textit{24} ~ \ \ & \textit{22} ~ \ \ \\ 
	 ~ Area equipped for irrigation (1000 ha) &  ~ \ \ &  ~ \ \ & \textit{2\,295} ~ \ \ \\ 
	 ~ Area irrigated (\%) &  ~ \ \ & \textit{100} ~ \ \ &  ~ \ \ \\ 
	 ~ Employment in agriculture (\%) & 69.1 ~ \ \ & \textit{62.7} ~ \ \ &  ~ \ \ \\ 
	 ~ Employment in agriculture, female (\%) &  ~ \ \ &  ~ \ \ &  ~ \ \ \\ 
	 ~ Fertilizers, Nitrogen (nutrients per ha) &  ~ \ \ & 3 ~ \ \ & \textit{10.5} ~ \ \ \\ 
	 ~ Fertilizers, Phosphate (nutrients per ha) &  ~ \ \ & 0.8 ~ \ \ & \textit{1.7} ~ \ \ \\ 
	 ~ Fertilizers, Potash (nutrients per ha) &  ~ \ \ & 0.1 ~ \ \ & \textit{1.3} ~ \ \ \\ 
	 ~ Energy consump, power irrigation (mln kWh) & \textit{0} ~ \ \ & 0 ~ \ \ & \textit{0} ~ \ \ \\ 
	 ~ Agr value added per worker (constant US\$) &  ~ \ \ &  ~ \ \ &  ~ \ \ \\ 
	\multicolumn{4}{l}{\textcolor{FAOblue}{\textbf{\large{Hunger dimensions}}}} \\ 
	 ~ Dietary energy supply (kcal/pc/day) & 1\,708 ~ \ \ & 2\,006 ~ \ \ & 2\,598 ~ \ \ \\ 
	 ~ Average dietary energy supply adequacy (\%) & 78 ~ \ \ & 88 ~ \ \ & 112 ~ \ \ \\ 
	 ~ Dietary en supp, cereals/roots/tubers (\%) & 69 ~ \ \ & 66 ~ \ \ & \textit{51} ~ \ \ \\ 
	 ~ Prevalence of undernourishment (\%) & 62.7 ~ \ \ & 46.3 ~ \ \ & 14.9 ~ \ \ \\ 
	 ~ GDP per capita (US\$, PPP) &  ~ \ \ &  ~ \ \ &  ~ \ \ \\ 
	 ~ Domestic food price volatility (index) &  ~ \ \ & 6.4 ~ \ \ & \textit{8.1} ~ \ \ \\ 
	 ~ Cereal import dependency ratio (\%) & -2.7 ~ \ \ & -5.8 ~ \ \ & \textit{-2.9} ~ \ \ \\ 
	 ~ Underweight, children under-5 (\%) & \textit{38.7} ~ \ \ & \textit{29.6} ~ \ \ & \textit{22.6} ~ \ \ \\ 
	 ~ Improved water source (\% pop) & 55.7 ~ \ \ & 70.1 ~ \ \ & \textit{85.7} ~ \ \ \\ 
	\multicolumn{4}{l}{\textcolor{FAOblue}{\textbf{\large{Food Supply}}}} \\ 
	 ~ Food production value, (2004-2006 mln I\$) & 5\,924 ~ \ \ & 10\,368 ~ \ \ & \textit{16\,517} ~ \ \ \\ 
	 ~ Agriculture, value added (\% GDP) & 61 ~ \ \ & 55 ~ \ \ &  ~ \ \ \\ 
	 ~ Food exports (mln US\$)  & 194 ~ \ \ & 492 ~ \ \ & \textit{1\,584} ~ \ \ \\ 
	 ~ Food imports (mln US\$)  & 103 ~ \ \ & 191 ~ \ \ & \textit{898} ~ \ \ \\ 
	\multicolumn{4}{l}{\textit{\normalsize{Production indices (2004-06=100)}}} \\ 
	 ~ Net food & 47 ~ \ \ & 82 ~ \ \ & \textit{130} ~ \ \ \\ 
	 ~ Net crop & 46 ~ \ \ & 79 ~ \ \ & \textit{121} ~ \ \ \\ 
	 ~ Cereal & 52 ~ \ \ & 78 ~ \ \ & \textit{105} ~ \ \ \\ 
	 ~ Vegetable oils & 31 ~ \ \ & 71 ~ \ \ & \textit{141} ~ \ \ \\ 
	 ~ Roots and tubers & 33 ~ \ \ & 72 ~ \ \ & \textit{179} ~ \ \ \\ 
	 ~ Fruit and vegetables & 55 ~ \ \ & 89 ~ \ \ & \textit{120} ~ \ \ \\ 
	 ~ Sugar & 33 ~ \ \ & 84 ~ \ \ & \textit{130} ~ \ \ \\ 
	 ~ Livestock & 27 ~ \ \ & 63 ~ \ \ & \textit{192} ~ \ \ \\ 
	 ~ Milk & 54 ~ \ \ & 65 ~ \ \ & \textit{172} ~ \ \ \\ 
	 ~ Meat & 23 ~ \ \ & 62 ~ \ \ & \textit{193} ~ \ \ \\ 
	 ~ Fish  & 33 ~ \ \ & 65 ~ \ \ & \textit{208} ~ \ \ \\ 
	\multicolumn{4}{l}{\textit{\normalsize{Net trade (min US\$)}}} \\ 
	 ~ Cereals & 49 ~ \ \ & 122 ~ \ \ & \textit{-24} ~ \ \ \\ 
	 ~ Fruit and vegetables & 99 ~ \ \ & 304 ~ \ \ & \textit{1\,345} ~ \ \ \\ 
	 ~ Meat & 0 ~ \ \ & -1 ~ \ \ & \textit{-16} ~ \ \ \\ 
	 ~ Dairy products &  ~ \ \ &  ~ \ \ &  ~ \ \ \\ 
	 ~ Fish & 10 ~ \ \ & 251 ~ \ \ & \textit{638} ~ \ \ \\ 
	\multicolumn{4}{l}{\textcolor{FAOblue}{\textbf{\large{Environment}}}} \\ 
	 ~ Forest area (\%) & 59 ~ \ \ & 52 ~ \ \ & \textit{48} ~ \ \ \\ 
	 ~ Renewable water res withdrawn (\% of total) &  ~ \ \ & \textit{89} ~ \ \ & 89 ~ \ \ \\ 
	 ~ Terrestrial protect areas (\% total land area)  & 3 ~ \ \ & 6 ~ \ \ & \textit{7} ~ \ \ \\ 
	 ~ Organic area (\% total agricultural area) &  ~ \ \ &  ~ \ \ & \textit{0} ~ \ \ \\ 
	 ~ Water withdrawal by agriculture (\% of total) &  ~ \ \ & \textit{89} ~ \ \ & 89 ~ \ \ \\ 
	 ~ Biofuel production (thousand kt of oil eq.) & 3 ~ \ \ & 27 ~ \ \ & \textit{27} ~ \ \ \\ 
	 ~ Wood pellet prod. (min tonnes) &  ~ \ \ &  ~ \ \ &  ~ \ \ \\ 
	 ~ GHG emissions from ag (Co2 eq, gigagrams) & 151 ~ \ \ & 129 ~ \ \ & \textit{153} ~ \ \ \\ 
       \toprule
      \end{tabular}
      \clearpage
\CountryData{ Namibia }
      \rowcolors{1}{FAOblue!10}{white}
      \begin{tabular}{L{3.9cm} R{1cm} R{1cm} R{1cm}}
      \toprule
      \multicolumn{1}{c}{} & \multicolumn{1}{c}{ 1992 } & \multicolumn{1}{c}{ 2002 } & \multicolumn{1}{c}{ 2014 } \\
      \midrule
	\multicolumn{4}{l}{\textcolor{FAOblue}{\textbf{\large{The setting}}}} \\ 
	 ~ Population, total (mln) & 1.5 ~ \ \ & 2 ~ \ \ & 2.3 ~ \ \ \\ 
	 ~ Population, rural (\% total population) & 1.1 ~ \ \ & 1.3 ~ \ \ & 1.4 ~ \ \ \\ 
	 ~ Govt expenditure on ag (\% total outlays) &  ~ \ \ & 5.2 ~ \ \ & \textit{6.5} ~ \ \ \\ 
	 ~ Area harvested (mln ha) & 0 ~ \ \ & 0 ~ \ \ & 0 ~ \ \ \\ 
	 ~ Cropping intensity ratio (\%) & 0 ~ \ \ & 0 ~ \ \ &  ~ \ \ \\ 
	 ~ Water resources (m\textsuperscript{3}/person/year) & \textit{26} ~ \ \ & \textit{20} ~ \ \ & \textit{17} ~ \ \ \\ 
	 ~ Area equipped for irrigation (1000 ha) &  ~ \ \ &  ~ \ \ & \textit{8} ~ \ \ \\ 
	 ~ Area irrigated (\%) & 100 ~ \ \ &  ~ \ \ &  ~ \ \ \\ 
	 ~ Employment in agriculture (\%) & \textit{48.2} ~ \ \ & \textit{29.9} ~ \ \ & \textit{27.4} ~ \ \ \\ 
	 ~ Employment in agriculture, female (\%) & \textit{52} ~ \ \ & \textit{25.2} ~ \ \ & \textit{26.6} ~ \ \ \\ 
	 ~ Fertilizers, Nitrogen (nutrients per ha) &  ~ \ \ & 0.1 ~ \ \ & \textit{0.1} ~ \ \ \\ 
	 ~ Fertilizers, Phosphate (nutrients per ha) &  ~ \ \ & 0 ~ \ \ & \textit{0} ~ \ \ \\ 
	 ~ Fertilizers, Potash (nutrients per ha) &  ~ \ \ & 0 ~ \ \ & \textit{0} ~ \ \ \\ 
	 ~ Energy consump, power irrigation (mln kWh) & 8 ~ \ \ & 11 ~ \ \ & \textit{11} ~ \ \ \\ 
	 ~ Agr value added per worker (constant US\$) & 1.9 ~ \ \ & 2.7 ~ \ \ & \textit{2} ~ \ \ \\ 
	\multicolumn{4}{l}{\textcolor{FAOblue}{\textbf{\large{Hunger dimensions}}}} \\ 
	 ~ Dietary energy supply (kcal/pc/day) & 2\,034 ~ \ \ & 2\,245 ~ \ \ & 2\,043 ~ \ \ \\ 
	 ~ Average dietary energy supply adequacy (\%) & 92 ~ \ \ & 99 ~ \ \ & 88 ~ \ \ \\ 
	 ~ Dietary en supp, cereals/roots/tubers (\%) & 62 ~ \ \ & 60 ~ \ \ & \textit{55} ~ \ \ \\ 
	 ~ Prevalence of undernourishment (\%) & 36.7 ~ \ \ & 25.6 ~ \ \ & 42.3 ~ \ \ \\ 
	 ~ GDP per capita (US\$, PPP) & 6\,197 ~ \ \ & 6\,278 ~ \ \ & \textit{9\,276} ~ \ \ \\ 
	 ~ Domestic food price volatility (index) &  ~ \ \ & 6.1 ~ \ \ & \textit{7.2} ~ \ \ \\ 
	 ~ Cereal import dependency ratio (\%) & 69.9 ~ \ \ & 60.9 ~ \ \ & \textit{55.9} ~ \ \ \\ 
	 ~ Underweight, children under-5 (\%) & 21.5 ~ \ \ & \textit{20.3} ~ \ \ & \textit{13.2} ~ \ \ \\ 
	 ~ Improved water source (\% pop) & 69.6 ~ \ \ & 81.4 ~ \ \ & \textit{91.7} ~ \ \ \\ 
	\multicolumn{4}{l}{\textcolor{FAOblue}{\textbf{\large{Food Supply}}}} \\ 
	 ~ Food production value, (2004-2006 mln I\$) & 387 ~ \ \ & 374 ~ \ \ & \textit{406} ~ \ \ \\ 
	 ~ Agriculture, value added (\% GDP) & 8 ~ \ \ & 11 ~ \ \ & \textit{6} ~ \ \ \\ 
	 ~ Food exports (mln US\$)  & 197 ~ \ \ & 97 ~ \ \ & \textit{362} ~ \ \ \\ 
	 ~ Food imports (mln US\$)  & 119 ~ \ \ & 122 ~ \ \ & \textit{599} ~ \ \ \\ 
	\multicolumn{4}{l}{\textit{\normalsize{Production indices (2004-06=100)}}} \\ 
	 ~ Net food & 86 ~ \ \ & 84 ~ \ \ & \textit{91} ~ \ \ \\ 
	 ~ Net crop & 41 ~ \ \ & 79 ~ \ \ & \textit{107} ~ \ \ \\ 
	 ~ Cereal & 18 ~ \ \ & 66 ~ \ \ & \textit{54} ~ \ \ \\ 
	 ~ Vegetable oils & 64 ~ \ \ & 67 ~ \ \ & \textit{38} ~ \ \ \\ 
	 ~ Roots and tubers & 61 ~ \ \ & 91 ~ \ \ & \textit{114} ~ \ \ \\ 
	 ~ Fruit and vegetables & 18 ~ \ \ & 74 ~ \ \ & \textit{153} ~ \ \ \\ 
	 ~ Sugar &  ~ \ \ &  ~ \ \ &  ~ \ \ \\ 
	 ~ Livestock & 100 ~ \ \ & 86 ~ \ \ & \textit{84} ~ \ \ \\ 
	 ~ Milk & 70 ~ \ \ & 94 ~ \ \ & \textit{107} ~ \ \ \\ 
	 ~ Meat & 105 ~ \ \ & 85 ~ \ \ & \textit{81} ~ \ \ \\ 
	 ~ Fish  & 120 ~ \ \ & 115 ~ \ \ & \textit{89} ~ \ \ \\ 
	\multicolumn{4}{l}{\textit{\normalsize{Net trade (min US\$)}}} \\ 
	 ~ Cereals &  ~ \ \ & -25 ~ \ \ & \textit{-122} ~ \ \ \\ 
	 ~ Fruit and vegetables &  ~ \ \ & -11 ~ \ \ & \textit{-76} ~ \ \ \\ 
	 ~ Meat & 92 ~ \ \ & 11 ~ \ \ & \textit{127} ~ \ \ \\ 
	 ~ Dairy products &  ~ \ \ & -5 ~ \ \ & \textit{-43} ~ \ \ \\ 
	 ~ Fish & 54 ~ \ \ & 282 ~ \ \ & \textit{717} ~ \ \ \\ 
	\multicolumn{4}{l}{\textcolor{FAOblue}{\textbf{\large{Environment}}}} \\ 
	 ~ Forest area (\%) & 10 ~ \ \ & 10 ~ \ \ & \textit{9} ~ \ \ \\ 
	 ~ Renewable water res withdrawn (\% of total) &  ~ \ \ & 70 ~ \ \ & 70 ~ \ \ \\ 
	 ~ Terrestrial protect areas (\% total land area)  & 14 ~ \ \ & 14 ~ \ \ & \textit{43} ~ \ \ \\ 
	 ~ Organic area (\% total agricultural area) &  ~ \ \ &  ~ \ \ & \textit{0} ~ \ \ \\ 
	 ~ Water withdrawal by agriculture (\% of total) &  ~ \ \ & 70 ~ \ \ & 70 ~ \ \ \\ 
	 ~ Biofuel production (thousand kt of oil eq.) &  ~ \ \ &  ~ \ \ &  ~ \ \ \\ 
	 ~ Wood pellet prod. (min tonnes) &  ~ \ \ &  ~ \ \ &  ~ \ \ \\ 
	 ~ GHG emissions from ag (Co2 eq, gigagrams) & 15 ~ \ \ & 14 ~ \ \ & \textit{19} ~ \ \ \\ 
       \toprule
      \end{tabular}
      \clearpage
\CountryData{ Nepal }
      \rowcolors{1}{FAOblue!10}{white}
      \begin{tabular}{L{3.9cm} R{1cm} R{1cm} R{1cm}}
      \toprule
      \multicolumn{1}{c}{} & \multicolumn{1}{c}{ 1992 } & \multicolumn{1}{c}{ 2002 } & \multicolumn{1}{c}{ 2014 } \\
      \midrule
	\multicolumn{4}{l}{\textcolor{FAOblue}{\textbf{\large{The setting}}}} \\ 
	 ~ Population, total (mln) & 19.1 ~ \ \ & 24.1 ~ \ \ & 28.1 ~ \ \ \\ 
	 ~ Population, rural (\% total population) & 17.2 ~ \ \ & 20.7 ~ \ \ & 23.1 ~ \ \ \\ 
	 ~ Govt expenditure on ag (\% total outlays) &  ~ \ \ & 5.9 ~ \ \ & \textit{8.5} ~ \ \ \\ 
	 ~ Area harvested (mln ha) & 5 ~ \ \ & 7 ~ \ \ & 9 ~ \ \ \\ 
	 ~ Cropping intensity ratio (\%) & 1.2 ~ \ \ & 1.7 ~ \ \ &  ~ \ \ \\ 
	 ~ Water resources (m\textsuperscript{3}/person/year) & \textit{11} ~ \ \ & \textit{9} ~ \ \ & \textit{8} ~ \ \ \\ 
	 ~ Area equipped for irrigation (1000 ha) &  ~ \ \ &  ~ \ \ & \textit{1\,332} ~ \ \ \\ 
	 ~ Area irrigated (\%) &  ~ \ \ &  ~ \ \ &  ~ \ \ \\ 
	 ~ Employment in agriculture (\%) & \textit{81.2} ~ \ \ & \textit{65.7} ~ \ \ &  ~ \ \ \\ 
	 ~ Employment in agriculture, female (\%) & \textit{90.5} ~ \ \ & \textit{72.8} ~ \ \ &  ~ \ \ \\ 
	 ~ Fertilizers, Nitrogen (nutrients per ha) &  ~ \ \ & 5.1 ~ \ \ & \textit{12} ~ \ \ \\ 
	 ~ Fertilizers, Phosphate (nutrients per ha) &  ~ \ \ & 3.6 ~ \ \ & \textit{2.2} ~ \ \ \\ 
	 ~ Fertilizers, Potash (nutrients per ha) &  ~ \ \ & 0.4 ~ \ \ & \textit{0.4} ~ \ \ \\ 
	 ~ Energy consump, power irrigation (mln kWh) & \textit{0} ~ \ \ & 0 ~ \ \ & \textit{0} ~ \ \ \\ 
	 ~ Agr value added per worker (constant US\$) & 0.3 ~ \ \ & 0.3 ~ \ \ & \textit{0.3} ~ \ \ \\ 
	\multicolumn{4}{l}{\textcolor{FAOblue}{\textbf{\large{Hunger dimensions}}}} \\ 
	 ~ Dietary energy supply (kcal/pc/day) & 2\,211 ~ \ \ & 2\,298 ~ \ \ & 2\,670 ~ \ \ \\ 
	 ~ Average dietary energy supply adequacy (\%) & 105 ~ \ \ & 109 ~ \ \ & 122 ~ \ \ \\ 
	 ~ Dietary en supp, cereals/roots/tubers (\%) & 76 ~ \ \ & 73 ~ \ \ & \textit{69} ~ \ \ \\ 
	 ~ Prevalence of undernourishment (\%) & 23.4 ~ \ \ & 21.1 ~ \ \ & 7.7 ~ \ \ \\ 
	 ~ GDP per capita (US\$, PPP) & 1\,306 ~ \ \ & 1\,591 ~ \ \ & \textit{2\,173} ~ \ \ \\ 
	 ~ Domestic food price volatility (index) &  ~ \ \ & 10.9 ~ \ \ & 10.2 ~ \ \ \\ 
	 ~ Cereal import dependency ratio (\%) & 0.8 ~ \ \ & 0.9 ~ \ \ & \textit{3.9} ~ \ \ \\ 
	 ~ Underweight, children under-5 (\%) & \textit{44.1} ~ \ \ & \textit{43} ~ \ \ & \textit{29.1} ~ \ \ \\ 
	 ~ Improved water source (\% pop) & 68.6 ~ \ \ & 79 ~ \ \ & \textit{88.1} ~ \ \ \\ 
	\multicolumn{4}{l}{\textcolor{FAOblue}{\textbf{\large{Food Supply}}}} \\ 
	 ~ Food production value, (2004-2006 mln I\$) & 2\,781 ~ \ \ & 3\,892 ~ \ \ & \textit{5\,553} ~ \ \ \\ 
	 ~ Agriculture, value added (\% GDP) & 45 ~ \ \ & 39 ~ \ \ & \textit{35} ~ \ \ \\ 
	 ~ Food exports (mln US\$)  & 40 ~ \ \ & 128 ~ \ \ & \textit{170} ~ \ \ \\ 
	 ~ Food imports (mln US\$)  & 73 ~ \ \ & 209 ~ \ \ & \textit{829} ~ \ \ \\ 
	\multicolumn{4}{l}{\textit{\normalsize{Production indices (2004-06=100)}}} \\ 
	 ~ Net food & 65 ~ \ \ & 91 ~ \ \ & \textit{130} ~ \ \ \\ 
	 ~ Net crop & 62 ~ \ \ & 90 ~ \ \ & \textit{130} ~ \ \ \\ 
	 ~ Cereal & 61 ~ \ \ & 94 ~ \ \ & \textit{109} ~ \ \ \\ 
	 ~ Vegetable oils & 65 ~ \ \ & 97 ~ \ \ & \textit{115} ~ \ \ \\ 
	 ~ Roots and tubers & 45 ~ \ \ & 83 ~ \ \ & \textit{146} ~ \ \ \\ 
	 ~ Fruit and vegetables & 73 ~ \ \ & 86 ~ \ \ & \textit{168} ~ \ \ \\ 
	 ~ Sugar & 54 ~ \ \ & 94 ~ \ \ & \textit{124} ~ \ \ \\ 
	 ~ Livestock & 72 ~ \ \ & 93 ~ \ \ & \textit{136} ~ \ \ \\ 
	 ~ Milk & 69 ~ \ \ & 91 ~ \ \ & \textit{131} ~ \ \ \\ 
	 ~ Meat & 75 ~ \ \ & 95 ~ \ \ & \textit{138} ~ \ \ \\ 
	 ~ Fish  & 39 ~ \ \ & 82 ~ \ \ & \textit{135} ~ \ \ \\ 
	\multicolumn{4}{l}{\textit{\normalsize{Net trade (min US\$)}}} \\ 
	 ~ Cereals & -14 ~ \ \ & -6 ~ \ \ & \textit{-200} ~ \ \ \\ 
	 ~ Fruit and vegetables & 9 ~ \ \ & -19 ~ \ \ & \textit{-91} ~ \ \ \\ 
	 ~ Meat & 0 ~ \ \ & 0 ~ \ \ & \textit{7} ~ \ \ \\ 
	 ~ Dairy products & -5 ~ \ \ & -2 ~ \ \ & \textit{-11} ~ \ \ \\ 
	 ~ Fish & \textit{0} ~ \ \ & 0 ~ \ \ & \textit{-7} ~ \ \ \\ 
	\multicolumn{4}{l}{\textcolor{FAOblue}{\textbf{\large{Environment}}}} \\ 
	 ~ Forest area (\%) & 32 ~ \ \ & 26 ~ \ \ & \textit{25} ~ \ \ \\ 
	 ~ Renewable water res withdrawn (\% of total) &  ~ \ \ &  ~ \ \ & 98 ~ \ \ \\ 
	 ~ Terrestrial protect areas (\% total land area)  & 14 ~ \ \ & 17 ~ \ \ & \textit{16} ~ \ \ \\ 
	 ~ Organic area (\% total agricultural area) &  ~ \ \ & \textit{0} ~ \ \ & \textit{0} ~ \ \ \\ 
	 ~ Water withdrawal by agriculture (\% of total) &  ~ \ \ &  ~ \ \ & 98 ~ \ \ \\ 
	 ~ Biofuel production (thousand kt of oil eq.) & 11 ~ \ \ & 18 ~ \ \ & \textit{21} ~ \ \ \\ 
	 ~ Wood pellet prod. (min tonnes) &  ~ \ \ &  ~ \ \ &  ~ \ \ \\ 
	 ~ GHG emissions from ag (Co2 eq, gigagrams) & 50 ~ \ \ & 50 ~ \ \ & \textit{28} ~ \ \ \\ 
       \toprule
      \end{tabular}
      \clearpage
\CountryData{ Netherlands }
      \rowcolors{1}{FAOblue!10}{white}
      \begin{tabular}{L{3.9cm} R{1cm} R{1cm} R{1cm}}
      \toprule
      \multicolumn{1}{c}{} & \multicolumn{1}{c}{ 1992 } & \multicolumn{1}{c}{ 2002 } & \multicolumn{1}{c}{ 2014 } \\
      \midrule
	\multicolumn{4}{l}{\textcolor{FAOblue}{\textbf{\large{The setting}}}} \\ 
	 ~ Population, total (mln) & 15.1 ~ \ \ & 16 ~ \ \ & 16.8 ~ \ \ \\ 
	 ~ Population, rural (\% total population) & 4.5 ~ \ \ & 3.5 ~ \ \ & 2.6 ~ \ \ \\ 
	 ~ Govt expenditure on ag (\% total outlays) &  ~ \ \ &  ~ \ \ &  ~ \ \ \\ 
	 ~ Area harvested (mln ha) & 19 ~ \ \ & 27 ~ \ \ & 7 ~ \ \ \\ 
	 ~ Cropping intensity ratio (\%) & 9.6 ~ \ \ & 13.9 ~ \ \ &  ~ \ \ \\ 
	 ~ Water resources (m\textsuperscript{3}/person/year) & \textit{6} ~ \ \ & \textit{6} ~ \ \ & \textit{5} ~ \ \ \\ 
	 ~ Area equipped for irrigation (1000 ha) &  ~ \ \ &  ~ \ \ & \textit{486} ~ \ \ \\ 
	 ~ Area irrigated (\%) &  ~ \ \ &  ~ \ \ & \textit{44.2} ~ \ \ \\ 
	 ~ Employment in agriculture (\%) & 3.7 ~ \ \ & 2.6 ~ \ \ & \textit{2.5} ~ \ \ \\ 
	 ~ Employment in agriculture, female (\%) & 2.2 ~ \ \ & 1.8 ~ \ \ & \textit{1.5} ~ \ \ \\ 
	 ~ Fertilizers, Nitrogen (nutrients per ha) &  ~ \ \ & 149.9 ~ \ \ & \textit{150.7} ~ \ \ \\ 
	 ~ Fertilizers, Phosphate (nutrients per ha) &  ~ \ \ & 24.6 ~ \ \ & \textit{8.1} ~ \ \ \\ 
	 ~ Fertilizers, Potash (nutrients per ha) &  ~ \ \ & 27 ~ \ \ & \textit{11.4} ~ \ \ \\ 
	 ~ Energy consump, power irrigation (mln kWh) & 6 ~ \ \ & 6 ~ \ \ & \textit{6} ~ \ \ \\ 
	 ~ Agr value added per worker (constant US\$) & 34.3 ~ \ \ & 43.2 ~ \ \ & \textit{66.2} ~ \ \ \\ 
	\multicolumn{4}{l}{\textcolor{FAOblue}{\textbf{\large{Hunger dimensions}}}} \\ 
	 ~ Dietary energy supply (kcal/pc/day) &  ~ \ \ &  ~ \ \ &  ~ \ \ \\ 
	 ~ Average dietary energy supply adequacy (\%) & 129 ~ \ \ & 125 ~ \ \ & 123 ~ \ \ \\ 
	 ~ Dietary en supp, cereals/roots/tubers (\%) & 22 ~ \ \ & 23 ~ \ \ & \textit{27} ~ \ \ \\ 
	 ~ Prevalence of undernourishment (\%) & <5.0 ~ \ \ & <5.0 ~ \ \ & <5.0 ~ \ \ \\ 
	 ~ GDP per capita (US\$, PPP) & 33\,377 ~ \ \ & 41\,848 ~ \ \ & \textit{45\,021} ~ \ \ \\ 
	 ~ Domestic food price volatility (index) &  ~ \ \ & 30.4 ~ \ \ & 5.6 ~ \ \ \\ 
	 ~ Cereal import dependency ratio (\%) & 62.6 ~ \ \ & 77.5 ~ \ \ & \textit{84.3} ~ \ \ \\ 
	 ~ Underweight, children under-5 (\%) &  ~ \ \ &  ~ \ \ &  ~ \ \ \\ 
	 ~ Improved water source (\% pop) & 100 ~ \ \ & 100 ~ \ \ & \textit{100} ~ \ \ \\ 
	\multicolumn{4}{l}{\textcolor{FAOblue}{\textbf{\large{Food Supply}}}} \\ 
	 ~ Food production value, (2004-2006 mln I\$) & 12\,840 ~ \ \ & 11\,652 ~ \ \ & \textit{13\,343} ~ \ \ \\ 
	 ~ Agriculture, value added (\% GDP) & 4 ~ \ \ & 2 ~ \ \ & \textit{2} ~ \ \ \\ 
	 ~ Food exports (mln US\$)  & 22\,553 ~ \ \ & 19\,831 ~ \ \ & \textit{56\,768} ~ \ \ \\ 
	 ~ Food imports (mln US\$)  & 13\,955 ~ \ \ & 13\,879 ~ \ \ & \textit{41\,002} ~ \ \ \\ 
	\multicolumn{4}{l}{\textit{\normalsize{Production indices (2004-06=100)}}} \\ 
	 ~ Net food & 109 ~ \ \ & 99 ~ \ \ & \textit{113} ~ \ \ \\ 
	 ~ Net crop & 99 ~ \ \ & 99 ~ \ \ & \textit{109} ~ \ \ \\ 
	 ~ Cereal & 80 ~ \ \ & 99 ~ \ \ & \textit{104} ~ \ \ \\ 
	 ~ Vegetable oils & 130 ~ \ \ & 61 ~ \ \ & \textit{93} ~ \ \ \\ 
	 ~ Roots and tubers & 109 ~ \ \ & 106 ~ \ \ & \textit{100} ~ \ \ \\ 
	 ~ Fruit and vegetables & 93 ~ \ \ & 93 ~ \ \ & \textit{117} ~ \ \ \\ 
	 ~ Sugar & 140 ~ \ \ & 106 ~ \ \ & \textit{97} ~ \ \ \\ 
	 ~ Livestock & 112 ~ \ \ & 100 ~ \ \ & \textit{115} ~ \ \ \\ 
	 ~ Milk & 109 ~ \ \ & 97 ~ \ \ & \textit{114} ~ \ \ \\ 
	 ~ Meat & 115 ~ \ \ & 101 ~ \ \ & \textit{116} ~ \ \ \\ 
	 ~ Fish  & 83 ~ \ \ & 89 ~ \ \ & \textit{66} ~ \ \ \\ 
	\multicolumn{4}{l}{\textit{\normalsize{Net trade (min US\$)}}} \\ 
	 ~ Cereals & -297 ~ \ \ & -315 ~ \ \ & \textit{-853} ~ \ \ \\ 
	 ~ Fruit and vegetables & 1\,448 ~ \ \ & 2\,075 ~ \ \ & \textit{5\,085} ~ \ \ \\ 
	 ~ Meat & 4\,570 ~ \ \ & 2\,156 ~ \ \ & \textit{4\,732} ~ \ \ \\ 
	 ~ Dairy products & 1\,857 ~ \ \ & 1\,478 ~ \ \ & \textit{3\,839} ~ \ \ \\ 
	 ~ Fish & 517 ~ \ \ & 470 ~ \ \ & \textit{270} ~ \ \ \\ 
	\multicolumn{4}{l}{\textcolor{FAOblue}{\textbf{\large{Environment}}}} \\ 
	 ~ Forest area (\%) & 10 ~ \ \ & 11 ~ \ \ & \textit{11} ~ \ \ \\ 
	 ~ Renewable water res withdrawn (\% of total) &  ~ \ \ &  ~ \ \ & 1 ~ \ \ \\ 
	 ~ Terrestrial protect areas (\% total land area)  & 11 ~ \ \ & 12 ~ \ \ & \textit{20} ~ \ \ \\ 
	 ~ Organic area (\% total agricultural area) &  ~ \ \ & \textit{3} ~ \ \ & \textit{3} ~ \ \ \\ 
	 ~ Water withdrawal by agriculture (\% of total) &  ~ \ \ &  ~ \ \ & 1 ~ \ \ \\ 
	 ~ Biofuel production (thousand kt of oil eq.) & 7 ~ \ \ & 17 ~ \ \ & \textit{10\,350} ~ \ \ \\ 
	 ~ Wood pellet prod. (min tonnes) &  ~ \ \ &  ~ \ \ & \textit{300} ~ \ \ \\ 
	 ~ GHG emissions from ag (Co2 eq, gigagrams) & 24 ~ \ \ & 20 ~ \ \ & \textit{20} ~ \ \ \\ 
       \toprule
      \end{tabular}
      \clearpage
\CountryData{ New Zealand }
      \rowcolors{1}{FAOblue!10}{white}
      \begin{tabular}{L{3.9cm} R{1cm} R{1cm} R{1cm}}
      \toprule
      \multicolumn{1}{c}{} & \multicolumn{1}{c}{ 1992 } & \multicolumn{1}{c}{ 2002 } & \multicolumn{1}{c}{ 2014 } \\
      \midrule
	\multicolumn{4}{l}{\textcolor{FAOblue}{\textbf{\large{The setting}}}} \\ 
	 ~ Population, total (mln) & 3.5 ~ \ \ & 4 ~ \ \ & 4.6 ~ \ \ \\ 
	 ~ Population, rural (\% total population) & 0.5 ~ \ \ & 0.6 ~ \ \ & 0.6 ~ \ \ \\ 
	 ~ Govt expenditure on ag (\% total outlays) &  ~ \ \ &  ~ \ \ &  ~ \ \ \\ 
	 ~ Area harvested (mln ha) & 4 ~ \ \ & 2 ~ \ \ & 1 ~ \ \ \\ 
	 ~ Cropping intensity ratio (\%) & 0.3 ~ \ \ & 0.2 ~ \ \ &  ~ \ \ \\ 
	 ~ Water resources (m\textsuperscript{3}/person/year) & \textit{92} ~ \ \ & \textit{81} ~ \ \ & \textit{73} ~ \ \ \\ 
	 ~ Area equipped for irrigation (1000 ha) &  ~ \ \ &  ~ \ \ & \textit{722} ~ \ \ \\ 
	 ~ Area irrigated (\%) &  ~ \ \ &  ~ \ \ & \textit{82.2} ~ \ \ \\ 
	 ~ Employment in agriculture (\%) & 10.8 ~ \ \ & 8.8 ~ \ \ & \textit{6.6} ~ \ \ \\ 
	 ~ Employment in agriculture, female (\%) & 7.7 ~ \ \ & 5.7 ~ \ \ & \textit{4.4} ~ \ \ \\ 
	 ~ Fertilizers, Nitrogen (nutrients per ha) &  ~ \ \ & 25.5 ~ \ \ & \textit{26.7} ~ \ \ \\ 
	 ~ Fertilizers, Phosphate (nutrients per ha) &  ~ \ \ & 33.3 ~ \ \ & \textit{45.9} ~ \ \ \\ 
	 ~ Fertilizers, Potash (nutrients per ha) &  ~ \ \ & 13.5 ~ \ \ & \textit{3.8} ~ \ \ \\ 
	 ~ Energy consump, power irrigation (mln kWh) & 2 ~ \ \ & 2 ~ \ \ & \textit{1\,028} ~ \ \ \\ 
	 ~ Agr value added per worker (constant US\$) & 19.4 ~ \ \ & 25.6 ~ \ \ & \textit{27.2} ~ \ \ \\ 
	\multicolumn{4}{l}{\textcolor{FAOblue}{\textbf{\large{Hunger dimensions}}}} \\ 
	 ~ Dietary energy supply (kcal/pc/day) &  ~ \ \ &  ~ \ \ &  ~ \ \ \\ 
	 ~ Average dietary energy supply adequacy (\%) & 130 ~ \ \ & 126 ~ \ \ & 128 ~ \ \ \\ 
	 ~ Dietary en supp, cereals/roots/tubers (\%) & 28 ~ \ \ & 28 ~ \ \ & \textit{28} ~ \ \ \\ 
	 ~ Prevalence of undernourishment (\%) & <5.0 ~ \ \ & <5.0 ~ \ \ & <5.0 ~ \ \ \\ 
	 ~ GDP per capita (US\$, PPP) & 21\,958 ~ \ \ & 29\,156 ~ \ \ & \textit{33\,020} ~ \ \ \\ 
	 ~ Domestic food price volatility (index) &  ~ \ \ &  ~ \ \ &  ~ \ \ \\ 
	 ~ Cereal import dependency ratio (\%) & 18.3 ~ \ \ & 31.5 ~ \ \ & \textit{28.3} ~ \ \ \\ 
	 ~ Underweight, children under-5 (\%) &  ~ \ \ &  ~ \ \ &  ~ \ \ \\ 
	 ~ Improved water source (\% pop) & 100 ~ \ \ & 100 ~ \ \ & \textit{100} ~ \ \ \\ 
	\multicolumn{4}{l}{\textcolor{FAOblue}{\textbf{\large{Food Supply}}}} \\ 
	 ~ Food production value, (2004-2006 mln I\$) & 6\,718 ~ \ \ & 8\,641 ~ \ \ & \textit{10\,334} ~ \ \ \\ 
	 ~ Agriculture, value added (\% GDP) & 7 ~ \ \ & 6 ~ \ \ & \textit{7} ~ \ \ \\ 
	 ~ Food exports (mln US\$)  & 3\,916 ~ \ \ & 5\,671 ~ \ \ & \textit{17\,306} ~ \ \ \\ 
	 ~ Food imports (mln US\$)  & 468 ~ \ \ & 929 ~ \ \ & \textit{2\,780} ~ \ \ \\ 
	\multicolumn{4}{l}{\textit{\normalsize{Production indices (2004-06=100)}}} \\ 
	 ~ Net food & 71 ~ \ \ & 92 ~ \ \ & \textit{110} ~ \ \ \\ 
	 ~ Net crop & 78 ~ \ \ & 93 ~ \ \ & \textit{118} ~ \ \ \\ 
	 ~ Cereal & 85 ~ \ \ & 109 ~ \ \ & \textit{132} ~ \ \ \\ 
	 ~ Vegetable oils & 49 ~ \ \ & 89 ~ \ \ & \textit{93} ~ \ \ \\ 
	 ~ Roots and tubers & 64 ~ \ \ & 100 ~ \ \ & \textit{112} ~ \ \ \\ 
	 ~ Fruit and vegetables & 76 ~ \ \ & 91 ~ \ \ & \textit{117} ~ \ \ \\ 
	 ~ Sugar &  ~ \ \ &  ~ \ \ &  ~ \ \ \\ 
	 ~ Livestock & 73 ~ \ \ & 92 ~ \ \ & \textit{107} ~ \ \ \\ 
	 ~ Milk & 54 ~ \ \ & 93 ~ \ \ & \textit{126} ~ \ \ \\ 
	 ~ Meat & 89 ~ \ \ & 90 ~ \ \ & \textit{85} ~ \ \ \\ 
	 ~ Fish  & 84 ~ \ \ & 108 ~ \ \ & \textit{86} ~ \ \ \\ 
	\multicolumn{4}{l}{\textit{\normalsize{Net trade (min US\$)}}} \\ 
	 ~ Cereals & -42 ~ \ \ & -97 ~ \ \ & \textit{62} ~ \ \ \\ 
	 ~ Fruit and vegetables & 554 ~ \ \ & 555 ~ \ \ & \textit{1\,252} ~ \ \ \\ 
	 ~ Meat & 1\,668 ~ \ \ & 1\,949 ~ \ \ & \textit{4\,183} ~ \ \ \\ 
	 ~ Dairy products & 1\,308 ~ \ \ & 2\,382 ~ \ \ & \textit{9\,117} ~ \ \ \\ 
	 ~ Fish & 620 ~ \ \ & 654 ~ \ \ & \textit{1\,090} ~ \ \ \\ 
	\multicolumn{4}{l}{\textcolor{FAOblue}{\textbf{\large{Environment}}}} \\ 
	 ~ Forest area (\%) & 30 ~ \ \ & 31 ~ \ \ & \textit{31} ~ \ \ \\ 
	 ~ Renewable water res withdrawn (\% of total) &  ~ \ \ & 74 ~ \ \ & 74 ~ \ \ \\ 
	 ~ Terrestrial protect areas (\% total land area)  & 25 ~ \ \ & 26 ~ \ \ & \textit{27} ~ \ \ \\ 
	 ~ Organic area (\% total agricultural area) &  ~ \ \ & \textit{0} ~ \ \ & \textit{1} ~ \ \ \\ 
	 ~ Water withdrawal by agriculture (\% of total) &  ~ \ \ & 74 ~ \ \ & 74 ~ \ \ \\ 
	 ~ Biofuel production (thousand kt of oil eq.) & 26 ~ \ \ & 43 ~ \ \ & \textit{70} ~ \ \ \\ 
	 ~ Wood pellet prod. (min tonnes) &  ~ \ \ &  ~ \ \ &  ~ \ \ \\ 
	 ~ GHG emissions from ag (Co2 eq, gigagrams) & 18 ~ \ \ & 20 ~ \ \ & \textit{21} ~ \ \ \\ 
       \toprule
      \end{tabular}
      \clearpage
\CountryData{ Nicaragua }
      \rowcolors{1}{FAOblue!10}{white}
      \begin{tabular}{L{3.9cm} R{1cm} R{1cm} R{1cm}}
      \toprule
      \multicolumn{1}{c}{} & \multicolumn{1}{c}{ 1992 } & \multicolumn{1}{c}{ 2002 } & \multicolumn{1}{c}{ 2014 } \\
      \midrule
	\multicolumn{4}{l}{\textcolor{FAOblue}{\textbf{\large{The setting}}}} \\ 
	 ~ Population, total (mln) & 4.3 ~ \ \ & 5.2 ~ \ \ & 6.2 ~ \ \ \\ 
	 ~ Population, rural (\% total population) & 2 ~ \ \ & 2.4 ~ \ \ & 2.6 ~ \ \ \\ 
	 ~ Govt expenditure on ag (\% total outlays) &  ~ \ \ &  ~ \ \ &  ~ \ \ \\ 
	 ~ Area harvested (mln ha) & 3 ~ \ \ & 3 ~ \ \ & 7 ~ \ \ \\ 
	 ~ Cropping intensity ratio (\%) & 0.6 ~ \ \ & 0.6 ~ \ \ &  ~ \ \ \\ 
	 ~ Water resources (m\textsuperscript{3}/person/year) & \textit{37} ~ \ \ & \textit{31} ~ \ \ & \textit{27} ~ \ \ \\ 
	 ~ Area equipped for irrigation (1000 ha) &  ~ \ \ &  ~ \ \ & \textit{199} ~ \ \ \\ 
	 ~ Area irrigated (\%) &  ~ \ \ &  ~ \ \ & \textit{72.4} ~ \ \ \\ 
	 ~ Employment in agriculture (\%) & 38.9 ~ \ \ & \textit{28.9} ~ \ \ & \textit{32.2} ~ \ \ \\ 
	 ~ Employment in agriculture, female (\%) &  ~ \ \ & \textit{8.3} ~ \ \ & \textit{15.2} ~ \ \ \\ 
	 ~ Fertilizers, Nitrogen (nutrients per ha) &  ~ \ \ & 6.7 ~ \ \ & \textit{10.9} ~ \ \ \\ 
	 ~ Fertilizers, Phosphate (nutrients per ha) &  ~ \ \ & 2.7 ~ \ \ & \textit{2.2} ~ \ \ \\ 
	 ~ Fertilizers, Potash (nutrients per ha) &  ~ \ \ & 1.4 ~ \ \ & \textit{2.8} ~ \ \ \\ 
	 ~ Energy consump, power irrigation (mln kWh) &  ~ \ \ & 0 ~ \ \ & \textit{0} ~ \ \ \\ 
	 ~ Agr value added per worker (constant US\$) & \textit{1.8} ~ \ \ & 2.4 ~ \ \ & \textit{3.7} ~ \ \ \\ 
	\multicolumn{4}{l}{\textcolor{FAOblue}{\textbf{\large{Hunger dimensions}}}} \\ 
	 ~ Dietary energy supply (kcal/pc/day) & 1\,813 ~ \ \ & 2\,305 ~ \ \ & 2\,628 ~ \ \ \\ 
	 ~ Average dietary energy supply adequacy (\%) & 86 ~ \ \ & 106 ~ \ \ & 117 ~ \ \ \\ 
	 ~ Dietary en supp, cereals/roots/tubers (\%) & 49 ~ \ \ & 54 ~ \ \ & \textit{51} ~ \ \ \\ 
	 ~ Prevalence of undernourishment (\%) & 52.7 ~ \ \ & 28.3 ~ \ \ & 17.1 ~ \ \ \\ 
	 ~ GDP per capita (US\$, PPP) & 2\,942 ~ \ \ & 3\,517 ~ \ \ & \textit{4\,494} ~ \ \ \\ 
	 ~ Domestic food price volatility (index) &  ~ \ \ & 4.2 ~ \ \ & 6.4 ~ \ \ \\ 
	 ~ Cereal import dependency ratio (\%) & 25.7 ~ \ \ & 25.1 ~ \ \ & \textit{31.5} ~ \ \ \\ 
	 ~ Underweight, children under-5 (\%) & \textit{9.6} ~ \ \ & \textit{4.3} ~ \ \ & \textit{5.7} ~ \ \ \\ 
	 ~ Improved water source (\% pop) & 75.1 ~ \ \ & 81.2 ~ \ \ & \textit{85} ~ \ \ \\ 
	\multicolumn{4}{l}{\textcolor{FAOblue}{\textbf{\large{Food Supply}}}} \\ 
	 ~ Food production value, (2004-2006 mln I\$) & 571 ~ \ \ & 955 ~ \ \ & \textit{1\,579} ~ \ \ \\ 
	 ~ Agriculture, value added (\% GDP) & \textit{22} ~ \ \ & 18 ~ \ \ & \textit{17} ~ \ \ \\ 
	 ~ Food exports (mln US\$)  & 93 ~ \ \ & 260 ~ \ \ & \textit{1\,238} ~ \ \ \\ 
	 ~ Food imports (mln US\$)  & 184 ~ \ \ & 223 ~ \ \ & \textit{716} ~ \ \ \\ 
	\multicolumn{4}{l}{\textit{\normalsize{Production indices (2004-06=100)}}} \\ 
	 ~ Net food & 51 ~ \ \ & 85 ~ \ \ & \textit{141} ~ \ \ \\ 
	 ~ Net crop & 64 ~ \ \ & 88 ~ \ \ & \textit{141} ~ \ \ \\ 
	 ~ Cereal & 55 ~ \ \ & 103 ~ \ \ & \textit{117} ~ \ \ \\ 
	 ~ Vegetable oils & 28 ~ \ \ & 58 ~ \ \ & \textit{126} ~ \ \ \\ 
	 ~ Roots and tubers & 55 ~ \ \ & 98 ~ \ \ & \textit{355} ~ \ \ \\ 
	 ~ Fruit and vegetables & 137 ~ \ \ & 96 ~ \ \ & \textit{133} ~ \ \ \\ 
	 ~ Sugar & 62 ~ \ \ & 76 ~ \ \ & \textit{171} ~ \ \ \\ 
	 ~ Livestock & 44 ~ \ \ & 83 ~ \ \ & \textit{136} ~ \ \ \\ 
	 ~ Milk & 27 ~ \ \ & 87 ~ \ \ & \textit{123} ~ \ \ \\ 
	 ~ Meat & 50 ~ \ \ & 79 ~ \ \ & \textit{143} ~ \ \ \\ 
	 ~ Fish  & 19 ~ \ \ & 70 ~ \ \ & \textit{183} ~ \ \ \\ 
	\multicolumn{4}{l}{\textit{\normalsize{Net trade (min US\$)}}} \\ 
	 ~ Cereals & -58 ~ \ \ & -73 ~ \ \ & \textit{-231} ~ \ \ \\ 
	 ~ Fruit and vegetables & -12 ~ \ \ & 17 ~ \ \ & \textit{54} ~ \ \ \\ 
	 ~ Meat & 33 ~ \ \ & 84 ~ \ \ & \textit{434} ~ \ \ \\ 
	 ~ Dairy products & -19 ~ \ \ & 15 ~ \ \ & \textit{139} ~ \ \ \\ 
	 ~ Fish & 22 ~ \ \ & 41 ~ \ \ & \textit{193} ~ \ \ \\ 
	\multicolumn{4}{l}{\textcolor{FAOblue}{\textbf{\large{Environment}}}} \\ 
	 ~ Forest area (\%) & 36 ~ \ \ & 31 ~ \ \ & \textit{25} ~ \ \ \\ 
	 ~ Renewable water res withdrawn (\% of total) &  ~ \ \ &  ~ \ \ & 77 ~ \ \ \\ 
	 ~ Terrestrial protect areas (\% total land area)  & 29 ~ \ \ & 37 ~ \ \ & \textit{31} ~ \ \ \\ 
	 ~ Organic area (\% total agricultural area) &  ~ \ \ & \textit{1} ~ \ \ & \textit{1} ~ \ \ \\ 
	 ~ Water withdrawal by agriculture (\% of total) &  ~ \ \ &  ~ \ \ & 77 ~ \ \ \\ 
	 ~ Biofuel production (thousand kt of oil eq.) & 5 ~ \ \ & 8 ~ \ \ & \textit{13} ~ \ \ \\ 
	 ~ Wood pellet prod. (min tonnes) &  ~ \ \ &  ~ \ \ &  ~ \ \ \\ 
	 ~ GHG emissions from ag (Co2 eq, gigagrams) & 34 ~ \ \ & 36 ~ \ \ & \textit{37} ~ \ \ \\ 
       \toprule
      \end{tabular}
      \clearpage
\CountryData{ Niger }
      \rowcolors{1}{FAOblue!10}{white}
      \begin{tabular}{L{3.9cm} R{1cm} R{1cm} R{1cm}}
      \toprule
      \multicolumn{1}{c}{} & \multicolumn{1}{c}{ 1992 } & \multicolumn{1}{c}{ 2002 } & \multicolumn{1}{c}{ 2014 } \\
      \midrule
	\multicolumn{4}{l}{\textcolor{FAOblue}{\textbf{\large{The setting}}}} \\ 
	 ~ Population, total (mln) & 8.3 ~ \ \ & 11.8 ~ \ \ & 18.5 ~ \ \ \\ 
	 ~ Population, rural (\% total population) & 7 ~ \ \ & 9.9 ~ \ \ & 15.1 ~ \ \ \\ 
	 ~ Govt expenditure on ag (\% total outlays) &  ~ \ \ &  ~ \ \ &  ~ \ \ \\ 
	 ~ Area harvested (mln ha) & 8 ~ \ \ & 8 ~ \ \ & 10 ~ \ \ \\ 
	 ~ Cropping intensity ratio (\%) & 0.2 ~ \ \ & 0.2 ~ \ \ &  ~ \ \ \\ 
	 ~ Water resources (m\textsuperscript{3}/person/year) & \textit{4} ~ \ \ & \textit{3} ~ \ \ & \textit{2} ~ \ \ \\ 
	 ~ Area equipped for irrigation (1000 ha) &  ~ \ \ &  ~ \ \ & \textit{100} ~ \ \ \\ 
	 ~ Area irrigated (\%) &  ~ \ \ &  ~ \ \ & \textit{88} ~ \ \ \\ 
	 ~ Employment in agriculture (\%) &  ~ \ \ & \textit{56.9} ~ \ \ & \textit{56.9} ~ \ \ \\ 
	 ~ Employment in agriculture, female (\%) &  ~ \ \ & \textit{37.8} ~ \ \ & \textit{37.8} ~ \ \ \\ 
	 ~ Fertilizers, Nitrogen (nutrients per ha) &  ~ \ \ & 0.2 ~ \ \ & \textit{0.3} ~ \ \ \\ 
	 ~ Fertilizers, Phosphate (nutrients per ha) &  ~ \ \ & 0 ~ \ \ & \textit{0.1} ~ \ \ \\ 
	 ~ Fertilizers, Potash (nutrients per ha) &  ~ \ \ & 0 ~ \ \ & \textit{0.1} ~ \ \ \\ 
	 ~ Energy consump, power irrigation (mln kWh) &  ~ \ \ &  ~ \ \ &  ~ \ \ \\ 
	 ~ Agr value added per worker (constant US\$) &  ~ \ \ & \textit{0.2} ~ \ \ & \textit{0.2} ~ \ \ \\ 
	\multicolumn{4}{l}{\textcolor{FAOblue}{\textbf{\large{Hunger dimensions}}}} \\ 
	 ~ Dietary energy supply (kcal/pc/day) & 2\,064 ~ \ \ & 2\,351 ~ \ \ & 2\,583 ~ \ \ \\ 
	 ~ Average dietary energy supply adequacy (\%) & 99 ~ \ \ & 113 ~ \ \ & 124 ~ \ \ \\ 
	 ~ Dietary en supp, cereals/roots/tubers (\%) & 76 ~ \ \ & 67 ~ \ \ & \textit{61} ~ \ \ \\ 
	 ~ Prevalence of undernourishment (\%) & 31 ~ \ \ & 19.1 ~ \ \ & 9.7 ~ \ \ \\ 
	 ~ GDP per capita (US\$, PPP) & 830 ~ \ \ & 798 ~ \ \ & \textit{887} ~ \ \ \\ 
	 ~ Domestic food price volatility (index) &  ~ \ \ & 13.9 ~ \ \ & 9.4 ~ \ \ \\ 
	 ~ Cereal import dependency ratio (\%) & 6.2 ~ \ \ & 8.7 ~ \ \ & \textit{7.3} ~ \ \ \\ 
	 ~ Underweight, children under-5 (\%) & 41 ~ \ \ & \textit{43.6} ~ \ \ & \textit{37.9} ~ \ \ \\ 
	 ~ Improved water source (\% pop) & 35.9 ~ \ \ & 43.8 ~ \ \ & \textit{52.3} ~ \ \ \\ 
	\multicolumn{4}{l}{\textcolor{FAOblue}{\textbf{\large{Food Supply}}}} \\ 
	 ~ Food production value, (2004-2006 mln I\$) & 1\,135 ~ \ \ & 1\,944 ~ \ \ & \textit{2\,859} ~ \ \ \\ 
	 ~ Agriculture, value added (\% GDP) & 39 ~ \ \ & 40 ~ \ \ & \textit{37} ~ \ \ \\ 
	 ~ Food exports (mln US\$)  & 33 ~ \ \ & 66 ~ \ \ & \textit{191} ~ \ \ \\ 
	 ~ Food imports (mln US\$)  & 72 ~ \ \ & 140 ~ \ \ & \textit{460} ~ \ \ \\ 
	\multicolumn{4}{l}{\textit{\normalsize{Production indices (2004-06=100)}}} \\ 
	 ~ Net food & 53 ~ \ \ & 91 ~ \ \ & \textit{134} ~ \ \ \\ 
	 ~ Net crop & 54 ~ \ \ & 95 ~ \ \ & \textit{160} ~ \ \ \\ 
	 ~ Cereal & 65 ~ \ \ & 94 ~ \ \ & \textit{123} ~ \ \ \\ 
	 ~ Vegetable oils & 25 ~ \ \ & 86 ~ \ \ & \textit{206} ~ \ \ \\ 
	 ~ Roots and tubers & 57 ~ \ \ & 84 ~ \ \ & \textit{185} ~ \ \ \\ 
	 ~ Fruit and vegetables & 34 ~ \ \ & 88 ~ \ \ & \textit{166} ~ \ \ \\ 
	 ~ Sugar & 43 ~ \ \ & 97 ~ \ \ & \textit{84} ~ \ \ \\ 
	 ~ Livestock & 51 ~ \ \ & 87 ~ \ \ & \textit{103} ~ \ \ \\ 
	 ~ Milk & 58 ~ \ \ & 88 ~ \ \ & \textit{129} ~ \ \ \\ 
	 ~ Meat & 48 ~ \ \ & 87 ~ \ \ & \textit{95} ~ \ \ \\ 
	 ~ Fish  & 6 ~ \ \ & 54 ~ \ \ & \textit{103} ~ \ \ \\ 
	\multicolumn{4}{l}{\textit{\normalsize{Net trade (min US\$)}}} \\ 
	 ~ Cereals & -37 ~ \ \ & -67 ~ \ \ & \textit{-236} ~ \ \ \\ 
	 ~ Fruit and vegetables & -11 ~ \ \ & 19 ~ \ \ & \textit{60} ~ \ \ \\ 
	 ~ Meat & -1 ~ \ \ & -1 ~ \ \ & \textit{-1} ~ \ \ \\ 
	 ~ Dairy products & -7 ~ \ \ & -11 ~ \ \ & \textit{-38} ~ \ \ \\ 
	 ~ Fish & -1 ~ \ \ & 4 ~ \ \ & \textit{-1} ~ \ \ \\ 
	\multicolumn{4}{l}{\textcolor{FAOblue}{\textbf{\large{Environment}}}} \\ 
	 ~ Forest area (\%) & 1 ~ \ \ & 1 ~ \ \ & \textit{1} ~ \ \ \\ 
	 ~ Renewable water res withdrawn (\% of total) &  ~ \ \ & \textit{67} ~ \ \ & 67 ~ \ \ \\ 
	 ~ Terrestrial protect areas (\% total land area)  & 7 ~ \ \ & 7 ~ \ \ & \textit{17} ~ \ \ \\ 
	 ~ Organic area (\% total agricultural area) &  ~ \ \ & \textit{0} ~ \ \ & \textit{0} ~ \ \ \\ 
	 ~ Water withdrawal by agriculture (\% of total) &  ~ \ \ & \textit{67} ~ \ \ & 67 ~ \ \ \\ 
	 ~ Biofuel production (thousand kt of oil eq.) &  ~ \ \ & \textit{1} ~ \ \ & \textit{0} ~ \ \ \\ 
	 ~ Wood pellet prod. (min tonnes) &  ~ \ \ &  ~ \ \ &  ~ \ \ \\ 
	 ~ GHG emissions from ag (Co2 eq, gigagrams) & 17 ~ \ \ & 17 ~ \ \ & \textit{22} ~ \ \ \\ 
       \toprule
      \end{tabular}
      \clearpage
\CountryData{ Nigeria }
      \rowcolors{1}{FAOblue!10}{white}
      \begin{tabular}{L{3.9cm} R{1cm} R{1cm} R{1cm}}
      \toprule
      \multicolumn{1}{c}{} & \multicolumn{1}{c}{ 1992 } & \multicolumn{1}{c}{ 2002 } & \multicolumn{1}{c}{ 2014 } \\
      \midrule
	\multicolumn{4}{l}{\textcolor{FAOblue}{\textbf{\large{The setting}}}} \\ 
	 ~ Population, total (mln) & 100.6 ~ \ \ & 129.2 ~ \ \ & 178.5 ~ \ \ \\ 
	 ~ Population, rural (\% total population) & 63.7 ~ \ \ & 72.7 ~ \ \ & 86.6 ~ \ \ \\ 
	 ~ Govt expenditure on ag (\% total outlays) &  ~ \ \ & 1.4 ~ \ \ & \textit{0.9} ~ \ \ \\ 
	 ~ Area harvested (mln ha) & 50 ~ \ \ & 69 ~ \ \ & 100 ~ \ \ \\ 
	 ~ Cropping intensity ratio (\%) & 0.8 ~ \ \ & 1 ~ \ \ &  ~ \ \ \\ 
	 ~ Water resources (m\textsuperscript{3}/person/year) & \textit{3} ~ \ \ & \textit{2} ~ \ \ & \textit{2} ~ \ \ \\ 
	 ~ Area equipped for irrigation (1000 ha) &  ~ \ \ &  ~ \ \ & \textit{293} ~ \ \ \\ 
	 ~ Area irrigated (\%) &  ~ \ \ & \textit{74.6} ~ \ \ &  ~ \ \ \\ 
	 ~ Employment in agriculture (\%) &  ~ \ \ & \textit{44.6} ~ \ \ &  ~ \ \ \\ 
	 ~ Employment in agriculture, female (\%) &  ~ \ \ & \textit{38.7} ~ \ \ &  ~ \ \ \\ 
	 ~ Fertilizers, Nitrogen (nutrients per ha) &  ~ \ \ & 1.8 ~ \ \ & \textit{1.6} ~ \ \ \\ 
	 ~ Fertilizers, Phosphate (nutrients per ha) &  ~ \ \ & 0.2 ~ \ \ & \textit{0.4} ~ \ \ \\ 
	 ~ Fertilizers, Potash (nutrients per ha) &  ~ \ \ & 0.2 ~ \ \ & \textit{0.3} ~ \ \ \\ 
	 ~ Energy consump, power irrigation (mln kWh) &  ~ \ \ & \textit{0} ~ \ \ & \textit{0} ~ \ \ \\ 
	 ~ Agr value added per worker (constant US\$) & 1.1 ~ \ \ & 2.4 ~ \ \ & \textit{4.6} ~ \ \ \\ 
	\multicolumn{4}{l}{\textcolor{FAOblue}{\textbf{\large{Hunger dimensions}}}} \\ 
	 ~ Dietary energy supply (kcal/pc/day) & 2\,434 ~ \ \ & 2\,609 ~ \ \ & 2\,656 ~ \ \ \\ 
	 ~ Average dietary energy supply adequacy (\%) & 113 ~ \ \ & 121 ~ \ \ & 124 ~ \ \ \\ 
	 ~ Dietary en supp, cereals/roots/tubers (\%) & 68 ~ \ \ & 63 ~ \ \ & \textit{66} ~ \ \ \\ 
	 ~ Prevalence of undernourishment (\%) & 17.9 ~ \ \ & 9 ~ \ \ & 6.7 ~ \ \ \\ 
	 ~ GDP per capita (US\$, PPP) & 2\,875 ~ \ \ & 2\,922 ~ \ \ & \textit{5\,423} ~ \ \ \\ 
	 ~ Domestic food price volatility (index) &  ~ \ \ & 14.2 ~ \ \ & \textit{4} ~ \ \ \\ 
	 ~ Cereal import dependency ratio (\%) & 6.1 ~ \ \ & 15.9 ~ \ \ & \textit{21.7} ~ \ \ \\ 
	 ~ Underweight, children under-5 (\%) & \textit{35.1} ~ \ \ & \textit{27.2} ~ \ \ & \textit{31} ~ \ \ \\ 
	 ~ Improved water source (\% pop) & 47.6 ~ \ \ & 56.5 ~ \ \ & \textit{64} ~ \ \ \\ 
	\multicolumn{4}{l}{\textcolor{FAOblue}{\textbf{\large{Food Supply}}}} \\ 
	 ~ Food production value, (2004-2006 mln I\$) & 18\,551 ~ \ \ & 26\,647 ~ \ \ & \textit{36\,075} ~ \ \ \\ 
	 ~ Agriculture, value added (\% GDP) & 27 ~ \ \ & 49 ~ \ \ & \textit{21} ~ \ \ \\ 
	 ~ Food exports (mln US\$)  & 128 ~ \ \ & 335 ~ \ \ & \textit{1\,219} ~ \ \ \\ 
	 ~ Food imports (mln US\$)  & 719 ~ \ \ & 1\,533 ~ \ \ & \textit{6\,402} ~ \ \ \\ 
	\multicolumn{4}{l}{\textit{\normalsize{Production indices (2004-06=100)}}} \\ 
	 ~ Net food & 59 ~ \ \ & 85 ~ \ \ & \textit{115} ~ \ \ \\ 
	 ~ Net crop & 59 ~ \ \ & 83 ~ \ \ & \textit{109} ~ \ \ \\ 
	 ~ Cereal & 75 ~ \ \ & 81 ~ \ \ & \textit{102} ~ \ \ \\ 
	 ~ Vegetable oils & 50 ~ \ \ & 81 ~ \ \ & \textit{87} ~ \ \ \\ 
	 ~ Roots and tubers & 57 ~ \ \ & 82 ~ \ \ & \textit{118} ~ \ \ \\ 
	 ~ Fruit and vegetables & 61 ~ \ \ & 87 ~ \ \ & \textit{103} ~ \ \ \\ 
	 ~ Sugar & 98 ~ \ \ & 82 ~ \ \ & \textit{158} ~ \ \ \\ 
	 ~ Livestock & 61 ~ \ \ & 100 ~ \ \ & \textit{124} ~ \ \ \\ 
	 ~ Milk & 84 ~ \ \ & 92 ~ \ \ & \textit{129} ~ \ \ \\ 
	 ~ Meat & 57 ~ \ \ & 102 ~ \ \ & \textit{122} ~ \ \ \\ 
	 ~ Fish  & 55 ~ \ \ & 89 ~ \ \ & \textit{174} ~ \ \ \\ 
	\multicolumn{4}{l}{\textit{\normalsize{Net trade (min US\$)}}} \\ 
	 ~ Cereals & -243 ~ \ \ & -573 ~ \ \ & \textit{-3\,211} ~ \ \ \\ 
	 ~ Fruit and vegetables & 3 ~ \ \ & -43 ~ \ \ & \textit{159} ~ \ \ \\ 
	 ~ Meat & 0 ~ \ \ & -7 ~ \ \ & \textit{-11} ~ \ \ \\ 
	 ~ Dairy products & -94 ~ \ \ & -172 ~ \ \ & \textit{-519} ~ \ \ \\ 
	 ~ Fish & -262 ~ \ \ & -343 ~ \ \ & \textit{-1\,142} ~ \ \ \\ 
	\multicolumn{4}{l}{\textcolor{FAOblue}{\textbf{\large{Environment}}}} \\ 
	 ~ Forest area (\%) & 18 ~ \ \ & 14 ~ \ \ & \textit{9} ~ \ \ \\ 
	 ~ Renewable water res withdrawn (\% of total) &  ~ \ \ & \textit{54} ~ \ \ & 54 ~ \ \ \\ 
	 ~ Terrestrial protect areas (\% total land area)  & 13 ~ \ \ & 13 ~ \ \ & \textit{14} ~ \ \ \\ 
	 ~ Organic area (\% total agricultural area) &  ~ \ \ &  ~ \ \ & \textit{0} ~ \ \ \\ 
	 ~ Water withdrawal by agriculture (\% of total) &  ~ \ \ & \textit{54} ~ \ \ & 54 ~ \ \ \\ 
	 ~ Biofuel production (thousand kt of oil eq.) & 1 ~ \ \ & 0 ~ \ \ & \textit{1} ~ \ \ \\ 
	 ~ Wood pellet prod. (min tonnes) &  ~ \ \ &  ~ \ \ &  ~ \ \ \\ 
	 ~ GHG emissions from ag (Co2 eq, gigagrams) & 219 ~ \ \ & 230 ~ \ \ & \textit{237} ~ \ \ \\ 
       \toprule
      \end{tabular}
      \clearpage
\CountryData{ North Korea }
      \rowcolors{1}{FAOblue!10}{white}
      \begin{tabular}{L{3.9cm} R{1cm} R{1cm} R{1cm}}
      \toprule
      \multicolumn{1}{c}{} & \multicolumn{1}{c}{ 1992 } & \multicolumn{1}{c}{ 2002 } & \multicolumn{1}{c}{ 2014 } \\
      \midrule
	\multicolumn{4}{l}{\textcolor{FAOblue}{\textbf{\large{The setting}}}} \\ 
	 ~ Population, total (mln) & 20.8 ~ \ \ & 23.2 ~ \ \ & 25 ~ \ \ \\ 
	 ~ Population, rural (\% total population) & 8.6 ~ \ \ & 9.4 ~ \ \ & 9.8 ~ \ \ \\ 
	 ~ Govt expenditure on ag (\% total outlays) &  ~ \ \ &  ~ \ \ &  ~ \ \ \\ 
	 ~ Area harvested (mln ha) & 9 ~ \ \ & 4 ~ \ \ & 5 ~ \ \ \\ 
	 ~ Cropping intensity ratio (\%) & 3.4 ~ \ \ & 1.7 ~ \ \ &  ~ \ \ \\ 
	 ~ Water resources (m\textsuperscript{3}/person/year) & \textit{4} ~ \ \ & \textit{3} ~ \ \ & \textit{3} ~ \ \ \\ 
	 ~ Area equipped for irrigation (1000 ha) &  ~ \ \ &  ~ \ \ & \textit{1\,460} ~ \ \ \\ 
	 ~ Area irrigated (\%) &  ~ \ \ &  ~ \ \ &  ~ \ \ \\ 
	 ~ Employment in agriculture (\%) &  ~ \ \ &  ~ \ \ &  ~ \ \ \\ 
	 ~ Employment in agriculture, female (\%) &  ~ \ \ &  ~ \ \ &  ~ \ \ \\ 
	 ~ Fertilizers, Nitrogen (nutrients per ha) &  ~ \ \ &  ~ \ \ &  ~ \ \ \\ 
	 ~ Fertilizers, Phosphate (nutrients per ha) &  ~ \ \ &  ~ \ \ &  ~ \ \ \\ 
	 ~ Fertilizers, Potash (nutrients per ha) &  ~ \ \ &  ~ \ \ &  ~ \ \ \\ 
	 ~ Energy consump, power irrigation (mln kWh) & \textit{0} ~ \ \ & 0 ~ \ \ & \textit{0} ~ \ \ \\ 
	 ~ Agr value added per worker (constant US\$) &  ~ \ \ &  ~ \ \ &  ~ \ \ \\ 
	\multicolumn{4}{l}{\textcolor{FAOblue}{\textbf{\large{Hunger dimensions}}}} \\ 
	 ~ Dietary energy supply (kcal/pc/day) & 2\,288 ~ \ \ & 2\,156 ~ \ \ & 2\,103 ~ \ \ \\ 
	 ~ Average dietary energy supply adequacy (\%) & 97 ~ \ \ & 92 ~ \ \ & 88 ~ \ \ \\ 
	 ~ Dietary en supp, cereals/roots/tubers (\%) & 65 ~ \ \ & 68 ~ \ \ & \textit{69} ~ \ \ \\ 
	 ~ Prevalence of undernourishment (\%) & 24.7 ~ \ \ & 37.2 ~ \ \ & 41.8 ~ \ \ \\ 
	 ~ GDP per capita (US\$, PPP) &  ~ \ \ &  ~ \ \ &  ~ \ \ \\ 
	 ~ Domestic food price volatility (index) &  ~ \ \ &  ~ \ \ &  ~ \ \ \\ 
	 ~ Cereal import dependency ratio (\%) & 16.1 ~ \ \ & 35.3 ~ \ \ & \textit{12.2} ~ \ \ \\ 
	 ~ Underweight, children under-5 (\%) &  ~ \ \ & 17.8 ~ \ \ & \textit{15.2} ~ \ \ \\ 
	 ~ Improved water source (\% pop) & 100 ~ \ \ & 99.4 ~ \ \ & \textit{98.1} ~ \ \ \\ 
	\multicolumn{4}{l}{\textcolor{FAOblue}{\textbf{\large{Food Supply}}}} \\ 
	 ~ Food production value, (2004-2006 mln I\$) & 3\,686 ~ \ \ & 3\,420 ~ \ \ & \textit{3\,651} ~ \ \ \\ 
	 ~ Agriculture, value added (\% GDP) &  ~ \ \ &  ~ \ \ &  ~ \ \ \\ 
	 ~ Food exports (mln US\$)  & 19 ~ \ \ & 15 ~ \ \ & \textit{18} ~ \ \ \\ 
	 ~ Food imports (mln US\$)  & 286 ~ \ \ & 330 ~ \ \ & \textit{510} ~ \ \ \\ 
	\multicolumn{4}{l}{\textit{\normalsize{Production indices (2004-06=100)}}} \\ 
	 ~ Net food & 104 ~ \ \ & 96 ~ \ \ & \textit{103} ~ \ \ \\ 
	 ~ Net crop & 120 ~ \ \ & 95 ~ \ \ & \textit{103} ~ \ \ \\ 
	 ~ Cereal & 187 ~ \ \ & 90 ~ \ \ & \textit{115} ~ \ \ \\ 
	 ~ Vegetable oils & 115 ~ \ \ & 101 ~ \ \ & \textit{99} ~ \ \ \\ 
	 ~ Roots and tubers & 45 ~ \ \ & 93 ~ \ \ & \textit{91} ~ \ \ \\ 
	 ~ Fruit and vegetables & 98 ~ \ \ & 97 ~ \ \ & \textit{97} ~ \ \ \\ 
	 ~ Sugar &  ~ \ \ &  ~ \ \ &  ~ \ \ \\ 
	 ~ Livestock & 79 ~ \ \ & 98 ~ \ \ & \textit{103} ~ \ \ \\ 
	 ~ Milk & 84 ~ \ \ & 95 ~ \ \ & \textit{89} ~ \ \ \\ 
	 ~ Meat & 73 ~ \ \ & 98 ~ \ \ & \textit{105} ~ \ \ \\ 
	 ~ Fish  & 157 ~ \ \ & 100 ~ \ \ & \textit{104} ~ \ \ \\ 
	\multicolumn{4}{l}{\textit{\normalsize{Net trade (min US\$)}}} \\ 
	 ~ Cereals & -205 ~ \ \ & -254 ~ \ \ & \textit{-346} ~ \ \ \\ 
	 ~ Fruit and vegetables & 15 ~ \ \ & -5 ~ \ \ & \textit{6} ~ \ \ \\ 
	 ~ Meat & -13 ~ \ \ & -8 ~ \ \ & \textit{-3} ~ \ \ \\ 
	 ~ Dairy products &  ~ \ \ &  ~ \ \ &  ~ \ \ \\ 
	 ~ Fish & 72 ~ \ \ & 187 ~ \ \ & \textit{14} ~ \ \ \\ 
	\multicolumn{4}{l}{\textcolor{FAOblue}{\textbf{\large{Environment}}}} \\ 
	 ~ Forest area (\%) & 66 ~ \ \ & 55 ~ \ \ & \textit{45} ~ \ \ \\ 
	 ~ Renewable water res withdrawn (\% of total) &  ~ \ \ & \textit{76} ~ \ \ & 76 ~ \ \ \\ 
	 ~ Terrestrial protect areas (\% total land area)  & 4 ~ \ \ & 4 ~ \ \ & \textit{2} ~ \ \ \\ 
	 ~ Organic area (\% total agricultural area) &  ~ \ \ &  ~ \ \ &  ~ \ \ \\ 
	 ~ Water withdrawal by agriculture (\% of total) &  ~ \ \ & \textit{76} ~ \ \ & 76 ~ \ \ \\ 
	 ~ Biofuel production (thousand kt of oil eq.) &  ~ \ \ &  ~ \ \ &  ~ \ \ \\ 
	 ~ Wood pellet prod. (min tonnes) &  ~ \ \ &  ~ \ \ &  ~ \ \ \\ 
	 ~ GHG emissions from ag (Co2 eq, gigagrams) & 21 ~ \ \ & 17 ~ \ \ & \textit{19} ~ \ \ \\ 
       \toprule
      \end{tabular}
      \clearpage
\CountryData{ Norway }
      \rowcolors{1}{FAOblue!10}{white}
      \begin{tabular}{L{3.9cm} R{1cm} R{1cm} R{1cm}}
      \toprule
      \multicolumn{1}{c}{} & \multicolumn{1}{c}{ 1992 } & \multicolumn{1}{c}{ 2002 } & \multicolumn{1}{c}{ 2014 } \\
      \midrule
	\multicolumn{4}{l}{\textcolor{FAOblue}{\textbf{\large{The setting}}}} \\ 
	 ~ Population, total (mln) & 4.3 ~ \ \ & 4.5 ~ \ \ & 5.1 ~ \ \ \\ 
	 ~ Population, rural (\% total population) & 1.2 ~ \ \ & 1 ~ \ \ & 1 ~ \ \ \\ 
	 ~ Govt expenditure on ag (\% total outlays) &  ~ \ \ &  ~ \ \ &  ~ \ \ \\ 
	 ~ Area harvested (mln ha) & 2 ~ \ \ & 4 ~ \ \ & 1 ~ \ \ \\ 
	 ~ Cropping intensity ratio (\%) & 2.5 ~ \ \ & 3.4 ~ \ \ &  ~ \ \ \\ 
	 ~ Water resources (m\textsuperscript{3}/person/year) & \textit{91} ~ \ \ & \textit{86} ~ \ \ & \textit{78} ~ \ \ \\ 
	 ~ Area equipped for irrigation (1000 ha) &  ~ \ \ &  ~ \ \ & \textit{90} ~ \ \ \\ 
	 ~ Area irrigated (\%) &  ~ \ \ &  ~ \ \ & \textit{47.8} ~ \ \ \\ 
	 ~ Employment in agriculture (\%) & 5.5 ~ \ \ & 3.9 ~ \ \ & \textit{2.2} ~ \ \ \\ 
	 ~ Employment in agriculture, female (\%) & 3.2 ~ \ \ & 2 ~ \ \ & \textit{0.8} ~ \ \ \\ 
	 ~ Fertilizers, Nitrogen (nutrients per ha) &  ~ \ \ & 94.8 ~ \ \ & \textit{95} ~ \ \ \\ 
	 ~ Fertilizers, Phosphate (nutrients per ha) &  ~ \ \ & 27 ~ \ \ & \textit{19.3} ~ \ \ \\ 
	 ~ Fertilizers, Potash (nutrients per ha) &  ~ \ \ & 50.8 ~ \ \ & \textit{39.3} ~ \ \ \\ 
	 ~ Energy consump, power irrigation (mln kWh) &  ~ \ \ &  ~ \ \ &  ~ \ \ \\ 
	 ~ Agr value added per worker (constant US\$) & 23.1 ~ \ \ & 36.8 ~ \ \ & \textit{65.9} ~ \ \ \\ 
	\multicolumn{4}{l}{\textcolor{FAOblue}{\textbf{\large{Hunger dimensions}}}} \\ 
	 ~ Dietary energy supply (kcal/pc/day) &  ~ \ \ &  ~ \ \ &  ~ \ \ \\ 
	 ~ Average dietary energy supply adequacy (\%) & 127 ~ \ \ & 136 ~ \ \ & 138 ~ \ \ \\ 
	 ~ Dietary en supp, cereals/roots/tubers (\%) & 33 ~ \ \ & 32 ~ \ \ & \textit{31} ~ \ \ \\ 
	 ~ Prevalence of undernourishment (\%) & <5.0 ~ \ \ & <5.0 ~ \ \ & <5.0 ~ \ \ \\ 
	 ~ GDP per capita (US\$, PPP) & 45\,154 ~ \ \ & 59\,514 ~ \ \ & \textit{62\,411} ~ \ \ \\ 
	 ~ Domestic food price volatility (index) &  ~ \ \ & 11.1 ~ \ \ & 11.3 ~ \ \ \\ 
	 ~ Cereal import dependency ratio (\%) & 22 ~ \ \ & 31.9 ~ \ \ & \textit{40.2} ~ \ \ \\ 
	 ~ Underweight, children under-5 (\%) &  ~ \ \ &  ~ \ \ &  ~ \ \ \\ 
	 ~ Improved water source (\% pop) & 100 ~ \ \ & 100 ~ \ \ & \textit{100} ~ \ \ \\ 
	\multicolumn{4}{l}{\textcolor{FAOblue}{\textbf{\large{Food Supply}}}} \\ 
	 ~ Food production value, (2004-2006 mln I\$) & 1\,288 ~ \ \ & 1\,269 ~ \ \ & \textit{1\,351} ~ \ \ \\ 
	 ~ Agriculture, value added (\% GDP) & 3 ~ \ \ & 2 ~ \ \ & \textit{2} ~ \ \ \\ 
	 ~ Food exports (mln US\$)  & 232 ~ \ \ & 207 ~ \ \ & \textit{449} ~ \ \ \\ 
	 ~ Food imports (mln US\$)  & 1\,039 ~ \ \ & 1\,414 ~ \ \ & \textit{4\,706} ~ \ \ \\ 
	\multicolumn{4}{l}{\textit{\normalsize{Production indices (2004-06=100)}}} \\ 
	 ~ Net food & 100 ~ \ \ & 98 ~ \ \ & \textit{104} ~ \ \ \\ 
	 ~ Net crop & 92 ~ \ \ & 93 ~ \ \ & \textit{77} ~ \ \ \\ 
	 ~ Cereal & 72 ~ \ \ & 86 ~ \ \ & \textit{67} ~ \ \ \\ 
	 ~ Vegetable oils & 69 ~ \ \ & 162 ~ \ \ & \textit{100} ~ \ \ \\ 
	 ~ Roots and tubers & 140 ~ \ \ & 108 ~ \ \ & \textit{88} ~ \ \ \\ 
	 ~ Fruit and vegetables & 99 ~ \ \ & 96 ~ \ \ & \textit{89} ~ \ \ \\ 
	 ~ Sugar &  ~ \ \ &  ~ \ \ &  ~ \ \ \\ 
	 ~ Livestock & 101 ~ \ \ & 100 ~ \ \ & \textit{109} ~ \ \ \\ 
	 ~ Milk & 121 ~ \ \ & 106 ~ \ \ & \textit{101} ~ \ \ \\ 
	 ~ Meat & 84 ~ \ \ & 95 ~ \ \ & \textit{115} ~ \ \ \\ 
	 ~ Fish  & 84 ~ \ \ & 108 ~ \ \ & \textit{109} ~ \ \ \\ 
	\multicolumn{4}{l}{\textit{\normalsize{Net trade (min US\$)}}} \\ 
	 ~ Cereals & -133 ~ \ \ & -248 ~ \ \ & \textit{-818} ~ \ \ \\ 
	 ~ Fruit and vegetables & -404 ~ \ \ & -516 ~ \ \ & \textit{-1\,400} ~ \ \ \\ 
	 ~ Meat & 1 ~ \ \ & -31 ~ \ \ & \textit{-198} ~ \ \ \\ 
	 ~ Dairy products & 65 ~ \ \ & 45 ~ \ \ & \textit{-25} ~ \ \ \\ 
	 ~ Fish & 2\,091 ~ \ \ & 2\,938 ~ \ \ & \textit{7\,547} ~ \ \ \\ 
	\multicolumn{4}{l}{\textcolor{FAOblue}{\textbf{\large{Environment}}}} \\ 
	 ~ Forest area (\%) & 25 ~ \ \ & 26 ~ \ \ & \textit{28} ~ \ \ \\ 
	 ~ Renewable water res withdrawn (\% of total) &  ~ \ \ &  ~ \ \ & 29 ~ \ \ \\ 
	 ~ Terrestrial protect areas (\% total land area)  & 8 ~ \ \ & 11 ~ \ \ & \textit{16} ~ \ \ \\ 
	 ~ Organic area (\% total agricultural area) &  ~ \ \ & \textit{4} ~ \ \ & \textit{6} ~ \ \ \\ 
	 ~ Water withdrawal by agriculture (\% of total) &  ~ \ \ &  ~ \ \ & 29 ~ \ \ \\ 
	 ~ Biofuel production (thousand kt of oil eq.) & 0 ~ \ \ & 1 ~ \ \ & \textit{1} ~ \ \ \\ 
	 ~ Wood pellet prod. (min tonnes) &  ~ \ \ &  ~ \ \ & \textit{56} ~ \ \ \\ 
	 ~ GHG emissions from ag (Co2 eq, gigagrams) & -8 ~ \ \ & -20 ~ \ \ & \textit{-20} ~ \ \ \\ 
       \toprule
      \end{tabular}
      \clearpage
\CountryData{ West Bank and Gaza Strip }
      \rowcolors{1}{FAOblue!10}{white}
      \begin{tabular}{L{3.9cm} R{1cm} R{1cm} R{1cm}}
      \toprule
      \multicolumn{1}{c}{} & \multicolumn{1}{c}{ 1992 } & \multicolumn{1}{c}{ 2002 } & \multicolumn{1}{c}{ 2014 } \\
      \midrule
	\multicolumn{4}{l}{\textcolor{FAOblue}{\textbf{\large{The setting}}}} \\ 
	 ~ Population, total (mln) & 2.3 ~ \ \ & 3.4 ~ \ \ & 4.4 ~ \ \ \\ 
	 ~ Population, rural (\% total population) & 0.7 ~ \ \ & 0.9 ~ \ \ & 1.1 ~ \ \ \\ 
	 ~ Govt expenditure on ag (\% total outlays) &  ~ \ \ &  ~ \ \ &  ~ \ \ \\ 
	 ~ Area harvested (mln ha) & 0 ~ \ \ & 1 ~ \ \ & 0 ~ \ \ \\ 
	 ~ Cropping intensity ratio (\%) & 0 ~ \ \ & 2.1 ~ \ \ &  ~ \ \ \\ 
	 ~ Water resources (m\textsuperscript{3}/person/year) & \textit{0} ~ \ \ & \textit{0} ~ \ \ & \textit{0} ~ \ \ \\ 
	 ~ Area equipped for irrigation (1000 ha) &  ~ \ \ &  ~ \ \ & \textit{24} ~ \ \ \\ 
	 ~ Area irrigated (\%) &  ~ \ \ &  ~ \ \ &  ~ \ \ \\ 
	 ~ Employment in agriculture (\%) &  ~ \ \ & 14.8 ~ \ \ & \textit{11.5} ~ \ \ \\ 
	 ~ Employment in agriculture, female (\%) &  ~ \ \ & 29.9 ~ \ \ & \textit{23.7} ~ \ \ \\ 
	 ~ Fertilizers, Nitrogen (nutrients per ha) &  ~ \ \ &  ~ \ \ &  ~ \ \ \\ 
	 ~ Fertilizers, Phosphate (nutrients per ha) &  ~ \ \ &  ~ \ \ &  ~ \ \ \\ 
	 ~ Fertilizers, Potash (nutrients per ha) &  ~ \ \ &  ~ \ \ &  ~ \ \ \\ 
	 ~ Energy consump, power irrigation (mln kWh) &  ~ \ \ &  ~ \ \ &  ~ \ \ \\ 
	 ~ Agr value added per worker (constant US\$) & \textit{2.1} ~ \ \ & 2.2 ~ \ \ & \textit{2.5} ~ \ \ \\ 
	\multicolumn{4}{l}{\textcolor{FAOblue}{\textbf{\large{Hunger dimensions}}}} \\ 
	 ~ Dietary energy supply (kcal/pc/day) &  ~ \ \ &  ~ \ \ &  ~ \ \ \\ 
	 ~ Average dietary energy supply adequacy (\%) &  ~ \ \ &  ~ \ \ &  ~ \ \ \\ 
	 ~ Dietary en supp, cereals/roots/tubers (\%) &  ~ \ \ &  ~ \ \ &  ~ \ \ \\ 
	 ~ Prevalence of undernourishment (\%) &  ~ \ \ &  ~ \ \ &  ~ \ \ \\ 
	 ~ GDP per capita (US\$, PPP) & \textit{2\,788} ~ \ \ & 3\,684 ~ \ \ & \textit{4\,484} ~ \ \ \\ 
	 ~ Domestic food price volatility (index) &  ~ \ \ &  ~ \ \ &  ~ \ \ \\ 
	 ~ Cereal import dependency ratio (\%) & \textit{91.3} ~ \ \ & 89.8 ~ \ \ & \textit{95.5} ~ \ \ \\ 
	 ~ Underweight, children under-5 (\%) &  ~ \ \ & \textit{3.6} ~ \ \ & 1.4 ~ \ \ \\ 
	 ~ Improved water source (\% pop) & 96.5 ~ \ \ & 89.5 ~ \ \ & \textit{81.8} ~ \ \ \\ 
	\multicolumn{4}{l}{\textcolor{FAOblue}{\textbf{\large{Food Supply}}}} \\ 
	 ~ Food production value, (2004-2006 mln I\$) & \textit{289} ~ \ \ & 595 ~ \ \ & \textit{551} ~ \ \ \\ 
	 ~ Agriculture, value added (\% GDP) & \textit{13} ~ \ \ & 9 ~ \ \ & \textit{5} ~ \ \ \\ 
	 ~ Food exports (mln US\$)  & 0 ~ \ \ & 49 ~ \ \ & \textit{91} ~ \ \ \\ 
	 ~ Food imports (mln US\$)  & 38 ~ \ \ & 362 ~ \ \ & \textit{461} ~ \ \ \\ 
	\multicolumn{4}{l}{\textit{\normalsize{Production indices (2004-06=100)}}} \\ 
	 ~ Net food & \textit{49} ~ \ \ & 101 ~ \ \ & \textit{93} ~ \ \ \\ 
	 ~ Net crop & \textit{34} ~ \ \ & 99 ~ \ \ & \textit{94} ~ \ \ \\ 
	 ~ Cereal & \textit{31} ~ \ \ & 120 ~ \ \ & \textit{41} ~ \ \ \\ 
	 ~ Vegetable oils &  ~ \ \ & 91 ~ \ \ & \textit{85} ~ \ \ \\ 
	 ~ Roots and tubers &  ~ \ \ & 107 ~ \ \ & \textit{88} ~ \ \ \\ 
	 ~ Fruit and vegetables & \textit{48} ~ \ \ & 102 ~ \ \ & \textit{92} ~ \ \ \\ 
	 ~ Sugar &  ~ \ \ &  ~ \ \ &  ~ \ \ \\ 
	 ~ Livestock & \textit{75} ~ \ \ & 103 ~ \ \ & \textit{91} ~ \ \ \\ 
	 ~ Milk & \textit{70} ~ \ \ & 92 ~ \ \ & \textit{102} ~ \ \ \\ 
	 ~ Meat & \textit{84} ~ \ \ & 114 ~ \ \ & \textit{95} ~ \ \ \\ 
	 ~ Fish  & 0 ~ \ \ & 100 ~ \ \ & \textit{93} ~ \ \ \\ 
	\multicolumn{4}{l}{\textit{\normalsize{Net trade (min US\$)}}} \\ 
	 ~ Cereals & \textit{-62} ~ \ \ & -119 ~ \ \ & \textit{-119} ~ \ \ \\ 
	 ~ Fruit and vegetables & -1 ~ \ \ & -32 ~ \ \ & \textit{-74} ~ \ \ \\ 
	 ~ Meat & \textit{-11} ~ \ \ & -12 ~ \ \ & \textit{-4} ~ \ \ \\ 
	 ~ Dairy products & \textit{-27} ~ \ \ & -30 ~ \ \ & \textit{-25} ~ \ \ \\ 
	 ~ Fish &  ~ \ \ &  ~ \ \ & \textit{-10} ~ \ \ \\ 
	\multicolumn{4}{l}{\textcolor{FAOblue}{\textbf{\large{Environment}}}} \\ 
	 ~ Forest area (\%) & 2 ~ \ \ & 2 ~ \ \ & \textit{2} ~ \ \ \\ 
	 ~ Renewable water res withdrawn (\% of total) &  ~ \ \ & \textit{45} ~ \ \ & 45 ~ \ \ \\ 
	 ~ Terrestrial protect areas (\% total land area)  & 1 ~ \ \ & 1 ~ \ \ & \textit{1} ~ \ \ \\ 
	 ~ Organic area (\% total agricultural area) &  ~ \ \ & \textit{0} ~ \ \ & \textit{2} ~ \ \ \\ 
	 ~ Water withdrawal by agriculture (\% of total) &  ~ \ \ & \textit{45} ~ \ \ & 45 ~ \ \ \\ 
	 ~ Biofuel production (thousand kt of oil eq.) &  ~ \ \ &  ~ \ \ &  ~ \ \ \\ 
	 ~ Wood pellet prod. (min tonnes) &  ~ \ \ &  ~ \ \ &  ~ \ \ \\ 
	 ~ GHG emissions from ag (Co2 eq, gigagrams) & 0 ~ \ \ & 0 ~ \ \ & \textit{0} ~ \ \ \\ 
       \toprule
      \end{tabular}
      \clearpage
\CountryData{ Pakistan }
      \rowcolors{1}{FAOblue!10}{white}
      \begin{tabular}{L{3.9cm} R{1cm} R{1cm} R{1cm}}
      \toprule
      \multicolumn{1}{c}{} & \multicolumn{1}{c}{ 1992 } & \multicolumn{1}{c}{ 2002 } & \multicolumn{1}{c}{ 2014 } \\
      \midrule
	\multicolumn{4}{l}{\textcolor{FAOblue}{\textbf{\large{The setting}}}} \\ 
	 ~ Population, total (mln) & 117.3 ~ \ \ & 149.7 ~ \ \ & 185.1 ~ \ \ \\ 
	 ~ Population, rural (\% total population) & 80.8 ~ \ \ & 99.3 ~ \ \ & 116.3 ~ \ \ \\ 
	 ~ Govt expenditure on ag (\% total outlays) &  ~ \ \ & 0.7 ~ \ \ & \textit{1.2} ~ \ \ \\ 
	 ~ Area harvested (mln ha) & 39 ~ \ \ & 48 ~ \ \ & 64 ~ \ \ \\ 
	 ~ Cropping intensity ratio (\%) & 1.5 ~ \ \ & 1.8 ~ \ \ &  ~ \ \ \\ 
	 ~ Water resources (m\textsuperscript{3}/person/year) & \textit{2} ~ \ \ & \textit{2} ~ \ \ & \textit{1} ~ \ \ \\ 
	 ~ Area equipped for irrigation (1000 ha) &  ~ \ \ &  ~ \ \ & \textit{20\,200} ~ \ \ \\ 
	 ~ Area irrigated (\%) &  ~ \ \ &  ~ \ \ &  ~ \ \ \\ 
	 ~ Employment in agriculture (\%) & 48.3 ~ \ \ & 42.1 ~ \ \ & \textit{43.7} ~ \ \ \\ 
	 ~ Employment in agriculture, female (\%) & 68.8 ~ \ \ & 64.6 ~ \ \ & \textit{75.7} ~ \ \ \\ 
	 ~ Fertilizers, Nitrogen (nutrients per ha) &  ~ \ \ & 87.5 ~ \ \ & \textit{105.3} ~ \ \ \\ 
	 ~ Fertilizers, Phosphate (nutrients per ha) &  ~ \ \ & 23.7 ~ \ \ & \textit{24.8} ~ \ \ \\ 
	 ~ Fertilizers, Potash (nutrients per ha) &  ~ \ \ & 0.4 ~ \ \ & \textit{0.7} ~ \ \ \\ 
	 ~ Energy consump, power irrigation (mln kWh) &  ~ \ \ &  ~ \ \ &  ~ \ \ \\ 
	 ~ Agr value added per worker (constant US\$) & 0.9 ~ \ \ & 1 ~ \ \ & \textit{1.1} ~ \ \ \\ 
	\multicolumn{4}{l}{\textcolor{FAOblue}{\textbf{\large{Hunger dimensions}}}} \\ 
	 ~ Dietary energy supply (kcal/pc/day) & 2\,300 ~ \ \ & 2\,293 ~ \ \ & 2\,444 ~ \ \ \\ 
	 ~ Average dietary energy supply adequacy (\%) & 108 ~ \ \ & 105 ~ \ \ & 108 ~ \ \ \\ 
	 ~ Dietary en supp, cereals/roots/tubers (\%) & 54 ~ \ \ & 50 ~ \ \ & \textit{50} ~ \ \ \\ 
	 ~ Prevalence of undernourishment (\%) & 25.7 ~ \ \ & 24.8 ~ \ \ & 22 ~ \ \ \\ 
	 ~ GDP per capita (US\$, PPP) & 3\,174 ~ \ \ & 3\,405 ~ \ \ & \textit{4\,454} ~ \ \ \\ 
	 ~ Domestic food price volatility (index) &  ~ \ \ & 9.5 ~ \ \ & 13.2 ~ \ \ \\ 
	 ~ Cereal import dependency ratio (\%) & 3.4 ~ \ \ & -13.4 ~ \ \ & \textit{-12.2} ~ \ \ \\ 
	 ~ Underweight, children under-5 (\%) & \textit{34.2} ~ \ \ & \textit{31.3} ~ \ \ & \textit{31.6} ~ \ \ \\ 
	 ~ Improved water source (\% pop) & 85.9 ~ \ \ & 88.9 ~ \ \ & \textit{91.4} ~ \ \ \\ 
	\multicolumn{4}{l}{\textcolor{FAOblue}{\textbf{\large{Food Supply}}}} \\ 
	 ~ Food production value, (2004-2006 mln I\$) & 18\,788 ~ \ \ & 25\,680 ~ \ \ & \textit{27\,300} ~ \ \ \\ 
	 ~ Agriculture, value added (\% GDP) & 26 ~ \ \ & 23 ~ \ \ & \textit{25} ~ \ \ \\ 
	 ~ Food exports (mln US\$)  & 582 ~ \ \ & 862 ~ \ \ & \textit{3\,621} ~ \ \ \\ 
	 ~ Food imports (mln US\$)  & 944 ~ \ \ & 957 ~ \ \ & \textit{4\,018} ~ \ \ \\ 
	\multicolumn{4}{l}{\textit{\normalsize{Production indices (2004-06=100)}}} \\ 
	 ~ Net food & 65 ~ \ \ & 88 ~ \ \ & \textit{94} ~ \ \ \\ 
	 ~ Net crop & 69 ~ \ \ & 84 ~ \ \ & \textit{107} ~ \ \ \\ 
	 ~ Cereal & 67 ~ \ \ & 84 ~ \ \ & \textit{108} ~ \ \ \\ 
	 ~ Vegetable oils & 60 ~ \ \ & 73 ~ \ \ & \textit{93} ~ \ \ \\ 
	 ~ Roots and tubers & 52 ~ \ \ & 93 ~ \ \ & \textit{187} ~ \ \ \\ 
	 ~ Fruit and vegetables & 65 ~ \ \ & 85 ~ \ \ & \textit{104} ~ \ \ \\ 
	 ~ Sugar & 81 ~ \ \ & 100 ~ \ \ & \textit{131} ~ \ \ \\ 
	 ~ Livestock & 62 ~ \ \ & 90 ~ \ \ & \textit{81} ~ \ \ \\ 
	 ~ Milk & 56 ~ \ \ & 91 ~ \ \ & \textit{51} ~ \ \ \\ 
	 ~ Meat & 76 ~ \ \ & 89 ~ \ \ & \textit{145} ~ \ \ \\ 
	 ~ Fish  & 99 ~ \ \ & 107 ~ \ \ & \textit{111} ~ \ \ \\ 
	\multicolumn{4}{l}{\textit{\normalsize{Net trade (min US\$)}}} \\ 
	 ~ Cereals & 63 ~ \ \ & 527 ~ \ \ & \textit{2\,125} ~ \ \ \\ 
	 ~ Fruit and vegetables & -38 ~ \ \ & -107 ~ \ \ & \textit{-163} ~ \ \ \\ 
	 ~ Meat & 1 ~ \ \ & 3 ~ \ \ & \textit{204} ~ \ \ \\ 
	 ~ Dairy products & -31 ~ \ \ & -7 ~ \ \ & \textit{-45} ~ \ \ \\ 
	 ~ Fish & 115 ~ \ \ & 131 ~ \ \ & \textit{277} ~ \ \ \\ 
	\multicolumn{4}{l}{\textcolor{FAOblue}{\textbf{\large{Environment}}}} \\ 
	 ~ Forest area (\%) & 3 ~ \ \ & 3 ~ \ \ & \textit{2} ~ \ \ \\ 
	 ~ Renewable water res withdrawn (\% of total) &  ~ \ \ &  ~ \ \ & 94 ~ \ \ \\ 
	 ~ Terrestrial protect areas (\% total land area)  & 10 ~ \ \ & 10 ~ \ \ & \textit{11} ~ \ \ \\ 
	 ~ Organic area (\% total agricultural area) &  ~ \ \ & \textit{0} ~ \ \ & \textit{0} ~ \ \ \\ 
	 ~ Water withdrawal by agriculture (\% of total) &  ~ \ \ &  ~ \ \ & 94 ~ \ \ \\ 
	 ~ Biofuel production (thousand kt of oil eq.) & 165 ~ \ \ & 218 ~ \ \ & \textit{255} ~ \ \ \\ 
	 ~ Wood pellet prod. (min tonnes) &  ~ \ \ &  ~ \ \ &  ~ \ \ \\ 
	 ~ GHG emissions from ag (Co2 eq, gigagrams) & 102 ~ \ \ & 124 ~ \ \ & \textit{161} ~ \ \ \\ 
       \toprule
      \end{tabular}
      \clearpage
\CountryData{ Panama }
      \rowcolors{1}{FAOblue!10}{white}
      \begin{tabular}{L{3.9cm} R{1cm} R{1cm} R{1cm}}
      \toprule
      \multicolumn{1}{c}{} & \multicolumn{1}{c}{ 1992 } & \multicolumn{1}{c}{ 2002 } & \multicolumn{1}{c}{ 2014 } \\
      \midrule
	\multicolumn{4}{l}{\textcolor{FAOblue}{\textbf{\large{The setting}}}} \\ 
	 ~ Population, total (mln) & 2.6 ~ \ \ & 3.2 ~ \ \ & 3.9 ~ \ \ \\ 
	 ~ Population, rural (\% total population) & 1.1 ~ \ \ & 1 ~ \ \ & 0.9 ~ \ \ \\ 
	 ~ Govt expenditure on ag (\% total outlays) &  ~ \ \ & \textit{3.5} ~ \ \ & \textit{1.8} ~ \ \ \\ 
	 ~ Area harvested (mln ha) & 2 ~ \ \ & 2 ~ \ \ & 2 ~ \ \ \\ 
	 ~ Cropping intensity ratio (\%) & 0.8 ~ \ \ & 0.7 ~ \ \ &  ~ \ \ \\ 
	 ~ Water resources (m\textsuperscript{3}/person/year) & \textit{53} ~ \ \ & \textit{43} ~ \ \ & \textit{36} ~ \ \ \\ 
	 ~ Area equipped for irrigation (1000 ha) &  ~ \ \ &  ~ \ \ & \textit{32} ~ \ \ \\ 
	 ~ Area irrigated (\%) &  ~ \ \ &  ~ \ \ & \textit{100} ~ \ \ \\ 
	 ~ Employment in agriculture (\%) & 26.3 ~ \ \ & 21.1 ~ \ \ & \textit{16.7} ~ \ \ \\ 
	 ~ Employment in agriculture, female (\%) & 3.1 ~ \ \ & 5.7 ~ \ \ & \textit{8.5} ~ \ \ \\ 
	 ~ Fertilizers, Nitrogen (nutrients per ha) &  ~ \ \ & 6.8 ~ \ \ & \textit{7.6} ~ \ \ \\ 
	 ~ Fertilizers, Phosphate (nutrients per ha) &  ~ \ \ & 1.8 ~ \ \ & \textit{4.7} ~ \ \ \\ 
	 ~ Fertilizers, Potash (nutrients per ha) &  ~ \ \ & 1.3 ~ \ \ & \textit{3} ~ \ \ \\ 
	 ~ Energy consump, power irrigation (mln kWh) &  ~ \ \ & 19 ~ \ \ & \textit{18} ~ \ \ \\ 
	 ~ Agr value added per worker (constant US\$) & 2.4 ~ \ \ & 3.4 ~ \ \ & \textit{4} ~ \ \ \\ 
	\multicolumn{4}{l}{\textcolor{FAOblue}{\textbf{\large{Hunger dimensions}}}} \\ 
	 ~ Dietary energy supply (kcal/pc/day) & 2\,282 ~ \ \ & 2\,326 ~ \ \ & 2\,745 ~ \ \ \\ 
	 ~ Average dietary energy supply adequacy (\%) & 103 ~ \ \ & 103 ~ \ \ & 120 ~ \ \ \\ 
	 ~ Dietary en supp, cereals/roots/tubers (\%) & 42 ~ \ \ & 45 ~ \ \ & \textit{44} ~ \ \ \\ 
	 ~ Prevalence of undernourishment (\%) & 25.9 ~ \ \ & 26.3 ~ \ \ & 10 ~ \ \ \\ 
	 ~ GDP per capita (US\$, PPP) & 8\,476 ~ \ \ & 9\,836 ~ \ \ & \textit{18\,793} ~ \ \ \\ 
	 ~ Domestic food price volatility (index) &  ~ \ \ & 4.2 ~ \ \ & 2.1 ~ \ \ \\ 
	 ~ Cereal import dependency ratio (\%) & 43.5 ~ \ \ & 61.8 ~ \ \ & \textit{69.4} ~ \ \ \\ 
	 ~ Underweight, children under-5 (\%) &  ~ \ \ & \textit{5.1} ~ \ \ & \textit{3.9} ~ \ \ \\ 
	 ~ Improved water source (\% pop) & 85.4 ~ \ \ & 91.2 ~ \ \ & \textit{94.3} ~ \ \ \\ 
	\multicolumn{4}{l}{\textcolor{FAOblue}{\textbf{\large{Food Supply}}}} \\ 
	 ~ Food production value, (2004-2006 mln I\$) & 713 ~ \ \ & 790 ~ \ \ & \textit{953} ~ \ \ \\ 
	 ~ Agriculture, value added (\% GDP) & 8 ~ \ \ & 8 ~ \ \ & \textit{3} ~ \ \ \\ 
	 ~ Food exports (mln US\$)  & 260 ~ \ \ & 247 ~ \ \ & \textit{287} ~ \ \ \\ 
	 ~ Food imports (mln US\$)  & 166 ~ \ \ & 313 ~ \ \ & \textit{1\,031} ~ \ \ \\ 
	\multicolumn{4}{l}{\textit{\normalsize{Production indices (2004-06=100)}}} \\ 
	 ~ Net food & 87 ~ \ \ & 97 ~ \ \ & \textit{117} ~ \ \ \\ 
	 ~ Net crop & 110 ~ \ \ & 96 ~ \ \ & \textit{98} ~ \ \ \\ 
	 ~ Cereal & 95 ~ \ \ & 116 ~ \ \ & \textit{125} ~ \ \ \\ 
	 ~ Vegetable oils & 5 ~ \ \ & 86 ~ \ \ & \textit{82} ~ \ \ \\ 
	 ~ Roots and tubers & 64 ~ \ \ & 93 ~ \ \ & \textit{87} ~ \ \ \\ 
	 ~ Fruit and vegetables & 126 ~ \ \ & 93 ~ \ \ & \textit{81} ~ \ \ \\ 
	 ~ Sugar & 94 ~ \ \ & 91 ~ \ \ & \textit{139} ~ \ \ \\ 
	 ~ Livestock & 66 ~ \ \ & 96 ~ \ \ & \textit{135} ~ \ \ \\ 
	 ~ Milk & 78 ~ \ \ & 99 ~ \ \ & \textit{115} ~ \ \ \\ 
	 ~ Meat & 66 ~ \ \ & 96 ~ \ \ & \textit{141} ~ \ \ \\ 
	 ~ Fish  & 69 ~ \ \ & 102 ~ \ \ & \textit{82} ~ \ \ \\ 
	\multicolumn{4}{l}{\textit{\normalsize{Net trade (min US\$)}}} \\ 
	 ~ Cereals & -58 ~ \ \ & -110 ~ \ \ & \textit{-376} ~ \ \ \\ 
	 ~ Fruit and vegetables & 181 ~ \ \ & 107 ~ \ \ & \textit{-21} ~ \ \ \\ 
	 ~ Meat & -7 ~ \ \ & -3 ~ \ \ & \textit{-49} ~ \ \ \\ 
	 ~ Dairy products & -10 ~ \ \ & -8 ~ \ \ & \textit{-60} ~ \ \ \\ 
	 ~ Fish & 69 ~ \ \ & 315 ~ \ \ & \textit{68} ~ \ \ \\ 
	\multicolumn{4}{l}{\textcolor{FAOblue}{\textbf{\large{Environment}}}} \\ 
	 ~ Forest area (\%) & 50 ~ \ \ & 45 ~ \ \ & \textit{43} ~ \ \ \\ 
	 ~ Renewable water res withdrawn (\% of total) &  ~ \ \ &  ~ \ \ & 43 ~ \ \ \\ 
	 ~ Terrestrial protect areas (\% total land area)  & 17 ~ \ \ & 19 ~ \ \ & \textit{21} ~ \ \ \\ 
	 ~ Organic area (\% total agricultural area) &  ~ \ \ & \textit{0} ~ \ \ & \textit{0} ~ \ \ \\ 
	 ~ Water withdrawal by agriculture (\% of total) &  ~ \ \ &  ~ \ \ & 43 ~ \ \ \\ 
	 ~ Biofuel production (thousand kt of oil eq.) & 4 ~ \ \ & 6 ~ \ \ & \textit{7} ~ \ \ \\ 
	 ~ Wood pellet prod. (min tonnes) &  ~ \ \ &  ~ \ \ &  ~ \ \ \\ 
	 ~ GHG emissions from ag (Co2 eq, gigagrams) & 22 ~ \ \ & 9 ~ \ \ & \textit{10} ~ \ \ \\ 
       \toprule
      \end{tabular}
      \clearpage
\CountryData{ Paraguay }
      \rowcolors{1}{FAOblue!10}{white}
      \begin{tabular}{L{3.9cm} R{1cm} R{1cm} R{1cm}}
      \toprule
      \multicolumn{1}{c}{} & \multicolumn{1}{c}{ 1992 } & \multicolumn{1}{c}{ 2002 } & \multicolumn{1}{c}{ 2014 } \\
      \midrule
	\multicolumn{4}{l}{\textcolor{FAOblue}{\textbf{\large{The setting}}}} \\ 
	 ~ Population, total (mln) & 4.5 ~ \ \ & 5.6 ~ \ \ & 6.9 ~ \ \ \\ 
	 ~ Population, rural (\% total population) & 2.2 ~ \ \ & 2.4 ~ \ \ & 2.5 ~ \ \ \\ 
	 ~ Govt expenditure on ag (\% total outlays) &  ~ \ \ &  ~ \ \ & \textit{3.4} ~ \ \ \\ 
	 ~ Area harvested (mln ha) & 3 ~ \ \ & 5 ~ \ \ & 9 ~ \ \ \\ 
	 ~ Cropping intensity ratio (\%) & 0.2 ~ \ \ & 0.2 ~ \ \ &  ~ \ \ \\ 
	 ~ Water resources (m\textsuperscript{3}/person/year) & \textit{85} ~ \ \ & \textit{68} ~ \ \ & \textit{57} ~ \ \ \\ 
	 ~ Area equipped for irrigation (1000 ha) &  ~ \ \ &  ~ \ \ & \textit{136} ~ \ \ \\ 
	 ~ Area irrigated (\%) &  ~ \ \ &  ~ \ \ & \textit{100} ~ \ \ \\ 
	 ~ Employment in agriculture (\%) & 1.9 ~ \ \ & 34.3 ~ \ \ & \textit{27.2} ~ \ \ \\ 
	 ~ Employment in agriculture, female (\%) & 0.4 ~ \ \ & 20 ~ \ \ & \textit{23} ~ \ \ \\ 
	 ~ Fertilizers, Nitrogen (nutrients per ha) &  ~ \ \ & 1.8 ~ \ \ & \textit{4.5} ~ \ \ \\ 
	 ~ Fertilizers, Phosphate (nutrients per ha) &  ~ \ \ & 3.5 ~ \ \ & \textit{7} ~ \ \ \\ 
	 ~ Fertilizers, Potash (nutrients per ha) &  ~ \ \ & 2.5 ~ \ \ & \textit{5.6} ~ \ \ \\ 
	 ~ Energy consump, power irrigation (mln kWh) &  ~ \ \ &  ~ \ \ &  ~ \ \ \\ 
	 ~ Agr value added per worker (constant US\$) & 1.7 ~ \ \ & 1.9 ~ \ \ & \textit{3.1} ~ \ \ \\ 
	\multicolumn{4}{l}{\textcolor{FAOblue}{\textbf{\large{Hunger dimensions}}}} \\ 
	 ~ Dietary energy supply (kcal/pc/day) & 2\,406 ~ \ \ & 2\,684 ~ \ \ & 2\,645 ~ \ \ \\ 
	 ~ Average dietary energy supply adequacy (\%) & 108 ~ \ \ & 118 ~ \ \ & 113 ~ \ \ \\ 
	 ~ Dietary en supp, cereals/roots/tubers (\%) & 43 ~ \ \ & 42 ~ \ \ & \textit{44} ~ \ \ \\ 
	 ~ Prevalence of undernourishment (\%) & 19.9 ~ \ \ & 12.1 ~ \ \ & 11.1 ~ \ \ \\ 
	 ~ GDP per capita (US\$, PPP) & 5\,988 ~ \ \ & 5\,742 ~ \ \ & \textit{7\,833} ~ \ \ \\ 
	 ~ Domestic food price volatility (index) &  ~ \ \ & 9.4 ~ \ \ & \textit{11.2} ~ \ \ \\ 
	 ~ Cereal import dependency ratio (\%) & 8.3 ~ \ \ & -42.5 ~ \ \ & \textit{-144.3} ~ \ \ \\ 
	 ~ Underweight, children under-5 (\%) & \textit{2.8} ~ \ \ & \textit{3.4} ~ \ \ & \textit{3.4} ~ \ \ \\ 
	 ~ Improved water source (\% pop) & 57.5 ~ \ \ & 77.2 ~ \ \ & \textit{93.8} ~ \ \ \\ 
	\multicolumn{4}{l}{\textcolor{FAOblue}{\textbf{\large{Food Supply}}}} \\ 
	 ~ Food production value, (2004-2006 mln I\$) & 2\,034 ~ \ \ & 2\,750 ~ \ \ & \textit{5\,562} ~ \ \ \\ 
	 ~ Agriculture, value added (\% GDP) & 16 ~ \ \ & 15 ~ \ \ & \textit{22} ~ \ \ \\ 
	 ~ Food exports (mln US\$)  & 231 ~ \ \ & 361 ~ \ \ & \textit{3\,786} ~ \ \ \\ 
	 ~ Food imports (mln US\$)  & 64 ~ \ \ & 93 ~ \ \ & \textit{471} ~ \ \ \\ 
	\multicolumn{4}{l}{\textit{\normalsize{Production indices (2004-06=100)}}} \\ 
	 ~ Net food & 64 ~ \ \ & 87 ~ \ \ & \textit{176} ~ \ \ \\ 
	 ~ Net crop & 60 ~ \ \ & 83 ~ \ \ & \textit{169} ~ \ \ \\ 
	 ~ Cereal & 46 ~ \ \ & 71 ~ \ \ & \textit{298} ~ \ \ \\ 
	 ~ Vegetable oils & 46 ~ \ \ & 83 ~ \ \ & \textit{215} ~ \ \ \\ 
	 ~ Roots and tubers & 52 ~ \ \ & 88 ~ \ \ & \textit{55} ~ \ \ \\ 
	 ~ Fruit and vegetables & 96 ~ \ \ & 94 ~ \ \ & \textit{92} ~ \ \ \\ 
	 ~ Sugar & 78 ~ \ \ & 90 ~ \ \ & \textit{156} ~ \ \ \\ 
	 ~ Livestock & 81 ~ \ \ & 93 ~ \ \ & \textit{134} ~ \ \ \\ 
	 ~ Milk & 68 ~ \ \ & 100 ~ \ \ & \textit{138} ~ \ \ \\ 
	 ~ Meat & 86 ~ \ \ & 93 ~ \ \ & \textit{134} ~ \ \ \\ 
	 ~ Fish  & 75 ~ \ \ & 105 ~ \ \ & \textit{95} ~ \ \ \\ 
	\multicolumn{4}{l}{\textit{\normalsize{Net trade (min US\$)}}} \\ 
	 ~ Cereals & -19 ~ \ \ & -23 ~ \ \ & \textit{891} ~ \ \ \\ 
	 ~ Fruit and vegetables & 1 ~ \ \ & -13 ~ \ \ & \textit{-42} ~ \ \ \\ 
	 ~ Meat & 47 ~ \ \ & 20 ~ \ \ & \textit{781} ~ \ \ \\ 
	 ~ Dairy products & -8 ~ \ \ & -7 ~ \ \ & \textit{-20} ~ \ \ \\ 
	 ~ Fish & -1 ~ \ \ & -1 ~ \ \ & \textit{-7} ~ \ \ \\ 
	\multicolumn{4}{l}{\textcolor{FAOblue}{\textbf{\large{Environment}}}} \\ 
	 ~ Forest area (\%) & 52 ~ \ \ & 48 ~ \ \ & \textit{43} ~ \ \ \\ 
	 ~ Renewable water res withdrawn (\% of total) &  ~ \ \ &  ~ \ \ & 79 ~ \ \ \\ 
	 ~ Terrestrial protect areas (\% total land area)  & 4 ~ \ \ & 4 ~ \ \ & \textit{6} ~ \ \ \\ 
	 ~ Organic area (\% total agricultural area) &  ~ \ \ &  ~ \ \ & \textit{0} ~ \ \ \\ 
	 ~ Water withdrawal by agriculture (\% of total) &  ~ \ \ &  ~ \ \ & 79 ~ \ \ \\ 
	 ~ Biofuel production (thousand kt of oil eq.) & 3 ~ \ \ & 3 ~ \ \ & \textit{4} ~ \ \ \\ 
	 ~ Wood pellet prod. (min tonnes) &  ~ \ \ &  ~ \ \ &  ~ \ \ \\ 
	 ~ GHG emissions from ag (Co2 eq, gigagrams) & 86 ~ \ \ & 92 ~ \ \ & \textit{100} ~ \ \ \\ 
       \toprule
      \end{tabular}
      \clearpage
\CountryData{ Peru }
      \rowcolors{1}{FAOblue!10}{white}
      \begin{tabular}{L{3.9cm} R{1cm} R{1cm} R{1cm}}
      \toprule
      \multicolumn{1}{c}{} & \multicolumn{1}{c}{ 1992 } & \multicolumn{1}{c}{ 2002 } & \multicolumn{1}{c}{ 2014 } \\
      \midrule
	\multicolumn{4}{l}{\textcolor{FAOblue}{\textbf{\large{The setting}}}} \\ 
	 ~ Population, total (mln) & 22.6 ~ \ \ & 26.7 ~ \ \ & 30.8 ~ \ \ \\ 
	 ~ Population, rural (\% total population) & 6.9 ~ \ \ & 7 ~ \ \ & 6.7 ~ \ \ \\ 
	 ~ Govt expenditure on ag (\% total outlays) &  ~ \ \ &  ~ \ \ &  ~ \ \ \\ 
	 ~ Area harvested (mln ha) & 6 ~ \ \ & 9 ~ \ \ & 11 ~ \ \ \\ 
	 ~ Cropping intensity ratio (\%) & 0.3 ~ \ \ & 0.4 ~ \ \ &  ~ \ \ \\ 
	 ~ Water resources (m\textsuperscript{3}/person/year) & \textit{82} ~ \ \ & \textit{70} ~ \ \ & \textit{62} ~ \ \ \\ 
	 ~ Area equipped for irrigation (1000 ha) &  ~ \ \ &  ~ \ \ & \textit{2\,580} ~ \ \ \\ 
	 ~ Area irrigated (\%) &  ~ \ \ &  ~ \ \ & \textit{70.1} ~ \ \ \\ 
	 ~ Employment in agriculture (\%) & 0.8 ~ \ \ & 0.9 ~ \ \ & \textit{25.8} ~ \ \ \\ 
	 ~ Employment in agriculture, female (\%) & 0.3 ~ \ \ & 0.5 ~ \ \ & \textit{22.5} ~ \ \ \\ 
	 ~ Fertilizers, Nitrogen (nutrients per ha) &  ~ \ \ & 9.3 ~ \ \ & \textit{11.5} ~ \ \ \\ 
	 ~ Fertilizers, Phosphate (nutrients per ha) &  ~ \ \ & 2.2 ~ \ \ & \textit{4.3} ~ \ \ \\ 
	 ~ Fertilizers, Potash (nutrients per ha) &  ~ \ \ & 2 ~ \ \ & \textit{2} ~ \ \ \\ 
	 ~ Energy consump, power irrigation (mln kWh) & \textit{70} ~ \ \ & 70 ~ \ \ & \textit{70} ~ \ \ \\ 
	 ~ Agr value added per worker (constant US\$) &  ~ \ \ &  ~ \ \ &  ~ \ \ \\ 
	\multicolumn{4}{l}{\textcolor{FAOblue}{\textbf{\large{Hunger dimensions}}}} \\ 
	 ~ Dietary energy supply (kcal/pc/day) & 2\,159 ~ \ \ & 2\,312 ~ \ \ & 2\,733 ~ \ \ \\ 
	 ~ Average dietary energy supply adequacy (\%) & 100 ~ \ \ & 105 ~ \ \ & 120 ~ \ \ \\ 
	 ~ Dietary en supp, cereals/roots/tubers (\%) & 59 ~ \ \ & 58 ~ \ \ & \textit{56} ~ \ \ \\ 
	 ~ Prevalence of undernourishment (\%) & 28.1 ~ \ \ & 20.6 ~ \ \ & 8.1 ~ \ \ \\ 
	 ~ GDP per capita (US\$, PPP) & 5\,160 ~ \ \ & 6\,693 ~ \ \ & \textit{11\,396} ~ \ \ \\ 
	 ~ Domestic food price volatility (index) &  ~ \ \ & 4.8 ~ \ \ & 3.4 ~ \ \ \\ 
	 ~ Cereal import dependency ratio (\%) & 60.2 ~ \ \ & 43.7 ~ \ \ & \textit{48.4} ~ \ \ \\ 
	 ~ Underweight, children under-5 (\%) & \textit{8.8} ~ \ \ & \textit{5.4} ~ \ \ & \textit{3.5} ~ \ \ \\ 
	 ~ Improved water source (\% pop) & 75.7 ~ \ \ & 81.8 ~ \ \ & \textit{86.8} ~ \ \ \\ 
	\multicolumn{4}{l}{\textcolor{FAOblue}{\textbf{\large{Food Supply}}}} \\ 
	 ~ Food production value, (2004-2006 mln I\$) & 3\,058 ~ \ \ & 5\,613 ~ \ \ & \textit{9\,145} ~ \ \ \\ 
	 ~ Agriculture, value added (\% GDP) &  ~ \ \ &  ~ \ \ & \textit{7} ~ \ \ \\ 
	 ~ Food exports (mln US\$)  & 144 ~ \ \ & 480 ~ \ \ & \textit{2\,738} ~ \ \ \\ 
	 ~ Food imports (mln US\$)  & 770 ~ \ \ & 790 ~ \ \ & \textit{3\,238} ~ \ \ \\ 
	\multicolumn{4}{l}{\textit{\normalsize{Production indices (2004-06=100)}}} \\ 
	 ~ Net food & 49 ~ \ \ & 90 ~ \ \ & \textit{147} ~ \ \ \\ 
	 ~ Net crop & 42 ~ \ \ & 91 ~ \ \ & \textit{140} ~ \ \ \\ 
	 ~ Cereal & 38 ~ \ \ & 98 ~ \ \ & \textit{137} ~ \ \ \\ 
	 ~ Vegetable oils & 55 ~ \ \ & 75 ~ \ \ & \textit{284} ~ \ \ \\ 
	 ~ Roots and tubers & 36 ~ \ \ & 102 ~ \ \ & \textit{138} ~ \ \ \\ 
	 ~ Fruit and vegetables & 44 ~ \ \ & 88 ~ \ \ & \textit{151} ~ \ \ \\ 
	 ~ Sugar & 77 ~ \ \ & 123 ~ \ \ & \textit{153} ~ \ \ \\ 
	 ~ Livestock & 59 ~ \ \ & 88 ~ \ \ & \textit{153} ~ \ \ \\ 
	 ~ Milk & 57 ~ \ \ & 88 ~ \ \ & \textit{133} ~ \ \ \\ 
	 ~ Meat & 60 ~ \ \ & 87 ~ \ \ & \textit{156} ~ \ \ \\ 
	 ~ Fish  & 86 ~ \ \ & 101 ~ \ \ & \textit{69} ~ \ \ \\ 
	\multicolumn{4}{l}{\textit{\normalsize{Net trade (min US\$)}}} \\ 
	 ~ Cereals & -493 ~ \ \ & -332 ~ \ \ & \textit{-1\,332} ~ \ \ \\ 
	 ~ Fruit and vegetables & 64 ~ \ \ & 312 ~ \ \ & \textit{1\,765} ~ \ \ \\ 
	 ~ Meat & -16 ~ \ \ & -27 ~ \ \ & \textit{-93} ~ \ \ \\ 
	 ~ Dairy products & -60 ~ \ \ & -44 ~ \ \ & \textit{-127} ~ \ \ \\ 
	 ~ Fish & 516 ~ \ \ & 1\,042 ~ \ \ & \textit{3\,177} ~ \ \ \\ 
	\multicolumn{4}{l}{\textcolor{FAOblue}{\textbf{\large{Environment}}}} \\ 
	 ~ Forest area (\%) & 55 ~ \ \ & 54 ~ \ \ & \textit{53} ~ \ \ \\ 
	 ~ Renewable water res withdrawn (\% of total) &  ~ \ \ &  ~ \ \ & 89 ~ \ \ \\ 
	 ~ Terrestrial protect areas (\% total land area)  & 5 ~ \ \ & 10 ~ \ \ & \textit{19} ~ \ \ \\ 
	 ~ Organic area (\% total agricultural area) &  ~ \ \ &  ~ \ \ & \textit{1} ~ \ \ \\ 
	 ~ Water withdrawal by agriculture (\% of total) &  ~ \ \ &  ~ \ \ & 89 ~ \ \ \\ 
	 ~ Biofuel production (thousand kt of oil eq.) & 11 ~ \ \ & 31 ~ \ \ & \textit{259} ~ \ \ \\ 
	 ~ Wood pellet prod. (min tonnes) &  ~ \ \ &  ~ \ \ &  ~ \ \ \\ 
	 ~ GHG emissions from ag (Co2 eq, gigagrams) & 63 ~ \ \ & 67 ~ \ \ & \textit{97} ~ \ \ \\ 
       \toprule
      \end{tabular}
      \clearpage
\CountryData{ Philippines }
      \rowcolors{1}{FAOblue!10}{white}
      \begin{tabular}{L{3.9cm} R{1cm} R{1cm} R{1cm}}
      \toprule
      \multicolumn{1}{c}{} & \multicolumn{1}{c}{ 1992 } & \multicolumn{1}{c}{ 2002 } & \multicolumn{1}{c}{ 2014 } \\
      \midrule
	\multicolumn{4}{l}{\textcolor{FAOblue}{\textbf{\large{The setting}}}} \\ 
	 ~ Population, total (mln) & 65 ~ \ \ & 81 ~ \ \ & 100.1 ~ \ \ \\ 
	 ~ Population, rural (\% total population) & 33.5 ~ \ \ & 42.1 ~ \ \ & 50.5 ~ \ \ \\ 
	 ~ Govt expenditure on ag (\% total outlays) &  ~ \ \ & 4.8 ~ \ \ & \textit{3.7} ~ \ \ \\ 
	 ~ Area harvested (mln ha) & 29 ~ \ \ & 28 ~ \ \ & 32 ~ \ \ \\ 
	 ~ Cropping intensity ratio (\%) & 2.6 ~ \ \ & 2.5 ~ \ \ &  ~ \ \ \\ 
	 ~ Water resources (m\textsuperscript{3}/person/year) & \textit{7} ~ \ \ & \textit{6} ~ \ \ & \textit{5} ~ \ \ \\ 
	 ~ Area equipped for irrigation (1000 ha) &  ~ \ \ &  ~ \ \ & \textit{1\,627} ~ \ \ \\ 
	 ~ Area irrigated (\%) & \textit{95} ~ \ \ &  ~ \ \ &  ~ \ \ \\ 
	 ~ Employment in agriculture (\%) & 45.4 ~ \ \ & 37 ~ \ \ & \textit{32.2} ~ \ \ \\ 
	 ~ Employment in agriculture, female (\%) & 31.7 ~ \ \ & 24.8 ~ \ \ & \textit{20.1} ~ \ \ \\ 
	 ~ Fertilizers, Nitrogen (nutrients per ha) &  ~ \ \ & 40 ~ \ \ & \textit{36.1} ~ \ \ \\ 
	 ~ Fertilizers, Phosphate (nutrients per ha) &  ~ \ \ & 20.4 ~ \ \ & \textit{8.1} ~ \ \ \\ 
	 ~ Fertilizers, Potash (nutrients per ha) &  ~ \ \ & 4.5 ~ \ \ & \textit{6.6} ~ \ \ \\ 
	 ~ Energy consump, power irrigation (mln kWh) & \textit{0} ~ \ \ & 0 ~ \ \ & \textit{37} ~ \ \ \\ 
	 ~ Agr value added per worker (constant US\$) & 0.8 ~ \ \ & 0.9 ~ \ \ & \textit{0.9} ~ \ \ \\ 
	\multicolumn{4}{l}{\textcolor{FAOblue}{\textbf{\large{Hunger dimensions}}}} \\ 
	 ~ Dietary energy supply (kcal/pc/day) & 2\,222 ~ \ \ & 2\,386 ~ \ \ & 2\,604 ~ \ \ \\ 
	 ~ Average dietary energy supply adequacy (\%) & 104 ~ \ \ & 110 ~ \ \ & 118 ~ \ \ \\ 
	 ~ Dietary en supp, cereals/roots/tubers (\%) & 57 ~ \ \ & 58 ~ \ \ & \textit{59} ~ \ \ \\ 
	 ~ Prevalence of undernourishment (\%) & 27 ~ \ \ & 19.7 ~ \ \ & 13.9 ~ \ \ \\ 
	 ~ GDP per capita (US\$, PPP) & 3\,813 ~ \ \ & 4\,340 ~ \ \ & \textit{6\,326} ~ \ \ \\ 
	 ~ Domestic food price volatility (index) &  ~ \ \ & 3.8 ~ \ \ & 2.6 ~ \ \ \\ 
	 ~ Cereal import dependency ratio (\%) & 15.3 ~ \ \ & 25 ~ \ \ & \textit{21.9} ~ \ \ \\ 
	 ~ Underweight, children under-5 (\%) & 29.8 ~ \ \ & \textit{20.7} ~ \ \ & \textit{20.2} ~ \ \ \\ 
	 ~ Improved water source (\% pop) & 84.4 ~ \ \ & 88.5 ~ \ \ & \textit{91.8} ~ \ \ \\ 
	\multicolumn{4}{l}{\textcolor{FAOblue}{\textbf{\large{Food Supply}}}} \\ 
	 ~ Food production value, (2004-2006 mln I\$) & 11\,976 ~ \ \ & 15\,391 ~ \ \ & \textit{20\,587} ~ \ \ \\ 
	 ~ Agriculture, value added (\% GDP) & 22 ~ \ \ & 13 ~ \ \ & \textit{11} ~ \ \ \\ 
	 ~ Food exports (mln US\$)  & 1\,171 ~ \ \ & 1\,285 ~ \ \ & \textit{3\,280} ~ \ \ \\ 
	 ~ Food imports (mln US\$)  & 874 ~ \ \ & 1\,979 ~ \ \ & \textit{4\,745} ~ \ \ \\ 
	\multicolumn{4}{l}{\textit{\normalsize{Production indices (2004-06=100)}}} \\ 
	 ~ Net food & 71 ~ \ \ & 91 ~ \ \ & \textit{122} ~ \ \ \\ 
	 ~ Net crop & 77 ~ \ \ & 91 ~ \ \ & \textit{118} ~ \ \ \\ 
	 ~ Cereal & 67 ~ \ \ & 88 ~ \ \ & \textit{127} ~ \ \ \\ 
	 ~ Vegetable oils & 81 ~ \ \ & 95 ~ \ \ & \textit{112} ~ \ \ \\ 
	 ~ Roots and tubers & 111 ~ \ \ & 96 ~ \ \ & \textit{128} ~ \ \ \\ 
	 ~ Fruit and vegetables & 77 ~ \ \ & 92 ~ \ \ & \textit{118} ~ \ \ \\ 
	 ~ Sugar & 89 ~ \ \ & 85 ~ \ \ & \textit{99} ~ \ \ \\ 
	 ~ Livestock & 56 ~ \ \ & 93 ~ \ \ & \textit{127} ~ \ \ \\ 
	 ~ Milk & 126 ~ \ \ & 90 ~ \ \ & \textit{160} ~ \ \ \\ 
	 ~ Meat & 54 ~ \ \ & 93 ~ \ \ & \textit{128} ~ \ \ \\ 
	 ~ Fish  & 80 ~ \ \ & 87 ~ \ \ & \textit{111} ~ \ \ \\ 
	\multicolumn{4}{l}{\textit{\normalsize{Net trade (min US\$)}}} \\ 
	 ~ Cereals & -359 ~ \ \ & -838 ~ \ \ & \textit{-1\,724} ~ \ \ \\ 
	 ~ Fruit and vegetables & 441 ~ \ \ & 604 ~ \ \ & \textit{1\,229} ~ \ \ \\ 
	 ~ Meat & -27 ~ \ \ & -129 ~ \ \ & \textit{-443} ~ \ \ \\ 
	 ~ Dairy products & -265 ~ \ \ & -299 ~ \ \ & \textit{-638} ~ \ \ \\ 
	 ~ Fish & 283 ~ \ \ & 326 ~ \ \ & \textit{566} ~ \ \ \\ 
	\multicolumn{4}{l}{\textcolor{FAOblue}{\textbf{\large{Environment}}}} \\ 
	 ~ Forest area (\%) & 22 ~ \ \ & 24 ~ \ \ & \textit{26} ~ \ \ \\ 
	 ~ Renewable water res withdrawn (\% of total) &  ~ \ \ &  ~ \ \ & 82 ~ \ \ \\ 
	 ~ Terrestrial protect areas (\% total land area)  & 9 ~ \ \ & 11 ~ \ \ & \textit{11} ~ \ \ \\ 
	 ~ Organic area (\% total agricultural area) &  ~ \ \ &  ~ \ \ & \textit{1} ~ \ \ \\ 
	 ~ Water withdrawal by agriculture (\% of total) &  ~ \ \ &  ~ \ \ & 82 ~ \ \ \\ 
	 ~ Biofuel production (thousand kt of oil eq.) & 56 ~ \ \ & 47 ~ \ \ & \textit{2\,975} ~ \ \ \\ 
	 ~ Wood pellet prod. (min tonnes) &  ~ \ \ &  ~ \ \ &  ~ \ \ \\ 
	 ~ GHG emissions from ag (Co2 eq, gigagrams) & 31 ~ \ \ & 43 ~ \ \ & \textit{50} ~ \ \ \\ 
       \toprule
      \end{tabular}
      \clearpage
\CountryData{ Poland }
      \rowcolors{1}{FAOblue!10}{white}
      \begin{tabular}{L{3.9cm} R{1cm} R{1cm} R{1cm}}
      \toprule
      \multicolumn{1}{c}{} & \multicolumn{1}{c}{ 1992 } & \multicolumn{1}{c}{ 2002 } & \multicolumn{1}{c}{ 2014 } \\
      \midrule
	\multicolumn{4}{l}{\textcolor{FAOblue}{\textbf{\large{The setting}}}} \\ 
	 ~ Population, total (mln) & 38.3 ~ \ \ & 38.3 ~ \ \ & 38.2 ~ \ \ \\ 
	 ~ Population, rural (\% total population) & 14.8 ~ \ \ & 14.6 ~ \ \ & 15 ~ \ \ \\ 
	 ~ Govt expenditure on ag (\% total outlays) &  ~ \ \ &  ~ \ \ &  ~ \ \ \\ 
	 ~ Area harvested (mln ha) & 23 ~ \ \ & 27 ~ \ \ & 28 ~ \ \ \\ 
	 ~ Cropping intensity ratio (\%) & 1.2 ~ \ \ & 1.6 ~ \ \ &  ~ \ \ \\ 
	 ~ Water resources (m\textsuperscript{3}/person/year) & \textit{2} ~ \ \ & \textit{2} ~ \ \ & \textit{2} ~ \ \ \\ 
	 ~ Area equipped for irrigation (1000 ha) &  ~ \ \ &  ~ \ \ & \textit{97} ~ \ \ \\ 
	 ~ Area irrigated (\%) &  ~ \ \ &  ~ \ \ & \textit{62.3} ~ \ \ \\ 
	 ~ Employment in agriculture (\%) & 25 ~ \ \ & 19.3 ~ \ \ & \textit{12.6} ~ \ \ \\ 
	 ~ Employment in agriculture, female (\%) & \textit{22.5} ~ \ \ & 18.8 ~ \ \ & \textit{11.7} ~ \ \ \\ 
	 ~ Fertilizers, Nitrogen (nutrients per ha) &  ~ \ \ & 49.2 ~ \ \ & \textit{102.6} ~ \ \ \\ 
	 ~ Fertilizers, Phosphate (nutrients per ha) &  ~ \ \ & 17.9 ~ \ \ & \textit{27.1} ~ \ \ \\ 
	 ~ Fertilizers, Potash (nutrients per ha) &  ~ \ \ & 22.3 ~ \ \ & \textit{30.7} ~ \ \ \\ 
	 ~ Energy consump, power irrigation (mln kWh) & 8 ~ \ \ & 8 ~ \ \ & \textit{25} ~ \ \ \\ 
	 ~ Agr value added per worker (constant US\$) &  ~ \ \ & 2.2 ~ \ \ & \textit{3.3} ~ \ \ \\ 
	\multicolumn{4}{l}{\textcolor{FAOblue}{\textbf{\large{Hunger dimensions}}}} \\ 
	 ~ Dietary energy supply (kcal/pc/day) &  ~ \ \ &  ~ \ \ &  ~ \ \ \\ 
	 ~ Average dietary energy supply adequacy (\%) & 136 ~ \ \ & 135 ~ \ \ & 139 ~ \ \ \\ 
	 ~ Dietary en supp, cereals/roots/tubers (\%) & 42 ~ \ \ & 42 ~ \ \ & \textit{41} ~ \ \ \\ 
	 ~ Prevalence of undernourishment (\%) & <5.0 ~ \ \ & <5.0 ~ \ \ & <5.0 ~ \ \ \\ 
	 ~ GDP per capita (US\$, PPP) & 9\,545 ~ \ \ & 14\,952 ~ \ \ & \textit{22\,835} ~ \ \ \\ 
	 ~ Domestic food price volatility (index) &  ~ \ \ & 7.7 ~ \ \ & 7 ~ \ \ \\ 
	 ~ Cereal import dependency ratio (\%) & 3.5 ~ \ \ & 2.4 ~ \ \ & \textit{-2.5} ~ \ \ \\ 
	 ~ Underweight, children under-5 (\%) &  ~ \ \ &  ~ \ \ &  ~ \ \ \\ 
	 ~ Improved water source (\% pop) &  ~ \ \ &  ~ \ \ &  ~ \ \ \\ 
	\multicolumn{4}{l}{\textcolor{FAOblue}{\textbf{\large{Food Supply}}}} \\ 
	 ~ Food production value, (2004-2006 mln I\$) & 15\,532 ~ \ \ & 15\,773 ~ \ \ & \textit{17\,813} ~ \ \ \\ 
	 ~ Agriculture, value added (\% GDP) & \textit{5} ~ \ \ & 3 ~ \ \ & \textit{3} ~ \ \ \\ 
	 ~ Food exports (mln US\$)  & 1\,599 ~ \ \ & 2\,482 ~ \ \ & \textit{17\,021} ~ \ \ \\ 
	 ~ Food imports (mln US\$)  & 1\,178 ~ \ \ & 1\,910 ~ \ \ & \textit{10\,716} ~ \ \ \\ 
	\multicolumn{4}{l}{\textit{\normalsize{Production indices (2004-06=100)}}} \\ 
	 ~ Net food & 94 ~ \ \ & 95 ~ \ \ & \textit{107} ~ \ \ \\ 
	 ~ Net crop & 99 ~ \ \ & 102 ~ \ \ & \textit{106} ~ \ \ \\ 
	 ~ Cereal & 74 ~ \ \ & 103 ~ \ \ & \textit{111} ~ \ \ \\ 
	 ~ Vegetable oils & 48 ~ \ \ & 60 ~ \ \ & \textit{170} ~ \ \ \\ 
	 ~ Roots and tubers & 200 ~ \ \ & 141 ~ \ \ & \textit{56} ~ \ \ \\ 
	 ~ Fruit and vegetables & 81 ~ \ \ & 86 ~ \ \ & \textit{118} ~ \ \ \\ 
	 ~ Sugar & 92 ~ \ \ & 112 ~ \ \ & \textit{88} ~ \ \ \\ 
	 ~ Livestock & 101 ~ \ \ & 97 ~ \ \ & \textit{108} ~ \ \ \\ 
	 ~ Milk & 110 ~ \ \ & 99 ~ \ \ & \textit{107} ~ \ \ \\ 
	 ~ Meat & 98 ~ \ \ & 96 ~ \ \ & \textit{109} ~ \ \ \\ 
	 ~ Fish  & 255 ~ \ \ & 128 ~ \ \ & \textit{124} ~ \ \ \\ 
	\multicolumn{4}{l}{\textit{\normalsize{Net trade (min US\$)}}} \\ 
	 ~ Cereals & -29 ~ \ \ & -32 ~ \ \ & \textit{845} ~ \ \ \\ 
	 ~ Fruit and vegetables & 322 ~ \ \ & 202 ~ \ \ & \textit{1\,228} ~ \ \ \\ 
	 ~ Meat & 80 ~ \ \ & 273 ~ \ \ & \textit{2\,657} ~ \ \ \\ 
	 ~ Dairy products & 112 ~ \ \ & 230 ~ \ \ & \textit{1\,186} ~ \ \ \\ 
	 ~ Fish & 147 ~ \ \ & -82 ~ \ \ & \textit{-53} ~ \ \ \\ 
	\multicolumn{4}{l}{\textcolor{FAOblue}{\textbf{\large{Environment}}}} \\ 
	 ~ Forest area (\%) & 29 ~ \ \ & 30 ~ \ \ & \textit{31} ~ \ \ \\ 
	 ~ Renewable water res withdrawn (\% of total) &  ~ \ \ &  ~ \ \ & 10 ~ \ \ \\ 
	 ~ Terrestrial protect areas (\% total land area)  & 18 ~ \ \ & 22 ~ \ \ & \textit{34} ~ \ \ \\ 
	 ~ Organic area (\% total agricultural area) &  ~ \ \ & \textit{1} ~ \ \ & \textit{5} ~ \ \ \\ 
	 ~ Water withdrawal by agriculture (\% of total) &  ~ \ \ &  ~ \ \ & 10 ~ \ \ \\ 
	 ~ Biofuel production (thousand kt of oil eq.) & 0 ~ \ \ & 1 ~ \ \ & \textit{7\,194} ~ \ \ \\ 
	 ~ Wood pellet prod. (min tonnes) &  ~ \ \ &  ~ \ \ & \textit{600} ~ \ \ \\ 
	 ~ GHG emissions from ag (Co2 eq, gigagrams) & 10 ~ \ \ & -13 ~ \ \ & \textit{-12} ~ \ \ \\ 
       \toprule
      \end{tabular}
      \clearpage
\CountryData{ Portugal }
      \rowcolors{1}{FAOblue!10}{white}
      \begin{tabular}{L{3.9cm} R{1cm} R{1cm} R{1cm}}
      \toprule
      \multicolumn{1}{c}{} & \multicolumn{1}{c}{ 1992 } & \multicolumn{1}{c}{ 2002 } & \multicolumn{1}{c}{ 2014 } \\
      \midrule
	\multicolumn{4}{l}{\textcolor{FAOblue}{\textbf{\large{The setting}}}} \\ 
	 ~ Population, total (mln) & 10 ~ \ \ & 10.4 ~ \ \ & 10.6 ~ \ \ \\ 
	 ~ Population, rural (\% total population) & 5.1 ~ \ \ & 4.6 ~ \ \ & 4 ~ \ \ \\ 
	 ~ Govt expenditure on ag (\% total outlays) &  ~ \ \ &  ~ \ \ &  ~ \ \ \\ 
	 ~ Area harvested (mln ha) & 2 ~ \ \ & 2 ~ \ \ & 1 ~ \ \ \\ 
	 ~ Cropping intensity ratio (\%) & 0.5 ~ \ \ & 0.6 ~ \ \ &  ~ \ \ \\ 
	 ~ Water resources (m\textsuperscript{3}/person/year) & \textit{8} ~ \ \ & \textit{7} ~ \ \ & \textit{7} ~ \ \ \\ 
	 ~ Area equipped for irrigation (1000 ha) &  ~ \ \ &  ~ \ \ & \textit{540} ~ \ \ \\ 
	 ~ Area irrigated (\%) &  ~ \ \ &  ~ \ \ & \textit{72.2} ~ \ \ \\ 
	 ~ Employment in agriculture (\%) & 11.5 ~ \ \ & 12.5 ~ \ \ & \textit{10.5} ~ \ \ \\ 
	 ~ Employment in agriculture, female (\%) & 12.9 ~ \ \ & 13.8 ~ \ \ & \textit{8.7} ~ \ \ \\ 
	 ~ Fertilizers, Nitrogen (nutrients per ha) &  ~ \ \ & 42.6 ~ \ \ & \textit{28} ~ \ \ \\ 
	 ~ Fertilizers, Phosphate (nutrients per ha) &  ~ \ \ & 20.6 ~ \ \ & \textit{10.2} ~ \ \ \\ 
	 ~ Fertilizers, Potash (nutrients per ha) &  ~ \ \ & 18 ~ \ \ & \textit{8.6} ~ \ \ \\ 
	 ~ Energy consump, power irrigation (mln kWh) & 45 ~ \ \ & 125 ~ \ \ & \textit{125} ~ \ \ \\ 
	 ~ Agr value added per worker (constant US\$) & \textit{6.5} ~ \ \ & 7.1 ~ \ \ & \textit{9.3} ~ \ \ \\ 
	\multicolumn{4}{l}{\textcolor{FAOblue}{\textbf{\large{Hunger dimensions}}}} \\ 
	 ~ Dietary energy supply (kcal/pc/day) &  ~ \ \ &  ~ \ \ &  ~ \ \ \\ 
	 ~ Average dietary energy supply adequacy (\%) & 137 ~ \ \ & 140 ~ \ \ & 133 ~ \ \ \\ 
	 ~ Dietary en supp, cereals/roots/tubers (\%) & 36 ~ \ \ & 32 ~ \ \ & \textit{32} ~ \ \ \\ 
	 ~ Prevalence of undernourishment (\%) & <5.0 ~ \ \ & <5.0 ~ \ \ & <5.0 ~ \ \ \\ 
	 ~ GDP per capita (US\$, PPP) & 21\,464 ~ \ \ & 26\,526 ~ \ \ & \textit{25\,933} ~ \ \ \\ 
	 ~ Domestic food price volatility (index) &  ~ \ \ & 5.6 ~ \ \ & 9 ~ \ \ \\ 
	 ~ Cereal import dependency ratio (\%) & 55.3 ~ \ \ & 71 ~ \ \ & \textit{77.1} ~ \ \ \\ 
	 ~ Underweight, children under-5 (\%) &  ~ \ \ &  ~ \ \ &  ~ \ \ \\ 
	 ~ Improved water source (\% pop) & 96.5 ~ \ \ & 98.3 ~ \ \ & \textit{99.8} ~ \ \ \\ 
	\multicolumn{4}{l}{\textcolor{FAOblue}{\textbf{\large{Food Supply}}}} \\ 
	 ~ Food production value, (2004-2006 mln I\$) & 3\,707 ~ \ \ & 4\,032 ~ \ \ & \textit{4\,240} ~ \ \ \\ 
	 ~ Agriculture, value added (\% GDP) & \textit{5} ~ \ \ & 3 ~ \ \ & \textit{2} ~ \ \ \\ 
	 ~ Food exports (mln US\$)  & 482 ~ \ \ & 827 ~ \ \ & \textit{3\,367} ~ \ \ \\ 
	 ~ Food imports (mln US\$)  & 2\,185 ~ \ \ & 3\,067 ~ \ \ & \textit{7\,452} ~ \ \ \\ 
	\multicolumn{4}{l}{\textit{\normalsize{Production indices (2004-06=100)}}} \\ 
	 ~ Net food & 91 ~ \ \ & 99 ~ \ \ & \textit{104} ~ \ \ \\ 
	 ~ Net crop & 93 ~ \ \ & 100 ~ \ \ & \textit{105} ~ \ \ \\ 
	 ~ Cereal & 113 ~ \ \ & 134 ~ \ \ & \textit{112} ~ \ \ \\ 
	 ~ Vegetable oils & 63 ~ \ \ & 83 ~ \ \ & \textit{116} ~ \ \ \\ 
	 ~ Roots and tubers & 239 ~ \ \ & 121 ~ \ \ & \textit{74} ~ \ \ \\ 
	 ~ Fruit and vegetables & 87 ~ \ \ & 94 ~ \ \ & \textit{109} ~ \ \ \\ 
	 ~ Sugar & 4 ~ \ \ & 124 ~ \ \ & \textit{3} ~ \ \ \\ 
	 ~ Livestock & 90 ~ \ \ & 102 ~ \ \ & \textit{103} ~ \ \ \\ 
	 ~ Milk & 80 ~ \ \ & 104 ~ \ \ & \textit{93} ~ \ \ \\ 
	 ~ Meat & 96 ~ \ \ & 100 ~ \ \ & \textit{108} ~ \ \ \\ 
	 ~ Fish  & 131 ~ \ \ & 91 ~ \ \ & \textit{88} ~ \ \ \\ 
	\multicolumn{4}{l}{\textit{\normalsize{Net trade (min US\$)}}} \\ 
	 ~ Cereals & -648 ~ \ \ & -547 ~ \ \ & \textit{-1\,293} ~ \ \ \\ 
	 ~ Fruit and vegetables & -233 ~ \ \ & -344 ~ \ \ & \textit{-151} ~ \ \ \\ 
	 ~ Meat & -324 ~ \ \ & -424 ~ \ \ & \textit{-838} ~ \ \ \\ 
	 ~ Dairy products & 3 ~ \ \ & -137 ~ \ \ & \textit{-249} ~ \ \ \\ 
	 ~ Fish & -477 ~ \ \ & -648 ~ \ \ & \textit{-858} ~ \ \ \\ 
	\multicolumn{4}{l}{\textcolor{FAOblue}{\textbf{\large{Environment}}}} \\ 
	 ~ Forest area (\%) & 37 ~ \ \ & 37 ~ \ \ & \textit{38} ~ \ \ \\ 
	 ~ Renewable water res withdrawn (\% of total) &  ~ \ \ & 73 ~ \ \ & 73 ~ \ \ \\ 
	 ~ Terrestrial protect areas (\% total land area)  & 6 ~ \ \ & 8 ~ \ \ & \textit{22} ~ \ \ \\ 
	 ~ Organic area (\% total agricultural area) &  ~ \ \ & \textit{6} ~ \ \ & \textit{6} ~ \ \ \\ 
	 ~ Water withdrawal by agriculture (\% of total) &  ~ \ \ & 73 ~ \ \ & 73 ~ \ \ \\ 
	 ~ Biofuel production (thousand kt of oil eq.) & 27 ~ \ \ & 30 ~ \ \ & \textit{8\,629} ~ \ \ \\ 
	 ~ Wood pellet prod. (min tonnes) &  ~ \ \ &  ~ \ \ & \textit{900} ~ \ \ \\ 
	 ~ GHG emissions from ag (Co2 eq, gigagrams) & 8 ~ \ \ & 8 ~ \ \ & \textit{6} ~ \ \ \\ 
       \toprule
      \end{tabular}
      \clearpage
\CountryData{ Republic of Korea }
      \rowcolors{1}{FAOblue!10}{white}
      \begin{tabular}{L{3.9cm} R{1cm} R{1cm} R{1cm}}
      \toprule
      \multicolumn{1}{c}{} & \multicolumn{1}{c}{ 1992 } & \multicolumn{1}{c}{ 2002 } & \multicolumn{1}{c}{ 2014 } \\
      \midrule
	\multicolumn{4}{l}{\textcolor{FAOblue}{\textbf{\large{The setting}}}} \\ 
	 ~ Population, total (mln) & 43.7 ~ \ \ & 46.4 ~ \ \ & 49.5 ~ \ \ \\ 
	 ~ Population, rural (\% total population) & 10.6 ~ \ \ & 9.1 ~ \ \ & 7.9 ~ \ \ \\ 
	 ~ Govt expenditure on ag (\% total outlays) &  ~ \ \ & 5.4 ~ \ \ & \textit{2.9} ~ \ \ \\ 
	 ~ Area harvested (mln ha) & 10 ~ \ \ & 11 ~ \ \ & 6 ~ \ \ \\ 
	 ~ Cropping intensity ratio (\%) & 4.7 ~ \ \ & 5.7 ~ \ \ &  ~ \ \ \\ 
	 ~ Water resources (m\textsuperscript{3}/person/year) & \textit{2} ~ \ \ & \textit{1} ~ \ \ & \textit{1} ~ \ \ \\ 
	 ~ Area equipped for irrigation (1000 ha) &  ~ \ \ &  ~ \ \ & \textit{778} ~ \ \ \\ 
	 ~ Area irrigated (\%) &  ~ \ \ &  ~ \ \ & \textit{100} ~ \ \ \\ 
	 ~ Employment in agriculture (\%) & 15.8 ~ \ \ & 9.3 ~ \ \ & \textit{6.6} ~ \ \ \\ 
	 ~ Employment in agriculture, female (\%) & 18.3 ~ \ \ & 10.7 ~ \ \ & \textit{6.9} ~ \ \ \\ 
	 ~ Fertilizers, Nitrogen (nutrients per ha) &  ~ \ \ & 189.4 ~ \ \ & \textit{169.3} ~ \ \ \\ 
	 ~ Fertilizers, Phosphate (nutrients per ha) &  ~ \ \ & 76.3 ~ \ \ & \textit{119.4} ~ \ \ \\ 
	 ~ Fertilizers, Potash (nutrients per ha) &  ~ \ \ & 93.9 ~ \ \ & \textit{120.8} ~ \ \ \\ 
	 ~ Energy consump, power irrigation (mln kWh) &  ~ \ \ & 0 ~ \ \ & \textit{0} ~ \ \ \\ 
	 ~ Agr value added per worker (constant US\$) & 7.1 ~ \ \ & 12.2 ~ \ \ & \textit{27.1} ~ \ \ \\ 
	\multicolumn{4}{l}{\textcolor{FAOblue}{\textbf{\large{Hunger dimensions}}}} \\ 
	 ~ Dietary energy supply (kcal/pc/day) & 2\,971 ~ \ \ & 3\,074 ~ \ \ & 3\,443 ~ \ \ \\ 
	 ~ Average dietary energy supply adequacy (\%) & 124 ~ \ \ & 127 ~ \ \ & 140 ~ \ \ \\ 
	 ~ Dietary en supp, cereals/roots/tubers (\%) & 54 ~ \ \ & 47 ~ \ \ & \textit{44} ~ \ \ \\ 
	 ~ Prevalence of undernourishment (\%) & <5.0 ~ \ \ & <5.0 ~ \ \ & <5.0 ~ \ \ \\ 
	 ~ GDP per capita (US\$, PPP) & 13\,744 ~ \ \ & 23\,008 ~ \ \ & \textit{32\,708} ~ \ \ \\ 
	 ~ Domestic food price volatility (index) &  ~ \ \ & 9.3 ~ \ \ & 9.1 ~ \ \ \\ 
	 ~ Cereal import dependency ratio (\%) & 67.3 ~ \ \ & 72.8 ~ \ \ & \textit{74.2} ~ \ \ \\ 
	 ~ Underweight, children under-5 (\%) &  ~ \ \ & \textit{0.9} ~ \ \ & \textit{0.6} ~ \ \ \\ 
	 ~ Improved water source (\% pop) & 89.8 ~ \ \ & 94.5 ~ \ \ & \textit{97.8} ~ \ \ \\ 
	\multicolumn{4}{l}{\textcolor{FAOblue}{\textbf{\large{Food Supply}}}} \\ 
	 ~ Food production value, (2004-2006 mln I\$) & 8\,478 ~ \ \ & 9\,866 ~ \ \ & \textit{10\,238} ~ \ \ \\ 
	 ~ Agriculture, value added (\% GDP) & 7 ~ \ \ & 4 ~ \ \ & \textit{2} ~ \ \ \\ 
	 ~ Food exports (mln US\$)  & 714 ~ \ \ & 1\,013 ~ \ \ & \textit{2\,705} ~ \ \ \\ 
	 ~ Food imports (mln US\$)  & 3\,649 ~ \ \ & 5\,586 ~ \ \ & \textit{16\,704} ~ \ \ \\ 
	\multicolumn{4}{l}{\textit{\normalsize{Production indices (2004-06=100)}}} \\ 
	 ~ Net food & 86 ~ \ \ & 100 ~ \ \ & \textit{104} ~ \ \ \\ 
	 ~ Net crop & 93 ~ \ \ & 99 ~ \ \ & \textit{94} ~ \ \ \\ 
	 ~ Cereal & 113 ~ \ \ & 103 ~ \ \ & \textit{86} ~ \ \ \\ 
	 ~ Vegetable oils & 141 ~ \ \ & 97 ~ \ \ & \textit{96} ~ \ \ \\ 
	 ~ Roots and tubers & 101 ~ \ \ & 95 ~ \ \ & \textit{102} ~ \ \ \\ 
	 ~ Fruit and vegetables & 80 ~ \ \ & 96 ~ \ \ & \textit{96} ~ \ \ \\ 
	 ~ Sugar &  ~ \ \ &  ~ \ \ &  ~ \ \ \\ 
	 ~ Livestock & 77 ~ \ \ & 103 ~ \ \ & \textit{119} ~ \ \ \\ 
	 ~ Milk & 82 ~ \ \ & 114 ~ \ \ & \textit{94} ~ \ \ \\ 
	 ~ Meat & 78 ~ \ \ & 101 ~ \ \ & \textit{126} ~ \ \ \\ 
	 ~ Fish  & 128 ~ \ \ & 94 ~ \ \ & \textit{95} ~ \ \ \\ 
	\multicolumn{4}{l}{\textit{\normalsize{Net trade (min US\$)}}} \\ 
	 ~ Cereals & -1\,441 ~ \ \ & -1\,638 ~ \ \ & \textit{-4\,752} ~ \ \ \\ 
	 ~ Fruit and vegetables & -153 ~ \ \ & -379 ~ \ \ & \textit{-2\,343} ~ \ \ \\ 
	 ~ Meat & -494 ~ \ \ & -1\,276 ~ \ \ & \textit{-2\,944} ~ \ \ \\ 
	 ~ Dairy products & -34 ~ \ \ & -141 ~ \ \ & \textit{-595} ~ \ \ \\ 
	 ~ Fish & 856 ~ \ \ & -825 ~ \ \ & \textit{-1\,723} ~ \ \ \\ 
	\multicolumn{4}{l}{\textcolor{FAOblue}{\textbf{\large{Environment}}}} \\ 
	 ~ Forest area (\%) & 66 ~ \ \ & 65 ~ \ \ & \textit{64} ~ \ \ \\ 
	 ~ Renewable water res withdrawn (\% of total) &  ~ \ \ & 62 ~ \ \ & 62 ~ \ \ \\ 
	 ~ Terrestrial protect areas (\% total land area)  & 2 ~ \ \ & 2 ~ \ \ & \textit{6} ~ \ \ \\ 
	 ~ Organic area (\% total agricultural area) &  ~ \ \ & \textit{0} ~ \ \ & \textit{1} ~ \ \ \\ 
	 ~ Water withdrawal by agriculture (\% of total) &  ~ \ \ & 62 ~ \ \ & 62 ~ \ \ \\ 
	 ~ Biofuel production (thousand kt of oil eq.) & 1 ~ \ \ & 34 ~ \ \ & \textit{8\,904} ~ \ \ \\ 
	 ~ Wood pellet prod. (min tonnes) &  ~ \ \ &  ~ \ \ & \textit{15} ~ \ \ \\ 
	 ~ GHG emissions from ag (Co2 eq, gigagrams) & -13 ~ \ \ & -19 ~ \ \ & \textit{-18} ~ \ \ \\ 
       \toprule
      \end{tabular}
      \clearpage
\CountryData{ Republic of Moldova }
      \rowcolors{1}{FAOblue!10}{white}
      \begin{tabular}{L{3.9cm} R{1cm} R{1cm} R{1cm}}
      \toprule
      \multicolumn{1}{c}{} & \multicolumn{1}{c}{ 1992 } & \multicolumn{1}{c}{ 2002 } & \multicolumn{1}{c}{ 2014 } \\
      \midrule
	\multicolumn{4}{l}{\textcolor{FAOblue}{\textbf{\large{The setting}}}} \\ 
	 ~ Population, total (mln) & 4.4 ~ \ \ & 4 ~ \ \ & 3.5 ~ \ \ \\ 
	 ~ Population, rural (\% total population) & 2.3 ~ \ \ & 2.2 ~ \ \ & 1.7 ~ \ \ \\ 
	 ~ Govt expenditure on ag (\% total outlays) &  ~ \ \ & 2.2 ~ \ \ & \textit{4.1} ~ \ \ \\ 
	 ~ Area harvested (mln ha) & 2 ~ \ \ & 3 ~ \ \ & 3 ~ \ \ \\ 
	 ~ Cropping intensity ratio (\%) & 0.8 ~ \ \ & 1 ~ \ \ &  ~ \ \ \\ 
	 ~ Water resources (m\textsuperscript{3}/person/year) & \textit{3} ~ \ \ & \textit{3} ~ \ \ & \textit{3} ~ \ \ \\ 
	 ~ Area equipped for irrigation (1000 ha) &  ~ \ \ &  ~ \ \ & \textit{228} ~ \ \ \\ 
	 ~ Area irrigated (\%) & 72.1 ~ \ \ &  ~ \ \ &  ~ \ \ \\ 
	 ~ Employment in agriculture (\%) & 40 ~ \ \ & 49.6 ~ \ \ & \textit{26.4} ~ \ \ \\ 
	 ~ Employment in agriculture, female (\%) &  ~ \ \ & 48.6 ~ \ \ & \textit{23.2} ~ \ \ \\ 
	 ~ Fertilizers, Nitrogen (nutrients per ha) &  ~ \ \ & 5.8 ~ \ \ & \textit{10.9} ~ \ \ \\ 
	 ~ Fertilizers, Phosphate (nutrients per ha) &  ~ \ \ & 0.1 ~ \ \ & \textit{2.3} ~ \ \ \\ 
	 ~ Fertilizers, Potash (nutrients per ha) &  ~ \ \ & 0 ~ \ \ & \textit{0.9} ~ \ \ \\ 
	 ~ Energy consump, power irrigation (mln kWh) & 14 ~ \ \ & 14 ~ \ \ & \textit{309} ~ \ \ \\ 
	 ~ Agr value added per worker (constant US\$) & 1.4 ~ \ \ & 1.1 ~ \ \ & \textit{2.2} ~ \ \ \\ 
	\multicolumn{4}{l}{\textcolor{FAOblue}{\textbf{\large{Hunger dimensions}}}} \\ 
	 ~ Dietary energy supply (kcal/pc/day) &  ~ \ \ &  ~ \ \ &  ~ \ \ \\ 
	 ~ Average dietary energy supply adequacy (\%) & 104 ~ \ \ & 104 ~ \ \ & 116 ~ \ \ \\ 
	 ~ Dietary en supp, cereals/roots/tubers (\%) & 54 ~ \ \ & 58 ~ \ \ & \textit{46} ~ \ \ \\ 
	 ~ Prevalence of undernourishment (\%) & <5.0 ~ \ \ & <5.0 ~ \ \ & <5.0 ~ \ \ \\ 
	 ~ GDP per capita (US\$, PPP) & 3\,809 ~ \ \ & 2\,667 ~ \ \ & \textit{4\,521} ~ \ \ \\ 
	 ~ Domestic food price volatility (index) &  ~ \ \ & 11.3 ~ \ \ & \textit{5.7} ~ \ \ \\ 
	 ~ Cereal import dependency ratio (\%) & 10.1 ~ \ \ & -6.9 ~ \ \ & \textit{-12.5} ~ \ \ \\ 
	 ~ Underweight, children under-5 (\%) &  ~ \ \ & \textit{3.2} ~ \ \ & \textit{2.2} ~ \ \ \\ 
	 ~ Improved water source (\% pop) & 92.7 ~ \ \ & 93.8 ~ \ \ & \textit{96.5} ~ \ \ \\ 
	\multicolumn{4}{l}{\textcolor{FAOblue}{\textbf{\large{Food Supply}}}} \\ 
	 ~ Food production value, (2004-2006 mln I\$) & 1\,961 ~ \ \ & 1\,401 ~ \ \ & \textit{1\,262} ~ \ \ \\ 
	 ~ Agriculture, value added (\% GDP) & 51 ~ \ \ & 24 ~ \ \ & \textit{15} ~ \ \ \\ 
	 ~ Food exports (mln US\$)  & 142 ~ \ \ & 182 ~ \ \ & \textit{604} ~ \ \ \\ 
	 ~ Food imports (mln US\$)  & 95 ~ \ \ & 94 ~ \ \ & \textit{464} ~ \ \ \\ 
	\multicolumn{4}{l}{\textit{\normalsize{Production indices (2004-06=100)}}} \\ 
	 ~ Net food & 136 ~ \ \ & 98 ~ \ \ & \textit{88} ~ \ \ \\ 
	 ~ Net crop & 122 ~ \ \ & 100 ~ \ \ & \textit{102} ~ \ \ \\ 
	 ~ Cereal & 73 ~ \ \ & 99 ~ \ \ & \textit{101} ~ \ \ \\ 
	 ~ Vegetable oils & 53 ~ \ \ & 86 ~ \ \ & \textit{151} ~ \ \ \\ 
	 ~ Roots and tubers & 84 ~ \ \ & 91 ~ \ \ & \textit{67} ~ \ \ \\ 
	 ~ Fruit and vegetables & 155 ~ \ \ & 104 ~ \ \ & \textit{102} ~ \ \ \\ 
	 ~ Sugar & 192 ~ \ \ & 110 ~ \ \ & \textit{98} ~ \ \ \\ 
	 ~ Livestock & 209 ~ \ \ & 95 ~ \ \ & \textit{96} ~ \ \ \\ 
	 ~ Milk & 177 ~ \ \ & 95 ~ \ \ & \textit{83} ~ \ \ \\ 
	 ~ Meat & 285 ~ \ \ & 96 ~ \ \ & \textit{115} ~ \ \ \\ 
	 ~ Fish  & 48 ~ \ \ & 38 ~ \ \ & \textit{152} ~ \ \ \\ 
	\multicolumn{4}{l}{\textit{\normalsize{Net trade (min US\$)}}} \\ 
	 ~ Cereals & -70 ~ \ \ & 33 ~ \ \ & \textit{-42} ~ \ \ \\ 
	 ~ Fruit and vegetables & 71 ~ \ \ & 46 ~ \ \ & \textit{149} ~ \ \ \\ 
	 ~ Meat & 28 ~ \ \ & -2 ~ \ \ & \textit{-23} ~ \ \ \\ 
	 ~ Dairy products & 7 ~ \ \ & 2 ~ \ \ & \textit{-28} ~ \ \ \\ 
	 ~ Fish & \textit{-6} ~ \ \ & -9 ~ \ \ & \textit{-53} ~ \ \ \\ 
	\multicolumn{4}{l}{\textcolor{FAOblue}{\textbf{\large{Environment}}}} \\ 
	 ~ Forest area (\%) & 10 ~ \ \ & 10 ~ \ \ & \textit{12} ~ \ \ \\ 
	 ~ Renewable water res withdrawn (\% of total) &  ~ \ \ &  ~ \ \ & 4 ~ \ \ \\ 
	 ~ Terrestrial protect areas (\% total land area)  & 1 ~ \ \ & 1 ~ \ \ & \textit{4} ~ \ \ \\ 
	 ~ Organic area (\% total agricultural area) &  ~ \ \ &  ~ \ \ & \textit{1} ~ \ \ \\ 
	 ~ Water withdrawal by agriculture (\% of total) &  ~ \ \ &  ~ \ \ & 4 ~ \ \ \\ 
	 ~ Biofuel production (thousand kt of oil eq.) &  ~ \ \ & \textit{0} ~ \ \ & \textit{0} ~ \ \ \\ 
	 ~ Wood pellet prod. (min tonnes) &  ~ \ \ &  ~ \ \ & \textit{0} ~ \ \ \\ 
	 ~ GHG emissions from ag (Co2 eq, gigagrams) & 4 ~ \ \ & 1 ~ \ \ & \textit{1} ~ \ \ \\ 
       \toprule
      \end{tabular}
      \clearpage
\CountryData{ Romania }
      \rowcolors{1}{FAOblue!10}{white}
      \begin{tabular}{L{3.9cm} R{1cm} R{1cm} R{1cm}}
      \toprule
      \multicolumn{1}{c}{} & \multicolumn{1}{c}{ 1992 } & \multicolumn{1}{c}{ 2002 } & \multicolumn{1}{c}{ 2014 } \\
      \midrule
	\multicolumn{4}{l}{\textcolor{FAOblue}{\textbf{\large{The setting}}}} \\ 
	 ~ Population, total (mln) & 23.3 ~ \ \ & 22.2 ~ \ \ & 21.6 ~ \ \ \\ 
	 ~ Population, rural (\% total population) & 10.7 ~ \ \ & 10.5 ~ \ \ & 10.2 ~ \ \ \\ 
	 ~ Govt expenditure on ag (\% total outlays) &  ~ \ \ &  ~ \ \ &  ~ \ \ \\ 
	 ~ Area harvested (mln ha) & 12 ~ \ \ & 14 ~ \ \ & 21 ~ \ \ \\ 
	 ~ Cropping intensity ratio (\%) & 0.8 ~ \ \ & 1 ~ \ \ &  ~ \ \ \\ 
	 ~ Water resources (m\textsuperscript{3}/person/year) & \textit{9} ~ \ \ & \textit{10} ~ \ \ & \textit{10} ~ \ \ \\ 
	 ~ Area equipped for irrigation (1000 ha) &  ~ \ \ &  ~ \ \ & \textit{3\,149} ~ \ \ \\ 
	 ~ Area irrigated (\%) &  ~ \ \ &  ~ \ \ & \textit{28.2} ~ \ \ \\ 
	 ~ Employment in agriculture (\%) & 33 ~ \ \ & 36.4 ~ \ \ & \textit{29} ~ \ \ \\ 
	 ~ Employment in agriculture, female (\%) & 37.7 ~ \ \ & 38.5 ~ \ \ & \textit{30.1} ~ \ \ \\ 
	 ~ Fertilizers, Nitrogen (nutrients per ha) &  ~ \ \ & 16.1 ~ \ \ & \textit{21.1} ~ \ \ \\ 
	 ~ Fertilizers, Phosphate (nutrients per ha) &  ~ \ \ & 4.9 ~ \ \ & \textit{8.2} ~ \ \ \\ 
	 ~ Fertilizers, Potash (nutrients per ha) &  ~ \ \ & 0.9 ~ \ \ & \textit{2.5} ~ \ \ \\ 
	 ~ Energy consump, power irrigation (mln kWh) & 5\,025 ~ \ \ & 5\,025 ~ \ \ & \textit{872} ~ \ \ \\ 
	 ~ Agr value added per worker (constant US\$) & 3.1 ~ \ \ & 5.6 ~ \ \ & \textit{9.1} ~ \ \ \\ 
	\multicolumn{4}{l}{\textcolor{FAOblue}{\textbf{\large{Hunger dimensions}}}} \\ 
	 ~ Dietary energy supply (kcal/pc/day) &  ~ \ \ &  ~ \ \ &  ~ \ \ \\ 
	 ~ Average dietary energy supply adequacy (\%) & 120 ~ \ \ & 133 ~ \ \ & 136 ~ \ \ \\ 
	 ~ Dietary en supp, cereals/roots/tubers (\%) & 47 ~ \ \ & 47 ~ \ \ & \textit{45} ~ \ \ \\ 
	 ~ Prevalence of undernourishment (\%) & <5.0 ~ \ \ & <5.0 ~ \ \ & <5.0 ~ \ \ \\ 
	 ~ GDP per capita (US\$, PPP) & 9\,253 ~ \ \ & 11\,429 ~ \ \ & \textit{18\,184} ~ \ \ \\ 
	 ~ Domestic food price volatility (index) &  ~ \ \ & 7.8 ~ \ \ & 4.3 ~ \ \ \\ 
	 ~ Cereal import dependency ratio (\%) & 11.4 ~ \ \ & 4.7 ~ \ \ & \textit{-22.6} ~ \ \ \\ 
	 ~ Underweight, children under-5 (\%) & \textit{5} ~ \ \ & 3.5 ~ \ \ &  ~ \ \ \\ 
	 ~ Improved water source (\% pop) & 77.4 ~ \ \ & 85.9 ~ \ \ & \textit{87.7} ~ \ \ \\ 
	\multicolumn{4}{l}{\textcolor{FAOblue}{\textbf{\large{Food Supply}}}} \\ 
	 ~ Food production value, (2004-2006 mln I\$) & 6\,939 ~ \ \ & 7\,349 ~ \ \ & \textit{8\,556} ~ \ \ \\ 
	 ~ Agriculture, value added (\% GDP) & 19 ~ \ \ & 13 ~ \ \ & \textit{6} ~ \ \ \\ 
	 ~ Food exports (mln US\$)  & 241 ~ \ \ & 362 ~ \ \ & \textit{4\,012} ~ \ \ \\ 
	 ~ Food imports (mln US\$)  & 757 ~ \ \ & 790 ~ \ \ & \textit{4\,203} ~ \ \ \\ 
	\multicolumn{4}{l}{\textit{\normalsize{Production indices (2004-06=100)}}} \\ 
	 ~ Net food & 80 ~ \ \ & 84 ~ \ \ & \textit{98} ~ \ \ \\ 
	 ~ Net crop & 68 ~ \ \ & 80 ~ \ \ & \textit{103} ~ \ \ \\ 
	 ~ Cereal & 58 ~ \ \ & 71 ~ \ \ & \textit{106} ~ \ \ \\ 
	 ~ Vegetable oils & 49 ~ \ \ & 63 ~ \ \ & \textit{168} ~ \ \ \\ 
	 ~ Roots and tubers & 51 ~ \ \ & 102 ~ \ \ & \textit{73} ~ \ \ \\ 
	 ~ Fruit and vegetables & 82 ~ \ \ & 85 ~ \ \ & \textit{93} ~ \ \ \\ 
	 ~ Sugar & 340 ~ \ \ & 112 ~ \ \ & \textit{121} ~ \ \ \\ 
	 ~ Livestock & 106 ~ \ \ & 94 ~ \ \ & \textit{86} ~ \ \ \\ 
	 ~ Milk & 66 ~ \ \ & 84 ~ \ \ & \textit{86} ~ \ \ \\ 
	 ~ Meat & 152 ~ \ \ & 104 ~ \ \ & \textit{86} ~ \ \ \\ 
	 ~ Fish  & 693 ~ \ \ & 118 ~ \ \ & \textit{111} ~ \ \ \\ 
	\multicolumn{4}{l}{\textit{\normalsize{Net trade (min US\$)}}} \\ 
	 ~ Cereals & -260 ~ \ \ & 5 ~ \ \ & \textit{944} ~ \ \ \\ 
	 ~ Fruit and vegetables & -87 ~ \ \ & -69 ~ \ \ & \textit{-478} ~ \ \ \\ 
	 ~ Meat & 35 ~ \ \ & -169 ~ \ \ & \textit{-226} ~ \ \ \\ 
	 ~ Dairy products & -33 ~ \ \ & -17 ~ \ \ & \textit{-203} ~ \ \ \\ 
	 ~ Fish & -3 ~ \ \ & -47 ~ \ \ & \textit{-184} ~ \ \ \\ 
	\multicolumn{4}{l}{\textcolor{FAOblue}{\textbf{\large{Environment}}}} \\ 
	 ~ Forest area (\%) & 28 ~ \ \ & 28 ~ \ \ & \textit{29} ~ \ \ \\ 
	 ~ Renewable water res withdrawn (\% of total) &  ~ \ \ &  ~ \ \ & 17 ~ \ \ \\ 
	 ~ Terrestrial protect areas (\% total land area)  & 4 ~ \ \ & 5 ~ \ \ & \textit{19} ~ \ \ \\ 
	 ~ Organic area (\% total agricultural area) &  ~ \ \ & \textit{1} ~ \ \ & \textit{2} ~ \ \ \\ 
	 ~ Water withdrawal by agriculture (\% of total) &  ~ \ \ &  ~ \ \ & 17 ~ \ \ \\ 
	 ~ Biofuel production (thousand kt of oil eq.) &  ~ \ \ & 8 ~ \ \ & \textit{330} ~ \ \ \\ 
	 ~ Wood pellet prod. (min tonnes) &  ~ \ \ &  ~ \ \ & \textit{520} ~ \ \ \\ 
	 ~ GHG emissions from ag (Co2 eq, gigagrams) & 24 ~ \ \ & 15 ~ \ \ & \textit{4} ~ \ \ \\ 
       \toprule
      \end{tabular}
      \clearpage
\CountryData{ Russian Federation }
      \rowcolors{1}{FAOblue!10}{white}
      \begin{tabular}{L{3.9cm} R{1cm} R{1cm} R{1cm}}
      \toprule
      \multicolumn{1}{c}{} & \multicolumn{1}{c}{ 1992 } & \multicolumn{1}{c}{ 2002 } & \multicolumn{1}{c}{ 2014 } \\
      \midrule
	\multicolumn{4}{l}{\textcolor{FAOblue}{\textbf{\large{The setting}}}} \\ 
	 ~ Population, total (mln) & 148.8 ~ \ \ & 145.5 ~ \ \ & 142.5 ~ \ \ \\ 
	 ~ Population, rural (\% total population) & 39.6 ~ \ \ & 38.8 ~ \ \ & 36.6 ~ \ \ \\ 
	 ~ Govt expenditure on ag (\% total outlays) &  ~ \ \ &  ~ \ \ &  ~ \ \ \\ 
	 ~ Area harvested (mln ha) & 104 ~ \ \ & 85 ~ \ \ & 90 ~ \ \ \\ 
	 ~ Cropping intensity ratio (\%) & 0.5 ~ \ \ & 0.4 ~ \ \ &  ~ \ \ \\ 
	 ~ Water resources (m\textsuperscript{3}/person/year) & \textit{30} ~ \ \ & \textit{31} ~ \ \ & \textit{32} ~ \ \ \\ 
	 ~ Area equipped for irrigation (1000 ha) &  ~ \ \ &  ~ \ \ & \textit{4\,300} ~ \ \ \\ 
	 ~ Area irrigated (\%) &  ~ \ \ &  ~ \ \ & \textit{39.5} ~ \ \ \\ 
	 ~ Employment in agriculture (\%) & 15.4 ~ \ \ & 11.3 ~ \ \ & \textit{9.7} ~ \ \ \\ 
	 ~ Employment in agriculture, female (\%) &  ~ \ \ & 8.9 ~ \ \ & \textit{6.7} ~ \ \ \\ 
	 ~ Fertilizers, Nitrogen (nutrients per ha) &  ~ \ \ & 3 ~ \ \ & \textit{5.5} ~ \ \ \\ 
	 ~ Fertilizers, Phosphate (nutrients per ha) &  ~ \ \ & 1.4 ~ \ \ & \textit{2} ~ \ \ \\ 
	 ~ Fertilizers, Potash (nutrients per ha) &  ~ \ \ & 3.3 ~ \ \ & \textit{1.3} ~ \ \ \\ 
	 ~ Energy consump, power irrigation (mln kWh) &  ~ \ \ &  ~ \ \ &  ~ \ \ \\ 
	 ~ Agr value added per worker (constant US\$) & 3.7 ~ \ \ & 4.5 ~ \ \ & \textit{6} ~ \ \ \\ 
	\multicolumn{4}{l}{\textcolor{FAOblue}{\textbf{\large{Hunger dimensions}}}} \\ 
	 ~ Dietary energy supply (kcal/pc/day) &  ~ \ \ &  ~ \ \ &  ~ \ \ \\ 
	 ~ Average dietary energy supply adequacy (\%) & 121 ~ \ \ & 120 ~ \ \ & 136 ~ \ \ \\ 
	 ~ Dietary en supp, cereals/roots/tubers (\%) & 48 ~ \ \ & 46 ~ \ \ & \textit{41} ~ \ \ \\ 
	 ~ Prevalence of undernourishment (\%) & <5.0 ~ \ \ & <5.0 ~ \ \ & <5.0 ~ \ \ \\ 
	 ~ GDP per capita (US\$, PPP) & 15\,661 ~ \ \ & 14\,619 ~ \ \ & \textit{23\,564} ~ \ \ \\ 
	 ~ Domestic food price volatility (index) &  ~ \ \ & 6.6 ~ \ \ & 5.2 ~ \ \ \\ 
	 ~ Cereal import dependency ratio (\%) & 12.3 ~ \ \ & -9.8 ~ \ \ & \textit{-27.5} ~ \ \ \\ 
	 ~ Underweight, children under-5 (\%) &  ~ \ \ &  ~ \ \ &  ~ \ \ \\ 
	 ~ Improved water source (\% pop) & 93.5 ~ \ \ & 95.5 ~ \ \ & \textit{97} ~ \ \ \\ 
	\multicolumn{4}{l}{\textcolor{FAOblue}{\textbf{\large{Food Supply}}}} \\ 
	 ~ Food production value, (2004-2006 mln I\$) & 48\,439 ~ \ \ & 37\,749 ~ \ \ & \textit{46\,439} ~ \ \ \\ 
	 ~ Agriculture, value added (\% GDP) & 7 ~ \ \ & 6 ~ \ \ & \textit{4} ~ \ \ \\ 
	 ~ Food exports (mln US\$)  & 434 ~ \ \ & 1\,518 ~ \ \ & \textit{11\,606} ~ \ \ \\ 
	 ~ Food imports (mln US\$)  & 11\,682 ~ \ \ & 7\,080 ~ \ \ & \textit{28\,907} ~ \ \ \\ 
	\multicolumn{4}{l}{\textit{\normalsize{Production indices (2004-06=100)}}} \\ 
	 ~ Net food & 128 ~ \ \ & 100 ~ \ \ & \textit{122} ~ \ \ \\ 
	 ~ Net crop & 100 ~ \ \ & 93 ~ \ \ & \textit{116} ~ \ \ \\ 
	 ~ Cereal & 126 ~ \ \ & 111 ~ \ \ & \textit{122} ~ \ \ \\ 
	 ~ Vegetable oils & 52 ~ \ \ & 58 ~ \ \ & \textit{195} ~ \ \ \\ 
	 ~ Roots and tubers & 98 ~ \ \ & 83 ~ \ \ & \textit{82} ~ \ \ \\ 
	 ~ Fruit and vegetables & 72 ~ \ \ & 88 ~ \ \ & \textit{97} ~ \ \ \\ 
	 ~ Sugar & 104 ~ \ \ & 64 ~ \ \ & \textit{160} ~ \ \ \\ 
	 ~ Livestock & 159 ~ \ \ & 101 ~ \ \ & \textit{122} ~ \ \ \\ 
	 ~ Milk & 150 ~ \ \ & 106 ~ \ \ & \textit{97} ~ \ \ \\ 
	 ~ Meat & 174 ~ \ \ & 97 ~ \ \ & \textit{150} ~ \ \ \\ 
	 ~ Fish  & 170 ~ \ \ & 103 ~ \ \ & \textit{138} ~ \ \ \\ 
	\multicolumn{4}{l}{\textit{\normalsize{Net trade (min US\$)}}} \\ 
	 ~ Cereals & -4\,776 ~ \ \ & 603 ~ \ \ & \textit{5\,160} ~ \ \ \\ 
	 ~ Fruit and vegetables & -2\,473 ~ \ \ & -1\,420 ~ \ \ & \textit{-9\,689} ~ \ \ \\ 
	 ~ Meat & -1\,020 ~ \ \ & -2\,273 ~ \ \ & \textit{-7\,087} ~ \ \ \\ 
	 ~ Dairy products & -568 ~ \ \ & -401 ~ \ \ & \textit{-2\,805} ~ \ \ \\ 
	 ~ Fish & 792 ~ \ \ & 972 ~ \ \ & \textit{411} ~ \ \ \\ 
	\multicolumn{4}{l}{\textcolor{FAOblue}{\textbf{\large{Environment}}}} \\ 
	 ~ Forest area (\%) & 49 ~ \ \ & 49 ~ \ \ & \textit{49} ~ \ \ \\ 
	 ~ Renewable water res withdrawn (\% of total) &  ~ \ \ & \textit{20} ~ \ \ & 20 ~ \ \ \\ 
	 ~ Terrestrial protect areas (\% total land area)  & 5 ~ \ \ & 9 ~ \ \ & \textit{11} ~ \ \ \\ 
	 ~ Organic area (\% total agricultural area) &  ~ \ \ &  ~ \ \ & \textit{0} ~ \ \ \\ 
	 ~ Water withdrawal by agriculture (\% of total) &  ~ \ \ & \textit{20} ~ \ \ & 20 ~ \ \ \\ 
	 ~ Biofuel production (thousand kt of oil eq.) &  ~ \ \ &  ~ \ \ &  ~ \ \ \\ 
	 ~ Wood pellet prod. (min tonnes) &  ~ \ \ &  ~ \ \ & \textit{810} ~ \ \ \\ 
	 ~ GHG emissions from ag (Co2 eq, gigagrams) & 288 ~ \ \ & 166 ~ \ \ & \textit{25} ~ \ \ \\ 
       \toprule
      \end{tabular}
      \clearpage
\CountryData{ Rwanda }
      \rowcolors{1}{FAOblue!10}{white}
      \begin{tabular}{L{3.9cm} R{1cm} R{1cm} R{1cm}}
      \toprule
      \multicolumn{1}{c}{} & \multicolumn{1}{c}{ 1992 } & \multicolumn{1}{c}{ 2002 } & \multicolumn{1}{c}{ 2014 } \\
      \midrule
	\multicolumn{4}{l}{\textcolor{FAOblue}{\textbf{\large{The setting}}}} \\ 
	 ~ Population, total (mln) & 6.5 ~ \ \ & 9 ~ \ \ & 12.1 ~ \ \ \\ 
	 ~ Population, rural (\% total population) & 6.2 ~ \ \ & 7.5 ~ \ \ & 9.7 ~ \ \ \\ 
	 ~ Govt expenditure on ag (\% total outlays) &  ~ \ \ &  ~ \ \ &  ~ \ \ \\ 
	 ~ Area harvested (mln ha) & 4 ~ \ \ & 3 ~ \ \ & 6 ~ \ \ \\ 
	 ~ Cropping intensity ratio (\%) & 2 ~ \ \ & 1.9 ~ \ \ &  ~ \ \ \\ 
	 ~ Water resources (m\textsuperscript{3}/person/year) & \textit{2} ~ \ \ & \textit{1} ~ \ \ & \textit{1} ~ \ \ \\ 
	 ~ Area equipped for irrigation (1000 ha) &  ~ \ \ &  ~ \ \ & \textit{10} ~ \ \ \\ 
	 ~ Area irrigated (\%) &  ~ \ \ &  ~ \ \ & \textit{82.5} ~ \ \ \\ 
	 ~ Employment in agriculture (\%) &  ~ \ \ & \textit{78.8} ~ \ \ & \textit{78.8} ~ \ \ \\ 
	 ~ Employment in agriculture, female (\%) &  ~ \ \ &  ~ \ \ &  ~ \ \ \\ 
	 ~ Fertilizers, Nitrogen (nutrients per ha) &  ~ \ \ & 0 ~ \ \ & \textit{1.2} ~ \ \ \\ 
	 ~ Fertilizers, Phosphate (nutrients per ha) &  ~ \ \ & 0 ~ \ \ & \textit{1.3} ~ \ \ \\ 
	 ~ Fertilizers, Potash (nutrients per ha) &  ~ \ \ & 0 ~ \ \ & \textit{0} ~ \ \ \\ 
	 ~ Energy consump, power irrigation (mln kWh) &  ~ \ \ &  ~ \ \ &  ~ \ \ \\ 
	 ~ Agr value added per worker (constant US\$) & 0.2 ~ \ \ & 0.3 ~ \ \ & \textit{0.3} ~ \ \ \\ 
	\multicolumn{4}{l}{\textcolor{FAOblue}{\textbf{\large{Hunger dimensions}}}} \\ 
	 ~ Dietary energy supply (kcal/pc/day) & 1\,832 ~ \ \ & 1\,946 ~ \ \ & 2\,232 ~ \ \ \\ 
	 ~ Average dietary energy supply adequacy (\%) & 88 ~ \ \ & 94 ~ \ \ & 104 ~ \ \ \\ 
	 ~ Dietary en supp, cereals/roots/tubers (\%) & 50 ~ \ \ & 57 ~ \ \ & \textit{50} ~ \ \ \\ 
	 ~ Prevalence of undernourishment (\%) & 53.9 ~ \ \ & 49 ~ \ \ & 32.7 ~ \ \ \\ 
	 ~ GDP per capita (US\$, PPP) & 974 ~ \ \ & 880 ~ \ \ & \textit{1\,426} ~ \ \ \\ 
	 ~ Domestic food price volatility (index) &  ~ \ \ & 11.9 ~ \ \ & \textit{10.5} ~ \ \ \\ 
	 ~ Cereal import dependency ratio (\%) & 20.5 ~ \ \ & 17.9 ~ \ \ & \textit{23.7} ~ \ \ \\ 
	 ~ Underweight, children under-5 (\%) & 24.3 ~ \ \ & \textit{18} ~ \ \ & \textit{11.7} ~ \ \ \\ 
	 ~ Improved water source (\% pop) & 61.3 ~ \ \ & 67.4 ~ \ \ & \textit{70.7} ~ \ \ \\ 
	\multicolumn{4}{l}{\textcolor{FAOblue}{\textbf{\large{Food Supply}}}} \\ 
	 ~ Food production value, (2004-2006 mln I\$) & 1\,231 ~ \ \ & 1\,415 ~ \ \ & \textit{2\,472} ~ \ \ \\ 
	 ~ Agriculture, value added (\% GDP) & 33 ~ \ \ & 35 ~ \ \ & \textit{33} ~ \ \ \\ 
	 ~ Food exports (mln US\$)  & 0 ~ \ \ & 0 ~ \ \ & \textit{49} ~ \ \ \\ 
	 ~ Food imports (mln US\$)  & 40 ~ \ \ & 57 ~ \ \ & \textit{224} ~ \ \ \\ 
	\multicolumn{4}{l}{\textit{\normalsize{Production indices (2004-06=100)}}} \\ 
	 ~ Net food & 86 ~ \ \ & 99 ~ \ \ & \textit{173} ~ \ \ \\ 
	 ~ Net crop & 91 ~ \ \ & 101 ~ \ \ & \textit{175} ~ \ \ \\ 
	 ~ Cereal & 61 ~ \ \ & 78 ~ \ \ & \textit{254} ~ \ \ \\ 
	 ~ Vegetable oils & 104 ~ \ \ & 90 ~ \ \ & \textit{127} ~ \ \ \\ 
	 ~ Roots and tubers & 51 ~ \ \ & 107 ~ \ \ & \textit{216} ~ \ \ \\ 
	 ~ Fruit and vegetables & 116 ~ \ \ & 100 ~ \ \ & \textit{148} ~ \ \ \\ 
	 ~ Sugar & 36 ~ \ \ & 76 ~ \ \ & \textit{195} ~ \ \ \\ 
	 ~ Livestock & 60 ~ \ \ & 81 ~ \ \ & \textit{138} ~ \ \ \\ 
	 ~ Milk & 66 ~ \ \ & 82 ~ \ \ & \textit{141} ~ \ \ \\ 
	 ~ Meat & 57 ~ \ \ & 80 ~ \ \ & \textit{138} ~ \ \ \\ 
	 ~ Fish  & 49 ~ \ \ & 100 ~ \ \ & \textit{325} ~ \ \ \\ 
	\multicolumn{4}{l}{\textit{\normalsize{Net trade (min US\$)}}} \\ 
	 ~ Cereals & -19 ~ \ \ & -25 ~ \ \ & \textit{-55} ~ \ \ \\ 
	 ~ Fruit and vegetables & -1 ~ \ \ & -7 ~ \ \ & \textit{-10} ~ \ \ \\ 
	 ~ Meat &  ~ \ \ & \textit{0} ~ \ \ & \textit{0} ~ \ \ \\ 
	 ~ Dairy products & -3 ~ \ \ & 0 ~ \ \ & \textit{-1} ~ \ \ \\ 
	 ~ Fish & 0 ~ \ \ & 0 ~ \ \ & \textit{-7} ~ \ \ \\ 
	\multicolumn{4}{l}{\textcolor{FAOblue}{\textbf{\large{Environment}}}} \\ 
	 ~ Forest area (\%) & 13 ~ \ \ & 15 ~ \ \ & \textit{18} ~ \ \ \\ 
	 ~ Renewable water res withdrawn (\% of total) &  ~ \ \ & \textit{68} ~ \ \ & 68 ~ \ \ \\ 
	 ~ Terrestrial protect areas (\% total land area)  & 10 ~ \ \ & 10 ~ \ \ & \textit{11} ~ \ \ \\ 
	 ~ Organic area (\% total agricultural area) &  ~ \ \ & \textit{0} ~ \ \ & \textit{0} ~ \ \ \\ 
	 ~ Water withdrawal by agriculture (\% of total) &  ~ \ \ & \textit{68} ~ \ \ & 68 ~ \ \ \\ 
	 ~ Biofuel production (thousand kt of oil eq.) & 0 ~ \ \ & 0 ~ \ \ & \textit{0} ~ \ \ \\ 
	 ~ Wood pellet prod. (min tonnes) &  ~ \ \ &  ~ \ \ &  ~ \ \ \\ 
	 ~ GHG emissions from ag (Co2 eq, gigagrams) & 10 ~ \ \ & -7 ~ \ \ & \textit{2} ~ \ \ \\ 
       \toprule
      \end{tabular}
      \clearpage
\CountryData{ Samoa }
      \rowcolors{1}{FAOblue!10}{white}
      \begin{tabular}{L{3.9cm} R{1cm} R{1cm} R{1cm}}
      \toprule
      \multicolumn{1}{c}{} & \multicolumn{1}{c}{ 1992 } & \multicolumn{1}{c}{ 2002 } & \multicolumn{1}{c}{ 2014 } \\
      \midrule
	\multicolumn{4}{l}{\textcolor{FAOblue}{\textbf{\large{The setting}}}} \\ 
	 ~ Population, total (mln) & 0.2 ~ \ \ & 0.2 ~ \ \ & 0.2 ~ \ \ \\ 
	 ~ Population, rural (\% total population) & 0.1 ~ \ \ & 0.1 ~ \ \ & 0.2 ~ \ \ \\ 
	 ~ Govt expenditure on ag (\% total outlays) &  ~ \ \ & 3.8 ~ \ \ & \textit{1.8} ~ \ \ \\ 
	 ~ Area harvested (mln ha) & 0 ~ \ \ & 0 ~ \ \ & 0 ~ \ \ \\ 
	 ~ Cropping intensity ratio (\%) & 5.8 ~ \ \ & 6.6 ~ \ \ &  ~ \ \ \\ 
	 ~ Water resources (m\textsuperscript{3}/person/year) &  ~ \ \ &  ~ \ \ &  ~ \ \ \\ 
	 ~ Area equipped for irrigation (1000 ha) &  ~ \ \ &  ~ \ \ &  ~ \ \ \\ 
	 ~ Area irrigated (\%) &  ~ \ \ &  ~ \ \ &  ~ \ \ \\ 
	 ~ Employment in agriculture (\%) &  ~ \ \ & \textit{39.9} ~ \ \ &  ~ \ \ \\ 
	 ~ Employment in agriculture, female (\%) &  ~ \ \ & \textit{15.8} ~ \ \ &  ~ \ \ \\ 
	 ~ Fertilizers, Nitrogen (nutrients per ha) &  ~ \ \ & 0.1 ~ \ \ & \textit{0.1} ~ \ \ \\ 
	 ~ Fertilizers, Phosphate (nutrients per ha) &  ~ \ \ & 0.1 ~ \ \ & \textit{0.1} ~ \ \ \\ 
	 ~ Fertilizers, Potash (nutrients per ha) &  ~ \ \ & 0.1 ~ \ \ & \textit{0.1} ~ \ \ \\ 
	 ~ Energy consump, power irrigation (mln kWh) &  ~ \ \ &  ~ \ \ &  ~ \ \ \\ 
	 ~ Agr value added per worker (constant US\$) & \textit{3.1} ~ \ \ & 3 ~ \ \ & \textit{2.6} ~ \ \ \\ 
	\multicolumn{4}{l}{\textcolor{FAOblue}{\textbf{\large{Hunger dimensions}}}} \\ 
	 ~ Dietary energy supply (kcal/pc/day) & 2\,452 ~ \ \ & 2\,810 ~ \ \ & 2\,908 ~ \ \ \\ 
	 ~ Average dietary energy supply adequacy (\%) & 109 ~ \ \ & 125 ~ \ \ & 128 ~ \ \ \\ 
	 ~ Dietary en supp, cereals/roots/tubers (\%) & 30 ~ \ \ & 31 ~ \ \ & \textit{29} ~ \ \ \\ 
	 ~ Prevalence of undernourishment (\%) & 12 ~ \ \ & <5.0 ~ \ \ & <5.0 ~ \ \ \\ 
	 ~ GDP per capita (US\$, PPP) & 3\,508 ~ \ \ & 4\,784 ~ \ \ & \textit{5\,584} ~ \ \ \\ 
	 ~ Domestic food price volatility (index) &  ~ \ \ &  ~ \ \ &  ~ \ \ \\ 
	 ~ Cereal import dependency ratio (\%) &  ~ \ \ &  ~ \ \ &  ~ \ \ \\ 
	 ~ Underweight, children under-5 (\%) &  ~ \ \ & \textit{1.7} ~ \ \ &  ~ \ \ \\ 
	 ~ Improved water source (\% pop) & 89.8 ~ \ \ & 94.2 ~ \ \ & \textit{98.5} ~ \ \ \\ 
	\multicolumn{4}{l}{\textcolor{FAOblue}{\textbf{\large{Food Supply}}}} \\ 
	 ~ Food production value, (2004-2006 mln I\$) & 35 ~ \ \ & 43 ~ \ \ & \textit{53} ~ \ \ \\ 
	 ~ Agriculture, value added (\% GDP) &  ~ \ \ &  ~ \ \ &  ~ \ \ \\ 
	 ~ Food exports (mln US\$)  & 4 ~ \ \ & 4 ~ \ \ & \textit{6} ~ \ \ \\ 
	 ~ Food imports (mln US\$)  & 16 ~ \ \ & 27 ~ \ \ & \textit{80} ~ \ \ \\ 
	\multicolumn{4}{l}{\textit{\normalsize{Production indices (2004-06=100)}}} \\ 
	 ~ Net food & 73 ~ \ \ & 89 ~ \ \ & \textit{110} ~ \ \ \\ 
	 ~ Net crop & 67 ~ \ \ & 87 ~ \ \ & \textit{110} ~ \ \ \\ 
	 ~ Cereal &  ~ \ \ &  ~ \ \ &  ~ \ \ \\ 
	 ~ Vegetable oils & 77 ~ \ \ & 73 ~ \ \ & \textit{125} ~ \ \ \\ 
	 ~ Roots and tubers & 56 ~ \ \ & 93 ~ \ \ & \textit{104} ~ \ \ \\ 
	 ~ Fruit and vegetables & 56 ~ \ \ & 100 ~ \ \ & \textit{97} ~ \ \ \\ 
	 ~ Sugar & 100 ~ \ \ & 100 ~ \ \ & \textit{100} ~ \ \ \\ 
	 ~ Livestock & 93 ~ \ \ & 97 ~ \ \ & \textit{111} ~ \ \ \\ 
	 ~ Milk & 74 ~ \ \ & 100 ~ \ \ & \textit{91} ~ \ \ \\ 
	 ~ Meat & 91 ~ \ \ & 97 ~ \ \ & \textit{114} ~ \ \ \\ 
	 ~ Fish  & 23 ~ \ \ & 103 ~ \ \ & \textit{106} ~ \ \ \\ 
	\multicolumn{4}{l}{\textit{\normalsize{Net trade (min US\$)}}} \\ 
	 ~ Cereals & -4 ~ \ \ & -5 ~ \ \ & \textit{-21} ~ \ \ \\ 
	 ~ Fruit and vegetables & 3 ~ \ \ & 1 ~ \ \ & \textit{-3} ~ \ \ \\ 
	 ~ Meat & -6 ~ \ \ & -10 ~ \ \ & \textit{-29} ~ \ \ \\ 
	 ~ Dairy products & -2 ~ \ \ & -2 ~ \ \ & \textit{-9} ~ \ \ \\ 
	 ~ Fish & -3 ~ \ \ & 9 ~ \ \ & \textit{1} ~ \ \ \\ 
	\multicolumn{4}{l}{\textcolor{FAOblue}{\textbf{\large{Environment}}}} \\ 
	 ~ Forest area (\%) & 49 ~ \ \ & 60 ~ \ \ & \textit{60} ~ \ \ \\ 
	 ~ Renewable water res withdrawn (\% of total) &  ~ \ \ &  ~ \ \ &  ~ \ \ \\ 
	 ~ Terrestrial protect areas (\% total land area)  & 2 ~ \ \ & 3 ~ \ \ & \textit{7} ~ \ \ \\ 
	 ~ Organic area (\% total agricultural area) &  ~ \ \ &  ~ \ \ & \textit{96} ~ \ \ \\ 
	 ~ Water withdrawal by agriculture (\% of total) &  ~ \ \ &  ~ \ \ &  ~ \ \ \\ 
	 ~ Biofuel production (thousand kt of oil eq.) & 0 ~ \ \ & 0 ~ \ \ & \textit{0} ~ \ \ \\ 
	 ~ Wood pellet prod. (min tonnes) &  ~ \ \ &  ~ \ \ &  ~ \ \ \\ 
	 ~ GHG emissions from ag (Co2 eq, gigagrams) & -1 ~ \ \ & 0 ~ \ \ & \textit{0} ~ \ \ \\ 
       \toprule
      \end{tabular}
      \clearpage
\CountryData{ Sao Tome and Principe }
      \rowcolors{1}{FAOblue!10}{white}
      \begin{tabular}{L{3.9cm} R{1cm} R{1cm} R{1cm}}
      \toprule
      \multicolumn{1}{c}{} & \multicolumn{1}{c}{ 1992 } & \multicolumn{1}{c}{ 2002 } & \multicolumn{1}{c}{ 2014 } \\
      \midrule
	\multicolumn{4}{l}{\textcolor{FAOblue}{\textbf{\large{The setting}}}} \\ 
	 ~ Population, total (mln) & 0.1 ~ \ \ & 0.1 ~ \ \ & 0.2 ~ \ \ \\ 
	 ~ Population, rural (\% total population) & 0.1 ~ \ \ & 0.1 ~ \ \ & 0.1 ~ \ \ \\ 
	 ~ Govt expenditure on ag (\% total outlays) &  ~ \ \ &  ~ \ \ &  ~ \ \ \\ 
	 ~ Area harvested (mln ha) & 0 ~ \ \ & 0 ~ \ \ & 0 ~ \ \ \\ 
	 ~ Cropping intensity ratio (\%) & 2.9 ~ \ \ & 2.5 ~ \ \ &  ~ \ \ \\ 
	 ~ Water resources (m\textsuperscript{3}/person/year) & \textit{17} ~ \ \ & \textit{15} ~ \ \ & \textit{11} ~ \ \ \\ 
	 ~ Area equipped for irrigation (1000 ha) &  ~ \ \ &  ~ \ \ & \textit{10} ~ \ \ \\ 
	 ~ Area irrigated (\%) &  ~ \ \ &  ~ \ \ &  ~ \ \ \\ 
	 ~ Employment in agriculture (\%) & \textit{39.9} ~ \ \ & \textit{27.9} ~ \ \ &  ~ \ \ \\ 
	 ~ Employment in agriculture, female (\%) & \textit{35} ~ \ \ & \textit{22.8} ~ \ \ &  ~ \ \ \\ 
	 ~ Fertilizers, Nitrogen (nutrients per ha) &  ~ \ \ &  ~ \ \ &  ~ \ \ \\ 
	 ~ Fertilizers, Phosphate (nutrients per ha) &  ~ \ \ &  ~ \ \ &  ~ \ \ \\ 
	 ~ Fertilizers, Potash (nutrients per ha) &  ~ \ \ &  ~ \ \ &  ~ \ \ \\ 
	 ~ Energy consump, power irrigation (mln kWh) &  ~ \ \ &  ~ \ \ &  ~ \ \ \\ 
	 ~ Agr value added per worker (constant US\$) &  ~ \ \ & \textit{0.7} ~ \ \ & \textit{0.7} ~ \ \ \\ 
	\multicolumn{4}{l}{\textcolor{FAOblue}{\textbf{\large{Hunger dimensions}}}} \\ 
	 ~ Dietary energy supply (kcal/pc/day) & 2\,201 ~ \ \ & 2\,458 ~ \ \ & 2\,655 ~ \ \ \\ 
	 ~ Average dietary energy supply adequacy (\%) & 102 ~ \ \ & 111 ~ \ \ & 120 ~ \ \ \\ 
	 ~ Dietary en supp, cereals/roots/tubers (\%) & 53 ~ \ \ & 49 ~ \ \ & \textit{47} ~ \ \ \\ 
	 ~ Prevalence of undernourishment (\%) & 23.3 ~ \ \ & 15 ~ \ \ & 6.4 ~ \ \ \\ 
	 ~ GDP per capita (US\$, PPP) &  ~ \ \ & 2\,191 ~ \ \ & \textit{2\,876} ~ \ \ \\ 
	 ~ Domestic food price volatility (index) &  ~ \ \ & 8.4 ~ \ \ & \textit{50.5} ~ \ \ \\ 
	 ~ Cereal import dependency ratio (\%) &  ~ \ \ &  ~ \ \ &  ~ \ \ \\ 
	 ~ Underweight, children under-5 (\%) &  ~ \ \ & \textit{10.1} ~ \ \ & \textit{14.4} ~ \ \ \\ 
	 ~ Improved water source (\% pop) & \textit{74} ~ \ \ & 81.8 ~ \ \ & \textit{97} ~ \ \ \\ 
	\multicolumn{4}{l}{\textcolor{FAOblue}{\textbf{\large{Food Supply}}}} \\ 
	 ~ Food production value, (2004-2006 mln I\$) & 16 ~ \ \ & 27 ~ \ \ & \textit{29} ~ \ \ \\ 
	 ~ Agriculture, value added (\% GDP) &  ~ \ \ & 21 ~ \ \ & \textit{20} ~ \ \ \\ 
	 ~ Food exports (mln US\$)  & 4 ~ \ \ & 5 ~ \ \ & \textit{5} ~ \ \ \\ 
	 ~ Food imports (mln US\$)  & 5 ~ \ \ & 10 ~ \ \ & \textit{32} ~ \ \ \\ 
	\multicolumn{4}{l}{\textit{\normalsize{Production indices (2004-06=100)}}} \\ 
	 ~ Net food & 64 ~ \ \ & 103 ~ \ \ & \textit{113} ~ \ \ \\ 
	 ~ Net crop & 63 ~ \ \ & 104 ~ \ \ & \textit{111} ~ \ \ \\ 
	 ~ Cereal & 144 ~ \ \ & 92 ~ \ \ & \textit{27} ~ \ \ \\ 
	 ~ Vegetable oils & 49 ~ \ \ & 92 ~ \ \ & \textit{142} ~ \ \ \\ 
	 ~ Roots and tubers & 29 ~ \ \ & 103 ~ \ \ & \textit{36} ~ \ \ \\ 
	 ~ Fruit and vegetables & 51 ~ \ \ & 97 ~ \ \ & \textit{128} ~ \ \ \\ 
	 ~ Sugar &  ~ \ \ &  ~ \ \ &  ~ \ \ \\ 
	 ~ Livestock & 65 ~ \ \ & 93 ~ \ \ & \textit{135} ~ \ \ \\ 
	 ~ Milk & 86 ~ \ \ & 99 ~ \ \ & \textit{121} ~ \ \ \\ 
	 ~ Meat & 68 ~ \ \ & 94 ~ \ \ & \textit{136} ~ \ \ \\ 
	 ~ Fish  &  ~ \ \ &  ~ \ \ &  ~ \ \ \\ 
	\multicolumn{4}{l}{\textit{\normalsize{Net trade (min US\$)}}} \\ 
	 ~ Cereals &  ~ \ \ &  ~ \ \ & \textit{-14} ~ \ \ \\ 
	 ~ Fruit and vegetables & -1 ~ \ \ & -1 ~ \ \ & \textit{-3} ~ \ \ \\ 
	 ~ Meat &  ~ \ \ &  ~ \ \ &  ~ \ \ \\ 
	 ~ Dairy products &  ~ \ \ &  ~ \ \ & \textit{-3} ~ \ \ \\ 
	 ~ Fish & -1 ~ \ \ & 0 ~ \ \ & \textit{0} ~ \ \ \\ 
	\multicolumn{4}{l}{\textcolor{FAOblue}{\textbf{\large{Environment}}}} \\ 
	 ~ Forest area (\%) & 28 ~ \ \ & 28 ~ \ \ & \textit{28} ~ \ \ \\ 
	 ~ Renewable water res withdrawn (\% of total) &  ~ \ \ &  ~ \ \ &  ~ \ \ \\ 
	 ~ Terrestrial protect areas (\% total land area)  &  ~ \ \ &  ~ \ \ &  ~ \ \ \\ 
	 ~ Organic area (\% total agricultural area) &  ~ \ \ &  ~ \ \ & \textit{8} ~ \ \ \\ 
	 ~ Water withdrawal by agriculture (\% of total) &  ~ \ \ &  ~ \ \ &  ~ \ \ \\ 
	 ~ Biofuel production (thousand kt of oil eq.) &  ~ \ \ &  ~ \ \ &  ~ \ \ \\ 
	 ~ Wood pellet prod. (min tonnes) &  ~ \ \ &  ~ \ \ &  ~ \ \ \\ 
	 ~ GHG emissions from ag (Co2 eq, gigagrams) & 0 ~ \ \ & 0 ~ \ \ & \textit{0} ~ \ \ \\ 
       \toprule
      \end{tabular}
      \clearpage
\CountryData{ Saudi Arabia }
      \rowcolors{1}{FAOblue!10}{white}
      \begin{tabular}{L{3.9cm} R{1cm} R{1cm} R{1cm}}
      \toprule
      \multicolumn{1}{c}{} & \multicolumn{1}{c}{ 1992 } & \multicolumn{1}{c}{ 2002 } & \multicolumn{1}{c}{ 2014 } \\
      \midrule
	\multicolumn{4}{l}{\textcolor{FAOblue}{\textbf{\large{The setting}}}} \\ 
	 ~ Population, total (mln) & 17.3 ~ \ \ & 21.8 ~ \ \ & 29.4 ~ \ \ \\ 
	 ~ Population, rural (\% total population) & 3.8 ~ \ \ & 4.3 ~ \ \ & 5 ~ \ \ \\ 
	 ~ Govt expenditure on ag (\% total outlays) &  ~ \ \ &  ~ \ \ &  ~ \ \ \\ 
	 ~ Area harvested (mln ha) & 5 ~ \ \ & 3 ~ \ \ & 1 ~ \ \ \\ 
	 ~ Cropping intensity ratio (\%) & 0 ~ \ \ & 0 ~ \ \ &  ~ \ \ \\ 
	 ~ Water resources (m\textsuperscript{3}/person/year) & \textit{0} ~ \ \ & \textit{0} ~ \ \ & \textit{0} ~ \ \ \\ 
	 ~ Area equipped for irrigation (1000 ha) &  ~ \ \ &  ~ \ \ & \textit{1\,620} ~ \ \ \\ 
	 ~ Area irrigated (\%) & 100 ~ \ \ &  ~ \ \ &  ~ \ \ \\ 
	 ~ Employment in agriculture (\%) & 7.7 ~ \ \ & 4.7 ~ \ \ & \textit{4.7} ~ \ \ \\ 
	 ~ Employment in agriculture, female (\%) & 0.4 ~ \ \ & 0.6 ~ \ \ & \textit{0.2} ~ \ \ \\ 
	 ~ Fertilizers, Nitrogen (nutrients per ha) &  ~ \ \ & 0.6 ~ \ \ & \textit{2.5} ~ \ \ \\ 
	 ~ Fertilizers, Phosphate (nutrients per ha) &  ~ \ \ & 0.6 ~ \ \ & \textit{4} ~ \ \ \\ 
	 ~ Fertilizers, Potash (nutrients per ha) &  ~ \ \ & 0 ~ \ \ & \textit{0} ~ \ \ \\ 
	 ~ Energy consump, power irrigation (mln kWh) & 2\,558 ~ \ \ & 2\,558 ~ \ \ & \textit{2\,204} ~ \ \ \\ 
	 ~ Agr value added per worker (constant US\$) & 9.6 ~ \ \ & 15.5 ~ \ \ & \textit{25.6} ~ \ \ \\ 
	\multicolumn{4}{l}{\textcolor{FAOblue}{\textbf{\large{Hunger dimensions}}}} \\ 
	 ~ Dietary energy supply (kcal/pc/day) & 2\,797 ~ \ \ & 3\,033 ~ \ \ & 3\,266 ~ \ \ \\ 
	 ~ Average dietary energy supply adequacy (\%) & 126 ~ \ \ & 133 ~ \ \ & 137 ~ \ \ \\ 
	 ~ Dietary en supp, cereals/roots/tubers (\%) & 51 ~ \ \ & 50 ~ \ \ & \textit{48} ~ \ \ \\ 
	 ~ Prevalence of undernourishment (\%) & <5.0 ~ \ \ & <5.0 ~ \ \ & <5.0 ~ \ \ \\ 
	 ~ GDP per capita (US\$, PPP) & 37\,762 ~ \ \ & 34\,443 ~ \ \ & \textit{52\,068} ~ \ \ \\ 
	 ~ Domestic food price volatility (index) &  ~ \ \ & 6.5 ~ \ \ & 3.8 ~ \ \ \\ 
	 ~ Cereal import dependency ratio (\%) & 36 ~ \ \ & 68.8 ~ \ \ & \textit{88.1} ~ \ \ \\ 
	 ~ Underweight, children under-5 (\%) & \textit{13.5} ~ \ \ & \textit{5.3} ~ \ \ & \textit{5.3} ~ \ \ \\ 
	 ~ Improved water source (\% pop) & 92.3 ~ \ \ & 95.7 ~ \ \ & \textit{97} ~ \ \ \\ 
	\multicolumn{4}{l}{\textcolor{FAOblue}{\textbf{\large{Food Supply}}}} \\ 
	 ~ Food production value, (2004-2006 mln I\$) & 2\,467 ~ \ \ & 2\,886 ~ \ \ & \textit{3\,526} ~ \ \ \\ 
	 ~ Agriculture, value added (\% GDP) & 6 ~ \ \ & 5 ~ \ \ & \textit{2} ~ \ \ \\ 
	 ~ Food exports (mln US\$)  & 401 ~ \ \ & 503 ~ \ \ & \textit{3\,308} ~ \ \ \\ 
	 ~ Food imports (mln US\$)  & 3\,294 ~ \ \ & 4\,486 ~ \ \ & \textit{18\,512} ~ \ \ \\ 
	\multicolumn{4}{l}{\textit{\normalsize{Production indices (2004-06=100)}}} \\ 
	 ~ Net food & 76 ~ \ \ & 88 ~ \ \ & \textit{108} ~ \ \ \\ 
	 ~ Net crop & 90 ~ \ \ & 85 ~ \ \ & \textit{89} ~ \ \ \\ 
	 ~ Cereal & 150 ~ \ \ & 92 ~ \ \ & \textit{31} ~ \ \ \\ 
	 ~ Vegetable oils & 59 ~ \ \ & 66 ~ \ \ & \textit{77} ~ \ \ \\ 
	 ~ Roots and tubers & 19 ~ \ \ & 73 ~ \ \ & \textit{108} ~ \ \ \\ 
	 ~ Fruit and vegetables & 71 ~ \ \ & 84 ~ \ \ & \textit{110} ~ \ \ \\ 
	 ~ Sugar &  ~ \ \ &  ~ \ \ &  ~ \ \ \\ 
	 ~ Livestock & 62 ~ \ \ & 92 ~ \ \ & \textit{129} ~ \ \ \\ 
	 ~ Milk & 42 ~ \ \ & 87 ~ \ \ & \textit{183} ~ \ \ \\ 
	 ~ Meat & 68 ~ \ \ & 94 ~ \ \ & \textit{107} ~ \ \ \\ 
	 ~ Fish  & 65 ~ \ \ & 86 ~ \ \ & \textit{131} ~ \ \ \\ 
	\multicolumn{4}{l}{\textit{\normalsize{Net trade (min US\$)}}} \\ 
	 ~ Cereals & -1\,004 ~ \ \ & -1\,157 ~ \ \ & \textit{-5\,783} ~ \ \ \\ 
	 ~ Fruit and vegetables & -414 ~ \ \ & -614 ~ \ \ & \textit{-1\,844} ~ \ \ \\ 
	 ~ Meat & -358 ~ \ \ & -514 ~ \ \ & \textit{-2\,435} ~ \ \ \\ 
	 ~ Dairy products & -186 ~ \ \ & -410 ~ \ \ & \textit{-891} ~ \ \ \\ 
	 ~ Fish & -53 ~ \ \ & -119 ~ \ \ & \textit{-571} ~ \ \ \\ 
	\multicolumn{4}{l}{\textcolor{FAOblue}{\textbf{\large{Environment}}}} \\ 
	 ~ Forest area (\%) & 0 ~ \ \ & 0 ~ \ \ & \textit{0} ~ \ \ \\ 
	 ~ Renewable water res withdrawn (\% of total) &  ~ \ \ &  ~ \ \ & 88 ~ \ \ \\ 
	 ~ Terrestrial protect areas (\% total land area)  & 8 ~ \ \ & 31 ~ \ \ & \textit{31} ~ \ \ \\ 
	 ~ Organic area (\% total agricultural area) &  ~ \ \ & \textit{0} ~ \ \ & \textit{0} ~ \ \ \\ 
	 ~ Water withdrawal by agriculture (\% of total) &  ~ \ \ &  ~ \ \ & 88 ~ \ \ \\ 
	 ~ Biofuel production (thousand kt of oil eq.) &  ~ \ \ &  ~ \ \ &  ~ \ \ \\ 
	 ~ Wood pellet prod. (min tonnes) &  ~ \ \ &  ~ \ \ &  ~ \ \ \\ 
	 ~ GHG emissions from ag (Co2 eq, gigagrams) & 6 ~ \ \ & 5 ~ \ \ & \textit{8} ~ \ \ \\ 
       \toprule
      \end{tabular}
      \clearpage
\CountryData{ Senegal }
      \rowcolors{1}{FAOblue!10}{white}
      \begin{tabular}{L{3.9cm} R{1cm} R{1cm} R{1cm}}
      \toprule
      \multicolumn{1}{c}{} & \multicolumn{1}{c}{ 1992 } & \multicolumn{1}{c}{ 2002 } & \multicolumn{1}{c}{ 2014 } \\
      \midrule
	\multicolumn{4}{l}{\textcolor{FAOblue}{\textbf{\large{The setting}}}} \\ 
	 ~ Population, total (mln) & 8 ~ \ \ & 10.4 ~ \ \ & 14.5 ~ \ \ \\ 
	 ~ Population, rural (\% total population) & 4.9 ~ \ \ & 6.2 ~ \ \ & 8.2 ~ \ \ \\ 
	 ~ Govt expenditure on ag (\% total outlays) &  ~ \ \ &  ~ \ \ &  ~ \ \ \\ 
	 ~ Area harvested (mln ha) & 1 ~ \ \ & 1 ~ \ \ & 1 ~ \ \ \\ 
	 ~ Cropping intensity ratio (\%) & 0.1 ~ \ \ & 0.1 ~ \ \ &  ~ \ \ \\ 
	 ~ Water resources (m\textsuperscript{3}/person/year) & \textit{5} ~ \ \ & \textit{4} ~ \ \ & \textit{3} ~ \ \ \\ 
	 ~ Area equipped for irrigation (1000 ha) &  ~ \ \ &  ~ \ \ & \textit{120} ~ \ \ \\ 
	 ~ Area irrigated (\%) &  ~ \ \ & \textit{96.6} ~ \ \ &  ~ \ \ \\ 
	 ~ Employment in agriculture (\%) &  ~ \ \ & \textit{45.6} ~ \ \ & \textit{33.7} ~ \ \ \\ 
	 ~ Employment in agriculture, female (\%) &  ~ \ \ & \textit{43.7} ~ \ \ & \textit{33} ~ \ \ \\ 
	 ~ Fertilizers, Nitrogen (nutrients per ha) &  ~ \ \ & 2.2 ~ \ \ & \textit{2.3} ~ \ \ \\ 
	 ~ Fertilizers, Phosphate (nutrients per ha) &  ~ \ \ & 1.1 ~ \ \ & \textit{0.2} ~ \ \ \\ 
	 ~ Fertilizers, Potash (nutrients per ha) &  ~ \ \ & 0.8 ~ \ \ & \textit{0.5} ~ \ \ \\ 
	 ~ Energy consump, power irrigation (mln kWh) & 1 ~ \ \ & 1 ~ \ \ & \textit{1} ~ \ \ \\ 
	 ~ Agr value added per worker (constant US\$) & 0.4 ~ \ \ & 0.3 ~ \ \ & \textit{0.4} ~ \ \ \\ 
	\multicolumn{4}{l}{\textcolor{FAOblue}{\textbf{\large{Hunger dimensions}}}} \\ 
	 ~ Dietary energy supply (kcal/pc/day) & 2\,191 ~ \ \ & 2\,180 ~ \ \ & 2\,208 ~ \ \ \\ 
	 ~ Average dietary energy supply adequacy (\%) & 100 ~ \ \ & 98 ~ \ \ & 99 ~ \ \ \\ 
	 ~ Dietary en supp, cereals/roots/tubers (\%) & 61 ~ \ \ & 62 ~ \ \ & \textit{60} ~ \ \ \\ 
	 ~ Prevalence of undernourishment (\%) & 25.5 ~ \ \ & 27.4 ~ \ \ & 23 ~ \ \ \\ 
	 ~ GDP per capita (US\$, PPP) & 1\,813 ~ \ \ & 1\,915 ~ \ \ & \textit{2\,170} ~ \ \ \\ 
	 ~ Domestic food price volatility (index) &  ~ \ \ & 10.8 ~ \ \ & 8.7 ~ \ \ \\ 
	 ~ Cereal import dependency ratio (\%) & 39.9 ~ \ \ & 53.5 ~ \ \ & \textit{46.9} ~ \ \ \\ 
	 ~ Underweight, children under-5 (\%) & 19 ~ \ \ & \textit{14.5} ~ \ \ & \textit{16.8} ~ \ \ \\ 
	 ~ Improved water source (\% pop) & 61.2 ~ \ \ & 67.6 ~ \ \ & \textit{74.1} ~ \ \ \\ 
	\multicolumn{4}{l}{\textcolor{FAOblue}{\textbf{\large{Food Supply}}}} \\ 
	 ~ Food production value, (2004-2006 mln I\$) & 771 ~ \ \ & 724 ~ \ \ & \textit{1\,369} ~ \ \ \\ 
	 ~ Agriculture, value added (\% GDP) & 19 ~ \ \ & 16 ~ \ \ & \textit{18} ~ \ \ \\ 
	 ~ Food exports (mln US\$)  & 71 ~ \ \ & 77 ~ \ \ & \textit{318} ~ \ \ \\ 
	 ~ Food imports (mln US\$)  & 322 ~ \ \ & 469 ~ \ \ & \textit{1\,533} ~ \ \ \\ 
	\multicolumn{4}{l}{\textit{\normalsize{Production indices (2004-06=100)}}} \\ 
	 ~ Net food & 72 ~ \ \ & 68 ~ \ \ & \textit{128} ~ \ \ \\ 
	 ~ Net crop & 71 ~ \ \ & 61 ~ \ \ & \textit{125} ~ \ \ \\ 
	 ~ Cereal & 75 ~ \ \ & 70 ~ \ \ & \textit{125} ~ \ \ \\ 
	 ~ Vegetable oils & 83 ~ \ \ & 33 ~ \ \ & \textit{117} ~ \ \ \\ 
	 ~ Roots and tubers & 21 ~ \ \ & 49 ~ \ \ & \textit{67} ~ \ \ \\ 
	 ~ Fruit and vegetables & 51 ~ \ \ & 88 ~ \ \ & \textit{162} ~ \ \ \\ 
	 ~ Sugar & 101 ~ \ \ & 99 ~ \ \ & \textit{107} ~ \ \ \\ 
	 ~ Livestock & 78 ~ \ \ & 86 ~ \ \ & \textit{133} ~ \ \ \\ 
	 ~ Milk & 82 ~ \ \ & 87 ~ \ \ & \textit{191} ~ \ \ \\ 
	 ~ Meat & 80 ~ \ \ & 87 ~ \ \ & \textit{123} ~ \ \ \\ 
	 ~ Fish  & 93 ~ \ \ & 99 ~ \ \ & \textit{117} ~ \ \ \\ 
	\multicolumn{4}{l}{\textit{\normalsize{Net trade (min US\$)}}} \\ 
	 ~ Cereals & -116 ~ \ \ & -254 ~ \ \ & \textit{-726} ~ \ \ \\ 
	 ~ Fruit and vegetables & -19 ~ \ \ & -24 ~ \ \ & \textit{-42} ~ \ \ \\ 
	 ~ Meat & -4 ~ \ \ & -14 ~ \ \ & \textit{-14} ~ \ \ \\ 
	 ~ Dairy products & -60 ~ \ \ & -29 ~ \ \ & \textit{-71} ~ \ \ \\ 
	 ~ Fish & 167 ~ \ \ & 224 ~ \ \ & \textit{251} ~ \ \ \\ 
	\multicolumn{4}{l}{\textcolor{FAOblue}{\textbf{\large{Environment}}}} \\ 
	 ~ Forest area (\%) & 48 ~ \ \ & 46 ~ \ \ & \textit{44} ~ \ \ \\ 
	 ~ Renewable water res withdrawn (\% of total) &  ~ \ \ & 93 ~ \ \ & 93 ~ \ \ \\ 
	 ~ Terrestrial protect areas (\% total land area)  & 24 ~ \ \ & 24 ~ \ \ & \textit{25} ~ \ \ \\ 
	 ~ Organic area (\% total agricultural area) &  ~ \ \ &  ~ \ \ & \textit{0} ~ \ \ \\ 
	 ~ Water withdrawal by agriculture (\% of total) &  ~ \ \ & 93 ~ \ \ & 93 ~ \ \ \\ 
	 ~ Biofuel production (thousand kt of oil eq.) & 2 ~ \ \ & 5 ~ \ \ & \textit{5} ~ \ \ \\ 
	 ~ Wood pellet prod. (min tonnes) &  ~ \ \ &  ~ \ \ &  ~ \ \ \\ 
	 ~ GHG emissions from ag (Co2 eq, gigagrams) & 17 ~ \ \ & 17 ~ \ \ & \textit{17} ~ \ \ \\ 
       \toprule
      \end{tabular}
      \clearpage
\CountryData{ Serbia }
      \rowcolors{1}{FAOblue!10}{white}
      \begin{tabular}{L{3.9cm} R{1cm} R{1cm} R{1cm}}
      \toprule
      \multicolumn{1}{c}{} & \multicolumn{1}{c}{ 1992 } & \multicolumn{1}{c}{ 2002 } & \multicolumn{1}{c}{ 2014 } \\
      \midrule
	\multicolumn{4}{l}{\textcolor{FAOblue}{\textbf{\large{The setting}}}} \\ 
	 ~ Population, total (mln) &  ~ \ \ &  ~ \ \ & 9.5 ~ \ \ \\ 
	 ~ Population, rural (\% total population) &  ~ \ \ &  ~ \ \ & 4 ~ \ \ \\ 
	 ~ Govt expenditure on ag (\% total outlays) &  ~ \ \ &  ~ \ \ & \textit{2.5} ~ \ \ \\ 
	 ~ Area harvested (mln ha) &  ~ \ \ &  ~ \ \ & 9 ~ \ \ \\ 
	 ~ Cropping intensity ratio (\%) &  ~ \ \ &  ~ \ \ &  ~ \ \ \\ 
	 ~ Water resources (m\textsuperscript{3}/person/year) &  ~ \ \ &  ~ \ \ & \textit{17} ~ \ \ \\ 
	 ~ Area equipped for irrigation (1000 ha) &  ~ \ \ &  ~ \ \ & \textit{95} ~ \ \ \\ 
	 ~ Area irrigated (\%) &  ~ \ \ &  ~ \ \ & \textit{37.2} ~ \ \ \\ 
	 ~ Employment in agriculture (\%) &  ~ \ \ & \textit{23.3} ~ \ \ & \textit{22.2} ~ \ \ \\ 
	 ~ Employment in agriculture, female (\%) &  ~ \ \ &  ~ \ \ &  ~ \ \ \\ 
	 ~ Fertilizers, Nitrogen (nutrients per ha) &  ~ \ \ &  ~ \ \ & \textit{70.1} ~ \ \ \\ 
	 ~ Fertilizers, Phosphate (nutrients per ha) &  ~ \ \ &  ~ \ \ & \textit{25.5} ~ \ \ \\ 
	 ~ Fertilizers, Potash (nutrients per ha) &  ~ \ \ &  ~ \ \ & \textit{18.2} ~ \ \ \\ 
	 ~ Energy consump, power irrigation (mln kWh) &  ~ \ \ &  ~ \ \ &  ~ \ \ \\ 
	 ~ Agr value added per worker (constant US\$) &  ~ \ \ &  ~ \ \ & \textit{3.9} ~ \ \ \\ 
	\multicolumn{4}{l}{\textcolor{FAOblue}{\textbf{\large{Hunger dimensions}}}} \\ 
	 ~ Dietary energy supply (kcal/pc/day) &  ~ \ \ &  ~ \ \ &  ~ \ \ \\ 
	 ~ Average dietary energy supply adequacy (\%) &  ~ \ \ & \textit{108} ~ \ \ & 108 ~ \ \ \\ 
	 ~ Dietary en supp, cereals/roots/tubers (\%) &  ~ \ \ & \textit{39} ~ \ \ & \textit{40} ~ \ \ \\ 
	 ~ Prevalence of undernourishment (\%) & <5.0 ~ \ \ & <5.0 ~ \ \ & <5.0 ~ \ \ \\ 
	 ~ GDP per capita (US\$, PPP) &  ~ \ \ &  ~ \ \ & \textit{12\,892} ~ \ \ \\ 
	 ~ Domestic food price volatility (index) &  ~ \ \ & 16.7 ~ \ \ & \textit{8.5} ~ \ \ \\ 
	 ~ Cereal import dependency ratio (\%) &  ~ \ \ & \textit{-22.8} ~ \ \ & \textit{-31.8} ~ \ \ \\ 
	 ~ Underweight, children under-5 (\%) &  ~ \ \ & \textit{1.8} ~ \ \ & 1.8 ~ \ \ \\ 
	 ~ Improved water source (\% pop) &  ~ \ \ &  ~ \ \ & \textit{99.2} ~ \ \ \\ 
	\multicolumn{4}{l}{\textcolor{FAOblue}{\textbf{\large{Food Supply}}}} \\ 
	 ~ Food production value, (2004-2006 mln I\$) &  ~ \ \ &  ~ \ \ & \textit{4\,062} ~ \ \ \\ 
	 ~ Agriculture, value added (\% GDP) &  ~ \ \ & 15 ~ \ \ & \textit{9} ~ \ \ \\ 
	 ~ Food exports (mln US\$)  &  ~ \ \ &  ~ \ \ & \textit{2\,207} ~ \ \ \\ 
	 ~ Food imports (mln US\$)  &  ~ \ \ &  ~ \ \ & \textit{932} ~ \ \ \\ 
	\multicolumn{4}{l}{\textit{\normalsize{Production indices (2004-06=100)}}} \\ 
	 ~ Net food &  ~ \ \ &  ~ \ \ & \textit{108} ~ \ \ \\ 
	 ~ Net crop &  ~ \ \ &  ~ \ \ & \textit{105} ~ \ \ \\ 
	 ~ Cereal &  ~ \ \ &  ~ \ \ & \textit{110} ~ \ \ \\ 
	 ~ Vegetable oils &  ~ \ \ &  ~ \ \ & \textit{122} ~ \ \ \\ 
	 ~ Roots and tubers &  ~ \ \ &  ~ \ \ & \textit{70} ~ \ \ \\ 
	 ~ Fruit and vegetables &  ~ \ \ &  ~ \ \ & \textit{107} ~ \ \ \\ 
	 ~ Sugar &  ~ \ \ &  ~ \ \ & \textit{94} ~ \ \ \\ 
	 ~ Livestock &  ~ \ \ &  ~ \ \ & \textit{101} ~ \ \ \\ 
	 ~ Milk &  ~ \ \ &  ~ \ \ & \textit{93} ~ \ \ \\ 
	 ~ Meat &  ~ \ \ &  ~ \ \ & \textit{104} ~ \ \ \\ 
	 ~ Fish  & 0 ~ \ \ & 0 ~ \ \ & \textit{441} ~ \ \ \\ 
	\multicolumn{4}{l}{\textit{\normalsize{Net trade (min US\$)}}} \\ 
	 ~ Cereals &  ~ \ \ &  ~ \ \ & \textit{737} ~ \ \ \\ 
	 ~ Fruit and vegetables &  ~ \ \ &  ~ \ \ & \textit{231} ~ \ \ \\ 
	 ~ Meat &  ~ \ \ &  ~ \ \ & \textit{-15} ~ \ \ \\ 
	 ~ Dairy products &  ~ \ \ &  ~ \ \ & \textit{21} ~ \ \ \\ 
	 ~ Fish &  ~ \ \ &  ~ \ \ & \textit{-94} ~ \ \ \\ 
	\multicolumn{4}{l}{\textcolor{FAOblue}{\textbf{\large{Environment}}}} \\ 
	 ~ Forest area (\%) &  ~ \ \ &  ~ \ \ & \textit{32} ~ \ \ \\ 
	 ~ Renewable water res withdrawn (\% of total) &  ~ \ \ &  ~ \ \ & 2 ~ \ \ \\ 
	 ~ Terrestrial protect areas (\% total land area)  &  ~ \ \ &  ~ \ \ & \textit{6} ~ \ \ \\ 
	 ~ Organic area (\% total agricultural area) &  ~ \ \ &  ~ \ \ & \textit{0} ~ \ \ \\ 
	 ~ Water withdrawal by agriculture (\% of total) &  ~ \ \ &  ~ \ \ & 2 ~ \ \ \\ 
	 ~ Biofuel production (thousand kt of oil eq.) &  ~ \ \ &  ~ \ \ & \textit{0} ~ \ \ \\ 
	 ~ Wood pellet prod. (min tonnes) &  ~ \ \ &  ~ \ \ & \textit{162} ~ \ \ \\ 
	 ~ GHG emissions from ag (Co2 eq, gigagrams) &  ~ \ \ &  ~ \ \ & \textit{-69} ~ \ \ \\ 
       \toprule
      \end{tabular}
      \clearpage
\CountryData{ Sierra Leone }
      \rowcolors{1}{FAOblue!10}{white}
      \begin{tabular}{L{3.9cm} R{1cm} R{1cm} R{1cm}}
      \toprule
      \multicolumn{1}{c}{} & \multicolumn{1}{c}{ 1992 } & \multicolumn{1}{c}{ 2002 } & \multicolumn{1}{c}{ 2014 } \\
      \midrule
	\multicolumn{4}{l}{\textcolor{FAOblue}{\textbf{\large{The setting}}}} \\ 
	 ~ Population, total (mln) & 4 ~ \ \ & 4.5 ~ \ \ & 6.2 ~ \ \ \\ 
	 ~ Population, rural (\% total population) & 2.7 ~ \ \ & 2.9 ~ \ \ & 3.7 ~ \ \ \\ 
	 ~ Govt expenditure on ag (\% total outlays) &  ~ \ \ & 1.1 ~ \ \ & \textit{1.1} ~ \ \ \\ 
	 ~ Area harvested (mln ha) & 1 ~ \ \ & 1 ~ \ \ & 4 ~ \ \ \\ 
	 ~ Cropping intensity ratio (\%) & 0.2 ~ \ \ & 0.2 ~ \ \ &  ~ \ \ \\ 
	 ~ Water resources (m\textsuperscript{3}/person/year) & \textit{40} ~ \ \ & \textit{34} ~ \ \ & \textit{26} ~ \ \ \\ 
	 ~ Area equipped for irrigation (1000 ha) &  ~ \ \ &  ~ \ \ & \textit{30} ~ \ \ \\ 
	 ~ Area irrigated (\%) &  ~ \ \ &  ~ \ \ &  ~ \ \ \\ 
	 ~ Employment in agriculture (\%) &  ~ \ \ & \textit{68.5} ~ \ \ &  ~ \ \ \\ 
	 ~ Employment in agriculture, female (\%) &  ~ \ \ & \textit{71.1} ~ \ \ &  ~ \ \ \\ 
	 ~ Fertilizers, Nitrogen (nutrients per ha) &  ~ \ \ &  ~ \ \ &  ~ \ \ \\ 
	 ~ Fertilizers, Phosphate (nutrients per ha) &  ~ \ \ &  ~ \ \ &  ~ \ \ \\ 
	 ~ Fertilizers, Potash (nutrients per ha) &  ~ \ \ &  ~ \ \ &  ~ \ \ \\ 
	 ~ Energy consump, power irrigation (mln kWh) & 0 ~ \ \ & 0 ~ \ \ & \textit{0} ~ \ \ \\ 
	 ~ Agr value added per worker (constant US\$) & 0.5 ~ \ \ & 0.6 ~ \ \ & \textit{0.9} ~ \ \ \\ 
	\multicolumn{4}{l}{\textcolor{FAOblue}{\textbf{\large{Hunger dimensions}}}} \\ 
	 ~ Dietary energy supply (kcal/pc/day) & 2\,016 ~ \ \ & 2\,025 ~ \ \ & 2\,408 ~ \ \ \\ 
	 ~ Average dietary energy supply adequacy (\%) & 95 ~ \ \ & 94 ~ \ \ & 111 ~ \ \ \\ 
	 ~ Dietary en supp, cereals/roots/tubers (\%) & 61 ~ \ \ & 63 ~ \ \ & \textit{63} ~ \ \ \\ 
	 ~ Prevalence of undernourishment (\%) & 41.7 ~ \ \ & 41 ~ \ \ & 22.7 ~ \ \ \\ 
	 ~ GDP per capita (US\$, PPP) & 1\,108 ~ \ \ & 1\,086 ~ \ \ & \textit{1\,495} ~ \ \ \\ 
	 ~ Domestic food price volatility (index) &  ~ \ \ & 14.9 ~ \ \ & 3.3 ~ \ \ \\ 
	 ~ Cereal import dependency ratio (\%) & 29.7 ~ \ \ & 43.3 ~ \ \ & \textit{19.7} ~ \ \ \\ 
	 ~ Underweight, children under-5 (\%) & \textit{25.4} ~ \ \ & \textit{28.3} ~ \ \ & \textit{18.1} ~ \ \ \\ 
	 ~ Improved water source (\% pop) & 38.8 ~ \ \ & 49.4 ~ \ \ & \textit{60.1} ~ \ \ \\ 
	\multicolumn{4}{l}{\textcolor{FAOblue}{\textbf{\large{Food Supply}}}} \\ 
	 ~ Food production value, (2004-2006 mln I\$) & 370 ~ \ \ & 398 ~ \ \ & \textit{1\,172} ~ \ \ \\ 
	 ~ Agriculture, value added (\% GDP) & 38 ~ \ \ & 50 ~ \ \ & \textit{59} ~ \ \ \\ 
	 ~ Food exports (mln US\$)  & 5 ~ \ \ & 4 ~ \ \ & \textit{30} ~ \ \ \\ 
	 ~ Food imports (mln US\$)  & 88 ~ \ \ & 171 ~ \ \ & \textit{278} ~ \ \ \\ 
	\multicolumn{4}{l}{\textit{\normalsize{Production indices (2004-06=100)}}} \\ 
	 ~ Net food & 54 ~ \ \ & 59 ~ \ \ & \textit{172} ~ \ \ \\ 
	 ~ Net crop & 54 ~ \ \ & 58 ~ \ \ & \textit{170} ~ \ \ \\ 
	 ~ Cereal & 61 ~ \ \ & 53 ~ \ \ & \textit{164} ~ \ \ \\ 
	 ~ Vegetable oils & 82 ~ \ \ & 77 ~ \ \ & \textit{121} ~ \ \ \\ 
	 ~ Roots and tubers & 15 ~ \ \ & 32 ~ \ \ & \textit{256} ~ \ \ \\ 
	 ~ Fruit and vegetables & 76 ~ \ \ & 86 ~ \ \ & \textit{128} ~ \ \ \\ 
	 ~ Sugar & 70 ~ \ \ & 99 ~ \ \ & \textit{110} ~ \ \ \\ 
	 ~ Livestock & 93 ~ \ \ & 53 ~ \ \ & \textit{198} ~ \ \ \\ 
	 ~ Milk & 128 ~ \ \ & 57 ~ \ \ & \textit{214} ~ \ \ \\ 
	 ~ Meat & 81 ~ \ \ & 51 ~ \ \ & \textit{191} ~ \ \ \\ 
	 ~ Fish  & 48 ~ \ \ & 58 ~ \ \ & \textit{140} ~ \ \ \\ 
	\multicolumn{4}{l}{\textit{\normalsize{Net trade (min US\$)}}} \\ 
	 ~ Cereals & -45 ~ \ \ & -114 ~ \ \ & \textit{-140} ~ \ \ \\ 
	 ~ Fruit and vegetables & -3 ~ \ \ & -10 ~ \ \ & \textit{-16} ~ \ \ \\ 
	 ~ Meat & -2 ~ \ \ & -10 ~ \ \ & \textit{-32} ~ \ \ \\ 
	 ~ Dairy products &  ~ \ \ &  ~ \ \ &  ~ \ \ \\ 
	 ~ Fish & 9 ~ \ \ & 13 ~ \ \ & \textit{9} ~ \ \ \\ 
	\multicolumn{4}{l}{\textcolor{FAOblue}{\textbf{\large{Environment}}}} \\ 
	 ~ Forest area (\%) & 43 ~ \ \ & 40 ~ \ \ & \textit{37} ~ \ \ \\ 
	 ~ Renewable water res withdrawn (\% of total) &  ~ \ \ & \textit{22} ~ \ \ & 22 ~ \ \ \\ 
	 ~ Terrestrial protect areas (\% total land area)  & 5 ~ \ \ & 5 ~ \ \ & \textit{10} ~ \ \ \\ 
	 ~ Organic area (\% total agricultural area) &  ~ \ \ &  ~ \ \ & \textit{2} ~ \ \ \\ 
	 ~ Water withdrawal by agriculture (\% of total) &  ~ \ \ & \textit{22} ~ \ \ & 22 ~ \ \ \\ 
	 ~ Biofuel production (thousand kt of oil eq.) & 0 ~ \ \ & 0 ~ \ \ & \textit{0} ~ \ \ \\ 
	 ~ Wood pellet prod. (min tonnes) &  ~ \ \ &  ~ \ \ &  ~ \ \ \\ 
	 ~ GHG emissions from ag (Co2 eq, gigagrams) & 7 ~ \ \ & 7 ~ \ \ & \textit{9} ~ \ \ \\ 
       \toprule
      \end{tabular}
      \clearpage
\CountryData{ Slovakia }
      \rowcolors{1}{FAOblue!10}{white}
      \begin{tabular}{L{3.9cm} R{1cm} R{1cm} R{1cm}}
      \toprule
      \multicolumn{1}{c}{} & \multicolumn{1}{c}{ 1992 } & \multicolumn{1}{c}{ 2002 } & \multicolumn{1}{c}{ 2014 } \\
      \midrule
	\multicolumn{4}{l}{\textcolor{FAOblue}{\textbf{\large{The setting}}}} \\ 
	 ~ Population, total (mln) & \textit{5.4} ~ \ \ & 5.4 ~ \ \ & 5.5 ~ \ \ \\ 
	 ~ Population, rural (\% total population) & \textit{2.3} ~ \ \ & 2.4 ~ \ \ & 2.5 ~ \ \ \\ 
	 ~ Govt expenditure on ag (\% total outlays) &  ~ \ \ &  ~ \ \ &  ~ \ \ \\ 
	 ~ Area harvested (mln ha) &  ~ \ \ & 3 ~ \ \ & 3 ~ \ \ \\ 
	 ~ Cropping intensity ratio (\%) &  ~ \ \ & 1.4 ~ \ \ &  ~ \ \ \\ 
	 ~ Water resources (m\textsuperscript{3}/person/year) & \textit{9} ~ \ \ & \textit{9} ~ \ \ & \textit{9} ~ \ \ \\ 
	 ~ Area equipped for irrigation (1000 ha) &  ~ \ \ &  ~ \ \ & \textit{87} ~ \ \ \\ 
	 ~ Area irrigated (\%) &  ~ \ \ &  ~ \ \ & \textit{22.7} ~ \ \ \\ 
	 ~ Employment in agriculture (\%) & \textit{9.2} ~ \ \ & 6.2 ~ \ \ & \textit{3.2} ~ \ \ \\ 
	 ~ Employment in agriculture, female (\%) & \textit{6.4} ~ \ \ & 4.1 ~ \ \ & \textit{1.7} ~ \ \ \\ 
	 ~ Fertilizers, Nitrogen (nutrients per ha) &  ~ \ \ & 36.3 ~ \ \ & \textit{55.6} ~ \ \ \\ 
	 ~ Fertilizers, Phosphate (nutrients per ha) &  ~ \ \ & 7.9 ~ \ \ & \textit{10.8} ~ \ \ \\ 
	 ~ Fertilizers, Potash (nutrients per ha) &  ~ \ \ & 8.1 ~ \ \ & \textit{7.6} ~ \ \ \\ 
	 ~ Energy consump, power irrigation (mln kWh) &  ~ \ \ &  ~ \ \ &  ~ \ \ \\ 
	 ~ Agr value added per worker (constant US\$) & \textit{5.7} ~ \ \ & 8.7 ~ \ \ & \textit{18} ~ \ \ \\ 
	\multicolumn{4}{l}{\textcolor{FAOblue}{\textbf{\large{Hunger dimensions}}}} \\ 
	 ~ Dietary energy supply (kcal/pc/day) &  ~ \ \ &  ~ \ \ &  ~ \ \ \\ 
	 ~ Average dietary energy supply adequacy (\%) & 112 ~ \ \ & 109 ~ \ \ & 114 ~ \ \ \\ 
	 ~ Dietary en supp, cereals/roots/tubers (\%) & 37 ~ \ \ & 38 ~ \ \ & \textit{35} ~ \ \ \\ 
	 ~ Prevalence of undernourishment (\%) & <5.0 ~ \ \ & <5.0 ~ \ \ & <5.0 ~ \ \ \\ 
	 ~ GDP per capita (US\$, PPP) & 11\,364 ~ \ \ & 16\,528 ~ \ \ & \textit{25\,759} ~ \ \ \\ 
	 ~ Domestic food price volatility (index) &  ~ \ \ & 7 ~ \ \ & 9.2 ~ \ \ \\ 
	 ~ Cereal import dependency ratio (\%) & -2.5 ~ \ \ & -5.6 ~ \ \ & \textit{-27.5} ~ \ \ \\ 
	 ~ Underweight, children under-5 (\%) &  ~ \ \ &  ~ \ \ &  ~ \ \ \\ 
	 ~ Improved water source (\% pop) & 99.8 ~ \ \ & 99.8 ~ \ \ & \textit{100} ~ \ \ \\ 
	\multicolumn{4}{l}{\textcolor{FAOblue}{\textbf{\large{Food Supply}}}} \\ 
	 ~ Food production value, (2004-2006 mln I\$) & \textit{1\,669} ~ \ \ & 1\,572 ~ \ \ & \textit{1\,386} ~ \ \ \\ 
	 ~ Agriculture, value added (\% GDP) & \textit{6} ~ \ \ & 5 ~ \ \ & \textit{4} ~ \ \ \\ 
	 ~ Food exports (mln US\$)  & \textit{414} ~ \ \ & 416 ~ \ \ & \textit{3\,875} ~ \ \ \\ 
	 ~ Food imports (mln US\$)  & \textit{433} ~ \ \ & 588 ~ \ \ & \textit{3\,864} ~ \ \ \\ 
	\multicolumn{4}{l}{\textit{\normalsize{Production indices (2004-06=100)}}} \\ 
	 ~ Net food & \textit{105} ~ \ \ & 99 ~ \ \ & \textit{87} ~ \ \ \\ 
	 ~ Net crop & \textit{109} ~ \ \ & 96 ~ \ \ & \textit{97} ~ \ \ \\ 
	 ~ Cereal & \textit{104} ~ \ \ & 95 ~ \ \ & \textit{101} ~ \ \ \\ 
	 ~ Vegetable oils & \textit{52} ~ \ \ & 81 ~ \ \ & \textit{124} ~ \ \ \\ 
	 ~ Roots and tubers & \textit{136} ~ \ \ & 158 ~ \ \ & \textit{46} ~ \ \ \\ 
	 ~ Fruit and vegetables & \textit{147} ~ \ \ & 98 ~ \ \ & \textit{90} ~ \ \ \\ 
	 ~ Sugar & \textit{75} ~ \ \ & 86 ~ \ \ & \textit{73} ~ \ \ \\ 
	 ~ Livestock & \textit{120} ~ \ \ & 106 ~ \ \ & \textit{79} ~ \ \ \\ 
	 ~ Milk & \textit{109} ~ \ \ & 110 ~ \ \ & \textit{88} ~ \ \ \\ 
	 ~ Meat & \textit{131} ~ \ \ & 107 ~ \ \ & \textit{67} ~ \ \ \\ 
	 ~ Fish  & 0 ~ \ \ & 92 ~ \ \ & \textit{110} ~ \ \ \\ 
	\multicolumn{4}{l}{\textit{\normalsize{Net trade (min US\$)}}} \\ 
	 ~ Cereals & \textit{124} ~ \ \ & 8 ~ \ \ & \textit{194} ~ \ \ \\ 
	 ~ Fruit and vegetables & \textit{-83} ~ \ \ & -121 ~ \ \ & \textit{-488} ~ \ \ \\ 
	 ~ Meat & \textit{-9} ~ \ \ & -52 ~ \ \ & \textit{-381} ~ \ \ \\ 
	 ~ Dairy products & \textit{20} ~ \ \ & 31 ~ \ \ & \textit{-25} ~ \ \ \\ 
	 ~ Fish & \textit{-32} ~ \ \ & -35 ~ \ \ & \textit{-78} ~ \ \ \\ 
	\multicolumn{4}{l}{\textcolor{FAOblue}{\textbf{\large{Environment}}}} \\ 
	 ~ Forest area (\%) & \textit{40} ~ \ \ & 40 ~ \ \ & \textit{40} ~ \ \ \\ 
	 ~ Renewable water res withdrawn (\% of total) &  ~ \ \ &  ~ \ \ & 3 ~ \ \ \\ 
	 ~ Terrestrial protect areas (\% total land area)  & 19 ~ \ \ & 23 ~ \ \ & \textit{36} ~ \ \ \\ 
	 ~ Organic area (\% total agricultural area) &  ~ \ \ &  ~ \ \ & \textit{9} ~ \ \ \\ 
	 ~ Water withdrawal by agriculture (\% of total) &  ~ \ \ &  ~ \ \ & 3 ~ \ \ \\ 
	 ~ Biofuel production (thousand kt of oil eq.) &  ~ \ \ & 81 ~ \ \ & \textit{3\,056} ~ \ \ \\ 
	 ~ Wood pellet prod. (min tonnes) &  ~ \ \ &  ~ \ \ & \textit{92} ~ \ \ \\ 
	 ~ GHG emissions from ag (Co2 eq, gigagrams) & \textit{-6} ~ \ \ & -6 ~ \ \ & \textit{-4} ~ \ \ \\ 
       \toprule
      \end{tabular}
      \clearpage
\CountryData{ Slovenia }
      \rowcolors{1}{FAOblue!10}{white}
      \begin{tabular}{L{3.9cm} R{1cm} R{1cm} R{1cm}}
      \toprule
      \multicolumn{1}{c}{} & \multicolumn{1}{c}{ 1992 } & \multicolumn{1}{c}{ 2002 } & \multicolumn{1}{c}{ 2014 } \\
      \midrule
	\multicolumn{4}{l}{\textcolor{FAOblue}{\textbf{\large{The setting}}}} \\ 
	 ~ Population, total (mln) & 2 ~ \ \ & 2 ~ \ \ & 2.1 ~ \ \ \\ 
	 ~ Population, rural (\% total population) & 1 ~ \ \ & 1 ~ \ \ & 1 ~ \ \ \\ 
	 ~ Govt expenditure on ag (\% total outlays) &  ~ \ \ & 3.1 ~ \ \ & \textit{3.3} ~ \ \ \\ 
	 ~ Area harvested (mln ha) & 1 ~ \ \ & 1 ~ \ \ & 0 ~ \ \ \\ 
	 ~ Cropping intensity ratio (\%) & 1.2 ~ \ \ & 1.2 ~ \ \ &  ~ \ \ \\ 
	 ~ Water resources (m\textsuperscript{3}/person/year) & \textit{16} ~ \ \ & \textit{16} ~ \ \ & \textit{15} ~ \ \ \\ 
	 ~ Area equipped for irrigation (1000 ha) &  ~ \ \ &  ~ \ \ & \textit{6} ~ \ \ \\ 
	 ~ Area irrigated (\%) &  ~ \ \ &  ~ \ \ &  ~ \ \ \\ 
	 ~ Employment in agriculture (\%) & \textit{10.4} ~ \ \ & 9.7 ~ \ \ & \textit{8.3} ~ \ \ \\ 
	 ~ Employment in agriculture, female (\%) & \textit{10.8} ~ \ \ & 9.7 ~ \ \ & \textit{7.8} ~ \ \ \\ 
	 ~ Fertilizers, Nitrogen (nutrients per ha) &  ~ \ \ & 65.3 ~ \ \ & \textit{54.2} ~ \ \ \\ 
	 ~ Fertilizers, Phosphate (nutrients per ha) &  ~ \ \ & 29.6 ~ \ \ & \textit{18.3} ~ \ \ \\ 
	 ~ Fertilizers, Potash (nutrients per ha) &  ~ \ \ & 39.3 ~ \ \ & \textit{22.4} ~ \ \ \\ 
	 ~ Energy consump, power irrigation (mln kWh) &  ~ \ \ &  ~ \ \ & \textit{15} ~ \ \ \\ 
	 ~ Agr value added per worker (constant US\$) & \textit{24.3} ~ \ \ & 58.7 ~ \ \ & \textit{153.3} ~ \ \ \\ 
	\multicolumn{4}{l}{\textcolor{FAOblue}{\textbf{\large{Hunger dimensions}}}} \\ 
	 ~ Dietary energy supply (kcal/pc/day) &  ~ \ \ &  ~ \ \ &  ~ \ \ \\ 
	 ~ Average dietary energy supply adequacy (\%) & 112 ~ \ \ & 122 ~ \ \ & 125 ~ \ \ \\ 
	 ~ Dietary en supp, cereals/roots/tubers (\%) & 39 ~ \ \ & 38 ~ \ \ & \textit{38} ~ \ \ \\ 
	 ~ Prevalence of undernourishment (\%) & <5.0 ~ \ \ & <5.0 ~ \ \ & <5.0 ~ \ \ \\ 
	 ~ GDP per capita (US\$, PPP) & \textit{18\,240} ~ \ \ & 23\,972 ~ \ \ & \textit{27\,368} ~ \ \ \\ 
	 ~ Domestic food price volatility (index) &  ~ \ \ & 6.7 ~ \ \ & 9.4 ~ \ \ \\ 
	 ~ Cereal import dependency ratio (\%) & 58.8 ~ \ \ & 48.7 ~ \ \ & \textit{36.9} ~ \ \ \\ 
	 ~ Underweight, children under-5 (\%) &  ~ \ \ &  ~ \ \ &  ~ \ \ \\ 
	 ~ Improved water source (\% pop) & 99.6 ~ \ \ & 99.6 ~ \ \ & \textit{99.6} ~ \ \ \\ 
	\multicolumn{4}{l}{\textcolor{FAOblue}{\textbf{\large{Food Supply}}}} \\ 
	 ~ Food production value, (2004-2006 mln I\$) & 573 ~ \ \ & 786 ~ \ \ & \textit{608} ~ \ \ \\ 
	 ~ Agriculture, value added (\% GDP) & \textit{4} ~ \ \ & 3 ~ \ \ & \textit{2} ~ \ \ \\ 
	 ~ Food exports (mln US\$)  & 304 ~ \ \ & 213 ~ \ \ & \textit{1\,207} ~ \ \ \\ 
	 ~ Food imports (mln US\$)  & 411 ~ \ \ & 495 ~ \ \ & \textit{1\,987} ~ \ \ \\ 
	\multicolumn{4}{l}{\textit{\normalsize{Production indices (2004-06=100)}}} \\ 
	 ~ Net food & 78 ~ \ \ & 107 ~ \ \ & \textit{83} ~ \ \ \\ 
	 ~ Net crop & 75 ~ \ \ & 105 ~ \ \ & \textit{74} ~ \ \ \\ 
	 ~ Cereal & 67 ~ \ \ & 112 ~ \ \ & \textit{83} ~ \ \ \\ 
	 ~ Vegetable oils & 64 ~ \ \ & 101 ~ \ \ & \textit{196} ~ \ \ \\ 
	 ~ Roots and tubers & 102 ~ \ \ & 118 ~ \ \ & \textit{58} ~ \ \ \\ 
	 ~ Fruit and vegetables & 74 ~ \ \ & 103 ~ \ \ & \textit{74} ~ \ \ \\ 
	 ~ Sugar & 39 ~ \ \ & 95 ~ \ \ & \textit{107} ~ \ \ \\ 
	 ~ Livestock & 77 ~ \ \ & 107 ~ \ \ & \textit{85} ~ \ \ \\ 
	 ~ Milk & 88 ~ \ \ & 111 ~ \ \ & \textit{91} ~ \ \ \\ 
	 ~ Meat & 66 ~ \ \ & 101 ~ \ \ & \textit{79} ~ \ \ \\ 
	 ~ Fish  & 187 ~ \ \ & 116 ~ \ \ & \textit{64} ~ \ \ \\ 
	\multicolumn{4}{l}{\textit{\normalsize{Net trade (min US\$)}}} \\ 
	 ~ Cereals & -79 ~ \ \ & -83 ~ \ \ & \textit{-167} ~ \ \ \\ 
	 ~ Fruit and vegetables & -21 ~ \ \ & -127 ~ \ \ & \textit{-246} ~ \ \ \\ 
	 ~ Meat & 47 ~ \ \ & 7 ~ \ \ & \textit{-122} ~ \ \ \\ 
	 ~ Dairy products & 26 ~ \ \ & 31 ~ \ \ & \textit{1} ~ \ \ \\ 
	 ~ Fish & -11 ~ \ \ & -26 ~ \ \ & \textit{-61} ~ \ \ \\ 
	\multicolumn{4}{l}{\textcolor{FAOblue}{\textbf{\large{Environment}}}} \\ 
	 ~ Forest area (\%) & 59 ~ \ \ & 61 ~ \ \ & \textit{62} ~ \ \ \\ 
	 ~ Renewable water res withdrawn (\% of total) &  ~ \ \ &  ~ \ \ & 0 ~ \ \ \\ 
	 ~ Terrestrial protect areas (\% total land area)  & 8 ~ \ \ & 10 ~ \ \ & \textit{55} ~ \ \ \\ 
	 ~ Organic area (\% total agricultural area) &  ~ \ \ & \textit{5} ~ \ \ & \textit{7} ~ \ \ \\ 
	 ~ Water withdrawal by agriculture (\% of total) &  ~ \ \ &  ~ \ \ & 0 ~ \ \ \\ 
	 ~ Biofuel production (thousand kt of oil eq.) &  ~ \ \ & 0 ~ \ \ & \textit{514} ~ \ \ \\ 
	 ~ Wood pellet prod. (min tonnes) &  ~ \ \ &  ~ \ \ & \textit{90} ~ \ \ \\ 
	 ~ GHG emissions from ag (Co2 eq, gigagrams) & 2 ~ \ \ & -12 ~ \ \ & \textit{-13} ~ \ \ \\ 
       \toprule
      \end{tabular}
      \clearpage
\CountryData{ Solomon Islands }
      \rowcolors{1}{FAOblue!10}{white}
      \begin{tabular}{L{3.9cm} R{1cm} R{1cm} R{1cm}}
      \toprule
      \multicolumn{1}{c}{} & \multicolumn{1}{c}{ 1992 } & \multicolumn{1}{c}{ 2002 } & \multicolumn{1}{c}{ 2014 } \\
      \midrule
	\multicolumn{4}{l}{\textcolor{FAOblue}{\textbf{\large{The setting}}}} \\ 
	 ~ Population, total (mln) & 0.3 ~ \ \ & 0.4 ~ \ \ & 0.6 ~ \ \ \\ 
	 ~ Population, rural (\% total population) & 0.3 ~ \ \ & 0.4 ~ \ \ & 0.4 ~ \ \ \\ 
	 ~ Govt expenditure on ag (\% total outlays) &  ~ \ \ &  ~ \ \ &  ~ \ \ \\ 
	 ~ Area harvested (mln ha) & 0 ~ \ \ & 0 ~ \ \ & 0 ~ \ \ \\ 
	 ~ Cropping intensity ratio (\%) & 3 ~ \ \ & 2.5 ~ \ \ &  ~ \ \ \\ 
	 ~ Water resources (m\textsuperscript{3}/person/year) & \textit{132} ~ \ \ & \textit{100} ~ \ \ & \textit{80} ~ \ \ \\ 
	 ~ Area equipped for irrigation (1000 ha) &  ~ \ \ &  ~ \ \ &  ~ \ \ \\ 
	 ~ Area irrigated (\%) &  ~ \ \ &  ~ \ \ &  ~ \ \ \\ 
	 ~ Employment in agriculture (\%) &  ~ \ \ &  ~ \ \ &  ~ \ \ \\ 
	 ~ Employment in agriculture, female (\%) &  ~ \ \ &  ~ \ \ &  ~ \ \ \\ 
	 ~ Fertilizers, Nitrogen (nutrients per ha) &  ~ \ \ &  ~ \ \ &  ~ \ \ \\ 
	 ~ Fertilizers, Phosphate (nutrients per ha) &  ~ \ \ &  ~ \ \ &  ~ \ \ \\ 
	 ~ Fertilizers, Potash (nutrients per ha) &  ~ \ \ &  ~ \ \ &  ~ \ \ \\ 
	 ~ Energy consump, power irrigation (mln kWh) &  ~ \ \ &  ~ \ \ &  ~ \ \ \\ 
	 ~ Agr value added per worker (constant US\$) & 1 ~ \ \ & 0.8 ~ \ \ & \textit{1.1} ~ \ \ \\ 
	\multicolumn{4}{l}{\textcolor{FAOblue}{\textbf{\large{Hunger dimensions}}}} \\ 
	 ~ Dietary energy supply (kcal/pc/day) & 2\,171 ~ \ \ & 2\,373 ~ \ \ & 2\,471 ~ \ \ \\ 
	 ~ Average dietary energy supply adequacy (\%) & 103 ~ \ \ & 111 ~ \ \ & 115 ~ \ \ \\ 
	 ~ Dietary en supp, cereals/roots/tubers (\%) & 66 ~ \ \ & 69 ~ \ \ & \textit{67} ~ \ \ \\ 
	 ~ Prevalence of undernourishment (\%) & 23.7 ~ \ \ & 14.7 ~ \ \ & 10.9 ~ \ \ \\ 
	 ~ GDP per capita (US\$, PPP) & 2\,005 ~ \ \ & 1\,454 ~ \ \ & \textit{2\,003} ~ \ \ \\ 
	 ~ Domestic food price volatility (index) &  ~ \ \ &  ~ \ \ &  ~ \ \ \\ 
	 ~ Cereal import dependency ratio (\%) & 95.4 ~ \ \ & 96 ~ \ \ & \textit{97} ~ \ \ \\ 
	 ~ Underweight, children under-5 (\%) &  ~ \ \ &  ~ \ \ & \textit{11.5} ~ \ \ \\ 
	 ~ Improved water source (\% pop) &  ~ \ \ & 79.8 ~ \ \ & \textit{80.5} ~ \ \ \\ 
	\multicolumn{4}{l}{\textcolor{FAOblue}{\textbf{\large{Food Supply}}}} \\ 
	 ~ Food production value, (2004-2006 mln I\$) & 73 ~ \ \ & 81 ~ \ \ & \textit{121} ~ \ \ \\ 
	 ~ Agriculture, value added (\% GDP) & 49 ~ \ \ & 31 ~ \ \ & \textit{36} ~ \ \ \\ 
	 ~ Food exports (mln US\$)  & 26 ~ \ \ & 21 ~ \ \ & \textit{83} ~ \ \ \\ 
	 ~ Food imports (mln US\$)  & 13 ~ \ \ & 17 ~ \ \ & \textit{88} ~ \ \ \\ 
	\multicolumn{4}{l}{\textit{\normalsize{Production indices (2004-06=100)}}} \\ 
	 ~ Net food & 71 ~ \ \ & 79 ~ \ \ & \textit{118} ~ \ \ \\ 
	 ~ Net crop & 70 ~ \ \ & 78 ~ \ \ & \textit{119} ~ \ \ \\ 
	 ~ Cereal & 39 ~ \ \ & 79 ~ \ \ & \textit{146} ~ \ \ \\ 
	 ~ Vegetable oils & 82 ~ \ \ & 88 ~ \ \ & \textit{112} ~ \ \ \\ 
	 ~ Roots and tubers & 70 ~ \ \ & 92 ~ \ \ & \textit{120} ~ \ \ \\ 
	 ~ Fruit and vegetables & 68 ~ \ \ & 86 ~ \ \ & \textit{115} ~ \ \ \\ 
	 ~ Sugar &  ~ \ \ &  ~ \ \ &  ~ \ \ \\ 
	 ~ Livestock & 84 ~ \ \ & 96 ~ \ \ & \textit{106} ~ \ \ \\ 
	 ~ Milk & 96 ~ \ \ & 111 ~ \ \ & \textit{103} ~ \ \ \\ 
	 ~ Meat & 85 ~ \ \ & 95 ~ \ \ & \textit{104} ~ \ \ \\ 
	 ~ Fish  & 135 ~ \ \ & 75 ~ \ \ & \textit{105} ~ \ \ \\ 
	\multicolumn{4}{l}{\textit{\normalsize{Net trade (min US\$)}}} \\ 
	 ~ Cereals & -8 ~ \ \ & -10 ~ \ \ & \textit{-51} ~ \ \ \\ 
	 ~ Fruit and vegetables & 0 ~ \ \ & 0 ~ \ \ & \textit{-3} ~ \ \ \\ 
	 ~ Meat &  ~ \ \ &  ~ \ \ & \textit{-11} ~ \ \ \\ 
	 ~ Dairy products &  ~ \ \ &  ~ \ \ & \textit{-3} ~ \ \ \\ 
	 ~ Fish &  ~ \ \ &  ~ \ \ &  ~ \ \ \\ 
	\multicolumn{4}{l}{\textcolor{FAOblue}{\textbf{\large{Environment}}}} \\ 
	 ~ Forest area (\%) & 83 ~ \ \ & 81 ~ \ \ & \textit{79} ~ \ \ \\ 
	 ~ Renewable water res withdrawn (\% of total) &  ~ \ \ &  ~ \ \ &  ~ \ \ \\ 
	 ~ Terrestrial protect areas (\% total land area)  & 0 ~ \ \ & 0 ~ \ \ & \textit{2} ~ \ \ \\ 
	 ~ Organic area (\% total agricultural area) &  ~ \ \ &  ~ \ \ & \textit{1} ~ \ \ \\ 
	 ~ Water withdrawal by agriculture (\% of total) &  ~ \ \ &  ~ \ \ &  ~ \ \ \\ 
	 ~ Biofuel production (thousand kt of oil eq.) & 2 ~ \ \ & 2 ~ \ \ & \textit{2} ~ \ \ \\ 
	 ~ Wood pellet prod. (min tonnes) &  ~ \ \ &  ~ \ \ &  ~ \ \ \\ 
	 ~ GHG emissions from ag (Co2 eq, gigagrams) & 2 ~ \ \ & 2 ~ \ \ & \textit{2} ~ \ \ \\ 
       \toprule
      \end{tabular}
      \clearpage
\CountryData{ Somalia }
      \rowcolors{1}{FAOblue!10}{white}
      \begin{tabular}{L{3.9cm} R{1cm} R{1cm} R{1cm}}
      \toprule
      \multicolumn{1}{c}{} & \multicolumn{1}{c}{ 1992 } & \multicolumn{1}{c}{ 2002 } & \multicolumn{1}{c}{ 2014 } \\
      \midrule
	\multicolumn{4}{l}{\textcolor{FAOblue}{\textbf{\large{The setting}}}} \\ 
	 ~ Population, total (mln) & 6.3 ~ \ \ & 7.8 ~ \ \ & 10.8 ~ \ \ \\ 
	 ~ Population, rural (\% total population) & 4.4 ~ \ \ & 5.2 ~ \ \ & 6.6 ~ \ \ \\ 
	 ~ Govt expenditure on ag (\% total outlays) &  ~ \ \ &  ~ \ \ &  ~ \ \ \\ 
	 ~ Area harvested (mln ha) & 0 ~ \ \ & 1 ~ \ \ & 0 ~ \ \ \\ 
	 ~ Cropping intensity ratio (\%) & 0 ~ \ \ & 0 ~ \ \ &  ~ \ \ \\ 
	 ~ Water resources (m\textsuperscript{3}/person/year) & \textit{2} ~ \ \ & \textit{2} ~ \ \ & \textit{1} ~ \ \ \\ 
	 ~ Area equipped for irrigation (1000 ha) &  ~ \ \ &  ~ \ \ & \textit{200} ~ \ \ \\ 
	 ~ Area irrigated (\%) &  ~ \ \ & \textit{32.5} ~ \ \ &  ~ \ \ \\ 
	 ~ Employment in agriculture (\%) &  ~ \ \ &  ~ \ \ &  ~ \ \ \\ 
	 ~ Employment in agriculture, female (\%) &  ~ \ \ &  ~ \ \ &  ~ \ \ \\ 
	 ~ Fertilizers, Nitrogen (nutrients per ha) &  ~ \ \ &  ~ \ \ &  ~ \ \ \\ 
	 ~ Fertilizers, Phosphate (nutrients per ha) &  ~ \ \ &  ~ \ \ &  ~ \ \ \\ 
	 ~ Fertilizers, Potash (nutrients per ha) &  ~ \ \ &  ~ \ \ &  ~ \ \ \\ 
	 ~ Energy consump, power irrigation (mln kWh) &  ~ \ \ &  ~ \ \ &  ~ \ \ \\ 
	 ~ Agr value added per worker (constant US\$) &  ~ \ \ &  ~ \ \ &  ~ \ \ \\ 
	\multicolumn{4}{l}{\textcolor{FAOblue}{\textbf{\large{Hunger dimensions}}}} \\ 
	 ~ Dietary energy supply (kcal/pc/day) &  ~ \ \ &  ~ \ \ &  ~ \ \ \\ 
	 ~ Average dietary energy supply adequacy (\%) &  ~ \ \ &  ~ \ \ &  ~ \ \ \\ 
	 ~ Dietary en supp, cereals/roots/tubers (\%) &  ~ \ \ &  ~ \ \ &  ~ \ \ \\ 
	 ~ Prevalence of undernourishment (\%) &  ~ \ \ &  ~ \ \ &  ~ \ \ \\ 
	 ~ GDP per capita (US\$, PPP) &  ~ \ \ &  ~ \ \ &  ~ \ \ \\ 
	 ~ Domestic food price volatility (index) &  ~ \ \ &  ~ \ \ &  ~ \ \ \\ 
	 ~ Cereal import dependency ratio (\%) & 56.7 ~ \ \ & 31.2 ~ \ \ & \textit{69.7} ~ \ \ \\ 
	 ~ Underweight, children under-5 (\%) &  ~ \ \ & \textit{22.8} ~ \ \ & \textit{32.8} ~ \ \ \\ 
	 ~ Improved water source (\% pop) & \textit{21} ~ \ \ & 25.3 ~ \ \ & \textit{31.7} ~ \ \ \\ 
	\multicolumn{4}{l}{\textcolor{FAOblue}{\textbf{\large{Food Supply}}}} \\ 
	 ~ Food production value, (2004-2006 mln I\$) & 1\,066 ~ \ \ & 1\,498 ~ \ \ & \textit{1\,881} ~ \ \ \\ 
	 ~ Agriculture, value added (\% GDP) & \textit{65} ~ \ \ &  ~ \ \ &  ~ \ \ \\ 
	 ~ Food exports (mln US\$)  & 60 ~ \ \ & 98 ~ \ \ & \textit{314} ~ \ \ \\ 
	 ~ Food imports (mln US\$)  & 137 ~ \ \ & 129 ~ \ \ & \textit{853} ~ \ \ \\ 
	\multicolumn{4}{l}{\textit{\normalsize{Production indices (2004-06=100)}}} \\ 
	 ~ Net food & 66 ~ \ \ & 93 ~ \ \ & \textit{117} ~ \ \ \\ 
	 ~ Net crop & 74 ~ \ \ & 99 ~ \ \ & \textit{124} ~ \ \ \\ 
	 ~ Cereal & 64 ~ \ \ & 130 ~ \ \ & \textit{114} ~ \ \ \\ 
	 ~ Vegetable oils & 31 ~ \ \ & 61 ~ \ \ & \textit{186} ~ \ \ \\ 
	 ~ Roots and tubers & 46 ~ \ \ & 92 ~ \ \ & \textit{103} ~ \ \ \\ 
	 ~ Fruit and vegetables & 97 ~ \ \ & 99 ~ \ \ & \textit{109} ~ \ \ \\ 
	 ~ Sugar & 153 ~ \ \ & 115 ~ \ \ & \textit{120} ~ \ \ \\ 
	 ~ Livestock & 65 ~ \ \ & 93 ~ \ \ & \textit{116} ~ \ \ \\ 
	 ~ Milk & 72 ~ \ \ & 96 ~ \ \ & \textit{110} ~ \ \ \\ 
	 ~ Meat & 57 ~ \ \ & 88 ~ \ \ & \textit{124} ~ \ \ \\ 
	 ~ Fish  &  ~ \ \ &  ~ \ \ &  ~ \ \ \\ 
	\multicolumn{4}{l}{\textit{\normalsize{Net trade (min US\$)}}} \\ 
	 ~ Cereals & -97 ~ \ \ & -40 ~ \ \ & \textit{-230} ~ \ \ \\ 
	 ~ Fruit and vegetables & -11 ~ \ \ & -3 ~ \ \ & \textit{-239} ~ \ \ \\ 
	 ~ Meat & 0 ~ \ \ & 0 ~ \ \ & \textit{0} ~ \ \ \\ 
	 ~ Dairy products & -4 ~ \ \ & 0 ~ \ \ & \textit{-37} ~ \ \ \\ 
	 ~ Fish & 14 ~ \ \ & 3 ~ \ \ & \textit{-5} ~ \ \ \\ 
	\multicolumn{4}{l}{\textcolor{FAOblue}{\textbf{\large{Environment}}}} \\ 
	 ~ Forest area (\%) & 13 ~ \ \ & 12 ~ \ \ & \textit{11} ~ \ \ \\ 
	 ~ Renewable water res withdrawn (\% of total) &  ~ \ \ & \textit{100} ~ \ \ & 100 ~ \ \ \\ 
	 ~ Terrestrial protect areas (\% total land area)  & 1 ~ \ \ & 1 ~ \ \ & \textit{1} ~ \ \ \\ 
	 ~ Organic area (\% total agricultural area) &  ~ \ \ &  ~ \ \ & \textit{0} ~ \ \ \\ 
	 ~ Water withdrawal by agriculture (\% of total) &  ~ \ \ & \textit{100} ~ \ \ & 100 ~ \ \ \\ 
	 ~ Biofuel production (thousand kt of oil eq.) & 1 ~ \ \ & 0 ~ \ \ & \textit{1} ~ \ \ \\ 
	 ~ Wood pellet prod. (min tonnes) &  ~ \ \ &  ~ \ \ &  ~ \ \ \\ 
	 ~ GHG emissions from ag (Co2 eq, gigagrams) & 32 ~ \ \ & 38 ~ \ \ & \textit{36} ~ \ \ \\ 
       \toprule
      \end{tabular}
      \clearpage
\CountryData{ South Africa }
      \rowcolors{1}{FAOblue!10}{white}
      \begin{tabular}{L{3.9cm} R{1cm} R{1cm} R{1cm}}
      \toprule
      \multicolumn{1}{c}{} & \multicolumn{1}{c}{ 1992 } & \multicolumn{1}{c}{ 2002 } & \multicolumn{1}{c}{ 2014 } \\
      \midrule
	\multicolumn{4}{l}{\textcolor{FAOblue}{\textbf{\large{The setting}}}} \\ 
	 ~ Population, total (mln) & 38.6 ~ \ \ & 46.2 ~ \ \ & 53.1 ~ \ \ \\ 
	 ~ Population, rural (\% total population) & 18.1 ~ \ \ & 19.5 ~ \ \ & 19.5 ~ \ \ \\ 
	 ~ Govt expenditure on ag (\% total outlays) &  ~ \ \ &  ~ \ \ &  ~ \ \ \\ 
	 ~ Area harvested (mln ha) & 13 ~ \ \ & 23 ~ \ \ & 18 ~ \ \ \\ 
	 ~ Cropping intensity ratio (\%) & 0.1 ~ \ \ & 0.2 ~ \ \ &  ~ \ \ \\ 
	 ~ Water resources (m\textsuperscript{3}/person/year) & \textit{1} ~ \ \ & \textit{1} ~ \ \ & \textit{1} ~ \ \ \\ 
	 ~ Area equipped for irrigation (1000 ha) &  ~ \ \ &  ~ \ \ & \textit{1\,670} ~ \ \ \\ 
	 ~ Area irrigated (\%) &  ~ \ \ &  ~ \ \ & \textit{95.9} ~ \ \ \\ 
	 ~ Employment in agriculture (\%) &  ~ \ \ & 12.6 ~ \ \ & \textit{4.6} ~ \ \ \\ 
	 ~ Employment in agriculture, female (\%) &  ~ \ \ & 10.2 ~ \ \ & \textit{3.9} ~ \ \ \\ 
	 ~ Fertilizers, Nitrogen (nutrients per ha) &  ~ \ \ & 4.9 ~ \ \ & \textit{4.5} ~ \ \ \\ 
	 ~ Fertilizers, Phosphate (nutrients per ha) &  ~ \ \ & 2.3 ~ \ \ & \textit{2} ~ \ \ \\ 
	 ~ Fertilizers, Potash (nutrients per ha) &  ~ \ \ & 1.4 ~ \ \ & \textit{1.3} ~ \ \ \\ 
	 ~ Energy consump, power irrigation (mln kWh) & 1\,939 ~ \ \ & 2\,406 ~ \ \ & \textit{2\,406} ~ \ \ \\ 
	 ~ Agr value added per worker (constant US\$) & 2.5 ~ \ \ & 4.1 ~ \ \ & \textit{6.7} ~ \ \ \\ 
	\multicolumn{4}{l}{\textcolor{FAOblue}{\textbf{\large{Hunger dimensions}}}} \\ 
	 ~ Dietary energy supply (kcal/pc/day) & 2\,817 ~ \ \ & 2\,916 ~ \ \ & 3\,136 ~ \ \ \\ 
	 ~ Average dietary energy supply adequacy (\%) & 121 ~ \ \ & 122 ~ \ \ & 131 ~ \ \ \\ 
	 ~ Dietary en supp, cereals/roots/tubers (\%) & 56 ~ \ \ & 56 ~ \ \ & \textit{53} ~ \ \ \\ 
	 ~ Prevalence of undernourishment (\%) & <5.0 ~ \ \ & <5.0 ~ \ \ & <5.0 ~ \ \ \\ 
	 ~ GDP per capita (US\$, PPP) & 9\,631 ~ \ \ & 10\,213 ~ \ \ & \textit{12\,454} ~ \ \ \\ 
	 ~ Domestic food price volatility (index) &  ~ \ \ & 6.5 ~ \ \ & 6.2 ~ \ \ \\ 
	 ~ Cereal import dependency ratio (\%) & 20.2 ~ \ \ & 7.7 ~ \ \ & \textit{2.8} ~ \ \ \\ 
	 ~ Underweight, children under-5 (\%) & \textit{8} ~ \ \ & \textit{11.6} ~ \ \ & \textit{8.7} ~ \ \ \\ 
	 ~ Improved water source (\% pop) & 81.7 ~ \ \ & 88.3 ~ \ \ & \textit{95.1} ~ \ \ \\ 
	\multicolumn{4}{l}{\textcolor{FAOblue}{\textbf{\large{Food Supply}}}} \\ 
	 ~ Food production value, (2004-2006 mln I\$) & 7\,258 ~ \ \ & 10\,187 ~ \ \ & \textit{12\,753} ~ \ \ \\ 
	 ~ Agriculture, value added (\% GDP) & 4 ~ \ \ & 4 ~ \ \ & \textit{2} ~ \ \ \\ 
	 ~ Food exports (mln US\$)  & 1\,269 ~ \ \ & 1\,611 ~ \ \ & \textit{4\,764} ~ \ \ \\ 
	 ~ Food imports (mln US\$)  & 1\,255 ~ \ \ & 921 ~ \ \ & \textit{4\,679} ~ \ \ \\ 
	\multicolumn{4}{l}{\textit{\normalsize{Production indices (2004-06=100)}}} \\ 
	 ~ Net food & 70 ~ \ \ & 98 ~ \ \ & \textit{122} ~ \ \ \\ 
	 ~ Net crop & 63 ~ \ \ & 103 ~ \ \ & \textit{115} ~ \ \ \\ 
	 ~ Cereal & 42 ~ \ \ & 110 ~ \ \ & \textit{125} ~ \ \ \\ 
	 ~ Vegetable oils & 43 ~ \ \ & 145 ~ \ \ & \textit{120} ~ \ \ \\ 
	 ~ Roots and tubers & 66 ~ \ \ & 91 ~ \ \ & \textit{124} ~ \ \ \\ 
	 ~ Fruit and vegetables & 75 ~ \ \ & 93 ~ \ \ & \textit{113} ~ \ \ \\ 
	 ~ Sugar & 64 ~ \ \ & 114 ~ \ \ & \textit{89} ~ \ \ \\ 
	 ~ Livestock & 82 ~ \ \ & 90 ~ \ \ & \textit{131} ~ \ \ \\ 
	 ~ Milk & 83 ~ \ \ & 94 ~ \ \ & \textit{119} ~ \ \ \\ 
	 ~ Meat & 82 ~ \ \ & 88 ~ \ \ & \textit{134} ~ \ \ \\ 
	 ~ Fish  & 90 ~ \ \ & 99 ~ \ \ & \textit{54} ~ \ \ \\ 
	\multicolumn{4}{l}{\textit{\normalsize{Net trade (min US\$)}}} \\ 
	 ~ Cereals & -523 ~ \ \ & -166 ~ \ \ & \textit{-855} ~ \ \ \\ 
	 ~ Fruit and vegetables & 756 ~ \ \ & 776 ~ \ \ & \textit{2\,404} ~ \ \ \\ 
	 ~ Meat & -116 ~ \ \ & 3 ~ \ \ & \textit{-533} ~ \ \ \\ 
	 ~ Dairy products & 12 ~ \ \ & 12 ~ \ \ & \textit{-77} ~ \ \ \\ 
	 ~ Fish & 66 ~ \ \ & 272 ~ \ \ & \textit{255} ~ \ \ \\ 
	\multicolumn{4}{l}{\textcolor{FAOblue}{\textbf{\large{Environment}}}} \\ 
	 ~ Forest area (\%) & 8 ~ \ \ & 8 ~ \ \ & \textit{8} ~ \ \ \\ 
	 ~ Renewable water res withdrawn (\% of total) &  ~ \ \ & \textit{63} ~ \ \ & 63 ~ \ \ \\ 
	 ~ Terrestrial protect areas (\% total land area)  & 7 ~ \ \ & 7 ~ \ \ & \textit{6} ~ \ \ \\ 
	 ~ Organic area (\% total agricultural area) &  ~ \ \ & \textit{0} ~ \ \ & \textit{0} ~ \ \ \\ 
	 ~ Water withdrawal by agriculture (\% of total) &  ~ \ \ & \textit{63} ~ \ \ & 63 ~ \ \ \\ 
	 ~ Biofuel production (thousand kt of oil eq.) & 43 ~ \ \ & 143 ~ \ \ & \textit{135} ~ \ \ \\ 
	 ~ Wood pellet prod. (min tonnes) &  ~ \ \ &  ~ \ \ & \textit{75} ~ \ \ \\ 
	 ~ GHG emissions from ag (Co2 eq, gigagrams) & 32 ~ \ \ & 33 ~ \ \ & \textit{32} ~ \ \ \\ 
       \toprule
      \end{tabular}
      \clearpage
\CountryData{ South Sudan }
      \rowcolors{1}{FAOblue!10}{white}
      \begin{tabular}{L{3.9cm} R{1cm} R{1cm} R{1cm}}
      \toprule
      \multicolumn{1}{c}{} & \multicolumn{1}{c}{ 1992 } & \multicolumn{1}{c}{ 2002 } & \multicolumn{1}{c}{ 2014 } \\
      \midrule
	\multicolumn{4}{l}{\textcolor{FAOblue}{\textbf{\large{The setting}}}} \\ 
	 ~ Population, total (mln) &  ~ \ \ &  ~ \ \ & 11.7 ~ \ \ \\ 
	 ~ Population, rural (\% total population) &  ~ \ \ &  ~ \ \ & 8.2 ~ \ \ \\ 
	 ~ Govt expenditure on ag (\% total outlays) &  ~ \ \ &  ~ \ \ &  ~ \ \ \\ 
	 ~ Area harvested (mln ha) &  ~ \ \ &  ~ \ \ &  ~ \ \ \\ 
	 ~ Cropping intensity ratio (\%) &  ~ \ \ &  ~ \ \ &  ~ \ \ \\ 
	 ~ Water resources (m\textsuperscript{3}/person/year) &  ~ \ \ &  ~ \ \ & \textit{4} ~ \ \ \\ 
	 ~ Area equipped for irrigation (1000 ha) &  ~ \ \ &  ~ \ \ & \textit{100} ~ \ \ \\ 
	 ~ Area irrigated (\%) &  ~ \ \ &  ~ \ \ &  ~ \ \ \\ 
	 ~ Employment in agriculture (\%) &  ~ \ \ &  ~ \ \ &  ~ \ \ \\ 
	 ~ Employment in agriculture, female (\%) &  ~ \ \ &  ~ \ \ &  ~ \ \ \\ 
	 ~ Fertilizers, Nitrogen (nutrients per ha) &  ~ \ \ &  ~ \ \ & \textit{0} ~ \ \ \\ 
	 ~ Fertilizers, Phosphate (nutrients per ha) &  ~ \ \ &  ~ \ \ & \textit{0} ~ \ \ \\ 
	 ~ Fertilizers, Potash (nutrients per ha) &  ~ \ \ &  ~ \ \ & \textit{0} ~ \ \ \\ 
	 ~ Energy consump, power irrigation (mln kWh) &  ~ \ \ &  ~ \ \ &  ~ \ \ \\ 
	 ~ Agr value added per worker (constant US\$) &  ~ \ \ &  ~ \ \ &  ~ \ \ \\ 
	\multicolumn{4}{l}{\textcolor{FAOblue}{\textbf{\large{Hunger dimensions}}}} \\ 
	 ~ Dietary energy supply (kcal/pc/day) &  ~ \ \ &  ~ \ \ &  ~ \ \ \\ 
	 ~ Average dietary energy supply adequacy (\%) &  ~ \ \ &  ~ \ \ &  ~ \ \ \\ 
	 ~ Dietary en supp, cereals/roots/tubers (\%) &  ~ \ \ &  ~ \ \ &  ~ \ \ \\ 
	 ~ Prevalence of undernourishment (\%) &  ~ \ \ &  ~ \ \ &  ~ \ \ \\ 
	 ~ GDP per capita (US\$, PPP) &  ~ \ \ &  ~ \ \ & \textit{1\,965} ~ \ \ \\ 
	 ~ Domestic food price volatility (index) &  ~ \ \ &  ~ \ \ &  ~ \ \ \\ 
	 ~ Cereal import dependency ratio (\%) &  ~ \ \ &  ~ \ \ &  ~ \ \ \\ 
	 ~ Underweight, children under-5 (\%) &  ~ \ \ &  ~ \ \ & \textit{27.6} ~ \ \ \\ 
	 ~ Improved water source (\% pop) &  ~ \ \ &  ~ \ \ & \textit{56.5} ~ \ \ \\ 
	\multicolumn{4}{l}{\textcolor{FAOblue}{\textbf{\large{Food Supply}}}} \\ 
	 ~ Food production value, (2004-2006 mln I\$) &  ~ \ \ &  ~ \ \ &  ~ \ \ \\ 
	 ~ Agriculture, value added (\% GDP) &  ~ \ \ &  ~ \ \ &  ~ \ \ \\ 
	 ~ Food exports (mln US\$)  &  ~ \ \ &  ~ \ \ & \textit{0} ~ \ \ \\ 
	 ~ Food imports (mln US\$)  &  ~ \ \ &  ~ \ \ & \textit{65} ~ \ \ \\ 
	\multicolumn{4}{l}{\textit{\normalsize{Production indices (2004-06=100)}}} \\ 
	 ~ Net food &  ~ \ \ &  ~ \ \ &  ~ \ \ \\ 
	 ~ Net crop &  ~ \ \ &  ~ \ \ &  ~ \ \ \\ 
	 ~ Cereal &  ~ \ \ &  ~ \ \ &  ~ \ \ \\ 
	 ~ Vegetable oils &  ~ \ \ &  ~ \ \ &  ~ \ \ \\ 
	 ~ Roots and tubers &  ~ \ \ &  ~ \ \ &  ~ \ \ \\ 
	 ~ Fruit and vegetables &  ~ \ \ &  ~ \ \ &  ~ \ \ \\ 
	 ~ Sugar &  ~ \ \ &  ~ \ \ &  ~ \ \ \\ 
	 ~ Livestock &  ~ \ \ &  ~ \ \ &  ~ \ \ \\ 
	 ~ Milk &  ~ \ \ &  ~ \ \ &  ~ \ \ \\ 
	 ~ Meat &  ~ \ \ &  ~ \ \ &  ~ \ \ \\ 
	 ~ Fish  &  ~ \ \ &  ~ \ \ &  ~ \ \ \\ 
	\multicolumn{4}{l}{\textit{\normalsize{Net trade (min US\$)}}} \\ 
	 ~ Cereals &  ~ \ \ &  ~ \ \ & \textit{-28} ~ \ \ \\ 
	 ~ Fruit and vegetables &  ~ \ \ &  ~ \ \ & \textit{-14} ~ \ \ \\ 
	 ~ Meat &  ~ \ \ &  ~ \ \ & \textit{0} ~ \ \ \\ 
	 ~ Dairy products &  ~ \ \ &  ~ \ \ & \textit{0} ~ \ \ \\ 
	 ~ Fish &  ~ \ \ &  ~ \ \ &  ~ \ \ \\ 
	\multicolumn{4}{l}{\textcolor{FAOblue}{\textbf{\large{Environment}}}} \\ 
	 ~ Forest area (\%) &  ~ \ \ &  ~ \ \ &  ~ \ \ \\ 
	 ~ Renewable water res withdrawn (\% of total) &  ~ \ \ &  ~ \ \ & 36 ~ \ \ \\ 
	 ~ Terrestrial protect areas (\% total land area)  &  ~ \ \ &  ~ \ \ &  ~ \ \ \\ 
	 ~ Organic area (\% total agricultural area) &  ~ \ \ &  ~ \ \ &  ~ \ \ \\ 
	 ~ Water withdrawal by agriculture (\% of total) &  ~ \ \ &  ~ \ \ & 36 ~ \ \ \\ 
	 ~ Biofuel production (thousand kt of oil eq.) &  ~ \ \ &  ~ \ \ &  ~ \ \ \\ 
	 ~ Wood pellet prod. (min tonnes) &  ~ \ \ &  ~ \ \ &  ~ \ \ \\ 
	 ~ GHG emissions from ag (Co2 eq, gigagrams) &  ~ \ \ &  ~ \ \ &  ~ \ \ \\ 
       \toprule
      \end{tabular}
      \clearpage
\CountryData{ Spain }
      \rowcolors{1}{FAOblue!10}{white}
      \begin{tabular}{L{3.9cm} R{1cm} R{1cm} R{1cm}}
      \toprule
      \multicolumn{1}{c}{} & \multicolumn{1}{c}{ 1992 } & \multicolumn{1}{c}{ 2002 } & \multicolumn{1}{c}{ 2014 } \\
      \midrule
	\multicolumn{4}{l}{\textcolor{FAOblue}{\textbf{\large{The setting}}}} \\ 
	 ~ Population, total (mln) & 39.1 ~ \ \ & 41.3 ~ \ \ & 47.1 ~ \ \ \\ 
	 ~ Population, rural (\% total population) & 9.5 ~ \ \ & 9.7 ~ \ \ & 10.4 ~ \ \ \\ 
	 ~ Govt expenditure on ag (\% total outlays) &  ~ \ \ &  ~ \ \ &  ~ \ \ \\ 
	 ~ Area harvested (mln ha) & 15 ~ \ \ & 22 ~ \ \ & 25 ~ \ \ \\ 
	 ~ Cropping intensity ratio (\%) & 0.5 ~ \ \ & 0.7 ~ \ \ &  ~ \ \ \\ 
	 ~ Water resources (m\textsuperscript{3}/person/year) & \textit{3} ~ \ \ & \textit{3} ~ \ \ & \textit{2} ~ \ \ \\ 
	 ~ Area equipped for irrigation (1000 ha) &  ~ \ \ &  ~ \ \ & \textit{3\,800} ~ \ \ \\ 
	 ~ Area irrigated (\%) &  ~ \ \ &  ~ \ \ & \textit{89.1} ~ \ \ \\ 
	 ~ Employment in agriculture (\%) & 9.7 ~ \ \ & 6 ~ \ \ & \textit{4.4} ~ \ \ \\ 
	 ~ Employment in agriculture, female (\%) & 8.1 ~ \ \ & 4.1 ~ \ \ & \textit{2.5} ~ \ \ \\ 
	 ~ Fertilizers, Nitrogen (nutrients per ha) &  ~ \ \ & 34.9 ~ \ \ & \textit{31.3} ~ \ \ \\ 
	 ~ Fertilizers, Phosphate (nutrients per ha) &  ~ \ \ & 20.6 ~ \ \ & \textit{14} ~ \ \ \\ 
	 ~ Fertilizers, Potash (nutrients per ha) &  ~ \ \ & 16.7 ~ \ \ & \textit{11.9} ~ \ \ \\ 
	 ~ Energy consump, power irrigation (mln kWh) & 309 ~ \ \ & 309 ~ \ \ & \textit{4\,708} ~ \ \ \\ 
	 ~ Agr value added per worker (constant US\$) & \textit{16} ~ \ \ & 28 ~ \ \ & \textit{40.1} ~ \ \ \\ 
	\multicolumn{4}{l}{\textcolor{FAOblue}{\textbf{\large{Hunger dimensions}}}} \\ 
	 ~ Dietary energy supply (kcal/pc/day) &  ~ \ \ &  ~ \ \ &  ~ \ \ \\ 
	 ~ Average dietary energy supply adequacy (\%) & 133 ~ \ \ & 132 ~ \ \ & 125 ~ \ \ \\ 
	 ~ Dietary en supp, cereals/roots/tubers (\%) & 28 ~ \ \ & 26 ~ \ \ & \textit{25} ~ \ \ \\ 
	 ~ Prevalence of undernourishment (\%) & <5.0 ~ \ \ & <5.0 ~ \ \ & <5.0 ~ \ \ \\ 
	 ~ GDP per capita (US\$, PPP) & 24\,831 ~ \ \ & 31\,866 ~ \ \ & \textit{31\,683} ~ \ \ \\ 
	 ~ Domestic food price volatility (index) &  ~ \ \ & 4.3 ~ \ \ & 8.4 ~ \ \ \\ 
	 ~ Cereal import dependency ratio (\%) & 12.6 ~ \ \ & 28 ~ \ \ & \textit{34.2} ~ \ \ \\ 
	 ~ Underweight, children under-5 (\%) &  ~ \ \ &  ~ \ \ &  ~ \ \ \\ 
	 ~ Improved water source (\% pop) & 100 ~ \ \ & 100 ~ \ \ & \textit{100} ~ \ \ \\ 
	\multicolumn{4}{l}{\textcolor{FAOblue}{\textbf{\large{Food Supply}}}} \\ 
	 ~ Food production value, (2004-2006 mln I\$) & 24\,544 ~ \ \ & 29\,492 ~ \ \ & \textit{32\,045} ~ \ \ \\ 
	 ~ Agriculture, value added (\% GDP) & \textit{4} ~ \ \ & 4 ~ \ \ & \textit{3} ~ \ \ \\ 
	 ~ Food exports (mln US\$)  & 7\,799 ~ \ \ & 13\,433 ~ \ \ & \textit{33\,621} ~ \ \ \\ 
	 ~ Food imports (mln US\$)  & 6\,445 ~ \ \ & 8\,498 ~ \ \ & \textit{22\,087} ~ \ \ \\ 
	\multicolumn{4}{l}{\textit{\normalsize{Production indices (2004-06=100)}}} \\ 
	 ~ Net food & 83 ~ \ \ & 100 ~ \ \ & \textit{109} ~ \ \ \\ 
	 ~ Net crop & 85 ~ \ \ & 100 ~ \ \ & \textit{115} ~ \ \ \\ 
	 ~ Cereal & 75 ~ \ \ & 115 ~ \ \ & \textit{134} ~ \ \ \\ 
	 ~ Vegetable oils & 76 ~ \ \ & 91 ~ \ \ & \textit{160} ~ \ \ \\ 
	 ~ Roots and tubers & 196 ~ \ \ & 117 ~ \ \ & \textit{84} ~ \ \ \\ 
	 ~ Fruit and vegetables & 86 ~ \ \ & 96 ~ \ \ & \textit{102} ~ \ \ \\ 
	 ~ Sugar & 108 ~ \ \ & 121 ~ \ \ & \textit{36} ~ \ \ \\ 
	 ~ Livestock & 76 ~ \ \ & 102 ~ \ \ & \textit{100} ~ \ \ \\ 
	 ~ Milk & 95 ~ \ \ & 102 ~ \ \ & \textit{106} ~ \ \ \\ 
	 ~ Meat & 70 ~ \ \ & 103 ~ \ \ & \textit{100} ~ \ \ \\ 
	 ~ Fish  & 109 ~ \ \ & 99 ~ \ \ & \textit{110} ~ \ \ \\ 
	\multicolumn{4}{l}{\textit{\normalsize{Net trade (min US\$)}}} \\ 
	 ~ Cereals & -486 ~ \ \ & -1\,030 ~ \ \ & \textit{-3\,199} ~ \ \ \\ 
	 ~ Fruit and vegetables & 3\,974 ~ \ \ & 6\,080 ~ \ \ & \textit{12\,711} ~ \ \ \\ 
	 ~ Meat & -515 ~ \ \ & 585 ~ \ \ & \textit{3\,573} ~ \ \ \\ 
	 ~ Dairy products & -400 ~ \ \ & -438 ~ \ \ & \textit{-1\,089} ~ \ \ \\ 
	 ~ Fish & -2\,085 ~ \ \ & -1\,963 ~ \ \ & \textit{-2\,467} ~ \ \ \\ 
	\multicolumn{4}{l}{\textcolor{FAOblue}{\textbf{\large{Environment}}}} \\ 
	 ~ Forest area (\%) & 29 ~ \ \ & 34 ~ \ \ & \textit{37} ~ \ \ \\ 
	 ~ Renewable water res withdrawn (\% of total) &  ~ \ \ &  ~ \ \ & 60 ~ \ \ \\ 
	 ~ Terrestrial protect areas (\% total land area)  & 8 ~ \ \ & 9 ~ \ \ & \textit{29} ~ \ \ \\ 
	 ~ Organic area (\% total agricultural area) &  ~ \ \ & \textit{2} ~ \ \ & \textit{6} ~ \ \ \\ 
	 ~ Water withdrawal by agriculture (\% of total) &  ~ \ \ &  ~ \ \ & 60 ~ \ \ \\ 
	 ~ Biofuel production (thousand kt of oil eq.) & 38 ~ \ \ & 2\,063 ~ \ \ & \textit{22\,769} ~ \ \ \\ 
	 ~ Wood pellet prod. (min tonnes) &  ~ \ \ &  ~ \ \ & \textit{350} ~ \ \ \\ 
	 ~ GHG emissions from ag (Co2 eq, gigagrams) & -3 ~ \ \ & 37 ~ \ \ & \textit{18} ~ \ \ \\ 
       \toprule
      \end{tabular}
      \clearpage
\CountryData{ Sri Lanka }
      \rowcolors{1}{FAOblue!10}{white}
      \begin{tabular}{L{3.9cm} R{1cm} R{1cm} R{1cm}}
      \toprule
      \multicolumn{1}{c}{} & \multicolumn{1}{c}{ 1992 } & \multicolumn{1}{c}{ 2002 } & \multicolumn{1}{c}{ 2014 } \\
      \midrule
	\multicolumn{4}{l}{\textcolor{FAOblue}{\textbf{\large{The setting}}}} \\ 
	 ~ Population, total (mln) & 17.7 ~ \ \ & 19.3 ~ \ \ & 21.4 ~ \ \ \\ 
	 ~ Population, rural (\% total population) & 14.7 ~ \ \ & 16.3 ~ \ \ & 18.2 ~ \ \ \\ 
	 ~ Govt expenditure on ag (\% total outlays) &  ~ \ \ & 4 ~ \ \ & \textit{5.7} ~ \ \ \\ 
	 ~ Area harvested (mln ha) & 2 ~ \ \ & 3 ~ \ \ & 5 ~ \ \ \\ 
	 ~ Cropping intensity ratio (\%) & 1 ~ \ \ & 1.2 ~ \ \ &  ~ \ \ \\ 
	 ~ Water resources (m\textsuperscript{3}/person/year) & \textit{3} ~ \ \ & \textit{3} ~ \ \ & \textit{2} ~ \ \ \\ 
	 ~ Area equipped for irrigation (1000 ha) &  ~ \ \ &  ~ \ \ & \textit{570} ~ \ \ \\ 
	 ~ Area irrigated (\%) &  ~ \ \ &  ~ \ \ & \textit{81.1} ~ \ \ \\ 
	 ~ Employment in agriculture (\%) & 43.6 ~ \ \ & 34.5 ~ \ \ & \textit{39.4} ~ \ \ \\ 
	 ~ Employment in agriculture, female (\%) & \textit{41.5} ~ \ \ & 40 ~ \ \ & \textit{34.7} ~ \ \ \\ 
	 ~ Fertilizers, Nitrogen (nutrients per ha) &  ~ \ \ & 79.9 ~ \ \ & \textit{60.8} ~ \ \ \\ 
	 ~ Fertilizers, Phosphate (nutrients per ha) &  ~ \ \ & 14.9 ~ \ \ & \textit{23.2} ~ \ \ \\ 
	 ~ Fertilizers, Potash (nutrients per ha) &  ~ \ \ & 26.1 ~ \ \ & \textit{8.7} ~ \ \ \\ 
	 ~ Energy consump, power irrigation (mln kWh) & \textit{0} ~ \ \ & 0 ~ \ \ & \textit{0} ~ \ \ \\ 
	 ~ Agr value added per worker (constant US\$) & 0.7 ~ \ \ & 0.8 ~ \ \ & \textit{1} ~ \ \ \\ 
	\multicolumn{4}{l}{\textcolor{FAOblue}{\textbf{\large{Hunger dimensions}}}} \\ 
	 ~ Dietary energy supply (kcal/pc/day) & 2\,145 ~ \ \ & 2\,339 ~ \ \ & 2\,585 ~ \ \ \\ 
	 ~ Average dietary energy supply adequacy (\%) & 95 ~ \ \ & 102 ~ \ \ & 114 ~ \ \ \\ 
	 ~ Dietary en supp, cereals/roots/tubers (\%) & 59 ~ \ \ & 56 ~ \ \ & \textit{57} ~ \ \ \\ 
	 ~ Prevalence of undernourishment (\%) & 31.3 ~ \ \ & 29.7 ~ \ \ & 22.9 ~ \ \ \\ 
	 ~ GDP per capita (US\$, PPP) & 3\,561 ~ \ \ & 5\,111 ~ \ \ & \textit{9\,426} ~ \ \ \\ 
	 ~ Domestic food price volatility (index) &  ~ \ \ & 8.3 ~ \ \ & 8.3 ~ \ \ \\ 
	 ~ Cereal import dependency ratio (\%) & 38 ~ \ \ & 36.9 ~ \ \ & \textit{25.4} ~ \ \ \\ 
	 ~ Underweight, children under-5 (\%) & \textit{29.3} ~ \ \ & \textit{22.8} ~ \ \ & \textit{26.3} ~ \ \ \\ 
	 ~ Improved water source (\% pop) & 69.9 ~ \ \ & 81.7 ~ \ \ & \textit{93.8} ~ \ \ \\ 
	\multicolumn{4}{l}{\textcolor{FAOblue}{\textbf{\large{Food Supply}}}} \\ 
	 ~ Food production value, (2004-2006 mln I\$) & 1\,602 ~ \ \ & 1\,821 ~ \ \ & \textit{2\,737} ~ \ \ \\ 
	 ~ Agriculture, value added (\% GDP) & 26 ~ \ \ & 14 ~ \ \ & \textit{11} ~ \ \ \\ 
	 ~ Food exports (mln US\$)  & 137 ~ \ \ & 182 ~ \ \ & \textit{626} ~ \ \ \\ 
	 ~ Food imports (mln US\$)  & 498 ~ \ \ & 679 ~ \ \ & \textit{1\,624} ~ \ \ \\ 
	\multicolumn{4}{l}{\textit{\normalsize{Production indices (2004-06=100)}}} \\ 
	 ~ Net food & 83 ~ \ \ & 94 ~ \ \ & \textit{142} ~ \ \ \\ 
	 ~ Net crop & 80 ~ \ \ & 94 ~ \ \ & \textit{136} ~ \ \ \\ 
	 ~ Cereal & 76 ~ \ \ & 93 ~ \ \ & \textit{154} ~ \ \ \\ 
	 ~ Vegetable oils & 89 ~ \ \ & 92 ~ \ \ & \textit{146} ~ \ \ \\ 
	 ~ Roots and tubers & 122 ~ \ \ & 105 ~ \ \ & \textit{124} ~ \ \ \\ 
	 ~ Fruit and vegetables & 92 ~ \ \ & 97 ~ \ \ & \textit{126} ~ \ \ \\ 
	 ~ Sugar & 70 ~ \ \ & 93 ~ \ \ & \textit{92} ~ \ \ \\ 
	 ~ Livestock & 88 ~ \ \ & 97 ~ \ \ & \textit{137} ~ \ \ \\ 
	 ~ Milk & 166 ~ \ \ & 95 ~ \ \ & \textit{195} ~ \ \ \\ 
	 ~ Meat & 68 ~ \ \ & 97 ~ \ \ & \textit{107} ~ \ \ \\ 
	 ~ Fish  & 74 ~ \ \ & 98 ~ \ \ & \textit{188} ~ \ \ \\ 
	\multicolumn{4}{l}{\textit{\normalsize{Net trade (min US\$)}}} \\ 
	 ~ Cereals & -196 ~ \ \ & -189 ~ \ \ & \textit{-347} ~ \ \ \\ 
	 ~ Fruit and vegetables & 23 ~ \ \ & -49 ~ \ \ & \textit{-85} ~ \ \ \\ 
	 ~ Meat & -1 ~ \ \ & -2 ~ \ \ & \textit{0} ~ \ \ \\ 
	 ~ Dairy products & -66 ~ \ \ & -109 ~ \ \ & \textit{-305} ~ \ \ \\ 
	 ~ Fish & -28 ~ \ \ & 13 ~ \ \ & \textit{55} ~ \ \ \\ 
	\multicolumn{4}{l}{\textcolor{FAOblue}{\textbf{\large{Environment}}}} \\ 
	 ~ Forest area (\%) & 37 ~ \ \ & 32 ~ \ \ & \textit{29} ~ \ \ \\ 
	 ~ Renewable water res withdrawn (\% of total) &  ~ \ \ & \textit{87} ~ \ \ & 87 ~ \ \ \\ 
	 ~ Terrestrial protect areas (\% total land area)  & 20 ~ \ \ & 21 ~ \ \ & \textit{22} ~ \ \ \\ 
	 ~ Organic area (\% total agricultural area) &  ~ \ \ & \textit{0} ~ \ \ & \textit{1} ~ \ \ \\ 
	 ~ Water withdrawal by agriculture (\% of total) &  ~ \ \ & \textit{87} ~ \ \ & 87 ~ \ \ \\ 
	 ~ Biofuel production (thousand kt of oil eq.) & 2 ~ \ \ & 1 ~ \ \ & \textit{1} ~ \ \ \\ 
	 ~ Wood pellet prod. (min tonnes) &  ~ \ \ &  ~ \ \ &  ~ \ \ \\ 
	 ~ GHG emissions from ag (Co2 eq, gigagrams) & 15 ~ \ \ & 14 ~ \ \ & \textit{13} ~ \ \ \\ 
       \toprule
      \end{tabular}
      \clearpage
\CountryData{ Sudan }
      \rowcolors{1}{FAOblue!10}{white}
      \begin{tabular}{L{3.9cm} R{1cm} R{1cm} R{1cm}}
      \toprule
      \multicolumn{1}{c}{} & \multicolumn{1}{c}{ 1992 } & \multicolumn{1}{c}{ 2002 } & \multicolumn{1}{c}{ 2014 } \\
      \midrule
	\multicolumn{4}{l}{\textcolor{FAOblue}{\textbf{\large{The setting}}}} \\ 
	 ~ Population, total (mln) &  ~ \ \ &  ~ \ \ & 38.8 ~ \ \ \\ 
	 ~ Population, rural (\% total population) &  ~ \ \ &  ~ \ \ & 27.1 ~ \ \ \\ 
	 ~ Govt expenditure on ag (\% total outlays) &  ~ \ \ &  ~ \ \ &  ~ \ \ \\ 
	 ~ Area harvested (mln ha) &  ~ \ \ &  ~ \ \ &  ~ \ \ \\ 
	 ~ Cropping intensity ratio (\%) &  ~ \ \ &  ~ \ \ &  ~ \ \ \\ 
	 ~ Water resources (m\textsuperscript{3}/person/year) &  ~ \ \ &  ~ \ \ & \textit{1} ~ \ \ \\ 
	 ~ Area equipped for irrigation (1000 ha) &  ~ \ \ &  ~ \ \ & \textit{1\,890} ~ \ \ \\ 
	 ~ Area irrigated (\%) &  ~ \ \ &  ~ \ \ &  ~ \ \ \\ 
	 ~ Employment in agriculture (\%) &  ~ \ \ &  ~ \ \ &  ~ \ \ \\ 
	 ~ Employment in agriculture, female (\%) &  ~ \ \ &  ~ \ \ &  ~ \ \ \\ 
	 ~ Fertilizers, Nitrogen (nutrients per ha) &  ~ \ \ &  ~ \ \ & \textit{1.3} ~ \ \ \\ 
	 ~ Fertilizers, Phosphate (nutrients per ha) &  ~ \ \ &  ~ \ \ & \textit{0.6} ~ \ \ \\ 
	 ~ Fertilizers, Potash (nutrients per ha) &  ~ \ \ &  ~ \ \ & \textit{0} ~ \ \ \\ 
	 ~ Energy consump, power irrigation (mln kWh) &  ~ \ \ &  ~ \ \ &  ~ \ \ \\ 
	 ~ Agr value added per worker (constant US\$) & 1 ~ \ \ & 1.3 ~ \ \ & \textit{1.7} ~ \ \ \\ 
	\multicolumn{4}{l}{\textcolor{FAOblue}{\textbf{\large{Hunger dimensions}}}} \\ 
	 ~ Dietary energy supply (kcal/pc/day) &  ~ \ \ &  ~ \ \ &  ~ \ \ \\ 
	 ~ Average dietary energy supply adequacy (\%) &  ~ \ \ &  ~ \ \ &  ~ \ \ \\ 
	 ~ Dietary en supp, cereals/roots/tubers (\%) &  ~ \ \ &  ~ \ \ &  ~ \ \ \\ 
	 ~ Prevalence of undernourishment (\%) &  ~ \ \ &  ~ \ \ &  ~ \ \ \\ 
	 ~ GDP per capita (US\$, PPP) & 2\,008 ~ \ \ & 2\,567 ~ \ \ & \textit{3\,265} ~ \ \ \\ 
	 ~ Domestic food price volatility (index) &  ~ \ \ &  ~ \ \ &  ~ \ \ \\ 
	 ~ Cereal import dependency ratio (\%) &  ~ \ \ &  ~ \ \ & \textit{25} ~ \ \ \\ 
	 ~ Underweight, children under-5 (\%) & \textit{31.8} ~ \ \ & \textit{38.4} ~ \ \ & \textit{27} ~ \ \ \\ 
	 ~ Improved water source (\% pop) & 67.4 ~ \ \ & 60.6 ~ \ \ & \textit{55.5} ~ \ \ \\ 
	\multicolumn{4}{l}{\textcolor{FAOblue}{\textbf{\large{Food Supply}}}} \\ 
	 ~ Food production value, (2004-2006 mln I\$) &  ~ \ \ &  ~ \ \ &  ~ \ \ \\ 
	 ~ Agriculture, value added (\% GDP) & 40 ~ \ \ & 42 ~ \ \ & \textit{28} ~ \ \ \\ 
	 ~ Food exports (mln US\$)  &  ~ \ \ &  ~ \ \ &  ~ \ \ \\ 
	 ~ Food imports (mln US\$)  &  ~ \ \ &  ~ \ \ &  ~ \ \ \\ 
	\multicolumn{4}{l}{\textit{\normalsize{Production indices (2004-06=100)}}} \\ 
	 ~ Net food &  ~ \ \ &  ~ \ \ &  ~ \ \ \\ 
	 ~ Net crop &  ~ \ \ &  ~ \ \ &  ~ \ \ \\ 
	 ~ Cereal &  ~ \ \ &  ~ \ \ &  ~ \ \ \\ 
	 ~ Vegetable oils &  ~ \ \ &  ~ \ \ &  ~ \ \ \\ 
	 ~ Roots and tubers &  ~ \ \ &  ~ \ \ &  ~ \ \ \\ 
	 ~ Fruit and vegetables &  ~ \ \ &  ~ \ \ &  ~ \ \ \\ 
	 ~ Sugar &  ~ \ \ &  ~ \ \ &  ~ \ \ \\ 
	 ~ Livestock &  ~ \ \ &  ~ \ \ &  ~ \ \ \\ 
	 ~ Milk &  ~ \ \ &  ~ \ \ &  ~ \ \ \\ 
	 ~ Meat &  ~ \ \ &  ~ \ \ &  ~ \ \ \\ 
	 ~ Fish  &  ~ \ \ &  ~ \ \ &  ~ \ \ \\ 
	\multicolumn{4}{l}{\textit{\normalsize{Net trade (min US\$)}}} \\ 
	 ~ Cereals &  ~ \ \ &  ~ \ \ &  ~ \ \ \\ 
	 ~ Fruit and vegetables &  ~ \ \ &  ~ \ \ &  ~ \ \ \\ 
	 ~ Meat &  ~ \ \ &  ~ \ \ &  ~ \ \ \\ 
	 ~ Dairy products &  ~ \ \ &  ~ \ \ &  ~ \ \ \\ 
	 ~ Fish &  ~ \ \ &  ~ \ \ & \textit{-5} ~ \ \ \\ 
	\multicolumn{4}{l}{\textcolor{FAOblue}{\textbf{\large{Environment}}}} \\ 
	 ~ Forest area (\%) &  ~ \ \ &  ~ \ \ &  ~ \ \ \\ 
	 ~ Renewable water res withdrawn (\% of total) &  ~ \ \ &  ~ \ \ & 96 ~ \ \ \\ 
	 ~ Terrestrial protect areas (\% total land area)  & 4 ~ \ \ & 4 ~ \ \ & \textit{7} ~ \ \ \\ 
	 ~ Organic area (\% total agricultural area) &  ~ \ \ &  ~ \ \ &  ~ \ \ \\ 
	 ~ Water withdrawal by agriculture (\% of total) &  ~ \ \ &  ~ \ \ & 96 ~ \ \ \\ 
	 ~ Biofuel production (thousand kt of oil eq.) &  ~ \ \ &  ~ \ \ &  ~ \ \ \\ 
	 ~ Wood pellet prod. (min tonnes) &  ~ \ \ &  ~ \ \ &  ~ \ \ \\ 
	 ~ GHG emissions from ag (Co2 eq, gigagrams) &  ~ \ \ &  ~ \ \ &  ~ \ \ \\ 
       \toprule
      \end{tabular}
      \clearpage
\CountryData{ Suriname }
      \rowcolors{1}{FAOblue!10}{white}
      \begin{tabular}{L{3.9cm} R{1cm} R{1cm} R{1cm}}
      \toprule
      \multicolumn{1}{c}{} & \multicolumn{1}{c}{ 1992 } & \multicolumn{1}{c}{ 2002 } & \multicolumn{1}{c}{ 2014 } \\
      \midrule
	\multicolumn{4}{l}{\textcolor{FAOblue}{\textbf{\large{The setting}}}} \\ 
	 ~ Population, total (mln) & 0.4 ~ \ \ & 0.5 ~ \ \ & 0.5 ~ \ \ \\ 
	 ~ Population, rural (\% total population) & 0.2 ~ \ \ & 0.2 ~ \ \ & 0.2 ~ \ \ \\ 
	 ~ Govt expenditure on ag (\% total outlays) &  ~ \ \ &  ~ \ \ &  ~ \ \ \\ 
	 ~ Area harvested (mln ha) & 0 ~ \ \ & 0 ~ \ \ & 0 ~ \ \ \\ 
	 ~ Cropping intensity ratio (\%) & 3.5 ~ \ \ & 5.4 ~ \ \ &  ~ \ \ \\ 
	 ~ Water resources (m\textsuperscript{3}/person/year) & \textit{233} ~ \ \ & \textit{203} ~ \ \ & \textit{184} ~ \ \ \\ 
	 ~ Area equipped for irrigation (1000 ha) &  ~ \ \ &  ~ \ \ & \textit{57} ~ \ \ \\ 
	 ~ Area irrigated (\%) &  ~ \ \ &  ~ \ \ & \textit{100} ~ \ \ \\ 
	 ~ Employment in agriculture (\%) & 4.2 ~ \ \ & \textit{8} ~ \ \ &  ~ \ \ \\ 
	 ~ Employment in agriculture, female (\%) & 3.2 ~ \ \ & \textit{4.5} ~ \ \ &  ~ \ \ \\ 
	 ~ Fertilizers, Nitrogen (nutrients per ha) &  ~ \ \ & 52.9 ~ \ \ & \textit{30.5} ~ \ \ \\ 
	 ~ Fertilizers, Phosphate (nutrients per ha) &  ~ \ \ & 1.6 ~ \ \ & \textit{0.7} ~ \ \ \\ 
	 ~ Fertilizers, Potash (nutrients per ha) &  ~ \ \ & 1.6 ~ \ \ & \textit{13.8} ~ \ \ \\ 
	 ~ Energy consump, power irrigation (mln kWh) &  ~ \ \ & 2 ~ \ \ & \textit{2} ~ \ \ \\ 
	 ~ Agr value added per worker (constant US\$) & 3.4 ~ \ \ & 3 ~ \ \ & \textit{3.8} ~ \ \ \\ 
	\multicolumn{4}{l}{\textcolor{FAOblue}{\textbf{\large{Hunger dimensions}}}} \\ 
	 ~ Dietary energy supply (kcal/pc/day) & 2\,532 ~ \ \ & 2\,541 ~ \ \ & 2\,773 ~ \ \ \\ 
	 ~ Average dietary energy supply adequacy (\%) & 108 ~ \ \ & 108 ~ \ \ & 116 ~ \ \ \\ 
	 ~ Dietary en supp, cereals/roots/tubers (\%) & 50 ~ \ \ & 44 ~ \ \ & \textit{43} ~ \ \ \\ 
	 ~ Prevalence of undernourishment (\%) & 14.5 ~ \ \ & 13.8 ~ \ \ & 8.3 ~ \ \ \\ 
	 ~ GDP per capita (US\$, PPP) & 10\,588 ~ \ \ & 10\,502 ~ \ \ & \textit{15\,556} ~ \ \ \\ 
	 ~ Domestic food price volatility (index) &  ~ \ \ & 18.6 ~ \ \ & \textit{9.7} ~ \ \ \\ 
	 ~ Cereal import dependency ratio (\%) & -4.7 ~ \ \ & 12.8 ~ \ \ & \textit{6.3} ~ \ \ \\ 
	 ~ Underweight, children under-5 (\%) &  ~ \ \ & \textit{11.4} ~ \ \ & \textit{5.8} ~ \ \ \\ 
	 ~ Improved water source (\% pop) & \textit{87.3} ~ \ \ & 90.1 ~ \ \ & \textit{95.2} ~ \ \ \\ 
	\multicolumn{4}{l}{\textcolor{FAOblue}{\textbf{\large{Food Supply}}}} \\ 
	 ~ Food production value, (2004-2006 mln I\$) & 122 ~ \ \ & 75 ~ \ \ & \textit{136} ~ \ \ \\ 
	 ~ Agriculture, value added (\% GDP) & 11 ~ \ \ & 7 ~ \ \ & \textit{7} ~ \ \ \\ 
	 ~ Food exports (mln US\$)  & 42 ~ \ \ & 34 ~ \ \ & \textit{95} ~ \ \ \\ 
	 ~ Food imports (mln US\$)  & 55 ~ \ \ & 67 ~ \ \ & \textit{207} ~ \ \ \\ 
	\multicolumn{4}{l}{\textit{\normalsize{Production indices (2004-06=100)}}} \\ 
	 ~ Net food & 138 ~ \ \ & 85 ~ \ \ & \textit{154} ~ \ \ \\ 
	 ~ Net crop & 136 ~ \ \ & 81 ~ \ \ & \textit{155} ~ \ \ \\ 
	 ~ Cereal & 152 ~ \ \ & 91 ~ \ \ & \textit{153} ~ \ \ \\ 
	 ~ Vegetable oils & 471 ~ \ \ & 119 ~ \ \ & \textit{126} ~ \ \ \\ 
	 ~ Roots and tubers & 84 ~ \ \ & 94 ~ \ \ & \textit{193} ~ \ \ \\ 
	 ~ Fruit and vegetables & 110 ~ \ \ & 52 ~ \ \ & \textit{159} ~ \ \ \\ 
	 ~ Sugar & 81 ~ \ \ & 138 ~ \ \ & \textit{139} ~ \ \ \\ 
	 ~ Livestock & 147 ~ \ \ & 97 ~ \ \ & \textit{115} ~ \ \ \\ 
	 ~ Milk & 216 ~ \ \ & 110 ~ \ \ & \textit{72} ~ \ \ \\ 
	 ~ Meat & 139 ~ \ \ & 93 ~ \ \ & \textit{122} ~ \ \ \\ 
	 ~ Fish  & 36 ~ \ \ & 98 ~ \ \ & \textit{126} ~ \ \ \\ 
	\multicolumn{4}{l}{\textit{\normalsize{Net trade (min US\$)}}} \\ 
	 ~ Cereals & 20 ~ \ \ & 0 ~ \ \ & \textit{-5} ~ \ \ \\ 
	 ~ Fruit and vegetables & 4 ~ \ \ & 17 ~ \ \ & \textit{31} ~ \ \ \\ 
	 ~ Meat & \textit{-3} ~ \ \ & \textit{-19} ~ \ \ & \textit{-38} ~ \ \ \\ 
	 ~ Dairy products & \textit{-8} ~ \ \ & \textit{-8} ~ \ \ & \textit{-15} ~ \ \ \\ 
	 ~ Fish & 2 ~ \ \ & 35 ~ \ \ & \textit{67} ~ \ \ \\ 
	\multicolumn{4}{l}{\textcolor{FAOblue}{\textbf{\large{Environment}}}} \\ 
	 ~ Forest area (\%) & 95 ~ \ \ & 95 ~ \ \ & \textit{95} ~ \ \ \\ 
	 ~ Renewable water res withdrawn (\% of total) &  ~ \ \ &  ~ \ \ & 70 ~ \ \ \\ 
	 ~ Terrestrial protect areas (\% total land area)  & 4 ~ \ \ & 11 ~ \ \ & \textit{15} ~ \ \ \\ 
	 ~ Organic area (\% total agricultural area) &  ~ \ \ &  ~ \ \ &  ~ \ \ \\ 
	 ~ Water withdrawal by agriculture (\% of total) &  ~ \ \ &  ~ \ \ & 70 ~ \ \ \\ 
	 ~ Biofuel production (thousand kt of oil eq.) & 0 ~ \ \ & 0 ~ \ \ & \textit{0} ~ \ \ \\ 
	 ~ Wood pellet prod. (min tonnes) &  ~ \ \ &  ~ \ \ &  ~ \ \ \\ 
	 ~ GHG emissions from ag (Co2 eq, gigagrams) & 3 ~ \ \ & 3 ~ \ \ & \textit{6} ~ \ \ \\ 
       \toprule
      \end{tabular}
      \clearpage
\CountryData{ Swaziland }
      \rowcolors{1}{FAOblue!10}{white}
      \begin{tabular}{L{3.9cm} R{1cm} R{1cm} R{1cm}}
      \toprule
      \multicolumn{1}{c}{} & \multicolumn{1}{c}{ 1992 } & \multicolumn{1}{c}{ 2002 } & \multicolumn{1}{c}{ 2014 } \\
      \midrule
	\multicolumn{4}{l}{\textcolor{FAOblue}{\textbf{\large{The setting}}}} \\ 
	 ~ Population, total (mln) & 0.9 ~ \ \ & 1.1 ~ \ \ & 1.3 ~ \ \ \\ 
	 ~ Population, rural (\% total population) & 0.7 ~ \ \ & 0.8 ~ \ \ & 1 ~ \ \ \\ 
	 ~ Govt expenditure on ag (\% total outlays) &  ~ \ \ & 3.8 ~ \ \ & \textit{4.9} ~ \ \ \\ 
	 ~ Area harvested (mln ha) & 4 ~ \ \ & 5 ~ \ \ & 5 ~ \ \ \\ 
	 ~ Cropping intensity ratio (\%) & 3.2 ~ \ \ & 3.8 ~ \ \ &  ~ \ \ \\ 
	 ~ Water resources (m\textsuperscript{3}/person/year) & \textit{5} ~ \ \ & \textit{4} ~ \ \ & \textit{4} ~ \ \ \\ 
	 ~ Area equipped for irrigation (1000 ha) &  ~ \ \ &  ~ \ \ & \textit{50} ~ \ \ \\ 
	 ~ Area irrigated (\%) &  ~ \ \ & 89.9 ~ \ \ &  ~ \ \ \\ 
	 ~ Employment in agriculture (\%) &  ~ \ \ &  ~ \ \ &  ~ \ \ \\ 
	 ~ Employment in agriculture, female (\%) &  ~ \ \ &  ~ \ \ &  ~ \ \ \\ 
	 ~ Fertilizers, Nitrogen (nutrients per ha) &  ~ \ \ &  ~ \ \ &  ~ \ \ \\ 
	 ~ Fertilizers, Phosphate (nutrients per ha) &  ~ \ \ &  ~ \ \ &  ~ \ \ \\ 
	 ~ Fertilizers, Potash (nutrients per ha) &  ~ \ \ &  ~ \ \ &  ~ \ \ \\ 
	 ~ Energy consump, power irrigation (mln kWh) & \textit{0} ~ \ \ & 58 ~ \ \ & \textit{58} ~ \ \ \\ 
	 ~ Agr value added per worker (constant US\$) & 1 ~ \ \ & 1.1 ~ \ \ & \textit{1.4} ~ \ \ \\ 
	\multicolumn{4}{l}{\textcolor{FAOblue}{\textbf{\large{Hunger dimensions}}}} \\ 
	 ~ Dietary energy supply (kcal/pc/day) & 2\,306 ~ \ \ & 2\,380 ~ \ \ & 2\,239 ~ \ \ \\ 
	 ~ Average dietary energy supply adequacy (\%) & 109 ~ \ \ & 107 ~ \ \ & 98 ~ \ \ \\ 
	 ~ Dietary en supp, cereals/roots/tubers (\%) & 57 ~ \ \ & 55 ~ \ \ & \textit{59} ~ \ \ \\ 
	 ~ Prevalence of undernourishment (\%) & 16.6 ~ \ \ & 17.7 ~ \ \ & 26.5 ~ \ \ \\ 
	 ~ GDP per capita (US\$, PPP) & 5\,362 ~ \ \ & 5\,899 ~ \ \ & \textit{6\,471} ~ \ \ \\ 
	 ~ Domestic food price volatility (index) &  ~ \ \ &  ~ \ \ &  ~ \ \ \\ 
	 ~ Cereal import dependency ratio (\%) & 57.4 ~ \ \ & 70.4 ~ \ \ & \textit{72.9} ~ \ \ \\ 
	 ~ Underweight, children under-5 (\%) &  ~ \ \ & \textit{9.1} ~ \ \ & \textit{5.8} ~ \ \ \\ 
	 ~ Improved water source (\% pop) & 38.9 ~ \ \ & 55.5 ~ \ \ & \textit{74.1} ~ \ \ \\ 
	\multicolumn{4}{l}{\textcolor{FAOblue}{\textbf{\large{Food Supply}}}} \\ 
	 ~ Food production value, (2004-2006 mln I\$) & 257 ~ \ \ & 259 ~ \ \ & \textit{308} ~ \ \ \\ 
	 ~ Agriculture, value added (\% GDP) & 9 ~ \ \ & 10 ~ \ \ & \textit{7} ~ \ \ \\ 
	 ~ Food exports (mln US\$)  & 286 ~ \ \ & 174 ~ \ \ & \textit{234} ~ \ \ \\ 
	 ~ Food imports (mln US\$)  & 83 ~ \ \ & 125 ~ \ \ & \textit{152} ~ \ \ \\ 
	\multicolumn{4}{l}{\textit{\normalsize{Production indices (2004-06=100)}}} \\ 
	 ~ Net food & 94 ~ \ \ & 95 ~ \ \ & \textit{112} ~ \ \ \\ 
	 ~ Net crop & 89 ~ \ \ & 95 ~ \ \ & \textit{111} ~ \ \ \\ 
	 ~ Cereal & 142 ~ \ \ & 96 ~ \ \ & \textit{118} ~ \ \ \\ 
	 ~ Vegetable oils & 106 ~ \ \ & 112 ~ \ \ & \textit{59} ~ \ \ \\ 
	 ~ Roots and tubers & 80 ~ \ \ & 97 ~ \ \ & \textit{127} ~ \ \ \\ 
	 ~ Fruit and vegetables & 132 ~ \ \ & 104 ~ \ \ & \textit{117} ~ \ \ \\ 
	 ~ Sugar & 78 ~ \ \ & 92 ~ \ \ & \textit{109} ~ \ \ \\ 
	 ~ Livestock & 112 ~ \ \ & 96 ~ \ \ & \textit{117} ~ \ \ \\ 
	 ~ Milk & 117 ~ \ \ & 103 ~ \ \ & \textit{110} ~ \ \ \\ 
	 ~ Meat & 113 ~ \ \ & 94 ~ \ \ & \textit{119} ~ \ \ \\ 
	 ~ Fish  & 167 ~ \ \ & 100 ~ \ \ & \textit{167} ~ \ \ \\ 
	\multicolumn{4}{l}{\textit{\normalsize{Net trade (min US\$)}}} \\ 
	 ~ Cereals & -22 ~ \ \ & -38 ~ \ \ & \textit{-96} ~ \ \ \\ 
	 ~ Fruit and vegetables & 18 ~ \ \ & 7 ~ \ \ & \textit{11} ~ \ \ \\ 
	 ~ Meat & -8 ~ \ \ & -8 ~ \ \ & \textit{-2} ~ \ \ \\ 
	 ~ Dairy products & -9 ~ \ \ & -11 ~ \ \ & \textit{-3} ~ \ \ \\ 
	 ~ Fish &  ~ \ \ & -3 ~ \ \ & \textit{-4} ~ \ \ \\ 
	\multicolumn{4}{l}{\textcolor{FAOblue}{\textbf{\large{Environment}}}} \\ 
	 ~ Forest area (\%) & 28 ~ \ \ & 31 ~ \ \ & \textit{33} ~ \ \ \\ 
	 ~ Renewable water res withdrawn (\% of total) &  ~ \ \ & \textit{96} ~ \ \ & 96 ~ \ \ \\ 
	 ~ Terrestrial protect areas (\% total land area)  & 3 ~ \ \ & 3 ~ \ \ & \textit{3} ~ \ \ \\ 
	 ~ Organic area (\% total agricultural area) &  ~ \ \ &  ~ \ \ & \textit{0} ~ \ \ \\ 
	 ~ Water withdrawal by agriculture (\% of total) &  ~ \ \ & \textit{96} ~ \ \ & 96 ~ \ \ \\ 
	 ~ Biofuel production (thousand kt of oil eq.) & 8 ~ \ \ & 10 ~ \ \ & \textit{11} ~ \ \ \\ 
	 ~ Wood pellet prod. (min tonnes) &  ~ \ \ &  ~ \ \ &  ~ \ \ \\ 
	 ~ GHG emissions from ag (Co2 eq, gigagrams) & 1 ~ \ \ & 1 ~ \ \ & \textit{1} ~ \ \ \\ 
       \toprule
      \end{tabular}
      \clearpage
\CountryData{ Sweden }
      \rowcolors{1}{FAOblue!10}{white}
      \begin{tabular}{L{3.9cm} R{1cm} R{1cm} R{1cm}}
      \toprule
      \multicolumn{1}{c}{} & \multicolumn{1}{c}{ 1992 } & \multicolumn{1}{c}{ 2002 } & \multicolumn{1}{c}{ 2014 } \\
      \midrule
	\multicolumn{4}{l}{\textcolor{FAOblue}{\textbf{\large{The setting}}}} \\ 
	 ~ Population, total (mln) & 8.7 ~ \ \ & 8.9 ~ \ \ & 9.6 ~ \ \ \\ 
	 ~ Population, rural (\% total population) & 1.4 ~ \ \ & 1.4 ~ \ \ & 1.4 ~ \ \ \\ 
	 ~ Govt expenditure on ag (\% total outlays) &  ~ \ \ &  ~ \ \ &  ~ \ \ \\ 
	 ~ Area harvested (mln ha) & 4 ~ \ \ & 5 ~ \ \ & 5 ~ \ \ \\ 
	 ~ Cropping intensity ratio (\%) & 1.1 ~ \ \ & 1.7 ~ \ \ &  ~ \ \ \\ 
	 ~ Water resources (m\textsuperscript{3}/person/year) & \textit{20} ~ \ \ & \textit{19} ~ \ \ & \textit{18} ~ \ \ \\ 
	 ~ Area equipped for irrigation (1000 ha) &  ~ \ \ &  ~ \ \ & \textit{164} ~ \ \ \\ 
	 ~ Area irrigated (\%) &  ~ \ \ &  ~ \ \ & \textit{33.9} ~ \ \ \\ 
	 ~ Employment in agriculture (\%) & 3.3 ~ \ \ & 2.1 ~ \ \ & \textit{2} ~ \ \ \\ 
	 ~ Employment in agriculture, female (\%) & 1.9 ~ \ \ & 1 ~ \ \ & \textit{1} ~ \ \ \\ 
	 ~ Fertilizers, Nitrogen (nutrients per ha) &  ~ \ \ & 58.3 ~ \ \ & \textit{48.6} ~ \ \ \\ 
	 ~ Fertilizers, Phosphate (nutrients per ha) &  ~ \ \ & 12.1 ~ \ \ & \textit{3.4} ~ \ \ \\ 
	 ~ Fertilizers, Potash (nutrients per ha) &  ~ \ \ & 14 ~ \ \ & \textit{7.1} ~ \ \ \\ 
	 ~ Energy consump, power irrigation (mln kWh) &  ~ \ \ &  ~ \ \ &  ~ \ \ \\ 
	 ~ Agr value added per worker (constant US\$) & 18.4 ~ \ \ & 27 ~ \ \ & \textit{40.4} ~ \ \ \\ 
	\multicolumn{4}{l}{\textcolor{FAOblue}{\textbf{\large{Hunger dimensions}}}} \\ 
	 ~ Dietary energy supply (kcal/pc/day) &  ~ \ \ &  ~ \ \ &  ~ \ \ \\ 
	 ~ Average dietary energy supply adequacy (\%) & 120 ~ \ \ & 123 ~ \ \ & 126 ~ \ \ \\ 
	 ~ Dietary en supp, cereals/roots/tubers (\%) & 27 ~ \ \ & 28 ~ \ \ & \textit{28} ~ \ \ \\ 
	 ~ Prevalence of undernourishment (\%) & <5.0 ~ \ \ & <5.0 ~ \ \ & <5.0 ~ \ \ \\ 
	 ~ GDP per capita (US\$, PPP) & 29\,813 ~ \ \ & 37\,941 ~ \ \ & \textit{43\,540} ~ \ \ \\ 
	 ~ Domestic food price volatility (index) &  ~ \ \ & 6.8 ~ \ \ & 6.7 ~ \ \ \\ 
	 ~ Cereal import dependency ratio (\%) & -21 ~ \ \ & -25.5 ~ \ \ & \textit{-15} ~ \ \ \\ 
	 ~ Underweight, children under-5 (\%) &  ~ \ \ &  ~ \ \ &  ~ \ \ \\ 
	 ~ Improved water source (\% pop) & 100 ~ \ \ & 100 ~ \ \ & \textit{100} ~ \ \ \\ 
	\multicolumn{4}{l}{\textcolor{FAOblue}{\textbf{\large{Food Supply}}}} \\ 
	 ~ Food production value, (2004-2006 mln I\$) & 2\,546 ~ \ \ & 2\,858 ~ \ \ & \textit{2\,630} ~ \ \ \\ 
	 ~ Agriculture, value added (\% GDP) & 3 ~ \ \ & 2 ~ \ \ & \textit{1} ~ \ \ \\ 
	 ~ Food exports (mln US\$)  & 788 ~ \ \ & 1\,391 ~ \ \ & \textit{3\,841} ~ \ \ \\ 
	 ~ Food imports (mln US\$)  & 2\,289 ~ \ \ & 3\,133 ~ \ \ & \textit{8\,537} ~ \ \ \\ 
	\multicolumn{4}{l}{\textit{\normalsize{Production indices (2004-06=100)}}} \\ 
	 ~ Net food & 90 ~ \ \ & 102 ~ \ \ & \textit{93} ~ \ \ \\ 
	 ~ Net crop & 87 ~ \ \ & 104 ~ \ \ & \textit{102} ~ \ \ \\ 
	 ~ Cereal & 74 ~ \ \ & 108 ~ \ \ & \textit{99} ~ \ \ \\ 
	 ~ Vegetable oils & 124 ~ \ \ & 72 ~ \ \ & \textit{154} ~ \ \ \\ 
	 ~ Roots and tubers & 138 ~ \ \ & 101 ~ \ \ & \textit{89} ~ \ \ \\ 
	 ~ Fruit and vegetables & 71 ~ \ \ & 92 ~ \ \ & \textit{110} ~ \ \ \\ 
	 ~ Sugar & 93 ~ \ \ & 117 ~ \ \ & \textit{102} ~ \ \ \\ 
	 ~ Livestock & 96 ~ \ \ & 102 ~ \ \ & \textit{93} ~ \ \ \\ 
	 ~ Milk & 97 ~ \ \ & 102 ~ \ \ & \textit{90} ~ \ \ \\ 
	 ~ Meat & 93 ~ \ \ & 104 ~ \ \ & \textit{92} ~ \ \ \\ 
	 ~ Fish  & 116 ~ \ \ & 111 ~ \ \ & \textit{70} ~ \ \ \\ 
	\multicolumn{4}{l}{\textit{\normalsize{Net trade (min US\$)}}} \\ 
	 ~ Cereals & 37 ~ \ \ & 42 ~ \ \ & \textit{227} ~ \ \ \\ 
	 ~ Fruit and vegetables & -1\,024 ~ \ \ & -997 ~ \ \ & \textit{-2\,003} ~ \ \ \\ 
	 ~ Meat & -199 ~ \ \ & -390 ~ \ \ & \textit{-1\,413} ~ \ \ \\ 
	 ~ Dairy products & -71 ~ \ \ & -116 ~ \ \ & \textit{-581} ~ \ \ \\ 
	 ~ Fish & -260 ~ \ \ & -277 ~ \ \ & \textit{-752} ~ \ \ \\ 
	\multicolumn{4}{l}{\textcolor{FAOblue}{\textbf{\large{Environment}}}} \\ 
	 ~ Forest area (\%) & 67 ~ \ \ & 68 ~ \ \ & \textit{69} ~ \ \ \\ 
	 ~ Renewable water res withdrawn (\% of total) &  ~ \ \ &  ~ \ \ & 4 ~ \ \ \\ 
	 ~ Terrestrial protect areas (\% total land area)  & 6 ~ \ \ & 10 ~ \ \ & \textit{15} ~ \ \ \\ 
	 ~ Organic area (\% total agricultural area) &  ~ \ \ & \textit{7} ~ \ \ & \textit{16} ~ \ \ \\ 
	 ~ Water withdrawal by agriculture (\% of total) &  ~ \ \ &  ~ \ \ & 4 ~ \ \ \\ 
	 ~ Biofuel production (thousand kt of oil eq.) & 233 ~ \ \ & 222 ~ \ \ & \textit{5\,563} ~ \ \ \\ 
	 ~ Wood pellet prod. (min tonnes) &  ~ \ \ &  ~ \ \ & \textit{1\,306} ~ \ \ \\ 
	 ~ GHG emissions from ag (Co2 eq, gigagrams) & 10 ~ \ \ & -15 ~ \ \ & \textit{-16} ~ \ \ \\ 
       \toprule
      \end{tabular}
      \clearpage
\CountryData{ Switzerland }
      \rowcolors{1}{FAOblue!10}{white}
      \begin{tabular}{L{3.9cm} R{1cm} R{1cm} R{1cm}}
      \toprule
      \multicolumn{1}{c}{} & \multicolumn{1}{c}{ 1992 } & \multicolumn{1}{c}{ 2002 } & \multicolumn{1}{c}{ 2014 } \\
      \midrule
	\multicolumn{4}{l}{\textcolor{FAOblue}{\textbf{\large{The setting}}}} \\ 
	 ~ Population, total (mln) & 6.8 ~ \ \ & 7.2 ~ \ \ & 8.2 ~ \ \ \\ 
	 ~ Population, rural (\% total population) & 1.8 ~ \ \ & 1.9 ~ \ \ & 2.1 ~ \ \ \\ 
	 ~ Govt expenditure on ag (\% total outlays) &  ~ \ \ &  ~ \ \ &  ~ \ \ \\ 
	 ~ Area harvested (mln ha) & 2 ~ \ \ & 1 ~ \ \ & 1 ~ \ \ \\ 
	 ~ Cropping intensity ratio (\%) & 1.1 ~ \ \ & 0.9 ~ \ \ &  ~ \ \ \\ 
	 ~ Water resources (m\textsuperscript{3}/person/year) & \textit{8} ~ \ \ & \textit{7} ~ \ \ & \textit{7} ~ \ \ \\ 
	 ~ Area equipped for irrigation (1000 ha) &  ~ \ \ &  ~ \ \ & \textit{63} ~ \ \ \\ 
	 ~ Area irrigated (\%) &  ~ \ \ &  ~ \ \ & \textit{59.3} ~ \ \ \\ 
	 ~ Employment in agriculture (\%) & 4.3 ~ \ \ & 4.1 ~ \ \ & \textit{3.5} ~ \ \ \\ 
	 ~ Employment in agriculture, female (\%) & 3.9 ~ \ \ & 3.2 ~ \ \ & \textit{2.9} ~ \ \ \\ 
	 ~ Fertilizers, Nitrogen (nutrients per ha) &  ~ \ \ & 26 ~ \ \ & \textit{31.8} ~ \ \ \\ 
	 ~ Fertilizers, Phosphate (nutrients per ha) &  ~ \ \ & 10.2 ~ \ \ & \textit{10.6} ~ \ \ \\ 
	 ~ Fertilizers, Potash (nutrients per ha) &  ~ \ \ & 15.2 ~ \ \ & \textit{12.8} ~ \ \ \\ 
	 ~ Energy consump, power irrigation (mln kWh) &  ~ \ \ &  ~ \ \ &  ~ \ \ \\ 
	 ~ Agr value added per worker (constant US\$) & 19.3 ~ \ \ & 22.5 ~ \ \ & \textit{27.3} ~ \ \ \\ 
	\multicolumn{4}{l}{\textcolor{FAOblue}{\textbf{\large{Hunger dimensions}}}} \\ 
	 ~ Dietary energy supply (kcal/pc/day) &  ~ \ \ &  ~ \ \ &  ~ \ \ \\ 
	 ~ Average dietary energy supply adequacy (\%) & 132 ~ \ \ & 134 ~ \ \ & 137 ~ \ \ \\ 
	 ~ Dietary en supp, cereals/roots/tubers (\%) & 24 ~ \ \ & 24 ~ \ \ & \textit{24} ~ \ \ \\ 
	 ~ Prevalence of undernourishment (\%) & <5.0 ~ \ \ & <5.0 ~ \ \ & <5.0 ~ \ \ \\ 
	 ~ GDP per capita (US\$, PPP) & 45\,080 ~ \ \ & 49\,177 ~ \ \ & \textit{54\,993} ~ \ \ \\ 
	 ~ Domestic food price volatility (index) &  ~ \ \ & 5.7 ~ \ \ & 6.6 ~ \ \ \\ 
	 ~ Cereal import dependency ratio (\%) & 34.9 ~ \ \ & 43.1 ~ \ \ & \textit{51.5} ~ \ \ \\ 
	 ~ Underweight, children under-5 (\%) &  ~ \ \ &  ~ \ \ &  ~ \ \ \\ 
	 ~ Improved water source (\% pop) & 100 ~ \ \ & 100 ~ \ \ & \textit{100} ~ \ \ \\ 
	\multicolumn{4}{l}{\textcolor{FAOblue}{\textbf{\large{Food Supply}}}} \\ 
	 ~ Food production value, (2004-2006 mln I\$) & 2\,870 ~ \ \ & 2\,652 ~ \ \ & \textit{2\,675} ~ \ \ \\ 
	 ~ Agriculture, value added (\% GDP) & 2 ~ \ \ & 1 ~ \ \ & \textit{1} ~ \ \ \\ 
	 ~ Food exports (mln US\$)  & 1\,355 ~ \ \ & 1\,509 ~ \ \ & \textit{3\,613} ~ \ \ \\ 
	 ~ Food imports (mln US\$)  & 2\,793 ~ \ \ & 3\,050 ~ \ \ & \textit{6\,756} ~ \ \ \\ 
	\multicolumn{4}{l}{\textit{\normalsize{Production indices (2004-06=100)}}} \\ 
	 ~ Net food & 108 ~ \ \ & 100 ~ \ \ & \textit{101} ~ \ \ \\ 
	 ~ Net crop & 131 ~ \ \ & 102 ~ \ \ & \textit{89} ~ \ \ \\ 
	 ~ Cereal & 112 ~ \ \ & 104 ~ \ \ & \textit{80} ~ \ \ \\ 
	 ~ Vegetable oils & 62 ~ \ \ & 91 ~ \ \ & \textit{111} ~ \ \ \\ 
	 ~ Roots and tubers & 179 ~ \ \ & 113 ~ \ \ & \textit{71} ~ \ \ \\ 
	 ~ Fruit and vegetables & 144 ~ \ \ & 99 ~ \ \ & \textit{94} ~ \ \ \\ 
	 ~ Sugar & 66 ~ \ \ & 103 ~ \ \ & \textit{101} ~ \ \ \\ 
	 ~ Livestock & 104 ~ \ \ & 100 ~ \ \ & \textit{104} ~ \ \ \\ 
	 ~ Milk & 98 ~ \ \ & 100 ~ \ \ & \textit{102} ~ \ \ \\ 
	 ~ Meat & 113 ~ \ \ & 101 ~ \ \ & \textit{107} ~ \ \ \\ 
	 ~ Fish  & 137 ~ \ \ & 93 ~ \ \ & \textit{114} ~ \ \ \\ 
	\multicolumn{4}{l}{\textit{\normalsize{Net trade (min US\$)}}} \\ 
	 ~ Cereals & -184 ~ \ \ & -128 ~ \ \ & \textit{-384} ~ \ \ \\ 
	 ~ Fruit and vegetables & -1\,144 ~ \ \ & -1\,058 ~ \ \ & \textit{-2\,035} ~ \ \ \\ 
	 ~ Meat & -368 ~ \ \ & -398 ~ \ \ & \textit{-794} ~ \ \ \\ 
	 ~ Dairy products & 231 ~ \ \ & 159 ~ \ \ & \textit{266} ~ \ \ \\ 
	 ~ Fish & -384 ~ \ \ & -351 ~ \ \ & \textit{-710} ~ \ \ \\ 
	\multicolumn{4}{l}{\textcolor{FAOblue}{\textbf{\large{Environment}}}} \\ 
	 ~ Forest area (\%) & 29 ~ \ \ & 30 ~ \ \ & \textit{32} ~ \ \ \\ 
	 ~ Renewable water res withdrawn (\% of total) &  ~ \ \ & \textit{2} ~ \ \ & 2 ~ \ \ \\ 
	 ~ Terrestrial protect areas (\% total land area)  & 16 ~ \ \ & 23 ~ \ \ & \textit{26} ~ \ \ \\ 
	 ~ Organic area (\% total agricultural area) &  ~ \ \ & \textit{8} ~ \ \ & \textit{8} ~ \ \ \\ 
	 ~ Water withdrawal by agriculture (\% of total) &  ~ \ \ & \textit{2} ~ \ \ & 2 ~ \ \ \\ 
	 ~ Biofuel production (thousand kt of oil eq.) & 1 ~ \ \ & 4 ~ \ \ & \textit{196} ~ \ \ \\ 
	 ~ Wood pellet prod. (min tonnes) &  ~ \ \ &  ~ \ \ & \textit{168} ~ \ \ \\ 
	 ~ GHG emissions from ag (Co2 eq, gigagrams) & 3 ~ \ \ & 3 ~ \ \ & \textit{2} ~ \ \ \\ 
       \toprule
      \end{tabular}
      \clearpage
\CountryData{ Syria }
      \rowcolors{1}{FAOblue!10}{white}
      \begin{tabular}{L{3.9cm} R{1cm} R{1cm} R{1cm}}
      \toprule
      \multicolumn{1}{c}{} & \multicolumn{1}{c}{ 1992 } & \multicolumn{1}{c}{ 2002 } & \multicolumn{1}{c}{ 2014 } \\
      \midrule
	\multicolumn{4}{l}{\textcolor{FAOblue}{\textbf{\large{The setting}}}} \\ 
	 ~ Population, total (mln) & 13.2 ~ \ \ & 17 ~ \ \ & 22 ~ \ \ \\ 
	 ~ Population, rural (\% total population) & 6.7 ~ \ \ & 8 ~ \ \ & 9.4 ~ \ \ \\ 
	 ~ Govt expenditure on ag (\% total outlays) &  ~ \ \ & \textit{5.7} ~ \ \ & \textit{4.4} ~ \ \ \\ 
	 ~ Area harvested (mln ha) & 4 ~ \ \ & 6 ~ \ \ & 4 ~ \ \ \\ 
	 ~ Cropping intensity ratio (\%) & 0.3 ~ \ \ & 0.4 ~ \ \ &  ~ \ \ \\ 
	 ~ Water resources (m\textsuperscript{3}/person/year) & \textit{1} ~ \ \ & \textit{1} ~ \ \ & \textit{1} ~ \ \ \\ 
	 ~ Area equipped for irrigation (1000 ha) &  ~ \ \ &  ~ \ \ & \textit{1\,428} ~ \ \ \\ 
	 ~ Area irrigated (\%) &  ~ \ \ & \textit{95.5} ~ \ \ &  ~ \ \ \\ 
	 ~ Employment in agriculture (\%) & \textit{28.4} ~ \ \ & 31.2 ~ \ \ & \textit{14.3} ~ \ \ \\ 
	 ~ Employment in agriculture, female (\%) & \textit{60.2} ~ \ \ & 61.2 ~ \ \ & \textit{22.2} ~ \ \ \\ 
	 ~ Fertilizers, Nitrogen (nutrients per ha) &  ~ \ \ & 16.1 ~ \ \ & \textit{7.9} ~ \ \ \\ 
	 ~ Fertilizers, Phosphate (nutrients per ha) &  ~ \ \ & 6.7 ~ \ \ & \textit{2.1} ~ \ \ \\ 
	 ~ Fertilizers, Potash (nutrients per ha) &  ~ \ \ & 0.1 ~ \ \ & \textit{0} ~ \ \ \\ 
	 ~ Energy consump, power irrigation (mln kWh) & \textit{77} ~ \ \ & 77 ~ \ \ & \textit{719} ~ \ \ \\ 
	 ~ Agr value added per worker (constant US\$) &  ~ \ \ & \textit{4.7} ~ \ \ & \textit{4.7} ~ \ \ \\ 
	\multicolumn{4}{l}{\textcolor{FAOblue}{\textbf{\large{Hunger dimensions}}}} \\ 
	 ~ Dietary energy supply (kcal/pc/day) &  ~ \ \ &  ~ \ \ &  ~ \ \ \\ 
	 ~ Average dietary energy supply adequacy (\%) &  ~ \ \ &  ~ \ \ &  ~ \ \ \\ 
	 ~ Dietary en supp, cereals/roots/tubers (\%) &  ~ \ \ &  ~ \ \ &  ~ \ \ \\ 
	 ~ Prevalence of undernourishment (\%) &  ~ \ \ &  ~ \ \ &  ~ \ \ \\ 
	 ~ GDP per capita (US\$, PPP) &  ~ \ \ &  ~ \ \ &  ~ \ \ \\ 
	 ~ Domestic food price volatility (index) &  ~ \ \ &  ~ \ \ &  ~ \ \ \\ 
	 ~ Cereal import dependency ratio (\%) & 21.2 ~ \ \ & 3.5 ~ \ \ & \textit{43.2} ~ \ \ \\ 
	 ~ Underweight, children under-5 (\%) & \textit{11.3} ~ \ \ & \textit{11.1} ~ \ \ & \textit{10.1} ~ \ \ \\ 
	 ~ Improved water source (\% pop) & 85.8 ~ \ \ & 88 ~ \ \ & \textit{90.1} ~ \ \ \\ 
	\multicolumn{4}{l}{\textcolor{FAOblue}{\textbf{\large{Food Supply}}}} \\ 
	 ~ Food production value, (2004-2006 mln I\$) & 3\,631 ~ \ \ & 6\,012 ~ \ \ & \textit{5\,121} ~ \ \ \\ 
	 ~ Agriculture, value added (\% GDP) & 34 ~ \ \ & 27 ~ \ \ & \textit{18} ~ \ \ \\ 
	 ~ Food exports (mln US\$)  & 403 ~ \ \ & 832 ~ \ \ & \textit{571} ~ \ \ \\ 
	 ~ Food imports (mln US\$)  & 534 ~ \ \ & 574 ~ \ \ & \textit{1\,878} ~ \ \ \\ 
	\multicolumn{4}{l}{\textit{\normalsize{Production indices (2004-06=100)}}} \\ 
	 ~ Net food & 58 ~ \ \ & 97 ~ \ \ & \textit{82} ~ \ \ \\ 
	 ~ Net crop & 66 ~ \ \ & 103 ~ \ \ & \textit{76} ~ \ \ \\ 
	 ~ Cereal & 73 ~ \ \ & 103 ~ \ \ & \textit{71} ~ \ \ \\ 
	 ~ Vegetable oils & 61 ~ \ \ & 98 ~ \ \ & \textit{78} ~ \ \ \\ 
	 ~ Roots and tubers & 71 ~ \ \ & 88 ~ \ \ & \textit{76} ~ \ \ \\ 
	 ~ Fruit and vegetables & 78 ~ \ \ & 91 ~ \ \ & \textit{87} ~ \ \ \\ 
	 ~ Sugar & 109 ~ \ \ & 122 ~ \ \ & \textit{25} ~ \ \ \\ 
	 ~ Livestock & 53 ~ \ \ & 79 ~ \ \ & \textit{84} ~ \ \ \\ 
	 ~ Milk & 58 ~ \ \ & 75 ~ \ \ & \textit{100} ~ \ \ \\ 
	 ~ Meat & 46 ~ \ \ & 80 ~ \ \ & \textit{78} ~ \ \ \\ 
	 ~ Fish  & 56 ~ \ \ & 89 ~ \ \ & \textit{51} ~ \ \ \\ 
	\multicolumn{4}{l}{\textit{\normalsize{Net trade (min US\$)}}} \\ 
	 ~ Cereals & -160 ~ \ \ & -56 ~ \ \ & \textit{-612} ~ \ \ \\ 
	 ~ Fruit and vegetables & 219 ~ \ \ & 169 ~ \ \ & \textit{-91} ~ \ \ \\ 
	 ~ Meat & -1 ~ \ \ & 0 ~ \ \ & \textit{-18} ~ \ \ \\ 
	 ~ Dairy products & -19 ~ \ \ & -41 ~ \ \ & \textit{-78} ~ \ \ \\ 
	 ~ Fish & -1 ~ \ \ & -25 ~ \ \ & \textit{-74} ~ \ \ \\ 
	\multicolumn{4}{l}{\textcolor{FAOblue}{\textbf{\large{Environment}}}} \\ 
	 ~ Forest area (\%) & 2 ~ \ \ & 2 ~ \ \ & \textit{3} ~ \ \ \\ 
	 ~ Renewable water res withdrawn (\% of total) &  ~ \ \ & \textit{88} ~ \ \ & 88 ~ \ \ \\ 
	 ~ Terrestrial protect areas (\% total land area)  & 0 ~ \ \ & 1 ~ \ \ & \textit{1} ~ \ \ \\ 
	 ~ Organic area (\% total agricultural area) &  ~ \ \ &  ~ \ \ & \textit{0} ~ \ \ \\ 
	 ~ Water withdrawal by agriculture (\% of total) &  ~ \ \ & \textit{88} ~ \ \ & 88 ~ \ \ \\ 
	 ~ Biofuel production (thousand kt of oil eq.) &  ~ \ \ &  ~ \ \ &  ~ \ \ \\ 
	 ~ Wood pellet prod. (min tonnes) &  ~ \ \ &  ~ \ \ &  ~ \ \ \\ 
	 ~ GHG emissions from ag (Co2 eq, gigagrams) & 5 ~ \ \ & 5 ~ \ \ & \textit{6} ~ \ \ \\ 
       \toprule
      \end{tabular}
      \clearpage
\CountryData{ Tajikistan }
      \rowcolors{1}{FAOblue!10}{white}
      \begin{tabular}{L{3.9cm} R{1cm} R{1cm} R{1cm}}
      \toprule
      \multicolumn{1}{c}{} & \multicolumn{1}{c}{ 1992 } & \multicolumn{1}{c}{ 2002 } & \multicolumn{1}{c}{ 2014 } \\
      \midrule
	\multicolumn{4}{l}{\textcolor{FAOblue}{\textbf{\large{The setting}}}} \\ 
	 ~ Population, total (mln) & 5.5 ~ \ \ & 6.4 ~ \ \ & 8.4 ~ \ \ \\ 
	 ~ Population, rural (\% total population) & 3.8 ~ \ \ & 4.7 ~ \ \ & 6.2 ~ \ \ \\ 
	 ~ Govt expenditure on ag (\% total outlays) &  ~ \ \ & 3.1 ~ \ \ & \textit{2} ~ \ \ \\ 
	 ~ Area harvested (mln ha) & 1 ~ \ \ & 1 ~ \ \ & 1 ~ \ \ \\ 
	 ~ Cropping intensity ratio (\%) & 0.2 ~ \ \ & 0.2 ~ \ \ &  ~ \ \ \\ 
	 ~ Water resources (m\textsuperscript{3}/person/year) & \textit{4} ~ \ \ & \textit{3} ~ \ \ & \textit{3} ~ \ \ \\ 
	 ~ Area equipped for irrigation (1000 ha) &  ~ \ \ &  ~ \ \ & \textit{742} ~ \ \ \\ 
	 ~ Area irrigated (\%) &  ~ \ \ &  ~ \ \ & \textit{90.9} ~ \ \ \\ 
	 ~ Employment in agriculture (\%) &  ~ \ \ & \textit{55.5} ~ \ \ &  ~ \ \ \\ 
	 ~ Employment in agriculture, female (\%) &  ~ \ \ & \textit{75.1} ~ \ \ &  ~ \ \ \\ 
	 ~ Fertilizers, Nitrogen (nutrients per ha) &  ~ \ \ & 0 ~ \ \ & \textit{10.4} ~ \ \ \\ 
	 ~ Fertilizers, Phosphate (nutrients per ha) &  ~ \ \ & 0 ~ \ \ & \textit{0} ~ \ \ \\ 
	 ~ Fertilizers, Potash (nutrients per ha) &  ~ \ \ & 0 ~ \ \ & \textit{0} ~ \ \ \\ 
	 ~ Energy consump, power irrigation (mln kWh) & \textit{0} ~ \ \ & 0 ~ \ \ & \textit{0} ~ \ \ \\ 
	 ~ Agr value added per worker (constant US\$) & 0.5 ~ \ \ & 0.6 ~ \ \ & \textit{1.2} ~ \ \ \\ 
	\multicolumn{4}{l}{\textcolor{FAOblue}{\textbf{\large{Hunger dimensions}}}} \\ 
	 ~ Dietary energy supply (kcal/pc/day) & 2\,068 ~ \ \ & 1\,953 ~ \ \ & 2\,140 ~ \ \ \\ 
	 ~ Average dietary energy supply adequacy (\%) & 98 ~ \ \ & 89 ~ \ \ & 96 ~ \ \ \\ 
	 ~ Dietary en supp, cereals/roots/tubers (\%) & 63 ~ \ \ & 70 ~ \ \ & \textit{62} ~ \ \ \\ 
	 ~ Prevalence of undernourishment (\%) & 28.1 ~ \ \ & 41 ~ \ \ & 34 ~ \ \ \\ 
	 ~ GDP per capita (US\$, PPP) & 2\,300 ~ \ \ & 1\,398 ~ \ \ & \textit{2\,432} ~ \ \ \\ 
	 ~ Domestic food price volatility (index) &  ~ \ \ &  ~ \ \ &  ~ \ \ \\ 
	 ~ Cereal import dependency ratio (\%) & 85.3 ~ \ \ & 38.3 ~ \ \ & \textit{43.7} ~ \ \ \\ 
	 ~ Underweight, children under-5 (\%) &  ~ \ \ & \textit{14.9} ~ \ \ & \textit{13.3} ~ \ \ \\ 
	 ~ Improved water source (\% pop) & \textit{57.7} ~ \ \ & 61.5 ~ \ \ & \textit{71.7} ~ \ \ \\ 
	\multicolumn{4}{l}{\textcolor{FAOblue}{\textbf{\large{Food Supply}}}} \\ 
	 ~ Food production value, (2004-2006 mln I\$) & 753 ~ \ \ & 695 ~ \ \ & \textit{1\,292} ~ \ \ \\ 
	 ~ Agriculture, value added (\% GDP) & 27 ~ \ \ & 25 ~ \ \ & \textit{27} ~ \ \ \\ 
	 ~ Food exports (mln US\$)  & 11 ~ \ \ & 23 ~ \ \ & \textit{41} ~ \ \ \\ 
	 ~ Food imports (mln US\$)  & 250 ~ \ \ & 118 ~ \ \ & \textit{572} ~ \ \ \\ 
	\multicolumn{4}{l}{\textit{\normalsize{Production indices (2004-06=100)}}} \\ 
	 ~ Net food & 91 ~ \ \ & 84 ~ \ \ & \textit{156} ~ \ \ \\ 
	 ~ Net crop & 76 ~ \ \ & 86 ~ \ \ & \textit{151} ~ \ \ \\ 
	 ~ Cereal & 29 ~ \ \ & 78 ~ \ \ & \textit{135} ~ \ \ \\ 
	 ~ Vegetable oils & 34 ~ \ \ & 107 ~ \ \ & \textit{93} ~ \ \ \\ 
	 ~ Roots and tubers & 30 ~ \ \ & 64 ~ \ \ & \textit{204} ~ \ \ \\ 
	 ~ Fruit and vegetables & 90 ~ \ \ & 75 ~ \ \ & \textit{202} ~ \ \ \\ 
	 ~ Sugar &  ~ \ \ &  ~ \ \ &  ~ \ \ \\ 
	 ~ Livestock & 114 ~ \ \ & 75 ~ \ \ & \textit{164} ~ \ \ \\ 
	 ~ Milk & 97 ~ \ \ & 82 ~ \ \ & \textit{159} ~ \ \ \\ 
	 ~ Meat & 127 ~ \ \ & 69 ~ \ \ & \textit{161} ~ \ \ \\ 
	 ~ Fish  & 942 ~ \ \ & 151 ~ \ \ & \textit{737} ~ \ \ \\ 
	\multicolumn{4}{l}{\textit{\normalsize{Net trade (min US\$)}}} \\ 
	 ~ Cereals &  ~ \ \ & -44 ~ \ \ & \textit{-316} ~ \ \ \\ 
	 ~ Fruit and vegetables & 5 ~ \ \ & 17 ~ \ \ & \textit{26} ~ \ \ \\ 
	 ~ Meat &  ~ \ \ & -28 ~ \ \ & \textit{-26} ~ \ \ \\ 
	 ~ Dairy products &  ~ \ \ & \textit{-2} ~ \ \ & \textit{-2} ~ \ \ \\ 
	 ~ Fish & \textit{0} ~ \ \ & 0 ~ \ \ & \textit{-3} ~ \ \ \\ 
	\multicolumn{4}{l}{\textcolor{FAOblue}{\textbf{\large{Environment}}}} \\ 
	 ~ Forest area (\%) & 3 ~ \ \ & 3 ~ \ \ & \textit{3} ~ \ \ \\ 
	 ~ Renewable water res withdrawn (\% of total) &  ~ \ \ &  ~ \ \ & 91 ~ \ \ \\ 
	 ~ Terrestrial protect areas (\% total land area)  & 4 ~ \ \ & 4 ~ \ \ & \textit{5} ~ \ \ \\ 
	 ~ Organic area (\% total agricultural area) &  ~ \ \ &  ~ \ \ & \textit{0} ~ \ \ \\ 
	 ~ Water withdrawal by agriculture (\% of total) &  ~ \ \ &  ~ \ \ & 91 ~ \ \ \\ 
	 ~ Biofuel production (thousand kt of oil eq.) &  ~ \ \ &  ~ \ \ &  ~ \ \ \\ 
	 ~ Wood pellet prod. (min tonnes) &  ~ \ \ &  ~ \ \ &  ~ \ \ \\ 
	 ~ GHG emissions from ag (Co2 eq, gigagrams) & 4 ~ \ \ & 3 ~ \ \ & \textit{5} ~ \ \ \\ 
       \toprule
      \end{tabular}
      \clearpage
\CountryData{ Tanzania }
      \rowcolors{1}{FAOblue!10}{white}
      \begin{tabular}{L{3.9cm} R{1cm} R{1cm} R{1cm}}
      \toprule
      \multicolumn{1}{c}{} & \multicolumn{1}{c}{ 1992 } & \multicolumn{1}{c}{ 2002 } & \multicolumn{1}{c}{ 2014 } \\
      \midrule
	\multicolumn{4}{l}{\textcolor{FAOblue}{\textbf{\large{The setting}}}} \\ 
	 ~ Population, total (mln) & 27.2 ~ \ \ & 35.8 ~ \ \ & 50.8 ~ \ \ \\ 
	 ~ Population, rural (\% total population) & 21.9 ~ \ \ & 27.6 ~ \ \ & 36.5 ~ \ \ \\ 
	 ~ Govt expenditure on ag (\% total outlays) &  ~ \ \ & 3 ~ \ \ & \textit{1.6} ~ \ \ \\ 
	 ~ Area harvested (mln ha) & 8 ~ \ \ & 7 ~ \ \ & 10 ~ \ \ \\ 
	 ~ Cropping intensity ratio (\%) & 0.2 ~ \ \ & 0.2 ~ \ \ &  ~ \ \ \\ 
	 ~ Water resources (m\textsuperscript{3}/person/year) & \textit{3} ~ \ \ & \textit{3} ~ \ \ & \textit{2} ~ \ \ \\ 
	 ~ Area equipped for irrigation (1000 ha) &  ~ \ \ &  ~ \ \ & \textit{184} ~ \ \ \\ 
	 ~ Area irrigated (\%) &  ~ \ \ &  ~ \ \ &  ~ \ \ \\ 
	 ~ Employment in agriculture (\%) & \textit{84.2} ~ \ \ & \textit{82.1} ~ \ \ & \textit{76.5} ~ \ \ \\ 
	 ~ Employment in agriculture, female (\%) & \textit{90.4} ~ \ \ & \textit{84} ~ \ \ & \textit{80} ~ \ \ \\ 
	 ~ Fertilizers, Nitrogen (nutrients per ha) &  ~ \ \ & 0.6 ~ \ \ & \textit{1.2} ~ \ \ \\ 
	 ~ Fertilizers, Phosphate (nutrients per ha) &  ~ \ \ & 0.2 ~ \ \ & \textit{0.2} ~ \ \ \\ 
	 ~ Fertilizers, Potash (nutrients per ha) &  ~ \ \ & 0.1 ~ \ \ & \textit{0.2} ~ \ \ \\ 
	 ~ Energy consump, power irrigation (mln kWh) &  ~ \ \ &  ~ \ \ &  ~ \ \ \\ 
	 ~ Agr value added per worker (constant US\$) & 0.2 ~ \ \ & 0.3 ~ \ \ & \textit{0.3} ~ \ \ \\ 
	\multicolumn{4}{l}{\textcolor{FAOblue}{\textbf{\large{Hunger dimensions}}}} \\ 
	 ~ Dietary energy supply (kcal/pc/day) & 2\,157 ~ \ \ & 2\,041 ~ \ \ & 2\,203 ~ \ \ \\ 
	 ~ Average dietary energy supply adequacy (\%) & 103 ~ \ \ & 97 ~ \ \ & 105 ~ \ \ \\ 
	 ~ Dietary en supp, cereals/roots/tubers (\%) & 69 ~ \ \ & 62 ~ \ \ & \textit{56} ~ \ \ \\ 
	 ~ Prevalence of undernourishment (\%) & 24.8 ~ \ \ & 37.8 ~ \ \ & 32.1 ~ \ \ \\ 
	 ~ GDP per capita (US\$, PPP) & 1\,403 ~ \ \ & 1\,594 ~ \ \ & \textit{2\,365} ~ \ \ \\ 
	 ~ Domestic food price volatility (index) &  ~ \ \ & 7.6 ~ \ \ & 4.8 ~ \ \ \\ 
	 ~ Cereal import dependency ratio (\%) & 4.9 ~ \ \ & 8.7 ~ \ \ & \textit{13.2} ~ \ \ \\ 
	 ~ Underweight, children under-5 (\%) & \textit{25.1} ~ \ \ & \textit{16.7} ~ \ \ & \textit{13.6} ~ \ \ \\ 
	 ~ Improved water source (\% pop) & 54.9 ~ \ \ & 54.2 ~ \ \ & \textit{53.2} ~ \ \ \\ 
	\multicolumn{4}{l}{\textcolor{FAOblue}{\textbf{\large{Food Supply}}}} \\ 
	 ~ Food production value, (2004-2006 mln I\$) & 3\,601 ~ \ \ & 5\,287 ~ \ \ & \textit{8\,650} ~ \ \ \\ 
	 ~ Agriculture, value added (\% GDP) & 48 ~ \ \ & 32 ~ \ \ & \textit{34} ~ \ \ \\ 
	 ~ Food exports (mln US\$)  & 48 ~ \ \ & 144 ~ \ \ & \textit{726} ~ \ \ \\ 
	 ~ Food imports (mln US\$)  & 116 ~ \ \ & 234 ~ \ \ & \textit{1\,115} ~ \ \ \\ 
	\multicolumn{4}{l}{\textit{\normalsize{Production indices (2004-06=100)}}} \\ 
	 ~ Net food & 64 ~ \ \ & 94 ~ \ \ & \textit{154} ~ \ \ \\ 
	 ~ Net crop & 61 ~ \ \ & 95 ~ \ \ & \textit{157} ~ \ \ \\ 
	 ~ Cereal & 56 ~ \ \ & 104 ~ \ \ & \textit{156} ~ \ \ \\ 
	 ~ Vegetable oils & 37 ~ \ \ & 87 ~ \ \ & \textit{336} ~ \ \ \\ 
	 ~ Roots and tubers & 107 ~ \ \ & 97 ~ \ \ & \textit{133} ~ \ \ \\ 
	 ~ Fruit and vegetables & 43 ~ \ \ & 86 ~ \ \ & \textit{133} ~ \ \ \\ 
	 ~ Sugar & 62 ~ \ \ & 78 ~ \ \ & \textit{130} ~ \ \ \\ 
	 ~ Livestock & 76 ~ \ \ & 85 ~ \ \ & \textit{143} ~ \ \ \\ 
	 ~ Milk & 44 ~ \ \ & 70 ~ \ \ & \textit{141} ~ \ \ \\ 
	 ~ Meat & 92 ~ \ \ & 92 ~ \ \ & \textit{150} ~ \ \ \\ 
	 ~ Fish  & 87 ~ \ \ & 90 ~ \ \ & \textit{109} ~ \ \ \\ 
	\multicolumn{4}{l}{\textit{\normalsize{Net trade (min US\$)}}} \\ 
	 ~ Cereals & -49 ~ \ \ & -65 ~ \ \ & \textit{-370} ~ \ \ \\ 
	 ~ Fruit and vegetables & 28 ~ \ \ & 54 ~ \ \ & \textit{313} ~ \ \ \\ 
	 ~ Meat & 0 ~ \ \ & -1 ~ \ \ & \textit{-7} ~ \ \ \\ 
	 ~ Dairy products & -8 ~ \ \ & -3 ~ \ \ & \textit{-14} ~ \ \ \\ 
	 ~ Fish & 6 ~ \ \ & 117 ~ \ \ & \textit{101} ~ \ \ \\ 
	\multicolumn{4}{l}{\textcolor{FAOblue}{\textbf{\large{Environment}}}} \\ 
	 ~ Forest area (\%) & 46 ~ \ \ & 41 ~ \ \ & \textit{37} ~ \ \ \\ 
	 ~ Renewable water res withdrawn (\% of total) &  ~ \ \ & 89 ~ \ \ & 89 ~ \ \ \\ 
	 ~ Terrestrial protect areas (\% total land area)  & 27 ~ \ \ & 27 ~ \ \ & \textit{32} ~ \ \ \\ 
	 ~ Organic area (\% total agricultural area) &  ~ \ \ & \textit{0} ~ \ \ & \textit{0} ~ \ \ \\ 
	 ~ Water withdrawal by agriculture (\% of total) &  ~ \ \ & 89 ~ \ \ & 89 ~ \ \ \\ 
	 ~ Biofuel production (thousand kt of oil eq.) & 3 ~ \ \ & 5 ~ \ \ & \textit{7} ~ \ \ \\ 
	 ~ Wood pellet prod. (min tonnes) &  ~ \ \ &  ~ \ \ &  ~ \ \ \\ 
	 ~ GHG emissions from ag (Co2 eq, gigagrams) & 134 ~ \ \ & 138 ~ \ \ & \textit{147} ~ \ \ \\ 
       \toprule
      \end{tabular}
      \clearpage
\CountryData{ Thailand }
      \rowcolors{1}{FAOblue!10}{white}
      \begin{tabular}{L{3.9cm} R{1cm} R{1cm} R{1cm}}
      \toprule
      \multicolumn{1}{c}{} & \multicolumn{1}{c}{ 1992 } & \multicolumn{1}{c}{ 2002 } & \multicolumn{1}{c}{ 2014 } \\
      \midrule
	\multicolumn{4}{l}{\textcolor{FAOblue}{\textbf{\large{The setting}}}} \\ 
	 ~ Population, total (mln) & 57.7 ~ \ \ & 63.8 ~ \ \ & 67.2 ~ \ \ \\ 
	 ~ Population, rural (\% total population) & 40.5 ~ \ \ & 43.7 ~ \ \ & 43.5 ~ \ \ \\ 
	 ~ Govt expenditure on ag (\% total outlays) &  ~ \ \ &  ~ \ \ &  ~ \ \ \\ 
	 ~ Area harvested (mln ha) & 47 ~ \ \ & 60 ~ \ \ & 100 ~ \ \ \\ 
	 ~ Cropping intensity ratio (\%) & 2.2 ~ \ \ & 3 ~ \ \ &  ~ \ \ \\ 
	 ~ Water resources (m\textsuperscript{3}/person/year) & \textit{8} ~ \ \ & \textit{7} ~ \ \ & \textit{7} ~ \ \ \\ 
	 ~ Area equipped for irrigation (1000 ha) &  ~ \ \ &  ~ \ \ & \textit{6\,415} ~ \ \ \\ 
	 ~ Area irrigated (\%) &  ~ \ \ &  ~ \ \ & \textit{78.9} ~ \ \ \\ 
	 ~ Employment in agriculture (\%) & 60.8 ~ \ \ & 46.1 ~ \ \ & \textit{39.6} ~ \ \ \\ 
	 ~ Employment in agriculture, female (\%) & 62.1 ~ \ \ & 44 ~ \ \ & \textit{37.8} ~ \ \ \\ 
	 ~ Fertilizers, Nitrogen (nutrients per ha) &  ~ \ \ & 51.7 ~ \ \ & \textit{73.1} ~ \ \ \\ 
	 ~ Fertilizers, Phosphate (nutrients per ha) &  ~ \ \ & 20.7 ~ \ \ & \textit{22.9} ~ \ \ \\ 
	 ~ Fertilizers, Potash (nutrients per ha) &  ~ \ \ & 13.9 ~ \ \ & \textit{20.1} ~ \ \ \\ 
	 ~ Energy consump, power irrigation (mln kWh) &  ~ \ \ &  ~ \ \ &  ~ \ \ \\ 
	 ~ Agr value added per worker (constant US\$) & 0.7 ~ \ \ & 0.8 ~ \ \ & \textit{1.2} ~ \ \ \\ 
	\multicolumn{4}{l}{\textcolor{FAOblue}{\textbf{\large{Hunger dimensions}}}} \\ 
	 ~ Dietary energy supply (kcal/pc/day) & 2\,265 ~ \ \ & 2\,617 ~ \ \ & 2\,821 ~ \ \ \\ 
	 ~ Average dietary energy supply adequacy (\%) & 95 ~ \ \ & 108 ~ \ \ & 115 ~ \ \ \\ 
	 ~ Dietary en supp, cereals/roots/tubers (\%) & 55 ~ \ \ & 50 ~ \ \ & \textit{49} ~ \ \ \\ 
	 ~ Prevalence of undernourishment (\%) & 33.2 ~ \ \ & 17.5 ~ \ \ & 7.9 ~ \ \ \\ 
	 ~ GDP per capita (US\$, PPP) & 7\,333 ~ \ \ & 9\,399 ~ \ \ & \textit{13\,932} ~ \ \ \\ 
	 ~ Domestic food price volatility (index) &  ~ \ \ & 3.7 ~ \ \ & 2.8 ~ \ \ \\ 
	 ~ Cereal import dependency ratio (\%) & -39.2 ~ \ \ & -44 ~ \ \ & \textit{-41.6} ~ \ \ \\ 
	 ~ Underweight, children under-5 (\%) & \textit{15.4} ~ \ \ & \textit{15.4} ~ \ \ & \textit{9.2} ~ \ \ \\ 
	 ~ Improved water source (\% pop) & 87.5 ~ \ \ & 92.8 ~ \ \ & \textit{95.8} ~ \ \ \\ 
	\multicolumn{4}{l}{\textcolor{FAOblue}{\textbf{\large{Food Supply}}}} \\ 
	 ~ Food production value, (2004-2006 mln I\$) & 17\,497 ~ \ \ & 21\,806 ~ \ \ & \textit{28\,642} ~ \ \ \\ 
	 ~ Agriculture, value added (\% GDP) & 12 ~ \ \ & 9 ~ \ \ & \textit{12} ~ \ \ \\ 
	 ~ Food exports (mln US\$)  & 4\,022 ~ \ \ & 5\,538 ~ \ \ & \textit{18\,764} ~ \ \ \\ 
	 ~ Food imports (mln US\$)  & 736 ~ \ \ & 1\,360 ~ \ \ & \textit{6\,443} ~ \ \ \\ 
	\multicolumn{4}{l}{\textit{\normalsize{Production indices (2004-06=100)}}} \\ 
	 ~ Net food & 78 ~ \ \ & 98 ~ \ \ & \textit{128} ~ \ \ \\ 
	 ~ Net crop & 70 ~ \ \ & 95 ~ \ \ & \textit{129} ~ \ \ \\ 
	 ~ Cereal & 69 ~ \ \ & 95 ~ \ \ & \textit{123} ~ \ \ \\ 
	 ~ Vegetable oils & 41 ~ \ \ & 76 ~ \ \ & \textit{197} ~ \ \ \\ 
	 ~ Roots and tubers & 98 ~ \ \ & 83 ~ \ \ & \textit{148} ~ \ \ \\ 
	 ~ Fruit and vegetables & 62 ~ \ \ & 101 ~ \ \ & \textit{111} ~ \ \ \\ 
	 ~ Sugar & 88 ~ \ \ & 111 ~ \ \ & \textit{185} ~ \ \ \\ 
	 ~ Livestock & 100 ~ \ \ & 101 ~ \ \ & \textit{130} ~ \ \ \\ 
	 ~ Milk & 16 ~ \ \ & 78 ~ \ \ & \textit{130} ~ \ \ \\ 
	 ~ Meat & 107 ~ \ \ & 102 ~ \ \ & \textit{129} ~ \ \ \\ 
	 ~ Fish  & 79 ~ \ \ & 93 ~ \ \ & \textit{71} ~ \ \ \\ 
	\multicolumn{4}{l}{\textit{\normalsize{Net trade (min US\$)}}} \\ 
	 ~ Cereals & 1\,305 ~ \ \ & 1\,556 ~ \ \ & \textit{3\,892} ~ \ \ \\ 
	 ~ Fruit and vegetables & 1\,804 ~ \ \ & 1\,136 ~ \ \ & \textit{3\,039} ~ \ \ \\ 
	 ~ Meat & 460 ~ \ \ & 1\,036 ~ \ \ & \textit{2\,522} ~ \ \ \\ 
	 ~ Dairy products & -205 ~ \ \ & -103 ~ \ \ & \textit{-448} ~ \ \ \\ 
	 ~ Fish & 2\,130 ~ \ \ & 2\,655 ~ \ \ & \textit{5\,022} ~ \ \ \\ 
	\multicolumn{4}{l}{\textcolor{FAOblue}{\textbf{\large{Environment}}}} \\ 
	 ~ Forest area (\%) & 38 ~ \ \ & 37 ~ \ \ & \textit{37} ~ \ \ \\ 
	 ~ Renewable water res withdrawn (\% of total) &  ~ \ \ &  ~ \ \ & 90 ~ \ \ \\ 
	 ~ Terrestrial protect areas (\% total land area)  & 16 ~ \ \ & 20 ~ \ \ & \textit{19} ~ \ \ \\ 
	 ~ Organic area (\% total agricultural area) &  ~ \ \ & \textit{0} ~ \ \ & \textit{0} ~ \ \ \\ 
	 ~ Water withdrawal by agriculture (\% of total) &  ~ \ \ &  ~ \ \ & 90 ~ \ \ \\ 
	 ~ Biofuel production (thousand kt of oil eq.) & 103 ~ \ \ & 166 ~ \ \ & \textit{13\,727} ~ \ \ \\ 
	 ~ Wood pellet prod. (min tonnes) &  ~ \ \ &  ~ \ \ & \textit{20} ~ \ \ \\ 
	 ~ GHG emissions from ag (Co2 eq, gigagrams) & 71 ~ \ \ & 61 ~ \ \ & \textit{70} ~ \ \ \\ 
       \toprule
      \end{tabular}
      \clearpage
\CountryData{ Timor-Leste }
      \rowcolors{1}{FAOblue!10}{white}
      \begin{tabular}{L{3.9cm} R{1cm} R{1cm} R{1cm}}
      \toprule
      \multicolumn{1}{c}{} & \multicolumn{1}{c}{ 1992 } & \multicolumn{1}{c}{ 2002 } & \multicolumn{1}{c}{ 2014 } \\
      \midrule
	\multicolumn{4}{l}{\textcolor{FAOblue}{\textbf{\large{The setting}}}} \\ 
	 ~ Population, total (mln) & 0.8 ~ \ \ & 0.9 ~ \ \ & 1.2 ~ \ \ \\ 
	 ~ Population, rural (\% total population) & 0.6 ~ \ \ & 0.7 ~ \ \ & 0.8 ~ \ \ \\ 
	 ~ Govt expenditure on ag (\% total outlays) &  ~ \ \ &  ~ \ \ &  ~ \ \ \\ 
	 ~ Area harvested (mln ha) & 0 ~ \ \ & 0 ~ \ \ & 0 ~ \ \ \\ 
	 ~ Cropping intensity ratio (\%) & 0.9 ~ \ \ & 0.6 ~ \ \ &  ~ \ \ \\ 
	 ~ Water resources (m\textsuperscript{3}/person/year) & \textit{10} ~ \ \ & \textit{9} ~ \ \ & \textit{7} ~ \ \ \\ 
	 ~ Area equipped for irrigation (1000 ha) &  ~ \ \ &  ~ \ \ & \textit{35} ~ \ \ \\ 
	 ~ Area irrigated (\%) &  ~ \ \ & 83.4 ~ \ \ &  ~ \ \ \\ 
	 ~ Employment in agriculture (\%) &  ~ \ \ &  ~ \ \ & \textit{50.6} ~ \ \ \\ 
	 ~ Employment in agriculture, female (\%) &  ~ \ \ &  ~ \ \ & \textit{50.2} ~ \ \ \\ 
	 ~ Fertilizers, Nitrogen (nutrients per ha) &  ~ \ \ &  ~ \ \ &  ~ \ \ \\ 
	 ~ Fertilizers, Phosphate (nutrients per ha) &  ~ \ \ &  ~ \ \ &  ~ \ \ \\ 
	 ~ Fertilizers, Potash (nutrients per ha) &  ~ \ \ &  ~ \ \ &  ~ \ \ \\ 
	 ~ Energy consump, power irrigation (mln kWh) &  ~ \ \ &  ~ \ \ &  ~ \ \ \\ 
	 ~ Agr value added per worker (constant US\$) &  ~ \ \ & 0.5 ~ \ \ & \textit{0.4} ~ \ \ \\ 
	\multicolumn{4}{l}{\textcolor{FAOblue}{\textbf{\large{Hunger dimensions}}}} \\ 
	 ~ Dietary energy supply (kcal/pc/day) & 1\,985 ~ \ \ & 2\,016 ~ \ \ & 2\,179 ~ \ \ \\ 
	 ~ Average dietary energy supply adequacy (\%) & 95 ~ \ \ & 101 ~ \ \ & 106 ~ \ \ \\ 
	 ~ Dietary en supp, cereals/roots/tubers (\%) & 74 ~ \ \ & 72 ~ \ \ & \textit{70} ~ \ \ \\ 
	 ~ Prevalence of undernourishment (\%) & 40.7 ~ \ \ & 38.1 ~ \ \ & 27.9 ~ \ \ \\ 
	 ~ GDP per capita (US\$, PPP) &  ~ \ \ & 1\,363 ~ \ \ & \textit{2\,040} ~ \ \ \\ 
	 ~ Domestic food price volatility (index) &  ~ \ \ &  ~ \ \ &  ~ \ \ \\ 
	 ~ Cereal import dependency ratio (\%) & 15.3 ~ \ \ & 29.2 ~ \ \ & \textit{10.6} ~ \ \ \\ 
	 ~ Underweight, children under-5 (\%) &  ~ \ \ & 40.6 ~ \ \ & \textit{45.3} ~ \ \ \\ 
	 ~ Improved water source (\% pop) & \textit{52.8} ~ \ \ & 56.9 ~ \ \ & \textit{70.5} ~ \ \ \\ 
	\multicolumn{4}{l}{\textcolor{FAOblue}{\textbf{\large{Food Supply}}}} \\ 
	 ~ Food production value, (2004-2006 mln I\$) & 103 ~ \ \ & 102 ~ \ \ & \textit{122} ~ \ \ \\ 
	 ~ Agriculture, value added (\% GDP) &  ~ \ \ & 27 ~ \ \ & \textit{18} ~ \ \ \\ 
	 ~ Food exports (mln US\$)  & 0 ~ \ \ & 4 ~ \ \ & \textit{0} ~ \ \ \\ 
	 ~ Food imports (mln US\$)  & 23 ~ \ \ & 38 ~ \ \ & \textit{73} ~ \ \ \\ 
	\multicolumn{4}{l}{\textit{\normalsize{Production indices (2004-06=100)}}} \\ 
	 ~ Net food & 102 ~ \ \ & 101 ~ \ \ & \textit{121} ~ \ \ \\ 
	 ~ Net crop & 88 ~ \ \ & 97 ~ \ \ & \textit{114} ~ \ \ \\ 
	 ~ Cereal & 96 ~ \ \ & 102 ~ \ \ & \textit{140} ~ \ \ \\ 
	 ~ Vegetable oils & 73 ~ \ \ & 86 ~ \ \ & \textit{96} ~ \ \ \\ 
	 ~ Roots and tubers & 116 ~ \ \ & 103 ~ \ \ & \textit{60} ~ \ \ \\ 
	 ~ Fruit and vegetables & 87 ~ \ \ & 81 ~ \ \ & \textit{167} ~ \ \ \\ 
	 ~ Sugar &  ~ \ \ &  ~ \ \ &  ~ \ \ \\ 
	 ~ Livestock & 108 ~ \ \ & 107 ~ \ \ & \textit{112} ~ \ \ \\ 
	 ~ Milk & 88 ~ \ \ & 73 ~ \ \ & \textit{114} ~ \ \ \\ 
	 ~ Meat & 109 ~ \ \ & 107 ~ \ \ & \textit{112} ~ \ \ \\ 
	 ~ Fish  & 0 ~ \ \ & 101 ~ \ \ & \textit{89} ~ \ \ \\ 
	\multicolumn{4}{l}{\textit{\normalsize{Net trade (min US\$)}}} \\ 
	 ~ Cereals &  ~ \ \ &  ~ \ \ &  ~ \ \ \\ 
	 ~ Fruit and vegetables & -6 ~ \ \ & -10 ~ \ \ & \textit{-6} ~ \ \ \\ 
	 ~ Meat &  ~ \ \ &  ~ \ \ &  ~ \ \ \\ 
	 ~ Dairy products &  ~ \ \ &  ~ \ \ &  ~ \ \ \\ 
	 ~ Fish &  ~ \ \ & \textit{0} ~ \ \ & \textit{-2} ~ \ \ \\ 
	\multicolumn{4}{l}{\textcolor{FAOblue}{\textbf{\large{Environment}}}} \\ 
	 ~ Forest area (\%) & 63 ~ \ \ & 56 ~ \ \ & \textit{48} ~ \ \ \\ 
	 ~ Renewable water res withdrawn (\% of total) &  ~ \ \ & \textit{91} ~ \ \ & 91 ~ \ \ \\ 
	 ~ Terrestrial protect areas (\% total land area)  &  ~ \ \ & 5 ~ \ \ & \textit{9} ~ \ \ \\ 
	 ~ Organic area (\% total agricultural area) &  ~ \ \ & \textit{6} ~ \ \ & \textit{6} ~ \ \ \\ 
	 ~ Water withdrawal by agriculture (\% of total) &  ~ \ \ & \textit{91} ~ \ \ & 91 ~ \ \ \\ 
	 ~ Biofuel production (thousand kt of oil eq.) &  ~ \ \ &  ~ \ \ &  ~ \ \ \\ 
	 ~ Wood pellet prod. (min tonnes) &  ~ \ \ &  ~ \ \ &  ~ \ \ \\ 
	 ~ GHG emissions from ag (Co2 eq, gigagrams) & 3 ~ \ \ & 6 ~ \ \ & \textit{6} ~ \ \ \\ 
       \toprule
      \end{tabular}
      \clearpage
\CountryData{ Togo }
      \rowcolors{1}{FAOblue!10}{white}
      \begin{tabular}{L{3.9cm} R{1cm} R{1cm} R{1cm}}
      \toprule
      \multicolumn{1}{c}{} & \multicolumn{1}{c}{ 1992 } & \multicolumn{1}{c}{ 2002 } & \multicolumn{1}{c}{ 2014 } \\
      \midrule
	\multicolumn{4}{l}{\textcolor{FAOblue}{\textbf{\large{The setting}}}} \\ 
	 ~ Population, total (mln) & 4 ~ \ \ & 5.1 ~ \ \ & 7 ~ \ \ \\ 
	 ~ Population, rural (\% total population) & 2.8 ~ \ \ & 3.4 ~ \ \ & 4.2 ~ \ \ \\ 
	 ~ Govt expenditure on ag (\% total outlays) &  ~ \ \ &  ~ \ \ &  ~ \ \ \\ 
	 ~ Area harvested (mln ha) & 1 ~ \ \ & 1 ~ \ \ & 2 ~ \ \ \\ 
	 ~ Cropping intensity ratio (\%) & 0.3 ~ \ \ & 0.4 ~ \ \ &  ~ \ \ \\ 
	 ~ Water resources (m\textsuperscript{3}/person/year) & \textit{4} ~ \ \ & \textit{3} ~ \ \ & \textit{2} ~ \ \ \\ 
	 ~ Area equipped for irrigation (1000 ha) &  ~ \ \ &  ~ \ \ & \textit{7} ~ \ \ \\ 
	 ~ Area irrigated (\%) &  ~ \ \ & \textit{85.6} ~ \ \ &  ~ \ \ \\ 
	 ~ Employment in agriculture (\%) &  ~ \ \ &  ~ \ \ & \textit{54.1} ~ \ \ \\ 
	 ~ Employment in agriculture, female (\%) &  ~ \ \ &  ~ \ \ & \textit{48.2} ~ \ \ \\ 
	 ~ Fertilizers, Nitrogen (nutrients per ha) &  ~ \ \ & 1.4 ~ \ \ & \textit{1.8} ~ \ \ \\ 
	 ~ Fertilizers, Phosphate (nutrients per ha) &  ~ \ \ & 1.1 ~ \ \ & \textit{1.4} ~ \ \ \\ 
	 ~ Fertilizers, Potash (nutrients per ha) &  ~ \ \ & 1.1 ~ \ \ & \textit{0.3} ~ \ \ \\ 
	 ~ Energy consump, power irrigation (mln kWh) & 1 ~ \ \ & 1 ~ \ \ & \textit{1} ~ \ \ \\ 
	 ~ Agr value added per worker (constant US\$) & 0.6 ~ \ \ & 0.6 ~ \ \ & \textit{0.6} ~ \ \ \\ 
	\multicolumn{4}{l}{\textcolor{FAOblue}{\textbf{\large{Hunger dimensions}}}} \\ 
	 ~ Dietary energy supply (kcal/pc/day) & 1\,929 ~ \ \ & 2\,171 ~ \ \ & 2\,600 ~ \ \ \\ 
	 ~ Average dietary energy supply adequacy (\%) & 91 ~ \ \ & 101 ~ \ \ & 120 ~ \ \ \\ 
	 ~ Dietary en supp, cereals/roots/tubers (\%) & 75 ~ \ \ & 74 ~ \ \ & \textit{72} ~ \ \ \\ 
	 ~ Prevalence of undernourishment (\%) & 41.3 ~ \ \ & 28.4 ~ \ \ & 12.9 ~ \ \ \\ 
	 ~ GDP per capita (US\$, PPP) & 1\,223 ~ \ \ & 1\,207 ~ \ \ & \textit{1\,346} ~ \ \ \\ 
	 ~ Domestic food price volatility (index) &  ~ \ \ & 26.1 ~ \ \ & 15.5 ~ \ \ \\ 
	 ~ Cereal import dependency ratio (\%) & 12.8 ~ \ \ & 16.2 ~ \ \ & \textit{14} ~ \ \ \\ 
	 ~ Underweight, children under-5 (\%) &  ~ \ \ & \textit{23.2} ~ \ \ & \textit{16.5} ~ \ \ \\ 
	 ~ Improved water source (\% pop) & 49.3 ~ \ \ & 54.4 ~ \ \ & \textit{60} ~ \ \ \\ 
	\multicolumn{4}{l}{\textcolor{FAOblue}{\textbf{\large{Food Supply}}}} \\ 
	 ~ Food production value, (2004-2006 mln I\$) & 379 ~ \ \ & 549 ~ \ \ & \textit{763} ~ \ \ \\ 
	 ~ Agriculture, value added (\% GDP) & 35 ~ \ \ & 38 ~ \ \ & \textit{31} ~ \ \ \\ 
	 ~ Food exports (mln US\$)  & 47 ~ \ \ & 38 ~ \ \ & \textit{147} ~ \ \ \\ 
	 ~ Food imports (mln US\$)  & 74 ~ \ \ & 76 ~ \ \ & \textit{185} ~ \ \ \\ 
	\multicolumn{4}{l}{\textit{\normalsize{Production indices (2004-06=100)}}} \\ 
	 ~ Net food & 62 ~ \ \ & 90 ~ \ \ & \textit{125} ~ \ \ \\ 
	 ~ Net crop & 67 ~ \ \ & 101 ~ \ \ & \textit{117} ~ \ \ \\ 
	 ~ Cereal & 58 ~ \ \ & 96 ~ \ \ & \textit{151} ~ \ \ \\ 
	 ~ Vegetable oils & 90 ~ \ \ & 109 ~ \ \ & \textit{120} ~ \ \ \\ 
	 ~ Roots and tubers & 60 ~ \ \ & 99 ~ \ \ & \textit{115} ~ \ \ \\ 
	 ~ Fruit and vegetables & 117 ~ \ \ & 102 ~ \ \ & \textit{101} ~ \ \ \\ 
	 ~ Sugar &  ~ \ \ &  ~ \ \ &  ~ \ \ \\ 
	 ~ Livestock & 61 ~ \ \ & 85 ~ \ \ & \textit{144} ~ \ \ \\ 
	 ~ Milk & 60 ~ \ \ & 96 ~ \ \ & \textit{105} ~ \ \ \\ 
	 ~ Meat & 60 ~ \ \ & 85 ~ \ \ & \textit{146} ~ \ \ \\ 
	 ~ Fish  & 40 ~ \ \ & 78 ~ \ \ & \textit{74} ~ \ \ \\ 
	\multicolumn{4}{l}{\textit{\normalsize{Net trade (min US\$)}}} \\ 
	 ~ Cereals & -19 ~ \ \ & -27 ~ \ \ & \textit{-47} ~ \ \ \\ 
	 ~ Fruit and vegetables & -5 ~ \ \ & -2 ~ \ \ & \textit{-7} ~ \ \ \\ 
	 ~ Meat & -2 ~ \ \ & -3 ~ \ \ & \textit{-12} ~ \ \ \\ 
	 ~ Dairy products & -5 ~ \ \ & -1 ~ \ \ & \textit{-6} ~ \ \ \\ 
	 ~ Fish & -16 ~ \ \ & 3 ~ \ \ & \textit{-32} ~ \ \ \\ 
	\multicolumn{4}{l}{\textcolor{FAOblue}{\textbf{\large{Environment}}}} \\ 
	 ~ Forest area (\%) & 12 ~ \ \ & 8 ~ \ \ & \textit{5} ~ \ \ \\ 
	 ~ Renewable water res withdrawn (\% of total) &  ~ \ \ & 45 ~ \ \ & 45 ~ \ \ \\ 
	 ~ Terrestrial protect areas (\% total land area)  & 11 ~ \ \ & 11 ~ \ \ & \textit{25} ~ \ \ \\ 
	 ~ Organic area (\% total agricultural area) &  ~ \ \ & \textit{0} ~ \ \ & \textit{0} ~ \ \ \\ 
	 ~ Water withdrawal by agriculture (\% of total) &  ~ \ \ & 45 ~ \ \ & 45 ~ \ \ \\ 
	 ~ Biofuel production (thousand kt of oil eq.) & 1 ~ \ \ & 1 ~ \ \ & \textit{1} ~ \ \ \\ 
	 ~ Wood pellet prod. (min tonnes) &  ~ \ \ &  ~ \ \ &  ~ \ \ \\ 
	 ~ GHG emissions from ag (Co2 eq, gigagrams) & 10 ~ \ \ & 10 ~ \ \ & \textit{2} ~ \ \ \\ 
       \toprule
      \end{tabular}
      \clearpage
\CountryData{ Trinidad and Tobago }
      \rowcolors{1}{FAOblue!10}{white}
      \begin{tabular}{L{3.9cm} R{1cm} R{1cm} R{1cm}}
      \toprule
      \multicolumn{1}{c}{} & \multicolumn{1}{c}{ 1992 } & \multicolumn{1}{c}{ 2002 } & \multicolumn{1}{c}{ 2014 } \\
      \midrule
	\multicolumn{4}{l}{\textcolor{FAOblue}{\textbf{\large{The setting}}}} \\ 
	 ~ Population, total (mln) & 1.2 ~ \ \ & 1.3 ~ \ \ & 1.3 ~ \ \ \\ 
	 ~ Population, rural (\% total population) & 1.1 ~ \ \ & 1.1 ~ \ \ & 1.1 ~ \ \ \\ 
	 ~ Govt expenditure on ag (\% total outlays) &  ~ \ \ &  ~ \ \ &  ~ \ \ \\ 
	 ~ Area harvested (mln ha) & 1 ~ \ \ & 1 ~ \ \ & 0 ~ \ \ \\ 
	 ~ Cropping intensity ratio (\%) & 16 ~ \ \ & 22.3 ~ \ \ &  ~ \ \ \\ 
	 ~ Water resources (m\textsuperscript{3}/person/year) & \textit{3} ~ \ \ & \textit{3} ~ \ \ & \textit{3} ~ \ \ \\ 
	 ~ Area equipped for irrigation (1000 ha) &  ~ \ \ &  ~ \ \ & \textit{7} ~ \ \ \\ 
	 ~ Area irrigated (\%) &  ~ \ \ & \textit{85} ~ \ \ &  ~ \ \ \\ 
	 ~ Employment in agriculture (\%) & 11.5 ~ \ \ & 6.9 ~ \ \ & \textit{3.8} ~ \ \ \\ 
	 ~ Employment in agriculture, female (\%) & 5.5 ~ \ \ & 2 ~ \ \ & \textit{1.8} ~ \ \ \\ 
	 ~ Fertilizers, Nitrogen (nutrients per ha) &  ~ \ \ & 377.4 ~ \ \ & \textit{234.6} ~ \ \ \\ 
	 ~ Fertilizers, Phosphate (nutrients per ha) &  ~ \ \ & 8.8 ~ \ \ & \textit{22.4} ~ \ \ \\ 
	 ~ Fertilizers, Potash (nutrients per ha) &  ~ \ \ & 12 ~ \ \ & \textit{29.8} ~ \ \ \\ 
	 ~ Energy consump, power irrigation (mln kWh) & 2 ~ \ \ & 2 ~ \ \ & \textit{2} ~ \ \ \\ 
	 ~ Agr value added per worker (constant US\$) & 2 ~ \ \ & 3 ~ \ \ & \textit{1.9} ~ \ \ \\ 
	\multicolumn{4}{l}{\textcolor{FAOblue}{\textbf{\large{Hunger dimensions}}}} \\ 
	 ~ Dietary energy supply (kcal/pc/day) & 2\,594 ~ \ \ & 2\,792 ~ \ \ & 2\,965 ~ \ \ \\ 
	 ~ Average dietary energy supply adequacy (\%) & 111 ~ \ \ & 115 ~ \ \ & 122 ~ \ \ \\ 
	 ~ Dietary en supp, cereals/roots/tubers (\%) & 40 ~ \ \ & 36 ~ \ \ & \textit{37} ~ \ \ \\ 
	 ~ Prevalence of undernourishment (\%) & 13.2 ~ \ \ & 11.3 ~ \ \ & 8 ~ \ \ \\ 
	 ~ GDP per capita (US\$, PPP) & 13\,426 ~ \ \ & 19\,773 ~ \ \ & \textit{29\,469} ~ \ \ \\ 
	 ~ Domestic food price volatility (index) &  ~ \ \ & 8 ~ \ \ & \textit{16.5} ~ \ \ \\ 
	 ~ Cereal import dependency ratio (\%) & 93.4 ~ \ \ & 97.9 ~ \ \ & \textit{98} ~ \ \ \\ 
	 ~ Underweight, children under-5 (\%) &  ~ \ \ & \textit{4.4} ~ \ \ &  ~ \ \ \\ 
	 ~ Improved water source (\% pop) & 90.7 ~ \ \ & 92.6 ~ \ \ & \textit{93.6} ~ \ \ \\ 
	\multicolumn{4}{l}{\textcolor{FAOblue}{\textbf{\large{Food Supply}}}} \\ 
	 ~ Food production value, (2004-2006 mln I\$) & 130 ~ \ \ & 176 ~ \ \ & \textit{143} ~ \ \ \\ 
	 ~ Agriculture, value added (\% GDP) & 3 ~ \ \ & 1 ~ \ \ & \textit{1} ~ \ \ \\ 
	 ~ Food exports (mln US\$)  & 77 ~ \ \ & 121 ~ \ \ & \textit{114} ~ \ \ \\ 
	 ~ Food imports (mln US\$)  & 217 ~ \ \ & 259 ~ \ \ & \textit{756} ~ \ \ \\ 
	\multicolumn{4}{l}{\textit{\normalsize{Production indices (2004-06=100)}}} \\ 
	 ~ Net food & 88 ~ \ \ & 118 ~ \ \ & \textit{96} ~ \ \ \\ 
	 ~ Net crop & 138 ~ \ \ & 133 ~ \ \ & \textit{60} ~ \ \ \\ 
	 ~ Cereal & 745 ~ \ \ & 172 ~ \ \ & \textit{141} ~ \ \ \\ 
	 ~ Vegetable oils & 239 ~ \ \ & 133 ~ \ \ & \textit{92} ~ \ \ \\ 
	 ~ Roots and tubers & 59 ~ \ \ & 120 ~ \ \ & \textit{118} ~ \ \ \\ 
	 ~ Fruit and vegetables & 78 ~ \ \ & 79 ~ \ \ & \textit{88} ~ \ \ \\ 
	 ~ Sugar & 214 ~ \ \ & 221 ~ \ \ & \textit{134} ~ \ \ \\ 
	 ~ Livestock & 54 ~ \ \ & 108 ~ \ \ & \textit{121} ~ \ \ \\ 
	 ~ Milk & 146 ~ \ \ & 135 ~ \ \ & \textit{69} ~ \ \ \\ 
	 ~ Meat & 50 ~ \ \ & 108 ~ \ \ & \textit{123} ~ \ \ \\ 
	 ~ Fish  & 87 ~ \ \ & 125 ~ \ \ & \textit{88} ~ \ \ \\ 
	\multicolumn{4}{l}{\textit{\normalsize{Net trade (min US\$)}}} \\ 
	 ~ Cereals & -43 ~ \ \ & -23 ~ \ \ & \textit{-145} ~ \ \ \\ 
	 ~ Fruit and vegetables & -39 ~ \ \ & -32 ~ \ \ & \textit{-106} ~ \ \ \\ 
	 ~ Meat & -17 ~ \ \ & -19 ~ \ \ & \textit{-99} ~ \ \ \\ 
	 ~ Dairy products & -37 ~ \ \ & -37 ~ \ \ & \textit{-93} ~ \ \ \\ 
	 ~ Fish & 0 ~ \ \ & 0 ~ \ \ & \textit{-30} ~ \ \ \\ 
	\multicolumn{4}{l}{\textcolor{FAOblue}{\textbf{\large{Environment}}}} \\ 
	 ~ Forest area (\%) & 47 ~ \ \ & 45 ~ \ \ & \textit{44} ~ \ \ \\ 
	 ~ Renewable water res withdrawn (\% of total) &  ~ \ \ & \textit{9} ~ \ \ & 9 ~ \ \ \\ 
	 ~ Terrestrial protect areas (\% total land area)  & 31 ~ \ \ & 31 ~ \ \ & \textit{33} ~ \ \ \\ 
	 ~ Organic area (\% total agricultural area) &  ~ \ \ &  ~ \ \ &  ~ \ \ \\ 
	 ~ Water withdrawal by agriculture (\% of total) &  ~ \ \ & \textit{9} ~ \ \ & 9 ~ \ \ \\ 
	 ~ Biofuel production (thousand kt of oil eq.) & 3 ~ \ \ & 4 ~ \ \ & \textit{0} ~ \ \ \\ 
	 ~ Wood pellet prod. (min tonnes) &  ~ \ \ &  ~ \ \ &  ~ \ \ \\ 
	 ~ GHG emissions from ag (Co2 eq, gigagrams) & 1 ~ \ \ & 0 ~ \ \ & \textit{1} ~ \ \ \\ 
       \toprule
      \end{tabular}
      \clearpage
\CountryData{ Tunisia }
      \rowcolors{1}{FAOblue!10}{white}
      \begin{tabular}{L{3.9cm} R{1cm} R{1cm} R{1cm}}
      \toprule
      \multicolumn{1}{c}{} & \multicolumn{1}{c}{ 1992 } & \multicolumn{1}{c}{ 2002 } & \multicolumn{1}{c}{ 2014 } \\
      \midrule
	\multicolumn{4}{l}{\textcolor{FAOblue}{\textbf{\large{The setting}}}} \\ 
	 ~ Population, total (mln) & 8.5 ~ \ \ & 9.8 ~ \ \ & 11.1 ~ \ \ \\ 
	 ~ Population, rural (\% total population) & 3.4 ~ \ \ & 3.5 ~ \ \ & 3.7 ~ \ \ \\ 
	 ~ Govt expenditure on ag (\% total outlays) &  ~ \ \ & 9.5 ~ \ \ & \textit{5.4} ~ \ \ \\ 
	 ~ Area harvested (mln ha) & 2 ~ \ \ & 2 ~ \ \ & 2 ~ \ \ \\ 
	 ~ Cropping intensity ratio (\%) & 0.2 ~ \ \ & 0.2 ~ \ \ &  ~ \ \ \\ 
	 ~ Water resources (m\textsuperscript{3}/person/year) & \textit{1} ~ \ \ & \textit{0} ~ \ \ & \textit{0} ~ \ \ \\ 
	 ~ Area equipped for irrigation (1000 ha) &  ~ \ \ &  ~ \ \ & \textit{459} ~ \ \ \\ 
	 ~ Area irrigated (\%) &  ~ \ \ &  ~ \ \ & \textit{89} ~ \ \ \\ 
	 ~ Employment in agriculture (\%) &  ~ \ \ & \textit{18.7} ~ \ \ & \textit{16.2} ~ \ \ \\ 
	 ~ Employment in agriculture, female (\%) &  ~ \ \ &  ~ \ \ &  ~ \ \ \\ 
	 ~ Fertilizers, Nitrogen (nutrients per ha) &  ~ \ \ & 3.8 ~ \ \ & \textit{8.7} ~ \ \ \\ 
	 ~ Fertilizers, Phosphate (nutrients per ha) &  ~ \ \ & 2.5 ~ \ \ & \textit{6.4} ~ \ \ \\ 
	 ~ Fertilizers, Potash (nutrients per ha) &  ~ \ \ & 0.8 ~ \ \ & \textit{0.6} ~ \ \ \\ 
	 ~ Energy consump, power irrigation (mln kWh) & 147 ~ \ \ & 366 ~ \ \ & \textit{366} ~ \ \ \\ 
	 ~ Agr value added per worker (constant US\$) & 3.5 ~ \ \ & 3.1 ~ \ \ & \textit{4.4} ~ \ \ \\ 
	\multicolumn{4}{l}{\textcolor{FAOblue}{\textbf{\large{Hunger dimensions}}}} \\ 
	 ~ Dietary energy supply (kcal/pc/day) & 3\,140 ~ \ \ & 3\,231 ~ \ \ & 3\,465 ~ \ \ \\ 
	 ~ Average dietary energy supply adequacy (\%) & 140 ~ \ \ & 138 ~ \ \ & 146 ~ \ \ \\ 
	 ~ Dietary en supp, cereals/roots/tubers (\%) & 56 ~ \ \ & 53 ~ \ \ & \textit{53} ~ \ \ \\ 
	 ~ Prevalence of undernourishment (\%) & <5.0 ~ \ \ & <5.0 ~ \ \ & <5.0 ~ \ \ \\ 
	 ~ GDP per capita (US\$, PPP) & 5\,920 ~ \ \ & 7\,752 ~ \ \ & \textit{10\,768} ~ \ \ \\ 
	 ~ Domestic food price volatility (index) &  ~ \ \ & 4 ~ \ \ & 4.7 ~ \ \ \\ 
	 ~ Cereal import dependency ratio (\%) & 29.9 ~ \ \ & 65.6 ~ \ \ & \textit{55.3} ~ \ \ \\ 
	 ~ Underweight, children under-5 (\%) & \textit{8.1} ~ \ \ & \textit{3.5} ~ \ \ & \textit{2.3} ~ \ \ \\ 
	 ~ Improved water source (\% pop) & 83.3 ~ \ \ & 90.8 ~ \ \ & \textit{96.8} ~ \ \ \\ 
	\multicolumn{4}{l}{\textcolor{FAOblue}{\textbf{\large{Food Supply}}}} \\ 
	 ~ Food production value, (2004-2006 mln I\$) & 2\,384 ~ \ \ & 2\,448 ~ \ \ & \textit{3\,916} ~ \ \ \\ 
	 ~ Agriculture, value added (\% GDP) & 19 ~ \ \ & 9 ~ \ \ & \textit{9} ~ \ \ \\ 
	 ~ Food exports (mln US\$)  & 288 ~ \ \ & 302 ~ \ \ & \textit{1\,309} ~ \ \ \\ 
	 ~ Food imports (mln US\$)  & 401 ~ \ \ & 772 ~ \ \ & \textit{2\,144} ~ \ \ \\ 
	\multicolumn{4}{l}{\textit{\normalsize{Production indices (2004-06=100)}}} \\ 
	 ~ Net food & 72 ~ \ \ & 74 ~ \ \ & \textit{119} ~ \ \ \\ 
	 ~ Net crop & 76 ~ \ \ & 64 ~ \ \ & \textit{117} ~ \ \ \\ 
	 ~ Cereal & 110 ~ \ \ & 25 ~ \ \ & \textit{65} ~ \ \ \\ 
	 ~ Vegetable oils & 69 ~ \ \ & 37 ~ \ \ & \textit{113} ~ \ \ \\ 
	 ~ Roots and tubers & 63 ~ \ \ & 88 ~ \ \ & \textit{111} ~ \ \ \\ 
	 ~ Fruit and vegetables & 68 ~ \ \ & 92 ~ \ \ & \textit{132} ~ \ \ \\ 
	 ~ Sugar &  ~ \ \ &  ~ \ \ &  ~ \ \ \\ 
	 ~ Livestock & 64 ~ \ \ & 102 ~ \ \ & \textit{120} ~ \ \ \\ 
	 ~ Milk & 51 ~ \ \ & 104 ~ \ \ & \textit{126} ~ \ \ \\ 
	 ~ Meat & 70 ~ \ \ & 102 ~ \ \ & \textit{117} ~ \ \ \\ 
	 ~ Fish  & 77 ~ \ \ & 87 ~ \ \ & \textit{109} ~ \ \ \\ 
	\multicolumn{4}{l}{\textit{\normalsize{Net trade (min US\$)}}} \\ 
	 ~ Cereals & -118 ~ \ \ & -395 ~ \ \ & \textit{-813} ~ \ \ \\ 
	 ~ Fruit and vegetables & 68 ~ \ \ & 72 ~ \ \ & \textit{299} ~ \ \ \\ 
	 ~ Meat & -24 ~ \ \ & 1 ~ \ \ & \textit{-41} ~ \ \ \\ 
	 ~ Dairy products & -57 ~ \ \ & -15 ~ \ \ & \textit{-7} ~ \ \ \\ 
	 ~ Fish & 70 ~ \ \ & 75 ~ \ \ & \textit{89} ~ \ \ \\ 
	\multicolumn{4}{l}{\textcolor{FAOblue}{\textbf{\large{Environment}}}} \\ 
	 ~ Forest area (\%) & 4 ~ \ \ & 6 ~ \ \ & \textit{7} ~ \ \ \\ 
	 ~ Renewable water res withdrawn (\% of total) &  ~ \ \ & \textit{82} ~ \ \ & 82 ~ \ \ \\ 
	 ~ Terrestrial protect areas (\% total land area)  & 1 ~ \ \ & 1 ~ \ \ & \textit{5} ~ \ \ \\ 
	 ~ Organic area (\% total agricultural area) &  ~ \ \ & \textit{1} ~ \ \ & \textit{1} ~ \ \ \\ 
	 ~ Water withdrawal by agriculture (\% of total) &  ~ \ \ & \textit{82} ~ \ \ & 82 ~ \ \ \\ 
	 ~ Biofuel production (thousand kt of oil eq.) &  ~ \ \ & 6 ~ \ \ & \textit{6} ~ \ \ \\ 
	 ~ Wood pellet prod. (min tonnes) &  ~ \ \ &  ~ \ \ &  ~ \ \ \\ 
	 ~ GHG emissions from ag (Co2 eq, gigagrams) & 3 ~ \ \ & 3 ~ \ \ & \textit{4} ~ \ \ \\ 
       \toprule
      \end{tabular}
      \clearpage
\CountryData{ Turkey }
      \rowcolors{1}{FAOblue!10}{white}
      \begin{tabular}{L{3.9cm} R{1cm} R{1cm} R{1cm}}
      \toprule
      \multicolumn{1}{c}{} & \multicolumn{1}{c}{ 1992 } & \multicolumn{1}{c}{ 2002 } & \multicolumn{1}{c}{ 2014 } \\
      \midrule
	\multicolumn{4}{l}{\textcolor{FAOblue}{\textbf{\large{The setting}}}} \\ 
	 ~ Population, total (mln) & 55.8 ~ \ \ & 65 ~ \ \ & 75.8 ~ \ \ \\ 
	 ~ Population, rural (\% total population) & 22 ~ \ \ & 22.2 ~ \ \ & 19.5 ~ \ \ \\ 
	 ~ Govt expenditure on ag (\% total outlays) &  ~ \ \ &  ~ \ \ &  ~ \ \ \\ 
	 ~ Area harvested (mln ha) & 29 ~ \ \ & 31 ~ \ \ & 37 ~ \ \ \\ 
	 ~ Cropping intensity ratio (\%) & 0.7 ~ \ \ & 0.7 ~ \ \ &  ~ \ \ \\ 
	 ~ Water resources (m\textsuperscript{3}/person/year) & \textit{4} ~ \ \ & \textit{3} ~ \ \ & \textit{3} ~ \ \ \\ 
	 ~ Area equipped for irrigation (1000 ha) &  ~ \ \ &  ~ \ \ & \textit{5\,215} ~ \ \ \\ 
	 ~ Area irrigated (\%) &  ~ \ \ &  ~ \ \ & \textit{98.9} ~ \ \ \\ 
	 ~ Employment in agriculture (\%) & 44.7 ~ \ \ & 34.9 ~ \ \ & \textit{23.6} ~ \ \ \\ 
	 ~ Employment in agriculture, female (\%) & 72.2 ~ \ \ & 60 ~ \ \ & \textit{37.2} ~ \ \ \\ 
	 ~ Fertilizers, Nitrogen (nutrients per ha) &  ~ \ \ & 29.1 ~ \ \ & \textit{38.4} ~ \ \ \\ 
	 ~ Fertilizers, Phosphate (nutrients per ha) &  ~ \ \ & 11.5 ~ \ \ & \textit{16} ~ \ \ \\ 
	 ~ Fertilizers, Potash (nutrients per ha) &  ~ \ \ & 1.8 ~ \ \ & \textit{2.5} ~ \ \ \\ 
	 ~ Energy consump, power irrigation (mln kWh) & \textit{637} ~ \ \ & 637 ~ \ \ & \textit{959} ~ \ \ \\ 
	 ~ Agr value added per worker (constant US\$) & 3.7 ~ \ \ & 4.5 ~ \ \ & \textit{6.9} ~ \ \ \\ 
	\multicolumn{4}{l}{\textcolor{FAOblue}{\textbf{\large{Hunger dimensions}}}} \\ 
	 ~ Dietary energy supply (kcal/pc/day) & 3\,721 ~ \ \ & 3\,606 ~ \ \ & 3\,703 ~ \ \ \\ 
	 ~ Average dietary energy supply adequacy (\%) & 164 ~ \ \ & 155 ~ \ \ & 156 ~ \ \ \\ 
	 ~ Dietary en supp, cereals/roots/tubers (\%) & 54 ~ \ \ & 54 ~ \ \ & \textit{47} ~ \ \ \\ 
	 ~ Prevalence of undernourishment (\%) & <5.0 ~ \ \ & <5.0 ~ \ \ & <5.0 ~ \ \ \\ 
	 ~ GDP per capita (US\$, PPP) & 10\,920 ~ \ \ & 12\,670 ~ \ \ & \textit{18\,567} ~ \ \ \\ 
	 ~ Domestic food price volatility (index) &  ~ \ \ & 25.2 ~ \ \ & 12.9 ~ \ \ \\ 
	 ~ Cereal import dependency ratio (\%) & -9 ~ \ \ & 4 ~ \ \ & \textit{0.8} ~ \ \ \\ 
	 ~ Underweight, children under-5 (\%) & \textit{9} ~ \ \ & \textit{3.5} ~ \ \ & \textit{1.9} ~ \ \ \\ 
	 ~ Improved water source (\% pop) & 86.4 ~ \ \ & 94.3 ~ \ \ & \textit{99.7} ~ \ \ \\ 
	\multicolumn{4}{l}{\textcolor{FAOblue}{\textbf{\large{Food Supply}}}} \\ 
	 ~ Food production value, (2004-2006 mln I\$) & 23\,381 ~ \ \ & 26\,726 ~ \ \ & \textit{37\,484} ~ \ \ \\ 
	 ~ Agriculture, value added (\% GDP) & 16 ~ \ \ & 12 ~ \ \ & \textit{8} ~ \ \ \\ 
	 ~ Food exports (mln US\$)  & 2\,899 ~ \ \ & 2\,834 ~ \ \ & \textit{12\,641} ~ \ \ \\ 
	 ~ Food imports (mln US\$)  & 763 ~ \ \ & 1\,394 ~ \ \ & \textit{7\,787} ~ \ \ \\ 
	\multicolumn{4}{l}{\textit{\normalsize{Production indices (2004-06=100)}}} \\ 
	 ~ Net food & 81 ~ \ \ & 93 ~ \ \ & \textit{130} ~ \ \ \\ 
	 ~ Net crop & 80 ~ \ \ & 96 ~ \ \ & \textit{115} ~ \ \ \\ 
	 ~ Cereal & 81 ~ \ \ & 87 ~ \ \ & \textit{109} ~ \ \ \\ 
	 ~ Vegetable oils & 64 ~ \ \ & 108 ~ \ \ & \textit{122} ~ \ \ \\ 
	 ~ Roots and tubers & 105 ~ \ \ & 117 ~ \ \ & \textit{90} ~ \ \ \\ 
	 ~ Fruit and vegetables & 75 ~ \ \ & 94 ~ \ \ & \textit{121} ~ \ \ \\ 
	 ~ Sugar & 105 ~ \ \ & 115 ~ \ \ & \textit{115} ~ \ \ \\ 
	 ~ Livestock & 83 ~ \ \ & 84 ~ \ \ & \textit{163} ~ \ \ \\ 
	 ~ Milk & 93 ~ \ \ & 75 ~ \ \ & \textit{162} ~ \ \ \\ 
	 ~ Meat & 77 ~ \ \ & 90 ~ \ \ & \textit{174} ~ \ \ \\ 
	 ~ Fish  & 74 ~ \ \ & 102 ~ \ \ & \textit{98} ~ \ \ \\ 
	\multicolumn{4}{l}{\textit{\normalsize{Net trade (min US\$)}}} \\ 
	 ~ Cereals & 443 ~ \ \ & -132 ~ \ \ & \textit{519} ~ \ \ \\ 
	 ~ Fruit and vegetables & 1\,513 ~ \ \ & 1\,791 ~ \ \ & \textit{5\,738} ~ \ \ \\ 
	 ~ Meat & -12 ~ \ \ & 14 ~ \ \ & \textit{469} ~ \ \ \\ 
	 ~ Dairy products & -17 ~ \ \ & -1 ~ \ \ & \textit{87} ~ \ \ \\ 
	 ~ Fish & 29 ~ \ \ & 87 ~ \ \ & \textit{137} ~ \ \ \\ 
	\multicolumn{4}{l}{\textcolor{FAOblue}{\textbf{\large{Environment}}}} \\ 
	 ~ Forest area (\%) & 13 ~ \ \ & 13 ~ \ \ & \textit{15} ~ \ \ \\ 
	 ~ Renewable water res withdrawn (\% of total) &  ~ \ \ & \textit{74} ~ \ \ & 74 ~ \ \ \\ 
	 ~ Terrestrial protect areas (\% total land area)  & 2 ~ \ \ & 2 ~ \ \ & \textit{2} ~ \ \ \\ 
	 ~ Organic area (\% total agricultural area) &  ~ \ \ & \textit{0} ~ \ \ & \textit{2} ~ \ \ \\ 
	 ~ Water withdrawal by agriculture (\% of total) &  ~ \ \ & \textit{74} ~ \ \ & 74 ~ \ \ \\ 
	 ~ Biofuel production (thousand kt of oil eq.) & 16 ~ \ \ & 54 ~ \ \ & \textit{236} ~ \ \ \\ 
	 ~ Wood pellet prod. (min tonnes) &  ~ \ \ &  ~ \ \ &  ~ \ \ \\ 
	 ~ GHG emissions from ag (Co2 eq, gigagrams) & 24 ~ \ \ & 9 ~ \ \ & \textit{14} ~ \ \ \\ 
       \toprule
      \end{tabular}
      \clearpage
\CountryData{ Turkmenistan }
      \rowcolors{1}{FAOblue!10}{white}
      \begin{tabular}{L{3.9cm} R{1cm} R{1cm} R{1cm}}
      \toprule
      \multicolumn{1}{c}{} & \multicolumn{1}{c}{ 1992 } & \multicolumn{1}{c}{ 2002 } & \multicolumn{1}{c}{ 2014 } \\
      \midrule
	\multicolumn{4}{l}{\textcolor{FAOblue}{\textbf{\large{The setting}}}} \\ 
	 ~ Population, total (mln) & 3.9 ~ \ \ & 4.6 ~ \ \ & 5.3 ~ \ \ \\ 
	 ~ Population, rural (\% total population) & 2.1 ~ \ \ & 2.5 ~ \ \ & 2.7 ~ \ \ \\ 
	 ~ Govt expenditure on ag (\% total outlays) &  ~ \ \ &  ~ \ \ &  ~ \ \ \\ 
	 ~ Area harvested (mln ha) & 1 ~ \ \ & 2 ~ \ \ & 5 ~ \ \ \\ 
	 ~ Cropping intensity ratio (\%) & 0 ~ \ \ & 0.1 ~ \ \ &  ~ \ \ \\ 
	 ~ Water resources (m\textsuperscript{3}/person/year) & \textit{6} ~ \ \ & \textit{5} ~ \ \ & \textit{5} ~ \ \ \\ 
	 ~ Area equipped for irrigation (1000 ha) &  ~ \ \ &  ~ \ \ & \textit{1\,995} ~ \ \ \\ 
	 ~ Area irrigated (\%) &  ~ \ \ &  ~ \ \ & \textit{100} ~ \ \ \\ 
	 ~ Employment in agriculture (\%) &  ~ \ \ &  ~ \ \ &  ~ \ \ \\ 
	 ~ Employment in agriculture, female (\%) &  ~ \ \ &  ~ \ \ &  ~ \ \ \\ 
	 ~ Fertilizers, Nitrogen (nutrients per ha) &  ~ \ \ &  ~ \ \ &  ~ \ \ \\ 
	 ~ Fertilizers, Phosphate (nutrients per ha) &  ~ \ \ &  ~ \ \ &  ~ \ \ \\ 
	 ~ Fertilizers, Potash (nutrients per ha) &  ~ \ \ &  ~ \ \ &  ~ \ \ \\ 
	 ~ Energy consump, power irrigation (mln kWh) & \textit{1} ~ \ \ & 1 ~ \ \ & \textit{0} ~ \ \ \\ 
	 ~ Agr value added per worker (constant US\$) & 1.7 ~ \ \ & 1.6 ~ \ \ & \textit{2.8} ~ \ \ \\ 
	\multicolumn{4}{l}{\textcolor{FAOblue}{\textbf{\large{Hunger dimensions}}}} \\ 
	 ~ Dietary energy supply (kcal/pc/day) & 2\,578 ~ \ \ & 2\,686 ~ \ \ & 3\,010 ~ \ \ \\ 
	 ~ Average dietary energy supply adequacy (\%) & 117 ~ \ \ & 116 ~ \ \ & 128 ~ \ \ \\ 
	 ~ Dietary en supp, cereals/roots/tubers (\%) & 59 ~ \ \ & 63 ~ \ \ & \textit{59} ~ \ \ \\ 
	 ~ Prevalence of undernourishment (\%) & 8.6 ~ \ \ & 7.9 ~ \ \ & <5.0 ~ \ \ \\ 
	 ~ GDP per capita (US\$, PPP) & 6\,402 ~ \ \ & 5\,477 ~ \ \ & \textit{13\,555} ~ \ \ \\ 
	 ~ Domestic food price volatility (index) &  ~ \ \ &  ~ \ \ &  ~ \ \ \\ 
	 ~ Cereal import dependency ratio (\%) &  ~ \ \ &  ~ \ \ &  ~ \ \ \\ 
	 ~ Underweight, children under-5 (\%) &  ~ \ \ & \textit{10.5} ~ \ \ &  ~ \ \ \\ 
	 ~ Improved water source (\% pop) & \textit{86.4} ~ \ \ & 80.1 ~ \ \ & \textit{71.1} ~ \ \ \\ 
	\multicolumn{4}{l}{\textcolor{FAOblue}{\textbf{\large{Food Supply}}}} \\ 
	 ~ Food production value, (2004-2006 mln I\$) & 792 ~ \ \ & 1\,581 ~ \ \ & \textit{2\,365} ~ \ \ \\ 
	 ~ Agriculture, value added (\% GDP) & 11 ~ \ \ & 22 ~ \ \ & \textit{15} ~ \ \ \\ 
	 ~ Food exports (mln US\$)  & 0 ~ \ \ & 7 ~ \ \ & \textit{9} ~ \ \ \\ 
	 ~ Food imports (mln US\$)  & 230 ~ \ \ & 61 ~ \ \ & \textit{373} ~ \ \ \\ 
	\multicolumn{4}{l}{\textit{\normalsize{Production indices (2004-06=100)}}} \\ 
	 ~ Net food & 42 ~ \ \ & 83 ~ \ \ & \textit{124} ~ \ \ \\ 
	 ~ Net crop & 74 ~ \ \ & 79 ~ \ \ & \textit{77} ~ \ \ \\ 
	 ~ Cereal & 22 ~ \ \ & 78 ~ \ \ & \textit{52} ~ \ \ \\ 
	 ~ Vegetable oils & 143 ~ \ \ & 78 ~ \ \ & \textit{63} ~ \ \ \\ 
	 ~ Roots and tubers & 18 ~ \ \ & 77 ~ \ \ & \textit{156} ~ \ \ \\ 
	 ~ Fruit and vegetables & 49 ~ \ \ & 81 ~ \ \ & \textit{115} ~ \ \ \\ 
	 ~ Sugar &  ~ \ \ & 95 ~ \ \ & \textit{103} ~ \ \ \\ 
	 ~ Livestock & 37 ~ \ \ & 85 ~ \ \ & \textit{136} ~ \ \ \\ 
	 ~ Milk & 27 ~ \ \ & 80 ~ \ \ & \textit{126} ~ \ \ \\ 
	 ~ Meat & 43 ~ \ \ & 90 ~ \ \ & \textit{150} ~ \ \ \\ 
	 ~ Fish  & 225 ~ \ \ & 86 ~ \ \ & \textit{100} ~ \ \ \\ 
	\multicolumn{4}{l}{\textit{\normalsize{Net trade (min US\$)}}} \\ 
	 ~ Cereals &  ~ \ \ &  ~ \ \ &  ~ \ \ \\ 
	 ~ Fruit and vegetables & -15 ~ \ \ & -4 ~ \ \ & \textit{-13} ~ \ \ \\ 
	 ~ Meat & \textit{-25} ~ \ \ & \textit{-25} ~ \ \ &  ~ \ \ \\ 
	 ~ Dairy products & \textit{-3} ~ \ \ & \textit{0} ~ \ \ &  ~ \ \ \\ 
	 ~ Fish & \textit{0} ~ \ \ & 0 ~ \ \ & \textit{-7} ~ \ \ \\ 
	\multicolumn{4}{l}{\textcolor{FAOblue}{\textbf{\large{Environment}}}} \\ 
	 ~ Forest area (\%) & 9 ~ \ \ & 9 ~ \ \ & \textit{9} ~ \ \ \\ 
	 ~ Renewable water res withdrawn (\% of total) &  ~ \ \ & \textit{94} ~ \ \ & 94 ~ \ \ \\ 
	 ~ Terrestrial protect areas (\% total land area)  & 3 ~ \ \ & 3 ~ \ \ & \textit{3} ~ \ \ \\ 
	 ~ Organic area (\% total agricultural area) &  ~ \ \ &  ~ \ \ &  ~ \ \ \\ 
	 ~ Water withdrawal by agriculture (\% of total) &  ~ \ \ & \textit{94} ~ \ \ & 94 ~ \ \ \\ 
	 ~ Biofuel production (thousand kt of oil eq.) &  ~ \ \ &  ~ \ \ &  ~ \ \ \\ 
	 ~ Wood pellet prod. (min tonnes) &  ~ \ \ &  ~ \ \ & \textit{0} ~ \ \ \\ 
	 ~ GHG emissions from ag (Co2 eq, gigagrams) & 4 ~ \ \ & 6 ~ \ \ & \textit{8} ~ \ \ \\ 
       \toprule
      \end{tabular}
      \clearpage
\CountryData{ Uganda }
      \rowcolors{1}{FAOblue!10}{white}
      \begin{tabular}{L{3.9cm} R{1cm} R{1cm} R{1cm}}
      \toprule
      \multicolumn{1}{c}{} & \multicolumn{1}{c}{ 1992 } & \multicolumn{1}{c}{ 2002 } & \multicolumn{1}{c}{ 2014 } \\
      \midrule
	\multicolumn{4}{l}{\textcolor{FAOblue}{\textbf{\large{The setting}}}} \\ 
	 ~ Population, total (mln) & 18.8 ~ \ \ & 25.9 ~ \ \ & 38.8 ~ \ \ \\ 
	 ~ Population, rural (\% total population) & 16.6 ~ \ \ & 22.8 ~ \ \ & 32.3 ~ \ \ \\ 
	 ~ Govt expenditure on ag (\% total outlays) &  ~ \ \ & 4.2 ~ \ \ & \textit{3.6} ~ \ \ \\ 
	 ~ Area harvested (mln ha) & 8 ~ \ \ & 11 ~ \ \ & 9 ~ \ \ \\ 
	 ~ Cropping intensity ratio (\%) & 0.7 ~ \ \ & 0.8 ~ \ \ &  ~ \ \ \\ 
	 ~ Water resources (m\textsuperscript{3}/person/year) & \textit{3} ~ \ \ & \textit{2} ~ \ \ & \textit{2} ~ \ \ \\ 
	 ~ Area equipped for irrigation (1000 ha) &  ~ \ \ &  ~ \ \ & \textit{14} ~ \ \ \\ 
	 ~ Area irrigated (\%) &  ~ \ \ &  ~ \ \ & \textit{95} ~ \ \ \\ 
	 ~ Employment in agriculture (\%) &  ~ \ \ & 65.5 ~ \ \ & \textit{65.6} ~ \ \ \\ 
	 ~ Employment in agriculture, female (\%) &  ~ \ \ & \textit{75.7} ~ \ \ &  ~ \ \ \\ 
	 ~ Fertilizers, Nitrogen (nutrients per ha) &  ~ \ \ & 0.3 ~ \ \ & \textit{0.5} ~ \ \ \\ 
	 ~ Fertilizers, Phosphate (nutrients per ha) &  ~ \ \ & 0.2 ~ \ \ & \textit{0.2} ~ \ \ \\ 
	 ~ Fertilizers, Potash (nutrients per ha) &  ~ \ \ & 0.1 ~ \ \ & \textit{0.2} ~ \ \ \\ 
	 ~ Energy consump, power irrigation (mln kWh) & 0 ~ \ \ & 1 ~ \ \ & \textit{5} ~ \ \ \\ 
	 ~ Agr value added per worker (constant US\$) & 0.2 ~ \ \ & 0.2 ~ \ \ & \textit{0.2} ~ \ \ \\ 
	\multicolumn{4}{l}{\textcolor{FAOblue}{\textbf{\large{Hunger dimensions}}}} \\ 
	 ~ Dietary energy supply (kcal/pc/day) & 2\,245 ~ \ \ & 2\,331 ~ \ \ & 2\,273 ~ \ \ \\ 
	 ~ Average dietary energy supply adequacy (\%) & 108 ~ \ \ & 112 ~ \ \ & 108 ~ \ \ \\ 
	 ~ Dietary en supp, cereals/roots/tubers (\%) & 45 ~ \ \ & 45 ~ \ \ & \textit{45} ~ \ \ \\ 
	 ~ Prevalence of undernourishment (\%) & 24.4 ~ \ \ & 27.2 ~ \ \ & 25.3 ~ \ \ \\ 
	 ~ GDP per capita (US\$, PPP) & 781 ~ \ \ & 1\,115 ~ \ \ & \textit{1\,621} ~ \ \ \\ 
	 ~ Domestic food price volatility (index) &  ~ \ \ & 12.2 ~ \ \ & 21.8 ~ \ \ \\ 
	 ~ Cereal import dependency ratio (\%) & -1.3 ~ \ \ & 6.7 ~ \ \ & \textit{9.1} ~ \ \ \\ 
	 ~ Underweight, children under-5 (\%) & \textit{21.5} ~ \ \ & \textit{19} ~ \ \ & \textit{14.1} ~ \ \ \\ 
	 ~ Improved water source (\% pop) & 44.7 ~ \ \ & 59.4 ~ \ \ & \textit{74.8} ~ \ \ \\ 
	\multicolumn{4}{l}{\textcolor{FAOblue}{\textbf{\large{Food Supply}}}} \\ 
	 ~ Food production value, (2004-2006 mln I\$) & 3\,269 ~ \ \ & 4\,538 ~ \ \ & \textit{5\,360} ~ \ \ \\ 
	 ~ Agriculture, value added (\% GDP) & 51 ~ \ \ & 25 ~ \ \ & \textit{25} ~ \ \ \\ 
	 ~ Food exports (mln US\$)  & 14 ~ \ \ & 32 ~ \ \ & \textit{446} ~ \ \ \\ 
	 ~ Food imports (mln US\$)  & 33 ~ \ \ & 135 ~ \ \ & \textit{793} ~ \ \ \\ 
	\multicolumn{4}{l}{\textit{\normalsize{Production indices (2004-06=100)}}} \\ 
	 ~ Net food & 69 ~ \ \ & 95 ~ \ \ & \textit{113} ~ \ \ \\ 
	 ~ Net crop & 72 ~ \ \ & 102 ~ \ \ & \textit{108} ~ \ \ \\ 
	 ~ Cereal & 72 ~ \ \ & 96 ~ \ \ & \textit{136} ~ \ \ \\ 
	 ~ Vegetable oils & 47 ~ \ \ & 72 ~ \ \ & \textit{126} ~ \ \ \\ 
	 ~ Roots and tubers & 58 ~ \ \ & 99 ~ \ \ & \textit{102} ~ \ \ \\ 
	 ~ Fruit and vegetables & 83 ~ \ \ & 105 ~ \ \ & \textit{100} ~ \ \ \\ 
	 ~ Sugar & 40 ~ \ \ & 78 ~ \ \ & \textit{141} ~ \ \ \\ 
	 ~ Livestock & 57 ~ \ \ & 80 ~ \ \ & \textit{127} ~ \ \ \\ 
	 ~ Milk & 44 ~ \ \ & 68 ~ \ \ & \textit{118} ~ \ \ \\ 
	 ~ Meat & 61 ~ \ \ & 84 ~ \ \ & \textit{130} ~ \ \ \\ 
	 ~ Fish  & 66 ~ \ \ & 56 ~ \ \ & \textit{129} ~ \ \ \\ 
	\multicolumn{4}{l}{\textit{\normalsize{Net trade (min US\$)}}} \\ 
	 ~ Cereals & -3 ~ \ \ & -61 ~ \ \ & \textit{-185} ~ \ \ \\ 
	 ~ Fruit and vegetables & 1 ~ \ \ & 2 ~ \ \ & \textit{13} ~ \ \ \\ 
	 ~ Meat & 0 ~ \ \ & 0 ~ \ \ & \textit{-1} ~ \ \ \\ 
	 ~ Dairy products & -3 ~ \ \ & -2 ~ \ \ & \textit{16} ~ \ \ \\ 
	 ~ Fish & 6 ~ \ \ & 86 ~ \ \ & \textit{61} ~ \ \ \\ 
	\multicolumn{4}{l}{\textcolor{FAOblue}{\textbf{\large{Environment}}}} \\ 
	 ~ Forest area (\%) & 23 ~ \ \ & 18 ~ \ \ & \textit{14} ~ \ \ \\ 
	 ~ Renewable water res withdrawn (\% of total) &  ~ \ \ &  ~ \ \ & 41 ~ \ \ \\ 
	 ~ Terrestrial protect areas (\% total land area)  & 8 ~ \ \ & 10 ~ \ \ & \textit{11} ~ \ \ \\ 
	 ~ Organic area (\% total agricultural area) &  ~ \ \ & \textit{1} ~ \ \ & \textit{2} ~ \ \ \\ 
	 ~ Water withdrawal by agriculture (\% of total) &  ~ \ \ &  ~ \ \ & 41 ~ \ \ \\ 
	 ~ Biofuel production (thousand kt of oil eq.) & 1 ~ \ \ & 6 ~ \ \ & \textit{11} ~ \ \ \\ 
	 ~ Wood pellet prod. (min tonnes) &  ~ \ \ &  ~ \ \ &  ~ \ \ \\ 
	 ~ GHG emissions from ag (Co2 eq, gigagrams) & 30 ~ \ \ & 32 ~ \ \ & \textit{42} ~ \ \ \\ 
       \toprule
      \end{tabular}
      \clearpage
\CountryData{ Ukraine }
      \rowcolors{1}{FAOblue!10}{white}
      \begin{tabular}{L{3.9cm} R{1cm} R{1cm} R{1cm}}
      \toprule
      \multicolumn{1}{c}{} & \multicolumn{1}{c}{ 1992 } & \multicolumn{1}{c}{ 2002 } & \multicolumn{1}{c}{ 2014 } \\
      \midrule
	\multicolumn{4}{l}{\textcolor{FAOblue}{\textbf{\large{The setting}}}} \\ 
	 ~ Population, total (mln) & 51.6 ~ \ \ & 48.2 ~ \ \ & 44.9 ~ \ \ \\ 
	 ~ Population, rural (\% total population) & 17.1 ~ \ \ & 15.8 ~ \ \ & 13.7 ~ \ \ \\ 
	 ~ Govt expenditure on ag (\% total outlays) &  ~ \ \ & 3.2 ~ \ \ & \textit{2.2} ~ \ \ \\ 
	 ~ Area harvested (mln ha) & 36 ~ \ \ & 38 ~ \ \ & 63 ~ \ \ \\ 
	 ~ Cropping intensity ratio (\%) & 0.8 ~ \ \ & 0.9 ~ \ \ &  ~ \ \ \\ 
	 ~ Water resources (m\textsuperscript{3}/person/year) & \textit{3} ~ \ \ & \textit{3} ~ \ \ & \textit{3} ~ \ \ \\ 
	 ~ Area equipped for irrigation (1000 ha) &  ~ \ \ &  ~ \ \ & \textit{2\,167} ~ \ \ \\ 
	 ~ Area irrigated (\%) &  ~ \ \ & \textit{33.6} ~ \ \ &  ~ \ \ \\ 
	 ~ Employment in agriculture (\%) & 20.8 ~ \ \ & 20.6 ~ \ \ & \textit{17.2} ~ \ \ \\ 
	 ~ Employment in agriculture, female (\%) &  ~ \ \ &  ~ \ \ &  ~ \ \ \\ 
	 ~ Fertilizers, Nitrogen (nutrients per ha) &  ~ \ \ & 10.7 ~ \ \ & \textit{22.5} ~ \ \ \\ 
	 ~ Fertilizers, Phosphate (nutrients per ha) &  ~ \ \ & 0.8 ~ \ \ & \textit{5.3} ~ \ \ \\ 
	 ~ Fertilizers, Potash (nutrients per ha) &  ~ \ \ & 1 ~ \ \ & \textit{4.7} ~ \ \ \\ 
	 ~ Energy consump, power irrigation (mln kWh) & 4\,012 ~ \ \ & 4\,012 ~ \ \ & \textit{4\,012} ~ \ \ \\ 
	 ~ Agr value added per worker (constant US\$) & 1.8 ~ \ \ & 2.4 ~ \ \ & \textit{4.4} ~ \ \ \\ 
	\multicolumn{4}{l}{\textcolor{FAOblue}{\textbf{\large{Hunger dimensions}}}} \\ 
	 ~ Dietary energy supply (kcal/pc/day) &  ~ \ \ &  ~ \ \ &  ~ \ \ \\ 
	 ~ Average dietary energy supply adequacy (\%) & 132 ~ \ \ & 121 ~ \ \ & 126 ~ \ \ \\ 
	 ~ Dietary en supp, cereals/roots/tubers (\%) & 51 ~ \ \ & 49 ~ \ \ & \textit{42} ~ \ \ \\ 
	 ~ Prevalence of undernourishment (\%) & <5.0 ~ \ \ & <5.0 ~ \ \ & <5.0 ~ \ \ \\ 
	 ~ GDP per capita (US\$, PPP) & 8\,647 ~ \ \ & 5\,646 ~ \ \ & \textit{8\,508} ~ \ \ \\ 
	 ~ Domestic food price volatility (index) &  ~ \ \ & 7.2 ~ \ \ & 3.9 ~ \ \ \\ 
	 ~ Cereal import dependency ratio (\%) & 4.6 ~ \ \ & -21.5 ~ \ \ & \textit{-60.3} ~ \ \ \\ 
	 ~ Underweight, children under-5 (\%) &  ~ \ \ & 0.9 ~ \ \ &  ~ \ \ \\ 
	 ~ Improved water source (\% pop) & \textit{96.6} ~ \ \ & 97.1 ~ \ \ & \textit{98} ~ \ \ \\ 
	\multicolumn{4}{l}{\textcolor{FAOblue}{\textbf{\large{Food Supply}}}} \\ 
	 ~ Food production value, (2004-2006 mln I\$) & 22\,580 ~ \ \ & 16\,296 ~ \ \ & \textit{23\,686} ~ \ \ \\ 
	 ~ Agriculture, value added (\% GDP) & 20 ~ \ \ & 15 ~ \ \ & \textit{10} ~ \ \ \\ 
	 ~ Food exports (mln US\$)  & 602 ~ \ \ & 2\,129 ~ \ \ & \textit{16\,140} ~ \ \ \\ 
	 ~ Food imports (mln US\$)  & 605 ~ \ \ & 644 ~ \ \ & \textit{4\,577} ~ \ \ \\ 
	\multicolumn{4}{l}{\textit{\normalsize{Production indices (2004-06=100)}}} \\ 
	 ~ Net food & 132 ~ \ \ & 95 ~ \ \ & \textit{138} ~ \ \ \\ 
	 ~ Net crop & 99 ~ \ \ & 88 ~ \ \ & \textit{163} ~ \ \ \\ 
	 ~ Cereal & 95 ~ \ \ & 105 ~ \ \ & \textit{178} ~ \ \ \\ 
	 ~ Vegetable oils & 46 ~ \ \ & 67 ~ \ \ & \textit{288} ~ \ \ \\ 
	 ~ Roots and tubers & 106 ~ \ \ & 79 ~ \ \ & \textit{116} ~ \ \ \\ 
	 ~ Fruit and vegetables & 100 ~ \ \ & 82 ~ \ \ & \textit{140} ~ \ \ \\ 
	 ~ Sugar & 158 ~ \ \ & 80 ~ \ \ & \textit{59} ~ \ \ \\ 
	 ~ Livestock & 171 ~ \ \ & 103 ~ \ \ & \textit{105} ~ \ \ \\ 
	 ~ Milk & 141 ~ \ \ & 104 ~ \ \ & \textit{85} ~ \ \ \\ 
	 ~ Meat & 228 ~ \ \ & 106 ~ \ \ & \textit{122} ~ \ \ \\ 
	 ~ Fish  & 201 ~ \ \ & 114 ~ \ \ & \textit{87} ~ \ \ \\ 
	\multicolumn{4}{l}{\textit{\normalsize{Net trade (min US\$)}}} \\ 
	 ~ Cereals & -220 ~ \ \ & 991 ~ \ \ & \textit{7\,003} ~ \ \ \\ 
	 ~ Fruit and vegetables & 7 ~ \ \ & -21 ~ \ \ & \textit{-882} ~ \ \ \\ 
	 ~ Meat & 116 ~ \ \ & 173 ~ \ \ & \textit{-361} ~ \ \ \\ 
	 ~ Dairy products & 38 ~ \ \ & 121 ~ \ \ & \textit{318} ~ \ \ \\ 
	 ~ Fish & \textit{-48} ~ \ \ & -73 ~ \ \ & \textit{-735} ~ \ \ \\ 
	\multicolumn{4}{l}{\textcolor{FAOblue}{\textbf{\large{Environment}}}} \\ 
	 ~ Forest area (\%) & 16 ~ \ \ & 16 ~ \ \ & \textit{17} ~ \ \ \\ 
	 ~ Renewable water res withdrawn (\% of total) &  ~ \ \ & \textit{6} ~ \ \ & 6 ~ \ \ \\ 
	 ~ Terrestrial protect areas (\% total land area)  & 2 ~ \ \ & 4 ~ \ \ & \textit{4} ~ \ \ \\ 
	 ~ Organic area (\% total agricultural area) &  ~ \ \ & \textit{1} ~ \ \ & \textit{1} ~ \ \ \\ 
	 ~ Water withdrawal by agriculture (\% of total) &  ~ \ \ & \textit{6} ~ \ \ & 6 ~ \ \ \\ 
	 ~ Biofuel production (thousand kt of oil eq.) &  ~ \ \ &  ~ \ \ &  ~ \ \ \\ 
	 ~ Wood pellet prod. (min tonnes) &  ~ \ \ &  ~ \ \ & \textit{210} ~ \ \ \\ 
	 ~ GHG emissions from ag (Co2 eq, gigagrams) & 88 ~ \ \ & 14 ~ \ \ & \textit{5} ~ \ \ \\ 
       \toprule
      \end{tabular}
      \clearpage
\CountryData{ United Arab Emirates }
      \rowcolors{1}{FAOblue!10}{white}
      \begin{tabular}{L{3.9cm} R{1cm} R{1cm} R{1cm}}
      \toprule
      \multicolumn{1}{c}{} & \multicolumn{1}{c}{ 1992 } & \multicolumn{1}{c}{ 2002 } & \multicolumn{1}{c}{ 2014 } \\
      \midrule
	\multicolumn{4}{l}{\textcolor{FAOblue}{\textbf{\large{The setting}}}} \\ 
	 ~ Population, total (mln) & 2 ~ \ \ & 3.2 ~ \ \ & 9.4 ~ \ \ \\ 
	 ~ Population, rural (\% total population) & 0.4 ~ \ \ & 0.6 ~ \ \ & 1.4 ~ \ \ \\ 
	 ~ Govt expenditure on ag (\% total outlays) &  ~ \ \ &  ~ \ \ &  ~ \ \ \\ 
	 ~ Area harvested (mln ha) & 1 ~ \ \ & 1 ~ \ \ & 1 ~ \ \ \\ 
	 ~ Cropping intensity ratio (\%) & 3 ~ \ \ & 1.8 ~ \ \ &  ~ \ \ \\ 
	 ~ Water resources (m\textsuperscript{3}/person/year) & \textit{0} ~ \ \ & \textit{0} ~ \ \ & \textit{0} ~ \ \ \\ 
	 ~ Area equipped for irrigation (1000 ha) &  ~ \ \ &  ~ \ \ & \textit{92} ~ \ \ \\ 
	 ~ Area irrigated (\%) &  ~ \ \ &  ~ \ \ & \textit{82.6} ~ \ \ \\ 
	 ~ Employment in agriculture (\%) & \textit{8} ~ \ \ & \textit{4.9} ~ \ \ & \textit{3.8} ~ \ \ \\ 
	 ~ Employment in agriculture, female (\%) & \textit{0.1} ~ \ \ & \textit{0.1} ~ \ \ & \textit{0.1} ~ \ \ \\ 
	 ~ Fertilizers, Nitrogen (nutrients per ha) &  ~ \ \ & 68.2 ~ \ \ & \textit{61.1} ~ \ \ \\ 
	 ~ Fertilizers, Phosphate (nutrients per ha) &  ~ \ \ & 6.1 ~ \ \ & \textit{0} ~ \ \ \\ 
	 ~ Fertilizers, Potash (nutrients per ha) &  ~ \ \ & 14 ~ \ \ & \textit{10.6} ~ \ \ \\ 
	 ~ Energy consump, power irrigation (mln kWh) & \textit{100} ~ \ \ & 100 ~ \ \ & \textit{481} ~ \ \ \\ 
	 ~ Agr value added per worker (constant US\$) &  ~ \ \ & 31.4 ~ \ \ & \textit{12.1} ~ \ \ \\ 
	\multicolumn{4}{l}{\textcolor{FAOblue}{\textbf{\large{Hunger dimensions}}}} \\ 
	 ~ Dietary energy supply (kcal/pc/day) & 3\,170 ~ \ \ & 3\,343 ~ \ \ & 3\,328 ~ \ \ \\ 
	 ~ Average dietary energy supply adequacy (\%) & 133 ~ \ \ & 134 ~ \ \ & 129 ~ \ \ \\ 
	 ~ Dietary en supp, cereals/roots/tubers (\%) & 32 ~ \ \ & 43 ~ \ \ & \textit{42} ~ \ \ \\ 
	 ~ Prevalence of undernourishment (\%) & <5.0 ~ \ \ & <5.0 ~ \ \ & <5.0 ~ \ \ \\ 
	 ~ GDP per capita (US\$, PPP) & 108\,273 ~ \ \ & 106\,186 ~ \ \ & \textit{57\,045} ~ \ \ \\ 
	 ~ Domestic food price volatility (index) &  ~ \ \ &  ~ \ \ &  ~ \ \ \\ 
	 ~ Cereal import dependency ratio (\%) & 99.6 ~ \ \ & 100 ~ \ \ & \textit{94.7} ~ \ \ \\ 
	 ~ Underweight, children under-5 (\%) &  ~ \ \ &  ~ \ \ &  ~ \ \ \\ 
	 ~ Improved water source (\% pop) & 99.7 ~ \ \ & 99.7 ~ \ \ & \textit{99.6} ~ \ \ \\ 
	\multicolumn{4}{l}{\textcolor{FAOblue}{\textbf{\large{Food Supply}}}} \\ 
	 ~ Food production value, (2004-2006 mln I\$) & 307 ~ \ \ & 660 ~ \ \ & \textit{439} ~ \ \ \\ 
	 ~ Agriculture, value added (\% GDP) & 1 ~ \ \ & 2 ~ \ \ & \textit{1} ~ \ \ \\ 
	 ~ Food exports (mln US\$)  & 502 ~ \ \ & 995 ~ \ \ & \textit{2\,579} ~ \ \ \\ 
	 ~ Food imports (mln US\$)  & 1\,598 ~ \ \ & 3\,110 ~ \ \ & \textit{10\,821} ~ \ \ \\ 
	\multicolumn{4}{l}{\textit{\normalsize{Production indices (2004-06=100)}}} \\ 
	 ~ Net food & 48 ~ \ \ & 103 ~ \ \ & \textit{68} ~ \ \ \\ 
	 ~ Net crop & 51 ~ \ \ & 106 ~ \ \ & \textit{43} ~ \ \ \\ 
	 ~ Cereal & 13\,624 ~ \ \ & 497 ~ \ \ & \textit{579\,943} ~ \ \ \\ 
	 ~ Vegetable oils &  ~ \ \ &  ~ \ \ &  ~ \ \ \\ 
	 ~ Roots and tubers & 61 ~ \ \ & 142 ~ \ \ & \textit{141} ~ \ \ \\ 
	 ~ Fruit and vegetables & 51 ~ \ \ & 106 ~ \ \ & \textit{38} ~ \ \ \\ 
	 ~ Sugar &  ~ \ \ &  ~ \ \ &  ~ \ \ \\ 
	 ~ Livestock & 47 ~ \ \ & 92 ~ \ \ & \textit{153} ~ \ \ \\ 
	 ~ Milk & 45 ~ \ \ & 89 ~ \ \ & \textit{147} ~ \ \ \\ 
	 ~ Meat & 46 ~ \ \ & 91 ~ \ \ & \textit{153} ~ \ \ \\ 
	 ~ Fish  & 102 ~ \ \ & 105 ~ \ \ & \textit{78} ~ \ \ \\ 
	\multicolumn{4}{l}{\textit{\normalsize{Net trade (min US\$)}}} \\ 
	 ~ Cereals & -165 ~ \ \ & -292 ~ \ \ & \textit{-1\,790} ~ \ \ \\ 
	 ~ Fruit and vegetables & -320 ~ \ \ & -596 ~ \ \ & \textit{-1\,907} ~ \ \ \\ 
	 ~ Meat & -165 ~ \ \ & -303 ~ \ \ & \textit{-1\,217} ~ \ \ \\ 
	 ~ Dairy products & -164 ~ \ \ & -253 ~ \ \ & \textit{-617} ~ \ \ \\ 
	 ~ Fish & -4 ~ \ \ & -69 ~ \ \ & \textit{-396} ~ \ \ \\ 
	\multicolumn{4}{l}{\textcolor{FAOblue}{\textbf{\large{Environment}}}} \\ 
	 ~ Forest area (\%) & 3 ~ \ \ & 4 ~ \ \ & \textit{4} ~ \ \ \\ 
	 ~ Renewable water res withdrawn (\% of total) &  ~ \ \ & \textit{83} ~ \ \ & 83 ~ \ \ \\ 
	 ~ Terrestrial protect areas (\% total land area)  & 0 ~ \ \ & 5 ~ \ \ & \textit{18} ~ \ \ \\ 
	 ~ Organic area (\% total agricultural area) &  ~ \ \ &  ~ \ \ & \textit{1} ~ \ \ \\ 
	 ~ Water withdrawal by agriculture (\% of total) &  ~ \ \ & \textit{83} ~ \ \ & 83 ~ \ \ \\ 
	 ~ Biofuel production (thousand kt of oil eq.) &  ~ \ \ &  ~ \ \ &  ~ \ \ \\ 
	 ~ Wood pellet prod. (min tonnes) &  ~ \ \ &  ~ \ \ &  ~ \ \ \\ 
	 ~ GHG emissions from ag (Co2 eq, gigagrams) & -1 ~ \ \ & 1 ~ \ \ & \textit{1} ~ \ \ \\ 
       \toprule
      \end{tabular}
      \clearpage
\CountryData{ United Kingdom }
      \rowcolors{1}{FAOblue!10}{white}
      \begin{tabular}{L{3.9cm} R{1cm} R{1cm} R{1cm}}
      \toprule
      \multicolumn{1}{c}{} & \multicolumn{1}{c}{ 1992 } & \multicolumn{1}{c}{ 2002 } & \multicolumn{1}{c}{ 2014 } \\
      \midrule
	\multicolumn{4}{l}{\textcolor{FAOblue}{\textbf{\large{The setting}}}} \\ 
	 ~ Population, total (mln) & 57.7 ~ \ \ & 59.7 ~ \ \ & 63.7 ~ \ \ \\ 
	 ~ Population, rural (\% total population) & 12.7 ~ \ \ & 12.8 ~ \ \ & 12.8 ~ \ \ \\ 
	 ~ Govt expenditure on ag (\% total outlays) &  ~ \ \ &  ~ \ \ &  ~ \ \ \\ 
	 ~ Area harvested (mln ha) & 22 ~ \ \ & 23 ~ \ \ & 20 ~ \ \ \\ 
	 ~ Cropping intensity ratio (\%) & 1.2 ~ \ \ & 1.4 ~ \ \ &  ~ \ \ \\ 
	 ~ Water resources (m\textsuperscript{3}/person/year) & \textit{3} ~ \ \ & \textit{2} ~ \ \ & \textit{2} ~ \ \ \\ 
	 ~ Area equipped for irrigation (1000 ha) &  ~ \ \ &  ~ \ \ & \textit{95} ~ \ \ \\ 
	 ~ Area irrigated (\%) &  ~ \ \ &  ~ \ \ & \textit{60.6} ~ \ \ \\ 
	 ~ Employment in agriculture (\%) & 2.2 ~ \ \ & 1.4 ~ \ \ & \textit{1.2} ~ \ \ \\ 
	 ~ Employment in agriculture, female (\%) & 1.1 ~ \ \ & 0.7 ~ \ \ & \textit{0.6} ~ \ \ \\ 
	 ~ Fertilizers, Nitrogen (nutrients per ha) &  ~ \ \ & 69.7 ~ \ \ & \textit{57.9} ~ \ \ \\ 
	 ~ Fertilizers, Phosphate (nutrients per ha) &  ~ \ \ & 16.1 ~ \ \ & \textit{11.3} ~ \ \ \\ 
	 ~ Fertilizers, Potash (nutrients per ha) &  ~ \ \ & 23 ~ \ \ & \textit{15.5} ~ \ \ \\ 
	 ~ Energy consump, power irrigation (mln kWh) & 11 ~ \ \ & 11 ~ \ \ & \textit{214} ~ \ \ \\ 
	 ~ Agr value added per worker (constant US\$) & 23.1 ~ \ \ & 25.8 ~ \ \ & \textit{28.2} ~ \ \ \\ 
	\multicolumn{4}{l}{\textcolor{FAOblue}{\textbf{\large{Hunger dimensions}}}} \\ 
	 ~ Dietary energy supply (kcal/pc/day) &  ~ \ \ &  ~ \ \ &  ~ \ \ \\ 
	 ~ Average dietary energy supply adequacy (\%) & 130 ~ \ \ & 137 ~ \ \ & 137 ~ \ \ \\ 
	 ~ Dietary en supp, cereals/roots/tubers (\%) & 28 ~ \ \ & 32 ~ \ \ & \textit{32} ~ \ \ \\ 
	 ~ Prevalence of undernourishment (\%) & <5.0 ~ \ \ & <5.0 ~ \ \ & <5.0 ~ \ \ \\ 
	 ~ GDP per capita (US\$, PPP) & 26\,062 ~ \ \ & 33\,954 ~ \ \ & \textit{36\,932} ~ \ \ \\ 
	 ~ Domestic food price volatility (index) &  ~ \ \ & 4.7 ~ \ \ & 5 ~ \ \ \\ 
	 ~ Cereal import dependency ratio (\%) & -15.2 ~ \ \ & -2.2 ~ \ \ & \textit{-3.2} ~ \ \ \\ 
	 ~ Underweight, children under-5 (\%) &  ~ \ \ &  ~ \ \ &  ~ \ \ \\ 
	 ~ Improved water source (\% pop) & 100 ~ \ \ & 100 ~ \ \ & \textit{100} ~ \ \ \\ 
	\multicolumn{4}{l}{\textcolor{FAOblue}{\textbf{\large{Food Supply}}}} \\ 
	 ~ Food production value, (2004-2006 mln I\$) & 17\,563 ~ \ \ & 16\,011 ~ \ \ & \textit{15\,878} ~ \ \ \\ 
	 ~ Agriculture, value added (\% GDP) &  ~ \ \ & 1 ~ \ \ & \textit{1} ~ \ \ \\ 
	 ~ Food exports (mln US\$)  & 7\,408 ~ \ \ & 6\,774 ~ \ \ & \textit{13\,824} ~ \ \ \\ 
	 ~ Food imports (mln US\$)  & 17\,232 ~ \ \ & 20\,036 ~ \ \ & \textit{42\,602} ~ \ \ \\ 
	\multicolumn{4}{l}{\textit{\normalsize{Production indices (2004-06=100)}}} \\ 
	 ~ Net food & 111 ~ \ \ & 101 ~ \ \ & \textit{100} ~ \ \ \\ 
	 ~ Net crop & 110 ~ \ \ & 105 ~ \ \ & \textit{93} ~ \ \ \\ 
	 ~ Cereal & 101 ~ \ \ & 108 ~ \ \ & \textit{91} ~ \ \ \\ 
	 ~ Vegetable oils & 77 ~ \ \ & 80 ~ \ \ & \textit{117} ~ \ \ \\ 
	 ~ Roots and tubers & 128 ~ \ \ & 115 ~ \ \ & \textit{92} ~ \ \ \\ 
	 ~ Fruit and vegetables & 145 ~ \ \ & 97 ~ \ \ & \textit{102} ~ \ \ \\ 
	 ~ Sugar & 121 ~ \ \ & 114 ~ \ \ & \textit{96} ~ \ \ \\ 
	 ~ Livestock & 108 ~ \ \ & 100 ~ \ \ & \textit{103} ~ \ \ \\ 
	 ~ Milk & 102 ~ \ \ & 103 ~ \ \ & \textit{96} ~ \ \ \\ 
	 ~ Meat & 113 ~ \ \ & 98 ~ \ \ & \textit{107} ~ \ \ \\ 
	 ~ Fish  & 104 ~ \ \ & 109 ~ \ \ & \textit{99} ~ \ \ \\ 
	\multicolumn{4}{l}{\textit{\normalsize{Net trade (min US\$)}}} \\ 
	 ~ Cereals & 367 ~ \ \ & -295 ~ \ \ & \textit{-2\,032} ~ \ \ \\ 
	 ~ Fruit and vegetables & -4\,958 ~ \ \ & -6\,085 ~ \ \ & \textit{-11\,552} ~ \ \ \\ 
	 ~ Meat & -2\,123 ~ \ \ & -3\,557 ~ \ \ & \textit{-6\,490} ~ \ \ \\ 
	 ~ Dairy products & -908 ~ \ \ & -703 ~ \ \ & \textit{-2\,053} ~ \ \ \\ 
	 ~ Fish & -761 ~ \ \ & -974 ~ \ \ & \textit{-1\,703} ~ \ \ \\ 
	\multicolumn{4}{l}{\textcolor{FAOblue}{\textbf{\large{Environment}}}} \\ 
	 ~ Forest area (\%) & 11 ~ \ \ & 12 ~ \ \ & \textit{12} ~ \ \ \\ 
	 ~ Renewable water res withdrawn (\% of total) &  ~ \ \ &  ~ \ \ & 10 ~ \ \ \\ 
	 ~ Terrestrial protect areas (\% total land area)  & 23 ~ \ \ & 25 ~ \ \ & \textit{28} ~ \ \ \\ 
	 ~ Organic area (\% total agricultural area) &  ~ \ \ & \textit{4} ~ \ \ & \textit{3} ~ \ \ \\ 
	 ~ Water withdrawal by agriculture (\% of total) &  ~ \ \ &  ~ \ \ & 10 ~ \ \ \\ 
	 ~ Biofuel production (thousand kt of oil eq.) & 10 ~ \ \ & 60 ~ \ \ & \textit{4\,306} ~ \ \ \\ 
	 ~ Wood pellet prod. (min tonnes) &  ~ \ \ &  ~ \ \ & \textit{301} ~ \ \ \\ 
	 ~ GHG emissions from ag (Co2 eq, gigagrams) & 58 ~ \ \ & 45 ~ \ \ & \textit{41} ~ \ \ \\ 
       \toprule
      \end{tabular}
      \clearpage
\CountryData{ United States of America }
      \rowcolors{1}{FAOblue!10}{white}
      \begin{tabular}{L{3.9cm} R{1cm} R{1cm} R{1cm}}
      \toprule
      \multicolumn{1}{c}{} & \multicolumn{1}{c}{ 1992 } & \multicolumn{1}{c}{ 2002 } & \multicolumn{1}{c}{ 2014 } \\
      \midrule
	\multicolumn{4}{l}{\textcolor{FAOblue}{\textbf{\large{The setting}}}} \\ 
	 ~ Population, total (mln) & 259.7 ~ \ \ & 290.3 ~ \ \ & 322.6 ~ \ \ \\ 
	 ~ Population, rural (\% total population) & 62.1 ~ \ \ & 58.7 ~ \ \ & 54.4 ~ \ \ \\ 
	 ~ Govt expenditure on ag (\% total outlays) &  ~ \ \ & 1.1 ~ \ \ & \textit{0.6} ~ \ \ \\ 
	 ~ Area harvested (mln ha) & 353 ~ \ \ & 297 ~ \ \ & 437 ~ \ \ \\ 
	 ~ Cropping intensity ratio (\%) & 0.8 ~ \ \ & 0.7 ~ \ \ &  ~ \ \ \\ 
	 ~ Water resources (m\textsuperscript{3}/person/year) & \textit{12} ~ \ \ & \textit{10} ~ \ \ & \textit{10} ~ \ \ \\ 
	 ~ Area equipped for irrigation (1000 ha) &  ~ \ \ &  ~ \ \ & \textit{26\,400} ~ \ \ \\ 
	 ~ Area irrigated (\%) &  ~ \ \ &  ~ \ \ & \textit{86} ~ \ \ \\ 
	 ~ Employment in agriculture (\%) & 2.9 ~ \ \ & 2.5 ~ \ \ & \textit{1.6} ~ \ \ \\ 
	 ~ Employment in agriculture, female (\%) & 1.3 ~ \ \ & 1.4 ~ \ \ & \textit{0.8} ~ \ \ \\ 
	 ~ Fertilizers, Nitrogen (nutrients per ha) &  ~ \ \ & 26.5 ~ \ \ & \textit{29.6} ~ \ \ \\ 
	 ~ Fertilizers, Phosphate (nutrients per ha) &  ~ \ \ & 9.7 ~ \ \ & \textit{9.8} ~ \ \ \\ 
	 ~ Fertilizers, Potash (nutrients per ha) &  ~ \ \ & 10.9 ~ \ \ & \textit{10.4} ~ \ \ \\ 
	 ~ Energy consump, power irrigation (mln kWh) &  ~ \ \ & \textit{26\,904} ~ \ \ & \textit{26\,904} ~ \ \ \\ 
	 ~ Agr value added per worker (constant US\$) &  ~ \ \ & 39.6 ~ \ \ & \textit{63.3} ~ \ \ \\ 
	\multicolumn{4}{l}{\textcolor{FAOblue}{\textbf{\large{Hunger dimensions}}}} \\ 
	 ~ Dietary energy supply (kcal/pc/day) &  ~ \ \ &  ~ \ \ &  ~ \ \ \\ 
	 ~ Average dietary energy supply adequacy (\%) & 141 ~ \ \ & 148 ~ \ \ & 147 ~ \ \ \\ 
	 ~ Dietary en supp, cereals/roots/tubers (\%) & 26 ~ \ \ & 25 ~ \ \ & \textit{25} ~ \ \ \\ 
	 ~ Prevalence of undernourishment (\%) & <5.0 ~ \ \ & <5.0 ~ \ \ & <5.0 ~ \ \ \\ 
	 ~ GDP per capita (US\$, PPP) & 37\,289 ~ \ \ & 46\,368 ~ \ \ & \textit{51\,340} ~ \ \ \\ 
	 ~ Domestic food price volatility (index) &  ~ \ \ & 0 ~ \ \ & 0 ~ \ \ \\ 
	 ~ Cereal import dependency ratio (\%) & -40.6 ~ \ \ & -32 ~ \ \ & \textit{-24} ~ \ \ \\ 
	 ~ Underweight, children under-5 (\%) & \textit{0.9} ~ \ \ & 1.3 ~ \ \ & \textit{0.5} ~ \ \ \\ 
	 ~ Improved water source (\% pop) & 98.5 ~ \ \ & 98.9 ~ \ \ & \textit{99.2} ~ \ \ \\ 
	\multicolumn{4}{l}{\textcolor{FAOblue}{\textbf{\large{Food Supply}}}} \\ 
	 ~ Food production value, (2004-2006 mln I\$) & 160\,418 ~ \ \ & 179\,287 ~ \ \ & \textit{215\,750} ~ \ \ \\ 
	 ~ Agriculture, value added (\% GDP) &  ~ \ \ & 1 ~ \ \ & \textit{1} ~ \ \ \\ 
	 ~ Food exports (mln US\$)  & 32\,009 ~ \ \ & 39\,771 ~ \ \ & \textit{111\,636} ~ \ \ \\ 
	 ~ Food imports (mln US\$)  & 17\,265 ~ \ \ & 27\,751 ~ \ \ & \textit{63\,704} ~ \ \ \\ 
	\multicolumn{4}{l}{\textit{\normalsize{Production indices (2004-06=100)}}} \\ 
	 ~ Net food & 84 ~ \ \ & 94 ~ \ \ & \textit{113} ~ \ \ \\ 
	 ~ Net crop & 87 ~ \ \ & 88 ~ \ \ & \textit{109} ~ \ \ \\ 
	 ~ Cereal & 97 ~ \ \ & 82 ~ \ \ & \textit{119} ~ \ \ \\ 
	 ~ Vegetable oils & 72 ~ \ \ & 87 ~ \ \ & \textit{102} ~ \ \ \\ 
	 ~ Roots and tubers & 96 ~ \ \ & 104 ~ \ \ & \textit{100} ~ \ \ \\ 
	 ~ Fruit and vegetables & 87 ~ \ \ & 102 ~ \ \ & \textit{98} ~ \ \ \\ 
	 ~ Sugar & 91 ~ \ \ & 105 ~ \ \ & \textit{98} ~ \ \ \\ 
	 ~ Livestock & 83 ~ \ \ & 99 ~ \ \ & \textit{108} ~ \ \ \\ 
	 ~ Milk & 85 ~ \ \ & 96 ~ \ \ & \textit{114} ~ \ \ \\ 
	 ~ Meat & 82 ~ \ \ & 100 ~ \ \ & \textit{106} ~ \ \ \\ 
	 ~ Fish  & 103 ~ \ \ & 100 ~ \ \ & \textit{104} ~ \ \ \\ 
	\multicolumn{4}{l}{\textit{\normalsize{Net trade (min US\$)}}} \\ 
	 ~ Cereals & 11\,005 ~ \ \ & 9\,002 ~ \ \ & \textit{16\,260} ~ \ \ \\ 
	 ~ Fruit and vegetables & -492 ~ \ \ & -1\,955 ~ \ \ & \textit{-2\,493} ~ \ \ \\ 
	 ~ Meat & 1\,302 ~ \ \ & 2\,210 ~ \ \ & \textit{11\,327} ~ \ \ \\ 
	 ~ Dairy products & 6 ~ \ \ & -508 ~ \ \ & \textit{2\,126} ~ \ \ \\ 
	 ~ Fish & -2\,442 ~ \ \ & -7\,374 ~ \ \ & \textit{-11\,803} ~ \ \ \\ 
	\multicolumn{4}{l}{\textcolor{FAOblue}{\textbf{\large{Environment}}}} \\ 
	 ~ Forest area (\%) & 32 ~ \ \ & 33 ~ \ \ & \textit{33} ~ \ \ \\ 
	 ~ Renewable water res withdrawn (\% of total) &  ~ \ \ & \textit{40} ~ \ \ & 40 ~ \ \ \\ 
	 ~ Terrestrial protect areas (\% total land area)  & 12 ~ \ \ & 12 ~ \ \ & \textit{14} ~ \ \ \\ 
	 ~ Organic area (\% total agricultural area) &  ~ \ \ & \textit{0} ~ \ \ & \textit{0} ~ \ \ \\ 
	 ~ Water withdrawal by agriculture (\% of total) &  ~ \ \ & \textit{40} ~ \ \ & 40 ~ \ \ \\ 
	 ~ Biofuel production (thousand kt of oil eq.) & 760 ~ \ \ & 2\,337 ~ \ \ & \textit{29\,835} ~ \ \ \\ 
	 ~ Wood pellet prod. (min tonnes) &  ~ \ \ &  ~ \ \ & \textit{5\,700} ~ \ \ \\ 
	 ~ GHG emissions from ag (Co2 eq, gigagrams) & 57 ~ \ \ & -22 ~ \ \ & \textit{-58} ~ \ \ \\ 
       \toprule
      \end{tabular}
      \clearpage
\CountryData{ Uruguay }
      \rowcolors{1}{FAOblue!10}{white}
      \begin{tabular}{L{3.9cm} R{1cm} R{1cm} R{1cm}}
      \toprule
      \multicolumn{1}{c}{} & \multicolumn{1}{c}{ 1992 } & \multicolumn{1}{c}{ 2002 } & \multicolumn{1}{c}{ 2014 } \\
      \midrule
	\multicolumn{4}{l}{\textcolor{FAOblue}{\textbf{\large{The setting}}}} \\ 
	 ~ Population, total (mln) & 3.2 ~ \ \ & 3.3 ~ \ \ & 3.4 ~ \ \ \\ 
	 ~ Population, rural (\% total population) & 0.3 ~ \ \ & 0.3 ~ \ \ & 0.2 ~ \ \ \\ 
	 ~ Govt expenditure on ag (\% total outlays) &  ~ \ \ & 1.5 ~ \ \ & \textit{1.4} ~ \ \ \\ 
	 ~ Area harvested (mln ha) & 2 ~ \ \ & 2 ~ \ \ & 4 ~ \ \ \\ 
	 ~ Cropping intensity ratio (\%) & 0.1 ~ \ \ & 0.1 ~ \ \ &  ~ \ \ \\ 
	 ~ Water resources (m\textsuperscript{3}/person/year) & \textit{54} ~ \ \ & \textit{52} ~ \ \ & \textit{51} ~ \ \ \\ 
	 ~ Area equipped for irrigation (1000 ha) &  ~ \ \ &  ~ \ \ & \textit{238} ~ \ \ \\ 
	 ~ Area irrigated (\%) &  ~ \ \ &  ~ \ \ & \textit{100} ~ \ \ \\ 
	 ~ Employment in agriculture (\%) & 4.5 ~ \ \ & 4.2 ~ \ \ & \textit{10.9} ~ \ \ \\ 
	 ~ Employment in agriculture, female (\%) & 1 ~ \ \ & 1.3 ~ \ \ & \textit{4.8} ~ \ \ \\ 
	 ~ Fertilizers, Nitrogen (nutrients per ha) &  ~ \ \ & 1.8 ~ \ \ & \textit{9.6} ~ \ \ \\ 
	 ~ Fertilizers, Phosphate (nutrients per ha) &  ~ \ \ & 3.9 ~ \ \ & \textit{10.5} ~ \ \ \\ 
	 ~ Fertilizers, Potash (nutrients per ha) &  ~ \ \ & 0 ~ \ \ & \textit{2.1} ~ \ \ \\ 
	 ~ Energy consump, power irrigation (mln kWh) &  ~ \ \ & 0 ~ \ \ & \textit{82} ~ \ \ \\ 
	 ~ Agr value added per worker (constant US\$) & 5.9 ~ \ \ & 6.6 ~ \ \ & \textit{11} ~ \ \ \\ 
	\multicolumn{4}{l}{\textcolor{FAOblue}{\textbf{\large{Hunger dimensions}}}} \\ 
	 ~ Dietary energy supply (kcal/pc/day) & 2\,743 ~ \ \ & 2\,827 ~ \ \ & 2\,900 ~ \ \ \\ 
	 ~ Average dietary energy supply adequacy (\%) & 116 ~ \ \ & 119 ~ \ \ & 120 ~ \ \ \\ 
	 ~ Dietary en supp, cereals/roots/tubers (\%) & 38 ~ \ \ & 43 ~ \ \ & \textit{45} ~ \ \ \\ 
	 ~ Prevalence of undernourishment (\%) & 6.4 ~ \ \ & <5.0 ~ \ \ & <5.0 ~ \ \ \\ 
	 ~ GDP per capita (US\$, PPP) & 10\,595 ~ \ \ & 11\,142 ~ \ \ & \textit{18\,966} ~ \ \ \\ 
	 ~ Domestic food price volatility (index) &  ~ \ \ & 8.3 ~ \ \ & 6.4 ~ \ \ \\ 
	 ~ Cereal import dependency ratio (\%) & -56 ~ \ \ & -80.4 ~ \ \ & \textit{-123.7} ~ \ \ \\ 
	 ~ Underweight, children under-5 (\%) &  ~ \ \ & 5.4 ~ \ \ & \textit{4.5} ~ \ \ \\ 
	 ~ Improved water source (\% pop) & 95.6 ~ \ \ & 97.6 ~ \ \ & \textit{99.5} ~ \ \ \\ 
	\multicolumn{4}{l}{\textcolor{FAOblue}{\textbf{\large{Food Supply}}}} \\ 
	 ~ Food production value, (2004-2006 mln I\$) & 2\,086 ~ \ \ & 2\,418 ~ \ \ & \textit{4\,211} ~ \ \ \\ 
	 ~ Agriculture, value added (\% GDP) & 9 ~ \ \ & 9 ~ \ \ & \textit{10} ~ \ \ \\ 
	 ~ Food exports (mln US\$)  & 536 ~ \ \ & 741 ~ \ \ & \textit{5\,368} ~ \ \ \\ 
	 ~ Food imports (mln US\$)  & 125 ~ \ \ & 204 ~ \ \ & \textit{701} ~ \ \ \\ 
	\multicolumn{4}{l}{\textit{\normalsize{Production indices (2004-06=100)}}} \\ 
	 ~ Net food & 64 ~ \ \ & 75 ~ \ \ & \textit{130} ~ \ \ \\ 
	 ~ Net crop & 58 ~ \ \ & 66 ~ \ \ & \textit{197} ~ \ \ \\ 
	 ~ Cereal & 57 ~ \ \ & 67 ~ \ \ & \textit{149} ~ \ \ \\ 
	 ~ Vegetable oils & 21 ~ \ \ & 51 ~ \ \ & \textit{384} ~ \ \ \\ 
	 ~ Roots and tubers & 96 ~ \ \ & 94 ~ \ \ & \textit{83} ~ \ \ \\ 
	 ~ Fruit and vegetables & 77 ~ \ \ & 79 ~ \ \ & \textit{102} ~ \ \ \\ 
	 ~ Sugar & 459 ~ \ \ & 120 ~ \ \ & \textit{239} ~ \ \ \\ 
	 ~ Livestock & 72 ~ \ \ & 79 ~ \ \ & \textit{104} ~ \ \ \\ 
	 ~ Milk & 68 ~ \ \ & 93 ~ \ \ & \textit{134} ~ \ \ \\ 
	 ~ Meat & 68 ~ \ \ & 74 ~ \ \ & \textit{95} ~ \ \ \\ 
	 ~ Fish  & 99 ~ \ \ & 85 ~ \ \ & \textit{47} ~ \ \ \\ 
	\multicolumn{4}{l}{\textit{\normalsize{Net trade (min US\$)}}} \\ 
	 ~ Cereals & 91 ~ \ \ & 132 ~ \ \ & \textit{1\,154} ~ \ \ \\ 
	 ~ Fruit and vegetables & 23 ~ \ \ & 6 ~ \ \ & \textit{-35} ~ \ \ \\ 
	 ~ Meat & 234 ~ \ \ & 296 ~ \ \ & \textit{1\,555} ~ \ \ \\ 
	 ~ Dairy products & 52 ~ \ \ & 125 ~ \ \ & \textit{751} ~ \ \ \\ 
	 ~ Fish & 96 ~ \ \ & 90 ~ \ \ & \textit{134} ~ \ \ \\ 
	\multicolumn{4}{l}{\textcolor{FAOblue}{\textbf{\large{Environment}}}} \\ 
	 ~ Forest area (\%) & 6 ~ \ \ & 8 ~ \ \ & \textit{10} ~ \ \ \\ 
	 ~ Renewable water res withdrawn (\% of total) &  ~ \ \ & \textit{87} ~ \ \ & 87 ~ \ \ \\ 
	 ~ Terrestrial protect areas (\% total land area)  & 0 ~ \ \ & 0 ~ \ \ & \textit{3} ~ \ \ \\ 
	 ~ Organic area (\% total agricultural area) &  ~ \ \ & \textit{5} ~ \ \ & \textit{6} ~ \ \ \\ 
	 ~ Water withdrawal by agriculture (\% of total) &  ~ \ \ & \textit{87} ~ \ \ & 87 ~ \ \ \\ 
	 ~ Biofuel production (thousand kt of oil eq.) & 2 ~ \ \ & 1 ~ \ \ & \textit{302} ~ \ \ \\ 
	 ~ Wood pellet prod. (min tonnes) &  ~ \ \ &  ~ \ \ & \textit{3} ~ \ \ \\ 
	 ~ GHG emissions from ag (Co2 eq, gigagrams) & 2 ~ \ \ & 13 ~ \ \ & \textit{4} ~ \ \ \\ 
       \toprule
      \end{tabular}
      \clearpage
\CountryData{ Uzbekistan }
      \rowcolors{1}{FAOblue!10}{white}
      \begin{tabular}{L{3.9cm} R{1cm} R{1cm} R{1cm}}
      \toprule
      \multicolumn{1}{c}{} & \multicolumn{1}{c}{ 1992 } & \multicolumn{1}{c}{ 2002 } & \multicolumn{1}{c}{ 2014 } \\
      \midrule
	\multicolumn{4}{l}{\textcolor{FAOblue}{\textbf{\large{The setting}}}} \\ 
	 ~ Population, total (mln) & 21.5 ~ \ \ & 25.3 ~ \ \ & 29.3 ~ \ \ \\ 
	 ~ Population, rural (\% total population) & 13 ~ \ \ & 15.9 ~ \ \ & 18.7 ~ \ \ \\ 
	 ~ Govt expenditure on ag (\% total outlays) &  ~ \ \ &  ~ \ \ &  ~ \ \ \\ 
	 ~ Area harvested (mln ha) & 4 ~ \ \ & 6 ~ \ \ & 7 ~ \ \ \\ 
	 ~ Cropping intensity ratio (\%) & 0.2 ~ \ \ & 0.2 ~ \ \ &  ~ \ \ \\ 
	 ~ Water resources (m\textsuperscript{3}/person/year) & \textit{2} ~ \ \ & \textit{2} ~ \ \ & \textit{2} ~ \ \ \\ 
	 ~ Area equipped for irrigation (1000 ha) &  ~ \ \ &  ~ \ \ & \textit{4\,215} ~ \ \ \\ 
	 ~ Area irrigated (\%) &  ~ \ \ & \textit{88.1} ~ \ \ & \textit{88.1} ~ \ \ \\ 
	 ~ Employment in agriculture (\%) & \textit{41.2} ~ \ \ & \textit{38.5} ~ \ \ &  ~ \ \ \\ 
	 ~ Employment in agriculture, female (\%) &  ~ \ \ &  ~ \ \ &  ~ \ \ \\ 
	 ~ Fertilizers, Nitrogen (nutrients per ha) &  ~ \ \ & 0 ~ \ \ & \textit{27.4} ~ \ \ \\ 
	 ~ Fertilizers, Phosphate (nutrients per ha) &  ~ \ \ & 0 ~ \ \ & \textit{4.5} ~ \ \ \\ 
	 ~ Fertilizers, Potash (nutrients per ha) &  ~ \ \ & 0 ~ \ \ & \textit{1.3} ~ \ \ \\ 
	 ~ Energy consump, power irrigation (mln kWh) & \textit{11} ~ \ \ & 11 ~ \ \ & \textit{11} ~ \ \ \\ 
	 ~ Agr value added per worker (constant US\$) & 0.9 ~ \ \ & 1.1 ~ \ \ & \textit{2.2} ~ \ \ \\ 
	\multicolumn{4}{l}{\textcolor{FAOblue}{\textbf{\large{Hunger dimensions}}}} \\ 
	 ~ Dietary energy supply (kcal/pc/day) & 2\,707 ~ \ \ & 2\,297 ~ \ \ & 2\,851 ~ \ \ \\ 
	 ~ Average dietary energy supply adequacy (\%) & 125 ~ \ \ & 101 ~ \ \ & 122 ~ \ \ \\ 
	 ~ Dietary en supp, cereals/roots/tubers (\%) & 61 ~ \ \ & 61 ~ \ \ & \textit{58} ~ \ \ \\ 
	 ~ Prevalence of undernourishment (\%) & <5.0 ~ \ \ & 16.8 ~ \ \ & <5.0 ~ \ \ \\ 
	 ~ GDP per capita (US\$, PPP) & 2\,571 ~ \ \ & 2\,622 ~ \ \ & \textit{5\,002} ~ \ \ \\ 
	 ~ Domestic food price volatility (index) &  ~ \ \ &  ~ \ \ &  ~ \ \ \\ 
	 ~ Cereal import dependency ratio (\%) & 59.7 ~ \ \ & 8.1 ~ \ \ & \textit{18.2} ~ \ \ \\ 
	 ~ Underweight, children under-5 (\%) &  ~ \ \ & 7.1 ~ \ \ & \textit{4.4} ~ \ \ \\ 
	 ~ Improved water source (\% pop) & 89.9 ~ \ \ & 88.4 ~ \ \ & \textit{87.3} ~ \ \ \\ 
	\multicolumn{4}{l}{\textcolor{FAOblue}{\textbf{\large{Food Supply}}}} \\ 
	 ~ Food production value, (2004-2006 mln I\$) & 4\,920 ~ \ \ & 5\,505 ~ \ \ & \textit{10\,915} ~ \ \ \\ 
	 ~ Agriculture, value added (\% GDP) & 35 ~ \ \ & 34 ~ \ \ & \textit{19} ~ \ \ \\ 
	 ~ Food exports (mln US\$)  & 68 ~ \ \ & 89 ~ \ \ & \textit{470} ~ \ \ \\ 
	 ~ Food imports (mln US\$)  & 955 ~ \ \ & 240 ~ \ \ & \textit{1\,133} ~ \ \ \\ 
	\multicolumn{4}{l}{\textit{\normalsize{Production indices (2004-06=100)}}} \\ 
	 ~ Net food & 74 ~ \ \ & 82 ~ \ \ & \textit{163} ~ \ \ \\ 
	 ~ Net crop & 83 ~ \ \ & 82 ~ \ \ & \textit{144} ~ \ \ \\ 
	 ~ Cereal & 36 ~ \ \ & 85 ~ \ \ & \textit{123} ~ \ \ \\ 
	 ~ Vegetable oils & 109 ~ \ \ & 85 ~ \ \ & \textit{90} ~ \ \ \\ 
	 ~ Roots and tubers & 36 ~ \ \ & 81 ~ \ \ & \textit{238} ~ \ \ \\ 
	 ~ Fruit and vegetables & 82 ~ \ \ & 78 ~ \ \ & \textit{216} ~ \ \ \\ 
	 ~ Sugar &  ~ \ \ &  ~ \ \ &  ~ \ \ \\ 
	 ~ Livestock & 77 ~ \ \ & 81 ~ \ \ & \textit{166} ~ \ \ \\ 
	 ~ Milk & 82 ~ \ \ & 81 ~ \ \ & \textit{173} ~ \ \ \\ 
	 ~ Meat & 70 ~ \ \ & 82 ~ \ \ & \textit{157} ~ \ \ \\ 
	 ~ Fish  & 574 ~ \ \ & 110 ~ \ \ & \textit{223} ~ \ \ \\ 
	\multicolumn{4}{l}{\textit{\normalsize{Net trade (min US\$)}}} \\ 
	 ~ Cereals & \textit{-272} ~ \ \ & -73 ~ \ \ & \textit{-483} ~ \ \ \\ 
	 ~ Fruit and vegetables & 31 ~ \ \ & 72 ~ \ \ & \textit{412} ~ \ \ \\ 
	 ~ Meat & \textit{-256} ~ \ \ & -6 ~ \ \ & \textit{-31} ~ \ \ \\ 
	 ~ Dairy products & \textit{-15} ~ \ \ & -21 ~ \ \ & \textit{-26} ~ \ \ \\ 
	 ~ Fish & \textit{0} ~ \ \ & -1 ~ \ \ & \textit{-3} ~ \ \ \\ 
	\multicolumn{4}{l}{\textcolor{FAOblue}{\textbf{\large{Environment}}}} \\ 
	 ~ Forest area (\%) & 7 ~ \ \ & 8 ~ \ \ & \textit{8} ~ \ \ \\ 
	 ~ Renewable water res withdrawn (\% of total) &  ~ \ \ & \textit{90} ~ \ \ & 90 ~ \ \ \\ 
	 ~ Terrestrial protect areas (\% total land area)  & 2 ~ \ \ & 2 ~ \ \ & \textit{3} ~ \ \ \\ 
	 ~ Organic area (\% total agricultural area) &  ~ \ \ &  ~ \ \ & \textit{0} ~ \ \ \\ 
	 ~ Water withdrawal by agriculture (\% of total) &  ~ \ \ & \textit{90} ~ \ \ & 90 ~ \ \ \\ 
	 ~ Biofuel production (thousand kt of oil eq.) &  ~ \ \ &  ~ \ \ &  ~ \ \ \\ 
	 ~ Wood pellet prod. (min tonnes) &  ~ \ \ &  ~ \ \ & \textit{0} ~ \ \ \\ 
	 ~ GHG emissions from ag (Co2 eq, gigagrams) & 15 ~ \ \ & 10 ~ \ \ & \textit{25} ~ \ \ \\ 
       \toprule
      \end{tabular}
      \clearpage
\CountryData{ Vanuatu }
      \rowcolors{1}{FAOblue!10}{white}
      \begin{tabular}{L{3.9cm} R{1cm} R{1cm} R{1cm}}
      \toprule
      \multicolumn{1}{c}{} & \multicolumn{1}{c}{ 1992 } & \multicolumn{1}{c}{ 2002 } & \multicolumn{1}{c}{ 2014 } \\
      \midrule
	\multicolumn{4}{l}{\textcolor{FAOblue}{\textbf{\large{The setting}}}} \\ 
	 ~ Population, total (mln) & 0.2 ~ \ \ & 0.2 ~ \ \ & 0.3 ~ \ \ \\ 
	 ~ Population, rural (\% total population) & 0.1 ~ \ \ & 0.2 ~ \ \ & 0.2 ~ \ \ \\ 
	 ~ Govt expenditure on ag (\% total outlays) &  ~ \ \ & 17.3 ~ \ \ & \textit{15} ~ \ \ \\ 
	 ~ Area harvested (mln ha) & 0 ~ \ \ & 0 ~ \ \ & 0 ~ \ \ \\ 
	 ~ Cropping intensity ratio (\%) & 1.7 ~ \ \ & 1.3 ~ \ \ &  ~ \ \ \\ 
	 ~ Water resources (m\textsuperscript{3}/person/year) &  ~ \ \ &  ~ \ \ &  ~ \ \ \\ 
	 ~ Area equipped for irrigation (1000 ha) &  ~ \ \ &  ~ \ \ &  ~ \ \ \\ 
	 ~ Area irrigated (\%) &  ~ \ \ &  ~ \ \ &  ~ \ \ \\ 
	 ~ Employment in agriculture (\%) &  ~ \ \ &  ~ \ \ & \textit{60.5} ~ \ \ \\ 
	 ~ Employment in agriculture, female (\%) &  ~ \ \ &  ~ \ \ & \textit{62.3} ~ \ \ \\ 
	 ~ Fertilizers, Nitrogen (nutrients per ha) &  ~ \ \ &  ~ \ \ &  ~ \ \ \\ 
	 ~ Fertilizers, Phosphate (nutrients per ha) &  ~ \ \ &  ~ \ \ &  ~ \ \ \\ 
	 ~ Fertilizers, Potash (nutrients per ha) &  ~ \ \ &  ~ \ \ &  ~ \ \ \\ 
	 ~ Energy consump, power irrigation (mln kWh) &  ~ \ \ &  ~ \ \ &  ~ \ \ \\ 
	 ~ Agr value added per worker (constant US\$) & 1.9 ~ \ \ & 2.3 ~ \ \ & \textit{2.8} ~ \ \ \\ 
	\multicolumn{4}{l}{\textcolor{FAOblue}{\textbf{\large{Hunger dimensions}}}} \\ 
	 ~ Dietary energy supply (kcal/pc/day) & 2\,543 ~ \ \ & 2\,710 ~ \ \ & 2\,868 ~ \ \ \\ 
	 ~ Average dietary energy supply adequacy (\%) & 120 ~ \ \ & 126 ~ \ \ & 131 ~ \ \ \\ 
	 ~ Dietary en supp, cereals/roots/tubers (\%) & 43 ~ \ \ & 49 ~ \ \ & \textit{47} ~ \ \ \\ 
	 ~ Prevalence of undernourishment (\%) & 11.6 ~ \ \ & 8.3 ~ \ \ & 6.4 ~ \ \ \\ 
	 ~ GDP per capita (US\$, PPP) & 2\,548 ~ \ \ & 2\,467 ~ \ \ & \textit{2\,895} ~ \ \ \\ 
	 ~ Domestic food price volatility (index) &  ~ \ \ &  ~ \ \ &  ~ \ \ \\ 
	 ~ Cereal import dependency ratio (\%) & 91.7 ~ \ \ & 95.1 ~ \ \ & \textit{95.8} ~ \ \ \\ 
	 ~ Underweight, children under-5 (\%) &  ~ \ \ & \textit{10.6} ~ \ \ & \textit{11.7} ~ \ \ \\ 
	 ~ Improved water source (\% pop) & 64.8 ~ \ \ & 78.8 ~ \ \ & \textit{90.7} ~ \ \ \\ 
	\multicolumn{4}{l}{\textcolor{FAOblue}{\textbf{\large{Food Supply}}}} \\ 
	 ~ Food production value, (2004-2006 mln I\$) & 58 ~ \ \ & 55 ~ \ \ & \textit{82} ~ \ \ \\ 
	 ~ Agriculture, value added (\% GDP) & 16 ~ \ \ & 27 ~ \ \ & \textit{28} ~ \ \ \\ 
	 ~ Food exports (mln US\$)  & 12 ~ \ \ & 8 ~ \ \ & \textit{31} ~ \ \ \\ 
	 ~ Food imports (mln US\$)  & 13 ~ \ \ & 13 ~ \ \ & \textit{54} ~ \ \ \\ 
	\multicolumn{4}{l}{\textit{\normalsize{Production indices (2004-06=100)}}} \\ 
	 ~ Net food & 97 ~ \ \ & 92 ~ \ \ & \textit{138} ~ \ \ \\ 
	 ~ Net crop & 97 ~ \ \ & 94 ~ \ \ & \textit{146} ~ \ \ \\ 
	 ~ Cereal & 75 ~ \ \ & 110 ~ \ \ & \textit{110} ~ \ \ \\ 
	 ~ Vegetable oils & 101 ~ \ \ & 93 ~ \ \ & \textit{161} ~ \ \ \\ 
	 ~ Roots and tubers & 82 ~ \ \ & 94 ~ \ \ & \textit{118} ~ \ \ \\ 
	 ~ Fruit and vegetables & 84 ~ \ \ & 97 ~ \ \ & \textit{106} ~ \ \ \\ 
	 ~ Sugar &  ~ \ \ &  ~ \ \ &  ~ \ \ \\ 
	 ~ Livestock & 96 ~ \ \ & 88 ~ \ \ & \textit{112} ~ \ \ \\ 
	 ~ Milk & 75 ~ \ \ & 101 ~ \ \ & \textit{115} ~ \ \ \\ 
	 ~ Meat & 98 ~ \ \ & 87 ~ \ \ & \textit{111} ~ \ \ \\ 
	 ~ Fish  & 20 ~ \ \ & 20 ~ \ \ & \textit{26} ~ \ \ \\ 
	\multicolumn{4}{l}{\textit{\normalsize{Net trade (min US\$)}}} \\ 
	 ~ Cereals & -6 ~ \ \ & -6 ~ \ \ & \textit{-26} ~ \ \ \\ 
	 ~ Fruit and vegetables & -2 ~ \ \ & -2 ~ \ \ & \textit{3} ~ \ \ \\ 
	 ~ Meat & 2 ~ \ \ & 0 ~ \ \ & \textit{-7} ~ \ \ \\ 
	 ~ Dairy products & \textit{-1} ~ \ \ & 0 ~ \ \ & \textit{-4} ~ \ \ \\ 
	 ~ Fish & 47 ~ \ \ & 44 ~ \ \ & \textit{35} ~ \ \ \\ 
	\multicolumn{4}{l}{\textcolor{FAOblue}{\textbf{\large{Environment}}}} \\ 
	 ~ Forest area (\%) & 36 ~ \ \ & 36 ~ \ \ & \textit{36} ~ \ \ \\ 
	 ~ Renewable water res withdrawn (\% of total) &  ~ \ \ &  ~ \ \ &  ~ \ \ \\ 
	 ~ Terrestrial protect areas (\% total land area)  & 4 ~ \ \ & 4 ~ \ \ & \textit{4} ~ \ \ \\ 
	 ~ Organic area (\% total agricultural area) &  ~ \ \ &  ~ \ \ & \textit{2} ~ \ \ \\ 
	 ~ Water withdrawal by agriculture (\% of total) &  ~ \ \ &  ~ \ \ &  ~ \ \ \\ 
	 ~ Biofuel production (thousand kt of oil eq.) &  ~ \ \ &  ~ \ \ &  ~ \ \ \\ 
	 ~ Wood pellet prod. (min tonnes) &  ~ \ \ &  ~ \ \ &  ~ \ \ \\ 
	 ~ GHG emissions from ag (Co2 eq, gigagrams) & 0 ~ \ \ & 0 ~ \ \ & \textit{0} ~ \ \ \\ 
       \toprule
      \end{tabular}
      \clearpage
\CountryData{ Venezuela }
      \rowcolors{1}{FAOblue!10}{white}
      \begin{tabular}{L{3.9cm} R{1cm} R{1cm} R{1cm}}
      \toprule
      \multicolumn{1}{c}{} & \multicolumn{1}{c}{ 1992 } & \multicolumn{1}{c}{ 2002 } & \multicolumn{1}{c}{ 2014 } \\
      \midrule
	\multicolumn{4}{l}{\textcolor{FAOblue}{\textbf{\large{The setting}}}} \\ 
	 ~ Population, total (mln) & 20.7 ~ \ \ & 25.3 ~ \ \ & 30.9 ~ \ \ \\ 
	 ~ Population, rural (\% total population) & 3 ~ \ \ & 2.3 ~ \ \ & 1.8 ~ \ \ \\ 
	 ~ Govt expenditure on ag (\% total outlays) &  ~ \ \ &  ~ \ \ &  ~ \ \ \\ 
	 ~ Area harvested (mln ha) & 7 ~ \ \ & 9 ~ \ \ & 7 ~ \ \ \\ 
	 ~ Cropping intensity ratio (\%) & 0.3 ~ \ \ & 0.4 ~ \ \ &  ~ \ \ \\ 
	 ~ Water resources (m\textsuperscript{3}/person/year) & \textit{63} ~ \ \ & \textit{51} ~ \ \ & \textit{44} ~ \ \ \\ 
	 ~ Area equipped for irrigation (1000 ha) &  ~ \ \ &  ~ \ \ & \textit{1\,055} ~ \ \ \\ 
	 ~ Area irrigated (\%) &  ~ \ \ &  ~ \ \ & \textit{92.8} ~ \ \ \\ 
	 ~ Employment in agriculture (\%) & 11.8 ~ \ \ & 9.9 ~ \ \ & \textit{7.7} ~ \ \ \\ 
	 ~ Employment in agriculture, female (\%) & 1.8 ~ \ \ & 1.9 ~ \ \ & \textit{1.8} ~ \ \ \\ 
	 ~ Fertilizers, Nitrogen (nutrients per ha) &  ~ \ \ & 12 ~ \ \ & \textit{13.1} ~ \ \ \\ 
	 ~ Fertilizers, Phosphate (nutrients per ha) &  ~ \ \ & 2 ~ \ \ & \textit{3.2} ~ \ \ \\ 
	 ~ Fertilizers, Potash (nutrients per ha) &  ~ \ \ & 2.8 ~ \ \ & \textit{4.6} ~ \ \ \\ 
	 ~ Energy consump, power irrigation (mln kWh) & 203 ~ \ \ & 203 ~ \ \ & \textit{694} ~ \ \ \\ 
	 ~ Agr value added per worker (constant US\$) & 4.8 ~ \ \ & 5.9 ~ \ \ & \textit{9.2} ~ \ \ \\ 
	\multicolumn{4}{l}{\textcolor{FAOblue}{\textbf{\large{Hunger dimensions}}}} \\ 
	 ~ Dietary energy supply (kcal/pc/day) & 2\,473 ~ \ \ & 2\,406 ~ \ \ & 2\,992 ~ \ \ \\ 
	 ~ Average dietary energy supply adequacy (\%) & 110 ~ \ \ & 104 ~ \ \ & 128 ~ \ \ \\ 
	 ~ Dietary en supp, cereals/roots/tubers (\%) & 39 ~ \ \ & 40 ~ \ \ & \textit{41} ~ \ \ \\ 
	 ~ Prevalence of undernourishment (\%) & 13.2 ~ \ \ & 15.1 ~ \ \ & <5.0 ~ \ \ \\ 
	 ~ GDP per capita (US\$, PPP) & 16\,143 ~ \ \ & 13\,129 ~ \ \ & \textit{17\,615} ~ \ \ \\ 
	 ~ Domestic food price volatility (index) &  ~ \ \ & 10.7 ~ \ \ & 12.8 ~ \ \ \\ 
	 ~ Cereal import dependency ratio (\%) & 53.2 ~ \ \ & 45.8 ~ \ \ & \textit{56.6} ~ \ \ \\ 
	 ~ Underweight, children under-5 (\%) & 4.5 ~ \ \ & 4.2 ~ \ \ & \textit{2.9} ~ \ \ \\ 
	 ~ Improved water source (\% pop) & 90.4 ~ \ \ & 92.5 ~ \ \ & \textit{92.9} ~ \ \ \\ 
	\multicolumn{4}{l}{\textcolor{FAOblue}{\textbf{\large{Food Supply}}}} \\ 
	 ~ Food production value, (2004-2006 mln I\$) & 4\,226 ~ \ \ & 5\,480 ~ \ \ & \textit{7\,244} ~ \ \ \\ 
	 ~ Agriculture, value added (\% GDP) & 5 ~ \ \ & 4 ~ \ \ & \textit{6} ~ \ \ \\ 
	 ~ Food exports (mln US\$)  & 146 ~ \ \ & 124 ~ \ \ & \textit{15} ~ \ \ \\ 
	 ~ Food imports (mln US\$)  & 905 ~ \ \ & 1\,188 ~ \ \ & \textit{7\,416} ~ \ \ \\ 
	\multicolumn{4}{l}{\textit{\normalsize{Production indices (2004-06=100)}}} \\ 
	 ~ Net food & 77 ~ \ \ & 99 ~ \ \ & \textit{131} ~ \ \ \\ 
	 ~ Net crop & 80 ~ \ \ & 90 ~ \ \ & \textit{127} ~ \ \ \\ 
	 ~ Cereal & 54 ~ \ \ & 68 ~ \ \ & \textit{97} ~ \ \ \\ 
	 ~ Vegetable oils & 46 ~ \ \ & 72 ~ \ \ & \textit{71} ~ \ \ \\ 
	 ~ Roots and tubers & 59 ~ \ \ & 93 ~ \ \ & \textit{149} ~ \ \ \\ 
	 ~ Fruit and vegetables & 94 ~ \ \ & 104 ~ \ \ & \textit{158} ~ \ \ \\ 
	 ~ Sugar & 76 ~ \ \ & 89 ~ \ \ & \textit{77} ~ \ \ \\ 
	 ~ Livestock & 76 ~ \ \ & 105 ~ \ \ & \textit{140} ~ \ \ \\ 
	 ~ Milk & 120 ~ \ \ & 104 ~ \ \ & \textit{197} ~ \ \ \\ 
	 ~ Meat & 70 ~ \ \ & 106 ~ \ \ & \textit{130} ~ \ \ \\ 
	 ~ Fish  & 74 ~ \ \ & 117 ~ \ \ & \textit{50} ~ \ \ \\ 
	\multicolumn{4}{l}{\textit{\normalsize{Net trade (min US\$)}}} \\ 
	 ~ Cereals & -309 ~ \ \ & -371 ~ \ \ & \textit{-1\,846} ~ \ \ \\ 
	 ~ Fruit and vegetables & -80 ~ \ \ & -147 ~ \ \ & \textit{-550} ~ \ \ \\ 
	 ~ Meat & 5 ~ \ \ & -10 ~ \ \ & \textit{-1\,248} ~ \ \ \\ 
	 ~ Dairy products & -123 ~ \ \ & -133 ~ \ \ & \textit{-1\,232} ~ \ \ \\ 
	 ~ Fish & 64 ~ \ \ & 87 ~ \ \ & \textit{-188} ~ \ \ \\ 
	\multicolumn{4}{l}{\textcolor{FAOblue}{\textbf{\large{Environment}}}} \\ 
	 ~ Forest area (\%) & 58 ~ \ \ & 55 ~ \ \ & \textit{52} ~ \ \ \\ 
	 ~ Renewable water res withdrawn (\% of total) &  ~ \ \ &  ~ \ \ & 74 ~ \ \ \\ 
	 ~ Terrestrial protect areas (\% total land area)  & 53 ~ \ \ & 54 ~ \ \ & \textit{53} ~ \ \ \\ 
	 ~ Organic area (\% total agricultural area) &  ~ \ \ &  ~ \ \ & \textit{0} ~ \ \ \\ 
	 ~ Water withdrawal by agriculture (\% of total) &  ~ \ \ &  ~ \ \ & 74 ~ \ \ \\ 
	 ~ Biofuel production (thousand kt of oil eq.) & 14 ~ \ \ & 28 ~ \ \ & \textit{28} ~ \ \ \\ 
	 ~ Wood pellet prod. (min tonnes) &  ~ \ \ &  ~ \ \ &  ~ \ \ \\ 
	 ~ GHG emissions from ag (Co2 eq, gigagrams) & 136 ~ \ \ & 148 ~ \ \ & \textit{144} ~ \ \ \\ 
       \toprule
      \end{tabular}
      \clearpage
\CountryData{ Viet Nam }
      \rowcolors{1}{FAOblue!10}{white}
      \begin{tabular}{L{3.9cm} R{1cm} R{1cm} R{1cm}}
      \toprule
      \multicolumn{1}{c}{} & \multicolumn{1}{c}{ 1992 } & \multicolumn{1}{c}{ 2002 } & \multicolumn{1}{c}{ 2014 } \\
      \midrule
	\multicolumn{4}{l}{\textcolor{FAOblue}{\textbf{\large{The setting}}}} \\ 
	 ~ Population, total (mln) & 71.9 ~ \ \ & 82.5 ~ \ \ & 92.5 ~ \ \ \\ 
	 ~ Population, rural (\% total population) & 56.8 ~ \ \ & 61.5 ~ \ \ & 62 ~ \ \ \\ 
	 ~ Govt expenditure on ag (\% total outlays) &  ~ \ \ &  ~ \ \ & \textit{2.5} ~ \ \ \\ 
	 ~ Area harvested (mln ha) & 22 ~ \ \ & 37 ~ \ \ & 49 ~ \ \ \\ 
	 ~ Cropping intensity ratio (\%) & 3.2 ~ \ \ & 3.9 ~ \ \ &  ~ \ \ \\ 
	 ~ Water resources (m\textsuperscript{3}/person/year) & \textit{12} ~ \ \ & \textit{11} ~ \ \ & \textit{10} ~ \ \ \\ 
	 ~ Area equipped for irrigation (1000 ha) &  ~ \ \ &  ~ \ \ & \textit{4\,600} ~ \ \ \\ 
	 ~ Area irrigated (\%) &  ~ \ \ & \textit{100} ~ \ \ & \textit{100} ~ \ \ \\ 
	 ~ Employment in agriculture (\%) &  ~ \ \ & 62 ~ \ \ & \textit{47.4} ~ \ \ \\ 
	 ~ Employment in agriculture, female (\%) &  ~ \ \ & 63.2 ~ \ \ & \textit{49.5} ~ \ \ \\ 
	 ~ Fertilizers, Nitrogen (nutrients per ha) &  ~ \ \ & 121.7 ~ \ \ & \textit{84.6} ~ \ \ \\ 
	 ~ Fertilizers, Phosphate (nutrients per ha) &  ~ \ \ & 54.1 ~ \ \ & \textit{44.4} ~ \ \ \\ 
	 ~ Fertilizers, Potash (nutrients per ha) &  ~ \ \ & 37.1 ~ \ \ & \textit{46.3} ~ \ \ \\ 
	 ~ Energy consump, power irrigation (mln kWh) & \textit{0} ~ \ \ & 0 ~ \ \ & \textit{3} ~ \ \ \\ 
	 ~ Agr value added per worker (constant US\$) & 0.3 ~ \ \ & 0.4 ~ \ \ & \textit{0.5} ~ \ \ \\ 
	\multicolumn{4}{l}{\textcolor{FAOblue}{\textbf{\large{Hunger dimensions}}}} \\ 
	 ~ Dietary energy supply (kcal/pc/day) & 1\,936 ~ \ \ & 2\,359 ~ \ \ & 2\,796 ~ \ \ \\ 
	 ~ Average dietary energy supply adequacy (\%) & 90 ~ \ \ & 104 ~ \ \ & 121 ~ \ \ \\ 
	 ~ Dietary en supp, cereals/roots/tubers (\%) & 78 ~ \ \ & 69 ~ \ \ & \textit{59} ~ \ \ \\ 
	 ~ Prevalence of undernourishment (\%) & 44.8 ~ \ \ & 23.3 ~ \ \ & 11.8 ~ \ \ \\ 
	 ~ GDP per capita (US\$, PPP) & 1\,667 ~ \ \ & 2\,920 ~ \ \ & \textit{5\,125} ~ \ \ \\ 
	 ~ Domestic food price volatility (index) &  ~ \ \ &  ~ \ \ &  ~ \ \ \\ 
	 ~ Cereal import dependency ratio (\%) & -9 ~ \ \ & -11.2 ~ \ \ & \textit{-11} ~ \ \ \\ 
	 ~ Underweight, children under-5 (\%) & \textit{40.6} ~ \ \ & 23.4 ~ \ \ & \textit{12} ~ \ \ \\ 
	 ~ Improved water source (\% pop) & 64.8 ~ \ \ & 80.5 ~ \ \ & \textit{95} ~ \ \ \\ 
	\multicolumn{4}{l}{\textcolor{FAOblue}{\textbf{\large{Food Supply}}}} \\ 
	 ~ Food production value, (2004-2006 mln I\$) & 11\,000 ~ \ \ & 18\,170 ~ \ \ & \textit{27\,498} ~ \ \ \\ 
	 ~ Agriculture, value added (\% GDP) & 34 ~ \ \ & 21 ~ \ \ & \textit{18} ~ \ \ \\ 
	 ~ Food exports (mln US\$)  & 617 ~ \ \ & 1\,347 ~ \ \ & \textit{7\,147} ~ \ \ \\ 
	 ~ Food imports (mln US\$)  & 112 ~ \ \ & 642 ~ \ \ & \textit{7\,062} ~ \ \ \\ 
	\multicolumn{4}{l}{\textit{\normalsize{Production indices (2004-06=100)}}} \\ 
	 ~ Net food & 53 ~ \ \ & 88 ~ \ \ & \textit{134} ~ \ \ \\ 
	 ~ Net crop & 51 ~ \ \ & 89 ~ \ \ & \textit{133} ~ \ \ \\ 
	 ~ Cereal & 58 ~ \ \ & 94 ~ \ \ & \textit{123} ~ \ \ \\ 
	 ~ Vegetable oils & 59 ~ \ \ & 83 ~ \ \ & \textit{105} ~ \ \ \\ 
	 ~ Roots and tubers & 58 ~ \ \ & 74 ~ \ \ & \textit{134} ~ \ \ \\ 
	 ~ Fruit and vegetables & 55 ~ \ \ & 89 ~ \ \ & \textit{150} ~ \ \ \\ 
	 ~ Sugar & 41 ~ \ \ & 109 ~ \ \ & \textit{128} ~ \ \ \\ 
	 ~ Livestock & 43 ~ \ \ & 79 ~ \ \ & \textit{147} ~ \ \ \\ 
	 ~ Milk & 30 ~ \ \ & 51 ~ \ \ & \textit{214} ~ \ \ \\ 
	 ~ Meat & 42 ~ \ \ & 78 ~ \ \ & \textit{145} ~ \ \ \\ 
	 ~ Fish  & 30 ~ \ \ & 73 ~ \ \ & \textit{176} ~ \ \ \\ 
	\multicolumn{4}{l}{\textit{\normalsize{Net trade (min US\$)}}} \\ 
	 ~ Cereals & 373 ~ \ \ & 524 ~ \ \ & \textit{2\,081} ~ \ \ \\ 
	 ~ Fruit and vegetables & 67 ~ \ \ & 281 ~ \ \ & \textit{1\,519} ~ \ \ \\ 
	 ~ Meat & 25 ~ \ \ & 9 ~ \ \ & \textit{-1\,474} ~ \ \ \\ 
	 ~ Dairy products & -25 ~ \ \ & -108 ~ \ \ & \textit{-581} ~ \ \ \\ 
	 ~ Fish & 305 ~ \ \ & 1\,926 ~ \ \ & \textit{5\,462} ~ \ \ \\ 
	\multicolumn{4}{l}{\textcolor{FAOblue}{\textbf{\large{Environment}}}} \\ 
	 ~ Forest area (\%) & 30 ~ \ \ & 39 ~ \ \ & \textit{45} ~ \ \ \\ 
	 ~ Renewable water res withdrawn (\% of total) &  ~ \ \ & \textit{95} ~ \ \ & 95 ~ \ \ \\ 
	 ~ Terrestrial protect areas (\% total land area)  & 5 ~ \ \ & 6 ~ \ \ & \textit{6} ~ \ \ \\ 
	 ~ Organic area (\% total agricultural area) &  ~ \ \ & \textit{0} ~ \ \ & \textit{0} ~ \ \ \\ 
	 ~ Water withdrawal by agriculture (\% of total) &  ~ \ \ & \textit{95} ~ \ \ & 95 ~ \ \ \\ 
	 ~ Biofuel production (thousand kt of oil eq.) & 12 ~ \ \ & 22 ~ \ \ & \textit{23} ~ \ \ \\ 
	 ~ Wood pellet prod. (min tonnes) &  ~ \ \ &  ~ \ \ & \textit{170} ~ \ \ \\ 
	 ~ GHG emissions from ag (Co2 eq, gigagrams) & -4 ~ \ \ & 38 ~ \ \ & \textit{46} ~ \ \ \\ 
       \toprule
      \end{tabular}
      \clearpage
\CountryData{ Yemen }
      \rowcolors{1}{FAOblue!10}{white}
      \begin{tabular}{L{3.9cm} R{1cm} R{1cm} R{1cm}}
      \toprule
      \multicolumn{1}{c}{} & \multicolumn{1}{c}{ 1992 } & \multicolumn{1}{c}{ 2002 } & \multicolumn{1}{c}{ 2014 } \\
      \midrule
	\multicolumn{4}{l}{\textcolor{FAOblue}{\textbf{\large{The setting}}}} \\ 
	 ~ Population, total (mln) & 13 ~ \ \ & 18.6 ~ \ \ & 25 ~ \ \ \\ 
	 ~ Population, rural (\% total population) & 10.2 ~ \ \ & 13.5 ~ \ \ & 16.5 ~ \ \ \\ 
	 ~ Govt expenditure on ag (\% total outlays) &  ~ \ \ & \textit{1} ~ \ \ & \textit{1.1} ~ \ \ \\ 
	 ~ Area harvested (mln ha) & 1 ~ \ \ & 1 ~ \ \ & 1 ~ \ \ \\ 
	 ~ Cropping intensity ratio (\%) & 0 ~ \ \ & 0 ~ \ \ &  ~ \ \ \\ 
	 ~ Water resources (m\textsuperscript{3}/person/year) & \textit{0} ~ \ \ & \textit{0} ~ \ \ & \textit{0} ~ \ \ \\ 
	 ~ Area equipped for irrigation (1000 ha) &  ~ \ \ &  ~ \ \ & \textit{680} ~ \ \ \\ 
	 ~ Area irrigated (\%) &  ~ \ \ &  ~ \ \ &  ~ \ \ \\ 
	 ~ Employment in agriculture (\%) & \textit{53.6} ~ \ \ & \textit{31} ~ \ \ & \textit{24.7} ~ \ \ \\ 
	 ~ Employment in agriculture, female (\%) & \textit{86.9} ~ \ \ & \textit{41.8} ~ \ \ & \textit{28} ~ \ \ \\ 
	 ~ Fertilizers, Nitrogen (nutrients per ha) &  ~ \ \ & 0.5 ~ \ \ & \textit{0.5} ~ \ \ \\ 
	 ~ Fertilizers, Phosphate (nutrients per ha) &  ~ \ \ & 0 ~ \ \ & \textit{0} ~ \ \ \\ 
	 ~ Fertilizers, Potash (nutrients per ha) &  ~ \ \ & 0 ~ \ \ & \textit{0} ~ \ \ \\ 
	 ~ Energy consump, power irrigation (mln kWh) & \textit{2} ~ \ \ & 2 ~ \ \ & \textit{2} ~ \ \ \\ 
	 ~ Agr value added per worker (constant US\$) &  ~ \ \ & \textit{0.8} ~ \ \ & \textit{0.8} ~ \ \ \\ 
	\multicolumn{4}{l}{\textcolor{FAOblue}{\textbf{\large{Hunger dimensions}}}} \\ 
	 ~ Dietary energy supply (kcal/pc/day) & 2\,077 ~ \ \ & 2\,072 ~ \ \ & 2\,212 ~ \ \ \\ 
	 ~ Average dietary energy supply adequacy (\%) & 104 ~ \ \ & 100 ~ \ \ & 102 ~ \ \ \\ 
	 ~ Dietary en supp, cereals/roots/tubers (\%) & 68 ~ \ \ & 64 ~ \ \ & \textit{63} ~ \ \ \\ 
	 ~ Prevalence of undernourishment (\%) & 28.6 ~ \ \ & 30 ~ \ \ & 25.9 ~ \ \ \\ 
	 ~ GDP per capita (US\$, PPP) & 3\,578 ~ \ \ & 4\,094 ~ \ \ & \textit{3\,832} ~ \ \ \\ 
	 ~ Domestic food price volatility (index) &  ~ \ \ & \textit{19.5} ~ \ \ & \textit{11} ~ \ \ \\ 
	 ~ Cereal import dependency ratio (\%) & 73.7 ~ \ \ & 81.6 ~ \ \ & \textit{81.2} ~ \ \ \\ 
	 ~ Underweight, children under-5 (\%) & \textit{29.6} ~ \ \ & \textit{43.1} ~ \ \ & \textit{35.5} ~ \ \ \\ 
	 ~ Improved water source (\% pop) & 65.1 ~ \ \ & 58.5 ~ \ \ & \textit{54.9} ~ \ \ \\ 
	\multicolumn{4}{l}{\textcolor{FAOblue}{\textbf{\large{Food Supply}}}} \\ 
	 ~ Food production value, (2004-2006 mln I\$) & 779 ~ \ \ & 1\,145 ~ \ \ & \textit{1\,774} ~ \ \ \\ 
	 ~ Agriculture, value added (\% GDP) & 24 ~ \ \ & 13 ~ \ \ & \textit{10} ~ \ \ \\ 
	 ~ Food exports (mln US\$)  & 12 ~ \ \ & 41 ~ \ \ & \textit{180} ~ \ \ \\ 
	 ~ Food imports (mln US\$)  & 836 ~ \ \ & 905 ~ \ \ & \textit{3\,682} ~ \ \ \\ 
	\multicolumn{4}{l}{\textit{\normalsize{Production indices (2004-06=100)}}} \\ 
	 ~ Net food & 61 ~ \ \ & 89 ~ \ \ & \textit{138} ~ \ \ \\ 
	 ~ Net crop & 65 ~ \ \ & 88 ~ \ \ & \textit{122} ~ \ \ \\ 
	 ~ Cereal & 131 ~ \ \ & 91 ~ \ \ & \textit{141} ~ \ \ \\ 
	 ~ Vegetable oils & 54 ~ \ \ & 96 ~ \ \ & \textit{121} ~ \ \ \\ 
	 ~ Roots and tubers & 84 ~ \ \ & 98 ~ \ \ & \textit{133} ~ \ \ \\ 
	 ~ Fruit and vegetables & 52 ~ \ \ & 87 ~ \ \ & \textit{117} ~ \ \ \\ 
	 ~ Sugar &  ~ \ \ &  ~ \ \ &  ~ \ \ \\ 
	 ~ Livestock & 56 ~ \ \ & 91 ~ \ \ & \textit{159} ~ \ \ \\ 
	 ~ Milk & 67 ~ \ \ & 87 ~ \ \ & \textit{126} ~ \ \ \\ 
	 ~ Meat & 55 ~ \ \ & 95 ~ \ \ & \textit{170} ~ \ \ \\ 
	 ~ Fish  & 33 ~ \ \ & 74 ~ \ \ & \textit{87} ~ \ \ \\ 
	\multicolumn{4}{l}{\textit{\normalsize{Net trade (min US\$)}}} \\ 
	 ~ Cereals & -380 ~ \ \ & -448 ~ \ \ & \textit{-1\,917} ~ \ \ \\ 
	 ~ Fruit and vegetables & -45 ~ \ \ & -9 ~ \ \ & \textit{-146} ~ \ \ \\ 
	 ~ Meat & -21 ~ \ \ & -75 ~ \ \ & \textit{-217} ~ \ \ \\ 
	 ~ Dairy products & -52 ~ \ \ & -74 ~ \ \ & \textit{-285} ~ \ \ \\ 
	 ~ Fish & 4 ~ \ \ & 112 ~ \ \ & \textit{193} ~ \ \ \\ 
	\multicolumn{4}{l}{\textcolor{FAOblue}{\textbf{\large{Environment}}}} \\ 
	 ~ Forest area (\%) & 1 ~ \ \ & 1 ~ \ \ & \textit{1} ~ \ \ \\ 
	 ~ Renewable water res withdrawn (\% of total) &  ~ \ \ & \textit{91} ~ \ \ & 91 ~ \ \ \\ 
	 ~ Terrestrial protect areas (\% total land area)  & \textit{0} ~ \ \ & 1 ~ \ \ & \textit{1} ~ \ \ \\ 
	 ~ Organic area (\% total agricultural area) &  ~ \ \ &  ~ \ \ &  ~ \ \ \\ 
	 ~ Water withdrawal by agriculture (\% of total) &  ~ \ \ & \textit{91} ~ \ \ & 91 ~ \ \ \\ 
	 ~ Biofuel production (thousand kt of oil eq.) &  ~ \ \ &  ~ \ \ &  ~ \ \ \\ 
	 ~ Wood pellet prod. (min tonnes) &  ~ \ \ &  ~ \ \ &  ~ \ \ \\ 
	 ~ GHG emissions from ag (Co2 eq, gigagrams) & 4 ~ \ \ & 5 ~ \ \ & \textit{7} ~ \ \ \\ 
       \toprule
      \end{tabular}
      \clearpage
\CountryData{ Zambia }
      \rowcolors{1}{FAOblue!10}{white}
      \begin{tabular}{L{3.9cm} R{1cm} R{1cm} R{1cm}}
      \toprule
      \multicolumn{1}{c}{} & \multicolumn{1}{c}{ 1992 } & \multicolumn{1}{c}{ 2002 } & \multicolumn{1}{c}{ 2014 } \\
      \midrule
	\multicolumn{4}{l}{\textcolor{FAOblue}{\textbf{\large{The setting}}}} \\ 
	 ~ Population, total (mln) & 8.2 ~ \ \ & 10.6 ~ \ \ & 15 ~ \ \ \\ 
	 ~ Population, rural (\% total population) & 5.1 ~ \ \ & 6.9 ~ \ \ & 8.9 ~ \ \ \\ 
	 ~ Govt expenditure on ag (\% total outlays) &  ~ \ \ & \textit{3.9} ~ \ \ & \textit{9.7} ~ \ \ \\ 
	 ~ Area harvested (mln ha) & 1 ~ \ \ & 2 ~ \ \ & 4 ~ \ \ \\ 
	 ~ Cropping intensity ratio (\%) & 0.1 ~ \ \ & 0.1 ~ \ \ &  ~ \ \ \\ 
	 ~ Water resources (m\textsuperscript{3}/person/year) & \textit{12} ~ \ \ & \textit{10} ~ \ \ & \textit{7} ~ \ \ \\ 
	 ~ Area equipped for irrigation (1000 ha) &  ~ \ \ &  ~ \ \ & \textit{156} ~ \ \ \\ 
	 ~ Area irrigated (\%) &  ~ \ \ & 100 ~ \ \ &  ~ \ \ \\ 
	 ~ Employment in agriculture (\%) & \textit{49.8} ~ \ \ & \textit{72.2} ~ \ \ & \textit{72.2} ~ \ \ \\ 
	 ~ Employment in agriculture, female (\%) & \textit{56} ~ \ \ & \textit{78.9} ~ \ \ & \textit{78.9} ~ \ \ \\ 
	 ~ Fertilizers, Nitrogen (nutrients per ha) &  ~ \ \ & 1.8 ~ \ \ & \textit{1.7} ~ \ \ \\ 
	 ~ Fertilizers, Phosphate (nutrients per ha) &  ~ \ \ & 0.6 ~ \ \ & \textit{0.7} ~ \ \ \\ 
	 ~ Fertilizers, Potash (nutrients per ha) &  ~ \ \ & 0.6 ~ \ \ & \textit{0.5} ~ \ \ \\ 
	 ~ Energy consump, power irrigation (mln kWh) & 43 ~ \ \ & 56 ~ \ \ & \textit{56} ~ \ \ \\ 
	 ~ Agr value added per worker (constant US\$) & 0.3 ~ \ \ & 0.4 ~ \ \ & \textit{0.3} ~ \ \ \\ 
	\multicolumn{4}{l}{\textcolor{FAOblue}{\textbf{\large{Hunger dimensions}}}} \\ 
	 ~ Dietary energy supply (kcal/pc/day) & 1\,990 ~ \ \ & 1\,867 ~ \ \ & 1\,942 ~ \ \ \\ 
	 ~ Average dietary energy supply adequacy (\%) & 95 ~ \ \ & 90 ~ \ \ & 92 ~ \ \ \\ 
	 ~ Dietary en supp, cereals/roots/tubers (\%) & 78 ~ \ \ & 77 ~ \ \ & \textit{71} ~ \ \ \\ 
	 ~ Prevalence of undernourishment (\%) & 34.9 ~ \ \ & 47.1 ~ \ \ & 48.4 ~ \ \ \\ 
	 ~ GDP per capita (US\$, PPP) & 2\,254 ~ \ \ & 2\,304 ~ \ \ & \textit{3\,800} ~ \ \ \\ 
	 ~ Domestic food price volatility (index) &  ~ \ \ & 40 ~ \ \ & \textit{3.2} ~ \ \ \\ 
	 ~ Cereal import dependency ratio (\%) & 23.9 ~ \ \ & 15.8 ~ \ \ & \textit{-8.2} ~ \ \ \\ 
	 ~ Underweight, children under-5 (\%) & 21.2 ~ \ \ & 23.3 ~ \ \ & \textit{14.9} ~ \ \ \\ 
	 ~ Improved water source (\% pop) & 49.8 ~ \ \ & 54.8 ~ \ \ & \textit{63.3} ~ \ \ \\ 
	\multicolumn{4}{l}{\textcolor{FAOblue}{\textbf{\large{Food Supply}}}} \\ 
	 ~ Food production value, (2004-2006 mln I\$) & 621 ~ \ \ & 813 ~ \ \ & \textit{1\,772} ~ \ \ \\ 
	 ~ Agriculture, value added (\% GDP) & 24 ~ \ \ & 17 ~ \ \ & \textit{10} ~ \ \ \\ 
	 ~ Food exports (mln US\$)  & 22 ~ \ \ & 61 ~ \ \ & \textit{756} ~ \ \ \\ 
	 ~ Food imports (mln US\$)  & 131 ~ \ \ & 129 ~ \ \ & \textit{378} ~ \ \ \\ 
	\multicolumn{4}{l}{\textit{\normalsize{Production indices (2004-06=100)}}} \\ 
	 ~ Net food & 65 ~ \ \ & 85 ~ \ \ & \textit{184} ~ \ \ \\ 
	 ~ Net crop & 46 ~ \ \ & 70 ~ \ \ & \textit{158} ~ \ \ \\ 
	 ~ Cereal & 45 ~ \ \ & 56 ~ \ \ & \textit{215} ~ \ \ \\ 
	 ~ Vegetable oils & 19 ~ \ \ & 64 ~ \ \ & \textit{196} ~ \ \ \\ 
	 ~ Roots and tubers & 61 ~ \ \ & 88 ~ \ \ & \textit{113} ~ \ \ \\ 
	 ~ Fruit and vegetables & 82 ~ \ \ & 90 ~ \ \ & \textit{121} ~ \ \ \\ 
	 ~ Sugar & 55 ~ \ \ & 97 ~ \ \ & \textit{169} ~ \ \ \\ 
	 ~ Livestock & 81 ~ \ \ & 99 ~ \ \ & \textit{216} ~ \ \ \\ 
	 ~ Milk & 102 ~ \ \ & 102 ~ \ \ & \textit{108} ~ \ \ \\ 
	 ~ Meat & 81 ~ \ \ & 99 ~ \ \ & \textit{234} ~ \ \ \\ 
	 ~ Fish  & 103 ~ \ \ & 97 ~ \ \ & \textit{153} ~ \ \ \\ 
	\multicolumn{4}{l}{\textit{\normalsize{Net trade (min US\$)}}} \\ 
	 ~ Cereals & -116 ~ \ \ & -68 ~ \ \ & \textit{443} ~ \ \ \\ 
	 ~ Fruit and vegetables & -1 ~ \ \ & -9 ~ \ \ & \textit{-27} ~ \ \ \\ 
	 ~ Meat & 0 ~ \ \ & 0 ~ \ \ & \textit{-13} ~ \ \ \\ 
	 ~ Dairy products & -3 ~ \ \ & -2 ~ \ \ & \textit{-22} ~ \ \ \\ 
	 ~ Fish & -1 ~ \ \ & -2 ~ \ \ & \textit{-32} ~ \ \ \\ 
	\multicolumn{4}{l}{\textcolor{FAOblue}{\textbf{\large{Environment}}}} \\ 
	 ~ Forest area (\%) & 71 ~ \ \ & 68 ~ \ \ & \textit{66} ~ \ \ \\ 
	 ~ Renewable water res withdrawn (\% of total) &  ~ \ \ & 73 ~ \ \ & 73 ~ \ \ \\ 
	 ~ Terrestrial protect areas (\% total land area)  & 36 ~ \ \ & 36 ~ \ \ & \textit{38} ~ \ \ \\ 
	 ~ Organic area (\% total agricultural area) &  ~ \ \ & \textit{0} ~ \ \ & \textit{0} ~ \ \ \\ 
	 ~ Water withdrawal by agriculture (\% of total) &  ~ \ \ & 73 ~ \ \ & 73 ~ \ \ \\ 
	 ~ Biofuel production (thousand kt of oil eq.) & 4 ~ \ \ & 6 ~ \ \ & \textit{10} ~ \ \ \\ 
	 ~ Wood pellet prod. (min tonnes) &  ~ \ \ &  ~ \ \ &  ~ \ \ \\ 
	 ~ GHG emissions from ag (Co2 eq, gigagrams) & 95 ~ \ \ & 88 ~ \ \ & \textit{94} ~ \ \ \\ 
       \toprule
      \end{tabular}
      \clearpage
\CountryData{ Zimbabwe }
      \rowcolors{1}{FAOblue!10}{white}
      \begin{tabular}{L{3.9cm} R{1cm} R{1cm} R{1cm}}
      \toprule
      \multicolumn{1}{c}{} & \multicolumn{1}{c}{ 1992 } & \multicolumn{1}{c}{ 2002 } & \multicolumn{1}{c}{ 2014 } \\
      \midrule
	\multicolumn{4}{l}{\textcolor{FAOblue}{\textbf{\large{The setting}}}} \\ 
	 ~ Population, total (mln) & 11 ~ \ \ & 12.6 ~ \ \ & 14.6 ~ \ \ \\ 
	 ~ Population, rural (\% total population) & 7.6 ~ \ \ & 8.3 ~ \ \ & 8.7 ~ \ \ \\ 
	 ~ Govt expenditure on ag (\% total outlays) &  ~ \ \ &  ~ \ \ & \textit{16} ~ \ \ \\ 
	 ~ Area harvested (mln ha) & 1 ~ \ \ & 4 ~ \ \ & 4 ~ \ \ \\ 
	 ~ Cropping intensity ratio (\%) & 0.1 ~ \ \ & 0.3 ~ \ \ &  ~ \ \ \\ 
	 ~ Water resources (m\textsuperscript{3}/person/year) & \textit{2} ~ \ \ & \textit{2} ~ \ \ & \textit{1} ~ \ \ \\ 
	 ~ Area equipped for irrigation (1000 ha) &  ~ \ \ &  ~ \ \ & \textit{174} ~ \ \ \\ 
	 ~ Area irrigated (\%) &  ~ \ \ & \textit{71.4} ~ \ \ &  ~ \ \ \\ 
	 ~ Employment in agriculture (\%) &  ~ \ \ & \textit{64.8} ~ \ \ &  ~ \ \ \\ 
	 ~ Employment in agriculture, female (\%) &  ~ \ \ & \textit{71.1} ~ \ \ &  ~ \ \ \\ 
	 ~ Fertilizers, Nitrogen (nutrients per ha) &  ~ \ \ & 4.6 ~ \ \ & \textit{4.1} ~ \ \ \\ 
	 ~ Fertilizers, Phosphate (nutrients per ha) &  ~ \ \ & 3.1 ~ \ \ & \textit{2.5} ~ \ \ \\ 
	 ~ Fertilizers, Potash (nutrients per ha) &  ~ \ \ & 0.7 ~ \ \ & \textit{0.5} ~ \ \ \\ 
	 ~ Energy consump, power irrigation (mln kWh) & 216 ~ \ \ & 305 ~ \ \ & \textit{305} ~ \ \ \\ 
	 ~ Agr value added per worker (constant US\$) & 0.3 ~ \ \ & 0.4 ~ \ \ & \textit{0.2} ~ \ \ \\ 
	\multicolumn{4}{l}{\textcolor{FAOblue}{\textbf{\large{Hunger dimensions}}}} \\ 
	 ~ Dietary energy supply (kcal/pc/day) & 1\,954 ~ \ \ & 2\,020 ~ \ \ & 2\,176 ~ \ \ \\ 
	 ~ Average dietary energy supply adequacy (\%) & 89 ~ \ \ & 90 ~ \ \ & 96 ~ \ \ \\ 
	 ~ Dietary en supp, cereals/roots/tubers (\%) & 65 ~ \ \ & 57 ~ \ \ & \textit{58} ~ \ \ \\ 
	 ~ Prevalence of undernourishment (\%) & 44.4 ~ \ \ & 43.2 ~ \ \ & 34 ~ \ \ \\ 
	 ~ GDP per capita (US\$, PPP) & 2\,316 ~ \ \ & 2\,304 ~ \ \ & \textit{1\,773} ~ \ \ \\ 
	 ~ Domestic food price volatility (index) &  ~ \ \ &  ~ \ \ &  ~ \ \ \\ 
	 ~ Cereal import dependency ratio (\%) & 20.4 ~ \ \ & 25.9 ~ \ \ & \textit{43.9} ~ \ \ \\ 
	 ~ Underweight, children under-5 (\%) & \textit{11.7} ~ \ \ & \textit{14} ~ \ \ & 11.2 ~ \ \ \\ 
	 ~ Improved water source (\% pop) & 79.4 ~ \ \ & 79.6 ~ \ \ & \textit{79.9} ~ \ \ \\ 
	\multicolumn{4}{l}{\textcolor{FAOblue}{\textbf{\large{Food Supply}}}} \\ 
	 ~ Food production value, (2004-2006 mln I\$) & 727 ~ \ \ & 1\,139 ~ \ \ & \textit{1\,261} ~ \ \ \\ 
	 ~ Agriculture, value added (\% GDP) & 7 ~ \ \ & 14 ~ \ \ & \textit{12} ~ \ \ \\ 
	 ~ Food exports (mln US\$)  & 65 ~ \ \ & 153 ~ \ \ & \textit{147} ~ \ \ \\ 
	 ~ Food imports (mln US\$)  & 376 ~ \ \ & 234 ~ \ \ & \textit{1\,121} ~ \ \ \\ 
	\multicolumn{4}{l}{\textit{\normalsize{Production indices (2004-06=100)}}} \\ 
	 ~ Net food & 56 ~ \ \ & 89 ~ \ \ & \textit{98} ~ \ \ \\ 
	 ~ Net crop & 67 ~ \ \ & 107 ~ \ \ & \textit{108} ~ \ \ \\ 
	 ~ Cereal & 25 ~ \ \ & 50 ~ \ \ & \textit{55} ~ \ \ \\ 
	 ~ Vegetable oils & 55 ~ \ \ & 123 ~ \ \ & \textit{128} ~ \ \ \\ 
	 ~ Roots and tubers & 58 ~ \ \ & 84 ~ \ \ & \textit{118} ~ \ \ \\ 
	 ~ Fruit and vegetables & 64 ~ \ \ & 92 ~ \ \ & \textit{110} ~ \ \ \\ 
	 ~ Sugar & 3 ~ \ \ & 111 ~ \ \ & \textit{103} ~ \ \ \\ 
	 ~ Livestock & 80 ~ \ \ & 94 ~ \ \ & \textit{105} ~ \ \ \\ 
	 ~ Milk & 109 ~ \ \ & 103 ~ \ \ & \textit{104} ~ \ \ \\ 
	 ~ Meat & 73 ~ \ \ & 93 ~ \ \ & \textit{105} ~ \ \ \\ 
	 ~ Fish  & 165 ~ \ \ & 105 ~ \ \ & \textit{157} ~ \ \ \\ 
	\multicolumn{4}{l}{\textit{\normalsize{Net trade (min US\$)}}} \\ 
	 ~ Cereals & -275 ~ \ \ & -143 ~ \ \ & \textit{-583} ~ \ \ \\ 
	 ~ Fruit and vegetables & 5 ~ \ \ & 20 ~ \ \ & \textit{-62} ~ \ \ \\ 
	 ~ Meat & 15 ~ \ \ & 13 ~ \ \ & \textit{-25} ~ \ \ \\ 
	 ~ Dairy products & 4 ~ \ \ & -1 ~ \ \ & \textit{-40} ~ \ \ \\ 
	 ~ Fish & -3 ~ \ \ & 0 ~ \ \ & \textit{-21} ~ \ \ \\ 
	\multicolumn{4}{l}{\textcolor{FAOblue}{\textbf{\large{Environment}}}} \\ 
	 ~ Forest area (\%) & 56 ~ \ \ & 47 ~ \ \ & \textit{39} ~ \ \ \\ 
	 ~ Renewable water res withdrawn (\% of total) &  ~ \ \ & 79 ~ \ \ & 79 ~ \ \ \\ 
	 ~ Terrestrial protect areas (\% total land area)  & 18 ~ \ \ & 28 ~ \ \ & \textit{27} ~ \ \ \\ 
	 ~ Organic area (\% total agricultural area) &  ~ \ \ &  ~ \ \ & \textit{0} ~ \ \ \\ 
	 ~ Water withdrawal by agriculture (\% of total) &  ~ \ \ & 79 ~ \ \ & 79 ~ \ \ \\ 
	 ~ Biofuel production (thousand kt of oil eq.) & 0 ~ \ \ & 14 ~ \ \ & \textit{8} ~ \ \ \\ 
	 ~ Wood pellet prod. (min tonnes) &  ~ \ \ &  ~ \ \ &  ~ \ \ \\ 
	 ~ GHG emissions from ag (Co2 eq, gigagrams) & 49 ~ \ \ & 47 ~ \ \ & \textit{48} ~ \ \ \\ 
       \toprule
      \end{tabular}
      \clearpage
